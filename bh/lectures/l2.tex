% lecture notes by Umut Özer
% course: bh
\lhead{Lecture 2: January 20}

For a static, spherically symmetric star, we use the metric \eqref{eq:static-sph-sym}.
To find $\Phi$ and  $\Psi$, we need to solve the Einstein equations. In order to find those, we need to determine what matter the star contains.

We will model the matter inside the star as a perfect fluid with energy-momentum tensor
 \begin{equation}
  T_{ab} = (\rho + p) u_a u_b + p g_{ab},
\end{equation}
where $u_a$ is the velocity of the fluid, obeying  $g_{ab} u^{a} u^{b} = -1$ .
The quantities $\rho$ and $p$  are, respectively, the energy density and the pressure in the fluid's rest frame.

Time-independence implies that $u^{a} = e^{-\Phi} (\frac{\partial }{\partial t})^{a}$.
The velocity is fixed by the symmetry assumptions, which also imply that $\rho = \rho(r)$  and $p = p(r)$  can only be functions of $r$.
Also  $\rho, p = 0$ for  $r > R$, where  $R$  is the radius of the star.

\section{Tolman--Oppenheimber--Volkoff Equations}%
\label{sec:tolman_oppenheimber_volkoff_equations}

Solving the Einstein equation imposes the fluid equations, so we do not separately need to deal with those.
However, in what follows, it will actually be slightly easier to derive one of the following equations by using the fluid equation $\nabla_{\mu} T^{\mu\nu} = 0$  instead of some components of the Einstein equations.

By symmetry, there are only really $3$  equations to solve. To write these down more concisely, we define $m(r)$  by the relation
\begin{equation}
  e^{2 \Psi(r)} = (1 - \frac{2m(r)}{r})^{-1} \stackrel{\text{LHS} > 0}{\implies} m(r) < r / 2.
\end{equation}

\begin{exercise}[Sheet 1]
  Using the $(\mu\nu)$  component of the Einstein equation, we can derive
  \begin{align}
    (tt): \qquad \dv{m}{r} &= 4 \pi r^2 \rho \tag{TOV 1} \label{eq:TOV1} \\
    (rr): \qquad \dv{\Phi}{r} &= \frac{m + 4 \pi r^3 \rho}{r (r - 2m)} \tag{TOV 2} \label{eq:TOV2} \\
    \nabla_{\mu} T^{\mu\nu} = 0 \qquad \dv{p}{r} &= -(p + \rho) \frac{m + 4\pi r^3 p}{r (r - 2m)} \tag{TOV 3} \label{eq:TOV3}
  \end{align}
  where instead of using a third Einstein equation, it is easiest to use $\nabla_{\mu} T^{\mu\nu} = 0$ to derive \eqref{eq:TOV3}.
\end{exercise}

We have three equations, but four unknowns $(m, \Phi, \rho, \pi)$ . However, luckily we have some extra information coming from \emph{thermodynamics}.

A cold star has $T = 0$ but  $T = T(\rho, p)$ , so $T = 0$ fixes some  relation $p = p(\rho)$ . This is known as a ``barotopic equation of state''.

We will not need much information about this relation. However, we will assume that  $\rho, p > 0$ and  $\dv{p}{\rho} >0$ .
\begin{leftbar}
  Else, we have an unstable fluid: an increase in density $\delta \rho > 0$ would cause a decrease in pressure $\delta p < 0$, which causes more fluid to flow into a given volume, causing in turn an even bigger increase in density.
\end{leftbar}

\section{Outside a Star: Schwarzschild Solution}%
\label{sec:outside_a_star_schwarzschild_solution}

Outside the star, at $r > R$, we have no matter and therefore  $\rho = p = 0$. 

From \eqref{eq:TOV1}, we then find that $m(r) = M$  is a constant.
One can then integrate \eqref{eq:TOV2} to find that $\Phi(r) = \frac{1}{2} \ln(1 - \frac{2M}{r}) + \Phi_0$ , where $\Phi_0$  is some constant of integration.

However, $\Phi_0$ is not physical: as  $r \to \infty$, $\Phi(r) \to \Phi_0$ , so $g_{tt} \to e^{-2 \Phi_0}$  as $r \to \infty$ .
This means that we can eliminate $\Phi_0$ by absorbing it into the time coordinate via the coordinate transform  $A' = e^{\Phi_0} t$ .
Without loss of generality, we may therefore set $\Phi_0 = 0$ .
The resulting metric is the \emph{Schwarzschild solution}
\begin{equation}
  ds^2 = - \left( 1 - \frac{2M}{r} \right) dt^2 + \left(1 - \frac{2M}{r}\right)^{-1} dr^2 + r^2 d\Omega^2.
\end{equation}

We interpret $M$  to be the mass of the star.

At $r = 2M$, the ``Schwarzschild radius'', the metric components  $g_{\mu\nu}$  (in a coordinate basis) are singular.
Since in our derivation every step was sound, this singularity must be inside the star, where the metric is not valid.
The star must have 
\begin{equation}
  \label{eq:2-ineq}
  R > 2M.
\end{equation}

\begin{remark}
  To get from GR to Newtonian physics, we take the limit of $c \to \infty$. The inequality
  \begin{equation}
    R > 2M \qquad \xleftrightarrow[\text{units}]{\text{reinstate}} \qquad \frac{GM}{c^2 R} < \frac{1}{2}
  \end{equation}
  then becomes trivial, meaning that there is no Newtonian analogue of this new GR effect.
\end{remark}

 \begin{leftbar}
  \begin{remark}
    This is not true for black holes: they violate the assumption of static spacetime.
  \end{remark}
\end{leftbar}

This inequality is certainly true for the sun, which has a Schwarzschild radius of $2 M_\odot \approx 3 km$ and a radius of $R_{\odot} \approx 7 \times 10^5$ km.

\section{Interior Solution}%
\label{sec:interior_solution}

From \eqref{eq:TOV1}, we have that
\begin{equation}
  m(r) = 4 \pi \int_{0}^r \rho(r') {r'}^2 \dd[]{r'} + m_{*} \label{eq:2-dag},
\end{equation}
where $m_*$ is some integration constant.

Let $\Sigma_t$  denote a surface of constant time $t$.
The metric induced on such a surface is
 \begin{equation}
  \left.ds^2\right\rvert_{\Sigma_t} = e^{2 \Psi(r)} dr^2 + r^2 d\Omega^2.
\end{equation}

We want the metric to be smooth at $r = 0$.
This implies that the spacetime is locally flat at  $r = 0$. For small  $r$ , this spacetime will look like Euclidean space $\mathbb{R}^3$.

As such, a point on $S^2$  of small radius $r$  must be a distance $r$  from the origin $r = 0$  (since this is true in $\mathbb{E}^3$).
For small $r$ we have
\begin{equation}
  \therefore \quad r \approx \int_0^r e^{\Phi(r')} \dd[]{r'} \approx e^{\Phi(0)} r \implies \Phi(0) = 0
\end{equation}
Else, there is some kind of singularity at the origin; the origin would not be smooth.

This means that $m(0) = 0 \implies m_* = 0$ in Eq.~\eqref{eq:2-dag}.

This was outside $R$. Continuity tells us that
\begin{equation}
  m(r) = M = 4 \pi \int_0^R \rho(r) r^2 \dd[]{r} \label{eq:2-star}.
\end{equation}
The fact that this is the same as in Newtonian physics is a coincidence; this is not in general true for general relativity.

More specifically, in general relativity, the total energy is obtained by integrating the energy density $\rho(r)$ over the appropriate volume form. On $\Sigma_t$, this is
 \begin{equation}
  e^{\Psi(r)} \underbrace{r^2 \sin \theta dr \wedge d\theta \wedge d\varphi}_{\mathclap{\text{usual volume form on } \mathbb{E}^3}}.
\end{equation}
The energy of matter on $\Sigma_t$ is then
\begin{equation}
  E = 4 \pi \int_0^R e^{\Psi(r)} \rho(r)r^2 \dd[]{r}.
\end{equation}
Since $m > 0$, we find that $e^{\Psi} > 1$.
Therefore $E > M$: the energy of the matter in the star is larger than the total energy of the star.
This means that there is some gravitational binding energy $E - M$.

Finally, reduces to \eqref{eq:TOV3} $\dv{p}{r} < 0$. Together with the previously mentioned assumption that $\dv{p}{\rho} > 0$, this implies that $\dv{\rho}{r} < 0$.
\begin{exercise}[Sheet 1]
  One can then show that
  \begin{equation}
    \frac{m(r)}{r} < \frac{2}{9} \left[ 1 - 6 \pi r^2 p(r) + (1 + 6 \pi r^2 p(r))^{1 / 2} \right] \label{eq:2-doubstar}
  \end{equation}
\end{exercise}
At $r = R$, the surface of the star, the pressure vanishes $p = 0$. This then reduces to the ``Buchdahl inequality''
\begin{equation}
  \label{eq:buchdahl}
  R > \frac{9}{4} M,
\end{equation}
which is an improvement on Eq.~\eqref{eq:2-ineq}.

Now \eqref{eq:TOV1} and \eqref{eq:TOV3} are coupled ordinary differential equations for $m$ and $\rho$ (via $p = p(\rho)$).
These can be solved numerically given initial conditions. Eq.~\eqref{eq:2-dag} automatically implies $m(0) = 0 $, so we only need to specify $\rho(0) = \rho_c$.

In particular, \eqref{eq:TOV3} implies that the pressure  $p$  decreases as we move out towards higher $r$.
We define the radius  $R$  by $p(R) = 0$  giving us $R = R(\rho_c)$ . Similarly, once we have done this Eq.~\eqref{eq:2-star} fixes $M = M(\rho_c)$ . Finally, we fix $\Phi$ by solving  \eqref{eq:TOV2} in $r < R$ with initial condition
 \begin{equation}
  \Phi(R) = \frac{1}{2} \ln(1 - \frac{2M}{R}).
\end{equation}
As such, for a given equation of state, cold stars form a one-parameter family labelled uniquely by the energy density $\rho_c$ at the center of the star.

\section{Maximum Mass of Cold Star}%
\label{sec:maximum_mass_of_cold_star}

The maximum mass $M_{\text{max}}$  depends on the equation of state.
\begin{leftbar}
  In particular, choosing the density of state for the degenerate electron gas gives the Chandrasekhar limit!
\end{leftbar}
\begin{figure}[tbhp]
  \centering
  \def\svgwidth{0.4\columnwidth}
  \input{lectures/l2f1.pdf_tex}
  \caption{}
  \label{fig:l2f1}
\end{figure}

Experimentally, we can only know the equation of state up to nuclear density $\rho_0$ .
\begin{claim}
  The maximum mass is always $M_{\text{max}} \lesssim 5 M_\odot$ whatever happens for $\rho > \rho_0$.
\end{claim}
\begin{proof}
  We know that $\rho$ decreases with $r$.
  Let us now define two regions:
  \begin{description}
    \item[core] region where $\rho > \rho_0$ ($r < r_0$)
    \item[envelope] region where $\rho < \rho_0$ ($r_0 < r < R$)
  \end{description}
  We then define the `core mass' to be $M_0 \coloneqq m(r_0)$.
  Then Eq.~\eqref{eq:2-dag} gives 
  \begin{equation}
    \label{eq:2-1}
    M_0 > \frac{4}{3} \pi r^3_0 \rho_0.
  \end{equation}
  The core mass has a higher density than nuclear density $\rho_0$.

  On the other hand, eq.~\eqref{eq:2-doubstar} for $r = r_0$ gives that $\frac{m_0}{r_0} < \frac{2}{9}\left[ 1 -6 \pi r_0^2 p_0 + (1 + 6 \pi r_0^2 p_0)^{\frac{1}{2}} \right]$, but we know the quantity $p_0 = p(r_0)$  from the equation of state.
  Now the right-hand side of this is a decreasing function of $p_0$ , so to simplify, we can evaluate this at $p_0 = 0$ . This then gives the \emph{Bookdahl bound}
  \begin{equation}
    \label{eq:2-2}
    m_0 < \frac{4}{9} r_0,
  \end{equation}
  which is satisfied by the core alone.
  We can of course get a sharper inequality by not restricting to $p_0 = 0$, but this is not needed here.
  \begin{figure}[tbhp]
    \centering
    \def\svgwidth{0.4\columnwidth}
    \input{lectures/l2f2.pdf_tex}
    \caption{}
    \label{fig:l2f2}
  \end{figure}
  Now the intersection of \eqref{eq:2-1} and \eqref{eq:2-2}, as illustrated in Fig.~\ref{fig:l2f2}, turns out to be
  \begin{equation}
    m_0 < \sqrt{\frac{16}{243 \pi \rho_0}} \simeq 5 M_\odot,
  \end{equation}
  where in evaluating this last expression we used the nuclear density $\rho_0$.

  For any $(m_0, r_0)$ in the allowed region, we solve \eqref{eq:TOV1} and \eqref{eq:TOV3} in the envelope region with $\rho = \rho_0$, $m = m_0$ at $r = r_0$. This fixes $M$ in terms of $(m_0, r_0)$.
  Numerically, we find that the total mass $M$ is maximised when the core mass $m_0$ is maximised.
  At this maximum, the envelope is small, so $M_{\text{max}} \lesssim 5 M_\odot$.
  If we possess extra information, we can lower this bound further. However, this result as it is holds independent of the densities at $\rho > \rho_0$.
\end{proof}

