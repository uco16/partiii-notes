% lecture notes by Umut Özer
% course: bh
\lhead{Lecture 7: January 31}

Examples of globally hyperbolic spacetimes include:
\begin{itemize}
  \item Minkowski spacetime ($t = \text{const.}$ is Cauchy)
  \item Kruskal spacetime
  \begin{figure}[tbhp]
    \centering
    \def\svgwidth{0.4\columnwidth}
    \input{lectures/l7f1.pdf_tex}
    \caption{}
    \label{fig:l7f1}
  \end{figure}
  \item spherical gravitational collapse %F2
\end{itemize}

Non-globally hyperbolic spacetimes are:
\begin{itemize}
  \item Minkowski with the origin removed: There is no Cauchy surface %F3
\end{itemize}

Globally hyperbolic spacetimes are \emph{nice} in the following sense:
\begin{theorem}[]
  Let $(M, g)$ be globally hyperbolic. Then 
  \begin{enumerate}[(i)]
    \item There exists a \emph{global time function} $t \colon M \to \mathbb{R}$ such that $-(dt)^{a}$ is future-directed timelike.
    \item The constant-$t$ surfaces are Cauchy and all have the same topology $\Sigma$.
    \item $M$ has topology $\mathbb{R} \times \Sigma$.
  \end{enumerate}
\end{theorem}

\begin{exercise}
  Show that the function $U + V$ is a global time function for the Kruskal spacetime.
\end{exercise}
In particular, the surface $U + V = 0$, a straight line through the origin in Kruskal spacetime, is an Einstein--Rosen bridge. As illustrated in Fig.~\ref{fig:l6f2}, we can think of this as a cylinder $\Sigma \simeq \mathbb{R} \times S^2$ . Therefore, the manifold is $M \simeq \mathbb{R}^2 \times S^2$ .

Let $x^{i}$  be coordinates on the $t = 0$ surface $\Sigma$, and let  $T^{a}$  be an arbitrary timelike vector field.
If we now pick an arbitrary $p \in M$, an integral curve of  $T^{a}$  through $p$  intersects $\Sigma$  at a unique point. Let the coordinates of that point be $x^{i}(p)$ .
\begin{figure}[tbhp]
  \centering
  \def\svgwidth{0.4\columnwidth}
  \input{lectures/l7f4.pdf_tex}
  \caption{}
  \label{fig:l7f4}
\end{figure}
This defines a map $x^{i} \colon M \to \mathbb{R}$, shown in Fig.~\ref{fig:l7f4}.

Use $(t, x^{i})$ as coordinates on $M$.
We write the metric as 
\begin{equation}
  ds^2-N -N^2 dt^2 h_{ij} (dx^{i} + N^{i} dt) (dx^{j} + N^{j} dt).
\end{equation}
We call $N(t, x)$ the \emph{lapse function} and $N^{i}(t, x)$ the \emph{shift vector}. Finally, $h_{ij}(t, x)$ is the metric on the surface of constant $t$.

\section{Extrinsic Curvature}%
\label{sec:extrinsic_curvature}

\begin{definition}[spacelike surface]
  A hypersurface $\Sigma$ is \emph{spacelike} if its normal $n_{a}$ is everywhere timelike.
\end{definition}

If $X^{a}$  is a vector that is tangent to the surface, then by definition $n_{a} X^{a} =0$  and if $n_{a}$ is timelike, then $X^{a}$  is spacelike.
Any vector tangent to the spacelike hypersurface is spacelike.

Assume that we have normalised the normal to be a unit vector, $n_{a} n^{a} = -1$ . Then define 
\begin{equation}
  h\indices{^{a}_{b}} \coloneqq \delta^{a}_{b} + n^{a} n_{b}.
\end{equation}
Lowering indices gives $h_{ab} = g_{ab} + n_{a} n_{b}$ , so it is a symmetric tensor.

If $X^{a}$ , $Y^{a}$  are tangent vectors, then
\begin{equation}
  h_{ab} X^{a} Y^{b} = g_{ab} X^{a} Y^{a}.
\end{equation}
Therefore, $h_{ab}$  is the \emph{induced metric} on $\Sigma$  (the pull-back of $g_{ab}$).

Since $h\indices{^{a}_{b}} n^{b} = 0$ and $h\indices{^{a}_{c}} h\indices{^{c}_{b}} = h\indices{^{a}_{b}}$ , the  $h\indices{^{a}_{b}}$  is a \emph{projection} onto $\Sigma$.

Using this tensor, we can decompose any vector into a parallel and perpendicular component
 \begin{equation}
  X^{a} = \delta^{a}_{b} X^{b} = \underbrace{h\indices{^{a}_{b}} X^{b}}_{\mathclap{X^{a}_{\parallel}}} - \underbrace{n^{a} n_{b} X^{b}}_{\mathclap{X^{a}_{\perp}}}.
\end{equation}

\begin{figure}[tbhp]
  \centering
  \def\svgwidth{0.4\columnwidth}
  \input{lectures/l7f5.pdf_tex}
  \caption{}
  \label{fig:l7f5}
\end{figure}

Let $N_a$  be perpendicular to $\Sigma$  at $p$.
Parallel transport  $N_a$  along $C \colon X^{b} \nabla_{b} N_{a} = 0$ .
Does $N_a$  remain perpendicular to $\Sigma$?
Let $Y^{a}$  be tangent to $\Sigma$. T hen
\begin{equation}
  X(N \cdot Y) = X^{b} \nabla_b (Y^{a} N_a) = N_{a} X^{b} \nabla_b Y^{a}.
\end{equation}
If $N \cdot Y \equiv 0$ , then $(\nabla_X Y)_{\perp} = 0$ .

\begin{definition}[extrinsic curvature tensor]
  Extend $n_a$ to a neighbourhood of $\Sigma$, with $n_a n^a = -1$.
  The \emph{extrinsic curvature tensor} $K_{ab}$ is defined for $p \in \Sigma$ by $K(X, Y) = -n_a (\nabla_{X_\parallel} Y_{\parallel})^{a}$.
\end{definition}

\begin{lemma}
  Independent of the extension of $n_a$, we have
  \begin{equation}
    K_{ab} = h\indices{_{a}^{c}} h\indices{_{b}^{d}} \nabla_c n_d.
  \end{equation}
\end{lemma}
\begin{proof}
  Using the definition of the parallel components, 
  \begin{equation}
    -n_d X^{c}_\parallel \nabla_c Y^{d}_\parallel = -X^{c}_\parallel \nabla_c (\cancel{n_d Y_\parallel^d}) + X^{c}_\parallel Y_\parallel^{d} \nabla_c n_d = (h\indices{_{a}^{c}} h\indices{_{b}^{d}} \nabla_c n_d) X^{a} Y^{b}.
  \end{equation}
\end{proof}
\begin{remark}
  $n^{b} \nabla_c n_b = \frac{1}{2} \nabla_c (n_b n^b) = 0$, so the second index is automatically a tangential index.
  We have
  \begin{equation}
    K_{ab} = h\indices{_{a}^{c}} \nabla_c n_b.
  \end{equation}
\end{remark}

\begin{lemma}
  The extrinsic curvature is a symmetric tensor
  \begin{equation}
    K_{ab} = K_{ba}.
  \end{equation}
\end{lemma}
\begin{proof}
  Let $\Sigma$  be a surface of constant $f$, with $df\rvert_\Sigma \neq 0$ .
  Then there is some $g$ (fixed by $n_a n^a = -1$) such that $N_a \rvert_{\Sigma} = g(df)_a$ .
  Use this to extend $n_a$  off $\Sigma$.
   \begin{equation}
    \nabla_c n_d = g \nabla_c \nabla_d f + \nabla_c g \underbrace{\nabla_d f}_{\mathclap{g^{-1} n_d}} \implies K_{ab} = g h\indices{_{a}^{c}} h\indices{_{b}^{d}} \nabla_c \nabla_d f.
  \end{equation}
\end{proof}

\begin{lemma}
  There is also a definition in terms of the Lie derivative
  \begin{equation}
    K_{ab} = \frac{1}{2} \mathcal{L}_n h_{ab},
  \end{equation}
  with respect to $n^a$.
\end{lemma}
\begin{proof}
  Exercise sheet 2.
\end{proof}

We think of the extrinsic curvature intuitively as how the manifold is bending.
Of course we also have the intrinsic curvature defined by the metric. The question of how they are related is the matter of the following section.

\section{Gauss--Codacci Equations}%
\label{sec:gauss_codacci_equations}

\begin{leftbar}
  Proofs not examinable and will be skipped but can be found in lecture notes.
\end{leftbar}

\begin{definition}[invariant]
  A tensor at $p \in \Sigma$ is \emph{invariant under projection} $h\indices{^{a}_{b}}$ if
  \begin{equation}
    T\indices{^{a_1 \dots a_r}_{b_1 \dots b_s}} = h\indices{^{a_1}_{c_1}} \dots h\indices{^{a_r}_{c_r}} h\indices{^{d_1}_{b_1}} \dots h\indices{^{d_s}_{b_s}} T\indices{^{c_1 \dots c_r}_{d_1 \dots d_s}}
  \end{equation}
  can be identified with tensors on $\Sigma$.
\end{definition}
\begin{definition}[covariant derivative]
  The \emph{covariant derivative} on $\Sigma$ is 
  \begin{equation}
    D_a T \indices{^{b_1 \dots b_r}_{c_1 \dots c_s}} = h\indices{_{a}^{d}} h\indices{^{b_1}_{e_1}} \dots h\indices{^{b_r}_{e_r}} h\indices{^{f_1}_{c_1}} \dots h\indices{^{f_s}_{c_s}} \nabla_d T \indices{^{e_1 \dots e_r}_{f_1 \dots f_s}}
  \end{equation}
\end{definition}

\begin{lemma}
  The covariant derivative $D_a$ is the Levi--Civita connection associated with $h_{ab}$.
\end{lemma}
\begin{claim}
  The Riemann tensor of $D_a$ is given by \emph{Gauss' equation}
  \begin{equation}
    R'{} \indices{^{a}_{bcd}} = h\indices{^{a}_{e}} h\indices{^{f}_{b}} h\indices{^{g}_{c}} h\indices{^{h}_{d}} R\indices{^{e}_{fgh}} -2 K\indices{_{[c}^{a}} K_{d] b}.
  \end{equation}
\end{claim}
\begin{lemma}
  The Ricci scalar of $D_a$ is
  \begin{equation}
    \label{eq:7-star}
    R' = R + 2 R\indices{_{ab}} n^{a} n^{b} - K^2 + K^{ab} K_{ab},
  \end{equation}
  where $K = K\indices{^{a}_{a}} = g^{ab} K_{ab} = h^{ab} K_{ab}$, where the contraction can be done with either metric since it is purely tangential.
\end{lemma}
\begin{claim}
  \emph{Codacci's equation}
  \begin{equation}
    D_a K_{bc} - D_{b} K_{ac} = h\indices{_{a}^{d}} h\indices{_{b}^{e}} h\indices{_{c}^{f}} n^{g} R_{defg}.
  \end{equation}
\end{claim}
\begin{lemma}
  Taking a contraction of this, we get
  \begin{equation}
    \label{eq:7-dag}
    D_a K\indices{^{a}_{b}} - D_{b} K = h\indices{^{c}_{b}} R^{cd} n^{d}.
  \end{equation}
\end{lemma}

\section{The Constraint Equations}%
\label{sec:the_constraint_equations}

Take Einstein's equations $G_{ab} = 8 \pi T_{ab}$. Contract with $n^{a}$ to get
\begin{equation}
  G_{ab} n^{a} n^{b} = R_{ab} n^{a} n^{b} + \frac{1}{2} R.
\end{equation}
Observe that we get this in Eq.~\eqref{eq:7-star}. Can rewrite this as 
\begin{equation}
  G_{ab} n^{a} n^{b} = \frac{1}{2} (R' -K^{ab} K_{ab} + K^2),
\end{equation}
rewritten in terms of intrinsic and extrinsic curvature of our hypersurface.

Then considering the right-hand side gives the \emph{Hamiltonian constraint}
\begin{equation}
  R' - K^{ab} K_{ab} + K^2 = 16 \pi \rho,
\end{equation}
where $\rho = T_{ab} n^{a} n^{b}$ is the energy density seen by an observer that moves with velocity $n^{a}$.
We see $K$ like a time derivative, so this is not an evolution equation. Instead, it is a constraint.
This comes from the normal-normal components of the Einstein equation.
Considering instead the 
\begin{equation}
  8 \pi h\indices{_{a}^{b}} T_{bc} n^{c} = h\indices{_{a}^{b}} G_{bc} n^{c} = h\indices{_{a}^{b}} R_{bc} n^{c},
\end{equation}
then Eq.~\eqref{eq:7-star} now tells us that
\begin{equation}
  D_b K\indices{^{b}_{a}} - D_{a} K = 8 \pi h\indices{_{a}^{b}} T_{bc} n^c,
\end{equation}
where $h\indices{_{a}^{b}} T_{bc} n^{c}$ is (minus) the momentum density seen by an observer with velocity $n^a$.
This can be seen as an equation involving a space and a time derivative, rather than two time derivatives. It is therefore not an evolution equation. We call this the \emph{momentum constraint}.
The remaining equations have two time derivatives and give the time-evolution.
