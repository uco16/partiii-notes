% lecture notes by Umut Özer
% course: bh
\lhead{Lecture 6: January 29}

\section{Einstein--Rosen Bridge}%
\label{sec:einstein_rosen_bridge}

Taking $t$ constant in I corresponds to $V / U$ being constant. This extends into IV.

Let $r = \rho + M + \frac{M^2}{4 \rho}$ and choose $\rho > \frac{M}{2}$ in I and $0 < \rho < \frac{M}{2}$ in IV.
\begin{figure}[tbhp]
  \centering
  \def\svgwidth{0.4\columnwidth}
  \input{lectures/l6f1.pdf_tex}
  \caption{}
  \label{fig:l6f1}
\end{figure}
\begin{exercise}
  Show that the Schwarzschild metric in \emph{isotropic coordinates} $(t, \rho, \theta, \phi)$  is
  \begin{equation}
    ds^2 = -\frac{\left( 1 - \frac{M}{2\rho} \right)^2}{\left( 1 + \frac{M}{2 \rho} \right)^2} dt^2 + \left( 1 + \frac{M}{2\rho} \right)^4 (d\rho^2 + \rho^2 d\Omega^2).
  \end{equation}
\end{exercise}

Taking $\rho \to \frac{M^2}{4 \pi}$ is an isometry I $\leftrightarrow$ IV.

For $t$ constant, we have 
\begin{equation}
  ds^2 = \left( 1 + \frac{M}{2\rho} \right)^4 (d\rho^2 + \rho^2 d\Omega^2),
\end{equation}
which is smooth $\forall \rho > 0$.

This gives us an \emph{Einstein--Rosen bridge} connecting two far-away regions of spacetime.
\begin{figure}[tbhp]
  \centering
  \def\svgwidth{0.4\columnwidth}
  \input{lectures/l6f2.pdf_tex}
  \caption{Einstein--Rosen bridge}
  \label{fig:l6f2}
\end{figure}

\section{Extendibility}%
\label{sec:extendibility}

\begin{definition}[extendible]
  A Riemannian manifold $(M, g)$ is \emph{extendible} if it is isometric to a proper subset of another spacetime $(M', g')$, which is called an \emph{extension} of $(M, g)$.
\end{definition}
\begin{example}[]
  Let $(M, g)$ be the $r> 2M$ Schwarzschild spacetime.
  Then $(M', g')$ can be taken to be the Kruskal extension of $(M, g)$.
  Kruskal itself is \emph{inextendible}---it is a \emph{maximal analytic extension} of $(M, g)$.
\end{example}

\section{Singularities}%
\label{sec:singularities}

\begin{definition}[singular]
  The metric $g_{\mu\nu}$ is \emph{singular} if it is not smooth or $\det g_{\mu\nu} = 0$ somewhere.
\end{definition} 
There are different kinds of singularity:
\begin{description}
  \item[Coordinate singularities] can be eliminated via a change of coordinates (e.g.~at $r = 2M$ in Schwarzschild coordinates). This is not a physical singularity.
  \item[Scalar curvature singularities] are points where the scalar built from $R\indices{^{a}_{bcd}}$ diverges (e.g.~at $r = 0$ in Schwarzschild spacetime).
  \item[Non-curvature singularities] 
\end{description}

\subsection{Conical Singularity}%

Let us give an example of a non-curvature singularity.
Let $M = \mathbb{R}^2$ be a manifold with metric $g = dr^2 + \lambda^2 r^2 d\phi^2$ in polar coordinates $(r, \phi)$, with the identification $\phi \sim \phi + 2\pi$.

For $\lambda >0$, we have a singularity $\det g_{\mu\nu} = 0$ at $r = 0$.
If $\lambda = 1$, then this is just Euclidean space $\mathbb{E}^2$ and we can switch to Cartesian coordinates, where the metric has $\det g_{\mu\nu} = 1$; in this case, $r = 0$ is a coordinate singularity.
However, whenever $\lambda \neq 1$, changing coordinates $\phi' = \lambda \phi$ gives the metric $g = dr^2 + r^2 d\phi'{}^2$, which is locally isometric to $\mathbb{E}^2$; in this case $r = 0$ is clearly not a coordinate singularity.
Moreover, the curvature tensor $R\indices{^{a}_{bcd}} = 0$ vanishes everywhere, which means that the singularity cannot be a curvature singularity either.

Now the change of coordinates means that the new angular coordinates has a different period $\phi' \sim \phi' + 2\pi \lambda$, so this is spacetime is not globally isometric to $\mathbb{E}^2$.  
Take a circle with radius $r = \epsilon$. The ratio of circumference and radius is 
\begin{equation}
  \frac{\text{circumference}}{\text{radius}} = \frac{2 \pi \lambda \epsilon}{\epsilon} = 2 \pi \lambda \not\longrightarrow 2 \pi \text{ as } \epsilon \to 0.
\end{equation}
As such, the manifold is not locally flat at $r = 0$ and the metric cannot be smoothly extended to $r = 0$. We call this a \emph{conical singularity}.

\subsection{Geodesic Compleneness}%
\label{sub:geodesic_compleneness}

\begin{definition}[future endpoint]
  A point $p \in M$ is a \emph{future endpoint} of a future-directed causal curve $\gamma \colon (a, b) \to M$ if, for any neighbourhood $\mathcal{O}$ of $p$, there exists a value $t_0$ such that $\gamma(t) \in \mathcal{O}$ for all $t > t_0$.
\end{definition}
\begin{figure}[tbhp]
  \centering
  \def\svgwidth{0.25\columnwidth}
  \input{lectures/l6f3.pdf_tex}
  \caption{Future endpoint}
  \label{fig:l6f3}
\end{figure}
\begin{definition}[future-inextendible]
  A curve $\gamma$ is \emph{future-inextendible} if it has no future endpoint.
\end{definition}

\begin{example}[]
  Take Minkowski space $\mathbb{M}^4$ with the curve $\gamma \colon (-\infty, 0) \to M$ defined by $\gamma(t) = (t, 0, 0, 0)$. Then $(0,0,0,0)$ is a future endpoint.
  Taking the manifold $\mathbb{M}^4 \setminus \{(0,0,0,0)\}$, $\gamma$ becomes future-inextendible.
\end{example}

\begin{definition}[complete]
  A geodesic is \emph{complete} if an affine parameter extends to $\pm \infty$.
\end{definition}
\begin{definition}[geodesically complete]
  A Riemannian manifold $(M, g)$ is \emph{geodesically complete} iff all inextendible causal geodesics are complete.
\end{definition}
\begin{example}[]
  Minkowski spacetime with a static spherical star is geodesically complete.
\end{example}
\begin{example}[]
  Kruskal is geodesically incomplete---geodesics reach $r = 0$ in finite affine parameter.
\end{example}

An extendible spacetime is geodesically incomplete in a boring way; we can just make it into a bigger spacetime.
This motivates the following definition:
\begin{definition}[singular]
  A spacetime $(M, g)$ is \emph{singular} if it is inextendible and geodesically incomplete.
\end{definition}
\begin{example}[]
  Kruskal is singular.
\end{example}

\chapter{The Initial Value Problem}%
\label{cha:the_initial_value_problem}

\section{Predictability}%
\label{sec:predictability}

\begin{definition}[partial Cauchy surface]
  Let $(M, g)$ be time-orientable. A \emph{partial Cauchy surface} $\Sigma$ is a hypersurface such that no two points are connected by a causal curve in $M$.
\end{definition}
\begin{definition}[domain of dependence]
  The \emph{future (past) domain of dependence} of $\Sigma$, denoted $D^{+ (-)}(\Sigma)$, is the set of points $p \in M$ such that every past-(future-)inextendible causal curve through $p$ intersects $\Sigma$.
  The full domain of dependence is $D(\Sigma) = D^+(\Sigma) \cup D^-(\Sigma)$.
\end{definition}

Any causal geodesic in $D(\Sigma)$  must intersect $\Sigma$ . Therefore, it is determined uniquely by a tangent vector on $\Sigma$.
\begin{leftbar}
  We can predict particle trajectories, say, on $D(\Sigma)$ by specifying initial data on $\Sigma$.
\end{leftbar}
\begin{figure}[tbhp]
  \centering
  \def\svgwidth{0.4\columnwidth}
  \input{lectures/l6f4.pdf_tex}
  \caption{}
  \label{fig:l6f4}
\end{figure}

\begin{definition}[hyperbolic PDEs]
  Take fields $T \indices{^{(i)a b \dots}_{c d \dots}}$, where $i = 1, \dots, N$, which obey equations of motion
  \begin{equation}
    g^{ef} \nabla_{e} \nabla_{f} T \indices{^{(i)a b \dots}_{c d \dots}} = \dots.
  \end{equation}
  The right hand side depends on $g$ and its derivatives. It depends \emph{linearly} on $T^{(i)}$ and their \emph{first} derivatives.
\end{definition}
\begin{example}[]
  The Klein--Gordon, and Maxwell equations in Lorentz gauge are Hyperbolic PDEs. This is a large class of equations encompassing most physical equations of motion.
\end{example}

Solutions of such equations are uniquely determined in $D(\Sigma)$  from initial data on $\Sigma$.
 \begin{example}[]
   Take $\mathbb{M}^2$  and $\Sigma$  to be the positive $x$-axis. This is illustrated in \ref{fig:l6f5}.
   \begin{figure}[tbhp]
     \centering
     \begin{minipage}[t]{0.5\columnwidth}
       \centering
       \def\svgwidth{0.8\columnwidth}
       \input{lectures/l6f5.pdf_tex}
       \caption{}
       \label{fig:l6f5}
     \end{minipage}%
     \begin{minipage}[t]{0.5\columnwidth}
       \centering
       \def\svgwidth{0.8\columnwidth}
       \input{lectures/l6f6.pdf_tex}
       \caption{}
       \label{fig:l6f6}
     \end{minipage}
   \end{figure}
   Taking $\Sigma'$ to be the whole $x$-axis, we have the case illustrated in \ref{fig:l6f6}.

  Consider the wave equation $\nabla^{a} \nabla_{a} \psi = -\partial_t^2 \psi + \partial_x^2 \psi = 0$.
  The solutions in $D(\Sigma)$ ($M$) are uniquely determined by data in $(\psi, \partial_t \psi)$ on $\Sigma$ ($\Sigma'$).
\end{example}

Generally, if $D(\Sigma) \neq M$ , then physics in $M \setminus D(\Sigma)$  is not determined by data on $\Sigma$.

 \begin{definition}[]
   A spacetime $(M, g)$ is \emph{globally hyperbolic} if there exists a \emph{Cauchy surface}---a partial Cauchy surface such that $D(\Sigma) = M$.
\end{definition}
\begin{definition}[]
  The \emph{Cauchy horizon} is the boundary of $D(\Sigma)$ in $M$.
\end{definition}

A spacetime $(M, g)$ is globally hyperbolic iff no Cauchy horizon for $\Sigma$.
