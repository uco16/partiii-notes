% lecture notes by Umut Özer
% course: bh
\lhead{Lecture 16: February 21}

\section{Ergosphere \& Penrose Process}%
\label{sec:ergosphere_&_penrose_process}

Let us look, in BL coorinates, at the norm of the Killing field:
\begin{equation}
  k^2 = g_{tt} = -\frac{(\Delta - a^2 \sin^2 \theta)}{\Sigma} = \leftrightarrow(1 - \frac{2 M r}{r^2 + a^2 \cos^2 \theta}).
\end{equation}
$k$ is timelike when $r^2 - 2M r + a^2 \cos^2 \theta > 0$.
This happens for $r > M + \sqrt{M^2 - a^2 \cos^2 \theta}$. This is greater than $r_+$. This means that there is a region $r_+ = M + \sqrt{M^2 -a^2} < r < M + \sqrt{M^2 - a^2 \cos^2\theta}$ outside the event horizon $\mathcal{H}^+$ where $k$ is spacelike.
\begin{figure}[ht]
    \centering
    \inkfig[0.4]{l16f1}
    \caption{Ergosphere of a Kerr black hole.}
    \label{fig:l16f1}
\end{figure}
This region, depicted in Fig.~\ref{fig:l16f1}, is called the \emph{ergosphere}.
Its boundary is the \emph{ergosurface}. A stationary observer $(u^{a} \parallel k^{a})$ cannot exist in the ergosphere. Any causal curve in the ergosphere must rotate relative to $\infty$, in the same direction as the black hole.

Consider a particle with $4$-momentum $P^{a} = \mu u^{a}$, where $\mu$ is its rest mass and $u^{a}$ its $4$-velocity.
Along a geodesic $E = -k \cdot P$ is conserved. This is the energy according to an observer at infinity.
Assume the particle decays inside the ergosphere. 
\begin{equation}
  \begin{gathered}
    \feynmandiagram[transform shape, scale=1][horizontal=a to b, layered layout] {
      a -- [fermion, edge label=$P^a$] b [dot, label=$p$] -- [fermion, edge label=$P_1^a$] c,
      b -- [fermion, edge label=$P_2^a$] d,
    };
  \end{gathered}
\end{equation}
The $4$-momentum is conserved at $p$:
\begin{equation}
  P^{a} = P^{a}_1 + P^{a}_2 \implies E = E_1 + E_2, \qquad E_i = -k \cdot P_i.
\end{equation}
Since $k$ is spacelike in the ergosphere, we can have $E_1 < 0$, which means that we can have $E_2 = E + \abs{E_1} > E$. Particle $1$ must fall into the hole, but $2$ can escape to $\infty$, carrying more energy than the initial particle! Particle 1 carries negative energy into the hole, so the energy (mass) of the hole decreases.
This extraction of energy from the black hole is called the \emph{Penrose process}.

Particle crossing $\mathcal{H}^+$: 
\begin{equation}
  -P \cdot \xi \geq 0, 
\end{equation}
since both $P^{a}, \xi^{a}$ are future-directed causal.
\begin{equation}
  \xi = k + \Omega_H m \qquad E - \Omega_H L \geq 0 \qquad L = m c P,
\end{equation}
where $L$ is the angular momentum of the particle. 
Therefore
\begin{equation}
  \therefore L  \leq E / \Omega_H \qquad \delta M = E, \quad \delta J = L
\end{equation}
\begin{equation}
  \therefore \delta J \leq \delta M / \Omega_H =  \frac{2 M (M^2 + \sqrt{M^4 - J^2})}{J} \delta M.
\end{equation}
\begin{exercise}
  Show that this inequality is equivalent to $\delta M_{\text{irr}} \geq 0$, where 
  \begin{equation}
    M_{\text{irr}} = \left[ \frac{1}{2} (M^2 + \sqrt{M^4 - J^2}) \right]^{\frac{1}{2}}
  \end{equation}
  is the \emph{irreducible mass}. In other words, in the Penrose process $M_{\text{irr}}$ cannot decrease.
\end{exercise}
Inverting this gives
\begin{equation}
  M^2 = M^2_{\text{irr}} + \frac{J^2}{4 M^2_{\text{irr}}} \geq M^2 _{\text{irr}}.
\end{equation}
The Penrose process cannot reduce $M$ below the initial value of $M_{\text{irr}}$; it gives an upper bound on the energy we can extract.
\begin{exercise}
  The quantity $A = 16 \pi M^2_{\text{irr}}$ is the \emph{area of the event horizon} (area of $\mathcal{H}^+ \cap \Sigma$, where $\Sigma$ is a partial Cauchy surface, e.g.~$v = \text{const.}$)
\end{exercise}
This means $\delta A \geq 0$ in the Penrose process (special case of the 2\textsuperscript{nd} law of black hole mechanics).
\begin{equation}
  A = 8 \pi (M^2 + \sqrt{M^4 - J^2}).
\end{equation}

\chapter{Charge, Mass, and Angular Momentum}%
\label{cha:charge_mass_and_angular_momentum}

\section{Charges in Curved Spacetime}%
\label{sec:charges_in_curved_spacetime}

\begin{definition}[]
  \label{def:charges}
  Let $(\Sigma, h_{ab}, K_{ab})$ be an asymptotically flat end. The \emph{electric and magnetic charges} associated to this end are 
  \begin{equation}
    Q = \frac{1}{4 \pi} \lim_{r \to \infty} \int_{S^2_r} \star F,  \qquad
    P = \frac{1}{4 \pi} \lim_{r \to \infty} \int_{S^2_r} F,
  \end{equation}
  where $S^2_r$ is the sphere $x^{i} x^{i} = r^2$ where $x^{i}$ are the coordinates in the definition of the asymptotically flat end.
\end{definition}
\begin{leftbar}
  We can have a non-zero charge even though no matter is present. In a sense, the topology of the spacetime can support charge, such as in the RN solution.
\end{leftbar}

\section{Komar Integrals}%
\label{sec:komar_integrals}

In a stationary spacetime $(M, g)$, we can define $J_a = - T_{ab} k^{b}$, the conserved energy-momentum current.
\begin{equation}
  \nabla_a J^{a} = 0 \qquad \iff \qquad d \star J = 0.
\end{equation}
Given this conserved current, we can define the related conserved charge, interpreted as the total energy of matter on a spacelike hypersurface $\Sigma$
\begin{equation}
  E[\Sigma] = -\int_\Sigma \star J.
\end{equation}
\begin{wrapfigure}{R}{0.3\columnwidth}
  \centering
  \inkfig[0.2]{l16f2}
  \caption{}
  \label{fig:l16f2}
\end{wrapfigure}
It is conserved in the sense that if we have Fig.~\ref{fig:l16f2}, then 
\begin{equation}
  E[\Sigma'] - E[\Sigma] = - \int_{\partial R} \star J = -\int_R d \star J  = 0.
\end{equation}
Consider the 1-form
\begin{equation}
  (\star d \star k)_a = - \nabla^{b} (d k)_{ab},
\end{equation}
using the identity
\begin{multline}
  \epsilon^{a_1 \dots a_p c_{p+1} \dots c_{n}} \epsilon_{b_1 \dots b_p c_{p+1} \dots c_p} = -p! (n-p)! \delta^{a_1}_{[b_1} \dots \delta^{a_p}_{b_p]} \\
  \implies (\star d \star X)_{a_1 \dots a_{p-1}} = -(-1)^{p (n-p)} \nabla^{b} X_{a_1 \dots a_{p-1} b}.
\end{multline}
Using the Killing vector field, we have
\begin{equation}
  (\star d \star d k)_a = -\nabla^{b} \nabla_{a} k_b + \nabla^{b} \nabla_{b} k_a = 2 \nabla^{b} \nabla_{\beta} k_a.
\end{equation}
\begin{lemma}
  For a Killing vector field $k^{a}$, we have 
  \begin{equation}
    \nabla_a \nabla_b k^{c} = R\indices{^{c}_{bad}} k^{d}.
  \end{equation}
\end{lemma}
Thus, using the Einstein equation, and assuming that $T$ is the energy-momentum tensor, we have
\begin{equation}
  (\star d \star d k)_a = -2 R_{ab} k^{b} = 8 \pi J'_a, \qquad J'_a = -2 (T_{ab} - \frac{1}{2} T\indices{^{c}_{c}} g_{ab}) k^{b}.
\end{equation}
Since $d \star d k = 8 \pi \star J'$, we find that $\star J'$ is \emph{exact} ($\implies $ conserved $d \star J' =0$).
Consider a surface $\Sigma$ with a single asymptotically flat end, 
\begin{equation}
  \label{eq:16-star}
  \therefore -\int_\Sigma \star J' = -\frac{1}{8 \pi} \int_\Sigma d \star d k = -\frac{1}{8 \pi} \int_{\partial \Sigma} \star dk.
\end{equation}
This equation tells us something about the energy of matter.
\begin{exercise}
  Consider a static, spherically symmetric perfect fluid star. Take $\Sigma = \{t = \text{const.} \suchthat r \leq r_0\}$, $r_0 > R$.
  Show that the RHS of \eqref{eq:16-star} is $M$.
  Show that in the Newtonian limit, where $P \ll \rho$, $\abs{\Phi} \ll 1$, $\abs{\Psi} \ll 1$, the LHS really is the total mass of the fluid.
\end{exercise}

\begin{definition}[Komar mass]
  Let $(\Sigma, h_{ab}, K_{ab})$ be an asymp.~flat end in a stationary spacetime. The \emph{Komar mass} (or energy) is
  \begin{equation}
    M_{\text{Komar}} = -\frac{1}{8\pi} \lim_{r \to \infty}  \int_{S^2_r} \star dk.
  \end{equation}
\end{definition}
This is entirely analogous to the definition \ref{def:charges} of electric and magnetic charges.
This is the total mass of matter and gravity in the spacetime.
We can do this for any Killing vector field. In particular, we could have done this for an axisymmetric spacetime.
\begin{definition}[Komar angular momentum]
  Let $(\Sigma, h_{ab}, K_{ab})$ be an asymp.~flat end in an axisymmetric spacetime. The \emph{Komar angular momentum} is
  \begin{equation}
    J_{\text{Komar}} = \frac{1}{16 \pi} \lim_{r \to \infty}  \int_{S^2_r} \star dm.
  \end{equation}
\end{definition}
The normalisation ensures that this reduces to Newtonian angular momentum in the appropriate limit.

It is not obvious why this should be the definition of energy or angular momentum.
Classically, the energy is the value of the Hamiltonian. Surely the `correct' way to understand energy then is to understand the Hamiltonian treatment of general relativity.
As we will see, this `correct' definition agrees with the above in a stationary spacetime, and these are ultimately the more useful for calculations.

\section{Hamilonian Formulation of GR}%
\label{sec:hamilonian_formulation}

Set $16 \pi G = 1$.  In a $3 + 1$ split, we define the lapse function and the shift and we have metric
\begin{equation}
  ds^2 = N^2 dt^2 + h_{ij} (dx^{i} + N^{i} dt) (dx^{j} + N^{j} dt).
\end{equation}
\begin{equation}
  S = \int \dd[4]{x} \sqrt{-g } R = \int \dd[]{t} \dd[3]{x} \sqrt{h} N ({}^{(3)} R + K_{ij} K^{ij} - K^2) = \int \dd{t} \dd[3]{x} \mathscr{L}.
\end{equation}

\begin{exercise}[Sheet 2]
  A formula for the extrinsic curvature tensor is
  \begin{equation}
    K_{ij} = \frac{1}{2 N} (\dot{h}_{ij} - D_{i} N_j - D_j N_i),
  \end{equation}
  where the dot denotes time-derivative.
\end{exercise}
\begin{remark}
  $S$ is independent of the time derivatives of the lapse and the shift, $\dot{N}, \dot{N}^{i}$. Hamilton's equations are constraints
  \begin{equation}
    \frac{\delta S}{\delta N} \rightarrow \text{Ham.~constraint}, \qquad 
    \frac{\delta S}{\delta N^{i}} \rightarrow \text{momentum constraint}
  \end{equation}
  They are not really dynamical variables; they are a gauge choice that we can choose to whatever is convenient.
  On the other hand $\frac{\delta S}{\delta h_{ij}}$ gives evolution equations for $h_{ij}$.

  The conjugate momenta of the lapse and shift are zero, since there are no time derivatives.
  We are left with
  \begin{equation}
    \pi^{ij} = \frac{\delta S}{\delta \dot{h}_{ij}} = \sqrt{h} (K^{ij} - K h^{ij}),
  \end{equation}
  where we used that $\dot{h}_{ij}$ only lives within $K_{ij}$.
  Because of the square root of the determinant, this is a tensor density rather than a tensor.
\end{remark}
