% lecture notes by Umut Özer
% course: bh
\lhead{Lecture 3: January 22}

\chapter{The Schwarzschild Black Hole}%
\label{cha:the_schwarzschild_black_hole}

In contrast to cold stars, which cannot have masses more than a few times $M_\odot$, hot stars will undergo complete gravitational collapse to form a  \emph{black hole}.
The simplest black hole solution is described by the Schwarzschild metric, which we will assume to be valid everywhere in this chapter.

\section{Birkhoff's Theorem}%
\label{sec:birkhoff_s_theorem}

In \emph{Schwarzschild coordinates} $(t, r, \theta, \phi)$ , the Schwarzschild metric is the one-parameter family
\begin{equation}
  ds^2 = - \left( 1 - \frac{2M}{r} \right) dt^2 + \left( 1- \frac{2M}{r} \right)^{-1} dt^2 + r^2 d\Omega^2,
\end{equation}
where the parameter $M > 0$  is interpreted as a mass.
This is a solution to the vacuum Einstein equations for $0 < r < r_S = 2M$, the  \emph{Schwarzschild radius}.
This is spherically symmetric, but it turns out that staticity is not required.

\begin{theorem}[Birkhoff]
  Any spherically symmetric solution of the vacuum Einstein equations is isometric to the Schwarzschild solution.
\end{theorem}
\begin{proof}
  See Hawking and Ellis.
\end{proof}
The theorem assumes only spherical symmetry, but the Schwarzschild solution has an additional isometry: $\partial / \partial t$ is a hypersurface-orthogonal Killing vector field, which is timelike for $r > 2M$, so the corresponding Schwarzschild solution is static.

Birkhoff's theorem implies that the spacetime outside any spherical body is the time-independent (exterior) Schwarzschild spacetime, even if the body itself is time-dependent.
In particular, the Schwarzschild solution is a good description of the spacetime outside a spherical star during its gravitational collapse collapse.


