% lecture notes by Umut Özer
% course: bh
\lhead{Lecture 17: February 24}

The Hamiltonian is given by the Legendre transform
\begin{equation}
  H = \int \dd[3]{x} (\pi^{ij} \dot{h}_{ij} - \mathscr{L}) = \int \dd[3]{x} \sqrt{h} (N \mathcal{H} + N^{i} \mathcal{H}) \label{eq:17-star}.
\end{equation}
In doing this calculation, we need to integrate by parts and neglect surface terms. Upon doing this, we get
\begin{align}
  \mathcal{H} &= - {}^(3) R + h^{-1} \pi^{ij} \pi_{ij} - \frac{1}{2} h^{-1} \pi^2, \qquad \pi = h^{ij} \pi^{ij}. \\
  \mathcal{H}_i &= - 2 h_{ik} D_{j} (h^{-1 / 2} \pi^{jk}).
\end{align}
The $N, N^{i}$ have the interpretation as Lagrange multipliers: varying the action $\frac{\delta S}{\delta N} = \frac{\delta S}{\delta N^{i}} = 0$ gives constraints $\mathcal{H} = \mathcal{H}_i = 0$.
We have rewritten the Hamiltonian and momentum constraints in terms of $\pi^{ij}$ and $h_{ij}$.

Hamilton's principle says that the time derivative of $h_{ij}$ and $\pi^{ij}$ are given by
\begin{equation}
  \dot{h}_{ij} = \frac{\delta H}{\delta \pi^{ij}} \qquad \dot{\pi}^{ij} =- \frac{\delta H}{\delta h_{ij}} \label{eq:17-dag}.
\end{equation}
The equation for $\dot{h}_{ij}$ recovers the definition of $\pi^{ij}$.

Given the Hamiltonian \eqref{eq:17-star}, we would like to define the energy to be the value of the Hamiltonian.
But there is a problem; the only terms involved in the Hamiltonian are constraint equations, which vanish if the constraints are satisfied.
So the energy of the spacetime is always zero with this definition.

The resolution of this problem is that we have been rather cavalier about neglecting surface terms.
When calculating variational derivatives, you generate surface terms that are generally non-zero and will contribute to $H$.
We can trust this conclusion if there are no surface terms. When the surface of constant $t$ is compact (i.e.~for a closed universe), then the energy vanishes exactly.

However, we are mostly interested in the case of black holes. Assume that our constant-$t$ surfaces are asymptotically flat with one end.
Therefore, asymptotically,
\begin{equation}
  h_{ij} = \delta_{ij} + \mathcal{O}\left(\frac{1}{r}\right) \qquad \pi^{ij} = \mathcal{O} \left( \frac{1}{r^2} \right).
\end{equation}
We would like to discuss variations $\delta_{ij} = \mathcal{O} \left( \frac{1}{r} \right)$ and $\delta \pi^{ij} = \mathcal{O} \left( \frac{1}{r^2} \right)$.
Assume $N =1 + \mathcal{O} \left( \frac{1}{r} \right)$ and $N^{i} \to 0$ as $r \to \infty$.

Vary the Hamiltonian in the region inside $S^2_r$, the two-sphere of radius $r$.
From the variation $ \frac{\delta H}{\delta \pi^{ij}} $, we obtain a surface term that vanishes as $r \to \infty$.
The variation $\frac{\delta H}{\delta h_{ij}}$ gives two surface terms $S_1 + S_2$. It turns out that $S_2 \to 0$ as $r \to \infty$.
However,
\begin{equation}
  \lim_{r \to \infty} S_1  =- \lim_{r \to \infty}   \int_{S_r^2} \dd[]{A} n_i (\partial_j \delta h_{ij} - \partial_{i} \delta h_{ij}),
\end{equation}
where $dA$ is the area element on $S^2_r$ and $n_i$ the outward unit normal to $S_r^2$.
The nice thing about this formula is that we can pull the variation outside of the integral:
\begin{equation}
  \lim_{r \to \infty} S_1 = - \delta E_{\text{ADM}},
\end{equation}
where
\begin{equation}
  E_{\text{ADM}} = \lim_{r \to \infty} \int_{S_r^2}  dA n_i (\partial_j h_{ij} - \partial_i h_{jj}),
\end{equation}
so the surface term $S_1$ is itself the variation of something.
Let $H' = H + E_{\text{ADM}}$, so that the variations cancel.
In other words, this quantity $H'$ is the correct Hamiltonian of GR for asymptotically flat data.

\section{ADM Energy}%
\label{sec:adm_energy}

Let us revert back to units $G = 1$, rather than $16 \pi G = 1$, so we will get some new factors appearing.
\begin{definition}[ADM energy]
  The \emph{ADM energy} of an asymptotically flat end is
  \begin{equation}
    E_{\text{ADM}} = \lim_{r \to \infty} \frac{1}{16 \pi} \int_{S_r^2} \dd[]{A} n_i (\partial_j h_{ij} - \partial_{i} h_{jj}).
  \end{equation}
\end{definition}
If we have multiple asymptotically flat ends, we have for each end an associated ADM energy.

For a stationary spacetime,
\begin{equation}
  E_{\text{ADM}} = M_{\text{Komar}},
\end{equation}
at least if $t$-constant surfaces are orthogonal to $k^{a}$ as $r \to \infty$.
For the Kerr--Newmann spacetime,
\begin{equation}
  E_{\text{ADM}} = M,
\end{equation}
which is why we called $M$ the mass in the first place.

This gives us a satisfactory notion of energy for any asymptotically flat end.

We can define a notion of energy, but also something analogous to the three-momentum:
\begin{definition}[ADM $3$-momentum]
  The \emph{ADM $3$-momentum} is 
  \begin{equation}
    P_i = \frac{1}{8 \pi} \lim_{r \to \infty} \int_{S_r^2}  \dd{A} (K_{ij} n_{j} - K n_{i}).
  \end{equation}
\end{definition}
This gives a non-zero value if we have objects moving, i.e.~are not working in the object's rest frame.

The energy of the Newtonian gravitational field is negative, so you might wonder whether the $E_{\text{ADM}}$ might be negative. Yes it can!
In fact, taking $M <0$ Schwarzschild, we get $E_{\text{ADM}}< 0$ .
However, we have already discussed how the spacetime is pathological; it has a naked singularity at $r  = 0$ and is not geodesically complete.
Another way you can get negative $E_{\text{ADM}}$ is by considering negative-energy matter.

The interesting case is if we have the dominant energy condition and the spacetime is non-singular. Can the $E_{\text{ADM}}$ be negative in that case?
\begin{theorem}[Positive Energy Theorem (Shoen--Yau '79, Witten '81)]
  Let $(\Sigma, h_{ab}, K_{ab})$ be geodesically complete, asymptotically flat initial data.
  Assume dominant energy condition. 
  Then
  \begin{equation}
    E_{\text{ADM}} \geq \sqrt{P_{i} P_{i}},
  \end{equation}
  with equality iff $(\Sigma, h_{ab}, K_{ab})$ is surface in Minkowski.
\end{theorem}

If we had a black hole, we might not want to assume what happens inside a black hole.
There is another version of the theorem which has an inner boundary on $\Sigma$, corresponding to the black hole horizon.

\begin{definition}[ADM mass]
  Finally, we can regard $(E_{\text{ADM}}, P_i)$ as a $4$-vector at spatial infinity $i^0$.  We can then look at the norm of this.
  The \emph{ADM} mass is
  \begin{equation}
    M_{\text{ADM}} = \sqrt{E_{\text{ADM}}^2 - P_{i} P_{i}} \geq 0.
  \end{equation}
\end{definition}

\chapter{Black Hole Mechanics}%
\label{cha:black_hole_mechanics}

\section{Killing Horizons and Surface Gravity}%
\label{sec:killing_horizons}

\begin{definition}[Killing horizon]
  A null hypersurface $\mathcal{N}$ is a \emph{Killing horizon} if there exists a Killing vector field $\xi^{a}$ in a neighbourhood of $\mathcal{N}$ such that $\xi^{a}$ is normal to $\mathcal{N}$.
\end{definition}
\begin{theorem}[Hawking '72]
  In a stationary, analytic, asymptotically flat vacuum black hole spacetime, the event horizon $\mathcal{H}^+$ is a Killing horizon.
\end{theorem}
\begin{remark}
  In fact, this is why Killing horizons are important.
\end{remark}

For the Kerr black hole, the event horizon $\mathcal{H}^+$ is a Killing horizon of $\xi^{a} = k^{a} + \Omega_H m^{a}$, not $k^{a}$.
We can fix the freedom $\xi^{a} \to c \xi^{a}$ by $\xi^{a} = k^{a} + \Omega_H m^{a}$ with $k^2 \to -1$ at $\infty$.

Looking at the normal of the Killing field $\xi$ on the hypersurface $\mathcal{N}$, we have
\begin{equation}
  \xi^{a} \xi_{a} \rvert_{\mathcal{N}} = 0,
\end{equation}
so $\mathcal{N}$ is a surface of constant $\xi^2$.
This means that
\begin{equation}
  d (\xi^{a} \xi_{a}) \rvert_{\mathcal{N}} \perp \mathcal{N}.
\end{equation}
Thus, since $\xi^{a}$ is the normal to $\mathcal{N}$, we must have
\begin{equation}
  \nabla_a (\xi^{b} \xi_{b}) \rvert_{\mathcal{N}} = -2 \kappa \xi_{a}, \label{eq:17-star2}
\end{equation}
for some $\kappa \colon \mathcal{N} \to \mathbb{R}$ called the \emph{surface gravity} of $\mathcal{N}$.

The left-hand side of \eqref{eq:17-star2} is
\begin{equation}
  2 \xi^{b} \nabla_a \xi_{b} \stackrel{\text{KVF}}{=} -2 \xi^{b} \nabla_b \xi_{a}.
\end{equation}
Therefore, we obtain the geodesic equation
\begin{equation}
  \xi^{b} \nabla_{b} \xi^{a} \rvert_{\mathcal{N}} = \kappa \xi^{a}.
\end{equation}
In other words, $\kappa$ measures the failure of $\xi$ to follow an affinely parametrised geodesic.

Let $n^{a}$ be tangent to the affinely parametrised generators of $\mathcal{N}$ .
Then $\xi^{a} \rvert_\mathcal{N} = f n^{a}$ for some $f \colon \mathcal{N} \to \mathbb{R}$.
\begin{equation}
  \therefore \xi^{b} \nabla_b \xi^{a} = f n^{b} \nabla_b (f n^{a}) = f^2 \cancel{n^{b} \nabla_{b} n^{a}} + f n^{a} n^{b} \nabla_b f = f^{-1} \xi^{a} \xi^{b} \partial_{b} f.
\end{equation}
From this we can read off
\begin{equation}
  \kappa = \xi \cdot \partial \log \abs{f},
\end{equation}
where $f$ might be negative, which is why we used the modulus here.

\begin{example}[RN, ingoing EF]
  The metric is
  \begin{equation}
    ds^2 = - \frac{\Delta}{r^2} d v^2 + d 2 v dr + r^2 d\Omega^2,
  \end{equation}
  with $\Delta = (r - r_+) (r - r_-)$ with $r_\pm = M \pm \sqrt{M^2 - e^2}$ and $k = \frac{\partial }{\partial v}$.
  \begin{equation}
    k_a \rvert_{r = r_{\pm}} = (d r)_a.
  \end{equation}
  Now $\Delta = 0$, therefore
  \begin{equation}
    k^{a} \perp r = r_{\pm},
  \end{equation}
  so this gives Killing horizons.
  \begin{equation}
    d (k^{b} k_{\beta}) = d \left( -\frac{\Delta}{r^2} \right) = \left( - \frac{\Delta'}{r^2} + \frac{2 \Delta}{r^3} \right) d r \stackrel{r = r_\pm}{=} -\frac{r_\pm - r_\mp}{r_\pm^2} d r = -\frac{(r_\pm - r_\mp)}{r_\pm^2} k \rvert_{r = r_\pm}
  \end{equation}
  Therefore, we have
  \begin{equation}
    \kappa = \kappa_{\pm} \equiv \frac{r_\pm - r_\mp}{2 r_\pm^2}
  \end{equation}
  which is why we called it $\kappa_{\pm}$ before; this is the surface gravity.
\end{example}

\begin{example}[Schw.]
  This is obtained from $e = 0$, giving
  \begin{equation}
    \kappa = \frac{1}{4 M}.
  \end{equation}
\end{example}
\begin{example}[extreme RN]
  Here $r_+ = r_{-}$, so $\kappa = 0$.
\end{example}

\begin{exercise}[Kruskal]
  Show that the horizons $\mathcal{H}^+ = \{U = 0\}$ and $\mathcal{H}^- = \{V = 0\}$ are Killing horizons of
  \begin{equation}
    k = \frac{1}{4 M} \left( V \frac{\partial }{\partial V} - U \frac{\partial }{\partial U} \right).
  \end{equation}
  Show further that the surface gravity is $\kappa = \pm \frac{1}{4M}$.
\end{exercise}


\subsection*{Bifurcate Killing Horizon}%

Let $N^{\pm}$ be Killing horizons of the \emph{same} $\xi^{a}$.
\begin{figure}[tbhp]
  \centering
  \inkfig[0.5]{bifurcate-killing-horizon}
  \caption{Bifurcate Killing horizon}
  \label{fig:bifurcate-killing-horizon}
\end{figure}
The bifurcation surface $B = \mathcal{N}^+ \cap \mathcal{N}^-$.
On it, $\xi^{a}\rvert_B = 0$.

\begin{equation}
  X^{a} \parallel B \implies X^{a} \parallel \mathcal{N}^+ \text{ and } \mathcal{N}^- \implies X^{a} \text{ spacelike}.
\end{equation}
However, since it cannot be normal to both $\mathcal{N}^{\pm}$ at $B$, we have that $B$ is a spacelike surface.

\begin{example}[Kruskal]
  For the Kruskal spacetime, this is the 2-sphere $B = \{U = V = 0\}$.
\end{example}
