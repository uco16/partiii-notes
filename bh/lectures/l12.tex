% lecture notes by Umut Özer
% course: bh
\lhead{Lecture 12: February 12}

\section{Asymptotic Flatness}%
\label{sec:asymptotic_flatness}

\begin{definition}[manifold with boundary]
  A \emph{manifold with boundary} has charts $\phi \colon M \to \mathbb{R}^n / 2$.
  The boundary $\partial M$ is the set of points with $x' = 0$ in some chart.
  \begin{equation}
    \partial M = \{(x^1, \dots, x^n) \suchthat x^1 \leq 0\}.
  \end{equation}
\end{definition}

\begin{definition}[asymptitically flat at null infinity]
  A time orientable manifold $(M, g)$  is \emph{asymptotically flat at null infinity} if there is $(\overline{M}{}, \overline{g}{})$  such that
  \begin{enumerate}[1.]
    \item There is $\Omega \colon M \to \mathbb{R}$ with $\Omega > 0$ such that $(\overline{M}{}, \overline{g}{})$ is an extension of $(M, \Omega^2 g)$.
      (We regard $M < \overline{M}{}$ and $\overline{g}{} = \Omega^2 g$ on $M$.)
    \item Can extend $M$ within $\overline{M}{}$ to obtain a manifold with boundary $M \cup \partial M$.
    \item $\Omega$ extends to a function on $\overline{M}{}$ such that $\Omega\rvert_{\partial M} = 0$ and $d \Omega \rvert_{\partial M} \neq 0$.
    \item $\partial M$ is the disjoint union\footnote{This just means that their intersection is empty.} of $\mathscr{I}^+, \mathscr{I}^-$, each of which are diffeomorphic to $\mathbb{R} \times S^2$.
    \item No past (future) directed causal curve starting in $M$ intersects $\mathscr{I}^+$ ($\mathscr{I}^-$).
    \item $\mathscr{I}^\pm$ are ``complete''.
  \end{enumerate}
\end{definition}

\begin{remark}
  Points $1-3$ say that there is a conformal compactification of the manifold, ensuring that the spacetime approaches Minkowski at the appropriate rate.
  The other points ensure that the spacetime has the same structure as $\mathbb{M}$, with $\mathscr{I}^+$ lying to the future and $\mathscr{I}^-$ lying to the past of $M$.
\end{remark}
\begin{example}[Schw.]
  For $\mathscr{I}^+$, use outgoing Eddington--Finkelstein coordinates $(u, r, \theta, \phi)$.
  $\mathscr{I}^+$ is $r \to \infty$, $u$ finite.
  \begin{equation}
    r = \frac{1}{x} \implies g = - (1 - 2 M x) du^2 + \frac{2 du dx}{x^2} + \frac{1}{x^2} d\omega^2.
  \end{equation}
  Let $\Omega = x$ to multiply by $x^2$:
  \begin{equation}
    \label{eq:12-star}
    \overline{g}{} = - x^2 (1 - 2 M x) du^2 + 2 du dx + d \omega^2.
  \end{equation}
  This extends across $\{x = 0\} \coloneqq \mathscr{I}^+$ parametrised by $\{u, \theta, \phi\}$.
  This means that $\mathscr{I}^+$ is indeed diffeomorphic to $\mathbb{R} \times S^2$.
  Similarly for $\mathscr{I}^-$ we use the ingoing EF coordinates.
  Importantly, the area radius function $r$ is the same. Therefore, exactly the same choice of $\Omega$ gives us an extension to $\mathscr{I}^-$.
\end{example}

\begin{exercise}[Sheet 2]
  \begin{equation}
    R_{ab} = \overline{R}{}_{ab} + 2 \Omega^{-1} \overline{\nabla}{}_a \overline{\nabla}{}_b \Omega + \overline{g}{}_{ab} \overline{g}{}^{cd} \left( \Omega^{-1} \overline{\nabla}{}_c \overline{\nabla}{}_c \Omega - 3 \Omega^{-2} \partial_c \Omega \partial_d \Omega \right).
  \end{equation}
\end{exercise}
Assume $R_{ab} = 0$ . Then
\begin{equation}
  0 = \Omega\overline{R}{}_{ab} + 2 \overline{\nabla}{}_a \overline{\nabla}{}_b \Omega + \overline{g}{}_{ab} \overline{g}{}^{cd} \left( \overline{\nabla}{}_c \overline{\nabla}{}_c \Omega - 3 \Omega^{-1} \partial_c \Omega \partial_d \Omega \right).
\end{equation}
Since the first three summands are smooth at $\Omega = 0$, i.e.~at$\mathscr{I}^{\pm}$ , the fourth term
\begin{equation}
  \Omega^{-1} \overline{g}{}^{cd} \partial_c \Omega \partial_d \Omega
\end{equation}
is also smooth at $\Omega = 0$.
Therefore, at $\mathscr{I}^{\pm}$, we have
 \begin{equation}
  \overline{g}{}^{cd} \partial_c \Omega \partial_d \Omega = 0
\end{equation}

But $d\Omega$  is normal to $\mathscr{I}^{\pm}$ . 
Therefore, $g^{\pm}$  are \emph{null hypersurfaces} in the unphysical spacetime $(\overline{M}{}, \overline{g}{})$ .

The point $6.$  ``complete'' means that the generators of $\mathscr{I}^{\pm}$  are complete.
\begin{example}[]
  Consider the null generators of \eqref{eq:12-star}. Let $n = d\Omega = dx$ be the normal to  $\mathscr{I}^+$ .
  \begin{exercise}
    Show that on $\mathscr{I}^+$, $n$ is tangent to affinely parametrised geodesics
    \begin{equation}
      n^b \nabla_b n_a \rvert_{x = 0} = 0 \qquad n^a \rvert_{x = 0} = \left( \frac{\partial }{\partial u} \right)^a.
    \end{equation}
  \end{exercise}
  This implies that $u$ is an affine parameter along the generators of $\mathscr{I}^+$. This extends to $\pm \infty$, which means it is complete.
\end{example}

\section{Definition of a Black Hole}%
\label{sec:definition_of_a_black_hole}

(We are going to use some definitions that we did not introduce in these lectures. They can be found in the printed notes in Section 4.11.)
\begin{definition}[]
  Let $(M, g)$ be time orientable and pick some subset $U \subset M$.
  The \emph{choronological future} of $U$ is the set of points $I^+(U)$ that you can reach along a future-directed timelike curve from $U$ to $p$.
\end{definition}
\begin{definition}[]
  The \emph{causal future} $J^+(U)$ is the set of points reachable along a future-directed causal curve from $U$, but it also includes all of $U$ as well.
\end{definition}
We define similarly the \emph{chronological / causal past} $I^-(U), J^-(U)$.
\begin{figure}[tbhp]
  \centering
  \def\svgwidth{0.4\columnwidth}
  \input{lectures/l12f1.pdf_tex}
  \caption{}
  \label{fig:l12f1}
\end{figure}

We also need to introduce some topological notions
\begin{definition}[open]
  A set $S \subset M$ is \emph{open} if $\forall p \in S$ there exists a neighbourhood\footnote{defined using the chart} $U$ of $p$ such that $V \subset S$ $I^{\pm}(U)$ are open.
\end{definition}
\begin{definition}[closure]
  The \emph{closure} $\overline{S}{}$ of $S$ is the union of $S$ with all its limit points.
\end{definition}
\begin{example}[Mink]
  In Minkowski space, $\overline{I^{\pm}(p)}{} = J^{\pm}(p)$, so $J^{\pm}(p)$ is closed.
  This is not true in general, as is illustrated in Fig.~\ref{fig:l12f2} for $\mathbb{M}^2$.
\end{example}
\begin{figure}[tbhp]
  \centering
  \def\svgwidth{0.4\columnwidth}
  \input{lectures/l12f2.pdf_tex}
  \caption{$2d$ Minkowski.}
  \label{fig:l12f2}
\end{figure}

\begin{definition}[interior point]
  A point $p \in S$ is an \emph{interior point} of $S$ if $\exists$ a neighbourhood $V$ of $p$ such that $V \subset S$
  \begin{equation}
    \text{int}(S) = \{\text{interior points of } S\} \qquad S \text{ open} \iff S = \text{int}(S)
  \end{equation}
\end{definition}
\begin{definition}[boundary]
  The \emph{boundary} of $S$ is
  \begin{equation}
    \dot{S} = \overline{S}{} \setminus \text{int}(S).
  \end{equation}
\end{definition}
\begin{remark}
  This is a different notion of boundary than we had for manifolds. 
\end{remark}

For $\mathscr{I}^+ \supset \overline{M}{}$  then we can define $J^- (\mathscr{I}^+) \subset \overline{M}{}$ . 
Region of $M$ that can send signal to  $\mathscr{I}^+$  is $M \cap J^- (\mathscr{I}^+)$ .

\begin{definition}[black hole]
  Let $(M, g)$ be asymptotically flat at null $\infty$. The \emph{black hole region} is
  \begin{equation}
    \mathcal{B} = M \setminus \left[ M \cap J^- (\mathscr{I}^+) \right],
  \end{equation}
  where $J^-(\mathscr{I}^+)$ defined with respect to $(\overline{M}{}, \overline{g}{})$.
  Its boundary $\mathcal{H}^+ = \dot{\mathcal{B}} (= M \cap \dot{J}^-(\mathscr{I}^+))$ is the \emph{future event horizon}.
\end{definition}
\begin{definition}[white hole]
  Analogously we define the \emph{white hole region} $\mathcal{W} = M \setminus \left[ M \cap J^+ (\mathscr{I}^-) \right]$ with \emph{past event horizon} $\mathcal{H}^- = \dot{\mathcal{W}} (= M \cap \dot{J}^+ (\mathscr{I}^-))$.
\end{definition}

\begin{example}[Kruskal]
  The black hole region is $\mathcal{B} = II \cup IV$ (including $U = 0$) with future event horizon $\mathcal{H}^+ = \{U = 0\}$.
  The white hole region is $\mathcal{W} = III \cup IV$ (including $V = 0$) with future event horizon $\mathcal{H}^- = \{V = 0\}$.
\end{example}

\begin{claim}
  Can show that $\mathcal{H}^{\pm}$ are null hypersurfaces.
  Furthermore, the generators of $\mathcal{H}^+$ cannot have future end points; light rays skimming the future end horizon cannot leave it.
  However, they can have past endpoints, an example of which is shown in Fig.~\ref{fig:l12f3}.
\end{claim}
\begin{figure}[tbhp]
  \centering
  \def\svgwidth{0.4\columnwidth}
  \input{lectures/l12f3.pdf_tex}
  \caption{The marked point is the past endpoint of generators of $\mathcal{H}^+$.}
  \label{fig:l12f3}
\end{figure}

\begin{definition}[]
  An \emph{asymptotically flat spacetime} $(M, g)$ is \emph{strongly asymptotically predictable} if there exists and open unphysical $\bar{V}{} \subset \overline{M}{}$ such that the closure
  \begin{equation}
    \overline{M \cap J^-(\mathscr{I}^+)}{} \supset \bar{V}{}
  \end{equation}
  and $(\overline{V}{}, \overline{g}{})$ is globally hyperbolic.
\end{definition}
\begin{remark}
  This means that ``the physics is predictable on and outside $\mathcal{H}^+$''.
\end{remark}

\begin{theorem}[]
  Let $(M, g)$ be strongly asymptotically predictable. Let $\Sigma_1$ and $\Sigma_2$ be surfaces for $\overline{V}{}$ such that $\Sigma_2 \subset I^+(\Sigma_1)$. 
  Let $B$ be a connected component of $\mathcal{B} \cap \Sigma_1$. Then $J^+(B) \cap \Sigma_2$ is contained within a connected component of $\mathcal{B} \cap \Sigma_2$.
\end{theorem}
\begin{remark}
  $\mathcal{B} \cap \Sigma_i$ is the ``Black hole region at time $\Sigma_1$''.
  This theorem says that black holes do not bifurcate.
\end{remark}
