% lecture notes by Umut Özer
% course: bh
\lhead{Lecture 23: March 09}

Even from the diagram, we can see that 
\begin{equation}
  (p_i^{(1)}, p_i^{(2)}) = 0,
\end{equation}
since $p_i^{(1)}$ and $p_{i}^{(2)}$ are separated at $\mathscr{I}^-$.
This means that $p_i^{(2)}$ has norm $T_i$ ($T_i^2 + R_i^2 = 1$).
In particular $p_i \vert_{\mathscr{I}^+}$ is as in Fig.~\ref{fig:l22f10} with infinitely many oscillations as $u \to \infty$.
Therefore, as shown in Fig.~\ref{fig:l23f1}, we have infinitely many oscillations between $u = u_i$ and $u = \infty$ (i.e.~$\mathcal{H}^+$).
\begin{figure}[tbhp]
  \centering
  \def\svgwidth{0.4\columnwidth}
  \input{lectures/l23f1.pdf_tex}
  \caption{}
  \label{fig:l23f1}
\end{figure}
Thus the frequency of $p_i$ measured by an infalling observer goes to infinity at $\mathcal{H}^+$.
Take $\gamma$ to be the generator of $\mathcal{H}^+$ extended to the past, intersecting $\mathscr{I}^-$. Without loss of generality we let $v =0$.
Then $p_i^{(2)}$ is localised around some $v_0 < 0$ on $\mathscr{I}^-$. The infinitely many oscillations are in $v_0 < v < 0$.

The phase of $p_i$ varies rapidly near $\gamma$. Thus we can use the \emph{geometric optics} approximation
\begin{equation}
  \phi(x) = A(x) e^{i \lambda S(x)}, \qquad \lambda \gg 1, \qquad \nabla^2 \phi = 0 \implies (\nabla S)^2 = 0 + O(\lambda).
\end{equation}
Therefore, surfaces of constant phase are null hypersurfaces.
Consider a null geodesic congruence containing generators of these hypersurfaces (including $\mathcal{H}^+ \colon S = \infty$).
Let $U^{a}$ be tangent, and take a deviation vector $N^{a}$ such that $U \cdot N = -1$, $U \cdot \nabla N^{a} = 0$.
The deviation 
\begin{equation}
  S^{a} = \alpha U^{a} + \hat{S}^{a} + \beta N^{a}.
\end{equation}
The sum $(\alpha U^{a} + \hat{S}^{a}) \perp U^{a}$ and $\beta N^{a}$ is parallelly transported.
On $\mathcal{H}^+$, $\alpha U^{a} + \hat{S}^{a}$ is tangent to $\mathcal{H}^+$, $\beta N^{a}$ points to a generator of nearby $S = \text{const}$ surface.
Take $\beta = -\epsilon$, because we want to point out of the congruence.
Then $- \epsilon N^{a}$ is a deviation vector from $\gamma$ to generator $\gamma'$ of the $S=\text{const.}$ surface.
\begin{figure}[tbhp]
  \centering
  \inkfig[0.3]{l23f2}
  \caption{l23f2}
  \label{fig:l23f2}
\end{figure}

Because of spherical symmetry, we can always choose $N^{\theta} = N^{\phi} = 0$.
Outside the star (i.e.~in the region of spacetime that is Kruskal), $\partial / \partial V$ is tangent to the affinely parametrised generators of the event horizon $\mathcal{H}^+$.
Therefore, we can choose $U^{a} = (\frac{\partial }{\partial V})^{a}$ there.

Since $N^{a}$ is null and not parallel to $U^{a}$, we must have $N^{a} = c (\frac{\partial }{\partial U})^{a}$ for some $c> 0$. ($U \cdot N = -1$ fixes $c$.)
Thus, outside the star, the deviation vector $- \epsilon N^{a}$ connects $\gamma$ ($U = 0$) to $\gamma'$ with $U = -c \epsilon$.

The definition of the Kruskal coordinate $U$ can be rearranged to find 
\begin{equation}
  u = - \frac{1}{\kappa} \ln(-U),
\end{equation}
where $\kappa$ is the surface gravity.
We deduce that at late time, $\gamma'$ is an outgoing radial null geodesic with $u = -\frac{1}{\kappa} \ln(c \epsilon)$.

We will use this to track the surface of the collapsing star.
Define $F(u)$ to be the phase of $p_i$ on $\mathscr{I}^+$. Since we are looking at surfaces of constant phase, the phase of $p_i$ along $\gamma'$ must be 
\begin{equation}
  \label{eq:23-star}
  S = F(-\frac{1}{\kappa} \ln(c \epsilon)).
\end{equation}

\begin{figure}[tbhp]
  \centering
  \inkfig[0.3]{l23f3}
  \caption{l23f3}
  \label{fig:l23f3}
\end{figure}

At $\mathscr{I}^-$: $\gamma, \gamma'$ ingoing radial null geodesics. Coordinates $(u, v, \theta, \phi)$,
\begin{equation}
  \implies U^{a} \propto \left( \frac{\partial }{\partial u} \right)^{a}, \qquad ds^2 \approx -du d v + \frac{1}{4} (u - v)^2 d \Omega^2.
\end{equation}
Spherical symmetry and $N^{a}$ null, not parallel to $U^{a}$ implies that 
\begin{equation}
  N = D^{-1} \frac{\partial }{\partial v}
\end{equation}
at $\mathscr{I}^-$, and $D$ is a positive constant.
Thus $\gamma'$ intersects $\mathscr{I}^-$ at $v = -D^{-1} \epsilon$.

Then \eqref{eq:23-star} implies that the phase of $p_i^{(2)}$ on $\mathscr{I}^-$ is $S = F(-\frac{1}{\kappa} \ln(-c D_v))$ (small $v < 0$).
\begin{equation}
  p_i^{(2)} \rvert_{\mathscr{I}^-} \approx
  \begin{cases}
    0, & \text{if } v > 0 \\
    A_i(v) \exp[i F(-\frac{1}{\kappa} \ln(-c D_v))], & \text{for small }  v < 0.
  \end{cases}
\end{equation}
Take $F(u) = - \omega u$ with $\omega > 0$.
Although we can continue to work with wavepackets, we do the more pedagogical thing and fudge things just in the same way as we did before, taking $p_i \rvert_{\mathscr{I}^+} \propto e^{i \omega u}$.
Write $p_\omega$ instead of $p_i$
\begin{equation}
  \label{eq:23-star2}
  p_{\omega}^{(2)} \approx
  \begin{cases}
    0 & v > 0 \\
    A_\omega (v) \exp[\frac{i \omega}{\kappa} \ln (-c D v)] & \text{small } v < 0
  \end{cases}
\end{equation}

Take ``in'' modes $f_\sigma$ such that $f_{\sigma} \rvert_{\mathscr{I}^-} = \frac{1}{2 \pi N_\sigma} e^{-i \sigma v}$ ($\sigma > 0$).
We want to write $p_\omega^{(2)}$ in terms of $\left\{f_\sigma, \overline{f_\sigma}{}\right\}$, which we do via the Fourier transform. To do that, we notice that $p_\omega^{(2)}$ is squeezed into small range of $v$ near $v = 0$, which means that the Fourier transform is dominated by large frequency $\abs{\sigma}$ coming from the rapid variation near $v = 0$, coming from the logarithm.
In particular, we can use \eqref{eq:23-star2} and treat the amplitude $A_\omega(v)$ as approximately constant.
\begin{equation}
  \label{eq:23-dag}
  \widetilde{p}_{\omega}(\sigma = A_\omega \int_{-\infty}^0 \dd[]{v} e^{i \sigma v} \exp[\frac{i \omega}{\kappa} \ln(- c D_v)]
\end{equation}
\begin{remark}
  As in Rindler space, this does not converge, since we are working with plane waves instead of with wave packets, in which case it would be convergent. As before, we just pretend it converges here.
\end{remark}

The inverse
\begin{equation}
  p_\omega^{(2)}(v) = \underbrace{\int_{-\infty}^{+\infty} \frac{\dd[]{\sigma}}{2\pi} e ^{-i \sigma v} \widetilde{p}_{\omega}(\sigma) = \int_0^{\infty} \dd[]{\sigma} N_\sigma \widetilde{p}_{\omega}^{(1)} f_\sigma(v)}_{\text{positive freq.}} + \underbrace{\int_0^{\infty} \dd[]{\sigma} \overline{N_\sigma}{} \widetilde{p}_\omega^{(2)} (-\sigma) \overline{f_\sigma(v)}{}}_{\text{negative freq.}}.
\end{equation}
Therefore, 
\begin{equation}
  \therefore \quad  A^{(2)}_{\omega \sigma} = N_\sigma \widetilde{p}_\omega^{(2)} (\sigma), \qquad B_{\omega \sigma} = B_{\omega \sigma}^{(2)} = \overline{N_\sigma}{} \widetilde{p}_\omega^{(2)} (-\sigma) \qquad (\omega, \sigma > 0 )
\end{equation}
We defined
\begin{equation}
  \ln z = \ln \abs{z} + i \arg z, \qquad \arg z \in \left( -\frac{\pi}{2}, \frac{3\pi}{2} \right),
\end{equation}
so the integrand in \eqref{eq:23-dag} is analytic in the lower half-plane.
The branch cut is as in Fig.~\ref{fig:branch-cut}.
For $\sigma > 0$, 
\begin{equation}
  \widetilde{p}_{\omega}^{(2)}(-\sigma) = A_\omega \int_{-\infty}^0 \dd[]{v} e^{-i \sigma v} \exp(\dots),
\end{equation}
the integrand decays for $\abs{v} \to \infty$ in the lower half-plane.
We close the contour as in Fig.~%F

The integral over the semicircle vanishes (for wavepackets via Jordan's lemma) .
Thus
\begin{equation}
  \int_{-\infty}^0 \dd[]{v} = - \int_0^{\infty} \dd[]{v}.
\end{equation}
Then
\begin{align}
  \widetilde{p}_{\omega}^{(2)}(-\sigma) &= -A_\omega \int_{0}^{\infty}\dd[]{v} e ^{-i \sigma v} \exp[\frac{i \omega}{\kappa} \ln(-c D_v)] \\
					&= -A_\omega \int_{0}^{\infty}\dd[]{v} e ^{-i \sigma v} \exp[\frac{i \omega}{\kappa} (\ln(-c D_v) + i \pi)] \\
					&\stackrel{v \to -v}{=} - A_\omega e^{- \omega \pi / \kappa} \int_{-\infty}^{0}\dd[]{v} e^{i \sigma v} \exp[\frac{i \omega}{\kappa} \ln(-C D_v)] \\
					&= - e^{- \pi \omega / \kappa} \widetilde{p}_{\omega}^{(2)} (\sigma).
\end{align}
This relates the Bogoliubov coefficients as
\begin{equation}
  \abs{B_{\omega\sigma}} = e^{- \pi \omega / \kappa} \abs{A^{(2)}_{\omega\sigma}}.
\end{equation}
Switching back to wavepackets,
\begin{equation}
  \abs{B_{ij}} = e^{- \pi \omega_i / \kappa} \abs{A^{(2)}_{ij}}.
\end{equation}
\begin{equation}
  T_i^{2}= (p_i^{(2)}, p_i^{(2)}) \stackrel{p = Af + B \overline{f}{}}{=} \sum_j \left( \abs{A_{ij}}^2 - \abs{B_{ij}}^2 \right) = (e ^{2 \pi \omega_i / \kappa} - 1) \sum_j \abs{B_{ij}}^2 = (e^{2 \pi \omega_i / \kappa} - 1) (B B^{\dagger})_{ii}.
\end{equation}
\begin{equation}
  \therefore \bra{0} b_i^{\dagger} b_i \ket{0} = (B B^{\dagger})_{ii} = \frac{\Gamma_i}{e^{2 \pi \omega_i / \kappa} - 1}, 
\end{equation}
where $\Gamma_i = T_i^2$ is the absorption cross-section of mode $f_i$ - fraction absorbed by BH.
This is the same spectrum of a blackbody at \emph{Hawking temperature}
\begin{equation}
  T_{H} = \frac{\kappa}{2 \pi}.
\end{equation}
\begin{remark}
  These derivations can be generalised.
\end{remark}
The Hawking temperature is tiny, except for tiny black holes,
\begin{equation}
  T_H = 6 \times 10^{-8} \frac{M_\odot}{M} \kappa.
\end{equation}
Any realistic black hole absorbs more energy from the CMB than it loses via this Hawking radiation.
This temperature is inversely proportional to the mass, so the heat capacity $\dv{M}{T} < 0$ is negative.
Any homogeneous thermodynamic system with this property would be unstable, but black holes are not homogeneous systems. In fact, there are even normal stars with this property.

\section{Black Hole Thermodynamics}%
\label{sec:black_hole_thermodynamics}

Black holes really are thermodynamic systems with a temperature.
In fact, the zero\textsuperscript{th} law of black hole dynamics is exactly the same as the zero\textsuperscript{th} law of thermodynamics for a black hole.

Let us rewrite the first law of black hole mechanics as
\begin{equation}
  d E = T_H S_{BH} + \Omega_H d J, \qquad S_{BH} = \frac{A}{4},
\end{equation}
which is the same as the 1\textsuperscript{st} law of thermodynamics if a black hole has entropy $S_{BH}$. Reinstating the various constants, the \emph{Beckenstein--Hawking entropy} of a black hole is
\begin{equation}
  S_{BH} = \frac{c^3 A}{4 G \hbar}.
\end{equation}

The second law of black hole mechanics states that the area $A$ of the event horizon, and therefore $S_{BH}$ can only increase. This looks like the second law of thermodynamics. 
However, this is a classical statement. Quantum mechanically, Hawking radiation causes the mass of the black hole, and therefore the area and the entropy to decrease.
This appears to violate the second law of thermodynamics. 
However, the second law is actually a statement about the total entropy of the system. In fact, we can show that the total entropy $S_{BH} + S_{\text{radiation}}$ can only increase.
This leads to the \emph{generalised 2\textsuperscript{nd} law} of thermodynamics: The total entropy $S = S_{\text{matter}} + S_{BH}$ of matter and Beckenstein--Hawking entropy is non-decreasing in any physical process. 

One might ask how big this black hole entropy is.
For $M = M_{\odot}$, $S_{BH} \sim 10^{77}$. Since we are looking at the black hole of one solar mass, let us compare it to the entropy of our sun, which is $S_\odot \sim 10^{58}$. In other words, black holes carry a huge amount of entropy compared to stars with the same amount of mass.

According to statistical physics, entropy can be interpreted microscopically as $S = \ln N$, where $N$ is the number of microstates corresponding to a given macrostate.
Therefore, a black hole has $N \sim e^{A / 4}$ quantum microstates.

\section{Black Hole Evaporation}%
\label{sec:black_hole_evaporation}

The Hawking calculation was done on a fixed black hole spacetime.
Presumably, the black hole is losing energy, so if we were to solve the Einstein equations including the energy-momentum tensor of the particles moving away from the black hole. In theory we could calculate the backreaction that tells us how much mass the black hole is losing, but this is an extremely difficult calculation that has not yet been done.
However, we can make a crude approximation, pretending that our black hole is just a black body in flat space. The energy loss of a blackbody of area $A$ is then
\begin{equation}
  \dv{E}{t} \approx - \alpha A T^4,
\end{equation}
where $\alpha$ is some constant. Taking $E = M$ and $A \propto M^2$, $T \propto 1  / M$, we have
\begin{equation}
  \dv{M}{t} \propto - \frac{1}{M}.
\end{equation}
Therefore $M \to 0$ in a time
\begin{equation}
  \tau \sim M^3 \sim 10^{71} \left( \frac{M}{M_\odot} \right)^3 \text{sec}
\end{equation}
This is much much much longer than the age of universe.
Nonetheless, this phenomenon that the black hole will eventually radiate away leads to the famous information paradox.

\subsection{Information Paradox}%
\label{sub:information_paradox}

Consider matter in a pure quantum state $\ket{\psi}$ that undergoes gravitational collapse and forms a black hole. After this incredibly long time, it will eventually radiate away completely and all that will be left will be the thermal Hawking radiation that is has emitted.
Thermal radiation is described not by a pure quantum state but by a \emph{mixed state}
\begin{equation}
  \sum_i e^{-\beta E_i} \ket{E_i} \bra{E_i}, \qquad \beta = \frac{1}{T}.
\end{equation}
However, the (unitary) evolution from a pure state to a mixed state is impossible in standard QM!
We lost information in going from the pure state to the mixed state.

Hawking's original response was that we needed a theory of quantum gravity, which does not obey the usual rules of quantum mechanics. However, the majority of physicists nowadays hold on to the unitarity of quantum mechanics and claim that there must be something wrong in this argument.

In fact, in recent years there has been a clean argument that the three following assumptions together lead to a contradiction:
\begin{enumerate}[1.]
  \item Quantum field theory holds in a blackhole spacetime.
  \item Information is not lost.
  \item The event horizon is not a special place.
\end{enumerate}
