% lecture notes by Umut Özer
% course: bh
\lhead{Lecture 11: February 10}


%Missed first 10 mins

\begin{description}
  \item[Weak energy condition:] $T_{ab} V^{a} V^{b} \geq 0$ for all causal $V^a$.
  \item[Null energy condition:] $T_{ab} V^{a} V^{b} \geq 0$ for all null $V^a$.
    \begin{equation}
      \text{DEC} \implies \text{WEC} \implies \text{NEC}
    \end{equation}
  \item[Strong energy condition:] $(T_{ab} - \frac{1}{2} T\indices{^{c}_{c}} g_{ab}) V^{a} V^{b} \geq 0$ for all causal $V^a$.

    If this is imposed and the Einstein equation is satisfied, then $ R_{ab} V^{a} V^{b} \geq 0$.
    This is the statement that ``gravity is attractive''; geodesics converge.
    The SEC does not imply the DEC.
\end{description}

\section{Penrose Singularity Theorem}%
\label{sec:penrose_singularity_theorem}

\begin{lemma}
  Suppose the spacetime $(M, g)$  satisfies the Einstein equation and the null energy condition. 
  The generators of a null hypersurface obey
  \begin{equation}
    \label{eq:11-star}
    \dv{\theta}{\lambda} \leq -\frac{1}{2} \theta^2.
  \end{equation}
\end{lemma}
\begin{proof}
  This is a consequence of the Raychandhuri equation
  \begin{equation}
    \dv{\theta}{\lambda} = -\frac{1}{2} \theta^2 - \hat{\sigma}^{ab} \hat{\sigma}_{ab} + \hat{\omega}^{ab} \hat{\omega}_{ab} - R_{ab} U^{a} U^{b}.
  \end{equation}
  Since the expansion of $\hat{\sigma}$ is in terms of a spacelike basis $T_\perp$, we have $\hat{\sigma}^{ab} \hat{\sigma}_{ab} \geq 0$.
  And $\hat{\omega}_{ab} = 0$.

  The null energy condition implies
  \begin{equation}
    0 \leq 8 \pi T_{ab} U^a U^b = (R_{ab} - \frac{1}{2} R g_{ab}) U^{a} U^{b} = R^{ab} U^{a} U^{b},
  \end{equation}
  since $U$ is null.
\end{proof}

\begin{corollary}
  If $\theta = \theta_0 < 0$ at a point $p$ on the generator $\gamma$ of the null hypersurface, then $\theta \to - \infty$ along $\gamma$ within affine parameter distance $2 / \abs{\theta_0}$ (provided the generator extends this far).
\end{corollary}
\begin{proof}
  Choose an affine parameter that has $\lambda = 0$  at $p$. 
  Equation \eqref{eq:11-star} gives an inequality, which we can integrate with respect to $\lambda$
  \begin{equation}
    \dv{\theta^{-1}}{\lambda} \geq \frac{1}{2} \implies \theta^{-1} \theta_0^{-1} \geq \frac{1}{2} \lambda \implies \theta \leq \frac{\theta_0}{1 + \lambda \theta_0 / 2}.
  \end{equation}
  For $\theta_0 < 0$, the right-hand side diverges to $- \infty$ as $\lambda \to 2/ \abs{\theta_0}$.
\end{proof}

\begin{theorem}[Penrose 1965]
  Let $(M, g)$ be a globally hyperbolic spacetime with a non-compact Cauchy surface $\Sigma$.
  Assume the Einstein equation and the null energy condition. Assume further that there exist a trapped surface $T \subset M$.
  Let $\theta_0 < 0$ be the maximum value of the expansion of both families of null geodesics orthogonal to $T$.

  Then at least one of these geodesics is future-inextendible with affine length $\leq 2 / \abs{\theta_0}$.
\end{theorem}

Let us discuss the consequences of this.
Trapped surfaces are very common. There is a numerical argument, but also a mathematical argument called `Cauchy stability', illustrated in Diagram ??.

\begin{remark}
  This theorem basically tells us that whenever we have a trapped surface we expect (assuming SCC) a singularity to show up. It does not tell us anything about the nature of this singularity, or anything about black holes.
  There is also the Weak Cosmic Censorship conjecture that states that any singularity forms inside a black hole.
\end{remark}

\chapter{Asymptotic Flatness}%
\label{cha:asymptotic_flatness}

We have discussed asymptotic flatness of initial data; we now want to discuss asymptotic flatness of spacetime.

\section{Conformal Compactification}%
\label{sec:conformal_compactification}

On $(M, g)$  let $\overline{g}{} = \Omega^2 g$ , where $\Omega \colon M \to \mathbb{R}$  and $\Omega > 0$.
We want to choose  $\Omega$  to understand the structure of $g$ near infinity.
We want $\Omega \to 0$ at infinity.

Choose  $\Omega$  such that $(M, \overline{g}{})$  is extendible to $(\overline{M}{}, \overline{g}{})$ .
\begin{figure}[tbhp]
  \centering
  \def\svgwidth{0.4\columnwidth}
  \input{lectures/l11f1.pdf_tex}
  \caption{}
  \label{fig:l11f1}
\end{figure}
The boundary $\partial M$  of $M$  in $\overline{M}{}$  is such that $\Omega \rvert_{\partial M} = 0$ .
\begin{example}[]
  Consider $\mathbb{M}^4$ with $g = - dt^2 + dr^2 + r^2 d \omega^2$, where $d\omega^2 = d \theta^2 + \sin ^2 \theta d \phi^2$.
  Then $(\overline{M}{}, \overline{g})$ is the Einstein static universe $\mathbb{R} \times S^{3}$ with 
  \begin{equation}
    \overline{g}{} = - dT^2 + d \chi^2 + \sin^2 \chi d\omega^2.
  \end{equation}
  \begin{figure}[tbhp]
    \centering
    \def\svgwidth{0.4\columnwidth}
    \input{lectures/l11f2.pdf_tex}
    \caption{}
    \label{fig:l11f2}
  \end{figure}
  Suppress $S^2$ to obtain the Penrose diagram.
   \begin{figure}[tbhp]
     \centering
     \begin{minipage}[t]{0.5\columnwidth}
       \centering
       \def\svgwidth{0.4\columnwidth}
       \input{lectures/l11f3.pdf_tex}
       \caption{Penrose diagram of Minkowski space $\mathbb{M}^4$ (GEODESICS MISSING).}
       \label{fig:l11f3}
     \end{minipage}%
     \begin{minipage}[t]{0.5\columnwidth}
       \centering
       \def\svgwidth{0.8\columnwidth}
       \input{lectures/l11f4.pdf_tex}
       \caption{Penrose diagram of $\mathbb{M}^2$.}
       \label{fig:l11f4}
     \end{minipage}
  \end{figure}
  Each point is a two-sphere $S^2$, boundary is axis of symmetry ($r =0$) or at $\infty$ with respect to $g$ (or singular).
  Radial null geodesics are lines at $45^\circ$ .
\end{example}

Penrose diagram for the Kruskal diagram: 
Let $P = P(U)$, $Q  = Q(V)$ such that $P, Q \in (-\frac{\pi}{2}, \frac{\pi}{2})$, say.
Find $\Omega$ such that $(M, \overline{g}{})$ is extendible to $(\overline{M}{}, \overline{g}{})$. The boundary $\partial M$ as $4$ components $P$ or $Q$ is $\pm \frac{\pi}{2}$ ($U$ or $V$ is $\pm \infty$).
Thus we have $\mathscr{I}^+, \mathscr{I}^-$ in $I$ and $\mathscr{I}^+, \mathscr{I}^-$ in IV.

The Penrose diagram is depicted in Fig.~\ref{fig:l11f5}.
\begin{figure}[tbhp]
  \centering
  \def\svgwidth{0.9\columnwidth}
  \input{lectures/l11f5.pdf_tex}
  \caption{The surfaces of constant $r$ are depicted in red, while radial null geodesics are in yellow.}
  \label{fig:l11f5}
\end{figure}

The extended metric $\overline{g}{}$ is singular at $\mathscr{I}^{\pm}, \mathscr{I}^{\pm}{}'$ and not smooth at $\mathscr{I}^0, \mathscr{I}^0{}'$.

Spherical symmetric collapse is depicted in Fig ??.
