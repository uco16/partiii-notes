% A shared preamble for the Part III Live-TeXing 2019

% Font
\usepackage[upint]{stix2}

% Graphics
\usepackage[dvipsnames]{xcolor}
\definecolor{Aqua}{rgb}{0.0, 1.0, 1.0}
\definecolor{CambridgeBlue}{RGB}{163, 193, 173}
\usepackage{graphicx}
\graphicspath{{lectures/}}
\usepackage{wrapfig}
\usepackage{keyval}
\usepackage[theorems]{tcolorbox}
\newtcolorbox{gatherbox}{colback=white, colframe=black, boxrule=0.5pt, sharp corners, ams gather}
\newtcolorbox{equationbox}{colback=white, colframe=black, boxrule=0.5pt, sharp corners, ams equation}
\newtcolorbox{alignbox}{colback=white, colframe=black, boxrule=0.5pt, sharp corners, ams align}
\newtcolorbox{simplebox}{colback=white, colframe=black, boxrule=0.5pt, sharp corners}
\usepackage{nccmath} % simply load to get the spacing in the simplebox around a multline environment right
\usepackage{subcaption}
\usepackage{soul}
\usepackage{empheq}

\usepackage{tikz}
%\usetikzlibrary{calc}
\usepackage[compat=1.1.0]{tikz-feynman}
\usepackage{tikz-cd}

% GILLES CASTELs inkscape-figure
% https://github.com/gillescastel/inkscape-figures
\usepackage{pdfpages}
%\usepackage{transparent}
\newcommand{\inkfig}[2][1]{%
  \def\svgwidth{#1\columnwidth}
  \input{lectures/#2.pdf_tex}
}
%\pdfsuppresswarningpagegroup=1

% Mathematical Symbols
\usepackage{amsmath}
\usepackage{mathtools}
\usepackage{physics} 
%\usepackage{amssymb} % made redundant by STIX2
\usepackage{mathrsfs}
\usepackage{cancel}
\usepackage{siunitx}
\usepackage{tensor}
\usepackage{bbold}
\usepackage{simpler-wick}
\usepackage[shortlabels]{enumitem}
\setlist[description]{leftmargin=\parindent,labelindent=\parindent}

% Links
\usepackage{hyperref}
\hypersetup{
  %hidelinks,
  pdfencoding=auto,
  pdfauthor={Umut \"Ozer},
  pdfsubject={\coursetitle{}},
  pdfcreator={Umut \"Ozer},
  pdfproducer={Umut \"Ozer},
  pdfkeywords={\coursetitle{},\term{},\lecturer{},Umut \"Ozer},
  colorlinks=true,
  linkcolor=MidnightBlue!50!BlueViolet,
  linktoc=all,
  urlcolor=blue,
}
\usepackage{bookmark} % faster updated bookmarks

% Page and Line Formatting
\setlength\parindent{1em}
\setlength{\parskip}{0.8em}
\usepackage[margin=35mm]{geometry}
\renewcommand{\baselinestretch}{1.3}

% Headers and Footers
\usepackage{fancyhdr}
\pagestyle{fancy}
\fancyhf{}
\chead{\coursetitle{}}
\rhead{Page \thepage}
%\lfoot{\leftmark}
%\rfoot{\rightmark}
\cfoot{\nouppercase\leftmark}
\renewcommand{\footrulewidth}{0.5pt}

% Chapter title formatting
%\renewcommand{\thechapter}{\Roman{chapter}}
\usepackage{titlesec}
\titleformat{\chapter}[hang]
{\normalfont\huge\bfseries}{\thechapter}{1em}{}

% Line Numbers (Pre-Publishing)
%\usepackage{lineno}
%\linenumbers

% Numbering by Section
%\usepackage{placeins}
%\let\Oldsection\section
%\renewcommand{\section}{\FloatBarrier\Oldsection}
%\let\Oldsubsection\subsection
%\renewcommand{\subsection}{\FloatBarrier\Oldsubsection}
%\let\Oldsubsubsection\subsubsection
%\renewcommand{\subsubsection}{\FloatBarrier\Oldsubsubsection}
%\numberwithin{equation}{chapter}
%\numberwithin{figure}{chapter}
%\numberwithin{table}{chapter}

%Footnotes
\usepackage{perpage}
\MakePerPage{footnote}

% Theorems and Definition
\usepackage{amsthm}
\newtheoremstyle{StandardStyle}% name of the style to be used
  {\topsep}% measure of space to leave above the theorem. E.g.: 3pt
  {\topsep}% measure of space to leave below the theorem. E.g.: 3pt
  {\normalfont}% name of font to use in the body of the theorem
  {}% measure of space to indent
  {\bfseries}% name of head font
  {:}% punctuation between head and body
  { }% space after theorem head; " " = normal interword space
  {}% Manually specify head
\theoremstyle{StandardStyle}

\newtheorem{theorem}{Theorem}
\newtheorem{lemma}[theorem]{Lemma}
\newtheorem{conjecture}{Conjecture}
\newtheorem{claim}{Claim}
\newtheorem*{proposition}{Proposition}
\newtheorem*{corollary}{Corollary}

\newtheorem{definition}{Definition}
\newtheorem*{notation}{Notation}

\newtheorem{example}{Example}[section]
\newtheorem{exercise}{Exercise}[chapter]

\usepackage{framed} % for \leftbar next to remark
\newtheorem*{remark}{Remark}
\newtheorem*{note}{Note}

% --- QUOTE ENVIRONMENT ---
% source: https://tex.stackexchange.com/questions/16964/block-quote-with-big-quotation-marks
\newcommand*\openquote{\makebox(25,-22){\scalebox{5}{``}}}
\newcommand*\closequote{\makebox(25,-22){\scalebox{5}{''}}}
\colorlet{shadecolor}{CambridgeBlue}

\makeatletter
\newif\if@right
\def\shadequote{\@righttrue\shadequote@i}
\def\shadequote@i{\begin{snugshade}\begin{quote}\openquote}
\def\endshadequote{%
  \if@right\hfill\fi\closequote\end{quote}\end{snugshade}}
\@namedef{shadequote*}{\@rightfalse\shadequote@i}
\@namedef{endshadequote*}{\endshadequote}
\makeatother

% differential (upright d):
\newcommand{\bdd}[2][]{\mathop{\mathrm{d\hspace*{-0.2em}\bar{}\hspace*{0.2em}}^{#1}\hspace*{-0.05em}{#2}}}

% bar delta
\newcommand{\bdelta}{\delta\hspace*{-0.2em}\bar{}\hspace*{0.2em}}

% definition of path integral measure
\newcommand*\pdd[1]{\mathop{\mathcal{D} #1}}

% normal ordering : A B C :
\newcommand{\normalorder}[1]{\mathop{:}\nolimits\!#1\!\mathop{:}\nolimits}

% Set delimiter | that resizes to \left\{ \right}:
\newcommand{\suchthat}{\ifnum\currentgrouptype=16 \;\middle|\;\else\mid\fi}

% Pfaffian
\newcommand{\Pf}[1]{\mathop{\text{Pfaff}\, #1}}

% Floor and Ceiling funciton
% NB: use \floor*{. . .} for resized brackets
\DeclarePairedDelimiter\ceil{\lceil}{\rceil}
\DeclarePairedDelimiter\floor{\lfloor}{\rfloor}

\usepackage{ifthen}
\makeatletter
\newcommand{\yearofterm}[1]{%
  \ifthenelse{\equal{#1}{Michaelmas}}
    {2019}
    {2020}%
}
\makeatother

\author{Report typos to: \href{mailto:uco21@cam.ac.uk}{\ul{uco21@cam.ac.uk}}\\More notes at: \href{https://uco21.user.srcf.net/notes}{\ul{uco21.user.srcf.net}}}
\title{{\Huge \coursetitle{}}\\
  Part III \term{} \yearofterm{\term} \\
Lectures by \lecturer{}}
