% A shared preamble for the Part III Live-TeXing 2019

% Mathematical Symbols
\usepackage{amsmath}
\usepackage{mathtools}
\usepackage{physics} 
\usepackage{amssymb} % disable if unicode-math is used
\usepackage{mathrsfs}
\usepackage{cancel}
\usepackage{siunitx}
\usepackage{tensor}
\usepackage{bbold}
\usepackage{simpler-wick}

% Graphics
\usepackage{graphicx}
\graphicspath{{lectures/}}
\usepackage{wrapfig}
\usepackage{xcolor}
\usepackage{keyval}
\usepackage{tcolorbox}
\usepackage{subcaption}

\usepackage{tikz}
\usepackage[compat=1.1.0]{tikz-feynman}

% Links
\usepackage[hidelinks, pdfencoding=auto]{hyperref}
\usepackage{bookmark} % faster updated bookmarks

% Page and Line Formatting
\setlength\parindent{1em}
\setlength{\parskip}{0.8em}
\usepackage[margin=35mm]{geometry}
\renewcommand{\baselinestretch}{1.3}

% Headers and Footers
\usepackage{fancyhdr}
\pagestyle{fancy}
\fancyhf{}
\chead{\coursetitle{}}
\rhead{Page \thepage}
%\lfoot{\leftmark}
%\rfoot{\rightmark}
\cfoot{\nouppercase\leftmark}
\renewcommand{\footrulewidth}{0.5pt}

% Chapter title formatting
%\renewcommand{\thechapter}{\Roman{chapter}}
\usepackage{titlesec}
\titleformat{\chapter}[hang]
{\normalfont\huge\bfseries}{\thechapter}{1em}{}

% Line Numbers (Pre-Publishing)
%\usepackage{lineno}
%\linenumbers

% Numbering by Section
%\usepackage{placeins}
%\let\Oldsection\section
%\renewcommand{\section}{\FloatBarrier\Oldsection}
%\let\Oldsubsection\subsection
%\renewcommand{\subsection}{\FloatBarrier\Oldsubsection}
%\let\Oldsubsubsection\subsubsection
%\renewcommand{\subsubsection}{\FloatBarrier\Oldsubsubsection}
%\numberwithin{equation}{chapter}
%\numberwithin{figure}{chapter}
%\numberwithin{table}{chapter}

% Theorems and Definition
\usepackage{amsthm}
\newtheoremstyle{StandardStyle}% name of the style to be used
  {\topsep}% measure of space to leave above the theorem. E.g.: 3pt
  {\topsep}% measure of space to leave below the theorem. E.g.: 3pt
  {\normalfont}% name of font to use in the body of the theorem
  {}% measure of space to indent
  {\bfseries}% name of head font
  {:}% punctuation between head and body
  { }% space after theorem head; " " = normal interword space
  {}% Manually specify head
\theoremstyle{StandardStyle}

\newtheorem{theorem}{Theorem}
\newtheorem{lemma}[theorem]{Lemma}
\newtheorem{claim}{Claim}
\newtheorem*{proposition}{Proposition}
\newtheorem*{corollary}{Corollary}

\newtheorem{definition}{Definition}
\newtheorem*{notation}{Notation}

\newtheorem*{example}{Example}
\newtheorem{exercise}{Exercise}[chapter]

\usepackage{framed} % for \leftbar next to remark
\newtheorem*{remark}{Remark}

% differential (upright d):
\newcommand{\bdd}[2][]{\mathop{\mathrm{d\hspace*{-0.2em}\bar{}\hspace*{0.2em}}^{#1}\hspace*{-0.05em}{#2}}}

% bar delta
\newcommand{\bdelta}{\delta\hspace*{-0.2em}\bar{}\hspace*{0.2em}}

% definition of path integral measure
\newcommand*\pdd[1]{\mathop{\mathcal{D} #1}}

% normal ordering : A B C :
\newcommand{\normalorder}[1]{\mathop{:}\nolimits\!#1\!\mathop{:}\nolimits}

\author{Umut C.~\"Ozer\\uco21@cam.ac.uk}
\title{{\Huge \coursetitle{}}\\
Part III \term{} 2019\\
Lectures by \lecturer{}}
