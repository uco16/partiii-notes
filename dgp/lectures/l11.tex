% lecture notes by Umut Özer
% course: dgp
\lhead{Lecture 11: February 26}

\begin{example}[]
  Consider maps between two-spheres: $f \colon S^2 \to S^2$.
  Let us think about this $S^2$ concretely as a round sphere in $\mathbb{R}^3$.
  The coordinates on the image sphere are,
  \begin{equation}
    f^a = f^a (\rho^i) \in \mathbb{R}^3, \qquad \abs{\vb{f}} = 1,
  \end{equation}
  where the coordinates on the domain are $\rho^{i} = (\theta, \phi)$, $i = 1, 2$.
  The volume (i.e.~area) form on a sphere is 
  \begin{equation}
    \sin \theta d\theta \wedge d\phi = \frac{1}{r^3} [x dy \wedge d z + y dz \wedge dy + z dx \wedge dy].
  \end{equation}
  To obtain the degree, we take \eqref{eq:10-deg} and divide both sides by $\text{vol}(S^2)$:
  \begin{align}
    \text{deg}(f) &= \frac{1}{\text{vol}(S^2)} \int \epsilon^{abc} f^{a} df^{b} \wedge df^{c} \\
		  &= \frac{1}{8\pi} \int \epsilon^{ij} \epsilon^{abc} f^{a} \frac{\partial f^{b}}{\partial \rho^{i}} \frac{\partial f^{c}}{\partial \rho^{j}} \dd[2]{\rho}.
  \end{align}
\end{example}

\begin{example}[]
  Let $f \colon M \to SU(2)$, where $\dim(M) = 3$ (compact, no boundary) and $SU(2) \simeq S^3$.
  \begin{exercise}
    \begin{equation}
      \text{deg}(f) = \frac{1}{24 \pi^2} \int_M \Tr[(f^{-1} df)^3]
    \end{equation}
  \end{exercise}
  $f$ is an $SU(2)$-valued matrix. Now $f^{-1} df$ is a Lie algebra valued one-form.
  Taking the (exterior) group makes it still Lie algebra valued. Taking the trace gives a three-form that we can integrate.
\end{example}

\section{Applications of Topological Degree in Physics}%
\label{sec:applications_of_topological_degree_in_physics}

\subsection{Sigma Model Lumps}%
\label{sub:sigma_model_lumps}

We will talk about a scalar Lagrangian field theory of the following type.
Our scalar will be a map 
\begin{equation}
  \phi \colon \mathbb{R}^{2, 1} \to S^{N_1} \subset \mathbb{R}^N.
\end{equation}
We will assume the potential $U(\phi) = 0$. However, the theory is still non-linear. This non-linearity enters on a more fundamental level because the target, on which the fields are constrained to live, is a non-linear space.
We take the Lagrangian to be
\begin{equation}
  \mathscr{L} = \frac{1}{2} \eta^{\mu\nu} \frac{\partial \phi^{a}}{\partial x^{\mu}} \frac{\partial \phi^{b}}{\partial x^{\nu}} \delta_{ab}
\end{equation}
with metric $ds^2 = \eta_{\mu\nu} dx^{\mu} dx^{\nu}$ and the constraint
\begin{equation}
  \sum_{a=1}^{N} \phi^{a} \phi^{a} = 1.
\end{equation}

\subsection*{Lagrange Multiplier}%

To incorporate this constraint, we may modify the Lagrangian by introducing a Lagrange multiplier $\lambda$
\begin{equation}
  \mathscr{L}' = \mathscr{L} - \frac{1}{2} \lambda (1 - \sum_{a} \abs{\phi}^2).
\end{equation}
The resulting equation of motion is
\begin{equation}
  \Box \phi^{a} - \lambda \phi^{a} = 0, \qquad \Box = \eta^{\mu\nu} \frac{\partial }{\partial x^{\mu}} \frac{\partial }{\partial x^{\nu}}.
\end{equation}
As it stands, this equation is not what we want yet; we need to solve for $\lambda$ and substitute it back into this to obtain the constrained equations of motion.
We then obtain the non-linear equations
\begin{equation}
  \Box \phi^{a} - (\phi^{b} \Box \phi^{b}) \phi^{a} =0.
\end{equation}
Moreover, the non-linearity is of a particularly nasty type; the non-linear term arises with the highest derivative, unlike quasi-linear equations like Einstein's equations.

\subsection*{Without Lagrange Multiplier}%

Alternatively, we may solve $\abs{\phi}^2 = 1$ for the final component
\begin{equation}
  \phi^N = \pm \sqrt{1 - \phi^{p} \phi^{p}}, \qquad p = 1, \dots, N-1.
\end{equation}
We obtain
\begin{align}
  \mathscr{L} = \frac{1}{2} g_{pq}(\phi) \eta^{\mu\nu} \partial^{\mu} \phi^{p} \partial^{\nu} \phi^{q}, \label{eq:11-lag}
\end{align}
where,
\begin{equation}
  g_{pq} = \delta_{pq} + \frac{\phi_p \phi_q}{1 - \sum_{i=1}^{N-1} \phi_i \phi_i}
\end{equation}
is the round metric on $S^{N-1}$.
We are not going to work much with the Lagrangian \eqref{eq:11-lag}, but we want to point it out since it provides a broad framework for
\begin{equation}
  (\Sigma, \eta) \to (M, g).
\end{equation}
It is useful in a variety of situations:
\begin{enumerate}[1)]
  \item Bosonic sector of superstring theory. In this framework, $(\Sigma, \eta)$ the string worldsheet (2D Riemann surface) and $(M, g)$ is a 10-dimensional space-time.
    Then \eqref{eq:11-lag} gives the basic scalar Lagrangian.
  \item Point particle moving on a curved space-time. Take $\Sigma = \mathbb{R}$ and $(M, g)$ to be space-time (or any manifold).
    Then \eqref{eq:11-lag} is the geodesic Lagrangian.
  \item If $\dim \Sigma = k$, then \eqref{eq:11-lag} describes the behaviour of ``$(k-1)$ branes''.
\end{enumerate}
These examples are all scalar theories, but we can also add gauge fields to this Lagrangian.

\subsection{Solitons}%
\label{sub:solitons}

GO back to $\Sigma = \mathbb{R}^{2}$ and $M= S^{N-1}$.
Take $N = 3$, so that the target space is a two-sphere.
We will be interested in solitons --- static solutions to the Euler--Lagrange equations with finite energy.
The total energy for static solutions is
\begin{equation}
  E = \int_{\mathbb{R}^2} (\partial_{i} \phi^{a} \cdot \partial_{i} \phi^{a}) dx dy < \infty,
\end{equation}
where $x^{i} = (x, y)$ and $\partial_{i} = \frac{\partial }{\partial x^{i}}$.
The metric is $\eta = dt^2 - dx^2 - dy^2$, but $\phi = \phi(x^{i})$.

We know that an integral of $r^{-p}$ is logarithmically divergent for $p = 1$, but not for $p > 1$.
For the integral to be finite, we choose the following boundary conditions
\begin{equation}
  r \abs{\nabla \phi^{a}} \to 0 \quad \text{as} \quad r \to \infty.
\end{equation}
Therefore, we also have $\nabla \phi^{a} \to 0$, so $\phi \to \phi^{\infty}$ for some constant $\phi^\infty$ at spatial infinity.
Choose $\phi^\infty = (0,0,1)$, which is the north pole of $S^2$.

We may reinterpret this boundary condition as saying that static, finite energy configurations $\phi$ extend to the one-point compactification of $\mathbb{R}^2$, which is a two-sphere $S^2 = \mathbb{R}^2 + \{\infty\}$, with $\phi(\{\infty\}) = (0,0,1)$.
We can view
\begin{equation}
  \phi \colon S^2 \to S^2.
\end{equation}
These maps are classified by the degree $\text{deg}(\psi)$ (topological charge). 
This is only a partial classification, as fields can have dive different energies in the same topological sector.

\begin{claim}
  The energy
  \begin{equation}
    E = \frac{1}{2} \int_{S^2} \partial_{i} \phi^{a} \partial_{i} \phi^{a} \dd[2]{x} \geq 4 \pi \abs{\text{deg}(\phi)},
  \end{equation}
  is bounded from below, with equality when the first order Bogomolny equations
  \begin{equation}
    \boxed{\partial_{i} \phi^{a} = \pm \epsilon_{ij} \epsilon^{abc} \phi^{b} \partial_{j} \phi^{c}}
  \end{equation}
  are satisfied. Here $i, j = 1, 2$, $\epsilon_{ij} = 
  \begin{pmatrix}
   0 & 1 \\
   -1 & 0 \\
  \end{pmatrix}$, and $\abs{ \phi} = 1$.
\end{claim}
\begin{leftbar}
  In this case, non-integrable second-order equations reduce to integrable first-order equations.
\end{leftbar}
\begin{proof}
  Consider the identity
  \begin{equation}
    \label{eq:11-id}
    \int (\partial_{i} \phi^{a} \pm \epsilon_{ij} \epsilon^{abc} \phi^{b} \partial_{j} \phi^{c}) \cdot (\partial_{i} \phi^{a} \pm \epsilon_{ik} \epsilon^{ade} \phi^{d} \partial_{k} \phi^{e}) \dd[2]{x} \geq 0.
  \end{equation}
  This is a square of scalar quantities, which is therefore non-negative.
\end{proof}
