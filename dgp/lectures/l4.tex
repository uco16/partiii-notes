% lecture notes by Umut Özer
% course: dgp
\lhead{Lecture 4: January 29}

\begin{definition}[Lie derivative]
  Let $V, W$ be vector fields. Let $V$ generate a flow $V = \dot{\gamma}$. The \emph{Lie derivative} is
  \begin{align}
    \mathcal{L}_V W \rvert_p &= \lim_{\epsilon \to 0} \frac{W(p) - \gamma(\epsilon)_* W(p)}{\epsilon} \\
  \end{align}
  We can extend this definition over the whole manifold.
\end{definition}
\begin{exercise}
  Show that $L_V W = [V, W]$.
\end{exercise}
\begin{definition}[]
  On functions $f \colon M \to \mathbb{R}$, we define the Lie derivative as $\mathcal{L}_V(f) = V(f)$.
\end{definition}
On differential forms, we can use the Leibniz rule to show that
\begin{equation}
  \mathcal{L}_V \Omega = d(\iota_V \Omega) + \iota_V (d\Omega).
\end{equation}
\begin{definition}[]
  We define the cotangent space $T^*_p M = \text{Span}\left\{dx^1, \dots, dx^n\right\}$ as the space of one-forms.
  The cotangent bundle is then
  \begin{equation}
    \bigcup_{p\in M} T^*_p M = T^* M.
  \end{equation}
\end{definition}
Using the wedge product, which is anti-commutative on one-forms $dx^{i} \wedge dx^{j} = -dx^{j} \wedge dx^{i}$, we can define an $r$ -form 
\begin{equation}
  \Omega = \frac{1}{r!} \Omega_{ij \dots k} dx^{i} \wedge dx^{j} \wedge \dots \wedge dx^k.
\end{equation}
\begin{definition}[contraction]
  We write a contraction as
  \begin{equation}
    \frac{\partial }{\partial x^{i}} (HOOK) dx^{j} = \iota_{\frac{\partial }{\partial x^{i}}} dx^{j} = \delta\indices{_{i}^{j}}.
  \end{equation}
\end{definition}
For a general vector field $V$ and one-form $ \Omega$, we have
\begin{equation}
  \iota_V \Omega = V^{i} \frac{\partial }{\partial x^{i}} (HOOK) \Omega_j dx^{j} = V^{j} \Omega_{j} \delta\indices{^{j}_{i}} = V^{i} \Omega_{i}.
\end{equation}
No metric is needed.

\begin{definition}[]
  A \emph{Lie algebra} $\mathfrak{g}$ of a Lie group $G$ is the tangent space $T_e G$ to $G$ at the identity.
  The Lie bracket on $\mathfrak{g}$ is the commutator of vector fields on $G$.
\end{definition}

\begin{definition}[Left translations]
  For all $\mathfrak{g} \in G$, we define the \emph{left translations} 
  \begin{align}
    L_g \colon G &\to G \\
    g &\mapsto g \cdot h.
  \end{align}
\end{definition}
\begin{definition}[left invariant vector fields]
  Using the left translation maps, we define their push forward $(L_g)_* \colon \mathfrak{g} \equiv T_e G \to T_g G$, which maps $v \in g$ to vector fields $(L_g)_* (V)$ on $G$.
  This defines \emph{left invariant vector fields}.
\end{definition}
\begin{equation}
  [(L_g)_* V, (L_g)_* W] = (L_g)_* [V, W]_\mathfrak{g}.
\end{equation}
\begin{remark}
  It is important to understand the notation!
\end{remark}

These form a basis of $\mathfrak{g}$ , meaning that $\dim(G)$  is the number of global, non-vanishing vector fields on $G$.
\begin{claim}
  Lie groups are parallelisable manifolds.
\end{claim}
\begin{claim}
  The converse is not true.
\end{claim}
\begin{proof}
  $S^1, S^3, S^7$  are the only parallelisable spheres.

  The first two are indeed manifolds:
  $S^1 = U(1)$ , $S^3 = SU(2) = \left\{ 
    \begin{pmatrix}
     a & b \\
     -b^* & a \\
    \end{pmatrix} \suchthat \abs{a}^2 + \abs{b}^2 = 1
  \right\}$,
  but $S^7$ is not a Lie group.
\end{proof}
\begin{leftbar}
  This has introduced the field of $k$-theory.
\end{leftbar}

\begin{claim}
  Let $L_{\alpha}$, $\alpha = 1, \dots, \dim \mathfrak{g}$ be a basis of left invariant vector fields with structure constants
  \begin{equation}
    [L_{\alpha}, L_{\beta}] = \sum_\gamma f\indices{_{\alpha\beta}^{\gamma}}L_{\gamma}.
  \end{equation}
  Let $\sigma^{\alpha}$ be a dual basis of left-invariant one-forms
  \begin{equation}
    L_{\alpha} (hook) \sigma^{\beta} = \delta\indices{_{\alpha}^{\beta}}.
  \end{equation}
  Then
  \begin{equation}
    d\sigma^{\alpha} + \frac{1}{2} f\indices{^{\alpha}_{\beta\gamma}} \sigma^{\beta} \wedge \sigma^{\gamma} = 0 \label{eq:4-star}.
  \end{equation}
\end{claim}
\begin{proof}[Proof (Sheet 1)]
  Use the identity
  \begin{equation}
    d \Omega( V, W) = V(\Omega(W)) - W(\Omega(V)) - \Omega([V, W]).
  \end{equation}
\end{proof}
\begin{leftbar}
  Watch out for signs and factors in the upcoming derivations! Things can easily go wrong.
\end{leftbar}

\begin{definition}[Maurer--Cartan]
  Assume that $G$ is a matrix Lie group.
  The \emph{Maurer--Cartan 1-form} on $G$ is
  \begin{equation}
    \rho \coloneqq g^{-1} d g.
  \end{equation}
\end{definition}
\begin{claim}
  This one-form
  \begin{itemize}
    \item is left invariant.
    \item takes values in the Lie algebra.
  \end{itemize}
\end{claim}
\begin{proof}
  \begin{itemize}
    \item $(g_0 g)^{-1} d(g_0 g) = g^{-1} dg$, $g_0 = ??$
    \item Take $C$  a smooth curve $g(s) \subset G$ .
    \begin{equation}
      g^{-1}(s) g(s + \epsilon) = \underbrace{\epsilon}_{\mathclap{\mathbb{1}}} + \epsilon \underbrace{g^{-1} \dv{g}{\epsilon}}_{\mathclap{\in T_e G \simeq \mathfrak{g}}} \rvert_{\epsilon = 0} + O(\epsilon^2).
    \end{equation}
    So $g^{-1} d g = \sum_\alpha \sigma^{\alpha} \otimes T_\alpha$, where $T_\alpha$ are matrices with $[T_\alpha, T_\beta] = \sum_\gamma f\indices{_{\alpha\beta}^\gamma} T_{\gamma}$.
  \end{itemize}
\end{proof}

\begin{claim}
  It obeys the Maurer--Cartan equation:
  \begin{equation}
    d\rho + \rho \wedge \rho = 0.
  \end{equation}
\end{claim}
\begin{proof}
  Consider first the exterior derivative term
  \begin{equation}
    d \rho = \sum_\alpha d\sigma^{\alpha} \cdot T_{\alpha} = -\frac{1}{2} f\indices{^{\alpha}_{\beta\gamma}} \sigma^{\beta} \wedge \sigma^{\gamma} \cdot T_{\alpha}.
  \end{equation}
  The wedge product term is
  \begin{equation}
    \rho \wedge \rho = \sigma^{\alpha} T_{\alpha} \wedge \sigma^{\beta} T_{\beta} = \frac{1}{2} \sigma^{\alpha} \wedge \sigma^{\beta} [T_{\alpha}, T_{\beta}]
    = \frac{1}{2} \sigma^{\alpha} \wedge \sigma^{\beta} f\indices{_{\alpha\beta}^{\gamma}} T_{\gamma}.
  \end{equation}
\end{proof}
\begin{example}[Heisenberg group]
  The \emph{Heisenberg group} (sometimes just called \emph{Nil}) is the group of upper triangular matrices
  \begin{equation}
    g = 
    \begin{pmatrix}
     1 & x & z \\
      & 1 & y \\
      &  & 1 \\
    \end{pmatrix}
    = \mathbb{1} + x T_1 + yT_2 + z T_3,
  \end{equation}
  with
  \begin{equation}
    T_1 = 
    \begin{pmatrix}
      & 1 &  \\
      &  &  \\
      &  &  \\
    \end{pmatrix}
    \qquad
    T_2 = 
    \begin{pmatrix}
      &  &  \\
      &  & 1 \\
      &  &  \\
    \end{pmatrix}
    \qquad
    T_3 = 
    \begin{pmatrix}
      &  & 3 \\
      &  &  \\
      &  &  \\
    \end{pmatrix}
  \end{equation}
  The commutation relations are
  \begin{equation}
    [T_1, T_2] = T_3 \quad [T_1, T_3] = 0 = [T_2, T_3].
  \end{equation}
  We can interpret $T_1$ as position, $T_2$ as momentum and $T_3$ as the identity $i \hbar$.
\end{example}
