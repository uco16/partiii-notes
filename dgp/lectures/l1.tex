% lecture notes by Umut Özer
% course: dgp
\lhead{Lecture 1: January 20}

\chapter{Differential Geometry}%
\label{cha:differential_geometry}

\section{Introduction: Kepler / Newton Orbits}%
\label{sec:kepler_newton_orbits}

The importance of the role of geometry in physics in undeniable.
For example, consider Kepler orbits $\vb{r}(t) \in \mathbb{R}^3$ obeying
\begin{equation}
  \ddot{\vb{r}} = -\frac{G M v}{r^3} \vb{r}.
\end{equation}
The solutions are conic sections, such as the one illustrated in Fig.~\ref{fig:l1f1}.

\begin{wrapfigure}{R}{0.35\textwidth}
  \centering
  \inkfig[0.3]{l1f1}
  \caption{Conic section}
  \label{fig:l1f1}
\end{wrapfigure}
A general conic section can be written as the set of solutions of the equation
\begin{equation}
  \label{eq:1-c}
  ax^2 + by^2 + cxy + dx + ey + f = 0,
\end{equation}
where $a, b, c, d, e, f \in \mathbb{R}$.
Apollonius of Perga asked `what is the unique conic thorugh five points, no three of which are co-linear?'
Since multiplying all coefficients by a real constant $y \neq 0$ gives the same conic, the space of conics is the projective 5-space $\mathbb{RP}^5 = (\mathbb{R}^6 \setminus \left\{ 0 \right\}) / \sim$, where $x \sim y$ iff $\exists \gamma \in \mathbb{R} \setminus \{0\}$ s.t.~$x = \gamma y$. In particular, we have homogeneous coordinates
\begin{equation}
  [a, b, c, d, e, f] \sim [\gamma a, \gamma b, \gamma c, \gamma d, \gamma c, \gamma f], \gamma \in \mathbb{R}^*,
\end{equation}
where $\mathbb{R}^* = \mathbb{R} \setminus \{0\}$.
This is an application of geometry, rather than an application of differential geometry.
\begin{remark}
  Apolonius proved this geometrically. 
\end{remark}
Equations like \eqref{eq:1-c} are studied in what we now call \emph{algebraic geometry}.
In this course however, we will be interested in the physical applications of \emph{differential geometry} and will look at the following:
\begin{enumerate}[1)]
  \item Hamiltonian mechanics ($~$ mid 19\textsuperscript{th}). This is an elegant way of reformulating Newton's mechanics, turning second order differential equations into first order differential equations with the use of a function $H(p, q)$. The system of ODEs is 
    \begin{equation}
      \dot{q} = \frac{\partial H}{\partial p} \qquad \dot{q} = - \frac{\partial H}{\partial q}
    \end{equation}
    This led to the development of symplectic geometry ($~$ 1960s).
    The connection is that the phase-space to which $p$  and $q$  belong has a $2$ -form $dp \wedge dq$ .
    Using the Hamiltonian function, one can find a vector field 
    \begin{equation}
      X_H = \frac{\partial H}{\partial p} \frac{\partial }{\partial q} - \frac{\partial H}{\partial q} \frac{\partial }{\partial p}
    \end{equation}
    and looks for a one-parameter group of transformations, called symplectomorphisms, generated by this vector field. Under these symplectomorphisms, the $2$ -form is unchanged meaning that the area illustrated in F2 is preserved.
    %F2
    \begin{figure}[tbhp]
      \centering
      \def\svgwidth{0.4\columnwidth}
      \input{lectures/l1f2.pdf_tex}
      \caption{}
      \label{fig:l1f2}
    \end{figure}
    Details of this are going to come within the course.
  \item General Relativity (1915) $\leftarrow$  Riemannian Geometry ($~1850$)
  \item Gauge theory (Maxwell, Yang Mills) $\leftrightarrow$  Connection on Principal Bundle (U(1) (Maxwell), SU(2), SU(3))
    \begin{equation}
      A_+ = A_- + d g \qquad g = \psi_+ - \psi_- \qquad 
      \omega = \left\{ 
	\begin{matrix}
	A_+ + d\psi_+ \\
	A_- + d\psi_- \\
	\end{matrix}
      \right.
    \end{equation}
    \begin{figure}[tbhp]
      \centering
      \def\svgwidth{0.4\columnwidth}
      \input{lectures/l1f3.pdf_tex}
      \caption{}
      \label{fig:l1f3}
    \end{figure}
\end{enumerate}

This course: cover 1, 2, 3 in some detail. Unifying feature: Lie groups.
\begin{itemize}
  \item Prove some theorems, \emph{lots of} examples (often instead of proofs)
  \item Want to be able to do calculations; compute characteristic classes etc.
\end{itemize}
We will assume that you took either Part III General Relativity, or Part III Differential Geometry, or some equivalent course. 

\section{Manifolds}%
\label{sec:manifolds}

\begin{definition}[manifold]
  An $n$ -dimensional \emph{smooth manifold} is a set $M$  and a collection\footnote{In all examples that we will look at, there will be finitely $\alpha$.} of open sets $U_{\alpha}$ , labelled by $\alpha = 1, 2, 3, \dots$ , called \emph{charts} such that
  \begin{itemize}
    \item $U_{\alpha}$  cover $M$
     \item $\exists$  $1$-$1$ maps  $\phi_{\alpha} \colon U_{\alpha} \to V_{\alpha} \in \mathbb{R}^n$  such that
      \begin{equation}
	\phi_{\beta} \circ \phi_{\alpha}^{-1} \colon \phi_{\alpha}(U_{\alpha} \cap U_{\beta}) \to \phi_{\beta}(U_{\alpha} \cap U_{\beta})
      \end{equation}
      is a smooth map from $\mathbb{R}^n$  to $\mathbb{R}^n$ .
      \begin{figure}[tbhp]
        \centering
        \def\svgwidth{0.4\columnwidth}
        \input{lectures/manifold.pdf_tex}
        \caption{Manifold}
        \label{fig:manifold}
      \end{figure}
  \end{itemize}
\end{definition}

As such, manifolds are topological spaces with additional structure, allowing us to do calculus.

\begin{example}[$M = \mathbb{R}^n$]
  There is the \emph{trivial manifold}, which can be covered by only one open set.
  There are other possibilities. In fact, there are infinitely many smooth structures on $\mathbb{R}^4$ (Proof by Donaldson in $1984$ in his PhD. He used Gauge theory).
\end{example}
\begin{example}[sphere $S^n = \left\{ \vb{r} \in \mathbb{R}^{n + 1}, \abs{\vb{r}} = 1 \right\}$ ]
  Intuitively, the $n$-sphere $S^n$ is an $n$-dimensional manifold. To show this, we will construct a map $\phi \colon S^n \to \mathbb{R}^n$ by projecting the north pole $N: (0, 0 \dots, 0, 1) \in \mathbb{R}^{n+1}$, through the point $p \in S^n$ onto the hyperplane defined by $r_{n+1} = 0$. This is illustrated for $S^2$ in Fig.~\ref{fig:l1f5}.
  \begin{figure}[tbhp]
    \centering
    \def\svgwidth{0.4\columnwidth}
    \input{lectures/l1f5.pdf_tex}
    \caption{The map $\phi$ projects the north pole $N$ through a point $p \in S^2$ onto $\phi(p)$, which lies in the $R^2$-plane defined by $z = 0$. This map is not defined for the north-pole itself.}
    \label{fig:l1f5}
  \end{figure}
  From this figure, we can already see that this projection map is ill-defined at the north pole itself.
  However, the definition of a manifold allows us to work with multiple charts. Let us cover $S^n$ with the following two open sets 
  \begin{equation}
    U = S^n / \left\{ 0,0,0,, \dots, 0, 1 \right\}, \qquad
    \widetilde{U} = S^n / \left\{ 0,0,0,, \dots, 0, - 1 \right\},
  \end{equation}
  so $U$ does not include the north pole and $\widetilde{U}$ does not include the south pole.
  We can then define two maps, $\phi$ and $\widetilde{\phi}$, which project from the north pole and the south pole respectively.
  Using $x_i$ to denote coordinates in $\mathbb{R}^n$ and $r_i$ to denote coordinates in $\mathbb{R}^{n+1}$ we take maps
  \begin{align}
    \text{defined on $U$:} \qquad \phi(r_1, \dots, r_{n+1}) &= \left( \frac{r_1}{1 - r_{n+1}}, \dots, \frac{r_n}{1-r_{n+1}} \right) = (x_1 , \dots, x_n) \\
    \text{defined on $\widetilde U$:} \qquad \widetilde{\phi}(r_1, \dots, r_{n+1}) &= \left( \frac{r_1}{1 + r_{n+1}}, \dots, \frac{r_n}{1+r_{n+1}} \right) = (\widetilde{x}_1 , \dots, \widetilde{x}_n).
  \end{align}
  On the overlap $U \cap \widetilde{U}$, the coordinates are related
  \begin{equation}
    \overbrace{\frac{r_k}{1 + r_{n+1}}}^{\mathclap{\widetilde{x}_k}} = \frac{1 - r_{n+1}}{1 + r_{n+1}} \overbrace{\frac{r_k}{1 - r_{n+1}}}^{\mathclap{x_k}}, \qquad k = 1, \dots, n,
  \end{equation}
  where we can write the transition factor on the right in terms of the coordinates $x_i$:
  \begin{equation}
    \frac{1 - r_{n+1}}{1 + r_{n+1}} = \frac{(1 - r_{n+1})^2}{r^2_1 + r^2_2 + \dots + r^2_n} = \frac{1}{x^2_1 + x^2_2 + \dots + x^2_n}.
  \end{equation}
  So on $U \cap \widetilde{U}$, the transition function is
  \begin{equation}
    \widetilde{\phi} \circ \phi^{-1}\colon 
    (x_1, \dots, x_n)
    \mapsto
    (\widetilde{x}_1, \dots, \widetilde{x}_n) = \left( \frac{x_1}{x^2_1 + \dots + x^2_n}, \dots, \frac{x_n}{x^2_1 + \dots + x^2_n} \right),
  \end{equation}
  which is a smooth map from $\mathbb{R}^n \to \mathbb{R}^n$.
\end{example}
\begin{example}[]
  A Cartesian product of manifolds is a manifold, for example we have the $n$-torus $T^n = S^1 \times S^1 \times \dots \times S^1$.
\end{example}
