% lecture notes by Umut Özer
% course: dgp
\lhead{Lecture 16: March 13}

\section{Back to Yang--Mills}%
\label{sec:back_to_yang_mills}

Recall that the boundary conditions at $r \to \infty$ ensure that the components $A_{a} \sim -\partial_{a} g \cdot g^{-1} + O(\frac{1}{r^2})$ are in pure gauge.  We have the map $g \colon S_{\infty}^3 \to SU(2)$, from which we obtained the instanton number as its degree.

There is a theorem by Uhlenbeck that relates boundary conditions on $\mathbb{R}^4$ to solutions to YM extended to $S^4 = \mathbb{R}^4 \cup \left\{\infty\right\}$.
We have seen something similar for sigma models, in the case of $\mathbb{R}^2$ to $S^2$.
\begin{theorem}[Uhlenbeck (1984?)]
  For any instanton on $\mathbb{R}^4$, $\exists c$ principal $SU(2)$ bundle with connection $\omega$ over $S^4$ such that pull backs of $\omega$ to $S^4$ stereographically project to $\mathbb{R}^4$.
\end{theorem}
\begin{leftbar}
  Although any bundle over $\mathbb{R}^4$ is trivial (since the base space is contractible), principal bundles over $S^4$, which is non-contractible, will in general be non-trivial. Thus the Yang--Mills bundle $E \to S^4$ can be non-trivial. We will show that they are classified by the instanton number.
\end{leftbar}
\begin{proof}
  Let us cover $S^4$ with two open sets:
  \begin{equation}
    U_+ = S^4 - \left\{N = (0,0,0,1)\right\}, \qquad
    U_- = S^4 - \left\{S = (0,0,0,-1)\right\}.
  \end{equation}
  We then have a stereographic projection
  \begin{equation}
    f_{\pm} \colon U_{pm} \to \mathbb{R}^4.
  \end{equation}
  \begin{figure}[tbhp]
    \centering
    \inkfig[0.4]{l16f1}
    \caption{Open sets covering $S^4$.}
    \label{fig:l16f1}
  \end{figure}
  Let us now consider two different open sets $U_{\pm}^{\epsilon}$ shown in Fig.~\ref{fig:l16f1}.
  In this case $U_+^\epsilon$ covers the upper half of the sphere through to $-\epsilon$ under the equator. $U_-^\epsilon$ covers the other half, up to $+\epsilon$ above the equator.
  They intersect in a range $(-\epsilon, \epsilon)$ around the equator.
  Taking $\epsilon$ infinitesimal, then 
  \begin{equation}
    U_+^{\epsilon} \cap U_-^{\epsilon} = S^3 \times (-\epsilon, \epsilon).
  \end{equation}
  Now take 
  \begin{equation}
    \gamma_{\pm} \colon U_{\pm}^{\epsilon} \to E.
  \end{equation}
  We then have a connection on $E$:
  \begin{equation}
    \omega = 
    \begin{cases}
      \gamma_+^{-1} (A_+ + d) \gamma_+, & \text{on } \pi^{-1}(U_+^3)  \\
      \gamma_-^{-1} (A_- + d) \gamma_-, & \text{on } \pi^{-1}(U_-^3)  \\
    \end{cases}, \qquad
    \Omega = d \omega + \omega \wedge \omega = \left\{
      \begin{matrix}
        \gamma_+^{-1} F_+ \psi_+ \\
        \gamma_-^{-1} ??\\
      \end{matrix}
    \right.
  \end{equation}
  and $A_+, A_-$ are $\mathfrak{g}$-valued $1$-forms on $U_+, U_-$.
  \begin{equation}
    \gamma_- = g \gamma_+, \qquad A_- = g A_+ g^{-1} - d g \cdot g^{-1}.
  \end{equation}
  This is the gauge transformation on $S^4$.
  And so
  \begin{align}
    k &= -\frac{1}{8 \pi^2} \int_{S^4} \tr(\Omega \wedge \Omega) \\
      &= -\frac{1}{8 \pi^2} \left[ \int_{U_+^{\epsilon}} \tr(F_+ \wedge F_+) + \int_{U_-^{\epsilon}} \tr(F_- \wedge F_-) \right].
  \end{align}
  Taking the limit $\epsilon \to 0$, we have
  \begin{equation}
    k = -\frac{1}{8 \pi^2} \left[ \int_{S^3} \tr(F_+ \wedge A_+ - \frac{1}{3} A_+^3) - \int_{S^3} \tr(F_- \wedge A_- - \frac{1}{3} A_-^3) \right],
  \end{equation}
  where the minus sign in the second term arises since the $S^3$-boundary of $U_-^{\epsilon}$ has the opposite orientation of $U_+^{\epsilon}$.
  \begin{exercise}
    Show that this equals
    \begin{equation}
      k = -\frac{1}{8 \pi^2} \int_{S^3} \tr(\frac{1}{3} (dg \cdot g^{-1})^3) - d(A_+ \wedge dg \cdot d^{-1}).
    \end{equation}
  \end{exercise}
  Now the second summand is a boundary term, which vanishes since $\partial S^3 = \emptyset$. (It arose as the boundary of $U_{\pm}^{\epsilon}$ and the boundary of a boundary is empty.)
  \begin{equation}
    k =-\frac{1}{24 \pi^2} \int_{S^3} \tr(dg \cdot g^{-1})^3 \in \mathbb{Z}.
  \end{equation}
  Here, $g$ has a different interpretation as before ($\mathbb{R}^4$ calculation).
  Now it is a transition function for $(E, B = S^4, \pi, G)$.
\end{proof}

We obtain topological information about / a topological invariant for $(E, B = S^4, \pi, G)$. This is the starting point for Donaldson's work on four-manifolds.
