% lecture notes by Umut Özer
% course: dgp
\lhead{Lecture 9: February 19}

\begin{proof}
  \textcolor{red}{MISSED 15 Mins}
  
  Through every point of $M$, there is exactly one torus. We say that $M$ is \emph{foliated} by tori (different choices of $c_i$).
  
  Moreover, $M_f \subset M$ is a \emph{Lagrangian} submanifold, meaning that $\omega \rvert_{M_f} = 0$.
  This is because we proved that the $n$ Hamiltonian vector fields are tangent to $M_f$. These are independent, so at each point $p \in M_f$ they span the tangent space $T_p M_f$.
  It is thus sufficient to show that
  \begin{equation}
    \omega (X_{f_i}, X_{f_j}) = 0,
  \end{equation}
  which is indeed the case since $\{f_i, f_j\} = 0$.
  
  Now let us construct the action-angle variables $(I, \phi)$.
  Local argument in a neighbourhood of $M_f$ in $M$.
  $\omega = d \theta$.
  
  Taking any closed contractable curve $C$, which does not wind around the whole of the torus, as shown in Fig.~\ref{fig:l9f2}, we have
  \begin{equation}
    \iint_{D \in M_f} \omega = \oint_C \theta = 0.
  \end{equation}
  \begin{figure}[tbhp]
    \centering
    \def\svgwidth{0.4\columnwidth}
    \input{lectures/l9f2.pdf_tex}
    \caption{}
    \label{fig:l9f2}
  \end{figure}
  
  \textcolor{red}{missed something}
  
  The canonical transformation is
  \begin{equation}
    (\vb{p}, \vb{q}) \to (I, \phi)\qquad
    \omega = d \theta = d (\vb{p} \cdot d\vb{q})
  \end{equation}
  In analogy to last time, we define the generating function to be
  \begin{equation}
    S (q, I) = \int_{q_0}^{q} \theta.
  \end{equation}
  The canonical transformations obtained from this generating function are independent of the path taken.
  This is because we can use Stokes' theorem to integrate to zero around the closed loop obtained by concatenating two different paths. Their difference is thus zero.
  If the path winds around the whole in the torus, we can add a bit to the action, which will not impact the derivatives with respect to the action, and thus will not impact the canonical transformations.
  
  We define the angles $\phi_k$ for $k = 1, \dots,  n$ as
  \begin{equation}
    \phi_k = \frac{\partial S}{\partial I_k}.
  \end{equation}
  These obey
  \begin{equation}
    \{ \phi_k, \phi_j \} = 0, \qquad \{ \phi_k, I_j \} = \delta_{kj}.
  \end{equation}
  The Hamiltonian $H = f_i = H(I_1, \dots, I_n)$ gives time-evolution:
  \begin{equation}
    \dot{I}_k = - \frac{\partial H}{\partial \phi_k} = 0, \qquad \dot{\phi}_k = \frac{\partial H}{\partial I_k} = \text{const.}
  \end{equation}
  As in A-L.
\end{proof}

\begin{example}[$M = \mathbb{R}^2$]
  All Hamiltonian systems are integrable.
  Take $H = \frac{1}{2}(p^2+ \omega_0^2 q^2)$, where $(p, q)$ are coordinates on $\mathbb{R}^2$ and $\omega_0$ is a constant frequency.
  Construct
  \begin{equation}
    M_f = \{(p, q) \in \mathbb{R}^2 \suchthat H = E\} = \Gamma.
  \end{equation}
  Imposing one condition on $\mathbb{R}^2$ gives a curve. So $M_f$ is a one-dimensional torus, which has just one cycle $\Gamma$.
  The integral is
  \begin{equation}
    I = \frac{1}{2\pi} \oint_\Gamma \phi d\phi.
  \end{equation}
  We can solve this in two ways. Either ??.
  Or we can use Stokes' theorem, by considering Fig.~\ref{fig:l9f3}.
  \begin{figure}[tbhp]
    \centering
    \def\svgwidth{0.4\columnwidth}
    \input{lectures/l9f3.pdf_tex}
    \caption{}
    \label{fig:l9f3}
  \end{figure}
  We have
  \begin{equation}
    I = \frac{1}{2\pi} \oint_\Gamma \phi d\phi = \frac{1}{2\pi} \iint_{\mathcal{E}} dp dq = \frac{E}{\omega}.
  \end{equation}
  So $I = \frac{H}{\omega}$ and $\dot{\phi} = \frac{\partial H}{\partial I} = \omega$. Therefore, $\phi(t) = \phi(0) + \omega (t)$.
\end{example}

\section{Geodesics in Non-Riemannian Geometries}%
\label{sec:geodesics_in_non_riemannian_geometries}

We want to look at trajectories in Newtonian physics.
These are governed by Newton's equations
\begin{equation}
  \label{eq:9-star}
  \ddot{\vb{x}} = - \grad V, \qquad V \colon \mathbb{R}^3 \to \mathbb{R}^3,
\end{equation}
where the dot represents derivative with respect to an absolute time $t$.
These are invariant under the Galilean transformations.
We can reformulate (after Cartan) \eqref{eq:9-star} as geodesic motion.
Newtonian spacetime, like the spacetime of GR, is also four-dimensional. However, the geometry is different.
We assemble time and space into a vector $x^a = (t, x^{i})$ and look for a connection such that the geodesic equation $ \ddot x^{a}  + \Gamma^{a}_{bc} \dot x^{b} \dot x^{c} = 0 $ is the same as \eqref{eq:9-star}.
The unparametrised solution (i.e.~parametrised by a coordinate $t$) is
\begin{equation}
  \Gamma^{i}_{00} = \delta^{ij} \frac{\partial V}{\partial x^{j}},
\end{equation}
and all other components vanishing.
Importantly, this is not a Levi-Civita connection, which means that there is no metric underlying it!
We call this a \emph{non-Riemannian} connection.
Cartan asked what geometry underlies Newtonian physics.

\section{Newton--Cartan Geometry}%
\label{sec:newton_cartan_geometry}

Newton--Cartan geometry in $n$ dimensions is a triple $(h, \theta, \nabla)$, where
\begin{itemize}
  \item $h = h^{ab} \frac{\partial }{\partial x^{a}} \odot \frac{\partial }{\partial x^{b}}$ is a degenerate rank-$(n-1)$ symmetric tensor
  \item $\theta$ is a closed 1-form, called the \emph{clock}, in the kernel of $h$. This means
    \begin{equation}
      h^{ab} \theta_{a} = 0 \qquad d\theta = 0 \rightarrow \theta = d t,
    \end{equation}
    where $t$ is \emph{global time}.
  \item torsion-free connection $\nabla$ preserving $\theta$ and $h$, meaning that
    \begin{equation}
      \nabla_a h^{bc} = 0 \qquad \nabla_a \theta_b = 0.
    \end{equation}
\end{itemize}
In our example, $h = \text{diag}(0,1, \dots, 1)$. The connection $\nabla$ has symbols
\begin{equation}
  \Gamma^i_{00} = h^{ij} \frac{\partial V}{\partial x^{j}}.
\end{equation}
The Newton--Cartan spacetime is drawn in Fig.~\ref{fig:l9f4}.
\begin{figure}[tbhp]
  \centering
  \def\svgwidth{0.4\columnwidth}
  \input{lectures/l9f4.pdf_tex}
  \caption{Newton--Cartan spacetime. The fibres define a notion of \emph{simultaneity}. A geodesic of $\nabla$ cuts through the fibres. A contravariant metric on the fibre is given by $h$.}
  \label{fig:l9f4}
\end{figure}
