% lecture notes by Umut Özer
% course: dgp
\lhead{Lecture 15: March 11}

\section{The Connection and Curvature on Principal Bundles}%
\label{sec:the_connection_and_curvature_on_principal_bundles}

\begin{definition}[connection]
  The \emph{connection} on a principal bundle $(E, B, \pi, G)$ is a $\mathfrak{g}$-valued 1-form $\omega$ whose vertical (fibre) component is the Maurer--Cartan 1-form on $G$.
\end{definition}
In local coordinates,
\begin{equation}
  \omega = \gamma^{-1} (A + d) \gamma,
\end{equation}
where $A$ is a $\mathfrak{g}$-valued $1$-form on $B$ and $\gamma \in G$ (fibre bundle).

Say $U, U'$ are overlapping open sets on $B$, $g_{U U'} = g$ is the transition function acting on $G$ by left multiplication, i.e.~$\gamma' = g \gamma$.

On $U \cap U'$.
\begin{equation}
  \gamma^{-1} (A + d) \gamma = (\gamma')^{-1} (A' + d) \gamma'.
\end{equation}
Let us compute the right-hand side:
\begin{equation}
  \gamma^{-1} g^{-1} A' g \gamma + \underbrace{\gamma^{-1} g^{-1} (d g \cdot \gamma + g d \gamma)}_{\gamma^{-1} g^{-1} d g \gamma + \gamma^{-1} d \gamma}.
\end{equation}
Comparing this to the left-hand side, the term $\gamma^{-1} d \gamma$ cancels. We are left with
\begin{equation}
  \gamma^{-1} A \gamma = \gamma^{-1} (g^{-1} A' g + g^{-1} d g) \gamma.
\end{equation}
From this we obtain
\begin{equation}
  \boxed{A' = g A g^{-1} - dg \cdot g^{-1}}
\end{equation}
In other words, $A, A'$ are related by a gauge transformation \eqref{eq:12-gauge-transformation} on $B$.

\begin{definition}[curvature]
  The \emph{curvature} $\Omega$ of a connection $\omega$ is a $\mathfrak{g}$-valued $2$-form on $E$ defined by
  \begin{equation}
    \Omega = d \omega + \omega \wedge \omega.
  \end{equation}
\end{definition}
Working in a local trivialisation,
\begin{align}
  \Omega &= d (\gamma^{-1} A \gamma + \gamma^{-1} d \gamma) + \omega \wedge \omega \\
  &= -\gamma^{-1} d \gamma \gamma^{-1} \wedge A \gamma + \gamma^{-1} d A \gamma - \gamma^{-1} A \wedge d \gamma - \gamma^{-1} d \gamma \gamma^{-1} \wedge d \gamma \nonumber \\
  &\qquad + \gamma^{-1} A \gamma \wedge \gamma^{-1} A \gamma + \gamma^{-1} A \gamma \wedge \gamma^{-1} d \gamma + \gamma^{-1} d \gamma \wedge \gamma^{-1} A \gamma + \gamma^{-1} d \gamma \wedge \gamma^{-1} d \gamma \\
  &= \gamma^{-1} (d A + A \wedge A) \gamma = \gamma^{-1} F \gamma,
\end{align}
where $F = dA + A \wedge A$ is a $\mathfrak{g}$-valued $2$-form on $B$.
The curvature $\Omega$ does not have the vertical (fibre) component.

Writing $\Omega = (\gamma')^{-1} F' \gamma'$ gives
\begin{equation}
  \boxed{F' = g F g^{-1}}
\end{equation}
The base-components of both $\omega$ and $\Omega$ are related by gauge transformations \eqref{eq:12-gauge-transformation}.

Let $\gamma \colon U \to G$ be any local section, $\gamma =\gamma(x)$, where $x \in U \subset B$.
This can be used to pull back $\omega$ and $\Omega$ from the total space $E$ to the base space $B$.
\begin{equation}
  A = \gamma^* (\omega), \qquad F = \gamma^* (\Omega).
\end{equation}
The $A$ and $F$ defined this way are called the \emph{gauge potential} and \emph{gauge field} respectively.
This reveals the bundle-theoretic origin of gauge transformations.
In physics we only work with the base space and say that the gauge potential is only defined locally up to gauge transformation. 
We see now that this gauge ambiguity is an artefact of an ambiguity in the choice of section on the fibre structure. A gauge transformation on $B$ is the same as a change of section of $E$.
In particular, if the bundle $E$ is non-trivial, then a global section does not exist, and the gauge potential can only be defined locally.

Given a connection $\omega$, we can split the tangent bundle $T E$ into vertical and horizontal sub-bundles
\begin{equation}
  T E = H(E) \oplus V(E),
\end{equation}
where $H(E), V(E)$ are defined as
\begin{equation}
  H(E) = \left\{X \in TE \suchthat X \intprod \omega = 0\right\}.
\end{equation}
A basis of $H(E)$ (in a trivialisation) is
\begin{equation}
  H(E) = \text{Span}\left\{D_a\right\}, \qquad D_a = \frac{\partial }{\partial x^{a}} - A \indices{_a^{\alpha}} R_{\alpha}, \qquad 
  \begin{gathered}
    a = 1, \dots, \dim B \\
    \alpha = 1, \dots, \dim \mathfrak{g}
  \end{gathered}
\end{equation}
where $x^{a}$ are local coordinates on $U$ and $R_{\alpha}$ are right-invariant vector fields on $G$ with
\begin{equation}
  [R_{\alpha}, R_{\beta}] = -f\indices{_{\alpha\beta}^{\gamma}} R_{\gamma}.
\end{equation}
The curvature is an obstruction to integrability in $H(E)$.
\begin{exercise}
  Compute the Lie bracket
  \begin{equation}
    [D_{a}, D_{b}] = -F\indices{_{ab}^{\alpha}} R_{\alpha}.
  \end{equation}
  (Use $[T_{\alpha}, T_{\beta}] = f\indices{_{\alpha\beta}^{\gamma}} T_{\gamma}$ and the component form of the pull-back.)
  The right-hand side belongs to the vertical component.
\end{exercise}
Other texts define the connection from the splitting above and then recover the $1$-form from it.
