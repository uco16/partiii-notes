% lecture notes by Umut Özer
% course: dgp
\lhead{Lecture 8: February 17}

\section{Geodesics, Killing vectors, Killing tensors}%
\label{sec:geodesics_killing_vectors_killing_tensors}

We will explore the connection between Riemannian and symplectic geometry.
Let $(M, g)$ be a (pseudo-)Riemannian manifold of dimension $n$.
In coordinates, we have
\begin{equation}
  g = g_{ij} (x) dx^{i} dx^{j}.
\end{equation}
Then we know from the \emph{General Relativity} course, that there exists at unique Levi-Civita connection $\Gamma^{i}_{jk}$ such that for geodesics $x^{i} = x^{i}(\tau)$, we have
\begin{equation}
  \ddot{x}^{i} + \Gamma^{i}_{jk} \dot{x}^{i} \dot{x}^{j} = 0.
\end{equation}
Geodesics on $M$ are integral curves of a Hamiltonian vector field on $(T^*M, \omega)$.

As illustrated in Fig.~\ref{fig:l8f1}, we are looking for a curve in $(T^*M, \omega)$ specified by a single point $(x^{i}, p_i)$ which projects down to the geodesic in $M$.
\begin{figure}[tbhp]
  \centering
  \def\svgwidth{0.6\columnwidth}
  \input{lectures/l8f1.pdf_tex}
  \caption{}
  \label{fig:l8f1}
\end{figure}

\begin{align}
  \dot{x}^{i} &= p^{i} = g^{ij} p_j = X_H (x^{i}) \\
  \dot{p}^{i} &= -\Gamma^{i}_{jk} p^{i} p^{k} = X_H (p^{i})
\end{align}

The Hamiltonian vector field is 
\begin{equation}
  X_H = g^{ij} p_i \frac{\partial }{\partial x^{j}} - \Gamma^{i}_{jk} p^{j} p^{k} \frac{\partial }{\partial p^{i}},
\end{equation}
where $H = \frac{1}{2} g^{ij}(x) p_{i} p_{j}$.

Canonical symplectic form $\omega = dp_i \wedge dx^{i}$.
\begin{align}
  H \rightarrow X_H &= \frac{\partial H}{\partial p_i} \frac{\partial }{\partial x^{i}} - \frac{\partial H}{\partial x^{i}} \frac{\partial }{\partial p_i} \\
  &= g^{ij} p_j \frac{\partial }{\partial x^{i}} - \frac{1}{2} \frac{\partial g^{jk}}{\partial x^{i}} p_j p_k \frac{\partial }{\partial p_i}.
\end{align}
Now use
\begin{equation}
  \label{eq:8-star}
  \nabla_k g_{ij} = \partial_k g_{ij} - \Gamma^{m}_{ki} g_{jm} - \Gamma^{m}_{kj} g_{im} = 0
\end{equation}

\begin{definition}[Killing vector]
  \emph{Killing vectors} $K$ satisfy
  \begin{equation}
    \mathcal{L}_K g = 0 \iff \nabla_{(i} K_{j)} = 0.
  \end{equation}
\end{definition}
These correspond to first integrals of the Hamiltonian flow which are linear in the momenta.
Poisson commuting with $H$.
\begin{align}
  \{ \underbrace{\kappa^{i} p_i}_{\mathclap{\kappa}}, H \}_{\text{PB}} &= \frac{\partial \kappa^{i}}{\partial x^{j}} p_{i} g^{jk} p_k - \frac{1}{2} \kappa^{i} \frac{\partial g_{jk}}{\partial x^{i}} p^{j} p^{k} \\
								       &\stackrel{\eqref{eq:8-star}}{=} \nabla_{(i} K_{j)} p^{i} p^{j} = 0
\end{align}
Killing vectors are symmetries of the Hamiltonian flow.

\subsection{Killing Tensors}%
\label{sub:killing_tensors}

\begin{definition}[Killing tensor]
  Write
  \begin{equation}
    \kappa = K^{ij \dots k} \underbrace{p_i p_j \dots p_k}_{\mathclap{r}}.
  \end{equation}
  We assume nothing about $\kappa$ except that it Poisson-commutes with the Hamiltonian $H$.
  We find that this corresponds to
  \begin{equation}
    \{ H, \kappa \}_{\text{PB}} = 0 \iff \nabla_{(i} \kappa_{jk \dots l)} = 0.
  \end{equation}
  A $\kappa$ satisfying this is called a rank-$r$ \emph{Killing tensor}.
\end{definition}
We will see that we can think of Killing tensors as higher / hidden symmetries, since we can see them in the cotangent bundle but not in the manifold itself.

The associated Hamiltonian vector field is
\begin{align}
  X_\kappa &= \frac{\partial \kappa}{\partial p_{i}} \frac{\partial }{\partial x^{i}} - \frac{\partial \kappa}{\partial x^{i}} \frac{\partial }{\partial p_i} \\
  &= r K^{i_1 \dots i_r} p_{i_1} \dots p_{i_{r-1}} \frac{\partial }{\partial x^{i_r}} - \frac{\partial K^{i_1 \dots i_r}}{\partial x^k} p_{i_1} \dots p_{i_r} \frac{\partial }{\partial p_k}.
\end{align}
Projecting this down onto the manifold gives
\begin{equation}
  \pi_k (X_\kappa) = 
  \begin{cases}
    0, & \text{if } r > 1 \\
    K^i \frac{\partial }{\partial x^{i}}, & \text{if } r = 1. 
  \end{cases}
\end{equation}
If $r >1$, there is no `geometric' symmetry on $M$, but $\kappa$ is constant along geodesic.

\begin{example}[]
  The Kerr black hole does not have enough Killing vectors to solve the geodesic equations, but using the Killing tensors allows us to find the geodesics.
\end{example}

\section{Integrability}%
\label{sec:integrability}

Intuitively, a Hamiltonian system (e.g.~geodesic motion) is \emph{integrable} if there exist sufficiently many first integrals, i.e.~functions constant along the flow of the Hamiltonian vector field $X_H$.

\begin{definition}[integrable system]
  An \emph{integrable system} is a symplectic manifold $(M, \omega)$ of dimension $\dim M = 2n$, together with $n$ functions $f_i \colon M \to \mathbb{R}$, $i = 1, \dots, n$, with the properties of
  \begin{description}
    \item[involution:] $\{f_i, f_j\} = 0$ for all $i, j$,
    \item[independence:] $df_i \wedge df_j \wedge \dots \wedge df_n \neq 0$.
  \end{description}
\end{definition}
This is a strange definition because it does not specify any dynamics, any equations of motion.

The point is that with such a system any $f_i$ in our system can be declared to be a Hamiltonian. The corresponding Hamilton's equations will be solvable!

\begin{theorem}[Arnold--Liouville]
  Let $(M, \omega, f_i)$ form an integrable system with the Hamiltonian chosen to be $H = f_1$.
  Then
  \begin{itemize}
    \item The level set $M_f = \{x \in M \suchthat f_1 = c_1, \dots, f_n = c_n\}$ (if connected\footnote{If $M_f$ is not connected, the theorem applies to each connected component.}) is diffeomorphic to $\mathbb{R}^k \times T^{n-k}$ for some $0 \leq k \leq n$.\footnote{Usually this theorem is stated with $M_f$ \emph{compact}. In that case, we have $k = 0$ and we just have a torus $T^n$.}
    \item There exists a canonical transformation to action-angle variables
      \begin{equation}
        \phi_1, \dots, \phi_n, I_1, \dots, I_n
      \end{equation}
      such that in a neighbourhood of $M_f$ in $M$, $\phi_i$ are the coordinates, called \emph{angles}, on $M_f$, and $I_i$ are first integrals, called the \emph{actions}.
    \item Hamilton's equations are solvable by quadratures 
      \begin{equation}
	\dot{I}_i = \frac{\partial H}{\partial \phi_i} = 0 \qquad \dot{\phi}_i = \frac{\partial H}{\partial I_i} = \Omega_i (I_1, \dots, I_n),
      \end{equation}
      so $I_i(t) = I_i(0)$ and $\phi_i(t) = \phi_i(0) + \Omega_i t$.
  \end{itemize}
\end{theorem}
\begin{proof}
  The $d f_k$ for $k = 1, \dots, n$ are independent, so $M_f$ is a manifold of dimension $n$.
  Any function $f_k$ gives rise to a Hamiltonian vector field $X_{f_k}$. 
  If we contract this with any differential, we have
  \begin{equation}
    X_{f_k} \rightharpoonup d f_j = X_{f_k} (f_j) = - \{ f_k, f_j \}_{\text{PB}} = 0, \qquad \forall j,k.
  \end{equation}
  If $d f_k$ are normal to $M_f$, then the corresponding Hamiltonian vector fields $X_{f_j}$ are tangent to $M_f$.
\end{proof}
