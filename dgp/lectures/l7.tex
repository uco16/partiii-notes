% lecture notes by Umut Özer
% course: dgp
\lhead{Lecture 7: February 10}

Last time, we discussd Poisson structures, which we can now specialise.

\begin{definition}[symplectic manifold]
  A \emph{symplectic manifold} is a smooth manifold $M$  of dimension $2n$  with a closed 2-form $\omega \in \Lambda^{2}(\mathcal{M})$ , which is non-degenerate, meaning that
  \begin{equation}
    \underbrace{\omega \wedge \omega \wedge \dots \wedge \omega}_{\mathclap{n}} \neq 0,
  \end{equation}
  or $\omega$ is a $2 \times 2$ matrix of maximal rank.
\end{definition}

The symplectic form $\omega$ provides an isomorphism between $TM$ and $T^*M$ as
\begin{equation}
  v \in TM \mapsto v \rightharpoonup \omega \in T^*M.
\end{equation}
However, we tave to be more careful as this is now antisymmetric.
If $f \colon M \to \mathbb{R}$, then $df$ is a 1-form and is naturally associated to a Hamiltonian vector field $X_f$,
\begin{equation}
  X_f \rightharpoonup \omega = -d f,
\end{equation}
where $f$ is the Hamiltonian.

\begin{claim}
  We can define a Poisson bracket by
  \begin{equation}
    \{ f, g \}_{\text{PB}} \coloneqq X_g(f) = \omega(X_g, X_f)
  \end{equation}
\end{claim}
\begin{remark}
  Note that this is antisymmetric since $\omega(X_g, X_f) = - \omega(X_f, X_g)$.
\end{remark}

In local coordinates, the Poisson bracket is
\begin{equation}
  \{f, g\} = \sum_{i, j = 1}^{2n} \omega^{ij} \frac{\partial f}{\partial x^{j}} \frac{\partial g}{\partial x^{i}}.
\end{equation}
\begin{exercise}
  The Jacobi identity follows from the closure $d\omega = 0$.
\end{exercise}

If we compute the Lie bracket of two vector fields $X_f, X_g$, we find the \emph{anti-homomorphism}
\begin{equation}
  [X_f, X_g] = - X_{\{f, g\}}.
\end{equation}

Hamiltonian vector fields preserve the symplectic form.
The finite way of saying this is that they generate a flow under which the symplectic form is invariant.
Infinitesimally this is expressed with the Lie derivative as
\begin{equation}
  \mathcal{L}_{X_f} \omega = d(X_f \rightharpoonup \omega) + X_f \rightharpoonup \cancel{d\omega} = - d(df) = 0.
\end{equation}

\begin{theorem}[Darboux]
  Let $(M, \omega)$  be a $2n$ -dimensional symplectic manifold. There exist \emph{local} coordinates $x^1 = q^1, \dots, x^n = q^n, x^{n+1} = p_1, \dots, x^{2n} = p_n$ around any point in $M$, such that 
  \begin{equation}
    \omega = \sum_{a=1}^{n} dp_a \wedge dq^a,
  \end{equation}
  and the Poisson bracket takes the standard form.
\end{theorem}
\begin{proof}
  The proof proceeds by induction with respect to half of the dimension of the symplectic manifold.
  We first choose an arbitrary function $p_1 \colon M \to \mathbb{R}$ (which will later become the first momentum coordinate).
  Given this function, we search for another function $q^1 \colon M \to \mathbb{R}$ such that 
  \begin{equation}
    \label{eq:7-1}
    X_{p_1}(q^1) = 1.
  \end{equation}
  We denote this as $\dot q^1 = 1$.
  This is an ordinary differential equation, which will have a solution\footnote{The assumption for the existence theorem would be that we work in a Lipshitz-class of functions.} subject to some initial condition.

  The second step is as follows.
  Consider the surface $M_1 = \{x \in M, p_1 = \text{const.}, q^1 = \text{const.}\}$.
  This is a submanifold $M_1 \subset M$, which we can prove from the embedding theorem; the maximal rank condition of the Jacobian is encoded in \eqref{eq:7-1}.
  This $M_1$ is locally symplectic with symplectic form $\omega_1 \equiv \omega\rvert_{p_1, q^1 \text{ const.}}$

  Now look for $p_2, q^2$ and so on.
  A full proof is given in the book by Arnold.
\end{proof}

\begin{claim}
  Let $Q$, which will play the role of configuration space, be an $n$-dimensional manifold.
  The cotangent bundle $T^* Q$ admits a global symplectic structure.
\end{claim}
\begin{proof}
  Our goal is to describe what this symplectic structure is.
  \begin{figure}[tbhp]
    \centering
    \def\svgwidth{0.6\columnwidth}
    \input{lectures/l7f1.pdf_tex}
    \caption{}
    \label{fig:l7f1}
  \end{figure}
  We have a projection $\pi \colon T^* Q \to Q$ acting as $\pi(q, p) = q$. 
  \begin{definition}[pull-back]
    Given a map $f \colon M \to N$, the  \emph{pull-back} $f^* \colon T^*_{f(p)} N \to T_p^* M$  defined as
    \begin{equation}
      f^*(p) (V) = p(f_* V).
    \end{equation}
    In local coordinates, taking $x^i$, with  $i = 1, \dots, \dim M$  coordinates on $M$ and  $y^a$ with  $a = 1, \dots, \dim N$  coordinates on $N$, we can use the chain rule to write explicitly
     \begin{equation}
      f^*(dy^a) = \sum_i \frac{\partial f^a}{\partial x^{i}} dx^{i}.
    \end{equation}
  \end{definition}

  The pull-back of $\pi$ is a map
  \begin{equation}
    \pi^* \colon T^*(Q) \to T^* (T^* Q).
  \end{equation}
  Say that $p \in T^* (Q)$ is a one-form on $Q$. We can define $\theta = \pi^* (p)$ and from this we get our canonical symplectic form
  \begin{equation}
    \omega = d \theta.
  \end{equation}
  This is manifestly closed.
  This construction does not depend on coordinates, but usually in $(p_{\dot{a}}, q^{\dot{a}})$ coordinates we have
  \begin{equation}
    \theta = p_{\dot{a}} dq^{\dot{a}}, \qquad \omega = d\theta = dp_a \wedge dq^a.
  \end{equation}
\end{proof}

We can define yet another structure that symplectic geometry gives us.
\begin{definition}[Canonical transformations]
  Let $(M, \omega)$ be a $2n$-dimensional symplectic manifold. An endomorphism $f \colon M \to M$ is called \emph{canonical} when $f^* (\omega) = \omega$.
\end{definition}

The one-parameter groups of canonical transformations are generated by Hamiltonian vector fields.
For dimension $n = 1$, these are area preserving maps.

Consider the canonical transformation  $(p_a, q^a) \xrightarrow{f} (P_a, Q^a)$ .
\begin{equation}
  d(\vb{p} \cdot d\vb{q}) = - d(\vb{Q} \cdot d\vb{P}),
\end{equation}
where $\vb{P} = P(p, q)$ and $Q = Q(p, q)$ .
Then
\begin{equation}
  d(\vb{p} \cdot d\vb{q} + \vb{Q} d \vb{P}) = 0.
\end{equation}
So the one-form $\vb{p} \cdot d \vb{q} + \vb{Q} \cdot d\vb{P}$  is a closed-one form. Locally, this implies that it is exact, meaning that there is some \emph{generating function} $S = S(\vb{q}, \vb{P})$  such that
\begin{equation}
  \vb{p} \cdot d \vb{q} + \vb{Q} d \vb{P} = d S.
\end{equation}
From this function we can define $Q$  and $p$ as
 \begin{equation}
  Q^a = \frac{\partial S}{\partial p_a}, \qquad p_a = \frac{\partial S}{\partial q^a}.
\end{equation}
This gives $\vb{P} (q, p)$ and $Q(q, p)$.
