% lecture notes by Umut Özer
% course: dgp
\lhead{Lecture 10: February 24}

\chapter{Topological Charges}%
\label{cha:topological_charges}

\section{Kinks}%
\label{sec:kinks}

Consider Minkowski space $\mathbb{M}^2$ with metric $ds^2 = dt^2 - dx^2$, and let $\phi \colon \mathbb{M}^2 \to \mathbb{R}$ be a scalar field with Lagrangian
\begin{equation}
  L = \int_\mathbb{R} \left( \frac{1}{2} (\phi_t^2 - \phi_x^2) - U(\phi) \right) \dd[]{x}.
\end{equation}
where we denote the partial derivatives $\phi_t = \pdv{\phi}{t}$ and $\phi_x = \frac{\partial \phi}{\partial x}$.
\begin{equation}
  L = T - V, \qquad T = \frac{1}{2} \int \phi_t^2 \dd[]{x}
\end{equation}
and $T + V$ is the total energy.
The Euler--Lagrange equations give the equation of motion
\begin{equation}
  \phi_{tt} - \phi_{xx} = -\dv{U}{\phi}.
\end{equation}

Assume $U(\phi) \geq U_0$ for stability and normalise $U_0 = 0$.
Assume further that $U^{-1}(0) = \{\phi_1, \phi_2, \dots\}$ is a discrete set with at least two elements.
\begin{figure}[ht]
    \centering
    \inkfig[0.5]{l10f1}
    \caption{}
    \label{fig:l10f1}
\end{figure}
We can approach this problem in various different ways.
\begin{description}
  \item[Perturbation theory:] Choose a ground state $\phi_1$ and let $\phi \approx \phi_1 + \delta \phi$, where $\delta \phi$ is small. The Euler--Lagrange quation becomes the Klein--Gordon equation for a scalar boson
    \begin{equation}
      (\Box + m^2) \delta \phi = 0.
    \end{equation}
  \item[Solitons:] non-singular, static, finite energy solutions to EL.
    In general, $\phi \to \varphi \in U^{-1}(0)$.
\end{description}

\begin{definition}[soliton]
  \emph{Solitons} are non-singular, static, finite energy solutions to the Euler--Lagrange equations.
\end{definition}

\begin{example}[Kink]
  The kink solution has the asymptotic behaviour
  $\phi \to \phi_1$ as $x \to - \infty$ and $\phi \to \phi_2$ as $x \to + \infty$.
  This is illustrated in Fig.~\ref{fig:l10f2}.
\begin{figure}[ht]
    \centering
    \inkfig[0.5]{l10f2}
    \caption{Scalar kink}
    \label{fig:l10f2}
\end{figure}

  There is no way we can obtain this in perturbation theory, since $\phi_2$ is finite distance away from $\phi_1$.
  This kink is inherently a non-perturbative configuration.
\end{example}

Let us try and get an analytical handle on this.
Assume
\begin{equation}
  U = \frac{1}{2} \left( \dv{W}{\phi} \right), 
\end{equation}
for some function $W(\phi)$, which is often called the \emph{superpotential}, although there is no supersymmetry here.
\begin{align}
  E &= \frac{1}{2} \int_\mathbb{R} \left[ \phi_t^2 + \phi_x^2 + (W_\phi)^2 \right] \dd[]{x}  \\
    &= \frac{1}{2} \int_\mathbb{R} \left[ \phi_t^2 + (\phi_x \mp W_\phi)^2 \right] \dd[]{x} 
    \pm \int_\mathbb{R} \underbrace{\dv{W}{\phi} \dv{\phi}{x} \dd[]{x}}_{\mathclap{\dd[]{W}}} \label{eq:10-1} \\
    &\underbrace{\geq W(\phi(\infty)) - W(\phi(-\infty))}_{\text{Bogomolny bound}}
\end{align}
The B-bound is saturated for the \emph{Bogomolny equations}
\begin{equation}
  \boxed{\pdv{\phi}{t} = 0, \qquad \dv{\phi}{x} = \pm \dv{W}{\phi}}
\end{equation}
From a variational perspective, they are absolute minimisers of the action.
Moreover, finding these absolute minima can be obtained by solving first order Bogomolny equations rather than second order Euler--Lagrange equations. We will find that this is often possible even when the theories are not integrable.
\begin{equation}
  x - x_0 = \pm \int^\phi \frac{1}{\sqrt{2 U}} \dd[]{\phi}.
\end{equation}

Let us consider again the boundary term that we managed to integrate in \eqref{eq:10-1}.
\begin{definition}[topological charge]
  Define $\phi_{\pm} = \lim_{x \to \pm \infty}\phi$. The \emph{topological charge} $N$ is the difference
  \begin{equation}
    N = \phi_+ - \phi_- = \int_{-\infty}^{+\infty} \dv{\phi}{x} \dd[]{x}.
  \end{equation}
\end{definition}

To define $N$, we do not need to know the field equations for $\phi$. We just need to know about their limiting behaviour.
It depends on boundary conditions and is conserved if the energy is finite.
However, this is not a Noether conservation law, since it does not involve the field equations.

If the field is continuous, we would need to have a continuous deformation of the red curve in Fig.~\ref{fig:l10f2}. Then ??.
This would imply that the energy is infinite.

There are three possibilities for the topological charge of solitons, depicted in Fig.~\ref{fig:l10f3}.
\begin{figure}[ht]
    \centering
    \inkfig[1]{l10f3}
    \caption{Topological charges $N$ of solitons.}
    \label{fig:l10f3}
\end{figure}

\section{Degree of a Map}%
\label{sec:degree_of_a_map}

\begin{definition}[topological degree]
  Let $M, M'$ be oriented, compact manifolds without boundary of the same dimension.
  Let $f \colon M \to M'$ be a smooth map.
  The topological degree $\text{def}(f)$ is a number defined by
  \begin{equation}
    \label{eq:10-deg}
    \int_M f^* (\text{vol}(M')) = \text{deg}(f) \int_{M'} \text{vol}(M').
  \end{equation}
\end{definition}

\begin{theorem}[]
  The topological degree is an integer, given by
  \begin{equation}
    \text{deg}(f) = \sum_{x \in f^{-1}(y)} \text{sign} (J(x)),
  \end{equation}
  with $J = \det(\frac{\partial y^{i}}{\partial x^{j}})$ and $y$ is regular.
  It does not depend on the volume form $\text{vol(M')}$.
\end{theorem}

\begin{example}[]
  Take both $M$ and $M'$ to be circles.
  We then have $f \colon S^1 \to S^1$. The degree of $f$ is then the winding number.
  \begin{figure}[ht]
    \centering
    \inkfig[0.7]{l10f4}
    \caption{A function $f \colon S^1 \to S^1$.}
    \label{fig:l10f4}
  \end{figure}
  For the function illustrated in Fig.~\ref{fig:l10f4}, the topological degree is
  \begin{equation}
    \text{deg}(f) = \sum_{\mathclap{\theta \colon f(\theta) = f_0}} \text{sign}\left( \frac{\partial f}{\partial \theta} \right) = 1 - 1 + 1 +1 -1 = 1,
  \end{equation}
  where we only include the identified point at $\theta = 0 \sim \theta = 2\pi$ once.
  The fact that $\text{def}(f) = 1$ shows that the function wraps up once.
  Equivalently, 
  \begin{equation}
    \text{deg}(f) = \frac{1}{\text{vol}(S^1)} \int_{S^1} df = \frac{1}{2\pi} \int_0^{2 \pi} \dv{f}{\theta} \dd[]{\theta}.
  \end{equation}
  If we view $S^1$ as a set of complex numbers with modulus $1$, and let $f(z) = z^k$, then $\text{deg}(f) = k$.
\end{example}
