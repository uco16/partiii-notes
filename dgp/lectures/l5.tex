% lecture notes by Umut Özer
% course: dgp
\lhead{Lecture 5: February 03}

%Example continued
\begin{remark}
  Examples such as the lecture today will be important for the exam!
  If someone gives you a matrix Lie group, you will proceed in this order.
\end{remark}
Taking the inverse of \eqref{eq:4-g}, we construct the Maurer--Cartan 1-form
\begin{align}
  \rho = g^{-1} dg &=
  \begin{pmatrix}
    1 & -x & -z+xy \\
   0 & 1 & -y \\
   0 & 0 & 1 \\
  \end{pmatrix}
  \begin{pmatrix}
    & dx & dz \\
    &  & dy \\
    &  &  \\
  \end{pmatrix} \\
		   &= T_1 dx + T_2 dy + T_3 (dz - x dy).
\end{align}
We define the dual basis of left-invariant one-forms
\begin{align}
  \sigma^1 &= dx, & \sigma^2 &= dy, & \sigma^3 &= dz - x dy \\
  d\sigma^1 &= 0 & d\sigma^2 &= 0 & d \sigma^3 &= -dx \wedge dy = d\sigma^1 \wedge \sigma^2.
\end{align}
From these we find the left-invariant vector fields
\begin{equation}
  L_1 = \frac{\partial }{\partial x}, \qquad L_2 = \frac{\partial }{\partial y} + x \frac{\partial }{\partial z}, \qquad L_3 = \frac{\partial }{\partial z}.
\end{equation}
We find the Lie algebra by computing the brackets
\begin{equation}
  [L_1, L_2] = L_3, \qquad [L_1, L_3] = 0, \qquad [L_2, L_3] = 0.
\end{equation}
This is the same as \eqref{eq:heis-la}, the Lie algebra of the Heisenberg group.
We have a choice of representing the Lie algebra either in terms of matrices, $T_i$ or left-invariant vector fields $L_i$.

\subsection{Right-Invariance}%
\label{sub:right_invariance}

We could have defined the \emph{right translations} $R_g(h) = h \cdot g$ , and associated right-invariant vector fields and one-forms.
\begin{claim}
  The commutation relations for right-invariant vector fields differs by a minus sign from left-invariant vector fields and they commute
  \begin{equation}
    [R_{\alpha}, R_{\beta}] = -f\indices{_{\alpha\beta}^{\gamma}} R_\gamma, \qquad [R_{\alpha}, L_{\beta}] = 0.
  \end{equation}
\end{claim}
\begin{example}[Nil]
  For the Heisenberg group, using $dg \cdot g^{-1}$
  \begin{equation}
    \label{eq:5-r}
    R_1 = \frac{\partial }{\partial x} + y \frac{\partial }{\partial z}, \qquad R_2 = \frac{\partial }{\partial y}, \qquad R_3 = \frac{\partial }{\partial z}.
  \end{equation}
\end{example}
\begin{notation}[\danger]
  Right-invariant vector fields are said to generate left-translations and vice-versa.
\end{notation}

\section{Metrics on Lie Groups}%
\label{sec:metrics_on_lie_groups}

We can do some local differential geometry by defining a left-invariant metric on $G$.
\begin{definition}[left-invariant metric]
  A \emph{left-invariant metric} $h$ on a Lie group $G$ is of the form
  \begin{equation}
    h = g_{\alpha\beta} \sigma^{\alpha} \odot \sigma^{\beta}, \qquad \alpha, \beta = 1, \dots , \dim G,
  \end{equation}
  where $g_{\alpha\beta}$ is a non-degenerate constant matrix.
\end{definition}
Given the group and left-invariant metric, the right-invariant vector fields are generating isometries for this metric, which means that they are Killing vectors.
Recall that the Maurer--Cartan 1-form is $\rho = g^{-1} dg = \sigma^{\alpha} \otimes T_{\alpha}$, where $\sigma^{\alpha}$ are invariant under $g \to g_0  g$.
Therefore 
\begin{equation}
  \mathcal{L}_{R_{\alpha}} \sigma^{\beta} = 0 \qquad \forall \alpha, \beta.
\end{equation}
Therefore, right invariant vector fields are Killing vectors for $h$, meaning that 
\begin{equation}
  \mathcal{L}_{R_{\alpha}} (h) = 0.
\end{equation}

\begin{example}[Nil]
  The metric is defined as
  \begin{equation}
    h = \delta_{\alpha\beta} \sigma^{\alpha} \cdot \sigma^{\beta} = dx^2 + dy^2 + (dz - x dy)^2.
  \end{equation}
  How do we find the isometries?
  \begin{itemize}
    \item We can see that the metric components do not involve $z$ , so it is invariant under $z \to z + \omega$, which is generated by  $\frac{\partial }{\partial z} = R_3$.

    \item Similarly, we can see the same for  $y \to y + \epsilon$ , which is generated by $\frac{\partial}{\partial y} = R_2$.

    \item Finally, let us consider what happens for $x \to x + \epsilon$. As it stands, this is not an isometry. The parenthesis includes a term $\delta dy$, which we can get rid off by introducing another transformation $z \to z + \epsilon y$.
      This is generated by $\frac{\partial}{\partial x} + y \frac{\partial }{\partial z} = R_1$.
  \end{itemize}
  These agree with the right-invariant vector fields of Eq.~\eqref{eq:5-r}.
\end{example}

\subsection{Kaluza--Klein Interpretation}%
\label{sub:kaluza_klein_interpretation}

Consider the motion of a charged particle on the space of orbits of $R_3 = \frac{\partial }{\partial z}$.

\begin{leftbar}
  This was introduced as a way to combine the two known forces at the time: gravity and electromagnetism.
  This is done by introducing extra dimensions, by looking at gravity in dimension $d = 5$, which reduces to gravity and electromagnetism in dimension  $d=4$.
  The higher dimension is taken to be compactified into a circle of very large radius, so it is not detected by experiment. This idea is still stuck with us today in  \emph{String Theory}.
  \begin{example}[]
    For the Heisenberg group, the metric is independent of $z$, so we can take our manifold to be periodic in $z$.
  \end{example}
\end{leftbar}

To find the equations of motion, it is useful to write down the geodesic Lagrangian 
\begin{equation}
  \mathscr{L} = \frac{1}{2}( \dot{x}^2 + \dot{y}^2 + (\dot{z} - x \dot{y})^2),
\end{equation}
where the dot denotes $\bullet = \dv{s}$.
The Euler--Lagrange equations are
\begin{align}
  \dv{s} \frac{\partial \mathscr{L}}{\partial \dot{x}^{i}} &= \frac{\partial \mathscr{L}}{\partial x^{i}} \\
  \ddot x &= - \dot{y} (\dot{z} - x \dot{y}) \\
  \dv{s} \left[ \dot{y} - x (\dot{z} - x \dot{y}) \right] &= 0 \\
  \dv{s}(\dot{z} - x \dot{y}) &= 0.
\end{align}
Using the final equation to introduce a constant $c = \dot{z} - x \dot{y}$ , the first two equations reduce to
\begin{equation}
  \label{eq:5-star}
  \ddot x =  - c \dot y \qquad \& \qquad
  \ddot y = c \dot x.
\end{equation}

Compare this with geodesic motion in a magnetic field.
Let the spacetime be the Riemannian manifold $(M, g = g_{ij} dx^{i} \odot dx^{j})$  and the magnetic field be the closed 2-form $F = \frac{1}{2} F_{ij} dx^{i} \wedge dx^{j}$ .
The components of the Levi--Civita connection associated to $g$ are
\begin{equation}
  \Gamma^{i}_{jk} = \frac{1}{2} g^{il} \left( \frac{\partial g_{jl}}{\partial x^{k}} + \frac{\partial g_{kl}}{\partial x^{j}} - \frac{\partial g_{jk}}{\partial x^{l}} \right).
\end{equation}
The geodesic equation of motion is then given by
\begin{equation}
  \ddot x^{i} + \Gamma^{i}_{jk} \dot{x}^{j} \dot{x}^{k} = c F\indices{^{i}_{j}} \dot{x}^{j},
\end{equation}
where we see $F\indices{^{i}_{j}}$  as an endomorphism rather than a 2-form.
This is the general form of geodesic motion in a magnetic field.
We want to compare this to \eqref{eq:5-star}, so we take $M = \mathbb{R}^2$ and $g_{ij} = \delta_{ij}$ , which gives $\Gamma_{ij}^{k} = 0$ .
Moreover, we take $F = -dx \wedge dy$ , meaning that $F_{ij} = - \epsilon_{ij}$  is the volume-form.
So geodesics of the left-invariant metric $h$ on $G = \text{Nil}$  projects to the trajectories of a charged particle in a constant magnetic field.
We think of this as in Fig.~\ref{fig:l5f1}.
\begin{figure}[tbhp]
  \centering
  \def\svgwidth{0.6\columnwidth}
  \input{lectures/l5f1.pdf_tex}
  \caption{Kaluza--Klein reduction.}
  \label{fig:l5f1}
\end{figure}

For the general case of Kaluza--Klein reduction, we have
\begin{equation}
  h = (dz + A)^2 + g_{ij} dx^{i} \odot dx^{j}.
\end{equation}
Take coordinates $(z, x^{i})$, where $z$ is the extra dimension.
Moreover, $A$ is a one-form on $M$ and we define $F = dA$.
The geodesic equation is
\begin{equation}
  \dot{z} + A_{i} \dot{x}^{i} = c \qquad  (\text{conserved charge})
\end{equation}
