% lecture notes by Umut Özer
% course: dgp
\lhead{Lecture 14: March 09}

\chapter{Fibre Bundles and Connections}%
\label{cha:fibre_bundles_and_connections}

\begin{leftbar}
  Exams: answer two out of three questions. He lectured from 2012-2016.
  Exam revision class in April.
\end{leftbar}

We will introduce some rather formal mathematics to show that we can also think of Yang--Mills theory as a theory of connections on principal bundles with gauge group $G$ as the fibre.

\section{Fibre Bundles}%
\label{sec:fibre_bundles}

Without getting into the definitions just yet, let us give a bit more of an overview.
\emph{Fibre bundles} are manifolds that locally look like a product of manifolds.
Examples include the cylinder $S^1 \times \mathbb{R}$ and the Möbius band, which are illustrated in Fig.~\ref{fig:fibre-bundles}.
\begin{figure}[tbhp]
  \centering
  \inkfig[0.5]{fibre-bundles}
  \caption{Fibre bundles over $B = S^1$ with a fibre $\mathbb{R}$.}
  \label{fig:fibre-bundles}
\end{figure}
These are locally equivalent, but globally different, since the Möbius band has a half twist while the cylinder is the trivial bundle. The Möbius band is non-trivial. 

\begin{definition}[fibre bundle]
  A \emph{fibre bundle} $\{E, \pi, B, F, G\}$ is a structure consisting of:
  \begin{itemize}
    \item a manifold $E$, called the \emph{total space} of the bundle,
    \item a smooth map $\pi \colon E \to B$, called the \emph{projection}, where
    \item $B$, called the \emph{base space} is also a manifold
    \item $\forall x \in B$, $\exists U_{\alpha} \subset B$ s.t.~$x \in U_{\alpha}$ and $\exists$ diffeomorphism 
      \begin{equation}
	\phi_{\alpha} \colon U_{\alpha} \times F \to \pi^{-1}(U_{\alpha}), \qquad \text{s.t.} \pi(\phi_{\alpha}(x, f)) = x,
      \end{equation}
      where $(x, f) \subset U_{\alpha} \times F$.
      The manifold (!) $F$ is called a \emph{fibre} and $\phi_{\alpha}$ are called \emph{trivialisations}.
    \item transition functions: for $x \in U_{\alpha} \cap U_{\beta}$,
      \begin{equation}
        \phi_{\alpha\beta} \coloneqq \phi^{-1}_{\alpha} \circ \phi_{\beta} \colon F \to F,
      \end{equation}
      which are elements $\phi_{\alpha\beta} \in G$, where the Lie group $G$ is called the \emph{structure group} of $E$.
      \begin{enumerate}[1)]
	\item $\phi_{\alpha\alpha} = \text{Id}$ (no summation over $\alpha$)
	\item cocycle relation: for $x \in U_{\alpha} \cap U_{\beta} \cap U_{\gamma}$,
	  \begin{equation}
	    \phi_{\alpha\beta} \phi_{\beta\gamma} = \phi_{\alpha\gamma}.
	  \end{equation}
      \end{enumerate}
  \end{itemize}
\end{definition}

For the example above, the total spaces $E$ are the cylinder and the Möbius band, the base space $B$ is a circle.
The fibre $F$ in both cases is a real line $\mathbb{R}$.

\begin{definition}[principal bundle]
  A \emph{principal bundle} has $F = G$, and the Lie group $G$ acts on $F$ by left-translations.
\end{definition}
\begin{definition}[vector bundle]
  A \emph{vector bundle} has $F = \mathbb{R}^n$ (real vector bundle) or $F = \mathbb{C}^n$ (complex vector bundle) and $G = GL(n, \mathbb{R})$ or $GL(n, \mathbb{C})$.
\end{definition}
\begin{definition}[trivial bundle]
  A \emph{trivial bundle} has $E = B \times F$.
\end{definition}

\begin{example}[tangent bundle]
  This is an example of a bundle that we have already met. The \emph{tangent bundle}
  \begin{align}
    &\text{total space:} & E &= TB = \bigcup_{x} T_x B, \\
    &\text{fibre:} & F_x &= \pi^{-1}(x) = \mathbb{R}^n, \\
    & & E &= \{x, \vb{v}\} \qquad (\text{position}, \text{velocity}), \\
    &\text{group:} & G &= GL(n, \mathbb{R}).
  \end{align}
\end{example}
\begin{example}[magnetic monopole bundle]
  The \emph{magnetic monopole bundle} is an example of a principal bundle:
  \begin{align}
    &\text{base manifold:} & B &= S^2 = U_+ \cap U_-, \\
    &\text{fibre:} & F &= U(1), \\
  \end{align}
  with coordinate $e^{i \psi}$. Local coordinates $(\theta, \phi)$ on $S^2$
  \begin{equation}
    0 \leq \theta \leq \pi, \qquad 0 \leq \phi \leq 2\pi.
  \end{equation}
  \begin{figure}[tbhp]
    \centering
    \inkfig[0.3]{l14f2}
    \caption{}
    \label{fig:l14f2}
  \end{figure}
  The base manifold is covered by the open sets $U_+, U_-$ shown in Fig.~\ref{fig:l14f2}.
  \begin{align}
    U_+ \times U(1):& \qquad (\theta, \phi, e^{i \psi_+}) \\
    U_- \times U(1):& \qquad (\theta, \phi, e^{i \psi_-}),
  \end{align}
  on $U_+ \cap U_-$, 
  \begin{equation}
    e^{i \psi_-} = e^{i n \phi} e^{i \psi_+}.
  \end{equation}
  Then $n$ must be an integer for $E$ to be a manifold.
  In fact, the \emph{monopole number} $n$ classifies $U(1)$ principal bundles over $S^2$.
  \begin{itemize}
    \item $n = 0$, $E = S^2 \times S^1$ (trivial bundle)
    \item $n = 1$, $E = S^3 \times S^1$ (Hopf fibration)
  \end{itemize}
\end{example}

\section{Hopf Fibration}%
\label{sec:hopf_fibration}

There are two descriptions of the Hopf fibration
\begin{enumerate}[1.]
  \item Let us take $E = SU(2)$, then
    \begin{equation}
      B = \frac{SU(2)}{U(1)} \simeq S^2,
    \end{equation}
    where $U(1) \subset SU(2)$ diagonal.
    Let us think of $S^3$ as
    \begin{equation}
      S^3 = \left\{ (z_1, z_2) \in \mathbb{C}^2 \suchthat \abs{z_1}^2 + \abs{z_2}^2 = 1 \right\}.
    \end{equation}
    We take the right action of $S^1$ on $SU(2)$ to be
    \begin{equation}
      (z_1, z_2) \to (z_1 e^{i \alpha}, z_2 e^{i \alpha}).
    \end{equation}
    Since the sum of the norms is still the same, this maps from $S^3 \to S^3$.
    This fixes the ratio
    \begin{equation}
      \frac{z_1}{z_2} \in \mathbb{C P}^1 \simeq S^2.
    \end{equation}
  \item Any complex line
    \begin{equation}
      A_1 z_2 + A_2 z_2 = 0
    \end{equation}
    through $0 \in \mathbb{C}^2$, which can be seen as a two-dimensional plane in $\mathbb{R}^4$, intersects $S^3 \subset \mathbb{C}^2 \simeq \mathbb{R}^4$ in a circle $S^1$, which we take to be the fibre over a point $[z_2, z_2] \in \mathbb{C P}^1 \simeq S^2$.
\end{enumerate}

\section{Section}%
\label{sec:section}

\begin{definition}[section]
  A \emph{section} $s$ of a bundle $\pi \colon E \to B$ is a map $s \colon B \to E$ such that $\pi \circ s = \text{Id}$.
\end{definition}
\begin{figure}[tbhp]
  \centering
  \inkfig[0.5]{l14f3}
  \caption{A section $s$ of a bundle.}
  \label{fig:l14f3}
\end{figure}

\begin{claim}
  A principal bundle is trivial iff it admits a global section.
\end{claim}
\begin{proof}
  Suppose $s = s(x)$ is this section.
  Then $\forall p \in G_x \simeq \pi^{-1}(x)$ may be represented as $p = s(x) R_{g(p)}$, where $R_{g(p)}$ is the right action of $G$ on $G_x$ moving $S(x)$ to $p$.
  \begin{equation}
    E = \left\{x = \pi(p), g_8p\right\} \in B \times G.
  \end{equation}
\end{proof}
