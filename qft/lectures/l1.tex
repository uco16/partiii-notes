\lhead{Lecture 1: October 10}

\section*{Notation}%
In the following chapters, we will make extensive use of Fourier theory. To avoid having to drag around factors of $(2\pi)^d$ all the time, which make the equations longer and obtuser than they need to be, we adopt the following notational convention.

\begin{notation}[barred differential]
  We understand the \emph{barred differential} $\bdd{q}$ to be the following normalisation of the ordinary differential measure
  \begin{equation}
    \bdd{q} a = \frac{\dd{q} a}{2\pi}.
  \end{equation}
  In particular, given a $d$-dimensional vector $\vb{q} = (q_1, \ldots, q_d)$---most often a Fourier mode---we will frequently use the $d$-dimensional integration measure in Fourier space:
  \begin{equation}
    \bdd[d]{q} a = \frac{\dd[d]{q} a}{(2\pi)^d} = \prod_{i=1}^d \frac{\dd{q_i}}{2\pi}.
  \end{equation}
\end{notation}

\begin{notation}[barred delta]
  We will denote the \emph{barred Dirac delta} $\bdelta(q)$ to include a normalisation of $2\pi$ in the following way
  \begin{equation}
    \bdelta(q) = 2\pi \delta(q).
  \end{equation}
  In particular, given a $d$-dimensional vector $\vb{q} = (q_1, \ldots, q_d)$, we will frequently use the $d$-dimensional distribution
  \begin{equation}
    \bdelta(\vb{q}) = (2\pi)^d \delta(\vb{q}) = \prod_{i=1}^d 2\pi \delta(q_i).
  \end{equation}
\end{notation}

\newpage

\chapter{Introduction}%
\label{cha:lecture_1}

\begin{description}
  \item[Reference text book:] Introduction to Quantum Field Theory - Peskin and Schröder
\end{description}

QFT is a quantum theory with an infinite number of degrees of freedom at each point in spacetime $x^{\mu} = (t, \vb x)$. They are constructed from classical field theories (eg.~EM field) incorporating special relativity.
QFT states are multi-particle states. The theory becomes relevant at relativistic energies, where we have a lot of exchange between energy $\leftrightarrow$ mass. This takes the form of particles being destroyed / created in interactions.
Particle creation and annihilation occurs in Quantum field theory (QFT) but not in QM, differentiating the two theories. Unlike in QM, particle number is not fixed.
The interactions themselves arise from the mathematical structure of the theory. Various principles determine this structure:
\begin{itemize}
  \item locality
  \item symmetry
  \item renormalisation group flow
\end{itemize}

\begin{definition}
A \emph{Free QFT} is the limit which has particles but no interactions. Weakly interacting theories are then built from these by using perturbation theory.
Strong coupling phenomena are an active area of research, hard to solve and relevant for nature, often approached by a discretisation of spacetime (lattice QCD).
Free QFT is a relativistic theory with infinitely many \emph{quantised harmonic oscillators}.
Non-relativistic QFTs also exist, describing quasi-particles like phonons.
\end{definition}

\begin{itemize}
  \item Based on D.~Tong's notes (videos at PIRSA useful as well)
\end{itemize}

\section*{Units in QFT}%
\label{sec:units_in_qft}

As a mathematician, you would use $\pi = i = -1 = 2$ in these lectures. However, since we care about having correct numbers for our experimental measurements, we use special units in which $\hbar = c = 1$.
\begin{equation}
  [c] = L T^{-1}, \qquad [\hbar] = L^2 M T^{-1} \implies L = T = M^{-1}.
\end{equation}
Length and time are measured in inverse mass---or equivalently ($E = m\cancel{c^2}$) inverse energy units.
To convert back to metres or seconds, insert relevant powers of $c$ and $\hbar$.
E.g. $\lambda = \frac{\hbar}{m c}$, so when we measure $m_e \sim \SI{1e6}{\electronvolt}$, we have $\lambda_e = \SI{2e-12}{\meter}$. Similarly $m_p \sim \SI{1}{\giga\electronvolt} \rightarrow \SI{1e9}{\electronvolt}$ .

In QFT, if a quantity $x$ has mass dimension $(\text{mass})^d$, we write $[x] = d$. E.g.~$[G_N] = -2$ since 
\begin{equation}
  G_N = \frac{\cancel{\hbar c}}{M_p^2} = \frac{1}{M_p^2}
\end{equation}
where $M_p \sim \SI{1e19}{\giga\electronvolt}$ corresponds to the \emph{Planck scale}, where quantum gravity effects became important ($\lambda_p \sim 10^{-35}$m).

Angular momentum, like $\hbar$, is dimensionless in these units: $e^-$ total spin: $\hbar/2 = 1/2$.

Relativistic Schrödinger equation does not work---something always goes wrong (causality violation / energy unbounded from below). In QFT, these faults are fixed by allowing for the creation and annihilation of particles.

\chapter{Classical Field Theory}%
\label{cha:classical_field_theory}

A field is defined at each point of space and time ($t, \vb x$). Classical particle mechanics yields a finite number of generalised coordinates $q_a(t)$, where $a = x, y, \dots$ is a label.

In field theory, we have a field $\phi_a(\vb x, t)$, where we can consider $\vb x$ as a label as well.
So $a$ and $\vb x$ are labels: we have an infinite number of degrees of freedom: one for each $\vb x$.
Position is relegated from a dynamical variable to a mere label.


\begin{example}[Electromagnetism]
\begin{equation}
  E_i (\vb x, t), B_i(\vb x, t) \qquad \{i, j, k\} \in \left\{ 1, 2, 3 \right\} \text{ label spatial position}
\end{equation}

These 6 fields are derived from 4 fields $A_\mu(\vb x, t) = (\phi, \vb A)$, where $\mu \in \left\{ 0, 1, 2, 3 \right\}$.
The relationship between the electric and magnetic fields and $A$ is
\begin{equation}
  E_i = -\pdv{A_i}{t} - \pdv{A_0}{x_i}, \qquad B_i = \frac{1}{2} \varepsilon_{ijk} \pdv{A_k}{x_j},
\end{equation}
where the Einstein summation convention is used.
\end{example}

\section{Dynamics of Fields}%
\label{sec:dynamics}

The dynamics of the field is governed by a Lagrangian $L$. This is a function of the fields $\phi_a(\vb x, t)$, their time derivatives $\dot \phi_a(\vb x, t)$ and spatial derivatives $\grad \phi_a(\vb x, t)$.
We can write it as an integral over the Lagrangian density $\mathcal{L}$:
\begin{equation}
  L = \int \dd[3]{x} \mathcal{L}(\phi_a, \partial_\mu \phi_a).
\end{equation}
Confusingly, $\mathcal{L}$ is also often just called the \emph{Lagrangian} in QFT. We will mostly be concerned with Lagrangian densities.

We define the action as 
\begin{equation}
  S = \int_{t_0}^{t_1} L \dd{t} = \sum_{a} \int \dd[4]{x} \mathcal{L}(\phi_a, \partial_\mu \phi_a).
\end{equation}

Since $[\mathcal{L}] = 4$ and $[\dd[4]{x}] = -4$, the action is dimensionless $[S] = 0$.
The \emph{Dynamical Principle} of classical field theory is that fields evolve in such a way that the action $S$ is stationary with respect to those variations of the fields that don't affect the initial and final values.

\begin{align}
  \label{eq:variation in S}
  \delta S &= \sum_a \int \dd[4]{x} \left\{ \pdv{\mathcal{L}}{\phi_a} \delta \phi_a + \pdv{\mathcal{L}}{(\partial_\mu \phi_a)} \delta(\partial_\mu \phi_a) \right\} \\
  &= \sum_a \int \dd[4]{x} \left\{ \pdv{\mathcal{L}}{\phi_a} \delta \phi_a + \partial_\mu \left( \pdv{\mathcal{L}}{(\partial_\mu \phi_a)} \delta \phi_a \right) - \partial_\mu \pdv{\mathcal{L}}{(\partial_\mu \phi_a)} \delta \phi_a \right\}
\end{align}
The total derivative vanishes for any term that decays at spatial $\infty$ and has $\delta \phi(\vb x, t_0) = 0 = \delta \phi(\vb x, t_1)$. So the dynamical principle is
\begin{equation}
  \label{eq:EL}
  \delta S = 0 \implies \boxed{\partial_\mu \left( \pdv{\mathcal{L}}{(\partial_\mu \phi_a)} - \pdv{\mathcal{L}}{\phi_a} = 0 \right)} \qquad \text{Euler-Lagrange equations}.
\end{equation}

\begin{example}[Klein-Gordon Field $\phi$]

  \begin{equation}
    \mathcal{L} = \frac{1}{2} \eta^{\mu \nu} \partial_\mu \phi \partial_\nu \phi - \frac{1}{2} m^2 \phi^2
  \end{equation}
  where we use the mostly minus signature $\eta^{\mu \nu} = \text{diag}(1, -1, -1, -1)$.

  \begin{equation}
    \mathcal{L} = \frac{1}{2} \dot\phi^2 - \frac{1}{2} (\grad \phi)^2 - \frac{1}{2} m^2 \phi^2
  \end{equation}

  \begin{leftbar}
    \begin{remark}
      $\mathcal{L} = T- V$ where $T = \frac{1}{2} \dot \phi^2$ is the kinetic and $V = \frac{1}{2} (\grad \phi)^2 + \frac{1}{2}m^2 \phi^2$ is the potential energy density.
    \end{remark}
  \end{leftbar}

  The relevant derivatives of $\mathcal{L}$ appearing the the Euler-Lagrange equations are
  \begin{equation}
    \pdv{\mathcal{L}}{\phi} = -m^2\phi, \qquad \pdv{\mathcal{L}}{(\partial_\mu \phi)} = \partial^\mu \phi,
  \end{equation}
  So substituting $\mathcal{L}$ into \eqref{eq:EL} yield
  \begin{equation}
    \boxed{\partial_\mu \partial^\mu \phi + m^2 \phi = 0} \qquad \text{Klein-Gordon eqn}
  \end{equation}
  The K-G eqn admits a wave-like solution $\phi = e^{-i p \vdot x}$.

  \begin{leftbar}
    \begin{remark}
      $p \vdot x = p^0 x^0 - \vb p \vdot \vb x^T = E t - \vb p \vdot \vb x$
    \end{remark}
  \end{leftbar}

  Substitute $\phi \rightarrow$ KG:
  \begin{equation}
    (-p^2 + m^2)\phi = 0.
  \end{equation}
  Since $p^2 = E^2 - \abs{\vb{p}}^2 = m^2$, this reproduces the relativistic energy dispersion relation for a particle of mass $m$ and three-momentum $\vb{p}$.
\end{example}



