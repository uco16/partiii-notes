% lecture notes by Umut Özer
% course: qft
\lhead{Lecture 8: October 26}
If $H_I$ were just a function, we could solve \eqref{eq:interaction-operators} by $U = \exp[-i \int_{t_0}^{t} H_I(t') \dd[]{t'}]$, but this does not work.
There are ordering ambiguities since $[H_I(t'), H_I(t'')] \neq 0$ for $t' \neq t''$.
From \eqref{eq:interaction-operators}, we see that $U(t, t_0)$ satisfies
\begin{equation}
  U(t, t_0) = 1 + (-i) \int_{t_0}^{t} \dd[]{t'} H_I(t') U(t', t_0).
\end{equation}
Substituting this back into itself to get an infinite series
\begin{equation}
  U(t, t_0) = 1 + (-i) \int_{t_0}^{t} \dd[]{t'} H_I(t') + (-i)^2 \int_{t_0}^{t} \dd[]{t'} \int_{t_0}^{t} \dd[]{t''} H_I(t') H_I(t'') + \dots
\end{equation}
From the ranges of integration, the $H_I$ product is automatically time ordered.

%F1 graph axes t' on x and y=t''. Then there is rectangle at x=t and y=t with diagonal line separating it into two triangles. Then both areas (upper and lower) are marked with T{} identities
\begin{wrapfigure}{R}{0.4\textwidth}
  \centering
  \def\svgwidth{0.3\columnwidth}
  \input{lectures/l8f1.pdf_tex}
  \caption{}
  \label{fig:l8f1}
\end{wrapfigure}
If we integrate over the upper patch in Figure \ref{fig:l8f1}, time ordering is $T \left\{ H_I(t') H_I(t'') \right\} = H_I(t'') H_I(t')$, whereas in the lower patch we have $T \left\{ H_I(t') H_I(t'') \right\} = H_I(t') H_I(t'')$.
The upper triangle is the initial volume of the integral. But by relabelling, we can show that this is the same as the lower triangle:
\begin{equation}
  \int_{t_0}^{t} \dd[]{t'} \int_{t_0}^{t} \dd[]{t''} H_I(t') H_I(t'')
  = \int_{t_0}^{t} \dd[]{t''} \int_{t_0}^{t} \dd[]{t'} H_I(t'') H_I(t').
\end{equation}
So $\frac{1}{2!}$ times the area of the full square is the area of the upper triangle
\begin{equation}
  U(t) = 1 + (-i) \int_{t_0}^{t} \dd[]{t'} H_I(t')
  + \frac{(-i)^2}{2!} \int_{t_0}^{t} \dd[]{t'} \int_{t_0}^{t} \dd[]{t''} T \left\{ H_I(t') H_I(t'') \right\}
\end{equation}
Therefore, we obtain \emph{Dyson's formula} by writing
\begin{equation}
  \label{eq:8-dyson}
  U(t, t_0) = T \exp{-i \int_{t_0}^{t} \dd[]{t'} H_I(t')} = T \exp{+i \int_{t_0}^{t} \dd[3]{x} \dd{t} \mathcal{L}_I(t')}
\end{equation}
\begin{example}
  In scalar Yukawa theory, this is $T \exp[-i g \int_{}^{} \dd[4]{x} \psi^* \psi \phi]$.
\end{example}
This is something of a formal result---we expand to finite order in $\lambda_n$.

\section{Scattering Amplitudes}%
\label{sec:scattering_amplitudes}

The time evolution used in scattering theory is called the \emph{S-matrix}.
Let us define the time evolution operator going from initial to final states as
\begin{equation}
  S = \lim_{\substack{t \to \infty,\\ t_0 \to - \infty}} \left\{ U(t, t_0) \right\}
\end{equation}
The initial state $\ket{i}$ and final state $\ket{f}$ are in some sense `far away' from each other and from the interaction.
We assume that $\ket{f}, \ket{i}$ behave like free particles; they are eigenstates of $H_0$.

The amplitude is 
\begin{equation}
  \lim_{\substack{t \to \infty \\ t_0 \to - \infty}} \left\{ \bra{f} U(t, t_0) \ket{ i} \right\} = \bra{f} S \ket{i},
\end{equation}
where $\bra{ f}S \ket{ i}$ is historically called the \emph{S-matrix} or \emph{matrix element}.

\begin{example}[]
  Let us go back to scalar Yukawa theory, 
  \begin{equation}
    H_I = g \psi^*_I \psi_I \phi_I.
  \end{equation}
  Physically, we will model mesons with $\phi$ and nucleons as $\psi$.
  Moreover, we will drop the subscript $I$ from now on for notational ease.
  Concentrate on creation and annihilation operators in the expansion
  The operator $\phi \sim a_{\vb{p}}, a^{\dagger}_{\vb{p}}$ destroys or creates a meson.
  For the other two it is slightly more complicated:
  $\psi \sim b_{\vb{p}}, c ^{\dagger}_{\vb{p}}$ destroys a nucleon or creates an anti-nucleon, while 
  $\psi^* \sim b_{\vb{p}}^{\dagger}, c_{\vb{p}}$ creates a nucleon or destroys an anti-nucleon.
  The only non-zero commutators are
  \begin{equation}
    [a_{\vb{p}}, a_{\vb{p}'}^{\dagger}] = [b_{\vb{p}}, b_{\vb{p}'}^{\dagger}] = [c_{\vb{p}}, c_{\vb{p}'}^{\dagger}] = \bdelta^3(\vb{p} - \vb{p}').
  \end{equation}
  The interaction Hamiltonian $H_I$ contains terms like $\int \dots b_{\vb{p}}^{\dagger} c_{\vb{p}'}^{\dagger} a_{\vb{p}''} + \dots$, which destroys a meson and produces a nucleon/anti-nucleon pair. This contributes to meson decay: $\phi \to \psi \bar \psi$.
  Figure \ref{fig:meson-decay} represents this in a diagrammatic form.
  %F2 phi with solid arrow going in and psi and psi bar with solid arrows going out
  \begin{leftbar}
    \begin{remark}
      We wrote $\bar \psi$ to denote the anti-particle, not the field. In general, particles are labelled like their fields and anti-particles are labelled like the fields with a bar on top.
    \end{remark}
  \end{leftbar}
  This interaction arose from the first term in $U(t, t_0)$.
  At second order in $g$, we have more complicated terms like $\int \dots (c b a^{\dagger}) (b^{\dagger} c ^{\dagger} a) + \dots$. 
  \begin{leftbar}
    \begin{remark}
      The momenta $\vb{p}, \vb{p}', \dots$ are implied.
    \end{remark}
  \end{leftbar}
  %F3 psi and psi bar going in, mediated by phi, psi and psi bar going out, all solid arrows
  This term, corresponding to the diagram depicted in Figure \ref{fig:nucleon-scattering}, contributes to nucleon / anti-nucleon scattering.
  \begin{figure}[t]
    \begin{subfigure}[t]{0.5\textwidth}
      \centering
      \feynmandiagram [horizontal=a to b] {
        {i1 [particle=\(\psi\)], i2 [particle=\(\bar\psi\)]} -- [fermion, edge label=\(\)] a,
        a -- [fermion, edge label=\(\phi\)] b,
        b -- [fermion, edge label=\(\)] { f1 [particle=\(\psi\)] , f2 [particle=\(\hat\psi\)]},
      };
      \caption{Scattering between a nucleon $\psi$ and anti-nucleon $\bar\psi$ via the meson $\phi$.}
      \label{fig:nucleon-scattering}
    \end{subfigure}
    ~
    \begin{subfigure}[t]{0.5\textwidth}
      \centering
      \feynmandiagram [horizontal=a to b] {
	a [particle=\(\phi\)] -- [fermion, edge label=\(\),  momentum=\(p\)] b,
	b -- [fermion, edge label=\(\), momentum=\(q_1\)] o1 [particle=\(\psi\)],
	b -- [fermion, edge label=\(\), momentum=\(q_2\)] o2 [particle=\(\bar \psi\)],
      };
      \caption{A meson $\phi$ with momentum decays into a nucleon and anti-nucleon pair.}
      \label{fig:meson-decay}
    \end{subfigure}
    \caption{Diagrammatic representations of interaction terms.}
  \end{figure}

\end{example} 

\subsection{Meson Decay}%
\label{sub:meson_decay}

Let us compute the amplitude for the diagram in Figure \ref{fig:meson-decay}.
We take the initial and final states to be
\begin{equation}
  \ket{i} = \sqrt{2 E_p} a_{\vb{p}}^{\dagger} \ket{0} \qquad \ket{f} = \sqrt{4 E_{q_1} E_{q_2}} b_{\vb{q}_1}^{\dagger} c_{\vb{q}_2} ^{\dagger} \ket{0}.
\end{equation}
We need to calculate the matrix element $\bra{ f} S \ket{i}$.
As previously mentioned, we do this by expanding in the coupling constant $g$.
The zeroth order term vanishes, since ${\bra{f}\ket{i} = \bra{0} b c a^{\dagger} \ket{0} = 0}$.
The first order term is
\begin{equation}
  -ig \bra{f}T \int_{}^{} \dd[4]{x} \psi^*(x) \psi(x) \phi(x) \ket{i} + O(g^2).
\end{equation}
Note that the time ordering operator $T$ has no effect here, whereas in the higher order terms it will matter.
Our general strategy is to expand all field and states, getting rid of all creation and annihilation operators with the commutation relations and the vacuum normalisation $\bra{0} \ket{0} = 1$.
We will thus reduce this QFT amplitude to a function of 4-momenta.

Let us expand the meson field $\phi$ and $\ket{i}$ first.
\begin{equation}
  \bra{f}S \ket{i} = -ig \bra{f} \int_{}^{} \dd[4]{x} \psi^*(x) \psi(x) \int_{}^{} \bdd[3]{k} \frac{\sqrt{2E_p}}{\sqrt{2E_k}} \left( a_{\vb{k}} a_{\vb{p}}^{\dagger} e^{-i k \cdot x} + a^{\dagger}_{\vb{k}} a^{\dagger}_{\vb{p}} e^{i k \cdot x} \right) \ket{0} + O(g^2)
\end{equation}
We can then commute the $a_{\vb{k}}^{\dagger}$ in the second summand directly into the vacuum on the left hand side, using that $[a_{\vb{k}}^{\dagger}, \psi(x)] = 0 = [a_{\vb{k}}^{\dagger}, \psi^*(x)]$. 
This annihilates the vacuum on the left hand side since
\begin{equation}
  a_{\vb{k}} \ket{0} = 0 \implies \bra{0} a_{\vb{k}}^{\dagger} = 0.
\end{equation}
In the first term, we can replace $[a_{\vb{k}}, a_{\vb{p}}^{\dagger}] \ket{0} = \bdelta^3(\vb{p}-\vb{k}) \ket{0}$, so to first order in $g$, 
\begin{equation}
  \bra{f}S \ket{i} = -ig \bra{f} \int_{}^{} \dd[4]{x}  \psi^*(x) \psi(x) e^{-ip \cdot x} \ket{0}.
\end{equation}
We now expand the $\psi$ fields and the $\bra{f}$.
\begin{align}
  \begin{split}
\bra{f}S \ket{i} &= -ig \bra{0} \int_{}^{} \dd[4]{x} \frac{\dd[3]{k_1} \dd[3]{k_2}}{\sqrt{4 E_{k_1} E_{k_2}}} \Bigr\{ \sqrt{4 E_{q_1} E_{q_2}} c_{\vb{q}_2} b_{\vb{q}_1} \left( b^{\dagger}_{\vb{k}_1} e^{i k_1 \cdot x} + \cancel{c_{\vb{k}_1}} e^{-i k_1 \cdot x} \right) \\
		   &\qquad\times\left( \cancel{b_{\vb{k}_1}}(...) + c_{\vb{k}_2}^{\dagger} e^{i k_2 \cdot x} \right) e^{-i p \cdot x} \ket{0} \Bigr\}
  \end{split}
  \\
		   &= -i g \bra{0} \int_{}^{} \dd[4]{x} \frac{\dd[3]{k_2}}{\sqrt{2 E_{k_2}}} \sqrt{2 E_{q_2}} [c_{\vb{q}_2}, c ^{\dagger}_{\vb{k}_2} ] e^{i (q_1 \cdot x + k_2 \cdot x - p \cdot x)} \ket{0} \\
		   &=- ig \bra{0} \int_{}^{} \dd[4]{x} e^{i(q_1 + q_2 -p) \cdot x} \ket{0} \\
		   &=-i g \bdelta^4(q_1 + q_2 -p).
\end{align}
This four dimensional delta function imposes 4-momentum conservation on initial and final states.
