% lecture notes by Umut Özer
% course: qft
\lhead{Lecture 14: November 12}

\section{The Spinor Representation}%
\label{sec:the_spinor_representation}

We will look for other matrices other than $M$ that satisfy the Lorentz algebra \eqref{eq:13-star}.

\begin{definition}[]
  The anti-commutator of two matrices $X, Y \in GL(n)$ is
  \begin{equation}
    \left\{ X, Y \right\} = X Y + Y X.
  \end{equation}
\end{definition}
\begin{definition}[]
  The \emph{Clifford algebra} is defined, in any number of dimensions, by the relation
  \begin{equation}
    \boxed{ \left\{ \gamma^{\mu}, \gamma^{\nu} \right\} = 2 \eta^{\mu\nu} \mathbb{1} },
  \end{equation}
  where $\gamma^{\mu}$ are a set of $4$ matrices, $\mu \in \left\{ 0, 1,2,3 \right\}$.
  We have $\gamma^{\mu} \gamma^{\nu} = - \gamma^{\nu} \gamma^{\mu}$ for $\mu \neq \nu$ and $(\gamma^0)^2 = \mathbb{1}$ and $(\gamma^i)^2 = -\mathbb{1}$, for $i \in \left\{ 1, 2,3 \right\}$.
\end{definition}
The simplest representation of the Clifford algebra is in terms of $4 \times 4$ matrices, for example the \emph{chiral representation}:
\begin{equation}
  \gamma^0 = 
  \begin{pmatrix}
   0 & \mathbb{1}_2 \\
   \mathbb{1}_2 & 0 \\
  \end{pmatrix}, 
  \qquad \gamma^i = 
  \begin{pmatrix}
   0 & \sigma^i \\
   -\sigma^i & 0 \\
  \end{pmatrix},
\end{equation}
where $\sigma^i$ are the \emph{Pauli matrices}:
\begin{equation}
  \sigma^i = 
  \begin{pmatrix}
   0 & 1 \\
   1 & 0 \\
  \end{pmatrix}, \qquad
  \sigma^2 =
  \begin{pmatrix}
   0 & -i \\
   i & 0 \\
  \end{pmatrix},
  \qquad \sigma^3 = 
  \begin{pmatrix}
   1 & 0 \\
   0 & -1 \\
  \end{pmatrix}.
\end{equation}
These satisfy $\left\{ \sigma^i, \sigma^j\right\} = 2 \delta^{ij} \mathbb{1}_2$ and $\left[ \sigma^j, \sigma^k \right] = 2 i \epsilon^{jkl} \sigma^l$, for $i,j,k,l \in \left\{ 1, 2,3 \right\}$.
Note that these are not the only matrices which work as a representation. In fact, any similarity transformation $U \gamma^{\mu} U^{-1}$, with constant invertible matrix $U$, will give a valid representation.

\begin{definition}[]
  \begin{align}
    S^{\rho\sigma} \coloneqq \frac{1}{4} [\gamma^{\rho}, \gamma^{\sigma}] &=
    \begin{cases}
      0, & \rho = \sigma \\
      \frac{1}{2} \gamma^{\rho} \gamma^{\sigma}, & \rho \neq \sigma,
    \end{cases} \\
									  &=\frac{1}{2} \gamma^{\rho} \gamma^{\sigma} - \frac{1}{2} \eta^{\rho\sigma} \mathbb{1} \label{eq:14-2star}
  \end{align}
\end{definition}

\begin{claim}
  \label{cl:14-1}
  \begin{equation}
    [S^{\mu\nu}, \gamma^{\rho}] = \gamma^{\mu}\eta^{\nu\rho} - \gamma^{\nu} \eta^{\rho\mu}
  \end{equation}
\end{claim}
\begin{claim}
  $S$ provides a representation of the Lorentz group, meaning that
  \begin{equation}
    [S^{\rho\sigma}, S^{\tau\nu}] = \eta^{\sigma\tau} S^{\rho\nu} - \eta^{\rho\tau} S^{\sigma\nu} + \eta^{\rho\nu} S^{\sigma\tau}- \eta^{\sigma\nu} S^{\rho\tau}.
  \end{equation}
\end{claim}
\begin{proof}
  Use Claim \ref{cl:14-1} and \eqref{eq:14-2star}.
\end{proof}

\begin{definition}[]
  The \emph{Dirac spinor} $\psi_{\alpha}(x)$, $\alpha \in \left\{ 1, 2, 3, 4 \right\}$ is defined by
  \begin{equation}
    \psi_{\alpha}(x) \xrightarrow{L.T.} S[\Lambda] \indices{^{\alpha}_{\beta}} \psi^{\beta}(\Lambda^{-1} x),
  \end{equation}
  where $\Lambda = \exp(\frac{1}{2} \Omega_{\rho\sigma} M^{\rho\sigma})$ and $S[\Lambda] = \exp(\frac{1}{2} \Omega^{\rho\sigma} S^{\rho\sigma})$.
\end{definition}

\begin{claim}
  The spinor representation is \emph{not} equivalent to the usual vector representation.
\end{claim}
\begin{proof}
  We can see this by looking at a specific LT, such as \emph{rotations}:
  \begin{equation}
    S^{ij} = \frac{1}{4} \left[	
      \begin{pmatrix}
       0 & \sigma^i \\
       -\sigma^i & 0 \\
      \end{pmatrix},
      \begin{pmatrix}
       0 & \sigma^j \\
       -\sigma^j & 0 \\
      \end{pmatrix}
    \right]
    = -\frac{i \epsilon^{ijk}}{2} 
    \begin{pmatrix}
     \sigma^k & 0 \\
     0 & \sigma^k \\
    \end{pmatrix},
  \end{equation}
  where we made use of the algebra of $\sigma^i$. Interpreting $\phi$ physically as the rotation angle, we then write $\Omega_{ij} \coloneqq -\epsilon_{ijk} \phi^k$, (eg. $\Omega_{12} = -\phi^3$). We then have
  \begin{equation}
    S[\Lambda] = \exp(\frac{1}{2} \Omega_{\rho\sigma} S^{\rho\sigma}) = 
    \begin{pmatrix}
     e^{i \phi \cdot \sigma/2} & 0 \\
     0 & e^{-i \phi \cdot \sigma /2} \\
    \end{pmatrix},
  \end{equation}
  \begin{leftbar}
    \begin{remark}
      check sign
    \end{remark}
  \end{leftbar}
  where we get an extra factor of two due to the two $\epsilon$ dotted together.
  Consider a rotation of $2\pi$ around the $x^3$-axis: $\phi = 0,0, 2 \pi$ gives
  \begin{equation}
    S[\Lambda] = 
    \begin{pmatrix}
      e^{i\pi \sigma^3} & 0 \\
      0 & e^{-i\pi \sigma^3} \\
    \end{pmatrix}
    = -\mathbb{1}.
  \end{equation}
  Therefore, a rotation of $2\pi$ takes $\psi_\alpha(x) \to -\psi_{\alpha}(x)$. This is different to a vector, which goes as 
  \begin{equation}
    \Lambda = \exp(\frac{1}{2} \Omega_{\mu\nu} M^{\mu\nu}) = \exp
    \begin{pmatrix}
     0 & 0 & 0 & 0 \\
     0 & 0 & 2\pi & 0 \\
     0 & -2\pi & 0 & 0 \\
     0 & 0 & 0 & 0 \\
    \end{pmatrix} = \mathbb{1}_4,
  \end{equation}
  as expected.

  Let us now consider what happens to \emph{boosts of spinors}
  Using \eqref{eq:14-2star}, we get
  \begin{equation}
    S^{01} = \frac{1}{2} 
    \begin{pmatrix}
     -\sigma^i & 0 \\
     0 & \sigma^i \\
    \end{pmatrix}.
  \end{equation}
  Writing the boost parameter as $\Omega_{0i} \coloneqq \chi_i (=- \Omega_{i0})$, we have
  \begin{equation}
    S[\Lambda] = 
    \begin{pmatrix}
     e^{-\chi \cdot \sigma / 2} & 0 \\
     0 & e^{\chi \cdot \sigma/2} \\
    \end{pmatrix}
  \end{equation}
  For rotations, $S[\Lambda]$ is unitary, since $S[\Lambda]^{\dagger} S[\Lambda] = \mathbb{1}$, but for boosts it is not!
\end{proof}
\begin{claim}
  There are no finite-dimensional unitary representations of the Lorentz boosts.
\end{claim}
\begin{proof}
  The $S[\Lambda] = \exp[\frac{1}{2} \Omega_{\rho\sigma} S^{\rho\sigma}]$ are only unitary if $S^{\sigma\rho}$ are anti-Hermitian, i.e.~$(S^{\rho\sigma})^{\dagger} = -S^{\rho\sigma}$. $(S^{\rho\sigma})^{\dagger} = -\frac{1}{4} [(\gamma^{\rho})^{\dagger}, (\gamma^{\sigma})^{\dagger}]$ can be anti-Hermitian only if \emph{all} $\gamma^{\mu}$ are Hermitian, or if all $\gamma^{\mu}$ are anti-Hermitian.
  This cannot be arranged: $(\gamma^0)^2 = \mathbb{1}$ implies that $\gamma^0$ has real eigenvalues. Therefore it cannot be anti-Hermitian, since these have imaginary eigenvalues.
  Similarly, $(\gamma^i)^2 = -\mathbb{1}$ implies that $\gamma^i$ cannot be Hermitian.
  In general, there is no way to pick $\gamma^{\mu}$ such that $S^{\mu\nu}$ are anti-Hermitian.
\end{proof}

\section{Constructing a Lorentz Invariant Action of \texorpdfstring{$\psi$}{psi}}%
\label{sec:constructing_a_lorentz_invariant_action_of_psi_}

\begin{equation}
  \psi^{\dagger}(x) = (\psi^*)^T(x)
\end{equation}
Is $\psi^{\dagger}(x) \psi(x)$ a Lorentz scalar?
Under a LT, we have
\begin{equation}
  \psi^{\dagger}(x) \psi(x) \xrightarrow{LT} \psi^{\dagger}(\Lambda^{-1}x) S[\Lambda]^{\dagger} S[\Lambda] \psi(\Lambda^{-1} x).
\end{equation}
This is not a Lorentz scalar; this is where the non-unitarity of $S$ matters! 
\begin{leftbar}
  \begin{remark}
    In the chiral representation, $(\gamma^0)^{\dagger} = (\gamma^0)$, and $\gamma^i = -(\gamma^i)^{\dagger}$. Using the Clifford algebra, and the fact that $(\gamma^0)^2 = \mathbb{1}$, we can encapsulate both of these facts as
    \begin{equation}
      \label{eq:14-3}
      (\gamma^{\mu})^{\dagger} = \gamma^0 \gamma^{\mu} \gamma^0.
    \end{equation}
  \end{remark}
\end{leftbar}
Using \eqref{eq:14-3}, we have
\begin{equation}
  (S^{\mu\nu})^{\dagger} = - \frac{1}{4} [(\gamma^{\mu})^{\dagger}, (\gamma^{\nu})^{\dagger}] = -\gamma^0 S^{\mu\nu} \gamma^0,
\end{equation}
which we can use to show, by repeatedly using $(\gamma^0)^2 = \mathbb{1}$, that
\begin{equation}
  S[\Lambda]^{\dagger} = \exp(\frac{1}{2} \Omega_{\mu\nu} (S^{\mu\nu})^{\dagger}) = \gamma^0 S[\Lambda]^{-1} \gamma^0.
\end{equation}
The fact that these $\gamma^0$'s pop up points to the existence of another adjoint. With this in mind, we define the following.
\begin{definition}[]
  The \emph{Dirac adjoint} of $\psi$ is $\bar \psi(x) \coloneqq \psi^{\dagger}(x) \psi^0$.
\end{definition}

\begin{claim}
  $\bar \psi \psi$ is a scalar.
\end{claim}
\begin{proof}
  Under a Lorentz transformation $x^{\mu} \to \Lambda\indices{^{\mu}_{\nu}} x^{\nu}$, this transforms as
  \begin{align}
    \bar\psi(x)\psi(x) = \psi^{\dagger}(x) \gamma^0 \psi(x) \to &\psi^{\dagger}(\Lambda^{-1} x) S[\Lambda]^{\dagger} \gamma^0 S[\Lambda] \psi(\Lambda^{-1} x) \\
								&= \psi^{\dagger}(\Lambda^{-1} x)\gamma^0 \psi(\Lambda^{-1} x) \\
								&= \bar \psi(\Lambda^{-1} x) \psi(\Lambda^{-1} x).
  \end{align}
\end{proof}
\begin{claim}
  $\bar \psi(x) \gamma^{\mu} \psi(x)$ is a Lorentz vector.
\end{claim}
\begin{proof}
  Let us suppress the argument. Under Lorentz transformations, we have
  \begin{equation}
    \bar\psi \gamma^{\mu} \psi \to \bar\psi S[\Lambda]^{-1} \gamma^{\mu} S[\Lambda]\psi.
  \end{equation}
  If $\bar\psi \gamma^{\mu} \psi$ is to be a Lorentz transformation, we must have
  \begin{equation}
    \label{eq:14-1}
    S[\Lambda]^{-1} \gamma^{\mu} S[\Lambda] = \Lambda \indices{^{\mu}_{\nu}} \gamma^{\nu}.
  \end{equation}
  Recall that $\Lambda \indices{^{\mu}_{\nu}} = \exp(\frac{1}{2} \Omega_{\rho\sigma} M^{\rho\sigma})$ and $S[\Lambda] = \exp(\frac{1}{2} \Omega_{\rho\sigma} S^{\rho\sigma})$.
  Infinitesimally (taking the $\Omega$ terms to be small and throwing out higher order terms in $\Omega$), we get
  \begin{align}
    (M^{\rho\sigma}) \indices{^{\mu}_{\nu}} \gamma^{\nu} &= - [S^{\rho\sigma}, \gamma^{\mu}]. \\
    (\eta^{\rho\mu} \delta^{\sigma}_{\nu} - \gamma^{\sigma\mu} \delta^{\rho}_{\nu}) \gamma^{\nu}) &= \eta^{\rho\mu} \gamma^{\sigma} - \gamma^{\rho} \eta^{\sigma\mu}.
  \end{align}
  But by Claim \ref{cl:14-1}, the right hand side is $-[S^{\rho\sigma}, \gamma^{\mu}]$.
\end{proof}
\begin{corollary}
  The action
  \begin{equation}
    \label{eq:14-dirac-lagrangian}
    \boxed{ S = \int \dd[4]{x} \underbrace{\bar\psi(x) (i \gamma^{\mu} \partial_{\mu} - m) \psi(x)}_{\mathcal{L}_D} }
  \end{equation}
  is Lorentz invariant.
\end{corollary}

\section{The Dirac Equation}%
\label{sec:the_dirac_equation}

The Euler-Lagrange equations obtained from the Dirac Lagrangian \eqref{eq:14-dirac-lagrangian} are the \emph{Dirac equation} and the conjugate equation:
\begin{align}
  \overline{\psi}\colon \qquad (i \gamma^{\mu} \partial_{\mu} - m) \psi &= 0 \\
  {\psi}\colon \qquad i \partial_{\mu} \overline{\psi} \gamma^{\mu} + m \overline{\psi} &= 0
\end{align}
\begin{leftbar}
  \begin{remark}
    The Dirac equation is a first order ODE, unlike Klein-Gordon, yet still Lorentz invariant.
  \end{remark}
\end{leftbar}
\begin{definition}[slash notation]
  We write $A_{\mu} \gamma^{\mu} \coloneqq \cancel{A}$.
\end{definition}
In slash notation, we have $(i \cancel{\partial} - m) \psi = 0$.

\begin{claim}
  Each component $\phi_{\alpha}$ of $\psi$ solves the Klein-Gordon equation.
\end{claim}
\begin{proof}
  Act on the Dirac equation with $(i \cancel{\partial} + m)$ to get
  \begin{align}
    (i \cancel{\partial} + m)(i \cancel{\partial} - m) \psi &= 0 \\
    \implies -(\gamma^{\mu} \gamma^{\nu} \partial_{\mu} \partial_{\nu} + m^2) \psi &= 0 \\
    - (\frac{1}{2} \left\{ \gamma^{\mu}, \gamma^{\nu} \right\} \partial_{\mu} \partial_{\nu} + m^2) \psi &= 0\\
    - (\partial^2 + m^2) \psi_{\alpha} &= 0.
  \end{align}
\end{proof}
