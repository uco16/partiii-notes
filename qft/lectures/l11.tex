% lecture notes by Umut Özer
% course: qft
\lhead{Lecture 11: November 05}
\begin{example}[$\psi \bar \psi \to \phi\phi$]
  The Feynman diagrams that describe this interaction are
  \begin{multline}
    i \mathcal{M} = 
    \left(
    \begin{gathered}
      \feynmandiagram[scale=0.7, transform shape][vertical=b to e] {
        a -- [fermion, momentum=$p_1$] b -- [scalar, momentum=$p_1'$] c,
	b -- [fermion, momentum=$p_1 - p_1'$] e,
        d -- [anti fermion, momentum'=$p_2$] e -- [scalar, momentum'=$p_2'$] f,
	a -- [draw=none] d,
	c -- [draw=none] f,
      };
    \end{gathered}
    +
    \begin{gathered}
      \feynmandiagram[scale=0.7, transform shape][vertical=b to e] {
        a -- [fermion, momentum=$p_1$] b -- [scalar, momentum=$p_2'$] c,
	b -- [fermion, momentum=$p_1 - p_1'$] e,
        d -- [anti fermion, momentum'=$p_2$] e -- [scalar, momentum'=$p_1'$] f,
	a -- [draw=none] d,
	c -- [draw=none] f,
      };
    \end{gathered}
    \right)
    \\
    = (-ig)^2 \left( \frac{i}{(p_1 - p_1')^2 - \mu^2 + i \epsilon} + \frac{i}{(p_1 - p_2')^2 - \mu^2 + i \epsilon} \right)
  \end{multline}
  where we evaluated these diagrams by using the Feynman rules.
\end{example}

\subsection{Feynman Rules for \texorpdfstring{$\phi^4$}{phi four} theory}%
\label{sub:feynman_rules_for_phi^4_phi_four_theory}

We now introduce the $\phi^4$ theory with the interaction Hamiltonian $\frac{1}{4!} \lambda \phi^4$.
This is now drawn as a four-point vertex:
\begin{equation}
  \begin{gathered}
    \feynmandiagram[inline=(v.base), layered layout, horizontal=a to b] {
      a -- v [dot] -- d,
      b -- v -- c,
    };
  \end{gathered}
  \sim -i \lambda
\end{equation}
\begin{leftbar}
  \begin{remark}
    Note that there is no factor of $4!$. We will see why this cancels in the upcoming discussion of symmetry factors.
  \end{remark}
\end{leftbar}

\begin{example}[$\phi\phi \to \phi\phi$]
  \begin{equation}
    %diagram of four point vertex
    i \mathcal{M} \sim \frac{-i \lambda}{4!} \bra{p_1', p_2'} \normalorder{\phi(x) \phi(x) \phi(x) \phi(x)} \ket{ p_1, p_2}
  \end{equation}
  (where we neglected $\int \dots$).
\end{example}

\subsection{Symmetry Factors}%
\label{sub:combinatoric_factors}

There can be other combinatoric factors, also called \emph{symmetry factors}. These are often $2$ or $4$.
Consider $\bra{0} T \left\{ \phi_1 \dots \phi_m S \right\} \ket{0}$, where $\phi_i \coloneqq \phi(x_i)$.
This is an example of a \emph{correlation function}.
The $n$\textsuperscript{th} term in the perturbation expansion for $S$ gives us 
\begin{equation}
  \frac{1}{n!} \left( \frac{-i\lambda}{4!} \right)^n \int \dd[4]{y_1} \dots \dd[4]{y_n} \bra{0} T \left\{ \phi_1 \dots \phi_m \phi^4 (y_1) \dots \phi^4 (y_n) \right\} \ket{0}
\end{equation}
For example, the $n = 1$ and $m = 4$ term is
\begin{equation}
  \frac{-i\lambda}{4!} \int \dd[4]{x} \bra{0} T \left\{ \phi_1\phi_2\phi_3\phi_4 \phi^4_x \right\} \ket{0}
\end{equation}
From Wick's theorem, we get many terms:
\begin{align}
  \begin{split}
      \dots &= \frac{-i \lambda}{4!} \int \dd[4]{x} \wick{\c1 \phi_1 \c2 \phi_2 \c3 \phi_3 \c4 \phi_4 \c1 \phi_x \c2 \phi_x \c3 \phi_x \c4 \phi_x} + (\text{perms of contractions between } \phi_i \text{ and } \phi_x) \\
    	& +\frac{(-i \lambda)}{4!} \int \dd[4]{x} \wick{\c1 \phi_1 \c1 \phi_2} \wick{ \c1 \phi_3 \c2 \phi_4 \c1 \phi_x \c2 \phi_x } \wick{\c \phi_x \c \phi_x} + (\text{perms where we only contract two } \phi_i \text{'s}) \\
    	& +\frac{(-i \lambda)}{4!} \int \dd[4]{x} \wick{\c \phi_1 \c \phi_2} \wick{\c \phi_3 \c \phi_4} \wick{ \c \phi_x \c \phi_x } \wick{\c \phi_x \c \phi_x} + (\text{perms where we contract all pairs } \phi_x)
  \end{split}\\[2ex]
  \begin{split} \label{eq:11-n1m4}
    &= -i \lambda \int \dd[4]{x} \Delta_F (x_1 - x) \Delta_F (x_2 - x) \Delta_F (x_3 - x) \Delta_F (x_4 - x) \\
    & +\underbrace{\frac{(-i \lambda)}{2}} \int \dd[4]{x} \Delta_F (x_1 - x_2) \Delta_F (x_3 - x) \Delta_F (x_4 - x) \Delta_F (x - x) + (\text{similar}) \\
    &\text{factor 2 since there are 12 ways of pairing $\phi_3$, $\phi_4$ with $\phi_x$} \\
    & +\underbrace{\frac{(-i\lambda)}{8}} \int \dd[4]{x} \Delta_F(x_1 - x_1) \Delta_F( x_3 - x_4) \Delta_F(x - x) \Delta_F(x - x) + (\text{perms}) \\
    &\text{factor 8 due to 3 ways of pairing $\phi_x$'s}.
  \end{split}
\end{align}
Diagrammatically, the result \eqref{eq:11-n1m4} can be written
\begin{multline}
  \dots = 
  \feynmandiagram[inline=(v.base), horizontal=a to b, layered layout] {
    a [particle=\(x_3\)] -- v [particle=\(x\)] -- d [particle=\(x_1\)],
    b [particle=\(x_4\)] -- v -- c [particle=\(x_2\)],
  };
  +
  \underbrace{
  \left(
    \feynmandiagram[inline=(v.base), horizontal=a to b] {
    a [particle=\(x_3\)] -- v [particle=\(x\)] -- [draw=none] d [particle=\(x_1\)],
    b [particle=\(x_4\)] -- v -- [draw=none] c [particle=\(x_2\)],
    a -- [draw=none] b,
    c -- d,
    v -- [loop, min distance=1cm] v
    };
    + 5 \text{ similar}
  \right)
  }_{\text{symmetry factor } 2}
  \\
  +
  \underbrace{
    \feynmandiagram[inline=(a.base), layered layout] {
      a [dot] -- [loop, min distance=2cm] a
        -- [loop, min distance=2cm, in=-135, out=-45] a,
    };
    \times
  \left(
    \begin{gathered}
      \feynmandiagram[small, horizontal=a to b] {
      a [particle=\(x_1\)] -- b [particle=\(x_2\)],
      c [particle=\(x_3\)] -- d [particle=\(x_4\)],
      a -- [draw=none] c,
      b -- [draw=none] d,
      };
    \end{gathered}
    + 2 \text{ similar}
  \right)
  }_{\text{symmetry factor } 8}
\end{multline}

As a second example, the case $n= 2$ and $m = 4$ gives
\begin{equation}
  \frac{1}{2!} \left(  \frac{-i \lambda}{4!} \right)^2 \int \bra{0} T \left\{ \phi_1\phi_2\phi_3\phi_4 \phi_x^4 \phi_y^4 \right\} \ket{0} \dd[4]{x} \dd[4]{y}
\end{equation}
One of the terms that contributes is
\begin{equation}
  \wick{\c1 \phi_1 \c2 \phi_2 \c3 \phi_3 \c4 \phi_4 \c1 \phi_x \c2 \phi_x \c2 \phi_x \c1 \phi_x \c1 \phi_y \c2 \phi_y \c3 \phi_y \c4 \phi_y}
   = 
   \begin{gathered}
     \feynmandiagram[scale=0.6, transform shape][horizontal=a to b] {
       a [particle=\(x_1\)] -- b [particle=\(x\)] -- c [particle=\(x_2\)],
       b -- [half left] e [particle=\(y\)] -- [half left] b,
       d [particle=\(x_3\)] -- e -- f [particle=\(x_4\)],
       a -- [draw=none] d,
       c -- [draw=none] f,
     };
   \end{gathered}
\end{equation}
This gives
\begin{equation}
  \frac{(-i \lambda)^2}{2} \int \dd[4]{x} \dd[4]{y} \Delta_F (x_1 - x) \Delta_F(x_2 - x) \Delta_F(x_3 - y) \Delta_F(x_4 - y) \Delta_F(x - y)^2
\end{equation}
The symmetry factor is 
\begin{equation}
  \frac{1}{2} = \frac{1}{2! (4!)^2} \times \underbrace{4 \times 3}_{x_1, x_2, x} \times \underbrace{12}_{x_3, x_4, y} \times \underbrace{2}_{x, y} \times \underbrace{2}_{x \leftrightarrow y}
\end{equation}
%D3 diagrams
are distinct diagrams, which also contribute.

The Feynman diagram rules are thus:
\begin{itemize}
  \item $\bra{0} T \left\{ \phi_1 \dots \phi_m \exp(\frac{-i \lambda}{4!} \int \dd[4]{x} \phi^4_x) \right\} \ket{0}$ is the sum of all diagrams with $m$ external points and any number of vertices connected by propagator lines.
  \item For each diagram, there exists and integral containing
    \begin{itemize}
      \item propagator 
	\begin{equation}
	  \feynmandiagram[inline=(a.base), horizontal=a to b] {
	    a [particle=\(x\)] -- b [particle=\(z\)],
	  };
	  = \Delta_F(y - z)
	\end{equation}
      \item vertex
	\begin{equation}
	  \feynmandiagram[layered layout, horizontal=a to b, inline=(v.base)]{
	    a -- v [particle=\(x\)] -- d,
	    b -- v -- c,
	  };
	  = -i \lambda \int \dd[4]{x}
	\end{equation}
    \end{itemize}
  \item Divide by symmetry factor
\end{itemize}
Since 
\begin{equation}
  \Delta_F(x - y) = \int \frac{\bdd[4]{p} i}{p^2 - m^2 + i\epsilon} e^{-i p \cdot (x - y)},
\end{equation}
we have
\subsection{\texorpdfstring{$\phi^4$}{Phi four} Feynman Rules with Momenta}%
\label{sub:$phi^4$_phi_four_feynman_rules_with_momenta}

\begin{itemize}
  \item propagator
    \begin{equation}
      \feynmandiagram[inline=(a.base), horizontal=a to b] {
	a [particle=\(x\)] -- [rmomentum=$p$] b [particle=\(y\)],
      };
    \end{equation}
    assign $e^{i p \cdot y}$ to $y$ vertex (arrows out) and $e^{-i p \cdot x}$ to $x$ vertex (arrows in) and $\frac{i}{p^2 - m^2 + i \epsilon}$ to the line, with $\int \bdd[4]{p}$.
\end{itemize}
The integral at vertex $x$ 
%TODO: D4
becomes
\begin{equation}
  \int \dd[4]{x} e^{-i (p_1 + p_3 - p_4 - p_2) \cdot x} = \bdelta^4(p_1 + p_4 - p_2 - p_4)
\end{equation}

\subsection*{Summary}%

\begin{enumerate}
  \item propagator: \(
    \feynmandiagram[inline=(a.base), horizontal=a to b] {
      a -- [rmomentum=$p$] b,
    };
    = \frac{i}{p^2 - m^2 + i \epsilon}
    \)
  \item vertex: \(
    \feynmandiagram[layered layout, horizontal=a to b, inline=(v.base)]{
      a -- v [dot] -- d,
      b -- v -- c,
    };
    = -i \lambda
    \)
  \item Impose $4$-momentum conservation at each vertex
  \item Integrate over all undetermined momenta $\int \bdd[4]{k}$
  \item Divide by symmetry factor
\end{enumerate}

\subsection{Vacuum Bubbles and Connected Diagrams}%
\label{sub:vacuum_bubbles_and_connected_diagrams}

The diagrammatic expansion of $\bra{0} S \ket{0}$ in $\phi^4$ theory is
\begin{equation}
  1 + 
  \underbrace{
    \feynmandiagram[inline=(c.base), layered layout] {
      c [dot] -- [loop, min distance=2cm] c,
      c -- [loop, in=-135, out=-45, min distance=2cm] c,
    };
  }_{1^{\text{st}}\text{ order}}
  + 
  \begin{gathered}
    \feynmandiagram[scale=0.7, transform shape][vertical=b to t, layered layout] {
      t [dot] -- [loop, min distance=2cm, in=-135, out=-45] t,
      t -- [half left] b [dot] -- [half left] t,
      b -- [loop, min distance=2cm] b,
    };
  \end{gathered}
  +
  \begin{gathered}
    \feynmandiagram[scale=0.8, transform shape][vertical=a to b, layered layout] {
      a [dot] -- [half left] b [dot] -- [half left] a,
      a -- [half left, looseness=0.5] b -- [half left, looseness=0.5]a,
    };
  \end{gathered}
  +
  \feynmandiagram[inline=(c.base), horizontal=c to d, layered layout] {
    c [dot] -- [loop, min distance=2cm] c,
    c -- [loop, in=-135, out=-45, min distance=2cm] c,
    c -- [draw=none] d [dot],
    d -- [loop, min distance=2cm] d,
    d -- [loop, in=-135, out=-45, min distance=2cm] d,
  };
  + \dots
\end{equation}
These are called \emph{vacuum bubbles}.
The combinatorial factors are such that
\begin{align}
  \bra{0} S \ket{0} &= \exp (
  \begin{gathered}
    \feynmandiagram[scale=0.6, transform shape][layered layout] {
      c [dot] -- [loop, min distance=2cm] c,
      c -- [loop, in=-135, out=-45, min distance=2cm] c,
    };
  \end{gathered}
  + 
  \begin{gathered}
    \feynmandiagram[scale=0.4, transform shape][vertical=b to t, layered layout] {
      t [dot] -- [loop, min distance=2cm, in=-135, out=-45] t,
      t -- [half left] b [dot] -- [half left] t,
      b -- [loop, min distance=2cm] b,
    };
  \end{gathered}
  +
  \begin{gathered}
    \feynmandiagram[scale=0.5, transform shape][vertical=a to b, layered layout] {
      a [dot] -- [half left] b [dot] -- [half left] a,
      a -- [half left, looseness=0.5] b -- [half left, looseness=0.5]a,
    };
  \end{gathered}
  +
  \begin{gathered}
    \feynmandiagram[scale=0.6, transform shape][horizontal=c to d, layered layout] {
      c [dot] -- [loop, min distance=2cm] c,
      c -- [loop, in=-135, out=-45, min distance=2cm] c,
      c -- [draw=none] d [dot],
      d -- [loop, min distance=2cm] d,
      d -- [loop, in=-135, out=-45, min distance=2cm] d,
    };
  \end{gathered}
  + \dots
  ) \\
  &= \exp(\text{sum over all distinct vacuum bubble types})
\end{align}
\begin{exercise}
  This is done in the second example sheet.
\end{exercise}

Remember that our correlation function $\bra{0} T \left\{ \phi_1 \dots \phi_m S \right\} \ket{0}$, also called the \emph{$m$-point function} is the sum over all the diagrams with $m$ external points.
A typical diagram has some vacuum bubbles in it.
Remarkably, these vacuum bubbles add to the \emph{same} exponential series as above.
In other words
\begin{equation}
  \bra{0} T \left\{ \phi_1 \dots \phi_m S \right\} \ket{0} = \sum( \text{connected diagrams}) \times \bra{0} S \ket{0}.
\end{equation}
\begin{example}[]
  Let us look at the second order term in a two point function.
  In general, there will be a term
  \begin{equation}
    \left( 
      \feynmandiagram[inline=(a.base), horizontal=a to b, layered layout] {
        a -- b [dot] -- [loop, min distance=2cm] b -- c,
      };
      \times
      \feynmandiagram[inline=(c.base), layered layout] {
	c [dot] -- [loop, min distance=2cm] c,
	c -- [loop, in=-135, out=-45, min distance=2cm] c,
      };
    \right).
  \end{equation}
  Only the first diagram is connected
\end{example}
\begin{definition}[]
  A diagrams is said to be \emph{connected} if every part of the diagram is connected to at least 1 external point.
\end{definition}
\begin{example}[]
  Connected diagrams for the two-point function are
  \begin{equation}
    \feynmandiagram[inline=(a.base), horizontal=a to b] {
      a -- b,
    };
    + 
    \feynmandiagram[inline=(b.base), horizontal=a to b, layered layout] {
      a -- b [dot] -- [loop, min distance=2cm] b -- c,
    };
    + 
    \feynmandiagram[inline=(a.base), horizontal=a to b, layered layout] {
      a -- b [dot] -- [loop, min distance=2cm] b -- c [dot] -- [loop, min distance=2cm] c -- d,
    };
    + \dots
    % see other's notes
  \end{equation}
\end{example}
