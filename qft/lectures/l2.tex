\lhead{Lecture 2: October 12}


\begin{example}[Maxwell's Equations]

The Lagrangian which allows us to recover the charge-free Maxwell equations is
\begin{equation}
  \mathcal{L} = -\frac{1}{2} (\partial_\mu A_\nu) (\partial^\mu A_\nu) + \frac{1}{2}(\partial_\mu A^\mu)^2
\end{equation}

Noting that $\mathcal{L}$ only depends on derivatives $\partial_\mu A_\nu$ of the scalar field, we only need to compute
\begin{equation}
  \pdv{\mathcal{L}}{(\partial_\mu A_\nu)} = -\partial^\mu A^\nu + \eta\indices{^\mu^\nu}\partial_\rho A^\rho.
\end{equation}
The Euler-Lagrange equations then give the Maxwell equations in four-vector notation:
\begin{align}
  \partial_\mu \left( \pdv{\mathcal{L}}{\left( \partial_\mu A_\nu \right)} \right) = 0 &= -\partial_\mu \partial^\mu A^\nu + \partial^\nu (\partial_\rho A^\rho) \\
  &= -\partial_\mu (\partial^\mu A^\nu - \partial_\nu A^\mu) \\
  &= -\partial_\mu F^{\mu \nu}
\end{align}
where $F_{\mu \nu} = \partial_\mu A_\nu - \partial_\nu A_\nu$ is the Maxwell field strength tensor.
\end{example}
In the previous examples, we only considered local Lagrangians, which means that the Lagrangian does not involve any products of fields like $\phi(\vb{x}, t) \phi(\vb{y}, t)$ where the positions differ $\vb{x} \neq \vb{y}$.

\section{Lorentz Invariance}%
\label{sec:lorentz_invariance}

Consider the Lorentz transformation (LT) $\Lambda$ of some scalar field $\phi(x) \equiv \phi(x^\mu)$.
Under $\Lambda$, the field changes as $\phi \to \phi'$ where $\phi'(x) = \phi(x')$ and $x'^\mu = (\Lambda^{-1})\indices{^\mu_\nu} x^\nu$.

The defining equation for Lorentz transformations is
\begin{equation}
  \Lambda\indices{^\mu_\sigma}\eta\indices{^\sigma^\rho} \Lambda\indices{_\rho ^\nu} = \eta\indices{^\mu^\nu}.
\end{equation}
Examples of a Lorentz transformation include rotation around the $x$-axis or Lorentz boosts, whose matrix representation is respectively
\begin{equation}
  \Lambda\indices{^\mu_\nu} = 
  \begin{pmatrix}
   1 &  &  &  \\
    & 1 &  &  \\
    &  & \cos\theta & \sin\theta \\
    &  & -\sin\theta & \cos\theta \\
  \end{pmatrix},
  \qquad
  \Lambda\indices{^\mu_\nu} = 
  \begin{pmatrix}
   \gamma & -\gamma v &  &  \\
   -\gamma v & \gamma &  &  \\
    &  & 1 &  \\
    &  &  & 1 \\
  \end{pmatrix},
\end{equation}
where $\gamma = (1- v^2)^{-1/2}$.
The Lorentz transformations form a Lie group under matrix multiplication. This means that applying two Lorentz transformations gives another, and any Lorentz transformation has an inverse.
The Lorentz transformations allow a representation on the fields.
For a scalar field, this is $\phi(x) \to \phi'(x) = \phi(\Lambda^{-1} x)$. This is an \emph{active transformation}, which genuinely rotates the field. This is why the inverse Lorentz transformation $\Lambda^{-1}$ is needed to used the new coordinates to the old.
A \emph{passive transformation} is one where we just relabel the coordinates. Under such a transformation, a scalar field changes as $\phi(x) \to \phi'(\Lambda x)$.
\emph{Lorentz invariant} theories are such that the action $S$, and the dynamics that are described by it, are unchanged under Lorentz transformations.
In particular, if $\phi(x)$ satisfies the equations of motion of a Lorentz invariant theory, then so does the transformed field $\phi(\Lambda^{-1} x)$.

\begin{example}[Klein-Gordon]
Taking the K-G Lagrangian density, we have the action
\begin{equation}
  S = \int_{\mathbb{R}^4}^{} \dd[4]{x} \frac{1}{2} \partial_\mu \phi \partial^\mu \phi - U(\phi)
\end{equation}
where $U$ is some polynomial.
Since we know how $\phi$ behaves under LT, we can deduce that $U$ behaves under LTs in the following way:
\begin{equation}
  U'(x) = U(\phi'(x)) = U(\phi(x')) = U(x').
\end{equation}
The first term in the action transforms as
\begin{align}
  ( \partial_\mu \phi )' &= \pdv{}{x^\mu}\phi(x') = \pdv{x'^\sigma}{x^\mu} \pdv{}{x'^\sigma} \phi(x') \\
  &= (\Lambda^{-1})\indices{^\sigma_\mu} \partial_\sigma' \phi(x'),
\end{align}
where we denoted $\partial_\sigma' = \pdv{}{x'^\sigma}$.
The kinetic term in the Lagrangian is thus
\begin{align}
  \mathcal{L}'_{\text{kin}} &= \frac{1}{2} \eta\indices{^\mu^\nu} (\partial_\mu \phi)' (\partial_\nu \phi)' \\
  &= \frac{1}{2} \underbrace{\eta\indices{^\mu^\nu}(\Lambda^{-1})\indices{^\sigma_\mu} (\Lambda^{-1})\indices{^\rho_\nu} }_{\eta\indices{^\sigma^\rho}} \partial_\sigma' \phi(x') \partial_\rho' \phi(x') \\
  &= \frac{1}{2} \eta\indices{^\sigma^\rho} \partial_\sigma' \phi(x) \partial_\rho' \phi(x').
\end{align}
The Lagrangian density transforms like the scalar field $\phi$; this transformation law means that $\mathcal{L}$ is itself a scalar field.
The action changes under LT as
\begin{equation}
  S' = \int_{}^{} \dd[4]{x} \mathcal{L}(x') = \int_{}^{} \dd[4]{x} \mathcal{L} (\Lambda^{-1}x).
\end{equation}
  We change variables to $y = \Lambda^{-1} x$. Since LTs are in the special orthogonal group, we have unit determinant $\det \Lambda = 1$ which means that the Jacobian is unity and $\int_{}^{} \dd[4]{x}  = \int_{}^{} \dd[4]{x}$. As a result, the action is invariant:
  \begin{equation}
    S' = \int_{}^{} \dd[4]{y} \mathcal{L}(y) = S.
  \end{equation}
\end{example} 

\begin{leftbar}
  \begin{remark}
    Under a LT, a vector field $A_\mu$ transforms like $\partial_\mu \phi$, so
    \begin{equation}
      A_\mu'(x) = (\Lambda^{-1})\indices{_\mu^\sigma} A_\sigma(\Lambda^{-1} x).
    \end{equation}
    If all indices are summed over, the result is Lorentz invariant.
  \end{remark}
\end{leftbar}

\begin{example}
Do Q1 in Ex.~sheet 1.
\end{example}

\section{Noether's theorem}%
\label{sec:noether_s_theorem}

Already noticeable in classical field theory: symmetries are important.

\begin{theorem}[Noether's theorem]
  Every continuous symmetry of $\mathcal{L}$ gives rise to a current $j^\mu(x)$ which is conserved:
  \begin{equation}
    \partial_\mu j^\mu = \pdv{(j^0)}{t} + \div \vb{j} = 0.
  \end{equation}
\end{theorem}

\begin{proof}

  Consider an infinitesimal variation of a field $\phi$:
  \begin{equation}
    \phi(x) \to \phi'(x) = \phi(x) + \alpha \Delta \phi(x),
  \end{equation} 
  where $\alpha$ is seen as an infinitesimal parameter.
  This is a symmetry if $S$ is unchanged, i.e.~ $\mathcal{L}$ should be invariant up to a total $4$-divergence (which integrates to a surface term and does not effect the E-L equations).
  In other words, the Lagrangian density changes as
  \begin{equation}
    \label{eq:Lchange}
    \mathcal{L}(x) \to \mathcal{L}(x) + \alpha \partial_\mu X^\mu(x) = 0.
  \end{equation}
  In fact, often we have $X^\mu = 0$.
  Recall that $\mathcal{L}$ only depends on $\phi$ and $\partial_\mu \phi$, so
  \begin{align}
    \mathcal{L}(x) &\to \mathcal{L}(x) + \alpha \pdv{\mathcal{L}}{\phi} \Delta \phi + \alpha \pdv{\mathcal{L}}{(\partial_\mu \phi)} \partial_\mu(\Delta \phi) \\
    \label{eq:Lchange_2}
    &= \mathcal{L}(x) + \alpha \partial_\mu \left( \pdv{\mathcal{L}}{(\partial_\mu \phi)} \Delta \phi \right) + \alpha \underbrace{\left( \pdv{\mathcal{L}}{\phi} - \partial_\mu \pdv{\mathcal{L}}{(\partial_\mu \phi)} \right)}_{\text{E-L: } = 0} \Delta \phi.
  \end{align}

  Combining \eqref{eq:Lchange} and \eqref{eq:Lchange_2}, we see that
  \begin{equation}
    j^\mu = \pdv{\mathcal{L}}{(\partial_\mu \phi)} \Delta \phi - X^\mu
  \end{equation}
  is conserved: $\partial_\mu j^\mu = 0$.
\end{proof}
\begin{leftbar}
  \begin{remark}
    Each conserved current has associated with it a conserved charge:
    \begin{equation}
      Q = \int_{\mathbb{R}^3}^{} \dd[3]{x} j^0.
    \end{equation}
    \begin{proof}
      The current changes in time as
      \begin{align}
	\dv{Q}{t} &= \dv{}{t} \int_{\mathbb{R}}^{} \dd[3]{x} \partial_0 j^0 \\
	&= - \int_{\mathbb{R}^3}^{} \dd[3]{x} \div \vb{j}
      \end{align}
      which vanishes by the divergence theorem.
    \end{proof}
    Starting from the Lagrangian density $\mathcal{L}$ and the symmetry transformation, we can work out what the conserved current is.
  \end{remark}
\end{leftbar}

\begin{example}[Scalar Field]
  \begin{equation}
    \psi (x) = \frac{1}{\sqrt{2}} (\phi_1(x) + i \phi_{28}(x))
  \end{equation}
  We could also use two real fields to describe this theory. However, when using complex fields, we can consider $\psi$ and $\psi^*$ as independent variables. This is because we really have two independent degrees of freedom $\phi_1$ and $\phi_2$.
  Using these variables, the Lagrangian is
  \begin{equation}
    \mathcal{L} = \partial_\mu \psi* \partial^\mu \psi - V(\abs{\psi}^2).
  \end{equation}
  \begin{leftbar}
    \begin{remark}
      The potential could be for example:
      \begin{equation}
	V(\abs{\psi}^2) = m^2 \psi* \psi + \frac{\lambda}{2} (\psi^* \psi)^2.
      \end{equation}
      We will see later that the first summand will be a mass term and the second will describe an interaction.
    \end{remark}
  \end{leftbar}
  The symmetry is a complex phase rotation:
  \begin{equation}
    \psi \to e^{i\alpha} \psi \implies \psi^* \to e^{-i\alpha} \psi^*.
  \end{equation}
  Since $\mathcal{L}$ is invariant under this transformation, this is a symmetry with $M^\mu = 0$.
  The fields change as
  \begin{equation}
    \Delta \psi = i \alpha \psi \qquad \Delta \psi^* = -i \alpha \psi^*.
  \end{equation}
  The currents from the two different fields will add:
  \begin{equation}
    j^\mu = (\psi \partial^\mu \psi - \psi^* \partial^\mu \psi).
  \end{equation}
  This conserved charge could be the electric charge or some particle number (baryon, lepton, etc.) for example. In QED, we will see something every similar.
\end{example} 

\begin{example}
Do questions 2 and 3 of Example sheet 1.
\end{example}
