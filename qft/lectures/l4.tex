%  _              _                    _  _   
% | |    ___  ___| |_ _   _ _ __ ___  | || |  
% | |   / _ \/ __| __| | | | '__/ _ \ | || |_ 
% | |__|  __/ (__| |_| |_| | | |  __/ |__   _|
% |_____\___|\___|\__|\__,_|_|  \___|    |_|  

% lecture notes by Umut Özer
% course: qft
\lhead{Lecture 4: October 17}

\section{The Free Scalar Field}%
\label{sec:the_free_scalar_field}

Let us apply the SHO to free fields. We define the Fourier transform of a scalar field as
\begin{equation}
  \phi(\vb{x}) = \int_{}^{} \bdd[3]{p} \frac{1}{\sqrt{2\omega_{p}}} \left( a_{\vb{p}} e^{i \vb{p} \cdot \vb{x}}+ a^{\dagger}_{\vb{p}} e^{-i \vb{p} \cdot \vb{x}} \right),
\end{equation}
where $\omega_{p}^2 = p^2 + m^2$, writing $p = \sqrt\abs{\vb{p}}^2$.
Since we explicitly added a Hermitian conjugate constant $a^\dagger$, this definition of the Fourier integral makes it evident that $\phi = \phi^\dagger$ is a real scalar field.
We also expand the conjugate momentum $\pi(\vb{x})$ in terms of these Fourier modes
\begin{equation}
  \pi(\vb{x}) = \int_{}^{} \bdd[3]{p} (-i) \sqrt{\frac{\omega_{p}}{2}} \left( a_{\vb{p}} e^{i \vb{p} \cdot \vb{x}} - a_{\vb{p}}^{\dagger} e^{-i \vb{p} \cdot \vb{x}} \right).
\end{equation}
The process of promoting the constants $a_{\vb{p}}$ and $a_{\vb{p}}^\dagger$ to quantum operators---as it turns out, these will be annihilating and creating particles of momentum $\vb{p}$--- acting on a quantum state is called \emph{second quantisation}; we have inserted an infinite number of QHOs into momentum space.
As a result, the fields $\phi$ and $\pi$ are also quantum operators. For these quantum fields, we want to impose commutation relations:
\begin{subequations}
  \begin{equation}
    [\phi(\vb{x}), \phi(\vb{y})] = 0, \qquad [\pi(\vb{x}), \pi(\vb{y})] = 0,
  \end{equation}
  \begin{equation} \label{eq:field-ccr}
    [\phi(\vb{x}), \pi(\vb{y})] = i \delta^3(\vb{x} - \vb{y}).
  \end{equation}
\end{subequations}

\begin{leftbar}
  \begin{remark}
    We are in the Schr\"odinger picture, where time $t = x^0$ is in the states, not in the fields.
  \end{remark}
\end{leftbar}

\begin{claim}
  These are equivalent to
  \begin{subequations}
    \begin{equation}
      [ a_{\vb{p}}, a_{\vb{q}} ] = 0, \qquad [a^{\dagger}_{\vb{p}}, a^{\dagger}_{\vb{q}}] = 0,
    \end{equation}
    \begin{equation} \label{eq:a-ccr}
      [ a_{\vb{p}}, a^{\dagger}_{\vb{q}} ] = \bdelta^3(\vb{p}-\vb{q}).
    \end{equation}
  \end{subequations}
\end{claim}
\begin{proof}
  Let us check that \eqref{eq:a-ccr} implies \eqref{eq:field-ccr} .
  \begin{align}
    [\phi(\vb{x}), \pi(\vb{y})] &= \int_{}^{} \bdd[3]{p} \bdd[3]{q} \frac{(-i)}{2} \frac{\omega_q}{\omega_p}
    \left\{ - \left[ {a_{\vb{p}}}, {a^{\dagger}_{\vb{q}}} \right] e^{i\vb{p}\cdot\vb{x} - i\vb{q}\cdot\vb{y}} + \left[ {a^{\dagger}_{\vb{p}}}, {a_{\vb{q}}} \right] e^{-i\vb{p}\cdot\vb{x} + i\vb{q}\cdot\vb{y}} \right\} \\
    &= \int_{}^{} \bdd[3]{p} \left( \frac{-i}{2} \right) \left\{ - e^{i \vb{p} \cdot(\vb{x} - \vb{y})} - e^{i\vb{p}\cdot(\vb{x}-\vb{y})} \right\} \\
    &= i \delta^{3}(\vb{x} - \vb{y})
  \end{align}
\begin{exercise}
  Complete the proof.
\end{exercise}
\end{proof}

Now, compute the Hamiltonian $H$ in terms of the ladder operators $a_{\vb{p}}$ and $a^{\dagger}_{\vb{q}}$:
\begin{align}
    H &= \frac{1}{2} \int_{}^{} \dd[3]{x} \left[ \pi^2 + (\grad \phi)^2 + m^2 \phi^2 \right] 
    \\[1em]
  \begin{split}
    &= \frac{1}{2} \int_{}^{} \dd[3]{x} \bdd[3]{p} \bdd[3]{q} \\
    & \qquad \times \Big\{ 
    -\frac{\sqrt{\omega_p \omega_2}}{2} \left( a_{\vb{p}} e^{i\vb{p}\cdot\vb{x}} - a^{\dagger}_{\vb{p}} e^{-i\vb{p}\cdot\vb{x}} \right)\left( a_{\vb{q}} e^{i\vb{q}\cdot\vb{x}} - a^{\dagger}_{\vb{q}} e^{i\vb{q}\cdot\vb{x}} \right) \qquad \text{from } \pi^2\\
    &\qquad-\frac{1}{2}\sqrt{\omega_p \omega_2} \left( i p a_{\vb{p}} e^{i\vb{p}\cdot\vb{x}} - ip a^{\dagger}_{\vb{p}} e^{-i\vb{p}\cdot\vb{x}} \right)
    \left(iq a_{\vb{q}} e^{i\vb{q}\cdot\vb{x}} - iqa^{\dagger}_{\vb{q}} e^{i\vb{q}\cdot\vb{x}} \right) \qquad \text{from } (\grad \phi)^2 \\
    &\qquad-\frac{m^2}{2\sqrt{\omega_p \omega_2}} \left( a_{\vb{p}} e^{i\vb{p}\cdot\vb{x}} + a^{\dagger}_{\vb{p}} e^{-i\vb{p}\cdot\vb{x}} \right)
    \left( a_{\vb{q}} e^{i\vb{q}\cdot\vb{x}} + a^{\dagger}_{\vb{q}} e^{-i\vb{q}\cdot\vb{x}} \right) \Big\} \qquad \text{from } m^2 \phi^2
  \end{split} 
  \\[1em]
  &= \cdots \nonumber
  \\[1em]
  \begin{split}
    &= \frac{1}{4} \int_{}^{} \bdd[3]{p} \Big\{ \underbrace{\left(\cancel{ -\omega_p + \frac{p^2}{\omega_p} + \frac{m^2}{\omega_p}})\right)}_{\text{since }\omega^2_p = p^2 + m^2} \left( a_{\vb{p}} a_{-\vb{p}} + a^{\dagger}_{\vb{p}} a_{-\vb{p}}^{\dagger} \right) + \left( \omega_p + \frac{p^2}{\omega_p} + \frac{m^2}{\omega_p} \right) (a_{\vb{p}} a^{\dagger}_{\vb{p}} + a^{\dagger}_{\vb{p}} a_{\vb{p}}) \Big\}
  \end{split}
  \\[1em]
  \therefore \quad H &= \frac{1}{2} \int_{}^{} \bdd[3]{p} \omega_p (a_{\vb{p}} a^{\dagger}_{\vb{p}} + a^{\dagger}_{\vb{p}} a_{\vb{p}}).
\end{align}
This is the Hamiltonian of an infinite number of uncoupled harmonic oscillators, each at frequency $\omega_p = \sqrt{p^2 + m^2}$.

\subsection{The Vacuum}%
\label{sub:the_vacuum}

\begin{definition}[vacuum]
  The \emph{vacuum} or \emph{ground state} $\ket{0}$ is defined to be the state with the property that it is annihilated by all annihilation operators:
  \begin{equation}
    a_{\vb{p}} \ket{0} = 0, \quad \forall p.
  \end{equation}
\end{definition}

Acting on the vacuum with the Hamiltonian, we expect to find the ground-state energy:
\begin{align}
  H \ket{0} &= \int_{}^{} \bdd[3]{p} \omega_p \left( \cancel{a^{\dagger}_{\vb{p}} a_{\vb{p}}} + \frac{1}{2} [a_{\vb{p}}, a^{\dagger}_{\vb{p}}] \right) \ket{0} \\
  &= \frac{1}{2}\int_{}^{} \bdd[3]{p} \omega_p \bdelta^3(0) \ket{0}.
\end{align}
We find an infinite ground-state energy!
And worse, this infinity is twofold; we have the obvious factor of $\bdelta^3(0)$, but also a less obvious infinity that arises because the integral $\int_{}^{} \bdd[3]{p} \omega_p$ does not converge; the energy of each harmonic oscillator diverges $\omega_{p} = \sqrt{p^2 + m^2} \rightarrow \infty$ as $p \to \infty$.
This latter divergence is called a \emph{high-frequency} or \emph{ultra-violet} divergence. We can resolve this issue by introducing a UV-cutoff $\Lambda$ as an upper integration bound, which tames the integral. 
\begin{leftbar}
  \begin{remark}
    Physically, this can be justified by postulating that $\Lambda$ corresponds to the highest energy (or equivalently, $\Lambda^{-1}$ the smallest length scale) on which this QFT is valid. On higher energy or smaller length scales, new physics will need to be introduced.
    In statistical field theory, this smallest length scale is on the order of the lattice spacing. On smaller length scales, the continuum assumptions of the theory break down and new physics---accounting for the lattice effects---will need to be introduced.
  \end{remark}
\end{leftbar}

Concerning the former infinity, $\bdelta^3(0)$, we have another resolution up our sleeves: Heuristically, one might claim that in non-gravitational physics, only energy differences are relevant for the dynamics of the system. As such, one might redefine the Hamiltonian to subtract the commutator $[a_{\vb{p}}, a^{\dagger}_{\vb{p}}]$ which led to the divergence.
More concretely, this arbitrary choice that we can make in the Hamiltonian points to an ambiguity arising in moving from the classical to the quantum theory:

Classically, the Hamiltonian $H = \frac{1}{2}(\omega q - i p)(\omega q + ip)$ is exactly the same as $H = \frac{1}{2}p^2 + \frac{1}{2} \omega^2 q^2$.
However, in QM, the operator order matters, and upon quantisation it gives us $H = \omega a^{\dagger} a$.
The analogous thing happens in QFT with momentum modes, which means that the Hamiltonian that we should really use is
\begin{equation}
  H = \int_{}^{} \bdd[3]{p} \omega_{\vb{p}} a^{\dagger}_{\vb{p}} a_{\vb{p}}.
\end{equation}
Using this Hamiltonian, the ground-state energy of the vacuum has been redefined to $H \ket{0} = 0$.

This ambiguity in the definition of QFT operators can be resolved by introducing the concept of normal ordering.
\begin{definition}[normal order]
  A \emph{normal ordered} string of operators $\normalorder{\phi_1(\vb{x}_1) \cdots \phi_n(\vb{x}_n)}$ is defined to be the same string of operators, except that all annihilation operators $a$ are moved to the RHS of all creation operators $a^{\dagger}$.
\end{definition}
Since the annihilation operators annihilate the vacuum, the $\bdelta^3(0)$ infinity is discarded when using the normally ordered Hamiltonian
\begin{equation}
  \normalorder{H} = \int_{}^{} \bdd[3]{p} \omega_p a^{\dagger}_{\vb{p}} a_{\vb{p}}.
\end{equation}

\subsection{Particles}%
\label{sub:particles}

It is easy to verify the commutation relation with the Hamiltonian
\begin{equation}
  [H, a_{\vb{p}}^{\dagger}] = \omega_p a_{\vb{p}}^{\dagger} \qquad \text{and} \qquad 
  [H, a_{\vb{p}}] = -\omega_p a_{\vb{p}}.
\end{equation}
The operator $a_{\vb{p}}^{\dagger}$ increases the energy by the constant value $\omega_p$.

Now consider the state $\ket{\vb{p}'} = a_{\vb{p}'}^{\dagger} \ket{0}$. The energy of this state is
\begin{align}
  H \ket{p'} &= \int_{}^{} \bdd[3]{p} \omega_p a_{\vb{p}}^{\dagger} a_{\vb{p}} a_{\vb{p}'}^{\dagger} \ket{0}\\
  &= \int_{}^{} \bdd[3]{p} \omega_p a_{\vb{p}}^{\dagger} \left( [a_{\vb{p}}, a_{\vb{p}'}^{\dagger}] - a_{\vb{p}'}^{\dagger} a_{\vb{p}} \right) \ket{0} \\
  &= \omega_{p'} a_{\vb{p}'}^{\dagger} \ket{0} = \omega_{p'} \ket{\vb{p}'}.
\end{align}
Since $\omega_p = \sqrt{p^2 + m^2}$ is the relativistic dispersion relation of a particle of mass $m$ and momentum $p$, we interpret the state  $\ket{\vb{p}}$ as the one-particle momentum eigenstate.

\begin{leftbar}
  \begin{remark}
    Mass $m$ comes from the term $\frac{1}{2} m^2 \phi^2$ in the Lagrangian; in general, the coefficient of the quadratic field term in $\mathcal{L}$ allows us to identify the mass.
  \end{remark}
\end{leftbar}

\subsubsection{Properties of $\ket{\vb{p}}$}%
\label{subsub:properties_of_p_m_}

We previously defined the momentum operator $\vb{P} = -\int_{}^{} \dd[3]{x} \pi(\vb{x}) \grad \phi(\vb{x})$.
\begin{exercise}
  Show that $\vb{P} = \int_{}^{} \bdd[3]{p} \vb{p} a_{\vb{p}}^{\dagger} a_{\vb{p}}$. Then show that $\ket{\vb{p}}$ is actually a momentum eigenstate, i.e.~that $P \ket{\vb{p}} = \vb{p} \ket{\vb{p}}$.
\end{exercise}
We could also act with the angular momentum operator $J^i$ to find $J^i \ket{\vb{p}} = 0$. This is a spin-zero state.
\begin{exercise}
  Do all of Example sheet 1.
\end{exercise}

\subsubsection{Multi-Particle States $\ket{\vb{p}_1, \ldots, \vb{p}_n}$}%
\label{subsub:multi_particle_states}

We define $\ket{\vb{p}_1, \cdots, \vb{p}_n} \coloneq a_{\vb{p}_1} \cdots a_{\vb{p}_n} \ket{0}$.
Since the creation operators $a_{\vb{p}_i}$ commute with each other, we have $\ket{\vb{p}, \vb{q}} = \ket{\vb{q}, \vb{p}}$. These multi-particle states are thus symmetric under interchange of particle label, which means that we are dealing with \emph{bosons}.

The full Hilbert space is spanned by the set
\begin{equation}
  \left\{ \ket{0}, \ket{\vb{p}_1}, \ket{\vb{p}_1, \vb{p}_2}, \cdots \right\}.
\end{equation}
A space that is built from these basis states is called a \emph{Fock space}.

We can also introduce a \emph{number operator}
\begin{equation}
  N = \int_{}^{} \bdd[3]{p} a_{\vb{p}}^{\dagger} a_{\vb{p}},
\end{equation}
which counts the number of particles.
\begin{exercise}
  Show that $N \ket{\vb{p}_1, \cdots, \vb{p}_{n}} = n{\vb{p}_1, \cdots, \vb{p}_{n}}$.
\end{exercise}
Also, the number operator commutes with the Hamiltonian, $[N, H] = 0$, which means that particle number is conserved in the free quantum field theory. This ceased to hold true once we allow interactions between the particles.
