% lecture notes by Umut Özer
% course: qft
\lhead{Lecture 18: November 21}

\begin{table}[htpb]
  \centering
  \begin{tabular}{c c}
  $\psi\psi \to \psi\psi$ &  $\phi$ \\
  $b, c$ & a \\
  $\{b, c\} = 0$ & $[a, a^{\dagger}] = \dots$ \\
  \end{tabular}
  \caption{}
  \label{tab:18-1}
\end{table}

In addition to the commutation relations of Table \ref{tab:18-1}, we can also work out the commutation relations 
\begin{equation}
  [a, b] = 0 = [a, c].
\end{equation}

\section{Momentum Space Feynman Rules for Fermionic Amplitudes}%
\label{sec:momentum_space_feynman_rules_for_fermionic_amplitudes}

%Diagrams:
%\begin{equation}
%  \begin{gathered}
%    \feynmandiagram[inline=(a.base), vertical=e to b] {
%      a -- [fermion,  momentum=\(p\ s\)] b -- [fermion,  momentum=\(p'\ s'\)] c,
%      d -- [fermion] e -- [fermion] f,
%      e -- [scalar] b,
%    };
%  \end{gathered}
%  +
%  \begin{gathered}
%    \feynmandiagram[inline=(a.base), vertical=e to b] {
%      a -- [momentum=\(p\ s\)] b -- [momentum=\(p'\ s'\)] c,
%      d -- e -- f,
%      b -- [scalar] e,
%    };
%  \end{gathered}
%  -
%  \begin{gathered}
%    \feynmandiagram[inline=(a.base), vertical=e to b] {
%      a -- b -- c,
%      d -- e -- f,
%      b -- [scalar] e,
%    };
%  \end{gathered}
%\end{equation}

\begin{itemize}
  \item No \emph{clash} allowed:
    $\qquad \feynmandiagram[inline=(a.base), horizontal=a to b, layered layout] {
        a -- [fermion] b -- [anti fermion] c,
      }; $
  \item Dirac fermions preserve fermion number.
  \item external lines:
    \begin{itemize}
      \item Incoming fermion $ (u^{s}_{\vb{p}})_{\alpha} \sim
	\feynmandiagram[inline=(a.base), horizontal=a to b] {
	  a -- [fermion, momentum=$p\ s$] b [blob],
	}; $
      \item Outgoing fermion $(\overline{u}{}^{s}_{\vb{p}})_{\alpha} \sim
	  \feynmandiagram[inline=(a.base), horizontal=a to b] {
	    a [blob] -- [fermion,  momentum=\(p\ s\)] b,
	  }; $
      \item Incoming anti-fermion $\overline{v}{}^{s}_{\vb{p}}$
      \item Outgoing anti-fermion $v_{\vb{p}}^{s}$
    \end{itemize}
  \item $4$-momentum conservation at each vertex $
    \begin{gathered}
      \feynmandiagram[scale=0.5, transform shape][inline=(c.base), horizontal=a to b] {
        a -- [fermion] c[dot] -- [fermion] b,
        c -- [scalar] d
      };
    \end{gathered}
    \sim (-i \lambda) $
  \item propagator
    $
    \feynmandiagram[inline=(a.base), horizontal=a to b] {
      a [particle=\(\alpha\)] -- [fermion,  momentum=\(p\)] b [particle=\(\beta\)],
    };
    \sim \frac{i (\cancel{p} + m)_{\beta\alpha}}{p^2 - m^2 + i \epsilon}
    $
  \item Indices are contracted in the \emph{opposite direction} to the arrows.
  \item Relative minuses between diagrams that have one particle swapped due to Fermi-Dirac statistics. To be sure of relative minuses, follow the ordering of operators.
  \item In a closed fermionic loop, e.g.
    \begin{equation}
      \feynmandiagram[inline=(a.base), horizontal=a to b, layered layout] {
	a -- [scalar] x [particle=\(x\)] -- [fermion,  momentum=\(k\), half left] y [particle=\(y\)] -- [fermion, half left] x,
	y -- [scalar] b,
      };
      \sim \normalorder{\overline{\psi}_{\alpha} (x) \wick{\c \psi_{\alpha}(x)\c{\overline{\psi}}_{\beta} (y)} \psi_{\beta}(y)}
    \end{equation}
    Contraction is defined on $\wick{\c \psi \c{\overline{\psi}}}$
    \begin{equation}
      = - \normalorder{\wick{\c\psi_{\beta}(y) \c{\overline{\psi}_{\alpha}(x)}} \wick{\c \psi_{\alpha}(x) \c{\overline{\psi}}_{\beta}(y)}}.
    \end{equation}
    Hence there is an additional minus sign for a closed fermionic loop, as well as the usual $\int \dd[4]{k}$ around the loop momenta.
\end{itemize}

What if $\mathcal{L}_{\text{int}} = -\lambda \phi \overline{\psi} _{\alpha} (\gamma^5)_{\alpha\beta} \psi_{\beta}$?
\begin{leftbar}
  \begin{note}
    This only preserves $P$ if $\phi$ is a pseudo-scalar, i.e.~$P\phi (\vb{x}, t) = -\phi(-\vb{x}, {t})$.
  \end{note}
\end{leftbar}

The Feynman rule for the interaction is
\begin{equation}
  \feynmandiagram[inline=(a.base), horizontal=a to b, layered layout] {
    a -- [scalar] b -- [fermion] c [particle=\(\alpha\)],
    b -- [fermion] d [particle=\(\beta\)],
  };
  \sim (-i \lambda) (\gamma^5)_{\alpha\beta}.
\end{equation}

How do we deal with spin and the $\abs{\mathcal{M}}^2$  in the cross-section calculation?

In most experiments, the initial spin states $\ket{i}$  are random. In that case we average over them (e.g.~for $\psi\psi \to \psi\psi$, it would be $\frac{1}{4} \sum_{r, s = 1}^{2}$ ). Typically, the final state particles $\ket{f}$ are not observed.
Any properties that are not observed can / do happen, so we have to sum over all unobserved possibilities in the final state.
\begin{leftbar}
  \begin{note}
    There are particular experiments with polarised beams, where we have some biases. Peskin \& Schroeder \cite{peskin95} deals with this.
  \end{note}
\end{leftbar}

\begin{remark}
  $\mathcal{M} = B - A$ in $\psi\psi \to \psi\psi$, where $B$ \& $A$ are the different terms.
  They can be written as dot products of spinors; the indices $\alpha, \beta$ are summed over.

  We write appropriate spin sums / averages with a bar:
  \begin{equation}
    \overline{\abs{\mathcal{M}}^2} = \overline{\abs{A}^2} + \overline{\abs{B}^2} - \overline{A^{\dagger} B} - \overline{B^{\dagger} A}.
  \end{equation}
  \begin{align}
    A &= \frac{\lambda^2 (\overline{u}^{s}_{p'})_{\alpha} (u^{r}_{r})_{\alpha} (\overline{u}^{r'}_{q'})_{\beta} (u^{s}_{p})_{\beta}}{(u - \mu_2 + i \epsilon)} \\
      &\coloneqq \lambda^2 \frac{[\overline{u}^{s'}_{p'} \cdot u^{r}_{q}][\overline{u}^{r'}_{q'} \cdot u^{s}_{p}]}{(u - \mu^2)},
  \end{align}
  where we leave the $i \epsilon$ as ``understood'' from now on.
\end{remark}

\begin{remark}
  Since $(\gamma^0)^{\dagger} = \gamma^0$ , $[\overline{u}^{s'}_{p} \cdot u^{r}_{q}]^{\dagger} = [\overline{u}^{r}_{q} \cdot u^{s'}_{p'}]$ 
\end{remark}
\begin{align}
  \overline{\abs{A}^2} &= \frac{\abs{\lambda}^4}{4} \frac{\sum_{rss'r'} (\overline{u}{}^{s'}_{p'})_{\alpha} \overbrace{(u^{r}_{q})_{\alpha} (\overline{u}{}^{r}_{q})_{\beta}}^{\mathclap{\text{spin sum over $r$ gives } (\cancel{q} + m)_{\alpha}b}} (u^{s'}_{p'})^{\beta} (\overline{u}{}^{r'}_{q'})_{\gamma} \overbrace{(u^{s}_{p})_{\gamma} (\overline{u}{}^{s}_{p})_{\delta}}^{\mathclap{(\cancel{p} + m)_{\gamma\delta}}} (u^{r}_{q'})_{\delta} }{(u - \mu^2)^2} \\
  &= \frac{\abs{\lambda}^4}{4} \frac{(\cancel{p'} + m)_{\alpha\beta} (\cancel{q} + m)_{\beta\alpha} (\cancel{q'} + m)_{\gamma\delta} (\cancel{p} + m)_{\delta\gamma}}{(u - \mu^2)^2}  \\
  &= \frac{\abs{\lambda}^4}{4} \frac{\Tr[(\cancel{p'} + m)(\cancel{q} + m)] \Tr[(\cancel{q'} + m )(\cancel{p} + m)]}{(u - \mu^2)^2}.
\end{align}
This is why knowing how to deal with traces (PS3) is important. They come out in the spin sums.

Often we are in the high energy limit, where we can neglect particle masses, $\mu, m \to 0$.
 \begin{example}[]
  At the LHC, $\sqrt{s} = 13$TeV $>> m_p = 1$GeV.
\end{example}
In that case, then 
\begin{equation}
  \overline{\abs{A}^2} = \frac{\abs{\lambda}^4}{4 u^2} \Tr(\cancel{p'} \cancel{q}) \Tr(\cancel{q'} \cancel{p}).
\end{equation}
\begin{exercise}
  Check that a similar calculation gives
  \begin{equation}
    \overline{\abs{B}^2} = \frac{\abs{\lambda}^4}{4t^2} \Tr(\cancel{q'} \cancel{q}) \Tr(\cancel{p'} \cancel{p}).
  \end{equation}
\end{exercise}
We also want to know $- \overline{A^{\dagger} B} - \overline{B^{\dagger} A} = -2 \Re \overline{(A^{\dagger} B)}$ .
\begin{align}
  \overline{A^{\dagger} B} &= \frac{\abs{\lambda}^4}{4 ut} \sum_{r r' s s'} (\overline{u}^{r}_{\vb{q}})_{\beta} (u^{s'}_{\vb{p'}})_{\beta} (\overline{u}{}^{s}_{\vb{p}})_{\alpha}(u^{r'}_{\vb{q}'})_{\alpha} (\overline{u}{}^{r'}_{\vb{q}'})_{\gamma} (u^{r}_{\vb{q}})_{\gamma} (\overline{u}{}^{s'}_{\vb{p}'})_{\delta} (u^{s}_{\vb{p}})_{\delta} \\
			   &= \frac{\abs{\lambda}^4}{4ut} \Tr (\cancel{q} \cancel{p'} \cancel{p} \cancel{q'}) \qquad \text{in limit } \mu, m \to 0.
\end{align}

\subsection{Feynman Rules for Spin Summed \texorpdfstring{$\abs{\mathcal{M}}^2$}{M squared} diagrams}%
\label{sub:feynman_rules_for_spin_summed_squared_diagrams}

\begin{itemize}
  \item Complex conjugation switches $\ket{i} \leftrightarrow \ket{f}$ in diagram
  \item Fermion lines are joined with identical momenta on LHS and RHS
  \item After a spin sum, a closed fermion line in the $\abs{\mathcal{M}}^2$ diagram is given by a trace over $\gamma$ matrices, with appropriate $\gamma^5$'s etc in vertices at the correct position in the trace.
\end{itemize}
Trace follows fermion arrows \emph{backwards}.
\begin{example}[]
  Omitting the $s$ labels, we have
  \begin{equation}
    B \sim
    \begin{gathered}
      \feynmandiagram[small, horizontal=a to b] {
        a -- [fermion,  momentum=\(p\)] b -- [fermion,  momentum=\(q'\)] c,
        d -- [fermion,  momentum'=\(q\)] e -- [fermion,  momentum'=\(p'\)] f,
        b -- [scalar] e,
        a -- [draw=none] d,
        c -- [draw=none] f,
      };
    \end{gathered}
    \quad \implies \quad B^{\dagger} \sim
    \begin{gathered}
      \feynmandiagram[small, horizontal=a to b] {
        a -- [fermion,  momentum=\(p'\)] b -- [fermion,  momentum=\(q\)] c,
        d -- [fermion,  momentum'=\(q'\)] e -- [fermion,  momentum'=\(p\)] f,
        b -- [scalar] e,
        a -- [draw=none] d,
        c -- [draw=none] f,
      };
    \end{gathered}
  \end{equation}
  Then the equation $\overline{\abs{\mathcal{M}}^2} = \overline{\abs{B}^2} + \overline{\abs{A}^2} - 2 \Re(\overline{A B^{\dagger}})$ can be expressed diagrammatically as
  \begin{align}
    \overline{\abs{\mathcal{M}}^2} &= 
    \begin{gathered}
      \begin{tikzpicture}[scale=0.8]
	\begin{feynman}
	  \diagram[small, horizontal=a to b] {
	    a -- [fermion,  momentum=\(p\)] b -- [fermion,  momentum=\(p'\)] c,
	    d -- [fermion,  momentum=\(q\)] e -- [fermion,  momentum'=\(q'\)] f,
	    b -- [scalar] e,
	    a -- [draw=none] d,
	    c -- [draw=none] f,
	    c -- [fermion,  momentum=\(p\)] i,
	    f -- [fermion,  momentum=\(q\)] j,
	    c -- [scalar] f,
	    i -- [draw=none] j,
	  };
	\end{feynman}
	%circle in middle
	%\node (v) at ($(a)!0.5!(j)$) {};
	% arcs
	\draw (i) arc (0:181.5:1.9);
	\draw (j) arc (0:-178:1.9);
      \end{tikzpicture}
    \end{gathered}
    \quad + \quad
    \begin{gathered}
      \begin{tikzpicture}[scale=0.8]
	\begin{feynman}
	  \diagram[small, horizontal=a to b] {
	    a -- [fermion,  momentum=\(p\)] b -- [fermion,  momentum=\(q'\)] c,
	    d -- [fermion,  momentum=\(q\)] e -- [fermion,  momentum'=\(p'\)] f,
	    b -- [scalar] e,
	    a -- [draw=none] d,
	    c -- [draw=none] f,
	    c -- [fermion,  momentum=\(p\)] i,
	    f -- [fermion,  momentum=\(q\)] j,
	    c -- [scalar] f,
	    i -- [draw=none] j,
	  };
	\end{feynman}
	%circle in middle
	%\node (v) at ($(a)!0.5!(j)$) {};
	% arcs
	\draw (i) arc (0:181.5:1.9);
	\draw (j) arc (0:-178:1.9);
      \end{tikzpicture}
    \end{gathered}
    \quad - 2 \Re \left(
    \begin{gathered}
      \begin{tikzpicture}[scale=0.8]
	\begin{feynman}
	  \diagram[small, horizontal=a to b] {
	    a -- [fermion,  momentum=\(p\)] b -- [draw=none] c,
	    d -- [fermion,  momentum=\(q\)] e -- [draw=none] f,
	    b -- [scalar] e,
	    a -- [draw=none] d,
	    c -- [draw=none] f,
	    c -- [fermion,  momentum=\(p\)] i,
	    f -- [fermion,  momentum=\(q\)] j,
	    c -- [scalar] f,
	    i -- [draw=none] j,
	  };
	  \diagram* {
	    (b) -- [solid, momentum'=\(q'\)] (f),
	    (e) -- [fermion, momentum'=\(p'\)] (c),
	  };
	\end{feynman}
	%circle in middle
	%\node (v) at ($(a)!0.5!(j)$) {};
	% arcs
	\draw (i) arc (0:181.5:1.9);
	\draw (j) arc (0:-178:1.9);
      \end{tikzpicture}
    \end{gathered} \right) \\
    &= \frac{\abs{\lambda}^4}{4} \left\{ \frac{\Tr(\cancel{q} \cancel{q'}) \Tr(\cancel{p}\cancel{p'})}{t^2} \ +\qquad  \frac{\Tr(\cancel{q'} \cancel{p}) \Tr(\cancel{p'} \cancel{q})}{u^2} \qquad - \qquad \frac{2 \Re \Tr(\cancel{p} \cancel{q'} \cancel{q} \cancel{p'})}{ut} \right\}
  \end{align}
  \begin{remark}
    The minus sign that we started with ended up in the $2 \Re$ term. It has an effect on the differential cross-section---a truly quantum interference effect.
  \end{remark}

  \begin{leftbar}
    \begin{note}
      In the massive case, this simply becomes $\cancel{p} \to (\cancel{p} + \mu)$ for all momenta inside the traces and $t \to (t - \mu^2)$ for all Mandelstam variables in the denominator.
    \end{note}
  \end{leftbar}
\end{example}
