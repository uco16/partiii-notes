% lecture notes by Umut Özer
% course: qft
\lhead{Lecture 10: October 31}

Why does the term
\begin{equation}
  \wick{\c1 \psi^*(x_1) \c2 \psi(x_1) \c2 \psi^*(x_2) \c1 \psi(x_2)}\wick{\c \phi(x_1) \c\phi(x_2)} \bra{f}\ket{i}
\end{equation}
not contribute?
The momenta do not change between $\bra{f}\ket{i}$.
\begin{align}
  M &\equiv \bra{\vb{p'}_1, \vb{p}_2'} \normalorder{\psi^*(x_1) \psi(x_1) \psi^*(x_2) \psi(x_2)} \ket{\vb{p}_1, \vb{p}_2}  \\
    &= \int \frac{\dd[3]{q_1} \dots \dd[3]{q_4}}{\sqrt{2 E_{q_{1} \dots 2 E_{q_4}}}} \bra{0} b_{\vb{p}_1'} \dots \ket{0} e^{i(q_1 \cdot x_1 + q_2 \cdot x_2 - q_3 \cdot x_1 - q_4 \cdot x_2)}
\end{align}
Using the commutation relations, we can show that
\begin{multline}
  \bra{0} b_{\vb{p}_1'} \dots b^{\dagger}_{\vb{p}_2} \ket{0} = [\bdelta^3(\vb{p}_1' - \vb{q}_2) \bdelta^3(\vb{p}_2' - \vb{q}_i) + (p_1 \to p_2) ] \\
  \times [\bdelta^3(\vb{q}_4 - \vb{p}_1) \bdelta^3(\vb{q}_3 - \vb{p}_2) + (p_1 \leftrightarrow p_2)]
\end{multline}
so 
\begin{equation}
  M = \left( e^{i(p_1' \cdot x_2 + p_1' \cdot x_1}  + e^{i(p'_2 \cdot x_2 + p_1' \cdot x_1) }\right) \times \left( e^{-i(p_1 \cdot x_2 + p_2 \cdot x_1)} + e^{-i (p_1 \cdot x_1 + p_2 \cdot x_2)} \right).
\end{equation}
And so 
\begin{align}
  \bra{f} (S-1) \ket{i} &= \frac{(-ig)^2}{2!} \int \dd[4]{x_1} \dd[4]{x_2} \left[ e^{i x_2 \cdot (p_1' - p_1) + i x_1 \cdot (p_2' - p_2)}  \right. \\
			& \left. + e^{i x_2 \cdot (p_2' - p_1) + i x_1 \cdot (p_1' -p_2)} + \underbrace{(x_1 \leftrightarrow x_2)}_{\text{cancels the } \frac{1}{2!}}\right] \\
			& \qquad \times \int \bdd[4]{k} \frac{i e^{ik \cdot (x_2 - x_1)}}{k^2 - m^2 + i \varepsilon} \\
			&= (-ig)^2 \int \bdd[4]{k} \frac{i}{k^2 - m^2 + i \varepsilon} \left[ \bdelta^4(p_1' - p_1 + k) \bdelta^4(p_2' - p_2 - k) \right. \\
			& \qquad \left. + \bdelta^4(p_2' - p_1 + k) \bdelta^4(p_1' - p_2 - k) \right] \\
			&= (-ig)^2 \left\{ \frac{i}{(p_1 - p_1')^2 - m^2 + i \varepsilon} + \frac{i}{(p_2' - p_1)^2 - m^2 + i\varepsilon} \right\} \bdelta^4(p_1 + p_2 - p_1' - p_2'),
			\label{eq:result}
\end{align} 
where we end up with a $4$-momentum preserving $\delta$-function.

\section{Feynman Diagrams}%
\label{sec:feynman_diagrams}

This was a fairly painful process. Doing many of these Wick's theorem calculations, we see that we can make this whole thing more streamlined by introducing Feynman rules.

We draw Feynman diagrams to represent the expansion of the matrix element $\bra{f}(S-1) \ket{i}$. We learn to associate functions to each diagram:
\begin{itemize}
  \item We draw an external line for each particle in $\ket{i}$ and $\ket{f}$, assigning a directional $4$-momentum to each.
  \item For a complex field, we add an arrow to indicate the flow of charge. We choose an ingoing (outgoing) arrow in initial states $\ket{i}$ for particles (anti-particles), and the opposite for final states $\ket{f}$.
  \item We join the lines together with vertices.
    \begin{example}[Scalar Yukawa Theory]
      Figure \ref{fig:sytv} represents the vertex for scalar Yukawa theory.
      \begin{figure}[htbp]
        \centering
        \feynmandiagram [horizontal=a to b] {
          a -- [scalar, edge label=\(\phi\)] b,
	  c [particle=\(\bar \psi\)] -- [fermion, edge label=\(\)] b -- [fermion] d [particle=\(\psi\)],
        };
        \caption{}
        \label{fig:sytv}
      \end{figure}
    \end{example}
    Each such diagram is in one-to one correspondence with terms in $\bra{f}(S-1)\ket{i}$.
\end{itemize}

\begin{example}[]
  For meson scattering $\phi + \phi \to \phi + \phi$, we have diagram expansions such as:
  \begin{equation}
    \label{eq:scat-diag}
    \begin{gathered}
      \feynmandiagram[scale=0.7, transform shape][vertical=e to b] {
	a -- [fermion, edge label=\(\),  momentum=\(p_1\)] b [particle=\(g\)] -- [fermion, edge label=\(\),  momentum=\(p_1'\)] c,
	d -- [fermion, edge label=\(\),  momentum=\(p_2\)] e [particle=\(g\)] -- [fermion, edge label=\(\),  momentum=\(p_2'\)] f,
	b -- [scalar, edge label=\(\phi\), momentum=$k$] e,
	a -- [draw=none] d,
	c -- [draw=none] f,
      };
    \end{gathered}
    +
    \begin{gathered}
      \feynmandiagram[scale=0.7, transform shape][vertical=e to b] {
	a -- [fermion, edge label=\(\),  momentum=\(p_1\)] b [particle=\(g\)] -- [fermion, edge label=\(\),  momentum=\(p_2'\)] f,
	d -- [fermion, edge label=\(\),  momentum=\(p_2\)] e [particle=\(g\)] -- [fermion, edge label=\(\),  momentum=\(p_1'\)] c,
	b -- [scalar, edge label=\(\phi\), momentum=$k$] e,
	a -- [draw=none] d,
	c -- [draw=none] f,
      };
    \end{gathered}
  \end{equation}
\end{example}
At higher orders in $H_I$, there are more complicated diagrams, such as the one-loop diagram to order $O(g^4)$ in Figure \ref{fig:oneloop} or the two-loop diagram to order $O(g^6)$ in Figure \ref{fig:twoloop}.
% Feynman diagrams

\begin{figure}[tbph]
  \begin{subfigure}[t]{0.5\textwidth}
    \centering
    \feynmandiagram [horizontal=a to b] {
      i1 -- [fermion, momentum=$p_1$] a -- [fermion] f1,
      a -- [scalar] b,
      b -- [fermion, half left] c -- [fermion, half left] b,
      c -- [scalar] d,
      i2 -- [fermion] d -- [fermion] f2,
    };
    \caption{``One-loop''}
    \label{fig:oneloop}
  \end{subfigure}
  \begin{subfigure}[t]{0.5\textwidth}
    \centering
    \feynmandiagram[scale=0.93, transform shape][horizontal=a to b] {
      i1 -- [fermion] a -- [fermion] f1,
      a -- [scalar] b,
      b -- [fermion, quarter right] c1 
	-- [fermion, quarter right]  c 
	-- [fermion, quarter right] c2 
	-- [fermion, quarter right] b,
      c1 -- [scalar] c2,
      c -- [scalar] d,
      i2 -- [fermion] d -- [fermion] f2,
    };
    \caption{Two-loop}
    \label{fig:twoloop}
  \end{subfigure}
  \caption{Higher order diagrams}
\end{figure}

\begin{leftbar}
  \begin{remark}
    Time flows from left to right in the conventions used in these lectures.
  \end{remark}
\end{leftbar}

To each diagram, we associate a function using the \emph{Feynman rules}:
\begin{enumerate}
  \item Associate a $4$-momentum to each internal line
  \item Assign a factor $(-ig) \bdelta^4(\sum_i \vb{k}_i)$ to each vertex, where $\sum_i \vb{k}_i$ is the sum of $4$-momenta flowing \emph{into} the vertex.
  \item For each internal line with $4$-momentum $\vb{k}$, write a factor $\int \bdd[4]{k} D(k^2)$, where the propagator is
    \begin{equation}
      D(k^2) = 
      \begin{cases}
	\frac{i}{k^2 - m^2 + i\varepsilon} & \text{for } \phi \\
	\frac{i}{k^2 - \mu^2 + i \varepsilon} & \text{for } \psi
      \end{cases}.
    \end{equation}
\end{enumerate}

\begin{example}[$\psi\psi$ scattering revisisted]
  We use the Feynman diagrams depicted in Equation \eqref{eq:scat-diag}. Applying the Feynman rules, these give
  \begin{align}
    &= (-ig)^2 \int \bdd[4]{k} \frac{i}{k^2 - m^2 \pm i\varepsilon}  \bdelta^4(p_1 - p_1' - k) \bdelta^4(p_2 + k - p_2')+ \bdelta^4(p_1 - p_2' - k) \bdelta^4(p_2 + k - p_1') \\
    &= i (-ig)^2 \left\{ \frac{1}{(p_1 - p_1')^2 - m^2 + i \varepsilon} + \frac{1}{(p_1 - p_2')^2 - m^2 + i \varepsilon} \right\} \bdelta^4(p_1 + p_2 - p_1' - p_2').
  \end{align}
  This is the same as our previous result of Equation \eqref{eq:result}.
\end{example}
\begin{leftbar}
  \begin{remark}
    The meson does not necessarily satisfy $k^2 - m^2$---if it does not, it is called \emph{off-shell} or a \emph{virtual} particle. This is a quantum effect.
    The second diagram enforces \emph{Bose statistics} for the identical final state particles.
  \end{remark}
\end{leftbar}

\begin{definition}[]
  We will define the \emph{matrix element} $\mathcal{M}$ by 
  \begin{equation}
    \bra{f} (S-1) \ket{i} = i \mathcal{M} + \bdelta^4 (\sum_{\text{final state particles } i} p_i - \sum_{\text{initial state pts } i} p_i),
  \end{equation}
  where the factor of $i$ is a convention which was designed to match non-relativistic quantum mechanics. The $\bdelta^4$-function follows from translation invariance and is common to all $S$-matrix elements.
\end{definition}

We now refine the Feynman rules to compute $i \mathcal{M}$:
\begin{enumerate}
  \item Draw all possible diagrams with appropriate external lefts and impose 4-momentum conservation at each vertex.
  \item Write a factor $(-ig)$ at each vertex.
  \item For each internal line, factor in the propagator
  \item Integrate over momentum $k$ flowing in any closed loop as $\int \bdd[4]{k}$.
\end{enumerate}

\begin{figure}[htbp]
  \centering
  \feynmandiagram [horizontal=a to b] {
    i1 -- [scalar, momentum=\(p_1\)] a -- [fermion, momentum=\(k + p_1'\)] b -- [scalar, momentum=$p_1'$] f1,
    i2 -- [scalar,  momentum=\(p_2\)] c -- [anti fermion, rmomentum'=\(k - p_2'\)] d -- [scalar, momentum=$p_2'$] f2,
    b -- [fermion, momentum=$k$] d,
    c -- [fermion, momentum=$k + p_1' -p_1$] a,
  };
  \caption{}
  \label{fig:abc}
\end{figure}

\begin{example}[$\phi \phi \to \phi \phi$]
  %Diagram
  The diagram in Figure \ref{fig:abc} gives
  \begin{equation}
    i \mathcal{M} = \int \frac{(-ig)^4 \bdd[4]{k} i^4}{(k^2 - \mu^2 + i\varepsilon) ((k + p'_2)^2 - \mu^2 + i \varepsilon) ((k + p_1' - p_1)^2 - \mu^2 + i\varepsilon)((k+p_1')^2 - \mu^2 + i\varepsilon)}.
  \end{equation}
  Amazingly, we can do these integrals!
  All the one-loop diagrams in four dimensions have been done, however, the two-loop diagrams are still subject to active research.
\end{example}
