% lecture notes by Umut Özer
% course: qft
\lhead{Lecture 16: November 16}

\begin{description}
  \item[Lorentz Transformations] $x \to \Lambda x$ \par
    The Dirac spinor transforms as
    $\psi^{\alpha} \to S[\Lambda] \indices{^{\alpha}_{\beta}} \psi^{\beta} (x^{\mu} - \omega \indices{^{\mu}_{\nu}} x^{\nu})$, where $\omega \indices{^{\mu}_{\nu}} = \frac{1}{2} \Omega_{\rho\sigma} (M^{\rho\sigma}) \indices{^{\mu}_{\nu}}$ and $(M^{\sigma\rho})\indices{^{\mu}^{\nu}}$ are the generators of the Lorentz algebra, given by \eqref{eq:13-lorentz-generators}.
    Substituting these into the above, we have $\omega^{\mu\nu} = \Omega^{\mu\nu}$, which means that under infinitesimal Lorentz transformation we have
    \begin{align}
      \delta \psi^{\alpha} &= - \omega \indices{^{\mu}_{\nu}} x^{\nu} \partial_{\mu} \psi^{\alpha} + \frac{1}{2} \Omega_{\rho\sigma} (S^{\rho\sigma}) \indices{^{\alpha}_{\beta}} \psi^{\beta} \\
			   &= - \omega^{\mu\nu} \left[ x_{\nu} \partial_{\mu} \psi_{\alpha} - \frac{1}{2} (S_{\mu\nu}) \indices{^{\alpha}_{\beta}} \psi^{\beta} \right] \\
      \delta \overline{\psi}_{\alpha} &= -\omega^{\mu\nu} \left[ x_{\nu} \partial_{\mu} \overline{\psi}_{\alpha} + \frac{1}{2} \overline{\psi}_{\beta} (S_{\mu\nu}) \indices{^{\beta}_{\alpha}} \right].
    \end{align}
    Again, the spinor equations of motion set $\mathcal{L} = 0$ and the conserved current can be calculated to be
    \begin{equation}
      (\mathcal{J}^{\mu})^{\rho\sigma} = \underbrace{x^{\rho} T^{\mu\sigma} - x^{\sigma} T^{\mu\rho}}_{\text{looks like } \mathbb{R} \text{ scalar}} -i \overline{\psi} \gamma^{\mu} S^{\rho\sigma} \psi.
    \end{equation}
    After quantisation, the last term will give rise to the internal angular momentum of spin-$\frac{1}{2}$ single particle states.
    \begin{equation}
      (\mathcal{J}^0)^{ij} = -i \overline{\psi} \gamma^0 S^{ij} \psi = \frac{1}{2} \varepsilon^{ijk} \psi^{\dagger}
      \begin{pmatrix}
       \sigma^{k} & 0 \\
       0 & \sigma^{k} \\
      \end{pmatrix} \psi
    \end{equation}
  \item[Internal Vector Symmetry]
    The Dirac Lagrangian is symmetric under rotation of the spinor phase
    \begin{equation}
      \psi \to e^{i\alpha} \psi \quad \implies \quad \delta \psi = i \alpha \psi
    \end{equation}
    By Noether's theorem, this implies the existence of the conserved current $j^{\mu}_V = \overline{\psi} \eta^{\mu} \psi$.
    Here, the index $V$ for \emph{vector} emphasises the fact that $\psi_L$ and $\psi_R$ transform the same way under this symmetry.
    The associated conserved charge is 
    \begin{equation}
      Q = \int \dd[3]{x} \overline{\psi} \gamma^0 \psi = \int \dd[3]{x} \psi^{\dagger} \psi.
    \end{equation}
    As we will see soon, this will be interpreted as electric charge.
  \item[Axial Symmetry]
    When $m = 0$, the Dirac Lagrangian is invariant under the axial symmetry transformation
    \begin{equation}
      \psi_{\alpha} \to (e^{i\alpha \gamma^5}) \indices{_{\alpha}^{\beta}} \psi_{\beta},
    \end{equation}
    which rotates left-handed and right-handed spinors in opposite directions. We have a conserved axial vector current $j^{\mu}_{A} = \overline{\psi} \gamma^{\mu} \gamma^5 \psi$.
    \begin{leftbar}
      \begin{remark}
	Once we couple this theory to gauge fields, we will see that this is the first example of an \emph{anomaly}---a symmetry that is present in the classical theory, but ceases to hold once we perform quantisation.
      \end{remark}
    \end{leftbar}
\end{description}

\subsection{Plane Wave Solutions}%
\label{sub:plane_wave_solutions}

\subsection*{Positve Frequency}%

We want to find solutions to the Dirac equation $(i \cancel{\partial} - m) \psi = 0$. Consider the plane wave ansatz $\psi = u_{\vb{p}} e^{-i p \cdot x}$, where $u_{\vb{p}}$ is a constant spinor depending on the three-momentum $\vb{p}$.
Substituting this into the Dirac equation (using the chiral representation of $\gamma^{\mu}$), we have
\begin{equation}
  \label{eq:16-star}
  (\gamma^{\mu} p_{\mu} - m) u_{\vb{p}} =  
  \begin{pmatrix}
   -m & p_{\mu}\sigma^{\mu} \\
   p_{\mu} \overline{\sigma}^{\mu} & -m \\
  \end{pmatrix}
  u_{\vb{p}} = 0.
\end{equation}
We have again used the definition that $\sigma^{\mu} = (1, \sigma^{i})$ and $\overline{\sigma}^{\mu} = (1, - \sigma^{i})$.

\begin{claim}
  The solution is $u_{\vb{p}} = 
  \begin{pmatrix}
  \sqrt{p \cdot \sigma} \xi \\
  \sqrt{p \cdot \overline{\sigma}} \xi \\
  \end{pmatrix}$ for any two-component spinor $\xi$, which is normalised such that $\xi^{\dagger} \xi = 1$.
\end{claim}
\begin{proof}
  Let us write $u_{\vb{p}} = 
  \begin{pmatrix}
  u^1 \\
  u^2\\
  \end{pmatrix}$, and sub into \eqref{eq:16-star}:
  \begin{subequations}
    \label{eq:16-doublestar}
    \begin{align}
      (p \cdot \sigma) u_2 &= m u_1 \\
      (p \cdot \overline{\sigma}) u_1 &= m u_2.
    \end{align}
  \end{subequations}
  Either of these implies the other, since
  \begin{align}
    (p \cdot \sigma) (p \cdot \overline{\sigma}) &= p_0^2 - p_{i} p_{j} \sigma^{i} \sigma^{j} \\
						 &= p_0^2 - p_{i} p_{j} \underbrace{\frac{1}{2} \left\{ \sigma^{i}, \sigma^{j} \right\}}_{\delta^{ij}} \\
						 &= p_{\mu} p^{\mu} = m^2.
  \end{align}
  Now we try the ansatz $u_1 = (p \cdot \sigma) \xi'$ for a two-component spinor $\xi'$. Sub into \eqref{eq:16-doublestar} to give $u_2 = \frac{1}{m} (p \cdot \overline{\sigma}) (p \cdot \sigma) \xi' = m \xi'$. So any vector of form $u_{\vb{p}} = A
  \begin{pmatrix}
    (p \cdot \sigma) \xi' \\
    m \xi' \\
  \end{pmatrix}
  $, where $A$ is constant is a solution.
  To make this look more symmetric, choose $A = \frac{1}{m}$ and $\xi' \coloneqq \sqrt{p \cdot \overline{\sigma}}\xi$, with $\xi$ constant. Then $u_{\vb{p}} = 
  \begin{pmatrix}
  \sqrt{p \cdot \sigma} \xi \\
  \sqrt{p \cdot \overline{\sigma}} \xi \\
  \end{pmatrix}
  $.
\end{proof}

\begin{example}[$\vb{p} = 0$]
  Have $u_{\vb{p} = 0} = \sqrt{m}
  \begin{pmatrix}
  \xi \\
  \xi \\
  \end{pmatrix}
  $ for any $\xi$. Under spatial rotations, $\xi \to e^{i \boldsymbol\sigma \cdot \boldsymbol \phi / 2} \xi$. After quantisation, $\xi$ describes the \emph{spin} of the spinor, e.g.~$\xi = 
  \begin{pmatrix}
  1 \\
  0 \\
  \end{pmatrix}
  $ is spin-$\uparrow$ along $x^3$-axis: to see this, consider the particle boosted along the $x^3$-direction.
  \begin{equation} \label{eq:16-m1}
    u_{\vb{p}} =
    \begin{pmatrix}
    \sqrt{E - p^3}
    \begin{pmatrix}
    1 \\
    0 \\
    \end{pmatrix}
    \\
    \sqrt{E + p^3}
    \begin{pmatrix}
    1 \\
    0 \\
    \end{pmatrix}
    \\
    \end{pmatrix}
    \xrightarrow[E \to p^3]{m \to 0} \sqrt{2 E} 
    \begin{pmatrix}
    0 \\
    0 \\
    1 \\
    0 \\
    \end{pmatrix}
  \end{equation}
  For $\xi = 
  \begin{pmatrix}
  0 \\
  1 \\
  \end{pmatrix}
  $, 
  \begin{equation}
    u_{\vb{p}} = 
    \begin{pmatrix}
    \sqrt{E + p^3}
    \begin{pmatrix}
    0 \\
    1 \\
    \end{pmatrix}
    \\
    \sqrt{E - p^3}
    \begin{pmatrix}
    0 \\
    1 \\
    \end{pmatrix}
    \\
    \end{pmatrix}
    \xrightarrow{m \to 0}
    \begin{pmatrix}
    0 \\
    1 \\
    0 \\
    0 \\
    \end{pmatrix} \sqrt{2 E}.
    \label{eq:16-m2}
  \end{equation}
\end{example}

\begin{definition}[helicity]
  The \emph{helicity} operator $h$ projects the angular momentum along the direction of motion.
  \begin{equation}
    h = \hat{\vb{p}} \cdot \vb{s} = \frac{1}{2} \hat p_{k}
    \begin{pmatrix}
     \sigma^{k} & 0 \\
     0 & \sigma^{k} \\
    \end{pmatrix}.
  \end{equation}
\end{definition}
A massless spin-$\uparrow$ operator has $h = + \frac{1}{2}$, whereas a massless spin-$\downarrow$ has $h = -\frac{1}{2}$.

\subsection*{Negative Frequency}%

The Dirac equation also admits negative frequency solution of the form $\psi = v_{\vb{p}} e^{+i p \cdot x}$ with
\begin{equation}
  v_{\vb{p}} = 
  \begin{pmatrix}
  \sqrt{p \cdot \sigma} \xi \\
  - \sqrt{p \cdot \overline{\sigma}} \xi \\
  \end{pmatrix},
\end{equation}
where $\xi^{\dagger} \xi = 1$ and $\xi$ is a constant two-spinor.
\begin{exercise}
  Go through the same derivation as above, but with the negative frequency ansatz.
\end{exercise}

\subsection{Quantising the Dirac Field}%
\label{sub:quantising_the_dirac_field}

In the Schrödinger picture, where the fields are not functions of time, we have
\begin{align}
  \psi(\vb{x}) &= \sum_{s=1}^{2} \int \frac{\bdd[3]{p}}{\sqrt{2E_{p}}} \left[ b_{\vb{p}}^{s} u^{s}_{\vb{p}} e^{+i \vb{p} \cdot \vb{x}} + (c_{\vb{p}}^{s})^{\dagger} v_{\vb{p}}^{s} e^{-i \vb{p} \cdot \vb{x}} \right] \\
  \psi^{\dagger}(\vb{x}) &= \sum_{s=1}^{2} \int \frac{\bdd[3]{p}}{\sqrt{2E_{p}}} \left[ (b_{\vb{p}}^{s})^{\dagger} (u^{s}_{\vb{p}})^{\dagger} e^{-i \vb{p} \cdot \vb{x}} + c_{\vb{p}}^{s} (v_{\vb{p}}^{s})^{\dagger} e^{+i \vb{p} \cdot \vb{x}} \right] \\
\end{align}
In the Heisenberg picture, we specify these at an initial time and use the Dirac equation to determine the full evolution.

At this stage, we would usually specify commutation relations. However, something pathological would go wrong if we did that.
It turns out that in spinor quantisation, we require \emph{anti-}commutation relations, $\left\{ A, B \right\} = AB + BA$, such as those we have already seen in the Clifford algebra.
We have
\begin{subequations} \label{eq:16-anticommrels}
  \begin{gather}
    \left\{ \psi_{\alpha}(\vb{x}), \psi_{\beta}(\vb{y}) \right\} = 0 = \left\{ \psi^{\dagger}_{\alpha}(\vb{x}), \psi^{\dagger}_{\beta}(\vb{y}) \right\} \\
    \left\{ \psi_{\alpha}(\vb{x}) , \psi^{\dagger}_{\beta}(\vb{y}) \right\} = \delta_{\alpha\beta} \delta^3 (\vb{x} - \vb{y}).
  \end{gather}
\end{subequations}
\begin{claim}
  These are equivalent to
  \begin{equation}
    \left\{ b_{\vb{p}}^{r}, (b^{s}_{\vb{q}})^{\dagger} \right\} = \bdelta^3(\vb{p} - \vb{q}) \delta^{rs} = \left\{ c^{r}_{\vb{p}}, (c_{\vb{q}}^{s})^{\dagger} \right\}
  \end{equation}
  and all other anti-commutators vanish.
\end{claim}

\subsection*{Hamiltonian}%

To show why we need to specify \emph{anti-}commutation relations, consider the Hamiltonian
\begin{align}
  \mathcal{H} &= \pi \dot{\psi} - \mathcal{L} \\
 &= \cancel{i \psi^{\dagger} \dot{\psi}} - \cancel{i \overline{\psi} \gamma^0 \partial_0 \psi} - i \overline{\psi} \gamma^{i} \partial_{i} \psi + m \overline{\psi} \psi \\
 &= \overline{\psi} (-i \gamma^{i} \partial_{i} + m) \psi
\end{align}
We then plug in $\psi, \overline{\psi}$ from above and use anti-commutation relations and some identities on the inner product of spinors, namely
\begin{subequations}
  \begin{align}
    (u_{\vb{p}}^{r})^{\dagger} u_{\vb{p}}^{s} &= (v_{\vb{p}}^{r})^{\dagger} v_{\vb{p}}^{s} = 2 p_0 \delta^{rs}, \\
    (u_{\vb{p}}^{r})^{\dagger} v_{\vb{p}}^{s} &= 0 = (v_{\vb{p}}^{r})^{\dagger} u_{\vb{p}}^{s},
  \end{align}
\end{subequations}
to obtain
\begin{equation}
  H = \int \bdd[3]{p} E_p \sum_{s=1}^{2} \left( (b_{\vb{p}}^{s})^{\dagger} b^{s}_{\vb{p}} + (c_{\vb{p}}^{s})^{\dagger} c^{s}_{\vb{p}} \right).
\end{equation}
This is where the difference between commutation and anti-commutation relations comes in; had we specified commutation relations instead of the anti-commutation relations \eqref{eq:16-anticommrels}, then we would have a minus sign in this Hamiltonian: $H= \dots b_{\vb{p}}^{s} - (c_{\vb{p}}^{s})^{\dagger} \dots$.
This would then mean that we can lower the energy indefinitely by creating anti-particles, thus making the theory unstable.
This is a first glimpse of how the spin-statistics theorem follows from QFT: scalars and vectors obey commutation relations, whereas spinors obey anti-commutation relations.
