% lecture notes by Umut Özer
% course: qft
\lhead{Lecture 20: November 28}

\section{Quantisation of the electromagnetic field}%
\label{sec:quantisation_of_the_electromagnetic_field}

Following the standard quantisation procedure, we first calculate the canonical conjugate momenta
\begin{align}
  \pi^0 &= \frac{\partial \mathcal{L}}{\partial \dot{A}_0} = 0 \\
  \pi^1 &= \frac{\partial \mathcal{L}}{\partial \dot{A}_{i}} = - \dot{A}^{i} + \partial^{i} A^0 = F^{i0} = +E^{i}
\end{align}
Meaning that the Hamiltonian is
\begin{align}
  H &= \int \dd[3]{x} \left( \pi^i \dot{A}_{i} - \mathcal{L} \right) \\
    &= \int \dd[3]{x} \frac{1}{2} \left( \abs{\vb{E}}^2 + \abs{\vb{B}}^2 \right) - A_0 (\grad \cdot \vb{E})
\end{align}
Note that the $A_0$ acts as a Lagrange multiplier, imposing $\grad \cdot \vb{E} = 0$.

Now a constraint of the system, treating $\vb{A}$ as the physical degree of freedom.

\subsection*{Lorentz Gauge}%

Imposing $\partial_{\mu} A^{\mu} = 0$ means that Maxwell's equations $\partial_{\mu} F^{\mu\nu}$ become 
\begin{equation}
  \label{eq:20-ddagger}
  \partial_{\mu} \partial^{\mu} A^{\nu} = 0.
\end{equation}
We now construct a different theory, where this comes directly from the Euler-Lagrange equations. This is given by the new Lagrangian
\begin{equation}
  \mathcal{L} = - \frac{1}{4} F_{\mu\nu} F^{\mu\nu} - \frac{1}{2} (\partial_{\mu} A^{\mu})^2.
\end{equation}
\begin{remark}
  This additional term spoils gauge invariance.
\end{remark}
The Euler-Lagrange equations are
\begin{equation}
  \partial_{\mu} F^{\mu\nu} + \partial^{\nu}(\partial_{\mu} A^{\mu}) = 0 \quad \iff \quad \partial_{\mu} \partial^{\mu} A^{\nu} = 0,
\end{equation}
which is \eqref{eq:20-ddagger}.

We now work with this new Lagrangian, only imposing $\partial_{\mu} A^{\mu} = 0$ later, at the operator level.
In general, $\mathcal{L} = - \frac{1}{4} F_{\mu\nu} F^{\mu\nu} - \frac{1}{2\alpha} (\partial_{\mu} A^{\mu})^2$.
 This gives us a continuum of different theories, which are different gauges.
\begin{description}
  \item[$\alpha = 1$] is called the \emph{Feynman gauge}.
  \item[$\alpha \to 0$] is the \emph{Landau gauge}.
\end{description}

Sometimes, people leave in the $\alpha$  in its most general form; any physical observable should be independent of this.
This is a bit too complicated for our purposes so we will work in the Feynman gauge $\alpha = 1$ for now.

The new theory has no gauge symmetry, but now $A_0$  and the spatial components $\vb{A}$  are dynamical:
\begin{equation}
  \pi^0 = \frac{\partial \mathcal{L}}{\partial \dot{A}_0} = - \partial_{\mu} A^{\mu}, \qquad \pi^i = \frac{\partial \mathcal{L}}{\partial \dot{A}_{i}} = - \dot{A}^{i} + \partial^{i} A^0
\end{equation}
When we consider the operators, we will set this extra piece to zero to get a physical limit.

Working in the Schrödinger picture, the commutators are

\begin{subequations}
  \label{eq:20-comms}
  \begin{equation}
    [A_{\mu}(\vb{x}), A_{\nu}(\vb{y})] = 0 = [\pi_{\mu}(\vb{x}), \pi_{\nu}(\vb{y})]
  \end{equation}
  \begin{equation}
    [A_{\mu}(\vb{x}), \pi_{\nu}(\vb{y})] = i \delta^3 (\vb{x} - \vb{y}) \eta_{\mu\nu}.
  \end{equation}
\end{subequations}

If we expand $A_{\mu}(\vb{x})$, we have, in the Schrödinger picture, 
\begin{equation}
  \label{eq:20-A}
  A_{\mu}(\vb{x}) = \int \frac{\bdd[3]{p}}{\sqrt{2 \abs{\vb{p}}}} \sum_{\lambda = 0}^{3} \left( \varepsilon_{\mu}^{\lambda} (\vb{p}) a_{\vb{p}}^{\lambda} e^{i \vb{p} \cdot \vb{x}} + (\varepsilon_{\mu}^{\lambda})^*(\vb{p}) (a_{\vb{p}}^{\lambda})^{\dagger} e^{-i \vb{p} \cdot \vb{x}} \right).
\end{equation}
where $\varepsilon_{\mu}^{\lambda = 0,1,2,3}$ are four polarisation $4$-vectors.
The energy in the photon, because it is massless, is just its momentum $E_{\vb{p}} = \abs{\vb{p}}$.
Similarly, its canonical momentum is
\begin{equation}
  \pi^{\mu}(\vb{x}) = \int \frac{(+i) \sqrt{\abs{\vb{p}}} \bdd[3]{p}}{\sqrt{2}} 
  \sum_{\lambda = 0}^{3} 
  \left( [\varepsilon^{\mu} _{\vb{p}}]^{\lambda} a_{\vb{p}}^{\lambda} e^{i \vb{p} \cdot \vb{x}} 
  -(a_{\vb{p}}^{\lambda})^{\dagger} e^{-i \vb{p} \cdot \vb{x}} [(\varepsilon^{\mu} _{\vb{p}})^*]^\lambda \right).
\end{equation}

We pick $\varepsilon^0_{\mu}$  to be time-like and $\varepsilon^{i}_{\mu}$  to be space-like, with normalisation $ \varepsilon^\lambda \cdot (\varepsilon^*)^{\lambda'} = \eta^{\lambda \lambda'}$.
We choose $\varepsilon^{1\mu}$  and $\varepsilon^{2 \mu}$  to be the transverse polarisation, meaning that $\varepsilon^1 \cdot p = 0 = \varepsilon^2 \cdot p$, and $\varepsilon^{3 \mu}$ to be the longitudinal polarisation.
\begin{example}[]
  Consider a photon $\gamma$ travelling in the $x^3$-direction:
  \begin{equation}
    p^\mu = \abs{\vb{p}} 
    \begin{pmatrix}
    1 \\
    0 \\
    0 \\
    1 \\
    \end{pmatrix}
    , \quad \varepsilon^{\rho\mu} = 
    \begin{pmatrix}
    1 \\
    0 \\
    0 \\
    0 \\
    \end{pmatrix}, \quad \varepsilon^{1\mu} = 
    \begin{pmatrix}
    0 \\
    1 \\
    0 \\
    0 \\
    \end{pmatrix}, \quad \varepsilon^{2\mu} = 
    \begin{pmatrix}
    0 \\
    0 \\
    1 \\
    0 \\
    \end{pmatrix}, \quad \varepsilon^{3\mu} = 
    \begin{pmatrix}
    0 \\
    0 \\
    0 \\
    1 \\
    \end{pmatrix}.
  \end{equation}
  For the other momenta, just do a rotation.
\end{example}

The commutation relations \eqref{eq:20-comms} then correspond to 
\begin{equation}
  [a_{\vb{p}}^{\lambda}, a_{\vb{q}}^{\lambda'}] = 0  = [a_{\vb{p}}^\lambda{}^{\dagger}, a_{\vb{q}}^{\lambda'}{}^{\dagger}]
\end{equation}
\begin{equation}
  [a_{\vb{p}}^{\lambda}, a_{\vb{q}}^{\lambda'}{}^{\dagger}] = - \eta^{\lambda \lambda'} \bdelta^3(\vb{p} - \vb{q}).
\end{equation}
This strange minus sign is OK for $\lambda = \lambda' = 1 , 2, 3$, but unusual for timelike photons.

As usual, the annihilation operator annihilates the vacuum $a_{\vb{p}}^{\lambda} \ket{0} = 0$ .
We describe a state with momentum $\vb{p}$  and polarisation $\lambda$ as  $\ket{\vb{p}, \lambda} = a_{\vb{p}}^{\lambda}{}^{\dagger} \ket{0}$ . This is fine for spacelike photons $\lambda = 1, 2, 3$, but when we consider timelike ones, the minus comes back to haunt us:
 \begin{equation}
  \bra{\vb{p}, \lambda = 0}\ket{\vb{q}, \lambda = 0} = \bra{0} a_{\vb{p}}^0 a_{\vb{q}}^0{}^{\dagger} \ket{} = - \bdelta^3(\vb{p} - \vb{q}).
\end{equation}
We have a negatively normed state!  
In general, a Hilbert space with a negative norm state is problematic, but in fact the constraint equation from our gauge comes to the rescue.

It will take us three approaches, each one becoming successively less naive.

Let us work in the Heisenberg picture.
\begin{enumerate}[]
  \item If we naively try to impose $\partial_{\mu} A^{\mu} = 0$, this does not work since $\pi^0 = -\partial_{\mu} A^{\mu}$ , so the commutation relations could not possibly be obeyed.
  \item We might instead try to impose the condition on the Hilbert space:
    We can split the states into `physical states' $\ket{\psi}$  and `others', which we hope will decouple:
    \begin{equation}
      \label{eq:20-star}
      \partial_{\mu} A^{\mu} \ket{\psi} = 0.
    \end{equation}
    It turns out that this condition will be too strong.

    Let us split the integral \eqref{eq:20-A} into two fields
    \begin{align}
      A_{\mu}({x}) &= \int \frac{\bdd[3]{p}}{ \sqrt{2 \abs{\vb{p}}}} \sum_{\lambda} \varepsilon_{\mu}^{\lambda} a_{\vb{p}} ^{\lambda} e^{-i {p} \cdot {x}}
      + \int \frac{\bdd[3]{p}}{\sqrt{2 \abs{\vb{p}}}} \sum_\lambda \varepsilon_{\mu}^{\lambda}{}^* a_{\vb{p}}^{\lambda}{}^{\dagger} e^{i {p} \cdot {x}} \\
		      &\coloneqq A_{\mu}^{\dagger}(x) + A_{\mu}^- (x)
    \end{align}
    Now we have $A_{\mu}^+ \ket{0} = 0$ automatically, but $\partial^{\mu} A_{\mu}^- \ket{0} \neq 0$ and therefore $\ket{0}$ is not physical!
    So this approach does not work either.

  \item Final attempt: define physical states $\ket{\psi}$ by 
    \begin{equation}
      \label{eq:20-phi}
      \partial^{\mu} A^{\dagger}_{\mu} (x) \ket{\psi} = 0
    \end{equation}
    Then we have $\bra{\psi'} \partial_{\mu} A^{\mu} \ket{\psi} = 0$ for physical states $\ket{\psi}$ and $\ket{\psi'}$. 
    Equation \eqref{eq:20-phi} is the \emph{Gupta-Bleuler condition}. Its linearity means that $\left\{ \ket{\psi} \right\}$ span a Hilbert space $\mathcal{H}_{\text{phys}}$.
\end{enumerate}

We decompose a generic state $\ket{\Psi} = \ket{\psi_I} \ket{\phi}$. The $\ket{\psi_T}$ are transverse photons with $a_{\vb{k}}^{1, 2}{}^{\dagger}$, whereas $\ket{\phi}$ are time-like photons with $a_{\vb{k}}^0{}^{\dagger}$ and longitudinal photons with $a_{\vb{k}}^{3}{}^{\dagger}$.
So now, $\partial_{\mu} (A^+)^{\mu} \ket{\Psi} =0$ gives (in two or three lines) that
\begin{equation}
  \label{eq:20-doubstar}
  (a_{\vb{k}}^3 - a^0_{\vb{k}})\ket{\phi} = 0.
\end{equation}
This means that physical states only contain timelike-longitudinal (TL) $\psi$-pairs.
So in general, 
\begin{equation}
  \ket{\phi} = \sum_{n = 0}^{\infty} c_n \ket{\phi_n},
\end{equation}
where $\ket{\phi_n}$ are $n$ TL-pairs.

Equation \eqref{eq:20-doubstar} implies that $\bra{\phi_m}\ket{\phi_n} = \delta_{m0} \delta_{n0}$. 
States with TL pairs have zero norm. These zero norm states are treated as an equivalence class: states that differ only by TL pairs are physically equivalent.
