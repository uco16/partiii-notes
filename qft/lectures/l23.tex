% lecture notes by Umut Özer
% course: qft
\lhead{Lecture 23: December 05}

\chapter{Summary}%

\section*{Classical Field Theory}%

\begin{itemize}
  \item Symmetries of $S$, Noether's theorem $\implies$ currents and charges
  \item EL equations of motion, Lorentz invariance
\end{itemize}

\section*{Canonical Quantisation}%

\begin{itemize}
  \item expand $\phi \sim a, a^{\dagger}$ (SHO for each momentum)
    \begin{itemize}
      \item second quantisation
    \end{itemize}
  \item CRs between fields / operators
  \item Physical things are $\bra{ \dots } \dots \phi \dots \ket{ \dots }$, same in all pictures
  \item Normal ordering introduced to get sensible (non-$\infty$) def\textsuperscript{n} of observables
  \item Time ordering $\rightarrow \bra{0} T \phi(x) \phi(y) \ket{0}$ propagator
    \begin{itemize}
      \item building blocks of theory
    \end{itemize}
\end{itemize}

\section*{Interacting Picture}%

\begin{equation}
  \bra{f}(S- 1)\ket{i} = \lim_{\substack{t \to +\infty \\ t_0 \to - \infty}} \bra{f} U(t, t_0) \ket{i}
\end{equation}
\begin{itemize}
  \item time evolution operator $U(t, t_0) = T\exp{-i \int_{t_0}^{t} \dd[]{t} H_I(t')} = T \exp{+i \int \dd[4]{x} \mathcal{L}_I}$ 
  \item Wick's theorem was important step in providing connection between QFT and Feynman rules \& diagrams.
  \item e.g. $\phi^4$ theory: $\bra{0} T (\phi_1 \dots \phi_4 \phi_x^4) \ket{0}$
    \begin{itemize}
      \item get diagrams, but only keep first piece
      \item throw away vacuum bubbles because fully interacting vacuum absorbs them
      \item disconnected pieces also get thrown away (LSZ reduction formula)
    \end{itemize}
\end{itemize}

\section*{Four Spinors}%

\begin{itemize}
  \item A new rep. of Lorentz group
\end{itemize}
If a field transformas a spinor, fields have to anti-commute in order to make sense.\par
$\implies$ anti-commuting creation / annihilation operators

\section*{Spinor cross-sections}%

Time / normal ordering of operators gives additional minus signs for fermions
\begin{equation}
  \left\{ \psi, \overline{\psi} \right\} \neq 0 \qquad \text{but} \qquad 
  \begin{gathered}
    \left\{ \psi_e, \overline{\psi}_{\mu} \right\} = 0 \\
    \left\{ \psi_e, \overline{\psi}_{e} \right\} \neq 0 \\
    \left\{ \psi_{\mu}, \overline{\psi}_{\mu} \right\} \neq 0
  \end{gathered}
\end{equation}

\begin{itemize}
  \item Spin sums give us traces of $\gamma$ matrices (see exercise sheet 3).
  \item Parity, chiral fermions $\implies \gamma^5 = i \gamma^0 \gamma^1 \gamma^2 \gamma^3$
\end{itemize}

\section*{QED}%

Gauge invariance 
\begin{align}
  A_{\mu} &\to A_{\mu} + \partial_{\mu} \lambda(x) \\
  \phi^a &\to e^{i q_a \lambda(x)} \phi^a,
\end{align}
where $q_a$ are charges of $\phi^a$.

\begin{equation}
  F_{\mu\nu} \to F_{\mu\nu} \qquad \text{and} \qquad D_{\mu} \phi^a \to e^{i e q_a \lambda} D_{\mu} \phi^a
\end{equation}
where
\begin{equation}
  D_{\mu} \coloneqq (\partial_{\mu} - i e q_a a_{\mu})
\end{equation}

\begin{itemize}
  \item Pick a gauge to make progress
    \begin{itemize}
      \item We enforced $\partial_{\mu} A^{\mu} = 0$ by changing $\mathcal{L}$
    \end{itemize}
  \item M\o ller scattering
\end{itemize}

\section*{Part III Exam}%
\label{sec:part_iii_exam}

Same format as previous years:
\begin{itemize}
  \item 3 hours
  \item answer 3 out of 4 questions, all equal weight
\end{itemize}
