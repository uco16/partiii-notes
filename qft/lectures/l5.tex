% lecture notes by Umut Özer
% course: qft
\lhead{Lecture 5: October 19}
\begin{leftbar}
  \begin{remark}
    The momentum eigenstates are not localised in space. We can create a localised state via Fourier transform
    \begin{equation}
      \ket{\vb{x}} = \int_{}^{} \bdd[3]{p} e^{i\vb{p}\cdot\vb{x}} \ket{\vb{p}}.
    \end{equation}
    More generally, when we talk about real particles, we describe a wave-packet partially localised in position and partially in momentum space:
    \begin{equation}
      \ket{\psi} = \int_{}^{} \bdd[3]{p} e^{i\vb{p}\cdot\vb{x}} \psi(\vb{p}) \ket{\vb{p}}
    \end{equation}
    where for example $\psi (\vb{p}) \propto e^{- \abs{\vb{p}}^2 / 2m^2}$.
    Neither $\ket{\vb{x}}$ nor $\ket{\psi}$ are eigenstates of $H$---like in usual QM.
  \end{remark}
\end{leftbar}

\section{Relativistic Normalisation}%
\label{sec:relativistic_normalisation}

We define the vacuum normalisation to be $\bra{0}\ket{0} = 1$. So we have for a general momentum eigenstate, 
\begin{equation}
  \bra{\vb{p}} \ket{\vb{q}} = \bra{0} [a_{\vb{p}}, a^{\dagger}_{\vb{q}}] \ket{0} = \bdelta^3 (\vb{p} - \vb{q}).
\end{equation}
But is this Lorentz invariant? If we perform a Lorentz transformation, the momenta change as $p^\mu \to \Lambda\indices{^\mu_\nu} p^\nu \coloneq p'^\mu$, such that $\ket{\vb{p}} \to \ket{\vb{p}'}$.
We want the states $\ket{\vb{p}}$ and $\ket{\vb{p}'}$ to be related by a unitary transformation:
\begin{equation}
  \ket{\vb{p}} \to \ket{\vb{p}'} = U(\Lambda) \ket{\vb{p}}.
\end{equation}
This is because the inner product would be invariant under such a transformation:
\begin{equation}
  \bra{\vb{p}} \ket{\vb{q}} \xrightarrow{\text{LT}} \bra{\vb{p}} U^{\dagger}(\Lambda) U(\Lambda) \ket{\vb{q}} = \bra{\vb{p}} \ket{\vb{q}}.
\end{equation}
Take the identity operator on one-particle states to be
\begin{equation}
  1 = \int_{}^{} \bdd[3]{p} \ket{\vb{p}} \bra{\vb{q}}.
\end{equation}
Notice that neither the integral measure $\int_{}^{} \bdd[3]{p} $, not the outer product $\ket{\vb{p}} \bra{\vb{q}}$ is Lorentz invariant by itself. However, the identity operator evidently is. Let us analyse this.

\begin{claim}
  $\int_{}^{} \frac{\dd[3]{p}}{2 E_p} $ is Lorentz invariant.
\end{claim}
\begin{proof}
  The integral over the full four momentum $\int_{}^{} \dd[4]{p} $ is obviously Lorentz invariant, since Lorentz transformations are elements of SO(1,3), meaning that $\det \Lambda = 1$ and the Jacobian is unity when changing frames.
  The relativistic dispersion relation for a massive particle $E_p^2 = p_0^2 = \abs{\vb{p}}^2 + m^2$ is Lorentz invariant (ie.~holds in all frames). The choice of branch for $p_0$ is also Lorentz invariant.
  \begin{equation}
    \int_{}^{} \dd[4]{p} \delta (p_0^2 - \abs{\vb{p}}^2 - m^2)\rvert_{p_0 >  0} \qquad \text{is LI}
  \end{equation}
  Finally, we use the properties of the delta function
  \begin{equation}
    \delta (g(x)) = \sum_{x_i \text{ roots of } g} \frac{\delta(x - x_i)}{\abs{g'(x_i)}}
  \end{equation}
  to show that the above integral is 
  \begin{equation}
    \int_{}^{} \dd[4]{p} \delta (p_0^2 - \abs{\vb{p}}^2 - m^2)\rvert_{p_0 >  0} = \int_{}^{} \frac{\dd[3]{p}}{2E_p} 
  \end{equation}
\end{proof}

\begin{claim}
  $2E_p \delta^3 (\vb{p}-\vb{q})$ is the Lorentz invariant delta function.
\end{claim}
\begin{proof}
  \begin{equation}
    \int_{}^{} \frac{\dd[3]{p}}{2E_p} 2 E_p \delta^3 (\vb{p}-\vb{q}) = 1
  \end{equation}
  Since the integral measure is Lorentz invariant by the previous claim, and the right hand side is just a number, the claim follows.
\end{proof}

Hence, we can define relativistically normalised states as
\begin{equation}
  \ket{p^\mu} \coloneq \ket{p} = \sqrt{2E_p} \ket{\vb{p}} \sqrt{2 E_p} a_{\vb{p}}^{\dagger} \ket{0}.
\end{equation}
Then the inner product is 
\begin{equation}
  \bra{p}\ket{q} = (2\pi)^3 2 \sqrt{E_p E_q} \delta^3(\vb{p}-\vb{q})
\end{equation}
and we re-write the completeness relation on single-particle states as
\begin{equation}
  1 = \int_{}^{} \frac{\bdd[3]{p}}{2E_p} \ket{p}\bra{p}.
\end{equation}

\section{Free Complex Scalar Field}%
\label{sec:free_complex_scalar_field}

Let $\psi \in \mathbb{C}$ be a complex scalar field. The associated Lagrangian density is
\begin{equation}
  \mathcal{L} = \partial_\mu \psi^* \partial^\mu \psi - \mu^2 \psi^* \psi,
\end{equation}
where $\mu \in \mathbb{R}$. Applying the Euler-Lagrange equations give
\begin{equation}
  \partial_\mu \partial^\mu \psi + \mu^2 \psi = 0, \qquad \partial_\mu \partial^\mu \psi + \mu^2 \psi = 0 .
\end{equation}
Again, we can expand this as a Fourier integral. However, this time, unlike in the case of the scalar field $\phi$, we want the conjugate fields to be different $\psi \neq \psi^*$. This is achieved by taking
\begin{equation}
  \psi = \int_{}^{} \frac{\bdd[3]{p}}{\sqrt{2E_p}} \left( b_{\vb{p}}e^{i\vb{p}\cdot\vb{x}} + c^{\dagger}_{\vb{p}}e^{-i\vb{p}\cdot\vb{x}} \right), \qquad
  \psi^{\dagger} = \int_{}^{} \frac{\bdd[3]{p}}{\sqrt{2E_p}} \left( b^{\dagger}_{\vb{p}}e^{-i\vb{p}\cdot\vb{x}} + c_{\vb{p}}e^{+i\vb{p}\cdot\vb{x}} \right).
\end{equation}
The conjugate momenta to these fields are
\begin{equation}
  \pi = \int_{}^{} \bdd[3]{p} (-i) \sqrt{\frac{E_p}{2}} \left( b^{\dagger}_{\vb{p}}e^{i\vb{p}\cdot\vb{x}} - c_{\vb{p}}e^{-i\vb{p}\cdot\vb{x}} \right), \qquad
  \pi^{\dagger} = \int_{}^{} \bdd[3]{p} (-i) \sqrt{\frac{E_p}{2}} \left( b_{\vb{p}}e^{i\vb{p}\cdot\vb{x}} - c^{\dagger}_{\vb{p}}e^{-i\vb{p}\cdot\vb{x}} \right).
\end{equation}
Finally, the commutation relations
\begin{equation}
  [\psi(\vb{x}), \pi(\vb{y}) ] = i \delta^3(\vb{x}-\vb{y}),
  \qquad
  [\psi(\vb{x}), \pi^{\dagger}(\vb{y}) ] = 0 , \qquad \text{etc\ldots}
\end{equation}
can then be shown to be equivalent to 
\begin{equation}
  [b_{\vb{p}}, b^{\dagger}_{\vb{q}} ] = (2\pi)^3 \delta^3(\vb{p} - \vb{q}) = [c_{\vb{p}}, c^{\dagger}_{\vb{q}}]
\end{equation}
with all other commutators vanishing.
The interpretation of this is that there are two different particles created by $b^{\dagger}_{\vb{p}}$ and $c^{\dagger}_{p}$. Both have the same mass $m$ and spin $0$.
These are interpreted as particle and anti-particle.
Retrospectively, we see that for a real scalar field, the particle is its own anti-particle.

The conserved charge associated with a phase rotation invariance of the Lagrangian is
\begin{align}
  Q &= i \int_{}^{} \dd[3]{x} (\dot \psi^* \psi - \psi^* \dot \psi) \\
    &= i \int_{}^{} \dd[3]{x} (\pi \psi - \psi^{\dagger} \pi^{\dagger}).
\end{align}
We then insert the respective Fourier expansions in terms of the ladder operators. After normal ordering, this conserved charge becomes
\begin{equation}
  Q = \int_{}^{} \bdd[3]{p} \left( c^{\dagger}_{\vb{p}} c_{\vb{p}} - b^{\dagger}_{\vb{p}} b_{\vb{p}} \right) = N_c - N_b.
\end{equation}
Since $[Q, H]$, the number of particles minus the number of anti-particles is conserved.
\begin{leftbar}
  \begin{remark}
    We are in the free theory, where $N_c$ and $N_b$ are individually conserved. In the interacting theory, this is not true any more, but the charge $Q$ will still be conserved.
    This corresponds to the conservation of electric charge, even in interacting theories.
  \end{remark}
\end{leftbar}

\section{The Heisenberg Picture}%
\label{sec:the_heisenberg_picture}

So far, in the Schrödinger picture, the Lorentz invariance has not been overt. In particular, the field operators $\phi(\vb{x})$ did only depend on space and not on time, whereas the states evolve in time via
\begin{equation}
  i \dv{\ket{p}}{t} = H \ket{p} = E_p \ket{p}.
\end{equation}
Physically, the expectation values $\bra{ \dots } \ket{\dots }$ corresponding to probability amplitudes that we will measure. We can leave these invariant by making the following transformations.
Taking the unitary transformation $U(t) = e^{iHt}$, we define operators $O_H$ in the Heisenberg picture as
\begin{equation}
  O_H(t) = U(t) O_S U^{\dagger}(t).
\end{equation}
This implies that the operators evolve in time via
\begin{equation}
  \dv{O_H}{t} = i [H, O_H].
\end{equation}
Note that at $t = 0$, the Schrödinger and Heisenberg operators coincide.
The operators in the Heisenberg picture satisfy equal-time commutation relations:
\begin{align}
  [\phi(\vb{x}, t), \phi(\vb{y}, t) ] &= 0 = [\pi(\vb{x}, t), \pi(\vb{y}, t)] \\
  [\phi(\vb{x}, t), \pi(\vb{y}, t)] &= i \delta^3(\vb{x} - \vb{y}).
\end{align}
\begin{exercise}
  We can check that $\dv{\phi}{t} = i [H, \phi]$ means that the Heisenberg operator $\phi$ satisfies the Klein-Gordon equation $\partial_\mu \partial^\mu \phi + m^2 \phi = 0$. This is now a vector equation.
\end{exercise}
\begin{notation}[]
  We denote the full four-vector spacetime dependence as $\phi(\vb{x}, t) = \phi(x)$.
\end{notation}
We write the Fourier transform of $\phi(x)$ by deriving
\begin{equation}
  U(t) a_{\vb{p}} U^{\dagger}(t) = e^{-iE_p t} a_{\vb{p}}, \qquad
  U(t) a^{\dagger}_{\vb{p}} U^{\dagger}(t) = e^{+iE_p t} a^{\dagger}_{\vb{p}}.
\end{equation}
Then, in the Heisenberg picture, $\phi(x)$ is given by
\begin{equation}
  \phi(x) = \int_{}^{} \frac{\bdd[3]{p}}{\sqrt{2E_p}} \left( a_{\vb{p}} e^{-ip \cdot x} + a_{\vb{p}} e^{+i p \cdot x} \right).
\end{equation}

