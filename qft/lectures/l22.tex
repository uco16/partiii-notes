% lecture notes by Umut Özer
% course: qft
\lhead{Lecture 22: December 03}

\section{External Photons}%
\label{sec:external_photons}

We want to be able to calculate $\gamma-e$ scattering.

\begin{equation}
  \feynmandiagram[inline=(a.base), horizontal=a to b, layered layout] {
    a -- [boson,  momentum=\(k\ \lambda\)] b [blob],
  };
  \sim \varepsilon^{\lambda}_{\mu}(k)
  \qquad
  \feynmandiagram[inline=(a.base), horizontal=a to b, layered layout] {
    a [blob] -- [boson,  momentum=\(k\ \lambda\)] b,
  };
  \sim \varepsilon^{\lambda^*}_{\mu}(k)
\end{equation}

Typically, we do not resolve the photon polarisations. We therefore want to average over the initial ones and sum over the final ones.
The prescription, which we will motivate later, is
\begin{claim}
  We can replace the sum
\begin{equation}
  \sum_{\lambda} \varepsilon^{\lambda}_{\mu}{}^* (k) \varepsilon^{\lambda}_{\nu} (k) \to -\eta_{\mu\nu}
\end{equation}
\end{claim}
\begin{proof}
  To see how this is derived, consider an arbitrary process with an external photon $\gamma$.
   \begin{equation}
    \feynmandiagram[inline=(a.base), horizontal=a to b, layered layout] {
      a -- [fermion] b [blob] -- [boson,  momentum=\(k\ \lambda\)] c,
      f -- [fermion] b,
      d -- [fermion] b,
      e -- [fermion] b,
    };
    \begin{gathered}
      \mathcal{M} \coloneqq M^{\mu}(k) \varepsilon^{\lambda}_{\mu}{}^* (k) \\
      d\sigma \propto \sum_{\lambda} \abs{\mathcal{M}}^2 = \sum_{\lambda} \varepsilon^{\lambda}_{\mu}{}^* (k) \varepsilon^{\lambda}_{\nu} (k) \mathcal{M}^{\mu}(k) \mathcal{M}^{\nu}{}^*(k)
    \end{gathered}
  \end{equation}
  We rotate coordinates such that the photon direction is $x^3$:  $k^{\mu}  (k, 0, 0, k)$ .
  Physical photon polarisations are $\varepsilon^{1\mu} = (0, 1, 0, 0)$  and $\varepsilon^{2 \mu} = (0, 0, 1, 0)$ .
  I.e.~$\sum_{\lambda} \abs{\mathcal{M}}^2 = \abs{M^2(k)}^2 + \abs{\mathcal{M}^2(k)}^2$ . 
  
  Recall that external photons are created by $\int \dd[4]{x} j^{\mu} A_{\mu}$ .
  Therefore,
  \begin{equation}
    M^{\mu}(k) \propto \int \dd[4]{x} e^{i k \cdot x} \bra{f} j^{\mu} \ket{i},
  \end{equation}
  where the external photon is \emph{omitted} from $\ket{i}$  and $\ket{f}$ .

  Classical electromagnetism implies that the current is conserved, $\partial_{\mu} j^{\mu}(x) = 0$ . This still holds in \texttt{QFT}---it is called a \emph{Ward Identity} and is a consequence of gauge invariance.
  \begin{remark}
    This will be covered in the Lent term's \texttt{AQFT} course.
  \end{remark}
  \begin{remark}
    Integrating by parts, we have
    \begin{align}
      k_{\mu} \mathcal{M}^{\mu}(k) &\propto \int \dd[4]{x} (\partial_{\mu} e^{i k \cdot x}) \bra{f} j^{\mu}(x) \ket{i}  \\
      &= - \int \dd[4]{x} e^{i k \cdot x} \bra{f} \partial_{\mu} j^{\mu} \ket{i} =0
    \end{align} 
    where we used the Ward identity in the last line.
    Thus, the Ward identity implies that $\mathcal{M} = 0$ if we replace the photon polarisation with its $4$-momentum.
  \end{remark}

  For our particular $k^{\mu} = (k, 0, 0, k)$ , the Ward identity states 
  \begin{equation}
    k \mathcal{M}^0(k) - k \mathcal{M}^3(k) = 0 \qquad \text{or} \qquad \mathcal{M}^0 = \mathcal{M}^3.
  \end{equation}
  Moreover, 
  \begin{align}
    \sum_{\lambda} \varepsilon^{\lambda}_{\mu}{}^* (k) \varepsilon^{\lambda}_{\nu} (k) \mathcal{M}^{\mu}(k) \mathcal{M}^{\nu}(k)^* &= \abs{\mathcal{M}^1}^2 + \abs{\mathcal{M}^2}^2 + 0 \\
    &= \abs{\mathcal{M}^1}^2 + \abs{\mathcal{M}^2}^2 + \abs{\mathcal{M}^3}^2  - \abs{\mathcal{M}^0}^2 \\
    &= -\eta_{\mu\nu} \mathcal{M}^{\mu}(k) (\mathcal{M}^{\nu}(k))^*.
  \end{align}
  Hence, we may replace $\sum_\lambda \varepsilon^{\lambda}_{\mu}{}^* (k) \varepsilon^{\lambda}_{\nu} (k) \to -\eta_{\mu\nu}$.
\end{proof}

In other words, the timelike and longitudinal photons can be consistently \emph{omitted} from \texttt{QED} calculations, since the squared amplitudes cancel.

\subsection{Electron-Muon Scattering in QED}%
\label{sub:electron_muon_scattering_in_qed}

Both are spinors. However, they have different interactions and therefore should be described by different fermionic fields $\psi_e$ and $\psi_\mu$ of mass $m_e$ and $m_\mu$ respectively.
All of the electron creation and annihilation operators will anti-commute with the ones for the muon.

\begin{equation}
  \begin{gathered}
    \feynmandiagram[inline=(a.base), horizontal=a to b] {
      a [particle=\(e\)] -- [fermion,  momentum=\(p\ s\)] b -- [fermion, momentum=\(p'\ s'\)] c,
      d [particle=\(\mu\)] -- [fermion,  momentum=\(q\ r\)] e -- [fermion, momentum=\(q'\ r'\)] f,
      b -- [boson] e,
      a -- [draw=none] d,
      c -- [draw=none] f,
    };
  \end{gathered}
  \qquad \mathcal{M} \sim (-i e)^2 [\overline{u}^{s'}(p') \gamma^{\mu} u^{s}(p)] \left( \frac{-i \eta_{\mu\nu}}{(p - p')^2 + i \epsilon} \right) [\overline{u}^{r'}(q') \gamma^{\nu} u^r (q)]
\end{equation}

\begin{equation}
  \begin{gathered}
    \begin{tikzpicture}[scale=0.8]
      \begin{feynman}
	\diagram[small, horizontal=a to b] {
      a -- [fermion,  momentum=\(p\)] b -- [fermion,  momentum=\(p'\)] c,
      d -- [fermion,  momentum=\(q\)] e -- [fermion,  momentum'=\(q'\)] f,
      b -- [boson] e,
      a -- [draw=none] d,
      c -- [draw=none] f,
      c -- [fermion,  momentum=\(p\)] i,
      f -- [fermion,  momentum=\(q\)] j,
      c -- [boson] f,
      i -- [draw=none] j,
	};
	\end{feynman}
	%circle in middle
	%\node (v) at ($(a)!0.5!(j)$) {};
	% arcs
	\draw (i) arc (0:181.5:1.9);
	\draw (j) arc (0:-178:1.9);
      \end{tikzpicture}
  \end{gathered}
  \qquad
  \begin{gathered}
    \begin{aligned}
      \overline{\abs{\mathcal{M}}^2} = \frac{e^4}{4 t^2} &\Tr[(\cancel{p'} + m_e) \eta^{\mu} (\cancel{p} + m_e) \gamma^{\nu}] \\
      \times &\Tr[(\cancel{q'} + m_{\mu}) \gamma_{\mu} (\cancel{q} + m_{\mu}) \gamma_{\nu}]
    \end{aligned}
  \end{gathered}
\end{equation}
The first trace is
\begin{equation}
  4 (p'{}^{\nu} p^{\mu} + p'{}^{\mu} p^{\nu} - p \cdot p' \eta^{\mu\nu} + m_e^2 \eta^{\mu\nu}).
\end{equation}
The second one is the same except for $m_e \to m_{\mu}$ and $p \to q$, $p' \to q'$, and indices downstairs.

In the massless limit, 
\begin{equation}
  \overline{\abs{\mathcal{M}}^2} = \frac{8 e^4}{t^2} \left[ \frac{s^2}{4} + \frac{u^2}{4} \right] = 2 e^4 \frac{s ^2 + u ^2}{t^2}.
\end{equation}
In this limit, we had derived an expression, in which we substitute the above to get
\begin{equation}
  \boxed{\dv{\sigma}{t} = \frac{\overline{\abs{\mathcal{M}}^2}}{16 \pi s^2} = \frac{e^4}{8 \pi s^2} \frac{s^2 + (s + t)^2}{t^2}}
\end{equation}
This means that the cross-section is
\begin{equation}
  \sigma = \frac{e^4}{8 \pi s^2} \int_{-s / 2}^0 \dd[]{t} \frac{s^2 + (s + t)^2}{t^2},
  \qquad t = 
  \begin{cases}
    -2 p \cdot p' & \text{for } m_e, m_\mu \to 0 \\
    -\frac{s}{2}(1 - \cos \theta)
  \end{cases}.
\end{equation}
Note that $\sigma$ has a singularity that disappears if we include $m_e \neq 0$ and $m_{\mu} \neq 0$ because we would not integrate to zero.
In the centre of mass frame, 
\begin{align}
  p' &= (\sqrt{\abs{\vb{p}}^2 + m_e^2}, \vb{p}'),  &p &= (\sqrt{\abs{\vb{p}}^2+ m_e^2}, \vb{p}) \\
  q' &= (\sqrt{\abs{\vb{p}}^2 + m_{\mu}^2}, -\vb{p}'), &q &= (\sqrt{\abs{\vb{p}}^2+ m_e^2}, -\vb{p})
\end{align}
\begin{align}
  s = (p' + q')^2 &= (\sqrt{\abs{\vb{p}}^2 + m_e^2} + \sqrt{\abs{\vb{p}'}^2 + m_{\mu}^2})^2 \\
  \implies \abs{\vb{p}'}^2 &= \frac{\lambda(s, m_e^2, m_{\mu}^2)}{4 (s + m_e^2 +m_{\mu}^2)} = \abs{\vb{p}}^2
\end{align}

\begin{remark}
  \begin{align}
    t &= m_e^2 + m_{\mu}^2 - 2 p \cdot p'  \\
      &= m_e^2 + m_{\mu}^2 - 2 \left( \frac{\lambda(s, m_e^2, m_{\mu}^2)}{4(s + m_e^2 + m_{\mu}^2)} + m_e^2 \right) + \frac{2 \cos\theta \lambda(s, m_e^2, m_{\mu}^2)}{4(s + m_e^2 + m_{\mu}^2)}
  \end{align}
  upper limit is now $m_{\mu}^2 - m_e^2$ instead of $0$.
\end{remark}

%Alternative explanation erratum 5 Dec
\subsection*{Erratum}%

The ranges of integration should be $- \frac{s}{2} ( 1+  \cos(\pi - \theta_0)) \simeq -\frac{s}{2}$ to $-\frac{s}{2}(1 - \cos(\theta_0)) \simeq -\frac{s}{4} \theta_0^2$. Then 
\begin{align}
  \sigma &= \frac{e^4}{8 \pi s^2} \int_{-s / 2}^{-s \theta_0^2 / 4} \frac{2s^2}{t^2} + \frac{2s}{t} + 1 \\
	 &= \frac{e^{4}}{8 \pi s} [- \frac{2s^2}{t} + 2 s \ln t + t]_{+s / 2}^{-s \theta_0^2 / 4} \\
	 &= \frac{e^4}{ 8\pi s} \left\{ \frac{8}{\theta_0^2}  \frac{7}{2} + 2 \ln( \frac{2}{\theta_0^2}) + O(\theta_0^2) \right\}
\end{align}

\section{Other Processes in QED}%
\label{sec:other_processes_in_qed}

\begin{equation*}
  e^- e^- \to e^- e^-: \qquad
  \begin{gathered}
    \feynmandiagram [small, vertical=b to e] {
      d -- [anti fermion] e -- [boson] f,
      a -- [fermion] b -- [boson] c,
      b -- [fermion] e,
      a -- [draw=none] d,
      c -- [draw=none] f,
    };
  \end{gathered}
  \qquad + \qquad
  \begin{gathered}
    \begin{tikzpicture}
      \begin{feynman}
	\diagram [small, vertical=b to e] {
	  d -- [anti fermion] e -- [draw=none] f,
	  a -- [fermion] b -- [draw=none] c,
	  b -- [fermion] e,
	  a -- [draw=none] d,
	  c -- [draw=none] f,
	};
	\diagram* {
	  (b) -- [boson] (f),
	  (e) -- [boson] (c),
	};
      \end{feynman}
      \end{tikzpicture}
  \end{gathered}
\end{equation*}
\begin{equation*}
  e^- e^- \to e^- e^-: \qquad
  \begin{gathered}
    \feynmandiagram [small, vertical=b to e] {
      d -- [fermion] e -- [fermion] f,
      a -- [fermion] b -- [fermion] c,
      b -- [boson] e,
      a -- [draw=none] d,
      c -- [draw=none] f,
    };
  \end{gathered}
  \qquad + \qquad
  \begin{gathered}
	\begin{tikzpicture}
	  \begin{feynman}
	    \diagram [small, vertical=b to e] {
	d -- [fermion] e -- [draw=none] f,
	a -- [fermion] b -- [draw=none] c,
	b -- [boson] e,
	a -- [draw=none] d,
	c -- [draw=none] f,
      };
      \diagram* {
	(b) -- (f),
	(e) -- [fermion] (c),
      };
	  \end{feynman}
	\end{tikzpicture}
  \end{gathered}
\end{equation*}
\begin{equation*}
  e^+ e^- \to e^+ e^-: \qquad
  \begin{gathered}
    \feynmandiagram [small, horizontal=a to b] {
      i [particle=\(e^-\)] -- [fermion] a -- [fermion] ii [particle=\(e^+\)],
      a -- [boson] b,
      f [particle=\(e^+\)] -- [anti fermion] b -- [anti fermion] ff [particle=\(e^+\)],
    };
  \end{gathered}
  \qquad + \qquad
  \begin{gathered}
    \feynmandiagram [small, vertical=b to e] {
      d [particle=\(e^+\)] -- [anti fermion] e -- [anti fermion] f [particle=\(e^{+}\)],
      a [particle=\(e^{-}\)] -- [fermion] b -- [fermion] c [particle=\(e^{-}\)],
      b -- [boson] e,
      a -- [draw=none] d,
      f -- [draw=none] c,
    };
  \end{gathered}
\end{equation*}
\begin{equation*}
  e^+ e^- \to \mu^+ \mu^-: \qquad
  \begin{gathered}
    \feynmandiagram [horizontal=a to b] {
      i [particle=\(e^-\)] -- [fermion] a -- [fermion] ii [particle=\(e^+\)],
      a -- [boson] b,
      f [particle=\(\mu^+\)] -- [fermion] b -- [fermion] ff [particle=\(\mu^-\)],
    };
  \end{gathered}
\end{equation*}
This diagram is historically important since quark diagram is the same, but comes in three colours. Comparison to muon production showed that QCD has three colours.

We can even scatter photons\footnote{This has only been observed in 2018 at the LHC}, but not at tree level; we have to consider loops:
\begin{equation*}
  \gamma\gamma \to \gamma\gamma: \qquad
  \begin{gathered}
    \feynmandiagram [small, vertical=b to e] {
      d -- [boson] e -- [anti fermion] g -- [boson] f,
      a -- [boson] b -- [fermion] h -- [boson] c,
      b -- [fermion] e,
      g -- [fermion] h,
      a -- [draw=none] d,
      c -- [draw=none] f,
    };
  \end{gathered}
  \qquad + \qquad \left(
    \begin{gathered}
      \text{permutations} \\
      \text{of external } \gamma
    \end{gathered}
      \right) \quad \propto \quad \int \frac{\bdd[4]{k}}{k^4 + \dots}
\end{equation*}
This is naively $\ln$ divergent! However, gauge invariance \emph{cancels} these pieces. This renders the result finite.

Finally, we also have \emph{Compton scattering}
\begin{equation*}
  \gamma e^- \to \gamma e^-: \qquad
  \begin{gathered}
    \begin{tikzpicture}
      \begin{feynman}
	\diagram[small, horizontal=b to c] {
	  e -- [draw=none] f -- [draw=none] g -- [draw=none] h,
	  a -- [fermion] b -- [fermion] c -- [fermion] d,
	  a -- [draw=none] e,
	  d -- [draw=none] h,
	  b -- [draw=none] f,
	  c -- [draw=none] g,
	};
	\diagram* {
	  (e) -- [boson] (b),
	  (h) -- [boson] (c),
	};
      \end{feynman}
    \end{tikzpicture}
  \end{gathered}
  \qquad + \qquad
  \begin{gathered}
    \begin{tikzpicture}
      \begin{feynman}
	\diagram[small, horizontal=b to c] {
	  e -- [draw=none] f -- [draw=none] g -- [draw=none] h,
	  a -- [fermion] b -- [fermion] c -- [fermion] d,
	  a -- [draw=none] e,
	  d -- [draw=none] h,
	  b -- [draw=none] f,
	  c -- [draw=none] g,
	};
	\diagram* {
	  (b) -- [boson] (g),
	  (f) -- [boson] (c),
	};
      \end{feynman}
    \end{tikzpicture}
  \end{gathered}
\end{equation*}

