% lecture notes by Umut Özer
% course: qft
\lhead{Lecture 12: November 07}
\section{Green's Functions and the Vacuum}%
\label{sec:green_s_functions_and_the_vacuum}

Up until now, we have been describing amplitudes in terms of operators sandwiched between the \emph{free} vacuum states: $\bra{0} \dots \ket{0}$.
What if we have interactions? We will see that we have been doing so far is actually correct, but we should show that explicitly.
Let $\ket{\Omega}$ be the vacuum of the \emph{interacting} theory.
In analogy to $H_0 \ket{0} = 0$ for the free Hamiltonian, we normalise the Hamiltonian with interactions as $H = H_0 + H_{\text{int}}$ such that $H \ket{\Omega}= 0$ and $\bra{\Omega}\ket{\Omega} = 1$.
The $n$-point Green function is then defined to be
\begin{equation}
  G^{(n)} (x_1, \dots, x_n) \coloneqq \bra{\Omega} T \left\{ \phi_H (x_1) \dots \phi_H(x_n) \right\} \ket{\Omega}.
\end{equation}
We will write $\phi_H(x_i) \coloneqq \phi_{iH}$ for the fields in the interacting picture.
\begin{claim}
  Removing the vacuum bubbles takes us back to the free vacuum:
  \begin{align}
    \bra{\Omega} T \left\{ \phi_{1H} \dots \phi_{m H} \right\} \ket{\Omega} &=
    \frac{\bra{0} T \left\{ \phi_{1I} \dots \phi_{mI}S \right\}\ket{0}}{\bra{0}S \ket{0}} \\
									    &= \sum (\text{connected diagrams with $m$ external points}).
  \end{align}
\end{claim}
\begin{leftbar}
  \begin{remark}
    This explains why the computations that we have been doing make sense at all.
  \end{remark}
\end{leftbar}
\begin{proof}
  Without loss of generality, we may order the coordinates in time as $x_1^0 > x_2^0 > \dots > x_m^0$.
  Remember, $U(t, t_0) = e^{iH_0 t} e^{iH(t - t_0)} e^{-i H_0 t_0}$.
  Expanding out the $S$ operator, the left hand side of the claim is equal to
  \begin{align}
    \bra{0} T \left\{ \phi_{1I} \dots \phi_{mI} S \right\} \ket{0} &=
    \bra{0} U(\infty, t_1) \phi_{1I} U(t_1, t_2) \phi_{2I} \dots U(t_{m-1}, t_m) \phi_{mI} U(t_m, -\infty) \ket{0} \\
    \begin{split}
      &= \bra{0} U(\infty, t_1) U(t_1, 0) \underbrace{U(0, t_1) \phi_{1I} U(t_1, 0)}_{\phi_{1H}} U(0, t_2) \phi_{2I} \dots \\
      & \qquad \dots \underbrace{U(0, t_m) \phi_{mI} U(t_m 0)}_{\phi_{mH}} U(0, -\infty) \ket{0}
    \end{split}
  \end{align}
  This insertion in the last line allows us to convert back into the Heisenberg picture.
  \begin{align}
    \dots &= \underbrace{U(\infty, 0) \phi_{1H} \dots \phi_{mH}}_{\coloneqq \bra{\psi}} U(0, -\infty) \ket{0} \\
	  &= \lim_{t_0 \to -\infty} \left\{ \bra{\psi} U(0, t_0) \ket{0} \right\} \\
	  &= \lim_{t_0 \to -\infty} \left\{ \bra{\psi} e^{i H t_0} \ket{0} \right\},
  \end{align}
  where we used that $H_0 \ket{0} = 0$. We now insert a complete set of interacting states to get
  \begin{align}
    \dots &= \lim_{t_0 \to - \infty} \biggl\{ \bra{\psi} e^{-i Ht_0} \Bigl[ \ket{\Omega} \bra{\Omega} + \sum_{n=1}^\infty \left[ \prod_{j=1}^n \frac{\bdd[3]{p_j}}{\sqrt{2E_{p_j}}} \right] \ket{p_1, \dots, p_n} \bra{p_1, \dots, p_n} \Bigr] \ket{0} \biggr\}  \\
	  &= \bra{\langle} \ket{\Omega} + \lim_{t_0 \to -\infty} \biggl\{ \sum_n \int \left( \prod_j \frac{\bdd[3]{p_j}}{\sqrt{2E_{p_j}}} \right) e^{-i \sum_{j=1}^n E_{p_j} t_0} \bra{\psi} \ket{p_1, \dots, p_n}\bra{p_1, \dots, p_n} \ket{0} \biggr\}
  \end{align}
  The second term vanishes because of the Riemann-Lebesgue lemma:
  \begin{equation}
    \lim_{\mu \to \infty} \left\{ \int_{a}^{b}  \dd[]{x} f(x) e^{i\mu x} \right\} = 0
  \end{equation}
  Therefore, we find 
  \begin{equation}
    \dots = \bra{0} U(\infty, 0) \underbrace{\phi_{1H} \dots \phi_{mH} \ket{\Omega} \bra{\Omega} \ket{0}}_{\ket{\psi}}.
  \end{equation}
  Similar to above we have
  \begin{align}
    \bra{0} U(\infty, 0) \ket{\psi} &= \lim_{t_0 \to \infty} \left\{ \bra{0} e^{i H t_0} \ket{\psi} \right\} \\
				    &= \bra{\Omega} \phi_{1H} \dots \phi_{mH} \ket{\Omega}\bra{\Omega}\ket{0} \bra{0} \ket{\Omega}.
  \end{align}
  The denominator in the right hand side of the claim is ($m = 0$)
  \begin{equation}
    \bra{0}S \ket{0} = \bra{0}\ket{\Omega}\bra{\Omega}\ket{0},
  \end{equation}
  which completes the proof.
\end{proof}
\begin{leftbar}
  \begin{remark}
    Green's functions are ultimately the simplest way to think about scattering.
  \end{remark}
\end{leftbar}
\begin{example}[]
  Let us consider the previous example, we have
  \begin{equation}
    \bra{\Omega} T \left\{ \phi_{1H} \dots \phi_{4H} \ket{\Omega} \right\} = 
    \begin{gathered}
      \feynmandiagram[transform shape, scale=0.5][inline=(v.base), horizontal=a to b] {
        a -- v [dot] -- b,
        c -- v -- d,
      };
    \end{gathered}
    + \left(
    \begin{gathered}
      \feynmandiagram[transform shape, scale=0.5][inline=(v.base), horizontal=a to b] {
	a -- [draw = none] v [dot] -- [draw = none] b,
	c -- v -- d,
	a -- b,
	v -- [loop, min distance=2cm] v,
      };
    \end{gathered}
    + 5 \text{ similar}
  \right) + 
    \cancel{%F_2
    }
  \end{equation}
\end{example}

\subsection*{LSZ Reduction}%

To describe scattering in the interacting theory, the external states (eg $\ket{p_1, p_2}$) should be those of the interacting theory.
This means that when we have a Feynman diagram with an interaction on an external leg, then that loop gets absorbed into the definition of an external interacting state.
In other words, we should exclude loops on external legs.
\begin{leftbar}
  \begin{remark}
    A more detailed explanation of these \emph{amputated diagrams} is described further in the \emph{Advanced Quantum Field Theory} course in Lent term.
  \end{remark}
\end{leftbar}

\section{Scattering}%
\label{sec:scattering}

\subsection*{Mandelstam Variables}%

We would like to describe the kinematics in terms of Lorentz invariant theories. This means we do not have to worry about switching frames.
Consider $2 \to 2$ scattering depicted in Figure \ref{fig:12-1}.
\begin{figure}[htbp]
  \centering
  \feynmandiagram [vertical=a to c, layered layout] {
    a -- [fermion, edge label=\(m_1\),  momentum'=\(p_1\)] v [blob] -- [fermion, edge label=\(m_1'\),  momentum'=\(q_1'\)] b,
    c -- [fermion, edge label=\(m_2\), momentum'=$p_2$] v -- [fermion, edge label=$m_2'$, momentum'=$q_2$] d,
  };
  \caption{The Feynman diagrams associated with $2 \to 2$ scattering are obtained by replacing the central blob with all possible valid vertex / propagator combinations.}
  \label{fig:12-1}
\end{figure}
By momentum conservation, we must have $p_1^{\mu} + p_2^{\mu} = q_1^{\mu} + q_2^{\mu}$. The Mandelstam variables are then defined as
\begin{equation}
  s \coloneqq (p_1 + p_2)^2 \qquad t \coloneqq (p_1-q_1)^2 \qquad u \coloneqq (p_1 - q_2)^2.
\end{equation}
\begin{exercise}
  Let $s + t + u = m_1^2 + m_2^2 + m_1'^2 + m_2'^2$.
  Nucleon scattering is then given as
  \begin{align}
    &
    \begin{gathered}
      \feynmandiagram[inline=(a.base), horizontal=a to b] {
        a -- [fermion,  momentum=\(p_1\)] t -- [fermion,  momentum=\(q_1\)] b,
        c -- [fermion,  momentum=\(p_2\)] e -- [fermion,  momentum=\(q_2\)] d,
        t -- [scalar] e,
        a -- [draw=none] c,
        b -- [draw=none] d,
      };
    \end{gathered}
    \quad + \quad
    \begin{gathered}
      \feynmandiagram[inline=(a.base), horizontal=a to b] {
        a -- [fermion,  momentum=\(p_1\)] t -- [fermion,  momentum=\(q_2\)] b,
        c -- [fermion,  momentum=\(p_2\)] e -- [fermion,  momentum=\(q_1\)] d,
        t -- [scalar] e,
        a -- [draw=none] c,
        b -- [draw=none] d,
      };
    \end{gathered} \\
    i \mathcal{M} &= (-ig)^2 \left\{ \frac{1}{(p_1 - q_1)^2 - m^2} + \frac{1}{(p_1 - q_2)^2 - m^2} \right\} \\
     &= (-ig)^2 \left\{ \frac{1}{t - m^2} + \frac{1}{u - m^2} \right\} \\
  \end{align}
\end{exercise}

\subsection*{Decay Rates and Cross-Sections}%

Probability of scattering is expressed in terms of 
\begin{equation}
  \bra{f} (S-1) \ket{i} = i \mathcal{M} \bdelta^4 (p_i - \sum_i q_i)
\end{equation}
Now $\ket{f}$ and $\ket{i}$ are states of definite $4$-momentum.
We then get a normalised probability as
\begin{equation}
  P = \frac{\abs{\bra{f}(S-1)\ket{i}}^2}{\bra{f}\ket{f}\bra{i}\ket{i}}.
\end{equation}

The effective cross-sectional area $d\sigma$ for scattering into $\ket{f}$ can be shown \cite{peskin95} to be
\begin{equation}
  d\sigma = \frac{\bdelta^4 (p_1 + p_2 - \sum_i q_i) \abs{\mathcal{M}}^2}{\mathcal{F}}
\end{equation}
where we introduced the \emph{flow factor} $ \mathcal{F} = 4 \sqrt{(p_1 \cdot p_2)^2 - m_1^2 m_2^2}$.
To find the total $i \to f$ cross-section, we integrate over the final state momentum in the usual Lorentz invariant manner:
\begin{equation}
  \sigma = \int \frac{1}{\mathcal{F}} \dd[]{\Pi} \abs{\mathcal{M}}^2,
\end{equation}
where\footnote{David Tong and Peskin and Schroeder use $\dd[]{\Pi}$. We used $\dd[]{p_f}$ in Lectures.}
\begin{equation}
  \dd[]{\Pi} = \dd[]{p_f} \coloneqq \left( \prod_{f=1}^n \frac{\bdd[3]{q_f}}{2E_{q_f}} \bdelta^4( q_f - p_i) \right).
\end{equation}
