% lecture notes by Umut Özer
% course: qft
\lhead{Lecture 9: October 29}

\section{Wick's Theorem}%
\label{sec:wick_s_theorem}

We want to compute various quantities such as 
\begin{equation}
  \bra{f} T \left\{ H_I (x_1) \dots H_I(x_n) \right\} \ket{i}.
\end{equation}
Wick's theorem will simplify these computations.
It is handy if we write this in terms of normal ordered products.
Take the case of a real scalar field $\phi$ to illustrate this.
Separate the field into creation and annihilation operators
\begin{equation}
  \phi(x) = \phi^+(x) + \phi^-(x),
\end{equation}
where the notation is slightly counterintuitive since $\phi^+$ are the annihilation operators and $\phi^-$ the creation ones:
\begin{equation}
  \phi^+(x) \coloneqq \int_{}^{} \bdd[3]{p} \frac{1}{\sqrt{2 E_p}} a_{\vb{p}} e^{-ip \cdot x} \qquad
  \phi^-(x) \coloneqq \int_{}^{} \bdd[3]{p} \frac{1}{\sqrt{2 E_p}} a^{\dagger}_{\vb{p}} e^{+ip \cdot x}.
\end{equation}
For $x^0 > y^0$, the time ordered product of two fields is $T \left\{ \phi(x) \phi(y) \right\} = \phi(x) \phi(y)$.
Expanding this out in terms of $\phi^+$ and $\phi^-$, we get
\begin{align}
  T \left\{ \phi(x) \phi(y) \right\} &= (\phi^+(x) + \phi^-(x)) (\phi^+(y) + \phi^-(y)) \\
				     &=\phi^+(x) \phi^+(y) + \phi^-(x) \phi^+(y) + \phi^-(y) \phi^+(x) + \phi^-(x) \phi^-(y) + [\phi^+(x), \phi^-(y)]. \\
				     &= \normalorder{\phi(x) \phi(y) } + \int \frac{\bdd[3]{p'}}{\sqrt{2 E_{p'}}} \frac{\bdd[3]{p}}{\sqrt{2 E_p}} [a_{\vb{p}}, a_{\vb{p}'}^{\dagger}] e^{-i p \cdot x + i p' \cdot y} \\
				     &= \normalorder{\phi(x) \phi(y) } + \int \frac{\bdd[3]{p}}{2 E_p} e^{-ip \cdot (x - y)} \\
				     &= \normalorder{\phi(x) \phi(y) } + D(x - y).
\end{align}
For $y^0 > x^0$, we can go through the same argument to give
\begin{equation}
  T \left\{ \phi(x) \phi(y) \right\} = \normalorder{\phi(x) \phi(y) } + D(y - x).
\end{equation}
Putting these two results together, we find that
\begin{equation}
  T \left\{ \phi(x) \phi(y) \right\} = \normalorder{\phi(x) \phi(y) } + \Delta_F(x - y).
\end{equation}
Note that the operators $T \left\{ \phi(x) \phi(y) \right\}$ and $\normalorder{\phi(x) \phi(y) }$ only differ by a function $\Delta_F(x - y)$.
This is the simplest example of Wick's theorem.
\begin{notation}[Wick contractions]
  We use a bracket to denote a \emph{contraction} of a pair of fields; the string $\wick{\phi(x_1) \c\phi(x_2) \phi(x_3)\dots \c\phi(x_j) \dots \phi(x_n)}$ denotes the same product of fields, except that the two contracted fields are replaced by their Feynman propagator $\wick{\c\phi(x_2) \c\phi(x_j)} = \Delta_F(x_2 - x_j)$.
\end{notation}
Using this notation, we can write the above result as
\begin{equation}
  T \left\{ \phi(x) \phi(y) \right\} = \normalorder{\phi(x) \phi(y) } + \wick{\c\phi(x) \c\phi(y)}
\end{equation}
\begin{theorem}[Wick's theorem]
  In general, Wick's theorem states that
  \begin{equation}
    T \left\{ \phi(x_1) \phi(x_2) \dots \phi(x_n) \right\} = \normalorder{\phi(x_1) \dots \phi(x_n) } + \normalorder{\text{all possible contractions} }.
  \end{equation}
\end{theorem}
\begin{example}[four fields]
  For a more concise notation, denote $\phi_i \coloneqq \phi(x_i)$, for the four fields $i = 1, 2, 3, 4$.
  The time ordered product of the four fields is then
  \begin{multline}
    T \left\{ \phi_1 \phi_2 \phi_3 \phi_4 \right\} 
    = \normalorder{\phi_1 \phi_2 \phi_3\phi_4 }
    + \left( \wick{\c\phi_1 \c\phi_2} \normalorder{\phi_3 \phi_4} + \wick{\c\phi_1 \c\phi_3} \normalorder{\phi_2 \phi_4}
    + \text{4 similar terms}\right)\\ 
    + \wick{\c\phi_1 \c\phi_2} \wick{\c\phi_3\c\phi_4}
    + \wick{\c\phi_1 \c\phi_3} \wick{\c\phi_2 \c\phi_4} + \wick{\c\phi_1\c\phi_4}\wick{\c\phi_2\c\phi_3}
  \end{multline}
  This is useful because all normal ordered terms that are sandwiched between vacuum states vanish; only the propagators survive:
  \begin{multline}
    \label{eq:four-field-tordered-product}
    \bra{0} T \left\{ \phi_1\phi_2\phi_3\phi_4 \right\} \ket{0} 
    = \Delta_F (x_1 - x_2) \Delta_F(x_3-x_4) \\ 
    + \Delta_F(x_1 - x_3) \Delta_F(x_2-x_4)
    + \Delta_F(x_1-x_4) \Delta_F(x_2 - x_3).
  \end{multline}
\end{example}
\begin{proof}[Proof of Wick's theorem]
  Let us proceed by induction. We have already shown that Wick's theorem holds for $n = 2$. 
  Now suppose it is true for $T \left\{ \phi_2 \dots \phi_n \right\}$ and left-multiply with $\phi_1$ with $x_1^0 > x_k^0$, $\forall k \in \left\{ 2, \dots, n \right\}$.
  Then 
  \begin{equation}
    T \left\{ \phi_1 \dots \phi_n \right\} = (\phi^+_1 + \phi^-_1) \left[ \normalorder{\phi_2 \dots \phi_n } + \normalorder{\text{all contractions of }\phi_2 \dots \phi_n } \right]
  \end{equation}
  The term $\phi^-_1$ can stay on the left hand side and go inside the normal ordering. The $\phi^+_1$ has to be commuted through to the right hand side of all $\phi^-_2 \dots \phi^-_n$, so that the right hand side can be written as a sum over normal-ordered products.
  Each commutation gives us $D(x_1 - x_k)$, which, for this time ordering, is $\Delta_F(x_1 - x_k) = \wick{\c\phi_1 \c\phi_k}$.
  \begin{leftbar}
    \begin{remark}
      The textbook `Relativistic quantum fields' (1965) by Bj\"rken and Drell contains a clear exposition of Wick's theorem.
    \end{remark}
  \end{leftbar}
\end{proof}

\subsection*{Consequences of Wick's theorem}%

\begin{itemize}
  \item For odd $n$, the time ordered expectation values just vanish
  \begin{equation}
    \bra{0} T \phi_1 \dots \phi_n \ket{0} = 0
  \end{equation}
  \item For even $n$, we have a sum over all possible Feynman propagators
  \begin{equation}
    \label{eq:time-ordered-product-wicks-theorem-sandwiched}
    \bra{0} T \phi_1 \dots \phi_n \ket{0} = \sum_{} \Delta_F (x_{i_1} - x_{i_2}) \Delta_F (x_{i_3} - x_{i_4}) \dots \Delta_F( x_{i_{n-1}} - x_{i_n}),
  \end{equation}
  where the sum is over all symmetric (even) permutations of $\left\{ i_1, i_2, \dots, i_n \right\}$.
\end{itemize}
There is a nice graphical representation of this: We mark $x_1, x_2, \dots, x_n$ as points $\to \bullet$. The sum of equation \eqref{eq:time-ordered-product-wicks-theorem-sandwiched} is then obtained by connecting pairs of points in all possible ways and assigning to each line a propagator factor.
\begin{example}[]
  For four fields, we draw four dots, representing $x_1, \dots, x_4$, and connect them in all possible ways:
  \begin{equation}
    \bra{0} T \phi_1 \phi_2 \phi_3 \phi_4 \ket{0} = 
    \quad
    \begin{gathered}
      \feynmandiagram[scale=0.5, transform shape][small, horizontal=a to b] {
	a [dot] -- b [dot],
	c [dot] --  d [dot],
	a -- [draw=none] c,
	b -- [draw=none] d,
      };
    \end{gathered}
    \quad
    + 
    \quad
    \begin{gathered}
      \feynmandiagram[scale=0.5,transform shape][small, horizontal=a to b] {
	a [dot] -- [draw=none] b [dot],
	c [dot] -- [draw=none] d [dot],
	a -- c,
	b -- d,
      };
    \end{gathered}
    \quad
    +
    \quad
    \begin{gathered}
      \feynmandiagram[scale=0.5,transform shape][small, horizontal=a to b] {
	a [dot] -- [draw=none] b [dot],
	c [dot] -- [draw=none] d [dot],
	a -- [draw=none] c,
	b -- [draw=none] d,
	a -- [solid] d,
	c -- [solid] b,
      };
    \end{gathered}
  \end{equation}
  Replacing each line with a propagator, we see that this equation is equivalent to our previous result Equation \eqref{eq:four-field-tordered-product}.
\end{example}

\begin{example}[]
  For a $\mathbb{C}$ scalar $\psi$, the time ordered product is
  \begin{equation}
    T \left\{ \psi(x) \psi^*(y) \right\} = \normalorder{\psi(x) \psi^*(y) } + \Delta_F(x - y).
  \end{equation}
  So here we define $ \wick{\c\psi(x)\c\psi^*(y)} = \Delta_F(x - y)$, whilst $\wick{\c\psi(x)\c\psi(y)} = 0$ and $\wick{\c\psi^*(x)\c\psi^*(y)} = 0$.
\end{example}

\section{Nucleon-Nucleon Scattering}%
\label{sec:nucleon_nucleon_scattering}

Let us apply Wick's theorem to nucleon-nucleon scattering:
\begin{equation}
  \psi(p_1) \psi(p_2) \to \psi(p_1') \psi(p_2').
\end{equation}
Then the initial and final states are
\begin{align}
  \ket{i} &= \sqrt{4 E_{p_1} E_{p_2}} b_{\vb{p}_1}^{\dagger} b_{\vb{p}_2}^{\dagger} \ket{0} \coloneqq \ket{\vb{p}_1, \vb{p}_2} \\
  \ket{f} &= \sqrt{4 E_{p_1'} E_{p_2'}} b^{\dagger}_{\vb{p}_1'} b^{\dagger}_{\vb{p}_2'} \ket{0} \coloneqq \ket{\vb{p}_1', \vb{p}_2'}.
\end{align}
We are not interested in processes where no scattering happens. Therefore, we look at terms in $\bra{f}(S-1)\ket{i}$ at $O(g^2)$:
\begin{equation}
  \frac{(-ig)^2}{2!} \int \dd[4]{x_1} \dd[4]{x_2} \bra{\vb{p}_1', \vb{p}_2'} T \left\{ \psi^*(x_1) \psi(x_1) \phi(x_1) \psi^*(x_2) \psi(x_2) \phi(x_2) \right\} \ket{\vb{p}_1, \vb{p}_2}.
\end{equation}
Using Wick's theorem, there exists a term in this of the form
\begin{equation}
  \normalorder{\psi^* (x_1) \psi(x_2) \psi^*(x_2) \psi(x_2) } \wick{\c\phi(x_1)\c\phi(x_2)},
\end{equation}
which will contribute: the $\psi$s will annihilate the initial nucleons whereas the $\psi^*$s will create the final ones. All other terms will vanish once they are commuted through to the vacuum.
\begin{align}
  M &=\bra{\vb{p}_1', \vb{p}_2'} \normalorder{\psi^*(x_1) \psi(x_1) \psi^*(x_2) \psi(x_2) } \ket{\vb{p}_1, \vb{p}_2} \\
  \begin{split}
    &= \int \frac{\dd[3]{q_1} \dots \dd[3]{q_4}}{\sqrt{2 E_{q_1} \dots 2E_{q_4}}} \sqrt{16 E_{p_1} E_{p_2} E_{p_1'} E_{p_2'}} \bra{0} b_{\vb{p}_1'} b_{\vb{p}_2'} b^{\dagger}_{\vb{q}_1} b^{\dagger}_{\vb{q}_2} b_{\vb{q}_3} b_{\vb{q}_4} b^{\dagger}_{\vb{p}_1} b^{\dagger}_{\vb{p}_2} \ket{0} \\
    & \qquad \times e^{i (q_1 \cdot x_1 + q_2 \cdot x_2 - q_3 \cdot x_1 - q_4 \cdot x_2)}.
  \end{split}
\end{align}
