% lecture notes by Umut Özer
% course: qft
\lhead{Lecture 15: November 14}

\section{Chiral Spinors}%
\label{sec:chiral_spinors}

Since $S[\Lambda]$ is block-diagonal, it is \emph{reducible}.
In other words, we may decompose it into two irreps, $\psi = 
\begin{pmatrix}
u_L \\
u_{R} \\
\end{pmatrix}
$, where $u_{L} , u_{R}$ are two $\mathbb{C}$-component objects: Weyl or Dirac spinors. They transform identically under rotations, but oppositely under boosts:
\begin{equation} \label{eq:15-1}
u_{L/R} \to e^{i \boldsymbol\phi \cdot \boldsymbol \sigma / 2} u_{L/R} \qquad
u_{L/R} \to e^{\mp\boldsymbol \chi \cdot \boldsymbol \sigma / 2} u_{L/R}
\end{equation}

\begin{equation}
  (SU(2) , SU(2)) \colon \quad u_L \sim \qty(\frac{1}{2}, 0),\; u_R \sim \qty(0, \frac{1}{2}),\quad \psi \text{ is in } \qty(\frac{1}{2}, 0) \oplus \qty(0, \frac{1}{2}).
\end{equation}

\section{The Weyl Equation}%
\label{sec:the_weyl_equation}

\begin{equation}
  \mathcal{L}_D = \bar \psi (i \cancel{\partial} - m) \psi = i u^{\dagger}_L \sigma^{\mu} \partial_{\mu} u_L + i u_R^{\dagger} \overline{\sigma}^{\mu} \partial_{\mu} u_R - m(u_L^{\dagger} u_R - u_R^{\dagger} u-L),
\end{equation}
where $\sigma^{\mu} \coloneqq (\mathbb{1}_2, \boldsymbol\sigma)$, $\overline{\sigma}^{\mu} \coloneqq (\mathbb{1}_2, -\boldsymbol\sigma)$
A massive fermion requires both $u_L, u_R$. A massless particle $m = 0$ requires only $u_L$ or $u_R$.
\begin{equation}
  m = 0: \qquad
  \left.
  \begin{aligned}
    i \sigma^{\mu} \partial_{\mu} u_L &= 0 \\
    i \overline{\sigma}^{\mu} \partial_{\mu} u_R &= 0
  \end{aligned}
  \;
  \right\}
  \text{Weyl's equations}
\end{equation}
In classical particle mechanics, the number of degrees of freedom $N$ is half the dimension of phase space
(If it is a field theory, this is really the number of degrees of freedom at each spacetime point).
For a scalar field $\phi$, we have $\pi_\phi = \dot\phi$, so $N = \frac{1}{2} \times 2 = 1$.
For a spinor $\psi_{\alpha}$, we have $\pi_{\psi} = i \psi^{\dagger}$ (not $\dot{\psi}$), so the number of degrees of freedom is $N = \frac{1}{2} \times 8 = 4$ (spin-$\uparrow$, spin-$\downarrow$ particle $\times$ spin-$\uparrow$, spin-$\downarrow$ anti-particle).

\subsection{\texorpdfstring{$\gamma^5$-}{Gamma five }Matrix}%
\label{sub:gamma_five_matrix}

For other bases, $S[\Lambda]$ is not necessarily block-diagonal.
$\gamma^{\mu} \to U \gamma^{\mu} U^{-1}$, $\psi \to U \psi$. Let $\gamma^5 \coloneqq i \gamma^0 \gamma^1 \gamma^2 \gamma^3$ to define Weyl spinors in a basis-independent way.
\begin{equation}
  \left\{ \gamma^{\mu}, \gamma^5 \right\} = 0, \qquad (\gamma^5)^2 = \mathbb{1}_4.
\end{equation}
Define projection operators $P_L \coloneqq \frac{1}{2} (\mathbb{1}) - \gamma^5$, and $P_R \coloneqq \frac{1}{2} (\mathbb{1} + \gamma^5)$. These satisfy $P_{L/R}^2 = P_{L/R}$ and $P_L P_R = 0 = P_R P_L$.
We then call $\psi_L \coloneqq P_L \psi$ \emph{left-handed} and $\psi_R \coloneqq P_R \psi$ \emph{right-handed}.
Under Lorentz transformation, this behaves like a \emph{pseudo scalar}, which means that
\begin{align}
  \overline{\psi}(x) \gamma^5 \psi(x) \xrightarrow{LT} &\overline{\psi}(\Lambda^{-1}x) S[\Lambda]^{-1} \gamma^{-5} S[\Lambda] \psi \\
						       &=\overline{\psi}(\Lambda^{-1} x) \gamma^5 \psi (\Lambda^{-1} x).
\end{align}
Check: $[S_{\mu\nu,} \gamma^5] = 0$, $\overline{\psi}\gamma^5 \gamma^{\mu} \psi$. We call this an \emph{axial vector}.

\subsection{Parity}%
\label{sub:parity}

$\psi_L$ and $\psi_R$ are related by \emph{parity} transformations. This is a symmetry similar to others we have already met:
\begin{align}
  &\text{Lorentz group:} &x^{\mu} &\to \Lambda \indices{^{\mu}_{\nu}} x^{\nu} \text{ s.t } \Lambda \indices{^{\mu}_{\nu}} \Lambda \indices{^{\rho}_{\sigma}} \eta^{\nu\sigma} = \eta^{\mu\rho} \\
  &\text{Time reversal $T$:} &x^0 &\to - x^0, \quad x^i \to x^i \\
  &\text{Parity $P$:} &x^0 &\to x^0, \quad x^i \to -x^i.
\end{align}
Note that the last two, time reversal and parity, are disconnected from identity!
Under $P$, rotations do not change sign, but boosts do. Using \eqref{eq:15-1}, this means that $P$ exchanges $u_L \leftrightarrow u_R$.
Recall that we defined $\psi_{L/R} \coloneqq \frac{1}{2} (\mathbb{1} \mp \gamma^5)$. Hence, $P$ exchanges left-handed and right-handed spinors:
\begin{equation}
  P\colon \quad \psi_{L/R} (\vb{x,} t) \to \psi_{R/L} (-\vb{x,} t), \qquad \text{i.e. }\; P \colon \quad \psi(\vb{x,} t) \to \gamma^0 \psi(-\vb{x,} t).
\end{equation}
Terms in $\mathcal{L}$:
\begin{align}
  \overline{\psi}\psi(\vb{x}, t) &\xrightarrow{P} \overline{\psi}\psi(-\vb{x}, t) \qquad &\text{scalar} \\
  \overline{\psi}\gamma^{\mu} \psi(\vb{x} , t) &\xrightarrow{P} 
  \begin{cases}
    \mu = 0: & \overline{\psi} \gamma^0 \psi(-\vb{x}, t) \\
    \mu = i: & \overline{\psi} \gamma^0 \gamma^i \gamma^0 \psi(-\vb{x}, t) = -\overline{\psi}\gamma^i \psi(-\vb{x}, t)
  \end{cases}
  &\text{vector} \\
  \overline{\psi} \gamma^5 \psi(\vb{x}, t) &\xrightarrow{P} \overline{\psi} \gamma^0 \gamma^5 \gamma^0 \psi(-\vb{x}, t) = -\overline{\psi} \gamma^5 \psi & \text{pseudo scalar} \\
  \overline{\psi}\gamma^5 \gamma^{\mu} \psi &\xrightarrow{P} \overline{\psi}\gamma^0 \gamma^5 \gamma^{\mu} \gamma^0 \psi = 
  \begin{cases}
    \mu =0: & -\overline{\psi} \gamma^5 \gamma^0 \psi \\
    \mu = i: & \overline{\psi}\gamma^5 \gamma^i \psi
  \end{cases}
					    & \text{axial vector}
\end{align}

\begin{table}[htpb]
  \centering
  \caption{}
  \label{tab:15-1}
  \begin{tabular}{c  c}
    Summary: & #\\
    $\overline{\psi}\psi$ & 1 \\
    $\overline{\psi} \gamma^{\mu} \psi$ & 4 \\
    $\overline{\psi} \delta^{\mu\nu} \psi$ & 6 \\
    $\overline{\psi} \gamma^5\psi$ & 1 \\
    $\overline{\psi}\gamma^5 \gamma^{\mu} \psi$ & 4 \\
  \end{tabular}
\end{table}
