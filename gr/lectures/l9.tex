% lecture notes by Umut Özer
% course: gr
\lhead{Lecture 9: November 01}
\begin{claim}
  Given a connection, we can construct two tensors.
  Let $\omega \in \Lambda^{1}(\mathcal{M})$ and $X, Y \in \mathfrak{X}(\mathcal{M})$, then
  \begin{description}
    \item[Torsion] is a rank $(1, 2)$-tensor (field)
      \begin{equation}
        T(\omega; X, Y) \coloneqq \omega(\nabla_X Y - \nabla_Y X - [X, Y])
      \end{equation}
      We can also think of $T$ as a map
      \begin{equation}
        \begin{split}
          T \colon \mathfrak{X}(\mathcal{M}) \times \mathfrak{X}(\mathcal{M}) \ &\to\  \mathfrak{X}(\mathcal{M}) \\
  	(X, Y) \ &\mapsto\  T(X, Y) = \nabla_X Y - \nabla_Y X - [X, Y].
        \end{split}
      \end{equation}
    \item[Curvature] is a rank $(1, 3)$-tensor
      \begin{equation}
        R(\omega; X, Y, Z) = \omega (\nabla_X \nabla_Y Z - \nabla_Y \nabla_X Z - \nabla_{[X, Y]} Z)
      \end{equation}
      This is the \emph{Riemann tensor}. We can also think of it as a map
      \begin{equation}
        \begin{split}
          R \colon \mathfrak{X}(\mathcal{M}) \times \mathfrak{X}(\mathcal{M})\ &\to\  \text{differential operator on } \mathfrak{X}(\mathcal{M}) \\
  	(X, Y) \ &\mapsto\  R(X, Y) = \nabla_X \nabla_Y - \nabla_Y \nabla_X - \nabla_{[X, Y]}
        \end{split}
      \end{equation}
  \end{description}
\end{claim}
\begin{proof}
  To show that these are tensors, we need to check linearity in all arguments.
  \begin{example}[]
    \begin{align}
      T(\omega; fX, Y) &= \omega(\nabla_{fX} Y - \nabla_{Y} (fX) - [fX, Y]) \\
		       &= \omega \left[f \nabla_X Y - f \nabla_Y X - Y(f) \cdot X - (f[X, Y] - Y(f) X)\right] \\
		       &= f \omega (\nabla_X Y - \nabla_Y X - [X, Y]) = f T(\omega; X, Y)
    \end{align}
  \end{example}
  Similar calculations for $X, Y, Z$ for torsion and curvature show that these are all linear.
\end{proof}

In a coordinate basis $\left\{ e_\mu = \partial_\mu \right\} $ and $\left\{ f^\mu = dx^\mu \right\}$, we have
\begin{align}
  T \indices{^\rho_\mu_\nu} &= T(f^\rho; e_\mu, e_\nu) \\
			    &= f^\rho (\nabla_{\mu} e_\nu - \nabla_{\nu} e_\mu - \underbrace{[e_\mu, e_\nu]}_{\text{vanishes in coord. basis}}) \\
			    &= \Gamma^\rho_{\mu\nu} - \Gamma^\rho_{\nu\mu}.
\end{align}
\begin{leftbar}
  \begin{remark}
    Therefore, we learn that although $\Gamma$ is not a tensor, the anti-symmetrisation of it is!
  \end{remark}
\end{leftbar}
\begin{leftbar}
  \begin{remark}
    In this lecture, since we never raise or lower indices on the Christoffel symbols, we do not care about the position of its indices.
  \end{remark}
\end{leftbar}
A connection with $\Gamma^\rho_{\mu\nu} = \Gamma^\rho_{\nu\mu}$ has $T \indices{^\rho_\mu_\nu} = 0$ and is said to be \emph{torsion free.}
We also have
\begin{equation}
  R \indices{^\sigma_\rho_\mu_\nu} = R (f^\sigma; e_\mu, e_\nu, e_\rho)
\end{equation}
\begin{leftbar}
  \begin{remark}
    This is a slightly odd ordering on the right hand side.
  \end{remark}
\end{leftbar}
\begin{align}
  \dots &= f^\sigma \left( \nabla_{\mu} \nabla_{\nu} e_\rho - \nabla_{\nu} \nabla_{\mu} e_\rho - \nabla_{[e_\mu, e_\nu]} e_\rho \right) \\
	&= f^\sigma (\nabla_{\mu} (\Gamma^\lambda_{\nu\rho} e_\lambda) - \nabla_{\nu} (\Gamma^\lambda_{\mu\rho} e_\lambda) \\
	&=\partial_\mu \Gamma^\sigma_\nu\rho - \partial_\nu \Gamma^\sigma_{\mu\rho} + \Gamma^\lambda_{\nu\rho} \Gamma^\sigma_{\mu\lambda} - \Gamma^\lambda_{\mu\rho} \Gamma^\sigma_{\nu\lambda}.
\end{align}
\begin{leftbar}
  \begin{remark}
    We mentioned that $\Gamma$ transforms similarly to the gauge potential. We have a similar thing in Yang-Mills theory. This Riemann tensor is to GR and curvature $R$ what the field strength $F$ is to Maxwell theory.
  \end{remark}
\end{leftbar}
Clearly, we have anti-symmetry in the last indices:
\begin{equation}
  R \indices{^\sigma_\rho_\mu_\nu} = -R \indices{^\sigma_\rho_\nu_\mu}.
\end{equation}
There are a few more subtle symmetry properties of $R$, which we will prove in the upcoming sections.

So far, the connection is completely independent of the metric. However, it turns out that, given a metric, we can define a natural connection from it.

\subsection{Levi-Civita Connection}%
\label{sub:levi_civita_connection}

\begin{theorem}[Fundamental Theorem of Riemannian Geometry]
  There exists a unique, torsion-free connection with the property $\nabla_{X} g = 0$ for all vector fields $X \in \mathfrak{X}(\mathcal{M})$.
\end{theorem}
\begin{proof}
  Suppose that the connection exists.
  We then follow a series of manipulation which gives us the desired result.
  Consider the object $X(g(Y, Z))$. Since $g(Y, Z)$ is a function, we have
  \begin{equation}
    X(g(Y,Z)) = \nabla_{X} [g(Y, Z)]
  \end{equation}
  By the Leibniz rule, we have
  \begin{align}
    \dots &= \nabla_{X} g(Y, Z) + g(\nabla_{X}Y, Z) + g(Y, \nabla_{X}Z) \\
	  &= g(\nabla_{X}Y, Z) + g(Y, \nabla_{X}Z) \label{eq:l9e1}
  \end{align}
  However, the torsion vanishes by assumption. Therefore we have
  \begin{equation}
    \nabla_{X}Y - \nabla_{Y}X = [X, Y].
  \end{equation} 
  We use this equation in \eqref{eq:l9e1} to find that
  \begin{equation}
    \label{eq:one}
    X(g(Y, Z)) = g(\nabla_{Y} X, Z ) + g(\nabla_{X} Z , Y) + g([X, Y], Z).
  \end{equation}
  Now we repeat these exact same calculations twice over, cyclically permuting $X, Y$, and $Z$. These give
  \begin{align}
    \label{eq:two}
    Y(g(Z, X)) &= g(\nabla_{Z} Y, X ) + g(\nabla_{Y} X , Z) + g([Z, Y], X), \\
    \label{eq:three}
    Z(g(X, Y)) &= g(\nabla_{X} Z, Y ) + g(\nabla_{Z} Y , X) + g([Z, X], Y).
  \end{align}
  Then, taking \eqref{eq:one} $+$ \eqref{eq:two} $-$ \eqref{eq:three}, we have
  \begin{multline}
    g(\nabla_{Y}X, Z) = \frac{1}{2} \bigl[ X(g(Y,Z) ) + Y(g(Z, X)) - Z(g(X,Y)) \\
    - g([X, Y], Z) - g([Y, Z], X) + g([Z, X], Y) \bigr].
  \end{multline}
  In a coordinate basis $\left\{ e_\mu = \partial_\mu \right\}$, this becomes
  \begin{align}
    g(\nabla_{\nu} e_\mu, e_\rho) &= \Gamma^\lambda_{\nu\mu} g_{\lambda\rho} = \frac{1}{2} (\partial_\mu g_{\nu\rho} + \partial_\nu g_{\mu\rho} - \partial_\rho g_{\mu\nu}) \\
				  &= \Gamma^\lambda_{\mu\nu} = \frac{1}{2} g^{\lambda\rho} (\partial_\mu g_{\nu\rho} + \partial_\nu g_{\mu\rho} - \partial_\rho g_{\mu\nu})
  \end{align}
  This is the \emph{Levi-Civita connection} and the explicit representation $\Gamma^\lambda_{\mu\nu}$ are the \emph{Christoffel symbols}.
  \begin{exercise}
    We showed that this is the unique form of the connection if it exists. We now still have to show that it is actually a connection, by considering how it transforms.
  \end{exercise}
\end{proof}

We basically only introduced torsion to show that given a metric, a torsion-free connection is this unique one.
From now on, we will talk about this object when we talk about a connection.

\begin{theorem}[Divergence theorem]
  \label{thm:divergence}
  Consider a Riemannian manifold $\mathcal{M}$ with metric $g$ and boundary $\partial \mathcal{M}$.
  Let $n^\mu$ be an outward-pointing unit vector orthogonal to (any tangent vectors in) the boundary.
  Then, for any $X^\mu$, 
  \begin{equation}
    \int_{\mathcal{M}} \dd[n]{x} \sqrt{\abs{g}} \nabla_{\mu} X^\mu = \int_{\partial \mathcal{M}} \dd[n-1]{x} \sqrt{\abs{\gamma}} n_\mu X^\mu,
  \end{equation}
  where $\gamma_{ij}$ is the pull-back of $g$ onto $\partial \mathcal{M}$.
  On a Lorentzian manifold, this also holds with $\sqrt{g} \to \sqrt{-g}$ provided that $\partial \mathcal{M}$ is purely timelike or purely spacelike.
\end{theorem}
\begin{leftbar}
  \begin{remark}
    Note that this follows on from Stokes' theorem. However, we did not prove Stokes' theorem, and since we will use the divergence theorem over and over again, we will prove this explicitly.
  \end{remark}
\end{leftbar}
\begin{lemma}
  \label{lem:chri}
  \begin{equation}
    \Gamma^\mu_{\mu\nu} = \frac{1}{\sqrt{g}} \partial_\nu \sqrt{g}
  \end{equation}
\end{lemma}
\begin{proof}[Proof of Lemma \ref{lem:chri}]
  Introducing the notation that $\hat g$ is a matrix, we have
  \begin{align}
    \Gamma^\mu_{\mu\nu} &= \frac{1}{2} g^{\mu\rho} \partial_\nu g_{\mu\rho} = \frac{1}{2} \tr(\hat g^{-1} \partial_\nu \hat g) \\
			&= \frac{1}{2} \tr (\partial_\nu \log \hat g) \\
			&= \frac{1}{2} \partial_\nu \log \det \hat g, \qquad \tr \log A = \log \det A \\
			&= \frac{1}{2} \frac{1}{\det \hat g} \partial_\nu \det \hat g \\
			&= \frac{1}{\sqrt{\det \hat g}} \partial_\nu \sqrt{\det \hat g} \\
			&= \frac{1}{\sqrt{g}} \partial_\nu \sqrt{g}
  \end{align}
\end{proof}
