% lecture notes by Umut Özer
% course: gr
\lhead{Lecture 3: October 16}
\section{Lie Derivatives}%
\label{sec:lie_derivatives}

We have learned that vector fields differentiate scalar functions. But how can we differentiate vector fields?
Given $X, Y \in \mathfrak{X}(\mathcal{M})$, is there some way to know how $Y$ changes in the direction of $X$?
There is a problem here. For a function $f\colon \mathbb{R} \to \mathbb{R}$, differentiation is defined as
\begin{equation}
\dv{f}{t} = \lim_{h \to 0} \left\{ \frac{f(t + h) - f(t)}{h} \right\}.
\end{equation}
Similarly, to differentiate $Y \in \mathfrak{X}(\mathcal{M})$, we need to compare $Y_p \in T_p \mathcal{M}$ with some neighbouring $Y_q \in T_q \mathcal{M}$. The problem is that these are in different vector spaces! As each vector space has different coordinates, simply working with component values gives us a bad definition of a derivative; one which is dependent of coordinate choice. We cannot just add or subtract one vector from another.
To differentiate $Y$, we need to understand how to map vectors in $T_q (\mathcal{M})$ to vectors in $T_p(\mathcal{M})$, so we can compare them.

\subsection{Push-Forward and Pull-Back}%
\label{sub:push_forward_and_pull_back}

First, we think more generally about maps of the form $\varphi: \mathcal{M} \to \mathcal{N}$, where $\mathcal{N}$ and $\mathcal{M}$ can be manifolds of different dimensions.
\begin{definition}[pull-back and push-forward]
  Let $\varphi\colon \mathcal{M} \to \mathcal{N}$ be a smooth map between manifolds, $f\colon \mathcal{N} \to \mathbb{R}$ a smooth function on $\mathcal{N}$, and $Y\colon C^\infty(\mathcal{M}) \to C^\infty(\mathcal{M})$ a vector field on $\mathcal{M}$.

  The \emph{pull-back} of $f$ by $\varphi$ is $(\varphi^* f) = f \circ \varphi \colon \mathcal{M} \to \mathbb{R}$, a smooth function on $\mathcal{M}$.

  The \emph{push-forward} of $Y$ by $\varphi$ is $(\varphi_* Y) \colon C^\infty(\mathcal{N}) \to C^\infty(\mathcal{N})$ is a vector field on $\mathcal{N}$ defined by
  \begin{equation}
    \varphi^* \left[ (\varphi_* Y)(f) \right] = Y(\varphi^*f) 
  \end{equation}
\end{definition}

Let $x^\mu$ be coordinates on $\mathcal{M}$ and $y^a$ be coordinates on $Y$ such that $\varphi(x) = y^a(x)$. Let $Y: Y^\mu \partial_\mu \in \mathfrak{X}(\mathcal{M})$ be a vector field on $\mathcal{M}$, then we can write the push-forward of $Y$ to $\mathcal{N}$ as 
\begin{equation}
  (\varphi_* Y) (f) = Y^\mu \pdv{f(y(x))}{x^\mu} = Y^\mu \pdv{y^a}{x^\mu} \pdv{f(y)}{y^a}.
\end{equation}
In components:
\begin{equation}
  \label{eq:push-forward-of-tangent-vector}
  (\varphi_* Y)^a = Y^\mu \pdv{y^a}{x^\mu}.
\end{equation}
Some objects are naturally pulled back and some are pushed forward. Note that if $\varphi$ is a diffeomorphism, then we can go both ways by using the inverse $\varphi^{-1}$.

\subsection{Differentiation by Lie Derivative}%
\label{sub:the_lie_derivative}

Now we can use this to differentiate. The idea is that given a vector field $X \in \mathfrak{X}(\mathcal{M})$, we can define a flow $\sigma_t\colon \mathcal{M} \to \mathcal{M}$. We use this to push forward vectors at $T_p \mathcal{M}$ to $T_{\sigma_t (p)} \mathcal{M}$.
This is called the \emph{Lie derivative}, denoted $\mathcal{L}_X$.
Let us first look at functions:
\begin{align}
  \mathcal{L}_X f &= \lim_{t \to 0} \left\{ \frac{f(\sigma_t(x)) - f(x)}{t} \right\} \\
		  &= \left.\dv{f(\sigma_t(x))}{t}\right\rvert_{t=0} = \left.\pdv{f}{x^\mu}\right. \left.\pdv{x^\mu}{t}\right\rvert_{t=0} \\
		  &= X^\mu \left.\pdv{f}{x^\mu}\right. = X(f)
\end{align}
As $X^\mu = \left.\dv{x^\mu}{t}\right.$, we have
\begin{equation}
  \mathcal{L}_X f = X^\mu(x) \left.\pdv{f}{x^\mu}\right. = X(f).
\end{equation}
The action of the Lie derivative $\mathcal{L}_X$ on a function is the same as the action of the vector field $X$.
Now we differentiate vector fields:
\begin{figure}[htpb]
  \centering
  \def\svgwidth{0.85\columnwidth}
  \input{lectures/liediffvf.pdf_tex}
  \caption{The Lie derivative uses the push-forward $(\sigma_{-t})_*$ to be able to compare vectors at a point $p \in \mathcal{M}$ and a bit further along the flow at $\sigma_t(p)$ defined by the vector field $X$.}
  \label{fig:liediffvf}
\end{figure}
The Lie derivative of a vector field $Y$ is defined by
\begin{equation}
  \mathcal{L}_X Y = \lim_{t\to 0} \left\{ \frac{((\sigma_{-t})_*Y) - Y}{t} \right\}.
\end{equation}
The minus sign is needed to push with the inverse flow map, $\sigma_{-t}\colon \sigma_t(p) \mapsto p$ as illustrated in Fig.~\ref{fig:liediffvf}.
Let us calculate the action of the Lie derivative on a basis $\{ \partial_\mu = \left.\pdv*{}{x^\mu}\right. \}$:
\begin{equation}
  \mathcal{L}_X \partial_\mu = \lim_{t \to 0} \left\{ \frac{(\sigma_{-t})_* \partial_\mu - \partial_\mu}{t} \right\}.
\end{equation}
The components of the push-forward are
\begin{equation}
  (\varphi_* f)^\alpha = Y^\mu \left.\pdv{y^\alpha}{x^\mu}\right.
\end{equation}
Now, we use the expression for the push-forward of a tangent vector as given by Equation \eqref{eq:push-forward-of-tangent-vector}, except that we replace the coordinates $y^a$ by the infinitesimal coordinate change induced by the flow $\sigma_t$:
\begin{equation}
  y^\mu = x^\mu -t X^\mu + \dots
\end{equation}
Therefore, we have for small $t$:
\begin{equation}
  (\sigma_{-t})_* \partial_\mu = (\delta^\nu_\mu -t \pdv{X^\nu}{x^\mu} + \dots) \partial_\nu.
\end{equation}
And finally, we see that the Lie derivative acts on a coordinate basis as
\begin{equation}
  \label{eq:3-1}
  \mathcal{L}_X \partial_\mu = -\pdv{X^\nu}{x^\mu} \partial_\nu.
\end{equation}
To find out how the $\mathcal{L}_X$ acts on a general vector field $Y = Y^\mu \partial_\mu$, we can use Leibniz' formula
\begin{align}
  \mathcal{L}_X Y &= \mathcal{L}_X (Y^\mu \partial_\mu) = \mathcal{L}_X (Y^\mu) \partial_\mu + Y^\mu \mathcal{L}_x \partial_\mu \\
		  &= X^\nu \pdv{Y^\mu}{x^\nu} \partial_\mu - Y^\mu \pdv{X^\nu}{x^\mu} \partial_\nu \\
		  &= [X, Y].
\end{align}
This is a surprising fact! We can think of the commutator as describing how $Y$ changes in the direction of $X$.
\begin{corollary}
  From the Jacobi identity of commutators, we have for vector fields $X, Y, Z$:
  \begin{equation}
    [\mathcal{L}_X, \mathcal{L}_Y] Z = \mathcal{L}_{[X, Y]}Z.
  \end{equation}
\end{corollary}
