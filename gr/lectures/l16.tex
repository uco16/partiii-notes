% lecture notes by Umut Özer
% course: gr
\lhead{Lecture 16: November 18}
\section{Symmetries}%
\label{sec:symmetries}

There are three basic spacetimes in GR.
We have intuition of why Minkowski is important; the other two have so far just been pulled out of thin air. The thing that makes these spaces special are the symmetries.
They are the \emph{maximally symmetric} spacetimes in general relativity.

Let us first define what a symmetry is.
Any intuition you have about this is probably correct. 
Consider a sphere; it has the symmetry $SO(3)$. A rugby ball on the other hand, with the same topology, has only the symmetry $SO(2)$.
(In general, this would be $O(3)$, but in these lectures we will only consider groups which are continuously connected to the identity.)

Consider a one-parameter family of diffeomorphisms $\sigma_t \colon \mathcal{M} \to \mathcal{M}$.
Recall that this is associated to a vector field
\begin{equation}
  K^{\mu} = \dv[]{x^{\mu}}{t},
\end{equation}
which is tangent to the flow lines.
We will define a symmetry by flowing from one point to a closeby one; if the point looks the same as the point we came from, this is a symmetry.
More specifically, this flow is called an \emph{isometry} if the metric looks the same at each point along the flow.
From Claim \ref{claim:14-metric-change-diffeo}, we know that this translates to 
\begin{equation}
  \label{eq:killing}
  \mathcal{L}_K g = 0 \quad \iff \quad \nabla_{\mu} K_{\nu} + \nabla_{\nu} K_{\mu} = 0.
\end{equation}
This is the \emph{Killing equation}. Any $K$ obeying this is called a \emph{Killing vector}. These describe the symmetries of the metric.
\begin{leftbar}
  \begin{remark}
    How is this related to the Killing form of \emph{Symmetries}?
  \end{remark}
\end{leftbar}
\begin{leftbar}
  \begin{remark}
    $\mathcal{L}_X \mathcal{L}_Y - \mathcal{L}_Y \mathcal{L}_X = \mathcal{L}_{[X, Y]}$.
    This means that there is a Lie algebra structure emerging from continuous symmetries of metrics.
  \end{remark}
\end{leftbar}

\begin{example}[Minkowski Space]
  The Killing equation \eqref{eq:killing} becomes $\partial_{\mu} K_{\nu} + \partial_{\nu} K_{\mu} = 0$. The solutions to this are
  \begin{equation}
    K_{\mu} = c_{\mu} + \omega_{\mu\nu} x^{\nu} ,
  \end{equation}
  where $c_{\mu}$ are translations and $\omega_{\mu\nu} = -\omega_{\nu\mu}$ correspond to rotations or boosts.
  These are the components of the Killing vector.
  Similarly, we can define the Killing vectors themselves as
  \begin{equation}
    P_{\mu} = \frac{\partial^{} }{\partial x^{\mu}} \quad \text{and} \quad M_{\mu\nu} = \eta_{\mu\rho} x^{\rho} \frac{\partial^{} }{\partial x^{\nu}} - \eta_{\nu\rho} x^{\rho} \frac{\partial^{} }{\partial x^{\mu}}.
  \end{equation}
  Once we have two Killing vectors, we can compute their commutator as 
  \begin{equation}
    \begin{gathered}
      [P_{\mu,} P_{\nu}] = 0 \qquad
      [M_{\mu\nu} , M_{\rho\sigma}] = \eta_{\mu\sigma} \\
      [M_{\mu\nu} , P_{\sigma}] = -\eta_{\mu\sigma} P_{\nu} + \eta _{\sigma\nu} P_{\mu}
    \end{gathered}
  \end{equation}
  These are the commutation relations of the Poincar\'e group.
\end{example}
\begin{example}[]
  The isometries of \texttt{dS} and \texttt{AdS} are inherited from the $5d$ embedding.
  \begin{description}
    \item[dS] has isometry group $SO(1, 4)$
    \item[AdS] has isometry group $SO(2, 3)$
    \item[Note:] both groups have $\dim(10)$, just like the Poincar\'e group of $\mathbb{M}^4$. Each of these three spaces is just as symmetric as the other two.
  \end{description}
  In $5d$, the Killing vectors are 
  \begin{equation}
    M_{AB} = \eta_{AC} X^{C} \frac{\partial^{} }{\partial X^{B}} - \eta_{BC} X^{C} \frac{\partial^{} }{\partial X^{A}},
  \end{equation}
  where $\eta = (- + + +)$ in \texttt{dS} and $\eta = (- - + + )$ in \texttt{AdS}. And $X^{A}$ with $A = 0,1,2,3,4$.
  The flows induced by $M_{AB}$ map the embedding hyperboloid to itself. This implies that these are isometries of \texttt{(A)dS}.
\end{example}
\begin{claim}
  If the metric $g_{\mu\nu}(x)$ does not depend on some coordinate $y$, then $K = \frac{\partial^{} }{\partial y}$ is a Killing vector.
\end{claim}
\begin{proof}
  This is easy to see when we look at what the Lie derivative does:
  \begin{equation}
    (\mathcal{L}_{\frac{\partial^{} }{\partial y}} g)_{\mu\nu} = \frac{\partial^{} g_{\mu\nu}}{\partial y} = 0.
  \end{equation}
\end{proof}
\begin{example}[de Sitter in static patch coordinates]
  For the static patch, we expect $\frac{\partial^{} }{\partial t}$ to be a Killing vector.
  We had $X^0 = \sqrt{R^2 - r^2} \sinh( \frac{t}{R})$ and $X^4 = \sqrt{R^2 - r^2} \cosh ( \frac{t}{R})$.
  Look at $\frac{\partial^{}}{\partial t} = \frac{\partial^{} X^{a}}{\partial t} \frac{\partial^{} }{\partial X^{A}} = \frac{1}{R}(X^4 \frac{\partial^{} }{\partial X^0} + X^0 \frac{\partial^{} }{\partial X^4})$.
\end{example}
\begin{description}
  \item[Comment:] Timelike Killing vectors, such that $g_{\mu\nu} K^{\mu} K^{\nu} < 0$, are used to define energy.
    Both $\mathbb{M}^4$ and \texttt{AdS} have such objects. And \texttt{dS} has such an object in the static patch, but \emph{not} globally!
    Consider the Killing vector $K = X^4 \frac{\partial^{} }{\partial X^0} + X^0 \frac{\partial^{} }{\partial X^4}$. The first term increases the timelike direction $X^0$ when $X^4 > 0$ and decreases $X^0$ when $X^4 <0$.
    The Killing vector is positive and timelike only in the static patch.
    Elsewhere, it is spacelike. Energy is a subtle concept in \texttt{dS}!
\end{description}
\begin{leftbar}
  \begin{remark}
    Timelike geodesics do not take us out of the static patch; the problems with energy only arise when we think more globally than what a single particle is doing.
  \end{remark}
\end{leftbar}

\subsection{Conserved Quantities}%
\label{sub:conserved_quantities}

\begin{claim}
  Consider a particle moving on a geodesic $x^{\mu}(\tau)$ in a spacetime with some Killing vector $K^{\mu}$.
  Then $Q = K_{\mu} \dv[]{x^{\mu}}{\tau}$ is conserved.
\end{claim}
\begin{proof}
  To see this,
  \begin{align}
    \dv[]{Q}{\tau} &= \partial_{\nu} K_{\mu} \dv[]{x^{\nu}}{\tau} \dv[]{x^{\mu}}{\tau} + K_{\mu} \dv[2]{x^{\mu}}{\tau} \\
    &= \partial_{\nu} K_{\mu} \dv[]{x^{\nu}}{\tau} \dv[]{x^{\mu}}{\tau}  K_{\mu} \Gamma^{\mu}_{\rho\sigma} \dv[]{x^{\rho}}{\tau} \dv[]{x^{\sigma}}{\tau} \\
    &= \nabla_{\nu} K_{\mu} \dv[]{x^{\nu}}{\tau} \dv[]{x^{\mu}}{\tau} \stackrel{\eqref{eq:killing}}{=} 0.
  \end{align}
  We can also see this from the action
  \begin{equation}
    S = \int \dd[]{\tau} \mathbf{g}_{\mu\nu} (x) \dot{x}^{\mu} \dot{x}^{\nu}.
  \end{equation}
  Consider $\delta x^{\mu} (\tau) = K^{\mu}(x)$. We have
  \begin{equation}
    \delta S = \int \dd[]{\tau} \left[ \partial_{\rho} g_{\mu\nu} \dv{x^{\mu}}{\tau} \dv{x^{\nu}}{\tau} K^{\rho} + 2g_{\mu\nu} \dv{x^{\mu}}{\tau} \dv{x^{\nu}}{\tau} \right].
  \end{equation}
  Using 
  \begin{align}
    g_{\mu\nu} \dv{K^{\nu}}{\tau} &=\dv{K_{\mu}}{\tau} - \dv{g_{\mu\nu}}{\tau} K^{\rho} \\
				  &= (\partial_{\nu} K_{\mu} - \partial_{\nu} g_{\mu\nu} K^{\rho}) \dv{x^{\nu}}{\tau}
  \end{align}
  in the second argument, we have
  \begin{equation}
    \rightarrow \delta S = \int \dd[]{\tau} 2 \nabla_{\mu} K_{\nu} \dv{x^{\mu}}{\tau} \dv{x^{\nu}}{\tau},
  \end{equation}
  where the $\Gamma$ 's in $\nabla$ come from the $\partial g$ 's.
  \begin{exercise}
    Check this!
  \end{exercise}
  This means that we have
  \begin{equation}
    \delta S = 0 \quad \iff \quad \nabla_{(\mu)} K^{\nu)} = 0,
  \end{equation}
  which is the Killing equation.
\end{proof}
