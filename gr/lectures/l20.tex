% lecture notes by Umut Özer
% course: gr
\lhead{Lecture 20: November 29}

\section{Linearised Theory}%
\label{sec:linearised_theory}

We want to build the theory of the linear ripples $h_{\mu\nu}$.
We work to leading order in $h_{\mu\nu}$. We have the inverse metric
\begin{equation}
  g^{\mu\nu} = \eta^{\mu\nu} - h^{\mu\nu},
\end{equation}
where again we raise indices with the Minkowski metric.
The Christoffel symbols are
\begin{equation}
  \Gamma^{\sigma}_{\nu\rho} = \frac{1}{2} \eta^{\sigma\lambda}(\partial_{\nu} h_{\lambda\rho} + \partial_{\rho} h_{\nu\lambda} - \partial_{\lambda} h_{\nu\rho})
\end{equation}
and the Riemann tensor is
\begin{align}
  R\indices{^{\sigma}_{\rho\mu\nu}} &= \partial_{\mu} \Gamma^{\sigma}_{\nu\rho} - \partial_{\nu} \Gamma^{\sigma}_{\mu\nu} + \underbrace{\Gamma^{\lambda}_{\nu\rho} \Gamma^{\sigma}_{\mu\lambda} - \Gamma^{\lambda}_{\mu\rho} \Gamma^{\sigma}_{\nu\lambda}}_{\mathclap{O(h^2)}}. \\
				    &= \frac{1}{2} \eta^{\sigma\lambda} (\partial_{\mu} \partial_{\rho} h_{\nu\lambda} - \partial_{\mu} \partial_{\lambda} h_{\nu\rho} - \partial_{\nu} \partial_{\rho} h_{\mu\lambda} + \partial_{\nu} \partial_{\lambda} h_{\mu\rho}).
\end{align}
It is very easy working in linearised theory; all the hard stuff disappears.
The Ricci tensor is
\begin{equation}
  \label{eq:20-riem}
  R_{\mu\nu} = \frac{1}{2} (\partial^{\rho} \partial_{\mu} h_{\nu\rho} + \partial^{\rho} \partial_{\nu} h_{\mu\rho} - \Box h_{\mu\nu} - \partial_{\mu} \partial_{\nu} h),
\end{equation}
and finally the Ricci scalar is
\begin{equation}
  R = \partial^{\mu} \partial^{\nu} h_{\mu\nu} - \Box h.
\end{equation}
We have introduced two new objects
\begin{align}
  \Box &= \partial_{\mu} \partial^{\mu} \\
  h &= h_{\mu}^{\mu} = \eta^{\mu\nu} h_{\mu\nu}.
\end{align}

Finally, the Einstein tensor
\begin{equation}
  G_{\mu\nu} = \frac{1}{2} \left[ \partial^{\rho} \partial_{\mu} h_{\nu\rho} + \partial^{\rho} \partial_{\nu} h_{\mu\rho} - \Box h_{\mu\nu} - \partial_{\mu} \partial_{\nu} h - (\partial^{\rho} \partial^{\sigma} h_{\rho\sigma} - \Box h) h_{\mu\nu} \right].
\end{equation}
This obeys the linearised Bianchi identity $\partial_{\mu} G^{\mu\nu} = 0$ , which is the equation of motion in the absence of matter.

In this case, we did not start with the action. However, we could have start with an action which does derive this equation of motion.

\subsection{The Fierz-Pauli Action}%
\label{sub:the_fierz_pauli_action}

The Einstein equation $G_{\mu\nu} = 8 \pi G T_{\mu\nu}$  follows from the following action by Fierz and Pauli, written down in the 1930s, 
\begin{equation}
  S_{\text{FP}} = \int \dd[4]{x} \frac{1}{8\pi G} \left[ - \frac{1}{4} \partial_{\rho} h_{\mu\nu} \partial^{\rho} h^{\mu\nu} + \frac{1}{2} \partial_{\rho} h_{\mu\nu} \partial^{\nu} h^{\rho\mu} + \frac{1}{4} \partial_{\mu} h \partial^{\mu} h - \frac{1}{2} \partial_{\nu} h^{\mu\nu} \partial_{\mu} h \right] + h_{\mu\nu} T^{\mu\nu}
\end{equation}

The claim is twofold: if we vary this action w.r. t $h$ , we get the Einstein equation. The hard part of the claim is that $S_{\text{FP}}$  arises from the Einstein-Hilbert action $S_{\text{EH}}$  by expanding to $O(h^2)$ .

In particular, we have to include a lot of the second order terms (for example in the Riemann tensor).

We should think about this in very much the same way as we think of the Maxwell action; expanding out $S_{\text{MW}}$  in terms of $A$ rather than  $F$, and also having a term that couples to the current.

This action has a particular symmetry.

\subsection{Gauge Symmetry}%
\label{sub:gauge_symmetry}

Under an infinitesimal diffeomorphism $x^{\mu} \mapsto x^{\mu} - \xi^{\mu} (x)$, the metric changes as 
\begin{equation}
  \delta g_{\mu\nu} = (\mathcal{L}_{\xi} g)_{\mu\nu} = \nabla_{\mu} \xi_{\nu} + \nabla_{\nu} \xi_{\mu}.
\end{equation}
Assuming that $\xi^{\mu}$ are `small', in the same sense that $h$ is small, we view this as a gauge transformation of $g^{\mu\nu}$ which, to leading order, is
\begin{equation}
  h_{\mu\nu} \mapsto h_{\mu\nu} + \partial_{\mu} \xi_{\nu} + \partial_{\nu} \xi_{\mu}.
\end{equation}
In particular, the covariant derivatives of the Christoffel symbols, which depend on $h$, are dropped since they are of order $O(h \xi)$.

\begin{exercise}
  You can check that the Riemann tensor $R\indices{^{\mu}_{\rho\nu\sigma}}$ given by \eqref{eq:20-riem} and therefore also $S_{\text{FP}}$ are both invariant under gauge transformations.
\end{exercise}

\begin{leftbar}
  \begin{note}
    We assume that the derivatives $\partial \xi$ are also small.
  \end{note}
\end{leftbar}

\begin{remark}
  There is a very nice story here, dating back to Feynman and Weinberg that goes beyond this: let us think about which terms we can act to this action that respect this gauge symmetry.
  You cannot have a mass term; you need to have derivatives of $h$. Moreover, any term you add forces you to add more terms. In the end, you will end up adding up infinitely many terms, which reconstitute the expansion of the Eintein-Hilbert action.
  The claim is that there is essentially a unique way to do this: the Eintein-Hilbert action.

  This is the statement that a massless spin-$2$ particle needs gauge-invariance to make sense of the quantum theory and if you want interactions, you are obliged to come out with general relativity.
\end{remark}

We use gauge symmetry to choose a gauge; we pick \emph{de Donder \texttt{(dD)} gauge}, given by
\begin{equation}
  \partial^{\mu} h_{\mu\nu} - \frac{1}{2} \partial_{\nu} h = 0.
\end{equation}

This is analogous to Lorentz gauge $\partial_{\mu} A^{\mu} = 0$ in electromagnetism.

\begin{remark}
  In the full non-linear theory, the generalisation of \texttt{dD} is something quite elegant: $g^{\mu\nu} \Gamma^{\rho}_{\mu\nu} = 0$.

  These are both 4 equations. Notice that it not a tensor equation! But this is exactly what you want for gauge fixing.

  In \texttt{dD} gauge, the linearised Einstein equations become
  \begin{equation}
    \Box h_{\mu\nu} - \frac{1}{2} \eta_{\mu\nu} \Box h = -16 \pi G T_{\mu\nu}
  \end{equation}
  \begin{definition}[]
    Define $\overline{h}_{\mu\nu} = h_{\mu\nu} - \frac{1}{2} \eta_{\mu\nu} h$ 
  \end{definition}
  We then have $\overline{h} = \eta_{\mu\nu} \overline{h}^{\mu\nu} = -h$ and so $h_{\mu\nu} = \overline{h}_{\mu\nu} - \frac{1}{2} \overline{h}\eta_{\mu\nu}$. 
  And \begin{equation}\Box \overline{h}_{\mu\nu} = - 16 \pi G T^{\mu\nu}.\end{equation}
\end{remark}

\subsection{The Newtonian Limit}%
\label{sub:the_newtonian_limit}

Our aim will be to rederive Newtonian gravity in this framework.
Consider stationary matter with $T_{00} = \rho(\vb{x})$. Then $\Box = -\partial_t^2 + \nabla^2$ and we look for solutions with $\partial / \partial t = 0$. 
\begin{equation}
  \nabla^2 \overline{h}_{00} = -16 \pi G \rho(\vb{x}) \quad \text{and} \quad \nabla^2 \overline{h}_{0i} = \nabla^2 \overline{h}_{ij} = 0.
\end{equation}
These have the solution $\overline{h}_{0i} = \overline{h}_{ij} = 0$ and $\overline{h}_{00} = - 4 \Phi (\vb{x})$, which $\Phi$ the Newtonian gravitational potential, defined by solvint the Poisson equation
\begin{equation}
  \nabla^2 \Phi = 4 \pi G \rho(\vb{x}).
\end{equation}

\begin{remark}
  We are hiding the speed of light $c = 1$. By the time we put the speed of light in, the energy density $\rho$ becomes the mass density.
\end{remark}
\begin{remark}
  This looks so similar to electromagnetism, since it is in some sense the only thing we could get.
  There are quantum mechanical arguments, but classically, gauge symmetry allows you to have a well-defined intial value problem.
\end{remark}

This then gives $h_{00} = - 2 \Phi$, $h_{ij} = -2 \Phi \delta_{ij}$ , $h_{0i} = 0$ . 
The metric then becomes
\begin{equation}
  ds^2 = -(1 + 2 \Phi) dt^2 + (1 - 2 \Phi) d\vb{x}^2.
\end{equation}

\begin{example}[]
  Suppose that the mass-density was a point mass $m$ sitting at the origin.
  We have $\Phi = - G M /r$ as the solutions to the Poisson equation. The metric then agrees with the Taylor equation of the Schwarzschild metric.
\end{example}

\begin{remark}
  You can argue in intuitive terms that the factor of two in $-(1 + 2 \Phi)dt^2$ has to be there. These arguments do not tell you about the factor in $d\vb{x}^2$; they follow from Einstein's equations.
  In the Newtonian limit, you get a lightbending answer that is off by a factor of two from the results of GR.
  The reason that this factor of two is the additional factor $(1 - 2\Phi)$ in front of $d\vb{x}^2$.

  This fact is also how Einstein found the factor of two before the full Schwarzschild metic was found; you don't need the full Schwarzschild metric to describe lightbending.
\end{remark}

\section{Gravitational Waves}%
\label{sec:gravitational_waves}

You may have heard tangentially: gravitational waves were discovered a couple of years ago. They are promising to revolutionise physics, so it might be a good idea to see how they are derived.

There are wave solutions in GR obeying
\begin{equation}
  \Box \overline{h} _{\mu\nu} = 0.
\end{equation}

The solution is $\overline{h}_{\mu\nu} = \Re(H_{\mu\nu} e_{i k_{\rho} x^{\rho}})$, where $H_{\mu\nu}$ is a complex, symmetric matrix that tells us about the polarisation of the gravitational waves.
This solves the wave equation providing $k_{\mu} k^{\mu} = 0$. In other words, the wavevector of these waves is null, which means that gravitational waves travel at the speed of light.

To derive this equation, we had to be in \texttt{dD} gauge. We have to make sure that this solution agrees with that gauge choice.
\begin{claim}
  This satisfies \texttt{dD} gauge $\partial_{\mu} \overline{h}_{\mu\nu} = 0$, provided that $k^{\mu} H_{\mu\nu} = 0$.
\end{claim}

\begin{remark}
  Again, analogously to electromagnetism, this means that the polarisation is transverse to the direction of propagation.
\end{remark}

We can make further gauge transformations that leave us in \texttt{dD} gauge
\begin{align}
  \eta_{\mu\nu} &\to \eta_{\mu\nu} + \partial_{\mu} \xi_{\nu} + \partial_{\nu} \xi_{\mu} \\
  \implies \overline{h}_{\mu\nu} &\to \overline{h}_{\mu\nu} + \partial_{\mu} \xi_{\nu} +  \partial_{\nu} \xi_{\mu} - \partial^{\rho} \xi_{\rho} \eta_{\mu\nu}.
\end{align}

This leaves us in $dD$ gauge, provided that $\Box \xi_{\mu} = 0$.

\begin{remark}
  This means that not all of these polarisations are physical.
\end{remark}

We can take, say, $\xi_{\mu} = \lambda_{\mu} e^{i k_{\rho} x^{\rho}}$. This shifts the polarisation vector
\begin{equation}
  H_{\mu\nu} \to H_{\mu\nu} + i (k_{\mu} \lambda_{\nu} + k_{\nu} \lambda_{\mu} - k^{\rho} \lambda_{\rho} \eta_{\mu\nu}).
\end{equation}

\begin{claim}
  We can choose $\lambda_{\mu}$ such that $H_{0\mu} = 0$ and also $H\indices{^{\mu}_{\mu}} = 0$.
  This is the \emph{transverse-traceless (\texttt{TT}) gauge}.
\end{claim}
\begin{proof}
  Problem sheet.
\end{proof}
