% lecture notes by Umut Özer
% course: gr
\lhead{Lecture 7: October 28}
\chapter{Introducing Riemannian Geometry}%
\label{cha:introducing_riemannian_geometry}

Riemannian geometry is differential geometry with a metric. You can tell that the metric is important, because it literally changes the name of the whole subject. As we will see, we should really be differentiating between Riemannian and Lorentzian manifolds, but one should be aware that people often just refer to a manifold with metric as a Riemannian manifold.

\section{The Metric}%
\label{sec:the_metric}

The metric defines an inner product on the space of tangent vectors.
It has a few properties, which guarantee that the metric is a good inner product.
\begin{definition}[metric]
  A \emph{metric} $g$ is a $(0,2)$-tensor that is
  \begin{itemize}
    \item symmetric: $g(X, Y) = g(Y, X)$
    \item non-degenerate: if $g(X, Y)_p = 0$ for all $Y_p \in T_p(\mathcal{M})$, then $X_p = 0$
  \end{itemize}
\end{definition}
\begin{leftbar}
  \begin{remark}
    The non-degeneracy ensures that the metric is invertible.
  \end{remark}
\end{leftbar}
In a particular coordinate basis, the metric gets two indices $g = g \indices{_\mu_\nu} dx^\mu \otimes dx^\nu$, with $g_{\mu \nu} = g \left( \pdv{}{x^\mu}, \pdv{}{x^\nu} \right)$.
\begin{notation}[]
  We often write this as a \emph{line element} $ds^2 = g_{\mu\nu} dx^\mu dx^\nu$.
\end{notation}
\begin{definition}[signature]
  If we diagonalise $g_{\mu\nu}$, it has positive and negative elements (none are zero, since it is non-degenerate). The number of negative elements is called the \emph{signature} of the metric.
\end{definition}

\begin{leftbar}
  \begin{remark}
    There is a theorem (Sylvester's law of inertia) in matrix algebra that says that the signature is unchanged under a change of basis.
  \end{remark}
\end{leftbar}

There are two metrics that are going to be of interest to us.

\subsection*{Riemannian Metrics}%

\begin{definition}
  A \emph{Riemannian manifold} is a manifold with metric with signature $(+ + \dots +)$.
\end{definition}
\begin{definition}[]
  \emph{Euclidean space} $\mathbb{E}^n$ is $\mathbb{R}^n$ with metric components $g_{ij} = \delta_{ij}$ 
  \begin{equation}
    g = dx^1 \otimes dx^1 + \dots + dx^n \otimes dx^n.
  \end{equation}
\end{definition}

A metric gives us a way to measure 
\begin{itemize}
  \item the length of a vector
  \begin{equation}
    \abs{X} = \sqrt{g(X, X)},
  \end{equation}
  \item the angle between vectors
  \begin{equation}
    g(X, Y) = \abs{X} \abs{Y} \cos\theta,
  \end{equation}
\item the distance between two points $p$ and $q$ along a curve $\sigma\colon [a, b] \to \mathcal{M}$, with end points $\sigma(a) = p$ and $\sigma(b) = q$
  \begin{equation}
    \text{distance} = \int_{a}^{b} \dd[]{t} \sqrt{g(X, X) \rvert_{\sigma(t)}},
  \end{equation}
  where $X$ is tangent to the curve.
  If the curve has coordinates $x^\mu(t)$, this is
  \begin{equation}
    \text{distance} = \int_{a}^{b} \dd[]{t} \sqrt{g_{\mu\nu}(t) \dv{x^\mu}{t} \dv{x^\nu}{t}}
  \end{equation}
\end{itemize}

\subsection*{Lorentzian Metrics}%

\begin{definition}
  A \emph{Lorentzian manifold} is a manifold equipped with a metric of signature $(- + + \dots +)$.
\end{definition}
\begin{definition}
  \emph{Minkowski space} $\mathbb{M}^n$ is $\mathbb{R}^n$ with metric components $\eta\indices{_\mu_\nu} = \text{diag}(-1, +1, \dots, +1)$
  \begin{equation}
    \eta = -dx⁰ \otimes dx^0 + dx^1 \otimes dx^1 + \dots + dx^{n-1} \otimes dx^{n-1}
  \end{equation}
\end{definition}
Because of this minus sign in the metric, `lengths' and `distances'---in the sense of inner products of vectors---can be negative.
We classify vectors $X \in T_p(\mathcal{M})$ as
\begin{equation}
  g(X, X) 
  \begin{cases}
    < 0 & \text{timelike} \\
    = 0 & \text{null} \\
    > 0 & \text{spacelike}
  \end{cases}.
\end{equation}
At each point $p \in \mathcal{M}$, we can draw null tangent vectors called \emph{lightcones}.
%F1
\begin{figure}[tbhp]
  \centering
  \def\svgwidth{0.2\columnwidth}
  \input{lectures/l7f1.pdf_tex}
  \caption{The causal structure around a point $p$ in a Lorentzian manifold.}
  \label{fig:causal-structure-M4}
\end{figure}
\begin{definition}[]
  A curve is called \emph{timelike} if its tangent vector is everywhere timelike.
\end{definition}

\begin{definition}[proper time]
  For timelike curves, we can measure the distance between two points
  \begin{equation}
    \tau = \int_{a}^{b} \dd[]{t} \sqrt{-g_{\mu\nu} \dv{x^\mu}{t} \dv{x^\nu}{t}}.
  \end{equation}
  This is the \emph{proper time} between $a$ and $b$.
\end{definition}
\begin{leftbar}
  \begin{remark}
    The parametrisation $t$ is arbitrary.
  \end{remark}
\end{leftbar}

\subsection{The Joys of a Metric}%
\label{sub:the_joys_of_a_metric}

The metric gives a natural (basis independent) isomorphism at every point
\begin{equation}
  \begin{split}
    g \colon T_p(\mathcal{M}) &\to T^*_{p}(\mathcal{M}) \\
    X &\mapsto g(X, \bullet).
  \end{split}
\end{equation}
Given a vector field $X \in \mathfrak{X}(\mathcal{M})$, we can construct $g(X, \bullet) \in \Lambda^1(\mathcal{M})$.
If $X = X^\mu \partial_\mu$, then the corresponding 1-form is
\begin{equation}
  g \indices{_\mu_\nu} X^\mu dx^\nu \coloneqq X_\nu dx^\nu.
\end{equation}
Because $g$ is non-degenerate, there is an inverse. In components, $g^{\mu\nu} g \indices{_\nu_\rho} = \delta^\mu_\rho$.
This defines a rank $(2, 0)$ tensor $\hat g = g \indices{^\mu^\nu} \partial_\mu \otimes \partial_\nu$. We use this to raise indices
\begin{equation}
  X^\mu \coloneqq g \indices{^\mu^\nu} X_\nu.
\end{equation}
\begin{leftbar}
  \begin{remark}
    As physicists, we often say that we can use the metric to \emph{raise} and \emph{lower} indices.
    This is what we really mean with that; the covariant and contravariant vectors really live in different mathematical spaces, and the \emph{lowering} of an index is the statement that the metric provides a natural isomorphism.
    This isomorphism allowed us to jump between vectors and 1-forms without even worrying about the fact that they are different objects. In spaces other than Euclidean space, their difference becomes important.
  \end{remark}
\end{leftbar}

The metric also gives a natural volume form. 
\begin{itemize}
  \item On a Riemannian manifold, 
  \begin{equation}
    v = \sqrt{\det g_{\mu\nu}} dx^1 \wedge \dots \wedge dx^n.
  \end{equation}
  We usually write $g = \det g_{\mu\nu}$.
  \item On a Lorentzian manifold, 
    \begin{equation}
      v = \sqrt{-g} dx^0 \wedge dx^1 \wedge \dots \wedge dx^{n-1}.
    \end{equation}
\end{itemize}

\begin{claim}
  This is independent of coordinates.
\end{claim}
\begin{proof}
  In new coordinates, $dx^\mu = A \indices{^\mu_\nu} d \tilde x^\nu$ with $A \indices{^\mu_\nu} = \pdv{x^\mu}{\tilde x^\nu}$, 
  \begin{align}
    dx^1 \wedge \dots \wedge dx^n &= A \indices{^1_{\mu_1}} \dots A \indices{^1_{\mu_n}} \underbrace{d\tilde x^{\mu_1} \wedge \dots \wedge d \tilde x^{\mu_n}}_{\text{rearrange to } d\tilde x^1 \wedge \dots \wedge d\tilde x^n} \\
				  &= \sum_{\text{perm }\pi} \text{sign}(\pi) A \indices{^1_{\pi(1)}} \dots A \indices{^n_{\pi(n)}} d \tilde x^1 \wedge \dots \wedge d \tilde x^n \\
				  &= \text{det}(A) d\tilde x^1 \wedge \dots \wedge d\tilde x^n.
  \end{align}
  The determinant $\det A > 0$ if the coordinate change preserves orientation.
  Meanwhile, 
  \begin{align}
    g_{\mu\nu} &= \pdv{\tilde x^\rho}{x^\mu} \pdv{\tilde x^\sigma}{x^\nu} \tilde g_{\rho\sigma} \\
	       &= (A^{-1}) \indices{^\rho_\mu} (A^{-1}) \indices{^\sigma_\nu} \tilde g_{\rho\sigma} \\
	       &\implies \det g_{\mu\nu} = (\det A^{-1})^2 \det \tilde g_{\rho\sigma} = \frac{\det \tilde g_{\rho\sigma}}{(\det A)^2} \\
	       &\implies v = \sqrt{\abs{\tilde g}} d\tilde x^1 \wedge \dots \wedge d\tilde x^n.
  \end{align}
\end{proof}
In components, $v = \frac{1}{n!} v_{\mu_1 \dots \mu_n} d x^{\mu_1} \wedge \dots \wedge d x^{\mu_n}$, with
\begin{equation}
  v_{\mu_1 \dots \mu_n} = \sqrt{ \abs{g}} \varepsilon_{\mu_1 \dots \mu_n},
\end{equation}
where $\varepsilon$ is the totally anti-symmetric tensor.
We can then integrate any function as
\begin{equation}
  \int_{\mathcal{M}} f v = \int_{\mathcal{M}} \dd[n]{x} \sqrt{\abs{g}} f(x).
\end{equation}
So the metric provides the measure in the form of $\sqrt{\abs{g}}$.
