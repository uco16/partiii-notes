% lecture notes by Umut Özer
% course: gr
\lhead{Lecture 22: December 04}

\subsection{Making Waves}%
\label{sub:making_waves}

The discussion of the production of gravitational waves will very much parallel the production of electromagnetic waves.
You generate electromagnetic waves by shaking charges; you generate gravitational waves by shaking mass.
In fact, the punchline of the previous subsection is that this is actually inaccurate in almost all circumstances. However, for now this is a good approximation.

We want to solve
\begin{equation}
  \label{eq:22-1}
  \Box \overline{h}_{\mu\nu} = -16\pi G T_{\mu\nu}.
\end{equation}

Everything is done in linearised analysis, meaning that the right hand side  must be small in an appropriate sense.
As we go on we will half-see what this means. In some sense we will deal with slowly moving things with low energy.

\begin{figure}[tbhp]
  \centering
  \def\svgwidth{0.4\columnwidth}
  \input{lectures/l22f1.pdf_tex}
  \caption{}
  \label{fig:l22f1}
\end{figure}

Consider some interesting region $\Sigma$ of space with an extend $d$, illustrated in Figure \ref{fig:l22f1} in which  $T_{\mu\nu} \neq 0$ . This could be two neutron stars orbiting each other.
Then there is a whole bunch of empty space $r \gg d$  between us and $\Sigma$.
\begin{leftbar}
  \begin{note}
    This is the same setup as when we try to understand radiation in electromagnetism (far-field approximation).
  \end{note}
\end{leftbar}

We will solve this using Green's functions.
Similar to the Green's function solution of electromagnetism, the solution is
\begin{equation}
  \overline{h}_{\mu\nu}(\vb{x}, t) = 4 G \int_\Sigma \dd[3]{x'} \frac{T_{\mu\nu}(\vb{x}', t_{\text{ret}})}{\abs{\vb{x} - \vb{x}'}},
\end{equation} 
with $t_{\text{ret}} = t - \abs{\vb{x} - \vb{x}'}$.
\begin{remark}
  This derivative $\partial_0$ is with respect to $t$, not $t_{\text{ret}}$.
\end{remark}

The equation \eqref{eq:22-1} was derived in \texttt{dD} gauge. And this solution indeed obeys the \texttt{dD} gauge condition provided that $\partial_{\mu} T^{\mu\nu} = 0$ .

For $r \gg d$, 
\begin{align}
  \abs{\vb{x} - \vb{x}'} &\approx r - \frac{1}{r} \vb{x} \cdot \vb{x}' + \dots \\
  \text{and} \quad \frac{1}{\abs{\vb{x} - \vb{x}'}} &\approx \frac{1}{r} + \frac{1}{r^3} \vb{x} \cdot \vb{x}' + \dots \\
  \text{and} \quad T_{\mu\nu} ( \vb{x}', t_{\text{ret}}) &= T_{\mu\nu} (\vb{x}', t - r) + \dot{T}_{\mu\nu} ( \vb{x}' , t - r) \frac{\vb{x} \cdot \vb{x}'}{r} + \dots,
\end{align} 
where we need $\dot{T}_{\mu\nu}$  to be suitably small.
At leading order, 
\begin{equation}
  \overline{h}_{\mu\nu} \approx \frac{4 G}{r} \int_{\Sigma} \dd[3]{x'} T_{\mu\nu}(\vb{x}', t - r)
\end{equation}
We get that
\begin{equation}
  \Rightarrow \overline{h}_{00} \approx \frac{4GE}{r} \qquad \text{with } E = \int_{\Sigma} \dd[3]{x'} T_{00} (\vb{x}', t - r).
\end{equation}
This is just recapitulating the Newtonian limit.
Similarly, we have
\begin{equation}
  \overline{h}_{0i} \approx - \frac{4GP_i}{r} \qquad \text{with } P_i = -\int_{\Sigma} \dd[3]{x'} T_{0i} (\vb{x}', t - r).
\end{equation}
This is just a long-distance fall off that you would see for stationary matter, or for matter moving along way away. 
The interesting physics comes from the following component
\begin{equation}
  \overline{h}_{ij} \approx \frac{4G}{r} \int_{\Sigma} \dd[3]{x'} T_{ij}(\vb{x}', t - r).
\end{equation}
Unlike the other two, $T_{ij}$  is not a conserved object; it is the current part of the stress-tensor rather than the charge part.

To solve this integral, we perform the same manipulations as in electromagnetism. In particular, we write
\begin{align}
  T^{ij} &= \partial_{k} (T^{ik} x^{j}) - (\partial_{k} T^{ik})x_{j} \\
	 &= \partial_{k} (T^{ik} x^{j}) + \partial_0 T^{0i} x^{j},
\end{align}
where we rewrote the second term by using the fact that the energy-momentum tensor is conserved $\partial_{\mu} T^{\mu\nu} = 0$ .
Symmetrising over $i$  and $j$ , 
\begin{align}
  T^{0(i} x^{j)} &= \frac{1}{2}\partial_{k} ( T^{0k} x^{i} x^{j}) - \frac{1}{2} (\partial_{k} T^{0k}) x^{i} x^{j} \\
  &= \frac{1}{2}\partial_{k} (T^{0k} x^{i} x^{j}) + \frac{1}{2} \partial_0 T^{00} x^{i} x^{j},
\end{align}
again using conservation of the energy-momentum tensor.

Put this into the integral $\int_{\Sigma} \dd[]{x'}$  and drop any total spatial derivative.
All that remains are the time derivatives $\partial_0$. What we get is that the metric where we are is given by
\begin{equation}
  \overline{h}_{ij}(\vb{x}, t) \approx \frac{2G}{r} \ddot{I}_{ij}(t - r), \qquad \text{where } I_{ij} = \int \dd[3]{x'} T^{00} (\vb{x}', t) x'_{i} x'_{j}.
\end{equation}
$I_{ij}$ is the \emph{quadrupole} of the energy distribution in the region $\Sigma$.
\begin{leftbar}
  \begin{note}
    The dependence on $\vb{x}$ of $\overline{h}$ comes from the $r$.
  \end{note}
\end{leftbar}

\begin{remark}
  We could now use the \texttt{dD} gauge condition $\partial_{\mu} \overline{h}{}^{\mu\nu} = 0$ to find the corrections to $\overline{h}_{00}$ and $\overline{h}_{0i}$ using $\overline{h}_{ij}$.
\end{remark}

\begin{remark}
  Shaking the matter quadrupole at some characteristic frequency $\omega$, creates waves of roughly frequency $\omega$ (possibly with some factors of two).
  We will see this in more detail in the next section.
\end{remark}

\begin{leftbar}
  \begin{note}
    In electromagnetism there is an analogous formula: "gauge field far away = first time derivative of the dipole of the charge density".
    We get the quadrupole rather than the dipole, since the dipole of a mass density is related to the momentum, which is conserved and cannot shake backwards and forwards.
    You have to go to the quadrupole to find the gravitational waves, which is why they are weak compared to electromagnetic waves.
  \end{note}
\end{leftbar}

\begin{example}[Binary System]
  Consider two objects, each with mass $M$ , in a circular orbit in the $(x-y)$ -plane at distance $R$. 
  This is illustrated in \ref{fig:l22f2}.
  \begin{figure}[tbhp]
    \centering
    \def\svgwidth{0.4\columnwidth}
    \input{lectures/l22f2.pdf_tex}
    \caption{}
    \label{fig:l22f2}
  \end{figure}

  Newtonian gravity gives $\omega^2 = \frac{2 G M }{R^3}$ . Viewed as point particles, 
  \begin{equation}
    T^{00}(\vb{x}, t) = M \delta(z) \left[ \delta(x - \frac{1}{2} R \cos(\omega t)) \delta(y - \frac{1}{2} R \sin(\omega t)) + \delta(x + \frac{1}{2} R \cos(\omega t)) \delta(y + \frac{1}{r} R \sin(\omega t)) \right]
  \end{equation}

  Compute the quadrupole $I_{ij}(t)$  to find 
  \begin{equation}
    I_{ij} = \frac{MR^2}{4}
    \begin{pmatrix}
     1 + \cos(2 \omega t) & \sin(2 \omega t) & 0 \\
     \sin(2 \omega t) & 1 - \cos(2\omega t) & 0 \\
     0 & 0 & 0 \\
    \end{pmatrix},
  \end{equation}
  where we have used the double-angle formula along the way.

  This tells us that 
  \begin{equation}
    \overline{h}_{ij} = - \frac{2 G M R^2 \omega^2}{r} 
    \begin{pmatrix}
     \cos(2 \omega t_{\text{ret}}) & \sin(2 \omega t_{\text{ret}}) & 0 \\
     \sin(2 \omega t_{\text{ret}}) & - \cos(2 \omega t_{\text{ret}}) & 0 \\
     0 & 0 & 0 \\
    \end{pmatrix}.
  \end{equation}
  This combination of polarisation is called \emph{circular polarisation} as in the case of light.
  We get circularly polarised gravitational waves travelling in the $z$-direction.
  Use $\omega^2 =\frac{2 GM}{R^3}$ to find the magnitude $\abs{h_{ij}} \sim \frac{G^2M^2}{R r}$.

  This is what you measure in LIGO.
  For light, what you measure is the square of the amplitude, meaning that you can only see as $~\frac{1}{r^2}$ far. As you start to improve LIGO, you start to see much further out than with optical telescopes!

  To get a `large' magnitude $h_{ij}$, we need compact objects nearby.
  The most compact objects are black holes. The closest they can come is the \emph{Schwarzschild radius} $R_S = 2 G M$. Neutron stars are not wildly different from this.

  As these objects approach, the size of the gravitational wave that we will see is given by $\abs{h_{ij}} \sim \frac{GM}{r}$.
  A black hole of a few solar masses ($R_S \sim 10$km) in our neighbouring Andromeda galaxy ($r \sim 10^{18}$km) gives a gravitational wave strength of $\abs{h_{ij}} \sim 10^{-17}$.
\end{example}
