% lecture notes by Umut Özer
% course: gr
\lhead{Lecture 11: November 06}
\section{Path Dependence and Curvature}%
\label{sec:path_dependence_and_curvature}

How does parallel transport depend on the path taken?
Let $X, Y \in \mathfrak{X}(\mathcal{M})$ be vector fields such that $[X, Y]$. Consider two curves generated by $X$ and $Y$.
\begin{figure}[tbph]
  \centering
  \def\svgwidth{0.5\columnwidth}
  \input{lectures/l11f1.pdf_tex}
  \caption{Path dependence of parallel transport}
  \label{fig:l11f1}
\end{figure}
Now consider a vector $Z_p$ living in the tangent space $T_p(\mathcal{M})$ at point $p$.
As illustrated in Figure~\ref{fig:l11f1}, the result of parallel transporting this vector from $p$ to another point $r$ on the manifold will depend on the path taken.
The expression for their difference is most clearly expressed in the case of infinitesimal translations.
Moreover, life will be much easier if we pick normal coordinates at $p$ with $x^{\mu} = (\tau, \lambda, 0, 0, \dots)$.
Along $p \to q$, the parallel transport equation is a first order differential equation:
\begin{equation}
  \dv[]{Z^{\mu}}{\tau} + X^{\nu} \Gamma^{\mu}_{\rho\nu} Z^{\rho} = 0.
\end{equation}
Let us first see which vector $Z_r$ we end up with when taking the path
\begin{equation}
  p \xrightarrow{\delta \tau} q \xrightarrow{\delta\lambda} r.
\end{equation}
The first step is to Taylor expand $Z_q$ about the point $p: (\tau = 0, \lambda = 0, \dots)$. Let
\begin{align}
  Z^{\mu}_q &= Z^{\mu}_p + \left.\dv[]{Z^{\mu}}{\tau}\right\rvert_{\tau = 0} d\tau 
  + \frac{1}{2} \left.\dv[2]{Z^{\mu}}{\tau}\right\rvert_{\tau = 0} d\tau^2 + \dots \\
  \left.\dv[]{Z^{\mu}}{\tau}\right\rvert_p &= - (X^{\nu} Z^{\rho} \Gamma^{\mu}_{\rho\nu})_p = 0.
\end{align}
The last expression vanishes since $\Gamma^{\mu}_{\rho\nu}(p) = 0$ in normal coordinates.
Moreover, we have
\begin{equation}
  \left.\dv[2]{Z^{\mu}}{\tau}\right\rvert_{\tau = 0} = - \left( X^{\nu}Z^{\rho} \dv[]{\Gamma^{\mu}_{\rho\nu}}{\tau} + \dv[]{X^{\nu}}{\tau} Z^{\rho} \Gamma^{\mu}_{\rho\nu} + X^{\nu} \dv[]{Z^{\rho}}{\tau} \Gamma^{\mu}_{\rho\nu} \right)_p.
\end{equation}
By the previous reasoning, the latter two summands vanish.
Using $\dv[]{}{\tau} = X^{\sigma} \partial_{\sigma}$, we obtain
\begin{equation}
  \left.\dv[2]{Z^{\mu}}{\tau}\right\rvert_{\tau =0} = - \left( X^{\nu}X^{\sigma} Z^{\rho} \Gamma^{\mu}_{\rho\nu, \sigma} \right)_p.
\end{equation}
Hence, we have
\begin{equation}
  Z^{\mu}_q = Z^{\mu}_p - \frac{1}{2} (X^{\nu}X^{\sigma} Z^{\rho} \Gamma^{\mu}_{\rho\nu, \sigma})_p d\tau^2 + \dots
\end{equation}
Next, we will go from $q \to r$:
\begin{equation}
  Z^{\mu}_r = Z^{\mu}_q + \left.\dv[]{Z^{\mu}}{\lambda}\right\rvert_{q} d\lambda + \frac{1}{2} \left.\dv[2]{Z^{\mu}}{\lambda}\right\rvert_{q} d\lambda^2 + \dots
\end{equation}
For the first summand, we use the earlier result, while for the second term we use parallel transport
\begin{align}
  \left.\dv[]{Z^{\mu}}{\lambda}\right\rvert_{q} &= -\left( Y^{\nu} Z^{\rho} \Gamma^{\mu}_{\rho\nu} \right)_q \\
						&= - \left( Y^{\nu} Z^{\rho} \dv{\Gamma^{\mu}_{\rho\nu}}{\tau} \right)_p d \tau + \dots \\
						&= - \left( Y^{\nu} Z^{\rho} X^{\sigma} \Gamma^{\mu}_{\rho\nu, \sigma} \right) d\tau + \dots
\end{align}
Similarly, the second derivative is
\begin{equation}
  \left.\dv[2]{Z^{\mu}}{\lambda}\right\rvert_{q} = - \left( Y^{\nu}Y^{\sigma} Z^{\rho} \Gamma^{\mu}_{\rho\nu, \sigma} \right)_q + \dots.
\end{equation}
The higher order terms $(\dots)$ include $\Gamma^{\mu}_{\rho\nu}(q) \sim \dv[]{}{t}\Gamma^{\mu}_{\rho\nu} d\tau$.
Putting all of this together, the components of the vector $Z$ at $r$ is
\begin{equation}
  Z^{\mu}_r = Z^{\mu}_p - \frac{1}{2} (\Gamma^{\mu}_{\rho\nu, \sigma})_p \left[ X^{\nu}X^{\sigma} Z^{\rho} d\tau^2 + 2 Y^{\nu}Z^{\rho}X^{\sigma} d\tau d\lambda + Y^{\nu}Y^{\sigma}Z^{\rho} d\lambda^2 \right]_p + \dots.
\end{equation}
If we go the other way, we have
\begin{equation}
  (Z')^{\mu}_r = Z^{\mu}_p - \frac{1}{2} (\Gamma^{\mu}_{\rho\nu, \sigma})_p \left[ X^{\nu}X^{\sigma} Z^{\rho} d\tau^2 + 2 X^{\nu} Z^{\rho} Y^{\sigma} d\lambda d\tau + Y^{\nu} Y^{\sigma} Z^{\rho} d\lambda^2 \right] + \dots
\end{equation}
The difference between the two is
\begin{align}
  \Delta Z^{\mu}_r &= Z^{\mu}_r - (Z')^{\mu}_r \\
		   &= \left( \Gamma^{\mu}_{\rho\nu, \sigma} - \Gamma^{\mu}_{\rho\sigma, \nu} \right)_p \left( Y^{\nu} Z^{\rho} X^{\sigma} \right)_p d\lambda d\tau + \dots \\
		   &= (R \indices{^{\mu}_{\rho\nu\sigma}} Y^{\nu}X^{\rho}X^{\sigma})_p d\lambda d\tau + \dots
\end{align}
We picked very special coordinates. However, the final expression is a tensor equation, which has to hold in all coordinate systems, even though we used special coordinates to derive it.
\begin{leftbar}
  \begin{remark}
    If a tensor evaluates to zero in one coordinate system, it does so in all coordinate systems. 
    This is not true of the Christoffel symbols.
  \end{remark}
\end{leftbar}
\begin{leftbar}
  \begin{remark}
    There is a whole area of geometry, called \emph{holonomy}, which deals with the possible rotations of the vector that can be obtained when going along certain paths along a manifold.
    Calabi-Yau manifolds are related to \emph{special holonmies}.
  \end{remark}
\end{leftbar}

\subsection{Geodesic Deviation}%
\label{sub:geodesic_deviation}

Consider the one-parameter family of geodesics $x^{\mu}(\tau; s)$; for a fixed $s$, $x^{\mu}(\tau;s)$ is a geodesic with affine parameter $\tau$.
%F2
\begin{figure}[tbhp]
  \centering
  \def\svgwidth{0.5\columnwidth}
  \input{lectures/l11f2.pdf_tex}
  \caption{}
  \label{fig:l11f2}
\end{figure}
Figure \ref{fig:l11f2} depicts these geodesics, which we can think of being integral curves generated by a vector field $X^{\mu} = \flatfrac{\partial^{} x^{\mu}}{\partial \tau} \rvert_s$. Moreover, the curves of fixed $\tau$ can be thought of as being generated by a vector fields $S^{\mu} = \left.\pdv*{x^{\mu}}{s}\right\rvert_{\tau}$
We will not prove it, but we can always pick $\tau, s$ such that $[S, X] =0$.

\begin{claim}
  \begin{align}
    \nabla_{X}\nabla_{X} S &= R(X, S) X \\
    \label{eq:11-1}
    \text{or} X^{\nu} \nabla_{\nu} (X^{\rho} \nabla_{\rho} S^{\mu}) &= R \indices{^{\mu}_{\nu\rho\sigma}} X^{\nu}X^{\rho} S^{\sigma} \\
    \text{or } \dv[2]{S^{\mu}}{\tau} &= R \indices{^{\mu}_{\nu\rho\sigma}} X^{\nu}X^{\rho} S^{\sigma}
  \end{align}
\end{claim}
\begin{proof}
  We are dealing with the Levi-Civita connection, which is torsion free. For a torsion-free connection, we have
  \begin{equation}
    [X, S] = 0 \implies \nabla_{X}S = \nabla_{S}X.
  \end{equation}
  We use this in the left hand side of \eqref{eq:11-1} to find
  \begin{equation}
    \nabla_{X} \nabla_{X} S = \nabla_{X} \nabla_{S} X = \nabla_{S} \nabla_{X} X + R(X, S) X.
  \end{equation}
  Now $\nabla_{S} \nabla_{X} X = 0$ as $X$ is a geodesic.
\end{proof}
\begin{leftbar}
  \begin{remark}
    In the physics-part of this lecture course, we will see this again when we talk about gravitational waves.
  \end{remark}
\end{leftbar}

\section{More on the Riemann Tensor}%
\label{sec:more_on_the_riemann_tensor}

Recall that the definition of the Riemann tensor in terms of the Christoffel symbols is
\begin{equation}
  R \indices{^{\sigma}_{\rho\mu\nu}} = \partial_{\mu} \Gamma^{\sigma}_{\rho\nu} - \partial_{\nu} \Gamma^{\sigma}_{\rho\mu} + \Gamma^{\lambda}_{\nu\rho} \Gamma^{\sigma}_{\mu\lambda} - \Gamma^{\lambda}_{\mu\rho} - \Gamma^{\lambda}_{\mu\rho} \Gamma^{\sigma}_{\nu\lambda}.
\end{equation}
This satisfies certain symmetry properties which are more evident once we lower the first index.
\begin{claim}
  $R_{\sigma\rho\mu\nu} = g_{\sigma\lambda} R \indices{^{\lambda}_{\rho\mu\nu}}$ obeys
  \begin{itemize}
    \item $R_{\sigma\rho\mu\nu} = -R_{\sigma\rho\nu\mu}$
    \item $R_{\sigma\rho\mu\nu} = - R_{\rho\sigma\mu\nu}$
    \item $R_{\sigma\rho\mu\nu} = R_{\mu\nu\sigma\rho}$
    \item $R_{\sigma[\rho\mu\nu]]} = 0$, the \emph{first Bianchi identity}
  \end{itemize}
\end{claim}
\begin{proof}
  Use normal coordinates
  \begin{align}
    R_{\sigma\rho\mu\nu} &= g_{\sigma\lambda} (\partial_{\mu} \Gamma^{\lambda}_{\nu\rho} -\partial_{\nu} \Gamma^{\lambda}_{\mu\rho}) \\
			 &= \frac{1}{2} (\partial_{\mu} \partial_{\rho} g_{\nu\sigma} - \partial_{\mu} \partial_{\sigma} g_{\nu\rho} - \partial_{\nu} \partial_{\rho} g_{\mu\sigma} + \partial_{\nu} \partial_{\sigma} g_{\mu\rho})
  \end{align}
  Now stare at this.
\end{proof}
\begin{claim}[The Second Bianchi Identity]
\begin{equation}
  \nabla_{[\lambda} R_{\sigma\rho] \mu\nu} = 0 \quad \text{or equivalently} \quad R_{\sigma\rho[\mu\nu;\lambda]} = 0
\end{equation}
\end{claim}
\begin{proof}
  The Riemann tensor is, schematically, $R \sim \partial \Gamma + \Gamma\Gamma$ and so its derivative is $\partial R \sim \partial^2 \Gamma + 2 \Gamma \partial\Gamma$. In normal coordinates at a point, we have $\Gamma = 0$ (but $\partial\Gamma \neq 0$ in general), so the last summand vanishes and the partial derivative of the Riemann tensor is
  \begin{equation}
    R_{\sigma\rho\mu\nu,\lambda} = \frac{1}{2} ( g_{\nu\sigma,\rho\mu\lambda} - g_{\nu\rho,\sigma\mu\lambda} - g_{\mu\sigma,\rho\nu\lambda} + g_{\mu\rho,\sigma\nu\lambda})
  \end{equation}
  The partial derivatives $\partial_{\mu} \partial_{\lambda}$ in the first term commute: $g_{\nu\sigma,\rho\mu\lambda} = g_{\nu\sigma,\rho\mu\lambda}$. Therefore, anti-symmetrising over $\mu$ and $\lambda$ makes this term vanish. In the same way, the other three terms vanish when we anti-symmetriese over $\mu, \nu$, and $\lambda$. Thus $R_{\sigma\rho[\mu\nu,\lambda]} = 0$. In normal coordinates, this is the same as $R_{\sigma\rho[\mu\nu;\lambda]} = 0$, which is a tensor equation and therefore holds in all coordinate systems.
\end{proof}
\begin{leftbar}
  \begin{remark}
    The Bianchi identity here implies a similar identity on the Ricci tensor, which we will see soon. There is a very elegant proof, which we will discuss, of that identity from the action principle of GR.
  \end{remark}
\end{leftbar}
