% lecture notes by Umut Özer
% course: gr
\lhead{Lecture 10: November 04}

\begin{proof}[Proof of Theorem \ref{thm:divergence}]
  Using Lemma \ref{lem:chri}, we have that 
  \begin{align}
    \sqrt{g} \nabla_{\mu} X^\mu &= \sqrt{g} (\partial_\mu X^\mu + \Gamma \indices{^\mu_\mu_\nu} X^\nu) \\
				&= \sqrt{g} (\partial_\mu X^\mu + X^\nu \frac{1}{\sqrt{g}} \partial_\nu \sqrt{g}) \\
				&= \partial_\mu (\sqrt{g} X^\mu).
  \end{align}
  Let us now restrict our discussion to a particular metric
  \begin{equation}
    g_{\mu\nu} = 
    \begin{pmatrix}
     \gamma_{ij} & 0 \\
     0 & N^2 \\
    \end{pmatrix}
  \end{equation}
  where the boundary $\partial \mathcal{M}$ is a surface with $x^n = \text{const.}$.
  We then use standard integration by parts over this boundary to give
  \begin{align}
    \int_{\mathcal{M}} \dd[n]{x} \sqrt{g} \nabla_{\mu} X^\mu &= \int_{\mathcal{M}} \dd[n]{x} \partial_\mu (\sqrt{g} X^\mu) \\
							     &= \int_{\partial \mathcal{M}} \dd[n-1]{x} \sqrt{\gamma N^2} X^n.
  \end{align}
  The unit normal vector is chosen $n^\mu = (0, 0, \dots, 0, 1/N)$, so that when measured with respect to the metric, we indeed satisfy the definition for a unit vector: $g_{\mu\nu} n^\mu n^\nu = 1$. This means that when we lower with the metric we have $n_\mu = (0, 0, \dots, N)$ and 
  \begin{equation}
    \int_{\mathcal{M}} \dd[n]{x} \partial_\mu (\sqrt{g} X^\mu) = \int_{\partial \mathcal{M}} \dd[n-1]{x} \sqrt{\gamma} n_\mu X^\mu.
  \end{equation}
  The final expression is covariant, and so holds in all coordinate systems.
\end{proof}

\subsection{Parallel Transport}%
\label{sub:parallel_transport}

The general story is that differentiation requires us to compare the vectors at two `neighbouring' points. But these points have different tangent spaces, and we cannot simply add vectors in these different tangent spaces.
The connection allows us to define a map between vector spaces that live at different points. This map is called \emph{parallel transport}.
\begin{definition}[parallel transport]
  Take a vector field $X$ and an integral curve $C$ with the property
  \begin{equation}
    \left.X^\mu\right\rvert_C = \dv{x^\mu(\tau)}{\tau}.
  \end{equation}
  A tensor $T$ is said to be \emph{parallel transported} along C if
  \begin{equation}
    \nabla_{X} T = 0.
  \end{equation}
\end{definition}
\begin{example}[]
  A vector field $Y \in \mathfrak{X}(\mathcal{M})$ is parallel transported along $C$ if it obeys $\nabla_{X} Y = 0$. In components, this is
  \begin{equation}
    \label{eq:10-vf}
    X^\nu(\partial_\nu Y^\mu + \Gamma^\mu_{\nu\rho} Y^\rho) = 0.
  \end{equation}
  In particular, if we restrict to the curve $C$, consider $Y^\mu(x(\tau))$, which, from obeys \eqref{eq:10-vf}
  \begin{equation}
    \label{eq:10-vfcoords}
    \dv[]{Y^\mu}{\tau} + X^\nu \Gamma^\mu_{\nu\rho} Y^\rho = 0.
  \end{equation}
  Given an initial condition $Y(x(\tau=0)) \in T_p(\mathcal{M})$, where $p = x(\tau=0)$, equation \eqref{eq:10-vfcoords} determines a unique vector at each point along $C$. This is simply because it is a first order differential equation.
  %L10F1
  \begin{figure}[tbhp]
    \centering
    \def\svgwidth{0.4\columnwidth}
    \input{lectures/l10f1.pdf_tex}
    \caption{Parallel transport}
    \label{fig:l10f1}
  \end{figure}
\end{example}
\begin{leftbar}
  \begin{remark}
    We can define parallel transport with only one curve. However, we need a vector field to cover the whole manifold.
    The parallel transport property depends on the path along which we parallel transport. Different paths will in general give different results.
  \end{remark}
\end{leftbar}

\subsection{Geodesics}%
\label{sub:geodesics}

\begin{definition}
  A \emph{geodesic} is a curve tangent to $X \in \mathfrak{X}(\mathcal{M})$ that is parallel transported along itself: it obeys $\nabla_{X}X = 0$.
\end{definition}
From \eqref{eq:10-vfcoords}, the coordinates $x^\mu(\tau)$ along the curve obey the (affinely parametrised) geodesic equation:
\begin{equation}
  \label{eq:geodesic}
  \dv[2]{x^\mu}{\tau} + \Gamma^\mu_{\rho\nu} \dv[]{x^\rho}{\tau} \dv[]{x^\nu}{\tau} = 0.
\end{equation}
The geodesic equation also arises as the equation of motion from the action
\begin{equation}
  \label{eq:action}
  S = \int \dd[]{\tau} g_{\mu\nu}(x) \dv[]{x^\mu}{\tau} \dv[]{x^\nu}{\tau}.
\end{equation}
\begin{leftbar}
  \begin{remark}
    If you are ever given a metric and you need to compute the Christoffel symbols, it gets quite difficult quickly. Do not compute the long way round The quick way round is to plug it into the action \eqref{eq:action} and then work out the equations of motion. Comparing this with the geodesic equation \eqref{eq:geodesic} gives the Christoffel symbols $\Gamma^\mu_{\rho\nu}$. There is a small subtlety with symmetrisation.
  \end{remark}
\end{leftbar}

\section{Normal coordinates}%
\label{sec:normal_coordinates}

\begin{claim}[Riemann Normal Coordinates]
  For any point $p \in \mathcal{M}$, we can always find \emph{normal coordinates} in which the metric components are
  \begin{equation}
    g_{\mu\nu}(p) = \delta_{\mu\nu} \quad \text{and} \quad g_{\mu\nu, \rho}(p) = 0.
  \end{equation}
\end{claim}
\begin{proof}[Motivation]
  To see that this is possible, start with coordinates $\widetilde x^\mu$ and change to $x^\mu$, such that $\widetilde x^\mu(p) = x^\mu(p) = 0$. Then
  \begin{equation}
    \widetilde{g}_{\rho\sigma} \pdv{\widetilde{x}^\rho}{x^\mu} \pdv{\widetilde{x}^\sigma}{x^\nu} = g_{\mu\nu}
  \end{equation}
  Taylor expanding we get
  \begin{equation}
    \widetilde{x}^\rho \left.\pdv{\widetilde{x}^\rho}{x^\mu}\right\rvert_{x = 0} x^\mu + \frac{1}{2} \left.\frac{\partial^2 \widetilde x^\rho}{\partial x^\mu \partial x^\nu}\right\rvert_{x = 0} x^\mu x^\nu + \dots
  \end{equation}
  At leading order, we want
  \begin{equation}
    \widetilde{g}_{\rho\sigma} \left.\pdv{\widetilde{x}^\rho}{x^\mu}\right\rvert_{x = 0} \left.\pdv{\widetilde{x}^\sigma}{x^\nu}\right\rvert_{x = 0} = \delta_{\mu\nu}.
  \end{equation}
  We have $\frac{1}{2} n(n+1)$ conditions, but $n^2$ coefficients in $\pdv*{\widetilde{x}^{\rho}}{x^\mu}$. We can do it with $\frac{1}{2} n(n-1) = \dim SO(n)$ left over.
  At second order, we want to set $g_{\mu\nu,\rho} = 0$. This gives $\frac{1}{2} n(n+1) \times n$ conditions. Meanwhile, $\flatfrac{\partial^2\widetilde{x}^\rho}{\partial x^\mu \partial x^\nu}$ has $n \times \frac{1}{2} n(n+1)$ components. In terms of degrees of freedom, this suggests that we can do it. To actually prove it we will have to do some more work. However, the usefulness of this counting comes in when we consider the second derivative of the metric:
  If we try to go further, we have $\frac{1}{2} n (n+1) \times \frac{1}{2} n(n+1)$ components in $g_{\mu\nu, \rho\sigma}$. And the third derivative of the coordinate transformation $\flatfrac{\partial^3 \widetilde x^\rho}{\partial x^\mu \partial x^\nu \partial x^\lambda}$ has $n \times \frac{1}{3!} n (n + 1) (n + 2)$ components.
  We find that we do not have enough degrees of freedom to do this; we are short by $\frac{1}{12}n^2 (n^2 - 1)$.
  This is precisely the number of independent coordinates of the Riemann tensor $R \indices{^\sigma_{\rho\mu\nu}}$.

  Let us now actually give the coordinates that satisfy the claim.
  We can construct normal coordinates using geodesics and the \emph{exponential map} (c.f.~\emph{Symmetries, Fields, and Particles}).
  \begin{equation}
    \text{Exp} \colon T_p(\mathcal{M}) \ \to\  \mathcal{M} \\
  \end{equation}
  We `follow along' the geodesic such that $\flatfrac{d \widetilde x^\mu}{d \tau} \rvert_{\tau = 0} = \widetilde X^\mu$ for distance $\tau = 1$.
  %F2
  \begin{figure}[tbhp]
    \centering
    \def\svgwidth{0.4\columnwidth}
    \input{lectures/l10f2.pdf_tex}
    \caption{}
    \label{fig:l10f2}
  \end{figure}
  Now, pick an orthonormal basis $\left\{ e_\mu \right\}$ of $T_p(\mathcal{M})$ such that $X_p = X^\mu e_\mu$. Normal coordinates are $x^\mu(q) = X^\mu$.
  The completion of the proof is in the printed notes.
\end{proof}
\begin{leftbar}
  \begin{remark}
    This is a mathematical version of the equivalence principle.
  \end{remark}
\end{leftbar}
\begin{corollary}
  In normal coordinates, $\Gamma^\mu_{\nu\rho}(p) =0$.
\end{corollary}
