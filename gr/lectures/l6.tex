% lecture notes by Umut Özer
% course: gr
\lhead{Lecture 6: October 25}
\begin{example}[one-form on $\mathbb{R}^3$]
  Consider the 1-form $\omega = \omega_\mu(x) dx^\mu$. Then
  \begin{equation}
    (d\omega)_{\mu \nu} = \partial_\mu \omega_\nu - \partial_\nu \omega_\mu \quad \text{or} \quad
    d\omega = \frac{1}{2} (\partial_\mu \omega_\nu - \partial_\nu \omega_\mu) dx^\mu \wedge dx^\nu.
  \end{equation}
  On $\mathbb{R}^3$, we have
  \begin{equation}
    d\omega = (\partial_1 \omega_2 - \partial_2 \omega_1) dx^1 \wedge dx^2
    + (\partial_2 \omega_3 - \partial_3 \omega_2) dx^2 \wedge dx^3 +
    (\partial_3 \omega_1 - \partial_1 \omega_3) dx^3 \wedge dx^1.
  \end{equation}
  Here we have used the property of the wedge product that $dx^i \wedge dx^i = 0$.
  These three terms are the components of the curl $\curl \boldsymbol \omega$.
  The correct way to think about the curl is as an exterior derivative as a one-form, which coincidentally has three components as well.
\end{example}

\begin{example}[two form on $\mathbb{R}^3$]
  Consider $B \in \Lambda^2(\mathbb{R}^3)$ with
  \begin{equation}
    B = B_1(x) dx^2 \wedge dx^3 +B_2(x) dx^3 \wedge dx^1 + B_3(x) dx^1 \wedge dx^2.
  \end{equation}
  Taking the exterior derivative of this two-form, we get a three form. However, this is a top-form in $\mathbb{R}^3$, so we only have once component.
  \begin{equation}
    dB = (\partial_1 B_1 + \partial_2 B_2 + \partial_3 B_3) dx^1 \wedge dx^2 \wedge dx^3.
  \end{equation}
  We have seen this before as well. These are the components of the divergence $\div \vb B$.
\end{example}

\begin{example}
  In electromagnetism, the gauge field $A^\mu$ should be thought of as a one-form $A \in \Lambda^1(\mathbb{R}^4)$.
  In components, this is $A = A_\mu dx^\mu$.
  Taking the exterior derivative we get 
  \begin{equation}
    F = dA = F_{\mu\nu} dx^\mu \wedge dx^\nu = \frac{1}{2} (\partial_\mu A_\nu - \partial_\nu A_\mu) dx^\mu \wedge dx^\nu.
  \end{equation}
  Gauge transformations act as $A \to A + d\alpha$, where $\alpha \in \Lambda^0(\mathcal{M}) = C^\infty(\mathcal{M})$.
  Under this transformation, the field strength is invariant since
  \begin{equation}
    F = dA \to d(A + d\alpha) = dA.
  \end{equation}
  Moreover, since $F = dA$ is exact, we have $dF = d^2A = 0$. This is the \emph{Bianchi identity}, which is equivalent to two of the Maxwell equations.
  We need one more ingredient to write the other two Maxwell equations in terms of differential forms.
\end{example}

\section{Integration}%
\label{sec:integration}

On a manifold, we integrate forms.
\begin{definition}[volume form]
  A \emph{volume form} $v$, also called an \emph{orientation}, is a nowhere-vanishing top form.
  Locally, it can be written as
  \begin{equation}
    v = v(x) dx^1 \wedge dx^2 \wedge \dots \wedge dx^\mu,
  \end{equation}
  with $v(x) \neq 0$ everywhere.
\end{definition}
There are a bunch of subtleties here; for some manifold, it is impossible to find a form like this. 
However, these are not useful in GR; which is why we will brush these subtleties under the carpet.
If such a form exists, $\mathcal{M}$ is said to be orientable. The subtleties mean that not all manifolds are orientable, e.g.~the Möbius strip, or real projective space $\mathbb{R} \mathbb{P}^n$, with $n$ even.

Given a volume form, we can integrate any function $f\colon \mathcal{M} \mapsto \mathbb{R}$ over $\mathcal{M}$.
To do this, we map it to $\mathbb{R}$ via a chart, and then integrate over $\mathbb{R}^n$ as we would usually do.
In a chart $\mathcal{O} \in \mathcal{M}$, we have
\begin{equation}
  \int_{\mathcal{O}} fv = \int_{U} dx^1 \dots dx^\mu f(x) v(x).
\end{equation}
We then sum over patches to integrate over $\mathcal{M}$.
\begin{leftbar}
  \begin{remark}
    In the language of integration, the volume form is the measure. This tells us how to weight functions on the manifold. This is because we have no notion of distance between points on the manifold without it.
  \end{remark}
\end{leftbar}

\begin{definition}[submanifold]
  A manifold $\Sigma$ of dimension $k < n$ is a \emph{submanifold} of $\mathcal{M}$ if there exists a bijection $\phi\colon \Sigma \to \mathcal{M}$, which embeds $\Sigma$ into $\mathcal{M}$, such that $\phi_*\colon T_p(\Sigma) \to T_p(\mathcal{M})$ is also a bijection.
\end{definition}
These definitions make sure that everything is nice and smooth.
We can then integrate any $\omega \in \Lambda^k(\mathcal{M})$ over the submanifold $\Sigma$ by
\begin{equation}
  \int_{\phi(\Sigma)} \omega = \int_{\Sigma} \phi^* \omega,
\end{equation}
where $\omega^*$ is the pull-back.

\begin{example}[]
  Let $\sigma$ be a map $\sigma \colon C \subset \mathbb{R} \to \mathcal{M}$. This defines a non-intersecting curve in $\mathcal{M}$, parametrised by $\tau$. Then, if $A \in \Lambda^1(\mathcal{M})$, in coordinates $x^\mu$, the integral is
  \begin{equation}
    \int_{\sigma(C)} A = \int_{C}\sigma^* A = \int \dd{\tau} A_\mu (x) \dv{x^\mu}{\tau}.
  \end{equation}
  The action in Minkowski space is obtained by pulling back the one-form in Minkowski space $\mathbb{M}^4$ to the real line, and then integrating over this line.
\end{example}

\begin{figure}[tbhp]
  \centering
  \def\svgwidth{0.5\columnwidth}
  \input{lectures/mfdwboundary.pdf_tex}
  \caption{Manifold $\mathcal{M}$ with boundary $\partial \mathcal{M}$.}
  \label{fig:mfdwboundary}
\end{figure}

\begin{theorem}[Stokes' Theorem]
  Let $\mathcal{M}$ be a manifold with boundary $\partial M$, as illustrated in \ref{fig:mfdwboundary}.
  If $\omega \in \Lambda^{n-1}(\mathcal{M})$
  \begin{equation}
    \int_{\mathcal{M}} d\omega = \int_{\partial \mathcal{M}} \omega.
  \end{equation}
\end{theorem}

\begin{example}[]
  Let $\mathcal{M}$ be the one-dimensional interval $I$ with $x \in [a, b]$. The zero-form $\omega(x)$ is a function, and $d\omega = \dv{\omega}{x} dx$ is a one-form.
  Stokes' theorem says that
  \begin{equation}
    \int_{\mathcal{M}} d\omega = \int_{a}^{b} \dv{\omega}{x} \dd[]{x} \quad \text{and} \quad 
    \int_{\partial \mathcal{M}} \omega = \omega(b) - \omega(a)
  \end{equation}
  where the minus sign is related to the subtleties at the boundary, which we have previously swept under the carpet.
  This is the \emph{fundamental theorem of calculus}.
\end{example}

\begin{example}[]
  Let $\mathcal{M} \subset \mathbb{R}^2$ and $\omega \in \Lambda^1(\mathcal{M})$. Then
  \begin{align}
    \int_{\mathcal{M}} d\omega &= \int_{\mathcal{M}} \left( \pdv{\omega_2}{x^1} - \pdv{\omega_1}{x^2} \right) dx^1 \wedge dx^2 \\
    \text{ and } \quad \int_{\partial \mathcal{M}} \omega &= \int_{\partial \mathcal{M}} \omega_1 dx^1 + \omega_2 dx^2.
  \end{align}
  The equality between left and right hand sides is \emph{Green's theorem} in the plane.
\end{example}

\begin{example}[]
  Let $\mathcal{M} \subset \mathbb{R}^3$ and $\omega \in \Lambda^2 (\mathcal{M})$. Then 
  \begin{align}
    \int_{\mathcal{M}} d\omega &= \int dx^1 dx^2 dx^3 \left( \partial_1 \omega_1 + \partial_2 \omega_2 + \partial_3 \omega_3 \right) \\
    \text{and} \quad \int_{\partial \mathcal{M}} \omega &= \int_{\partial \mathcal{M}} \omega_1 dx^2 dx^3 + \omega_2 dx^3 dx^1 + \omega_3 dx^1 dx^2.
  \end{align}
  The equality between these two sides is \emph{Gauss' divergence theorem}.
\end{example}
