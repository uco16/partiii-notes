% lecture notes by Umut Özer
% course: gr
\lhead{Lecture 23: December 06}

\subsection{Power Radiated}%
\label{sub:power_radiated}

We want to understand how much energy is emitted in gravitational waves.
This is a subtle subject; the definition of energy in GR is problematic.

To put this into perspective, we might recall how we did this calculation in Maxwell theory.
In electromagnetism, we compute the power by integrating the momentum carried away (or equivalently the energy flux) by the electromagnetic field through at two-sphere.
\begin{equation}
  \mathcal{P} = \int_{S^2} \dd[2]{S_i} T^{0i},
\end{equation}
where $T^{0i}$ is the \emph{Poynting vector}.

We need to defined the analogue of the Poynting vector for gravitational waves.
We define an energy-momentum tensor $t_{\mu\nu}$ for the gravitational field, which, in the linearised theory, should obey the conservation law
\begin{equation}
  \partial_{\mu} t^{\mu\nu} = 0.
\end{equation}
Unfortunately, there is no such object that is gauge invariant!
Surely, the power emitted should be independent of the coordinates we choose.

The proper way to proceed is to work in $\mathbb{M}^4$ and look at $\mathscr{I}^+$. There is a gauge invariant meaning to the amount of energy that hits $\mathscr{I}^+$ at any given time.
A correct treatment this has to work asymptotically. This is called \emph{Bondi energy}.

We will do things in a more hand-wavy and sloppy way.
This is because we only really care about order-of-magnitude estimates.

There is an obvious approach to defining $t^{\mu\nu}$. From the Fierz-Pauli action, which can be viewed as a theory of the field $h_{\mu\nu}$ in a flat $\mathbb{M}^4$background, we can calculate the Noether current for translations in exactly the same way as you would for any other quantum field theory.

The result with neither be symmetric in $\mu, \nu$, nor gauge invariant.
This is not a surprise; the same thing happens for Maxwell theory; however for Maxwell theory we can add terms that give us both. For the Fierz-Pauli action, we can only add a term that makes it symmetric---we can not get gauge invariance.

Let us take a shortcut. In \texttt{TT} gauge, with $h = 0$ and $\partial_{\mu} h^{\mu\nu} = 0$,the Fierz-Pauli action is 
\begin{equation}
  S_{\text{FP}} = -\frac{1}{8 \pi G} \int \dd[4]{x} \frac{1}{4} \partial_{\rho} h_{\mu\nu} \partial^{\rho} h^{\mu\nu}.
\end{equation}
We now just pretend that this is the action for $10$scalar fields $h^{\mu\nu}$.
The energy density is going to be of the form
\begin{equation}
  t^{00} \sim \frac{1}{G} \dot{h}_{\mu\nu} \dot{h}^{\mu\nu} + (\text{gradient terms}).
\end{equation}
For wave solutions, the gradient term $\nabla h^{\mu\nu} \cdot \nabla h^{\mu\nu}$contributes to the same.

Previously, we had the solutions $\overline{h}_{ij} = \flatfrac{2G}{r} \ddot{I}_{ij}$. This is not in \texttt{TT} gauge. If we put it in this form, we get
\begin{equation}
  h_{ij} \sim \frac{G}{r} \ddot{\mathcal{Q}}_{ij},
\end{equation}
where $\mathcal{Q}_{ij} = I_{ij} - \frac{1}{3} I_{kk} \delta_{ij}$ is the traceless part of the quadrupole.
This suggests that the energy density in gravitational waves far from the source is
\begin{equation}
  t^{00} \sim \frac{G}{r^2} \dddot{\mathcal{Q}}_{ij}^2.
\end{equation}
Integrated over a sphere, this suggests that the power emitted is
\begin{equation}
  \mathcal{P} \sim G \dddot{\mathcal{Q}}^2_{ij}.
\end{equation}
In fact, the correct result is $\mathcal{P} = \frac{G}{5} \dddot{Q}^2_{ij}$, using the Bondi energy at $\mathscr{I}^+$ .
\begin{leftbar}
  \begin{note}
    With squared quantities, the indices are contracted.
  \end{note}
\end{leftbar}

\begin{remark}
  The problem with this derivation is that at every single step we obtain things that depend on the coordinates.
  In fact, if you do things more carefully, you get this term and a bunch of other stuff that depends on the coordinates we are working with. This other stuff vanishes when you take the limit to $\mathscr{I}^+$.
\end{remark}

\begin{remark}
  There are multiple other ways of getting this. One is the Landau-Lifshitz pseudo-tensor.
\end{remark}

\begin{remark}
  The fifth is important; there was a nobel prize awarded to the Holst-Taylor binary.
  Long before LIGO existed, we could see the energy being emitted by the fact that neutron star orbits were decaying.
\end{remark}

\begin{example}[]
  A binary system, two objects that are orbiting each other, with centrifugal force $\omega^2 R \sim GM / R^2$.
  The quadrupole $\mathcal{Q} \sim M R^2$. This means that the third time derivative goes like $\dddot{\mathcal{Q}} \sim \omega^3 M R^2$. The power emitted is
  \begin{equation}
    \mathcal{P} \sim G \dddot{\mathcal{Q}}^2 \sim \frac{G^4M^5}{R^5}.
  \end{equation}
  Since we want actual numbers, we have to put $c$ back in. The above formula is actually correct, however, the Schwarzschild radius for a black hole has a factor $R_S = \flatfrac{2GM}{c^2}$. 
  \begin{equation}
    \label{eq:23-1}
    \mathcal{P} = \left(\frac{R_s}{R}\right)^5 L_{\text{planck}},
  \end{equation}
  where the Planck luminosity is $L_{\text{planck}} = \frac{c^5}{G} \approx 3.6 \times 10^{52} Js^{-1}$.
 To get a feel for the enormity of this, the Sun emits a luminosity $L_\odot \approx 10^{-26} L_{\text{planck}}$.
 All the stars in the observable universe emit $L \sim 10^{-5} L_{\text{planck}}$.

  Nonetheless, when two black holes collide, their distance is roughly $R \approx R_S$ and so the power they emit for that brief fraction of a second is $L \approx L_{\text{planck}}$. To say that LIGO observed violent events in an understatement.
\end{example}

\begin{example}[]
  Two objects with different masses $M^1 \gg M^2$ have a power
  \begin{equation}
    \mathcal{P} \sim \frac{G^4 M_1^3 M_2^2}{R^5}.
  \end{equation}
  We can now start calculating the gravitational waves emitted by other things.
  For Jupiter, its mass is $M^2 \approx 10^{-3} M_\odot$ and its orbit is $R\approx 10^8 km$.
  The power emitted $\mathcal{P} \approx 10^{-44} L_{\text{planck}} = 10^{-18} L_\odot$, which is negligible. What is killing this here is the factor of $5$ is the exponent of \eqref{eq:23-1}.
\end{example}

\begin{example}[]
  Suppose that hit up one of the excellent nightclubs in Cambridge to celebrate the end of Michaelmas term.
  You give it everything you have and wave your arms wildly. How many gravitational waves do you emit?
  $\mathcal{Q} \approx 1 kg m^2$  and $\dddot{\mathcal{Q}} \approx 1 kg m^2 s^{-3}$ . The power emitted while you dance is
  \begin{equation}
    \mathcal{P} \sim \frac{G \dddot{\mathcal{Q}}^2}{c^5} \approx 10^{-52} J s^{-1}.
  \end{equation}
  This is a really low amount. To see how low, consider the next example.
\end{example}

\begin{example}[]
  There is a lower bound to the gravitational wave power you can emit. A single graviton has $E = \hbar \omega$ so if $\omega = 1 s^{-1}$.
  Therefore $E \approx 10^{-34} J$. If you want to dance until you emit a single graviton, you should wave your arms for
  \begin{equation}
    T \approx 10^{18}s \approx 10 \text{ billion years}.
  \end{equation}
  This is one of the examples that exemplify that quantum gravity will never be important in our lifetimes.
\end{example}
