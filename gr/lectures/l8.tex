% lecture notes by Umut Özer
% course: gr
\lhead{Lecture 8: October 30}

\subsection{A Glimpse of Hodge Theory}%
\label{sub:hodge_theory}

There are a few more things we can define with a metric.
\begin{definition}[]
  The \emph{Hodge dual} is a map $\star\colon \Lambda^p(\mathcal{M}) \to \Lambda^{n-p}(\mathcal{M})$, where $n = \dim \mathcal{M}$, defined as
  \begin{equation}
    (\star\omega)_{\mu_1 \dots \mu_{n-p}} = \frac{1}{p!} \sqrt{\abs{g}} \varepsilon_{\mu_1 \dots \mu_{n-p} \nu_1 \dots \nu_p} \omega^{\nu_1 \dots \nu_p}.
  \end{equation}
\end{definition}
\begin{leftbar}
  \begin{remark}
    For $\mathbb{R}^3$, this is what is really going on when we were able to conflate 1-forms and 2-forms.
  \end{remark}
\end{leftbar}
\begin{exercise}
  You can check that $\star(\star\omega) = (-1)^{p(n-p)} s\omega$, where the sign depends on the signature of the metric; $s = +1$ for Riemannian and $s = -1$ for Lorentzian metrics.
\end{exercise}
\begin{corollary}
  The inverse of $\star$ is given by $\star^{-1} \omega = (-1)^{p(n-p)} s \star \omega$
\end{corollary}
\begin{definition}[]
  The \emph{codifferential} $d^{\dagger} = (-1)^k \star^{-1} d \star {}$ is a map from $p$-forms to $(p-1)$-forms.
\end{definition}
\begin{definition}
  The \emph{inner product} of two $p$-forms $\omega, \eta \in \Lambda^p(\mathcal{M})$ is defined as
  \begin{equation}
    \langle \eta, \omega \rangle = \int_{\mathcal{M}} \underbrace{\eta \wedge \star\omega}_{\text{this is a top-form}}.
  \end{equation}
\end{definition}

\begin{claim}
  For $\omega \in \Lambda^p(\mathcal{M})$ and $\alpha \in \Lambda^{p-1}(\mathcal{M})$, we then have
  \begin{equation}
    \langle d\alpha, \omega \rangle = \langle \alpha, d^{\dagger}\omega \rangle.
  \end{equation}
\end{claim}
\begin{proof}
  On a closed manifold, Stokes' theorem implies that
  \begin{align}
    0 &= \int_{\mathcal{M}} d(\alpha \wedge \star\omega) = \int_{\mathcal{M}} \left[ d\alpha \wedge (\star \omega) + (-1)^{p-1} \alpha \wedge d(\star w) \right]   \\
      &= \langle d\alpha, \omega\rangle 
      + (-1)^{p-1} s \langle \alpha, \star d\star\omega \rangle \\
      &= \left\langle d \alpha, \omega \right\rangle - \left\langle \alpha, d^{\dagger} \omega \right\rangle
  \end{align}
\end{proof}

\begin{definition}[]
  The \emph{Laplace-deRham} operator is $\Delta = (d^{\dagger} + d)^2 = d^{\dagger} d + d d^{\dagger}$.
\end{definition}
Like the Laplacian, it is the divergence of the gradient in the language of differential forms. It is self adjoint, which gives $(\Delta \omega, \alpha) = (\omega, \Delta \alpha)$, and the study of \emph{harmonic forms}, which satisfy $\Delta \omega = 0 $ lies at the heart of Hodge theory.

\begin{leftbar}
  \begin{remark}
    Compare this to the Hermitian conjugate of an operator in quantum mechanical inner products on a Hilbert space. Most of the time, we do not play with forms in QM.
    However, there is a close relationship between forms in differential geometry and fermions in QFT; this relationship will be fleshed out in the Lent term's \emph{Supersymmetry} course.
  \end{remark}
\end{leftbar}

\section{Connections}%
\label{sec:connections_and_curvature}

We have met two derivatives so far. We will now meet the final, and ultimately most useful one.

\begin{definition}
  A \emph{connection} is a map $\nabla \colon \mathfrak{X}(\mathcal{M}) \times \mathfrak{X}(\mathcal{M}) \to \mathfrak{X}(\mathcal{M})$
  We write this as $\nabla(X, Y) = \nabla_X Y$ to make it look more like differentiation.
  Here, $\nabla_X$ is called the \emph{covariant derivative} and it satisfies the following properties:
  \begin{itemize}
    \item linearity on the first argument: $\nabla_{fX + gY} Z = f\nabla_X Z + g \nabla_Y Z $ for all functions $g, g$
    \item some linearity on the second argument: $\nabla_X (Y + Z) = \nabla_X Y + \nabla_X Z$
    \item but functions in the second argument obey the Leibniz rule: $\nabla_X (f Y) = f\nabla_X Y + (\nabla_X f) Y$, where $\nabla_X f \coloneqq X(f)$.
  \end{itemize}
\end{definition}
\begin{leftbar}
  \begin{remark}
    This last term in the Leibniz rule is the reason why this is not a linear map, and thus not a tensor.
  \end{remark}
\end{leftbar}
\begin{leftbar}
  \begin{remark}
    We often refer to the abstract object and also its components both simply as the \emph{connection}.
  \end{remark}
\end{leftbar}
On a basis $\{ e_\mu\}$ of vector fields, not necessarily a coordinate basis, we write
\begin{equation}
  \nabla_{e_\rho} e_\nu = \Gamma \indices{_{\rho\nu}^{\mu}} e_{\mu}.
\end{equation}
This defines $\Gamma$, but it is simply the most general form it can be.
We also use the notation $\nabla_{e_\mu} = \nabla_\mu$ to make the connection look like a partial derivative.
Then for a general vector
\begin{align}
  \nabla_X Y &= \nabla_X (Y^\mu e_\mu) = X (Y^\mu) e_\mu + Y^\mu \nabla_X e_\mu. \\
	     &= X^\nu e_\nu (Y^\mu) e_\mu + Y^\mu X^\nu \nabla_\nu e_\mu = X^\nu (e_\nu (Y^\mu) + \Gamma \indices{^\mu_\nu_\rho} Y^\rho) e_\mu.
\end{align}
Because the components $X^\nu$ sits out front, we can write $\nabla_X Y = X^\nu \nabla_n Y$ with 
\begin{equation}
  \nabla_n Y = (e_\nu (Y^\mu) + \Gamma \indices{^\mu_{\nu\rho}} Y^\rho) e_\mu,
\end{equation}
or with a slightly slippery notation $(\nabla_\nu Y)^\mu \coloneqq \nabla_\nu Y^\mu = e_\nu (Y^\mu) + \Gamma \indices{^\mu_\nu_\rho} Y^\rho$.

\begin{leftbar}
  \begin{remark}
    For $\mathcal{L}_X$ depends on $X$ and $\partial X$, so we cannot write something like ``$\mathcal{L}_X = X^\mu \mathcal{L}_\mu$''.
    This is ultimately why the covariant derivative is more useful. The flip side is that the Lie derivative comes for free, while the numbers $\Gamma$ are extra structure that needs to be specified on the manifold.
  \end{remark}
\end{leftbar}
If we take a coordinate basis $\{e_\mu\} = \{\partial_\mu\}$, then
\begin{equation}
  \nabla_\nu Y^\mu = \partial_\nu Y^\mu + \Gamma \indices{^\mu_\nu_\rho} Y^\rho.
\end{equation}
\begin{leftbar}
  \begin{remark}
    This should be familiar from previous courses in GR or fluid dynamics.
  \end{remark}
\end{leftbar}
\begin{notation}[]
  We often write the covariant derivative as $\nabla_\nu Y^\mu \coloneqq Y \indices{^\mu_{; \nu}}$ with a semicolon and the partial derivative with a comma as $\partial_\nu Y^\mu \coloneqq Y \indices{^\mu_{, \nu}}$.
\end{notation}
\begin{leftbar}
  \begin{remark}
    The name \emph{connection} hints that we can use this to connect tangent spaces at different points in the manifold.
  \end{remark}
\end{leftbar}
\begin{claim}
  The connection is \emph{not} a tensor.
\end{claim}
\begin{proof}
  Consider a change of basis $\tilde e_\nu = A \indices{^\mu_\nu} e_\mu$ with $A \indices{^\mu_\nu} = \pdv{x^\mu}{\tilde x^\nu}$ for a coordinate bases.
  Then 
  \begin{align}
    \nabla_{\tilde e_\rho} \tilde e_\nu &= \tilde \Gamma \indices{^\mu_\rho_\nu} \tilde e_\mu \\
					&= \nabla_{(A \indices{^\sigma_\rho} e_\sigma)} (A \indices{^\lambda_\nu} e_\lambda) \\
					&= A \indices{^\sigma_\rho} \nabla_\sigma (A \indices{^\lambda_\nu} e_\lambda)
  \end{align}
  Using the Leibniz rule we have
  \begin{align}
    \dots &= A \indices{^\sigma_\rho} (A \indices{^\lambda_\nu} \Gamma \indices{^\tau_\sigma_\lambda} e_\tau + e_\lambda \partial_\sigma A \indices{^\lambda_\nu}) \\
	  &= A \indices{^\sigma_\rho} (A \indices{^\lambda _\nu} \Gamma \indices{^\tau_\sigma_\lambda} + \partial_\sigma A \indices{^\tau_\nu}) e_\tau, \qquad e_\tau = (A^{-1}) \indices{^\mu_\tau} \tilde e_\mu\\
	  &\implies \tilde\Gamma \indices{^\mu_\rho_\nu} = (A^{-1}) \indices{^\mu_\tau} A \indices{^\sigma_\rho} A \indices{^\lambda_\nu} \Gamma \indices{^\tau_\sigma_\lambda} + \underbrace{(A^{-1}) \indices{^\mu_\tau} A \indices{^\sigma_\rho} \partial_\sigma A \indices{^\tau_\nu}}_{\text{this is why it is not a tensor}}.
  \end{align}
\end{proof}
\begin{leftbar}
  \begin{remark}
    There are other objects, where things transform as $\text{something} \to \text{something} + \partial\text{other}$. The Maxwell tensor and the Yang-Mills gauge potential are examples of these. 
    Mathematicians actually call both of these objects \emph{connections}, precisely because of this property. The connections will become more obvious in the Lent term's \emph{Applications of Differential Geometry to Physics} course.
  \end{remark}
\end{leftbar}

We can also use the connection to differentiate other tensors. We simply ask that it obeys the Leibniz rule.
\begin{example}[]
  We can use the covariant derivative to differentiate a function. Let us think of a one-form $\omega \in \Lambda^1(\mathcal{M})$ acting on a vector field $Y \in \mathfrak{X}(\mathcal{M})$ as that function. Using the Leibniz rule,
  \begin{align}
    \nabla_X (\omega(Y)) &= X(\omega(Y)) = (\nabla_X \omega)(Y) + \omega (\nabla_X Y) \\
    \implies (\nabla_X \omega) (Y) &= (\omega(Y)) -\omega \nabla_X (Y).
  \end{align}
  In terms of coordinates, we have
  \begin{align}
    X^\mu(\nabla_\mu \omega_\nu) Y^\nu &= X^\mu \partial_\mu (\omega_\nu X^\nu) - \omega_\nu X^\mu (\partial_\mu Y^\nu + \Gamma \indices{^\nu_\mu_\rho} Y^\rho \\
				       &= X^\mu (\partial_\mu \omega_\rho - \Gamma \indices{^\nu_\mu_\rho} \omega_\nu) Y^\rho \\
				       \implies \nabla^\mu \omega_\rho &= \partial_\mu \omega_\rho - \Gamma \indices{^\nu_\mu_\rho} \omega_\nu.
  \end{align}
\end{example}
For a general $(p, q)$-tensor, we have one term for each index:
\begin{equation}
  \nabla_\rho T \indices{^{\mu_1 \dots \mu_p}_{\nu_1 \dots \nu_q}} = \partial_\rho T \indices{^{\mu_1 \dots \mu_p}_{\nu_1 \dots \nu_q}} + \Gamma \indices{^{\mu_1}_\rho_\sigma} T \indices{^{\sigma \mu_2 \dots \mu_p}_{\nu_1 \dots \nu_q}} + \dots - \Gamma^\sigma_{\rho \nu_1} T \indices{^{\mu_1 \dots \mu_p}_{\sigma \nu_2 \dots \nu_q}} - \dots.
\end{equation}

