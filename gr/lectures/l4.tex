% lecture notes by Umut Özer
% course: gr
\lhead{Lecture 4: October 21}

\section{Tensors}%
\label{sec:tensors}

\begin{definition}[Dual Space]
  Given a vector space $V$, there exists a dual vector space $V^*$ which is the space of all linear maps $V \to \mathbb{R}$.
\end{definition}

The easiest way to define $V^*$ is to come up with basis.
Given a basis $ \left\{ e_\mu: \mu = 1, \ldots, n \right\}$ of the first vector space $V$, we have a dual basis
$\left\{ f^\mu \right\}$ of $V^*$ defined by
\begin{equation}
  f^\mu(e_\nu) = \delta^\mu_\nu.
\end{equation}
Given a general element $X = X^\mu e_\mu \in V$, the linearity property means that we have $f^\nu(X) = X^\mu f^\nu (e_\mu) = X^\nu$.
\begin{leftbar}
  \begin{remark}
    Given a basis of $V$, this means that we have a natural basis of $V^*$, giving an isomorphism from $V$ to $V^*$. But the map defining this isomorphism depends on the basis.
  \end{remark}
\end{leftbar}

\begin{leftbar}
  \begin{remark}
    Most vector spaces that we meet in physics come endowed with an inner product. The vector space $V$ does not (yet) have an inner product structure.
    If you do have an inner product, defined by a metric, then there is a natural basis-independent isomorphism between $V$ and $V^*$.
  \end{remark}
\end{leftbar}

\begin{leftbar}
  \begin{remark}
    $V^{* *} = V$
  \end{remark}
\end{leftbar}

\begin{definition}[cotangent space]
  At each point $p \in \mathcal{M}$, the tangent space $T_p(\mathcal{M})$ is a vector space. The dual space $T^*_p (\mathcal{M})$ is called the \emph{cotangent space}.
\end{definition}

\begin{definition}[1-forms]
  A smooth assignment of cotangent vectors at each point is called a \emph{cotangent field}. However, most people just call this a \emph{1-form}.
\end{definition}

\begin{notation}
  We denote the space of all 1-forms over $\mathcal{M}$ as $\Lambda^i (\mathcal{M})$.
\end{notation}

\begin{definition}[]
  Given a smooth function $f \in C^\infty (\mathcal{M})$, there is a natural 1-form that we call $df \in \Lambda^1 (\mathcal{M})$, defined by
  \begin{equation}
    \underbrace{df}_{\text{1-form}}(\overbrace{X}^{\text{vector-field}}) = \overbrace{\underbrace{X}_{\text{vector field}} (\underbrace{f}_{\text{function}})}^{\text{function}}
  \end{equation}
\end{definition}

\begin{example}[]
  Suppose we have coordinates $x^\mu$. These define a basis $e_\mu = \pdv*{}{x^\mu}$ of vector fields.
  Taking a particular $f = x^\mu$ (this $f$ is a function) gives $dx^\mu(\pdv*{}{x^\nu}) = \pdv*{}{x^\nu} (x^\mu) = \delta^\mu_\nu$.
  So $f^\mu = dx^\mu$ (this $f^\mu$ is a basis element) provides a basis for $\Lambda^1(\mathcal{M})$. In general, $\omega \in \Lambda^1 (\mathcal{M})$ can be written (in a given chart) as 
  \begin{equation}
    \omega = \omega_\mu(x) dx^\mu.
  \end{equation}
  \begin{leftbar}
    \begin{remark}
      Note that this only holds locally, not necessarily globally.
    \end{remark}
  \end{leftbar}

  In this basis, the 1-form is $df = \pdv{f}{x^\mu} dx^\mu$, so that we have
  \begin{equation}
    df(X) = \pdv{f}{x^\mu} dx^\mu (X^\nu \partial_\nu) = X^\mu \pdv{f}{x^\mu} = X(f), \qquad \partial_\nu \coloneqq \pdv{}{x^\nu},
  \end{equation}
  as expected.
\end{example}

If we change coordinates from $x^\mu$ to $\tilde x^\mu(x)$, we have seen previously that the basis of vector fields changes as
\begin{equation}
  \pdv{}{\tilde x^\mu} = \pdv{x^\nu}{\tilde x^\mu} \pdv{}{x^\nu}.
\end{equation}
The basis of 1-forms changes as
\begin{equation}
  d\tilde x^\mu = \pdv{\tilde x^\mu}{x^\nu} d x^\nu.
\end{equation}
These are called \emph{covariant} transformations.

This ensures that
\begin{equation}
  d\tilde x^\mu \left( \pdv{}{\tilde x^\nu} \right) = \pdv{\tilde x^\mu}{x^\rho} dx^\rho \left( \pdv{x^\sigma}{\tilde x^\nu} \pdv{}{x^\sigma} \right) = \pdv{\tilde x^\mu}{x^\rho} \pdv{x^\sigma}{\tilde x^\nu} dx^\rho \left( \pdv{}{x^\sigma} \right) = \pdv{\tilde x^\mu}{x^\rho} \pdv{x^\rho}{x^\nu} = \delta^\mu_\nu
\end{equation}
as required.
\begin{leftbar}
  \begin{remark}
    This transformation looks very much like the Jacobian incurred by a change of variables in a multi-variate integral. We will see the relation between 1-forms and integration in the future.
  \end{remark}
\end{leftbar}

The question is now, how do we differentiate 1-forms?
Given a vector field $X \in \mathfrak{X}(\mathcal{M})$ and $\omega \in \Lambda^1 (\mu)$, we can take the Lie derivative $\mathcal{L}_X \omega$. \par
First, note that 1-forms are pulled-back: If $\varphi: \mathcal{M} \to \mathcal{N}$ and $\omega \in \Lambda^1(\mathcal{N})$, then $\varphi^* \omega \in \Lambda^1(\mathcal{M})$ with
\begin{equation}
  (\varphi^* \omega) X = \omega (\varphi_* X), \qquad \varphi_* X \in \mathfrak{X}(\mathcal{N}),
\end{equation}
with $(\varphi^* \omega)_\mu = \omega_\alpha \pdv{y^\alpha}{x^\mu}$, where $y^\alpha$ are coordinates on $\mathcal{N}$ and $x^\mu$ are coordinates on $\mathcal{M}$.

Now we have the Lie derivative
\begin{equation}
  \mathcal{L}_X \omega = \lim_{t \to 0} \left[ \frac{(\sigma_t^* \omega)_p - \omega_p}{t} \right].
\end{equation}
% F1

Now $x_\mu(t) = x^\mu(0) + X^\mu(x) t + \ldots$, which means that the push-forward of a 1-form
\begin{equation}
  \sigma_t^* dx^\mu = \left( \delta^\mu_\nu + t \pdv{X^\mu}{x^\nu} + \ldots \right) dx^\nu
\end{equation}
and the Lie derivative takes the form
\begin{equation}
  \mathcal{L}_X \omega = (X^\nu \partial_\nu \omega_\mu + \omega_\nu \partial_\mu X^\nu) dx^\mu.
\end{equation}
This is the \emph{Lie derivative} of a 1-form.

\subsection{Tensor Fields}%
\label{sub:tensor_fields}

\begin{definition}[tensor]
  A \emph{tensor} of rank $(r, s)$, $r, s \in \mathbb{N}_0^+$, at a point $p \in \mathcal{M}$ is a multi-linear map
  \begin{equation}
    T: \underbrace{T^*_p(\mathcal{M}) \times \cdots \times T_p^*(\mathcal{M})}_{r}
    \underbrace{T_p(\mathcal{M}) \times \cdots \times T_p(\mathcal{M})}_{s}
    \to \mathbb{R}.
  \end{equation}
\end{definition}

\begin{definition}[rank]
  The \emph{total rank} of the tensor is $r + s$.
\end{definition}

\begin{example}[]
  A cotangent vector has rank $(0, 1)$.\par
  A tangent vector has rank $(1, 0)$ (since $T^{**}_p(\mathcal{M}) = T_p(\mathcal{M})$)
\end{example}

\begin{definition}[tensor field]
  A \emph{tensor field} is a smooth assignment of an $(r, s)$ tensor to each point $p \in \mathcal{M}$.
\end{definition}

\begin{claim}
  Given a basis $\left\{ e_\mu \right\}$ for tangent vectors and a dual basis $\left\{ f^\mu \right\}$, a tensor has components
  \begin{equation}
    T \indices{^{\mu_1 \cdots \mu_r}_{\mu_1 \cdots \mu_s}} = T(f^{\mu_1}, \ldots, f^{\mu_r}, e_{\nu_1}, \ldots, e_{\nu_s}).
  \end{equation}
\end{claim}
