% lecture notes by Umut Özer
% course: gr
\lhead{Lecture 4: October 22}
% I deleted this file and lost my lecture notes for that day :'( 
% These notes are adapted from Prof Tong's lecture notes.

\section{Tensors}%
\label{sec:tensors}

\subsection*{Dual Spaces}%

\begin{definition}[dual space]
  Let $V$ be a vector space. The \emph{dual vector space} $V^*$ is the space of all linear maps from $V$ to $\mathbb{R}$.
\end{definition}

The easiest way to define the dual space is to work with a given basis $\{e_\mu\}$, $\mu = 1, \dots, n$ of $V$.
We can then introduce a dual basis $\{f^\mu\}$, $\mu = 1, \dots, n$ for $V^*$, defined by
\begin{equation}
  f^\nu (e_\mu) = \delta^\nu_\mu.
\end{equation}
The dual basis elements $f^\mu$ are linear maps which pick out their corresponding basis element $e_\mu$ and send it to unity. 
By definition, a general vector $X \in V$ can be decomposed in this basis as $X = X^\mu e_\mu$.
Since the maps are linear, the dual basis elements act on a general vector hence as $f^\nu(X) = X^\mu f^\nu (e_\mu) = X^\nu$.
For a given basis, the correspondence between the bases provides an isomorphism between $V$ and $V^*$. In particular, this means that $\dim V = \dim V^*$. However, this map $e_\mu \mapsto f^\mu$ depends on the basis and is thus a bad isomorphism to work with in a coordinate invariant theory.
We can repeat the construction to show that there is a natural isomorphism $(V^*)^* \to V$, which is independent of the choice of basis.

\subsection{One-Forms}%
\label{sub:one_forms}

\begin{definition}[cotangent space and vector]
  The \emph{cotangent space} $T_p^*(\mathcal{M})$ at a point $p \in \mathcal{M}$ is the dual space to the tangent space $T_p(\mathcal{M})$ at $p$. Elements of the cotangent space are called \emph{cotangent vectors} or \emph{covectors}.
\end{definition}
Given a basis $\{ e_\mu \}$ of the tangent space $T_p(\mathcal{M})$, we can introduce a dual basis $\{f^\mu\}$ for $T_p^*(\mathcal{M})$ and expand any covector as $\omega = \omega_\mu f^\mu$.

\begin{definition}[one-form]
  A \emph{one-form} or \emph{cotangent field} is a smooth assignment of cotangent vectors taken from the cotangent spaces across all points in the manifold.
  These one-forms are maps from vector fields to real numbers.
\end{definition}
\begin{notation}[]
  The set of all one-forms on $\mathcal{M}$ is denoted $\Lambda^1(M)$.
\end{notation}

\begin{definition}[differential]
  The \emph{differential} $df$ of a smooth function $f$ on $\mathcal{M}$ is a special one-form. It is defined by its action on a vector field $X$:
  \begin{equation}
    \label{eq:def_differential}
    \begin{split}
      df \in \Lambda^1(\mathcal{M}) \colon \mathfrak{X} &\to \mathbb{R} \\
      X &\mapsto X(f).
    \end{split}
  \end{equation}
\end{definition}
These differentials can be used to find a basis for the space $\lambda^1(\mathcal{M})$ of all one-forms on $\mathcal{M}$:
Let $\{x^\mu\}$ be coordinates on $\mathcal{M}$. This defines a coordinate basis $\{\pdv{}{x^\mu} = \partial_\mu \}$ of vector fields.
Then, for each value of $\mu$, we take the function $f$ that we used to build the differential to simply be that coordinate $x^\mu$ to give a one-form $dx^\mu$. By definition, this acts on the basis of vector fields as
\begin{equation}
  dx^\mu (\partial^\nu) = \partial^\nu(x^\mu) = \delta^\mu_\nu.
\end{equation}
Therefore, the set $\{dx^\mu\}$ provides a basis for $\Lambda^1 (\mathcal{M})$ that is dual to the coordinate basis $\{\partial_\mu\}$.
In general, an arbitrary one-form $\omega \in \Lambda^1(\mathcal{M})$ can hence be expanded as $\omega = \omega_\mu dx^\mu$.
\begin{claim}
  In this basis, the coefficients $\omega_\mu$ are the partial derivatives; the differential can be expanded as
  \begin{equation}
    df = \pdv{f}{x^\mu} dx^\mu.
  \end{equation}
\end{claim}
\begin{proof}
  We can check that this expansion satisfies Equation \eqref{eq:def_differential} by acting with it on a vector field $X$:
  \begin{equation}
    df(X) = \pdv{f}{x^\mu} dx^\mu (X^\nu \partial_\nu) = X^\mu \pdv{f}{x^\mu} = X(f).
  \end{equation}
\end{proof}
How do one-forms transform under a change of coordinates?
Recall that given two charts $\phi = (x^1, \dots, x^n)$ and $\widetilde\phi = (\tilde x^1, \dots, \tilde x^n)$, the partial derivatives, which form the basis of vector fields, are related via the chain rule:
\begin{equation}
  \pdv{}{\tilde x^\mu} = \pdv{x^\nu}{\tilde x^\mu} \pdv{}{x^\nu}.
\end{equation}
Since the differentials form the corresponding dual basis, they transform in the inverse manner
\begin{equation}
  d\tilde x^\mu = \pdv{\tilde x^\mu}{x^\nu} dx^\nu.
\end{equation}
\begin{claim}
  This transformation law ensures that the new basis $\{d \tilde x^\mu\}$ is also dual to the new chart $\widetilde \phi$.
\end{claim}
\begin{proof}
  This is an exercise in definitions and symbol manipulation:
  We need to show that 
  \begin{equation}
    d \tilde x^\mu \left( \pdv{}{\tilde x^\nu} \right) = \delta^\mu_\nu.
  \end{equation}
  Starting from the left hand side, we first use the two transformation laws
  \begin{equation}
    d \tilde x^\mu \left( \pdv{}{\tilde x^\nu} \right) = \pdv{\tilde x^\mu}{x^\rho} dx^\rho \left( \pdv{x^\sigma}{\tilde x^\nu} \pdv{}{x^\sigma} \right)
  \end{equation}
  Then we can pull out the factor $\pdv*{x^\sigma}{\tilde x^\nu}$ since the one-form $dx^\rho$ only acts on the vector field $\partial^\sigma$:
  \begin{equation}
    \dots = \pdv{\tilde x^\mu}{x^\rho} \pdv{x^\sigma}{\tilde x^\nu} dx^\rho \left( \pdv{}{x^\sigma} \right) = \pdv{\tilde x^\mu}{x^\rho} \pdv{x^\sigma}{\tilde x^\nu} \delta^\rho_\sigma = \pdv{\tilde x^\mu}{x^\rho} \pdv{x^\sigma}{\tilde x^\rho}.
  \end{equation}
  Lastly, we use the fact that these change of variable matrices are inverses
  \begin{equation}
    \dots =\pdv{\tilde x^\mu}{x^\rho} \pdv{x^\rho}{\tilde x^\nu} = \delta^\mu_\nu.
  \end{equation}
\end{proof}
Therefore, we can expand a one-form $\omega$ in the new basis as well
\begin{equation}
  \omega = \omega_\mu dx^\mu = \tilde \omega _\mu d\tilde x^\mu \quad \text{with} \quad \tilde \omega_\mu = \pdv{x^\nu}{\tilde x^\mu} \omega_\nu.
\end{equation}
We say that these transformations of covectors are \emph{covariant} transformations.

\subsection{Lie Derivative of One-Forms}%
\label{sub:lie_derivative_of_one_forms}


