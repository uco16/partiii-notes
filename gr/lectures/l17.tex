% lecture notes by Umut Özer
% course: gr
\lhead{Lecture 17: November 20}

\section{Asymptotics of Spacetime}%
\label{sec:asymptotics_of_spacetime}

Given a spacetime $\mathcal{M}$, with metric $g_{\mu\nu}(x)$, we consider the \emph{conformal transformation}
\begin{equation}
  \widetilde{g}_{\mu\nu}(x) = \Omega^2 (x) g_{\mu\nu}(x),
\end{equation}
where $\Omega$ is smooth, non-zero.
$g_{\mu\nu}$ and $\widetilde{g}_{\mu\nu}$ describe different spacetimes, but they have the same causal structure since
\begin{equation}
  g_{\mu\nu} X^{\mu} X^{\nu} = 0 \quad \iff \quad \widetilde{g}_{\mu\nu} X^{\mu} X^{\nu} = 0.
\end{equation}
So null / spacelike / timelike in $g_{\mu\nu}$ maps to null / spacelike / timelike in $\widetilde{g}_{\mu\nu}$.

\section{Penrose Diagrams}%
\label{sec:penrose_diagrams}

The idea is to use conformal transformations to bring infinity a little closer.
\begin{leftbar}
  \begin{note}
    We will try to draw a finite picture of the whole manifold $\mathcal{M}$ by stretching an squeezing our coordinates. But we still want the picture to accurately depict the causal structure of the spacetime ($\mathcal{M}, g$).
  \end{note}
\end{leftbar}

\subsection{Minkowski Space}%
\label{sub:minkowski_space}

Start with $\mathbb{R}^{1, 1}$ and metric $ds^2 = -dt^2 + dx^2$.
Introduce lightcone coordinates 
\begin{equation}
  u = t - x \qquad v = t + x
\end{equation}
in which the metric is $ds^2 = -du dv$ with $u, v \in (- \infty, \infty)$.
We now map this to a finite range
\begin{equation}
  u = \tan \widetilde{u} \quad \text{and} \quad v = \tan \widetilde{v}, 
\end{equation}
with $\widetilde{u}, \widetilde{v} \in (- \frac{\pi}{2}, + \frac{\pi}{2})$.
\begin{equation}
  ds^2 = - \frac{1}{\cos^2 \widetilde{u} \cos^2 \widetilde{v}} d \widetilde{u} d \widetilde{v}
\end{equation}
Consider the new metric
\begin{equation}
  d \widetilde{s}^2 = \cos ^2 \widetilde{u} \cos ^2 \widetilde{v} ds^2 = - d \widetilde{u} d \widetilde{v}.
\end{equation}
\begin{leftbar}
  \begin{note}
    The reason we use lightcone coordinates is the following: if we just proceed with $t = \tan\tilde{t}$ and $x = \tan\tilde{x}$, then we end up with $ds^2 =- \frac{1}{\cos^2 \widetilde{t}} d\tilde{t} + \frac{1}{\cos^2 \tilde{x}} d\tilde{x}$. And trying to pull out the time prefactor, we get an ugly factor $\frac{cos^2\tilde{t}}{cos^2\tilde{x}}$ in front of $d\tilde{x}$.
  \end{note}
\end{leftbar}
There is a bit of a technicality there. Strictly speaking the points at the edge of the spacetime were not included in the first place, whereas $\widetilde{u}, \widetilde{v} \in [-\frac{\pi}{2}, + \frac{\pi}{2}]$.
Adding the points $\pm \frac{\pi}{2}$, which used to be $\pm \infty$, is called a \emph{conformal compactification}.
\begin{leftbar}
  \begin{remark}
    There is a theorem by Penrose that shows that this is essentially the unique way of doing conformal complexification.
  \end{remark}
\end{leftbar}

\begin{figure}[tbhp]
  \centering
  \begin{minipage}[t]{0.5\columnwidth}
    \centering
    \def\svgwidth{0.9\textwidth}
    \input{lectures/l17f1.pdf_tex}
    \caption{}
    \label{fig:l17f1}
  \end{minipage}%
  \begin{minipage}[t]{0.5\columnwidth}
    \centering
    \def\svgwidth{0.9\textwidth}
    \input{lectures/l17f2.pdf_tex}
    \caption{}
    \label{fig:l17f2}
  \end{minipage}
  \caption{Penrose Diagrams}
  \label{fig:penrose-diagrams}
\end{figure}
Now we draw the spacetime with light rays at $45^\circ$ and time vertical. These diagrams, as depicted in Figure \ref{fig:l17f1} are called \emph{Penrose diagrams}.
Note that we cannot trust distances in these diagrams; things that look close might be very far apart in Minkowski space. However, we can trust the causal structure.
We can draw various geodesics on this diagram. 
In particular, we can draw timelike geodesics \textcolor{yellow}{(constant x)} %draws in yellow
and spacelike geodesics \textcolor{red}{(constant t)} % drawn in red.
as in Figure \ref{fig:l17f2}.
\begin{description}
  \item[Note:] All timelike geodesics start at $i^- \colon [- \frac{ \pi}{2}, - \frac{\pi}{2}]$ and end at $i^+ \colon [+ \frac{\pi}{2}, + \frac{\pi}{2}]$.
    These are called \emph{past / future timelike infinity}.
  \item[\bullet] All spacelike geodesics start / end at two points $i^0\colon [- \frac{\pi}{2}, + \frac{\pi}{2}]$ or $[+ \frac{\pi}{2}, - \frac{\pi}{2}]$. These are \emph{spacelike infinity}.
  \item[\bullet] All null curves start at $\mathscr{I}^-$ `scri-minus' and end at $\mathscr{I}^+$ `scri-plus'.\footnote{The strange names may be related to the fact that the symbol $\mathscr{I}$ is produced by the command \texttt{\textbackslash mathscr\{I\}} in \LaTeX.} These are \emph{past / future null infinity}.
\end{description}
\begin{leftbar}
  \begin{note}
    If these names do not make sense, consider that we write $\mathscr{I}$ as \texttt{\char`\\mathscr\{I\}} in \LaTeX.
  \end{note}
\end{leftbar}
\begin{leftbar}
  \begin{remark}
    In some sense, most of infinity is given by the diagonal lines $\mathscr{I}$.
  \end{remark}
\end{leftbar}

The Penrose diagram immediately tells us basic things about the spacetime.
For example, and two points in the spacetime have a common future and a common past.
%F3
For any two points, draw $45^\circ$ lines, as illustrated in Figure \ref{fig:l17f3}.
\begin{figure}[tbhp]
  \centering
  \begin{minipage}[t]{0.45\columnwidth}
    \centering
    \def\svgwidth{0.9\columnwidth}
    \input{lectures/l17f3.pdf_tex}
    \caption{Any two points, even spacelike, share a common future and common past---spacetime regions in which their lightcones overlap.}
    \label{fig:l17f3}
  \end{minipage}%
  \qquad
  \begin{minipage}[t]{0.45\columnwidth}
    \centering
    \def\svgwidth{0.65\columnwidth}
    \input{lectures/l17f4.pdf_tex}
    \caption{Penrose diagram for $\mathbb{R}^{1, 3}$ in polar coordinates. Since $r \geq 0$, we only draw the half where $\tilde{v} \geq \tilde{u}$.}
    \label{fig:l17f4}
  \end{minipage}
\end{figure}

Since we can trust the causal structure of Penrose diagrams, this illustrates clearly that any two points, even if they are spacelike separated, they have a common future and common past since their lightcones overlap.
If they follow timelike geodesics, they will eventually end up in their common future.

For $\mathbb{R}^{1, 3}$, we do something similar.
The metric is best written in polar coordinates
\begin{equation}
  ds^2 = - dt^2 + dt^2 + r^2 d\Omega^2_2,
\end{equation}
where $d\Omega^2_2$ is the metric on the two-sphere $S^2$.
Let $u = t - r = \tan \widetilde{u}$ and $v = t + r = \tan \widetilde{v}$. Then 
\begin{align}
  ds^2 &= -du dv + \frac{1}{4} (u-v)^2 d\Omega^2_2  \\
       &= \frac{1}{4 \cos^2 \widetilde{u} \cos^2 \widetilde{v}} \left( -4 d \widetilde{u} d \widetilde{v} + \sin^2 (\widetilde{u} - \widetilde{v}) d\Omega^2_2 \right).
\end{align}
\begin{leftbar}
  \begin{note}
    Recall the identity $\sin(\tilde{v})\cos(\tilde{u}) = \frac{1}{2}(\sin(\tilde{v} + \tilde{u}) + \sin(\tilde{v} - \tilde{u}))$ to show that
    \begin{equation}
      \tan(\tilde{v}) - \tan(\tilde{u}) = \frac{\sin(\tilde{v} - \tilde{u})}{\cos(\tilde{v})\cos(\tilde{u})}.
    \end{equation}
  \end{note}
\end{leftbar}

There is one other subtlety, which stems from the coordinates. Unlike in the $2d$ case, the radial coordinate $r$ has to be non-negative $r \geq 0$. This means that $v \geq u$ and therefore 
\begin{equation}
  -\frac{\pi}{2} \leq \widetilde{u} \leq \tilde{v} \leq \frac{\pi}{2}.
\end{equation}
We drop the $S^2$ and draw the Penrose diagram, which is now only the half for which $\tilde{v} \geq \tilde{u}$.
This is shown in \ref{fig:l17f4}.
%F4
The `boundary' on the left is \emph{not} a boundary of spacetime! It is $\tilde{u} = \tilde{v}$, so $r = 0$, and the two-sphere  $S^2$ shrinks to zero size.
In particular, notice the behaviour of a light ray.

\subsection{de Sitter}%
\label{sub:de_sitter}

We are going to work in global coordinates, since they cover the whole of spacetime:
\begin{equation}
  \label{eq:17-metric-dS}
  ds^2 = - d\tau^2 + R^2 \cosh^2 \left(\frac{\tau}{R}\right) d\Omega^2_3.
\end{equation}
We now introduce what cosmologists call \emph{conformal time} $\eta \in (- \frac{\pi}{2}, + \frac{\pi}{2})$. 
\begin{equation}
  \dv[]{\eta}{\tau} = \frac{1}{R \cosh(\tau/ R)} \quad \implies \cos \eta = \frac{1}{\cosh ( \tau / R)}.
\end{equation}
The purpose of conformal time is to pull out the factor of $R^2 \cosh^2$ to the front of the metric.
Plugging this into \eqref{eq:17-metric-dS}, we have
\begin{equation}
  ds^2 = \frac{R^2}{\cos^2 \eta} (- d\eta^2 + d\Omega^2_3).
\end{equation}
Here, the metric on the three-sphere is $d\Omega^2_3 = d\chi^2 + \sin^2 \chi d\Omega^2_2$, where $\chi \in [0, \pi]$.
This means that de Sitter is conformal to
 \begin{equation}
  ds^2 = -d\eta^2 + d\chi^2 + \sin^2 \chi d\Omega^2_2.
\end{equation}
The difference is that these are not lightcone coordinates. This means that the Penrose diagram for \texttt{dS} is just a square, as in Figure \ref{fig:l17f5}.
%F5
\begin{figure}[bthp]
  \centering
  \def\svgwidth{0.5\columnwidth}
  \input{lectures/l17f5.pdf_tex}
  \caption{}
  \label{fig:l17f5}
\end{figure}
The vertical lines are not boundaries of the spacetime. They correspond to the north pole of $S^3$ and south pole of $S^3$ respectively.
The horizontal lines do correspond the boundaries.
We see that the boundary of \texttt{dS} is spacelike.
There is no spatial infinty in \texttt{dS}; the spatial regions are compact manifolds without asymptotic region.
Rotating by $45^\circ$ has changed the physics dramatically. None of the above statements about $\mathbb{M}$ are true.
In particular, no matter how long you wait, you cannot see the whole space; nor can you influence the whole space.

Suppose we have an observer sitting at the north pole, as illustrated in Fig.~\ref{fig:l17f6}.
\begin{figure}[tbhp]
  \centering
  \begin{minipage}[t]{0.33\columnwidth}
    \centering
    \def\svgwidth{0.9\columnwidth}
    \input{lectures/l17f6.pdf_tex}
    \caption{}
    \label{fig:l17f6}
  \end{minipage}%
  \begin{minipage}[t]{0.33\columnwidth}
    \centering
    \def\svgwidth{0.9\columnwidth}
    \input{lectures/l17f7.pdf_tex}
    \caption{}
    \label{fig:l17f7}
  \end{minipage}%
  \begin{minipage}[t]{0.33\columnwidth}
    \centering
    \def\svgwidth{0.9\columnwidth}
    \input{lectures/l17f8.pdf_tex}
    \caption{}
    \label{fig:l17f8}
  \end{minipage}%
\end{figure}
There will be regions of \texttt{dS}, which she will never see; the boundary between what she can and cannot see is called the \emph{event horizon}. Unlike in the case of black holes, every observer has a different event horizon.
Similarly, if she sits on the south pole, we have the case of Figure \ref{fig:l17f7}.
This in cosmology is called the \emph{particle horizon}.
\begin{exercise}
  Check that the static patch coordinates map to the intersection of the event and particle horizons, as depicted in Fig.~\ref{fig:l17f8}.
\end{exercise}
In some sense, the static patch are the natural coordinates, since they are the region, which an observer can see and influence.
However, they come with a coordinate singularity at $r  = R$.
