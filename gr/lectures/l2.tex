% lecture notes by Umut Özer
% course: gr
\lhead{Lecture 2: October 14}

\section{Tangent Spaces}%
\label{sec:tangent_spaces}

So far, we have the idea of a manifold. It is a space that looks locally like $\mathbb{R}^n$, with a set of charts that allow us to map patches to $\mathbb{R}^n$ and differentiate (and later, integrate).

Let $C^{\infty}(\mathcal{M})$ denote the set of all smooth functions that assign each point in the manifold $\mathcal{M}$ a real number in $\mathbb{R}$. We know how to differentiate on $\mathbb{R}^n$. Smooth functions allow us to differentiate in $\mathbb{R}^n$ before we map back onto the manifold.

\begin{definition}[tangent vector]
  A \emph{tangent vector} $X_p$ is an object that differentiates functions at some point $p \in \mathcal{M}$ on the manifold.
  Specifically, it is a map $X_p: C^{\infty} (\mathcal{M}) \to \mathbb{R}$ with certain properties:
  \begin{enumerate}
    \item linearity: $X_p(f + g) = X_p(f) + X_p(g)$ for all $f, g \in C^{\infty} (\mathcal{M})$
    \item constants vanish: $X_p(f) = 0$ when $f$ is constant
    \item Leibniz rule: $X_p(fg) = f(p) X_p(g) + X_p(f) g(p)$ for all $f, g \in C^\infty(\mathcal{M})$
  \end{enumerate}
\end{definition}

\begin{definition}[tangent space]
  The set of all tangent vectors at $p$ is the \emph{tangent space} $T_p(\mathcal{M})$ at $p$.
\end{definition}

One of the early surprises of differential geometry is thinking of vectors as differential operators.
In $\mathbb{R}^3$, we used position vectors as displacement. However, this does not generalise to curved spaces.
The second idea of a vector in physics is the velocity of a particle. This change in time is analogous to the definition that we use here in differential geometry.
By differentiating, the tangent vector tells us how things change when moving in a particular direction.

\begin{description}
  \item[Claim:] In a chart $\phi= (x^1, \cdots, x^n)$, we can write a tangent vector as
    \begin{equation}
      X_p = X^\mu \partial_\mu \rvert_p,
    \end{equation}
    with $X^\mu = x_p(x^\mu)$ and $\partial_\mu f = \pdv{(f \circ \phi^{-1})}{x^\mu}$ for all functions $f \in C^\infty(\mathcal{M})$. This definition of $\partial_\mu f$ uses the fact that we can differentiate on $\mathbb{R}^n$, which we map to via $f \circ \phi^{-1}$.
\end{description}
\begin{proof}
  The idea is that given a function $f$, we use coordinates to push it to $\mathbb{R}^n$, and then we differentiate there.
  A detailed proof is in the notes.
\end{proof}

\begin{leftbar}
  \begin{remark}
    We are going to use the summation convention with indices up/down.
    The object with coefficients as superscripts are tangent vectors, different from objects with subscript coefficients. Hence, the position of the indices matter even more than in Special Relativity.
  \end{remark}
\end{leftbar}

Writing $X_p = X^\mu \partial_\mu$ clearly depends on coordinates $x^\mu$. What would happen if we chose different coordinates?
If we picked another set of coordinates $\widetilde x^\mu$, we could write
\begin{equation}
  X_p = X^\mu \left.\pdv{}{x^\mu}\right\rvert_p = \widetilde X^\mu \left.\pdv{}{\widetilde x^\mu}\right\rvert_p.
\end{equation}

\begin{notation}[]
  We write $\partial_\mu \coloneqq \pdv{}{x^\mu}$ and drop the implicit action of going to $\mathbb{R}^n$ via $\phi^{-1}$.
\end{notation}

Acting on a function $f$, we can use the chain rule
\begin{equation}
  X_p(f) = X^\mu \left.\pdv{f}{x^\mu} \right\rvert_p = X^\mu \left.\pdv{\widetilde x^\nu}{x^\mu} \right\rvert_{\phi(p)} \left.\pdv{f}{\widetilde x^\nu} \right\rvert_p.
\end{equation}
We can view this as a change of basis of $T_p(\mathcal{M})$
\begin{equation}
\left. \pdv{}{x^\mu} \right\rvert_p = \left.\pdv{\widetilde x^\nu}{x^\mu}\right\rvert_{\phi(p)} \left.\pdv{}{\widetilde x^\nu} \right\rvert_p.
\end{equation}
This is a \emph{covariant} transformation.
Alternatively, we can think of this as a change of components
\begin{equation}
  \widetilde X^\nu = X^\mu \left.\pdv{\widetilde x^\nu}{x^\mu} \right\rvert_{\phi(p)}.
\end{equation}
This is called a \emph{contravariant} transformation; the position of the index tells us which way things transform.

To see why $X_p$ is called a tangent vector, we are going to consider a path $\sigma(t): \mathbb{R} \to \mathcal{M}$ in the manifold such that $\sigma(0) = p$. We can think of the parameter $t$ as time for example.
%F1
Given a coordinate chart $\phi$, this becomes a path $x^\mu(t)$ in $\mathbb{R}^n$. The tangent in $\mathbb{R}^n$ is $X^\mu = \left.\dv{x^\mu(t)}{t}\right\rvert_{t=0}$. We can then use this to define $X_p \in T_p(\mathcal{M})$ as
\begin{equation}
  X_p = \left.\dv{x^\mu(t)}{t}\right\rvert_{t = 0 } \left.\pdv{}{x^\mu}\right\rvert_p.
\end{equation}
In this sense, $T_p(\mathcal{M})$ is the space of all tangent vectors at $p$.
By considering all possible paths, one can describe all possible tangent vectors. Physically, we can think about $T_p(\mathcal{M})$ as the space of all possible velocity vectors $X_p$ of a particle traversing a point $p \in \mathcal{M}$, as illustrated in Figure \ref{fig:l2f2}.
\begin{figure}[tbhp]
  \centering
  \def\svgwidth{0.4\columnwidth}
  \input{lectures/l2f2.pdf_tex}
  \caption{The tangent vector $X_p \in T_p(\mathcal{M})$ to a path $\sigma(t)$.}
  \label{fig:l2f2}
\end{figure}
% F2
\begin{leftbar}
  \begin{remark}
    Tangent spaces at two different points are two different vector spaces. It is meaningless to try to add vectors of two different tangent spaces \ldots for now.
  \end{remark}
\end{leftbar}

\section{Vector Fields}%
\label{sec:vector_fields}

\begin{definition}[vector field]
  A \emph{vector field} $X\colon C^{\infty}(\mathcal{M}) \to C^{\infty}(\mathcal{M})$ is a smooth assignment of tangent vectors $X_p$ at each point $p \in \mathcal{M}$ with $(Xf)(p) = X_p (f)$. In a given coordinate chart we can write
  \begin{equation}
    X = X^\mu (x) \partial_\mu.
  \end{equation} 
\end{definition}
\begin{description}
  \item[Notation:] We will denote the space of all vector fields as $\mathfrak{X} (\mathcal{M})$.
\end{description}

\begin{leftbar}
  \begin{remark}
    We will make the notion of \emph{smooth} more precise later. Intuitively, we want neighbouring tangent spaces to have similar directions.
  \end{remark}
\end{leftbar}

Given two vector fields $X, Y \in \mathfrak{X}(\mathcal{M})$, we can define the \emph{commutator}.
\begin{definition}[commutator]
  The \emph{commutator} $[X, Y] \in \mathfrak{X}(\mathcal{M})$ is a map
  \begin{equation}
    [X, Y] f = X(Y f) - Y(Xf).
  \end{equation}
\end{definition}

\begin{leftbar}
  \begin{remark}
    Both $X, Y$ are first order differential operator. Thus, each term on the RHS is a second order operator. However, by subtracting these terms from each other, the second order terms cancel by commutation of partial derivatives.
    Thus, the commutator is a first order differential operator.
  \end{remark}
\end{leftbar}

\begin{exercise}[]
In a coordinate basis, you can check
\begin{equation}
  [X, Y] = \left( X^\mu \pdv{Y^\nu}{x^\mu} - Y^\mu \pdv{X^\nu}{x^\mu} \right) \pdv{}{x^\nu}.
\end{equation}
\end{exercise}

\begin{leftbar}
  \begin{remark}
    As the course progresses, we will see connections with the \emph{Symmetries, Field and Particles} course and its treatment of Lie algebras though these commutators.
  \end{remark}
\end{leftbar}

\begin{exercise}
  Check that the \emph{Jacobi identity} holds:
  \begin{equation}
    [X, [Y, Z] ] + [Y, [Z, X] ] +  [Z, [X, Y] ] = 0
  \end{equation}
\end{exercise}

\subsection{Integral Curves}%
\label{sub:integral_curves}

There is a relationship between vector fields and streamlines, also known as flows, on a manifold $\mathcal{M}$.

\begin{definition}[flow]
  A \emph{flow} on $\mathcal{M}$ is a one-parameter family of diffeomorphisms $\sigma_t: \mathcal{M} \to \mathcal{M}$, labelled by $t \in \mathbb{R}$, such that $\sigma_{t = 0} = id$ and $\sigma_s \circ \sigma_t = \sigma_{s + t}.$ 
\end{definition}
This defines \emph{flow lines} on $\mathcal{M}$ as depicted in Figure \ref{fig:l2f3}.
\begin{figure}[tbhp]
  \centering
  \begin{minipage}[t]{0.5\textwidth}
    \centering
    \def\svgwidth{0.9\columnwidth}
    \input{lectures/l2f3.pdf_tex}
    \caption{Flow lines}
    \label{fig:l2f3}
  \end{minipage}%
  \begin{minipage}[t]{0.5\textwidth}
    \centering
    \def\svgwidth{0.6\columnwidth}
    \input{lectures/l2f4.pdf_tex}
    \caption{Integral curves of $X = \frac{\partial }{\partial \phi}$  on $S^2$.}
    \label{fig:l2f4}
  \end{minipage}
\end{figure}
For each point on the manifold, there is a unique such streamline that flows through it.
Each line has coordinates $x^\mu(t)$.

We can obtain a vector field $X$ by taking the tangent vector at each point:
 \begin{equation}
   \label{def:vecfield}
   X^\mu(x(t)) = \dv{x^\mu(t)}{t}.
\end{equation}
\begin{definition}[integral curves]
  Conversely, given a vector field $X$, we can integrate \eqref{def:vecfield} to find $x^\mu(t)$.
  These flow lines are called \emph{integral curves}.
\end{definition}

\begin{example}[]
  On $S^2$, with $(\theta, \phi)$ polar coordinates, consider $X = \pdv{}{\phi}$.
  From \eqref{def:vecfield}, $\dv{\theta}{t} = 0$ and $\dv{\phi}{t} = 1$. This means that we have constant $\theta = \theta$.
  \begin{equation}
    \phi = \phi_0 + t \implies \sigma_t (\theta, \phi) \rightarrow (\theta, \phi + t)
  \end{equation}
  As shown in Fig.~\ref{fig:l2f4}, the integral curves are lines of constant lattitude.
\end{example}
