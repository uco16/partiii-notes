% lecture notes by Umut Özer
% course: gr
\lhead{Lecture 12: November 08}
We can easily construct new tensors from the Riemann tensor. The Riemann tensor has four indices; to build a new one, we just raise index and contract it with one of the lower ones.
However, due to the symmetry properties, we have to contract one of the first pair indices to one index from the second pair.
\begin{definition}[]
  The \emph{Ricci tensor} $R_{\mu\nu} = R \indices{^{\rho}_{\mu\rho\nu}}$. This obeys $R_{\mu\nu} = R_{\nu\mu}$.
\end{definition}
\begin{definition}[]
  The \emph{Ricci scalar} $R= g^{\mu\nu} R_{\mu\nu}$.
\end{definition}
We are loosing information every time we contract indices like this. However, in some sense the information that is left over is the most essential part of the information. 
We can also apply the Bianchi identity to get
\begin{equation}
  \nabla^{\mu} R_{\mu\nu} = \frac{1}{2} \nabla_{\nu} R.
\end{equation}

\begin{definition}[]
  The \emph{Einstein tensor} $G_{\mu\nu} = R_{\mu\nu} - \frac{1}{2} Rg_{\mu\nu}$.
\end{definition}
The whole point of building this tensor is that, using the Riemann tensor's Bianchi identity, these two tensors cancel to give a Bianchi identity for the Einstein tensor: 
\begin{equation}
  \nabla^{\mu} G_{\mu\nu} = 0.
\end{equation}
Proving the Bianchi identity for the Riemann tensor is a pain, but in Section \ref{sec:diffeomorphisms_revisited} we will show that the Bianchi identity of the Einstein tensor follows from the diffeomorphism invariance of the Einstein-Hilbert action.

\section{Connection 1-forms}%
\label{sec:connection_1_forms}

These objects are, as we will see, closely related to two-forms.
Connection 1-forms are a technology designed to make it easier to calculate the Riemann tensor for a specific metric. 
It is not a slick trick like the one that we saw to find the Levi-Civita connection components.

Given a coordinate basis $\left\{ e_{\mu} \right\} = \left\{ \partial_{\mu} \right\}$.
The idea is that we will take a different basis of tangent vectors so that things start to look simpler.
Given the coordinate basis, we can always introduce a different basis $\left\{ \hat e_a = e \indices{_a^{\mu}} \partial_{\mu} \right\}$ .
There is a particularly clever basis to choose:
On a Riemannian / Lorentzian manifold, we pick a basis such that when we evaluate the metric on this new metric, we get
\begin{equation}
  g(\hat e_a, \hat e_b) = g_{\mu\nu} e\indices{_a^{\mu}} e \indices{_b^{\nu}} =
  \begin{cases}
    \delta_{ab} & \text{Riemannian} \\
    \eta_{ab} & \text{Lorentzian}
  \end{cases}
\end{equation}
Note that this is not the metric in a set of coordinates (such as in Riemann normal coordinates). This is rather a diagonalised version of the metric.
The components $e \indices{_a^{\mu}}$ are called \emph{vielbeins}.
In general, having an $n$-component basis is called an $n$-bein (where we use the German word for $n$, e.g. $n=2$: zweibeins).
We raise and lower greek indices with $g_{\mu\nu}$ and latin indices using $\delta_{ab}$.
The basis of one-forms $\left\{ \hat\theta^a \right\}$ obey 
\begin{equation}
  \hat \theta^a (\hat e_b) = \delta^a_b.
\end{equation}
The right hand side is a Krönicker delta for both Lorentzian and Riemannian metrics.
They are $\hat \theta^a = e \indices{^a_{\mu}} dx^{\mu}$ with $e \indices{^a_{\mu}} e \indices{_b^{\mu}} = \delta^a_b$ and $e \indices{^a_{\mu}} e \indices{_a^{\nu}} = \delta^{\nu}_{\mu}$. The metric is
\begin{equation}
  g = g_{\mu\nu} dx^{\mu} \otimes dx^{\nu} = \delta_{ab} \hat \theta^a \otimes \hat\theta^b \implies g_{\mu\nu} = e \indices{^a_{\mu}} e \indices{^b_{\nu}} \delta_{ab}.
\end{equation}

\begin{example}[]
  Consider the metric
  \begin{equation}
    ds^2 = -f(r)^2 dt^2 + f(r)^{-2} dr^2 + r^2 (d \theta^2 + \sin^2(\theta) d\phi^2)
  \end{equation}
  This is one of the most important class of metric in General relativity, since for a particular choice of $f$ we obtain the \emph{Schwarzschild solution}.
  In our new notation, this is $ds^2 = \eta_{ab} \hat\theta^a \otimes \hat \theta^b$ with non-coordinate 1-forms
  \begin{equation}
    \hat\theta^0 = f dt \quad \hat\theta^1 = f^{-1} dr \quad \hat\theta^2 = r d\theta \quad \hat\theta^3 = r \sin \theta d\phi.
  \end{equation}
  \begin{leftbar}
    \begin{remark}
      Note that this is a slightly annoying convention; $\hat\theta$ has nothing to do with $\theta$ or $d\theta$.
    \end{remark}
  \end{leftbar}
  In the basis $\left\{ \hat e_a \right\}$, the components of the connection are by definition
  \begin{equation}
    \nabla_{\hat e_c} \hat e_b \coloneqq \Gamma^a_{cb} \hat e_a.
  \end{equation}
  \begin{leftbar}
    \begin{remark}
      Another slightly annoying convention: the type of indices (Roman or Greek) determine which object $\Gamma$ we are looking at.
      $\Gamma \indices{^{\mu}_{\rho\nu}} \neq \Gamma \indices{^b_{cb}}$.
    \end{remark}
  \end{leftbar}
  \begin{definition}[]
    Given this, we define the \emph{connection 1-form} or \emph{spin connection} $\omega \indices{^a_b} = \Gamma^{a}_{cb} \hat\theta^c$.
  \end{definition}
  \begin{leftbar}
    \begin{remark}
      The reason that this is called the spin-connection is that we have to use this object when we want to couple to Dirac spinors.
    \end{remark}
  \end{leftbar}
\end{example}
\begin{claim}[First Cartan Structure Equation]
  Defining $\omega_{ab} = \delta_{ac} \omega \indices{^c_b}$, we have
  \begin{equation}
    d\hat\theta^a + \omega \indices{^a_b} \wedge \hat\theta^b = 0
  \end{equation}
\end{claim}
\begin{proof}
  No.
\end{proof}
\begin{claim}
  For the Levi-Civita connection, the lowered spin connection is anti-symmetric:
  \begin{equation}
    \omega_{ab} = -\omega_{ba}.
  \end{equation}
\end{claim}
\begin{definition}[]
  In the vielbein basis, $R \indices{^a_{bcd}} = R(\hat\theta^a; \hat e_c, \hat e_d, \hat e_b)$ (again, the indices tell us what kind of object this is), with $R \indices{^a_{bcd}} = - R \indices{^a_{bdc}}$.
  We define the \emph{curvature 2-form}
  \begin{equation}
    \mathcal{R} \indices{^a_b} = \frac{1}{2} R \indices{^a_{bcd}} \hat \theta^c \wedge \hat \theta^d.
  \end{equation}
\end{definition}
\begin{leftbar}
  \begin{remark}
    This has the full information of the Riemann tensor. Note that when we write this in components $\hat \theta = e \indices{^a_{\mu}} dx^{\mu}$, we find that the components of the curvature 2-form are $(\mathcal{R} \indices{^a_b})_{\mu\nu}$.
  \end{remark}
\end{leftbar}
\begin{claim}[Second Cartan Structure Equation]
  \begin{equation}
    \mathcal{R} \indices{^a_b} = d \omega \indices{^a_b} + \omega \indices{^a_c} \wedge \omega \indices{^c_b}
  \end{equation}
\end{claim}
\begin{proof}
  No.
\end{proof}

If we are given a metric, the quickest way to compute the Riemann tensor is to calculate the curvature one-forms $\hat \theta$. Then we can easily take the exterior derivative, which allows us to use the first Cartan structure equation.
Plugging this into the second structure equation, we can find the components of the Riemann tensor.

\subsection*{Back to the Example}%

Compute $d\hat \theta^a$:
\begin{align}
  \hat \theta^0 &= f' dr \wedge dt \quad & d\hat\theta^1 &= f' dr \wedge dr = 0 \\
  d \hat\theta^2 &= dr \wedge d\theta \quad &d \hat\theta^3 &= \sin \theta dr \wedge d\phi - r \cos \theta d \theta \wedge d\phi.
\end{align}
Use $d \hat\theta^0 = - \omega \indices{^0_b} \wedge \hat\theta^b$ to convince yourself that $\omega \indices{^0_1} = f' f dt = f' \hat\theta^0$ and $\omega \indices{^0_1} = -\omega_{01} = + \omega_{10} = \omega \indices{^1_0}$, where we used the Minkowski metric in the first equality.
\begin{description}
  \item[Check] $d\hat\theta^1 = - \omega \indices{^1_b} \wedge \hat\theta^b = -\omega \indices{^1_0} \wedge \hat\theta^0 + \dots = -f' \hat\theta^0 \wedge \hat\theta^0 + \dots$, where the $\hat\theta^0 \wedge \hat\theta^0$ vanishes. \checkmark
\end{description}
Proceed in the same way to find that the only non-vanishing ones are
\begin{align}
  \omega \indices{^0_1} &= \omega \indices{^1_0} = f' \hat\theta^0 \\
  \omega \indices{^2_1}&= - \omega \indices{^1_2} = \frac{f}{r} \hat\theta^2 \\
  \omega \indices{^3_1} &= - \omega \indices{^1_3} = \frac{f}{r} \hat\theta^3 \\
  \omega \indices{^3_2} &= - \omega \indices{^2_3} = \frac{\cot \theta}{r} \hat\theta^3
\end{align}
Now the curvature tensor is
\begin{align}
  \mathcal{R} \indices{^0_1} &= d \omega \indices{^0_1} + \overbrace{\omega \indices{^0_c} \wedge \omega \indices{^c_1}}^{\mathclap{\text{In this case, these vanish}}} \\
			     &= f' d\hat\theta^0 + f'' dr \wedge \hat\theta_0 \\
			     &= ((f')^2 + f'' f) dr \wedge dt \\
			     &= -((f')^2 + f'' f) \hat\theta^0 \wedge \hat\theta^1 \\
  \implies R_{0101} &= ff'' + (f')^2
\end{align}
We convert back to Greek indices using the vielbeins:
\begin{equation}
  R_{\mu\nu\rho\sigma}= e \indices{^a_\mu} e \indices{^b_{\nu}} e \indices{^c_\rho} e \indices{^d_\sigma} R_{abcd}
\end{equation}
In our case, we find $R_{trtr} = ff'' + (f')^2$.
