% lecture notes by Umut Özer
% course: susy
\lhead{Lecture 9: February 13}

\chapter{Nonlinear Sigma Models}%
\label{cha:nonlinear_sigma_models}

This is a theory of maps $x \colon [0, \beta] \to (M, g)$  to a Riemannian manifold.
\begin{figure}[tbhp]
  \centering
  \def\svgwidth{0.4\columnwidth}
  \input{lectures/l9f1.pdf_tex}
  \caption{A map $x\colon [0, \beta] \to (M, g)$.}
  \label{fig:l9f1}
\end{figure}
As illustrated in Fig.~\ref{fig:l9f1} the target space, in which we worldline lives, may now be curved.

The natural action in the bosonic case is
\begin{equation}
  S[x] = \frac{1}{2} g (\dot{x}, \dot{x}) \dd[]{\tau} = \frac{1}{2} \int g_{ab} \dot{x}^{a} \dot{x}^{b} \dd[]{\tau}
\end{equation}
as we know from general relativity.
Classically, trajectories are geodesics in $M$.
\begin{remark}
  This is an interacting theory on $[0, \beta]$. If we consider perturbing around the constant map $x([0, \beta]) = x_0 \in M$, use Riemann normal coordinates to write $g_{ab}(x_0 + \delta x) = \delta_{ab} - \frac{1}{3} R_{abcd} (x_0) \delta x^{c} \delta x^{d} + O(\delta x^2)$ and the $\delta x$ terms gives us interactions on the worldline.
\end{remark}
We can try to quantise as usual by finding the canonical momenta
\begin{equation}
  p_a = \frac{\delta L}{\delta \dot{x}^a} = g_{ab}(x) \dot{x}^b
\end{equation}
and then imposing canonical commutation relations
\begin{equation}
  [\hat{p}_a, \hat{x}^b] = i \delta^{b}_a.
\end{equation}
If we represent the Hilbert space as $\mathscr{H} = L^2 (M, \sqrt{g} \dd[n]{x})$, then the momentum operator is a derivative operator $\hat{p}_a = -i \frac{\partial }{\partial x^a}$.
However, the Hamiltonian is ambiguous:
classically, $H = \dot{x}^a p_a - L = \frac{1}{2} g^{ab}(x) p_a p_b$, but we have to decide how to order the $p$'s vs the $x$'s in the curved metric $g^{ab}(x)$.
Once we turn $p$ into a derivative operator all these choices of positioning matter.
There are a number of things we could require.
\begin{itemize}
  \item $\hat{H}$ is at most second order in derivatives acting on $\mathscr{H}$.
  \item We could ask that $\hat{H}$ is compatible with $\text{Diff}(M)$, so that $\hat{H}$ is some sort of covariant operator.
  \item Reduces to the usual Laplacian $-\frac{1}{2} \partial^{a} \partial_{a}$ in flat space when $g = \delta$.
\end{itemize}
However, these conditions are obeyed by the covariant Laplacian \emph{plus} any multiple of the scalar curvature: 
\begin{equation}
  \hat{H} \Psi = - \frac{1}{2} \nabla^a \nabla_a \Psi + \alpha R \Psi = -\frac{1}{\sqrt{g}} \partial_{a} (\sqrt{g} g^{ab} \partial_b \Psi) + \alpha R \Psi,
\end{equation}
for any $\alpha \in \mathbb{R}$.
Exactly which $\alpha$ we find depends on our normal ordering / path integral (from different regularisation procedures).
This ambiguity is annoying and we would like to do better. We \emph{can} do better in the $\mathcal{N} = 1$ supersymmetric version.

We have the action
\begin{align}
  S[x, \psi] &= \frac{1}{2} \int g(\dot{x}, \dot{x})  + i g (\psi, \nabla_\tau \psi) \dd[]{\tau} \\
	     &= \frac{1}{2} \int g_{ab}(x) \dot{x}^a \dot{x}^{b} + i g_{ab} (x) \psi^{a} \left( \dot{\psi}^{b} + \dot{x}^{c} \Gamma^{b}_{cd} \psi^{d} \right) \dd[]{\tau}, \label{eq:8-1}
\end{align}
where $\Gamma = \Gamma(x(\tau))$.
This action is invariant under the same supersymmetric transformations \eqref{eq:n1susy} as it was in flat space $(M, g) = (\mathbb{R}^n, \sigma)$, 
\begin{equation}
  \delta x^{a} =  \epsilon \psi^{a} \qquad \delta \psi^{a} = i \epsilon \dot{x}^{a}.
\end{equation}
In fact, not only is the action supersymmetric, but we also have
\begin{equation}
  \mathcal{Q} \left( -\frac{i}{2} \int g_{ab} \psi^{a} \dot{x}^{b} \dd[]{\tau} \right)  =\frac{1}{2} \int \left( g_{ab} \dot{x}^{a} \dot{x}^{b} + i g_{ab} \psi^{a} \dot{\psi}^{b} - i \partial_{c} g_{ab} \psi^{c} \psi^{a} \dot{x} ^{b} \right) \dd[]{\tau}.
\end{equation}
Comparing this to \eqref{eq:8-1}, we recognise the first two terms, but it is not obvious that the final term is the Christoffel symbol.
In fact it is:
we have
\begin{equation}
  \partial_{c} g_{ab} \psi^{c} \psi^{a} = -\frac{1}{2} \left( \partial_{a} g_{bc} + \partial_{b} g_{ac} - \partial_{c} g_{ba} \right) \psi^{c} \psi^{a} = -g_{cd} \Gamma^{d}_{ab} \psi^{c} \psi^{a}.
\end{equation}
So the action itself is $S = \mathcal{Q} (\dots)$, we say it is  \emph{$\mathcal{Q}$-exact}.
This will be crucial for localisation of the path integral.

It will be useful to do the canonical quantisation first. We are expecting to get some sort of spinors in curved space, like we did in flat space.
We expect the Dirac operators that we got in the flat case to turn into some sort of covariant Dirac operators.

The momenta $(p_a, \pi_a)$  are
\begin{equation}
  p_a = \frac{\delta L}{\delta \dot{x}^a} = g_{ab} \dot{x}^b + i g_{bc} \psi^{b} \Gamma^{c}_{ad} \psi^{d}, \qquad
  \pi_a = \frac{\delta L}{\delta \psi^a} = i g_{ab} \psi^{b}.
\end{equation}
Therefore, upon quantisation we find our usual commutation relations for the $x$  and $p$:  $[\hat{x}^a, \hat{p}_b] = i \delta^{c}_b$ , and similarly for the $\pi$'s and $\psi$'s, but the anticommutation relation for the fermions is now
\begin{equation}
  \{\hat{\psi}^a, \hat{\psi}^b\} = 2 g^{ab}(x).
\end{equation}
The fact that the $\hat{\psi}$  anticommutators involve the field $x$ is a little awkward. How can it be that the commutators of the different fields know about each other?

To do better, we introduce at each point  $x \in M$  an orthonormal set of basis vectors $\{e_i = e_i^a(x) \frac{\partial }{\partial x^{a}}\}$ on the tangent space $T_x M$, often called a \emph{frame}.
Orthonormality here means that 
\begin{equation}
  g(e_i, e_j) = g_{ab} e\indices{^{a}_{i}} e\indices{^{b}_{j}} = \delta_{ij}.
\end{equation}
We have a dual basis of 1-forms, often called \emph{vielbeins}, unfortunately conventionally denoted by the same letter $\{e^i = e\indices{^{i}_{a}} dx^{a}\}$, which are dual in the sense $e_{i} (e^{j}) = e\indices{_{i}^{a}} e\indices{_{a}^{j}} = \delta^{i}_{j}$.

Since the $e_i$ are a basis, we have completeness relations
\begin{align}
  g^{-1} &= \delta^{ij} e_{i} \otimes e_{j} & g &= \delta_{ij} e^{i} \otimes e^{j} \\
  \text{i.e.} \quad g^{ab} &= e\indices{^{a}_{i}} e\indices{^{b}_{j}} \delta^{ij} & \text{i.e.} \quad g_{\alpha\beta} &= e\indices{^{i}_{a}} e\indices{^{j}_{b}} \delta_{ij}.
\end{align}
\begin{remark}
  You might interpret this by saying that the $e_j$ are the `square root' of the metric.
\end{remark}

Expanding our fermions in this basis, we have $\psi^{a} = \psi^{i} e\indices{^{a}_{i}}$ or $\psi^{i} = e\indices{^{i}_{a}} \psi^{a}$, where the $\hat{\psi}^{i}$'s obey 
\begin{equation}
  \{\hat{\psi}^{i}, \hat{\psi}^{j}\} = \{e\indices{^{i}_{a}} \hat{\psi}^{a} , e\indices{^{j}_{b}} \hat{\psi}^{b}\} = e\indices{^{i}_{a}} e\indices{^{j}_{b}} \{\hat{\psi}^{a}, \hat{\psi}^{b}\} = 2 e\indices{^{i}_{a}} e\indices{^{j}_{b}} g^{ab} = 2 \delta^{ij}.
\end{equation}
In the vielbein basis, the $\hat{\psi}$'s obey exactly the same anticommutation relations as in flat space.

So far, we have an orthonormal frame $\{e_i (x)\}$ at each point $x \in M$, and in general this choice may vary over $M$.
To compare, we introduce a connection $\nabla$ on the tangent bundle $T M$ by 
\begin{equation}
  \label{eq:8-w}
  \nabla (e_{i}) = d e_i + \omega\indices{_{i}^{j}} e_{j},
\end{equation}
where the connection 1-form $\omega\indices{^{i}_{j}} = dx^{a} \omega_{a}(x)\indices{^{i}_{j}}$ is the \emph{spin connection}.
\begin{remark}
  In \eqref{eq:8-w} we supressed the spacetime index $a$.
\end{remark}
Each of the spin connection components is an antisymmetric matrix $\omega^{ij} = \delta^{jk} \omega\indices{^{i}_{k}} = - \omega^{ji}$, since it preserves orthonormality.
A priori, $\omega$ has nothing to do with $\Gamma$, but we usually (basically always) impose compatibility by the torsion-free condition
\begin{gather}
  \nabla_a e\indices{^{b}_{i}} = \partial_{a} e\indices{^{b}_{i}} + \Gamma\indices{^{b}_{ac}} e\indices{^{c}_{i}} + \omega_{a}{}\indices{^{j}_{i}} e\indices{^{b}_{j}} = 0 \\
  \text{i.e.} \quad (\omega_a)\indices{^{i}_{j}} = e\indices{^{i}_{b}} \Gamma\indices{^{b}_{ac}} e\indices{^{c}_{j}} + e\indices{^{i}_{b}} \partial_{a} e\indices{^{b}_{j}}.
\end{gather}
We get exactly the same expression from our fermion action:
\begin{align}
  g (\psi, \nabla_\tau \psi) = g_{ab} \psi^{a} \nabla_\tau \psi^{b} &= g_{ab} e\indices{^{a}_{i}} \psi^{i} \left( \frac{\partial }{\partial \tau} (e\indices{^{b}_{j}} \psi^{j}) + \dot{x}^{c} \Gamma\indices{^{b}_{cd}} e\indices{^{d}_{j}} \psi^{j} \right) \\
				       &= \delta_{ij} \psi^{i} \partial_{\tau} \psi^{j} + e_{bi} \dot{x}^{a} \partial_{a} e\indices{^{b}_{j}} \psi^{i} \psi^{j} + e_{ib} \dot{x}^{a} \Gamma\indices{^{b}_{ac}} e\indices{^{c}_{j}} \psi^{i} \psi^{j} \\
				       &= \delta_{ij} \psi^{i} \left( \partial_{\tau} \psi^{j} + \dot{x}^{a} (\omega_a)\indices{^{j}_{k}} \psi^{k} \right) \\
				       &= \delta(\psi, \nabla_\tau \psi)
\end{align}
where $e_{bi} = g_{ab} e\indices{^{a}_{i}}$.
