% lecture notes by Umut Özer
% course: susy
\lhead{Lecture 3: January 23}

If our action contains finitely many fermions, it is always easy to compute the fermionic integral exactly (unlike the bosonic case).
\begin{example}[]
  If we have a quartic action 
  \begin{equation}
    S(\psi^1, \dots, \psi^4) = A(\psi^1 \psi^2 + \psi^3 \psi^4) + \lambda \psi^1 \psi^2 \psi^3 \psi^4.
  \end{equation}
  Then $S^2 \neq 0$ , but $S^3 = 0$ . So the exponential measure truncates just after the second term
  \begin{align}
    e^{-S / \hbar} &= 1 - \frac{S}{\hbar} + \frac{S^2}{2\hbar^2} \\
		   &=1 - \frac{1}{\hbar} [A (\psi^1 \psi^2 + \psi^3 \psi^4) + \lambda \psi^1 \psi^2 \psi^3 \psi^4]
  \end{align}
  and hence, the integral extracts the piece
  \begin{equation}
    \int e^{-S (\psi) / \hbar} \dd[4]{\psi} = \frac{A^2}{\hbar^2} - \frac{\lambda}{\hbar}.
  \end{equation}
\end{example}

\subsection{Supersymmetric Theory}%
\label{sub:supersymmetric_theory}

A generic theory containing both fermions and bosons is intractable because of the bosonic integral. Even in $d = 0$, we get a complicated integral that is hard to solve.
However, let us consider a theory containing one bosonic and two fermionic fields $(x, \psi, \overline{\psi})$.
\begin{leftbar}
  These fermionic fields can be considered $\psi = \psi^1 + i \psi^2$ and $\overline{\psi} = \psi^1 - i \psi^2$.
\end{leftbar}
The most general action for these fields would be
\begin{equation}
  S(x, \psi \overline{\psi}) = f(x) + g(x) \overline{\psi} \psi,
\end{equation}
for some functions $f$ and $g$.
We can choose a very special relation between fermionic and bosonic fields by choosing the action
\begin{equation}
  S(x, \psi \overline{\psi}) = \frac{1}{2}(\partial w)^2 - \overline{\psi} \psi \partial^2 w,
\end{equation}
where $w = w(x)$ is a polynomial and $\partial w = \frac{\partial w}{\partial x}$.
\begin{claim}
  This action is invariant under the flow generated by the fermionic vector fields
  \begin{equation}
    Q = \psi \frac{\partial }{\partial x} + (\partial w) \frac{\partial }{\partial \overline{\psi}} \qquad \text{and} \qquad Q^{\dagger} = \overline{\psi} \frac{\partial }{\partial x} - (\partial w) \frac{\partial }{\partial \psi}.
  \end{equation}
  These are odd derivations of $\mathbb{R}^{1 \suchthat 2}$ with 
  \begin{align}
    \mathcal{Q}(x) &= \psi & \mathcal{Q}^{\dagger}(x) &= \overline{\psi} \\
    \mathcal{Q}(\psi) &= 0 &  \mathcal{Q}^{\dagger}(\psi) &= -\partial w (x) \\
    \mathcal{Q}(\overline{\psi}) &= \partial w (x) & \mathcal{Q}^{\dagger}(\overline{\psi}) &= 0
  \end{align}
\end{claim}
\begin{proof}
  We will only show this for $\mathcal{Q}^{\dagger}$.
  \begin{align}
    \mathcal{Q}^{\dagger}(S) &= \overline{\psi} \frac{\partial }{\partial x} \left( \frac{1}{2} (\partial w)^2 - \overline{\psi} \psi \partial^2 w \right) - (\partial w) \frac{\partial }{\partial \psi} \left( \frac{1}{2} (\partial w)^2 - \overline{\psi} \psi \partial^2 w \right) \\
			     &= \overline{\psi} \partial w \partial^2 w - \overline{\psi} (\partial w) \partial^2 w = 0.
  \end{align}
\end{proof}
We say that $\mathcal{Q}$ and $\mathcal{Q}^{\dagger}$ generate \emph{supersymmetries} of this action.

We can also calculate the anti-commutation relations
\begin{equation}
  \begin{gathered}
    \{\mathcal{Q} , \mathcal{Q}\} = 2 (\partial^2 w) \psi \frac{\partial }{\partial \overline{\psi}} \qquad 
    \{\mathcal{Q}^{\dagger}, \mathcal{Q}^{\dagger}\} = - 2 (\partial^2 w) \overline{\psi} \frac{\partial }{\partial \psi} \\
  \{\mathcal{Q}, \mathcal{Q}^{\dagger}\} = - \partial w \left( \psi \frac{\partial }{\partial \psi} - \overline{\psi} \frac{\partial }{\partial \overline{\psi}} \right),
  \end{gathered}
\end{equation}
which also generate bosonic symmetries.

This supersymmetry obeys $\mathcal{Q}^2 = 0 = \mathcal{Q}^{\dagger}{}^2$ only up to the $\psi, \overline{\psi}$ `equation of motion'\footnote{Of course we are in $d=0$ and there is no time, so we do not have any time-evolution.}. To do better, we will need superfields.

In a supersymmetric theory (in $d = 0$) we can compute the partition function $Z = \int e^{-S / \hbar} \dd[]{x} \dd[2]{\psi}$ exactly.
To do this, let us rescale $w \to \lambda w$, for $\lambda \in \mathbb{R}_{\geq 0}$.
Then the rescaled action $S_{\lambda} = \frac{\lambda^2}{2} (\partial w)^2 - \overline{\psi} \psi \lambda \partial^2 w$ is invariant under $Q_{\lambda} = \psi \frac{\partial }{\partial x} + \lambda ( \partial w) \frac{\partial }{\partial \overline{\psi}}$  and $\mathcal{Q}^{\dagger}_{\lambda}$ .
\begin{claim}
  The key point is that $Z_{\lambda} = \int e^{-S_{\lambda} / \hbar} \dd[]{x} \dd[2]{\psi}$ is independent of $\lambda$.
\end{claim}
\begin{leftbar}
  We will set $\hbar = 1$ from now on.\footnote{$\hbar$ is primarily useful to do a perturbative expansion around the classical limit and we will not do that.}
\end{leftbar}
\begin{proof}
  \begin{equation}
    -\dv{}{\lambda} Z_{\lambda} = \int \dv{S_{\lambda}}{\lambda} e^{-S_{\lambda}} \dd[]{x} \dd[2]{\psi} = \int \left( \lambda (\partial w)^2 - \overline{\psi} \psi \partial^2 w \right) e^{-S_\lambda} \dd[]{x} \dd[2]{\psi}.
  \end{equation}
  Observe that $\mathcal{Q}^{\dagger}_{\lambda}(\psi \partial w)= \overline{\psi} \psi \partial^2 w - \lambda(\partial w)^2 = - \dv{S_{\lambda}}{\lambda}$ .
  Hence, we can write this as
  \begin{equation}
    \dv{Z_{\lambda}}{\lambda} = \int \mathcal{Q}^{\dagger}_{\lambda} (\psi \partial w) e^{-S^{\lambda}} \dd[]{x} \dd[2]{\psi} = \int Q^{\dagger}_{\lambda} \left( \psi \partial w e^{-S_\lambda} \right) \dd[]{x} \dd[2]{\psi}
  \end{equation}
  where we used that the action itself is invariant under $Q_{\lambda}^{\dagger}$ .
  Since $Q^{\dagger}_{\lambda} = \overline{\psi} \frac{\partial }{\partial x} - \lambda (\partial w) \frac{\partial }{\partial \psi}$ , the second term does not survive the integral over $\dd[2]{\psi}$ .
  The first term is a total derivative in $x$, so it dies under integration over $\dd[]{x}$\footnote{We might need to worry about large $x$ behaviour, but $\partial w$ is some polynomial, so the exponential decay of $e^{-S_{\lambda}}$ will always dominate at large $x$.}.
  We conclude that $\dv{Z_{\lambda}}{\lambda} = 0$. 
\end{proof}

In particular, $ Z(1) = \lim_{\lambda \to \infty}  Z(\lambda) $. This is useful because it is easy to compute $Z(\lambda)$  at large $\lambda$: As $\lambda \to \infty$ , the term $e^{\frac{\lambda^2}{2} (\partial w)^2}$  suppresses all contributions, except near critical points $x_*$, where  $w'(x_*) = 0$ .
Suppose $w(x)$  is a generic polynomial of degree $D$ with isolated\footnote{For each critical point, there is always some open neighbourhood around it that does not contain any others.}, non-degenerate\footnote{This means that the second derivative of $w$ does \emph{not} vanish at such a point.} critical points  $x_*$ .
Then near any critical point, 
\begin{equation}
  w(x) = w(x_*) + \frac{c_*}{2} (x-x_*)^2 + \dots,
\end{equation} 
where $c_* = \partial^2 w(x_*)$.
Hence, near $x_*$,
\begin{equation}
  S(x, \psi, \overline{\psi}) = \frac{c_*^2}{2} (x - x_*)^2 - \overline{\psi}\psi c_* + \dots
\end{equation}
The higher order terms in $\delta x = x - x_*$ will be negligible as $\lambda \to \infty$.
Hence, near the critical point,
\begin{align}
  \frac{1}{\sqrt{2\pi}} \int e^{-S(x, \psi, \overline{\psi})} \dd[]{x} \dd[2]{\psi} &= \frac{1}{\sqrt{2\pi}} \int e^{-\frac{c_*}{2} (x  -x_*)^2} \left[ -1 + c_* \overline{\psi} \psi \right] \dd[]{x} \dd[2]{\psi} \\
										    &= \frac{c_*}{\sqrt{ 2 \pi}} e^{-\frac{c_*}{2} (x - x_*)^2} \dd[]{x} \\
										    &= \frac{c_*}{\sqrt{ 2 \pi}} \sqrt{\frac{2 \pi}{c_*^2}} = \frac{c_*}{\abs{c_*}} \\
										    &= \text{sgn}(\partial^2 w \rvert_{x_*}).
\end{align}
Summing over all critical points, 
\begin{equation}
  Z = \sum_{x_*: \partial w\rvert_{x_*} = 0} \text{sgn}(\partial^2 w\rvert_{x_*}).
\end{equation}
As seen in Figure XXX, the partition function only really cares about the degree of the polynomials.
%F1 F2

Let us think about what would have happened if we calculated this with the use of Feynman diagrams.
If you did that, you would see that to all orders in the loop expansions, the diagrams cancel exactly.
The reason for this is really this localisation.

We are just counting the number of things, this is the first sniff at some sort of index theorem.
