% lecture notes by Umut Özer
% course: susy
\lhead{Lecture 5: January 30}

\begin{example}[]
  Let use the localisation theorem to recompute the answer \eqref{eq:4-int} that we got for $S^2$.
  Critical points are the north and south poles.
  Near these, we can find (Darboux) coordinates such that $\omega = dq \wedge d p$ .
  In these coordinates,
  \begin{equation}
    X = k \left( q \frac{\partial }{\partial p} - p \frac{\partial }{\partial q} \right),
  \end{equation}
  for some $k \in \mathbb{Z}$.
  Again, we refer to the illustration in Fig.~\ref{fig:l4f2}.
  The associated Hamiltonian is
  \begin{equation}
    h(q, p) = \frac{1}{2} k (q^2 + p^2)
  \end{equation}
  and our $U(1)$-invariant positive definite metric
  \begin{equation}
    g = dq^2 + dp^2, \qquad g(X, X) = k^2 (q^2 + p^2).
  \end{equation}
  We then have
  \begin{equation}
    \epsilon^{ab} \partial_{a} (g_{bc} X^{c}) = \partial_{q} X_{p} - \partial_{p} X_{q} = 2k.
  \end{equation}
  \begin{equation}
    k^4 
    \begin{vmatrix}
     2 & 0 \\
     0 & 2 \\
    \end{vmatrix}
    = 4 k^4 \implies \sqrt{\det \partial_{a} \partial_{b} g(X, X)} = 2 k^2
  \end{equation}
  Now $k_N = +1$ and  $k_S = -1$, because we rotate anticlockwise at  $N$ and clockwise at $S$.
   \begin{equation}
     \lim_{\lambda \to \infty} Z_{\lambda} = (2\pi) \sum_{x_* = N, S} \frac{e^{i \alpha h(x_*)}}{k_*} 2 \pi \left( e^{i \alpha} - e^{-i\alpha} \right)
  \end{equation}
  Therefore, the integral is
  \begin{equation}
    \int_{S^2} e^{i \alpha \cos \theta} \sin \theta \; d\theta \wedge d\phi = \frac{2 \pi }{i \alpha} \left( e^{i\alpha} - e^{-i\alpha} \right).
  \end{equation}
\end{example}

\chapter{Quantum Mechanics (QFT in \texorpdfstring{$d = 1$}{One Dimension})}%
\label{cha:quantum_mechanics_qft_in_$d = 1$_one_dimension_}

Consider a single (free) particle in $\mathbb{R}^n$. 
In quantum mechanics, we describe this by some state $\Psi \in \mathscr{H} \simeq L^2(\mathbb{R}^n, d^n x)$ at any time $t$.
As time evolves, our state $\Psi$ changes under the action of some unitary operator $U \colon \mathscr{H} \to \mathscr{H}$, with $U(t) = e^{-i H t}$ if the Hamiltonian $H$ is time-independent.
We will often Wick rotate to Euclidean time $t \to -i \tau$, where the operator becomes $U(\tau) = e^{- H \tau}$, which is not unitary, but well behaved for Hermitian $H$.

If our particle is definitely at some $y_0 \in \mathbb{R}^n$  at $\tau = 0$, the amplitude to find it at  $y_1 \in \mathbb{R}^n$  at $\tau = \beta$ is
 \begin{equation}
  \bra{y_1} e^{-\beta H} \ket{y_0} = K_{\beta}(y_1, y_0),
\end{equation}
which is called the \emph{heat kernel} (or \emph{propagator} in the non-Euclidean version).
Explicitly, if our particle has $m = 1$  and $V(x) = 0$ , then the Hamiltonian is $H = -\frac{1}{2} \nabla^2$  and
\begin{equation}
  K_\tau (y_0, y_1) = \frac{1}{(2 \pi \tau)^{n / 2}} e^{-\frac{1}{2} \abs{y_0 - y_1}^2}.
\end{equation}

As illustrated in Fig.~\ref{fig:l5f1}, we break up our time interval into $N$  pieces with $\Delta \tau = \beta / N$ .
\begin{figure}[tbhp]
  \centering
  \def\svgwidth{0.4\columnwidth}
  \input{lectures/l5f1.pdf_tex}
  \caption{Constructing the path integral.}
  \label{fig:l5f1}
\end{figure} 
The heat kernel can then be expanded by inserting identity operator expansions in between each time step
\begin{align}
  \bra{y_1} e^{-\beta H} \ket{y_0} &= \bra{y_1} e^{- \Delta \tau H} e^{-\Delta \tau H} \dots e^{- \Delta \tau H} \ket{y_0} \\
  &= \int \bra{y_1} e^{-\Delta t H} \ket{x_{N-1}}
  \bra{x_{N-1}} e^{-\Delta t H} \ket{x_{N-2}} \dots
  \bra{x_1} e^{-\Delta t H} \ket{y_0} \dd[n]{x_1} \dd[n]{x_2} \dots \dd[n]{x_{N-1}} \\
  &= \int K_{\Delta \tau} (y_1, x_{N-1}) \dots K_{\Delta \tau} (x_2, x_1) K_{\Delta \tau} (x_1, y_0) \prod_{i = 1}^{N-1} \dd[n]{x_1}.
\end{align}
If we set $S_N (x) = \frac{1}{2} \sum_{i = 0}^{N-1} \frac{\abs{x_{i+1}-x_i}^2}{(\Delta \tau)^2} \Delta \tau$ and $D_N x = \frac{1}{(2 \pi \Delta \tau)^{n N / 2}} \prod_{i = 1}^{N-1} \dd[n]{x_i}$ then we can define the path integral as 
\begin{equation}
  \int e^{-S[x]} \pdd{x} \coloneqq \int \lim_{N \to \infty} \left( e^{-S_N(x)} D_N x \right) .
\end{equation}
\begin{leftbar}
  Although the limit of the action and the measure separately does not exist, the product of them does exist in quantum mechanics. This is not true in quantum field theory, which is why we always need to explain how we take this limit; either by putting it on a lattice, imposing a cutoff, or some other regularisation procedure. The job of renormalisation is to explain how our answer depends on the way we take this limit.
\end{leftbar}

Heuristically, the limit of the action is understood as $\lim_{N \to \infty}  S_N(x) = \frac{1}{2} \int_0^\beta \abs{\dot{x}}^2 \dd[]{\tau} = S[x]$. Hence, as a path integral 
\begin{equation}
  \bra{y_1} e^{- H \beta} \ket{y_0} = \int_{C_{\beta} [y_0, y_1]} e^{-S[x]} \pdd{x},
\end{equation}
where $C_{\beta} [y_0, y_1]$ is the space of continuous\footnote{The path in Fig.~\ref{fig:l5f1} is clearly continuous, but not necessarily smooth!} maps $x \colon [0, \beta] \to \mathbb{R}^n$  starting at $x(0) = y_0$  and ending at $x(\beta) = y_1$ .

\begin{definition}[partition function]
  Closely related to this is the \emph{partition function} of QM, given by the trace over the Hilbert space $\mathscr{H}$
  \begin{equation}
    \mathcal{Z}(\beta) = \tr_{\mathscr{H}}(e^{-\beta H}).
  \end{equation}
\end{definition}

We have
\begin{equation}
  \mathcal{Z}(\beta) = \int_{\mathbb{R}^n} \bra{y} e^{-\beta H} \ket{y} \dd[n]{y} = \int_{\mathbb{R}^n} \left( \int_{C_{\beta} [y, y]} e^{-S[x]} \pdd{x} \right) \dd[n]{y}.
\end{equation}
Hence
\begin{equation}
  \mathcal{Z} (\beta) = \int_{C_{S^1}} e^{-S[x]} \pdd{x}
\end{equation}
over the space $C_{S^1} = \text{Maps}(S^1_{\beta}, \mathbb{R})$ of continuous maps $x \colon S^1 \to \mathbb{R}^n$, where $S[x] = \oint_{S^1} \frac{1}{2} \abs{\dot{x}}^2 \dd[]{\tau}$ and the \emph{worldline} is now a \emph{worldcircle}, since the trace causes the integration to come back to the same point.

