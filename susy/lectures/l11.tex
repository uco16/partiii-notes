% lecture notes by Umut Özer
% course: susy
\lhead{Lecture 11: February 20}

Let us now examine this determinant.
We decompose the tangent space $T_{x_0} M$ into eigenspaces of $\mathcal{R}\indices{^{a}_{b}}$ such that restriction of $\mathcal{R}\rvert$ to the $i$\textsuperscript{th} eigenspace, which is even-dimensional since we are on an even-dimensional manifold, looks like
$
  \mathcal{R}_i = 
  \begin{pmatrix}
   0 & \omega_{i} \\
   -\omega_i & 0 \\
  \end{pmatrix}.
$
Let $D_i$ be the restriction of the operator we are interested in, $\delta\indices{^{a}_{b}}  \partial_{\tau} - \mathcal{R}\indices{^{a}_{b}}$ to this subspace.
Moreover, we write $\delta(x^{a})(\tau) = \sum_{k \neq 0} \delta x\indices{^{a}_{k}} e^{2 \pi i k \tau}$ in terms of (non-zero) Fourier modes.
Then the eigenvalues of $D_i$ are $2 \pi i k \pm \omega_i$, where the factor of $2\pi i k$ is picked up by the derivative. Hence, the determinant of the non-zero modes of the restricted operator is
\begin{equation}
  \label{eq:11-det}
  \text{det}'(D_i) = \prod_{k \in \mathbb{Z} \setminus \{0\}} (2 \pi i k + \omega_i) (2 \pi i k - \omega_i) = \prod_{k\neq 0} (- (2 \pi k)^2 - w_i^2) = \prod_{k = 1}^\infty (2 \pi k)^4 \prod_{k = 1}^\infty \left( 1 + \frac{\omega_i^2}{(2 \pi k)^2} \right)^2.
\end{equation}
Again we can use $\zeta$-function regularisation, from which we obtain
\begin{equation}
  \prod_{n = 1}^\infty (2 \pi k)^2 = (4 \pi^2)^{\zeta(0)} e^{-2 \zeta'(0)} = 1.
\end{equation}
This is us waving our hands and not looking too closely, so that we can drop the infinite divergent product.
\begin{leftbar}
If we were doing a more careful job, we would have had to regularise both the fermionic and the bosonic path integrals; if we are doing the regularisation for both in the same way everything works out.
\end{leftbar}
The remaining factor is the product expansion 
\begin{equation}
  \frac{\sinh (z)}{z} = \prod_{n = 1}^\infty \left( 1 + \frac{z^2}{\pi^2 n^2} \right).
\end{equation}
Comparing this to \eqref{eq:11-det}, with $z = \omega_i/2$, we find that
\begin{equation}
\sqrt{\text{det}' D_i} = \frac{\sinh (\omega_i / 2)}{(\omega_i / 2)}.
\end{equation}
Combining the factors from the $n / 2$ eigenspaces of $\mathcal{R}\indices{^{a}_{b}}$, we have that the term \eqref{eq:10-zmap} coming from the fluctuations around the zero-map is
\begin{equation}
  \frac{1}{\sqrt{\text{det}' (\delta\indices{^{a}_{b}} \partial_\tau - \mathcal{R}\indices{^{a}_{b}})}} = \prod_{i = 1}^{n / 2} \frac{(\omega_i / 2)}{\sinh (\omega_i / 2)} = \det( \left( \frac{\mathcal{R} / 2}{\sinh(\mathcal{R} / 2)} \right) \indices{^{a}_{b}}),
\end{equation}
where the matrix inside the determinant is understood via its Taylor expansion.

This was the term coming from fluctuations about constant zero modes. Finally, we must perform the integral over the zero modes $(x_0, \psi_0)$. The action vanishes on these zero modes, but they enter through $\mathcal{R}\indices{^{a}_{bcd}} (x_0)\; \psi_0^{c} \psi_0^{d}$. We obtain
\begin{equation}
  Z(\beta) = \int \det(\frac{R(x_0, \psi_0) / 2}{\sinh(\mathcal{R}(x_0, \psi_0) / 2)}) \dd[n]{x_0} \dd[n]{\psi_0}
\end{equation}
We expand the (slightly complicated) Taylor series until we hit the term with $n$ fermions.
Performing the fermionic integration then gives
\begin{equation}
  Z(\beta) = \int_M \det(\frac{\mathcal{R} / 2}{\sinh(\mathcal{R} / 2)})^{(n)} = \int \hat{A}(M).
\end{equation}
Where the fermionic integral can be written as extracting the top-form ($n$-form) part of an integral over $M$.
In other words, we now think of $\mathcal{R}\indices{^{a}_{b}}(x_0) = \mathcal{R}\indices{^{a}_{bcd}} dx^{c} \wedge dx^{d}$. This combination is often knows as the $\hat{A}$-genus of $M$.
Explicitly, we have
\begin{align}
  \hat{A}(M) &= 1 - \frac{1}{24} p_1 (M) + \frac{7 p_1^2(M) - 4 p_2(M)}{5760} + \dots , \\
  p_1(M) &= -\frac{1}{2} \frac{1}{(2 \pi)^2} \tr(\mathcal{R} \wedge \mathcal{R}), \\
  p_2(M) &= \frac{1}{8 (2 \pi)^4} \left( (\tr \mathcal{R} \wedge \mathcal{R})^2 - 2 \tr (\mathcal{R} \wedge \mathcal{R} \wedge \mathcal{R} \wedge \mathcal{R}) \right).
\end{align}
The \emph{Pontryagin classes} $p_1$ and $p_2$ are polynomials in traces of powers of the curvature.
We see that
\begin{equation}
  \text{ind}(\cancel{\nabla}) = \int_M \hat{A}(M).
\end{equation}
This is the \emph{Atiyah--Singer} theorem for the Dirac operator.
Since $z / \sinh(z)$ is an even function, we only get expansions in $z^2 \rightarrow \mathcal{R}^2$.
Hence $\text{ind}(\cancel{\nabla}) = 0$ whenever $\dim(M) = 4 k + 2$.

\subsection{Coupling to a Vector Bundle}%
\label{sub:coupling_to_a_vector_bundle}

We may wish to describe a charged spinor (electron!) moving on $M$.
To do this (in the Abelian case), we modify the action to include a coupling to a gauge field $A_a (x)$ on $M$.
\begin{equation}
  S[x, \psi] = \int \left[ \frac{1}{2} g(\dot{x}, \dot{x}) + \frac{i}{2} g(\psi, \nabla_\tau \psi) + i A_a (x) \dot{x}^{a} + \frac{1}{2} F_{ab}(x) \psi^{a} \psi^{b} \right] \dd[]{\tau},
\end{equation}
where $F = dA = \frac{1}{2} (\partial_{a} A_b - \partial_{b} A_a) dx^{a} \wedge dx^{b}$ is the EM fieldstrength, pulled back to the worldline.
The bosonic term involves $\dot{x}$, so modifies the momentum operator $p_{a} = g_{ab} \dot{x}^{b} + i A_a + \text{ferms}$.
Upon quantisation, the supercharge 
\begin{equation}
  Q = i g_{ab} \psi^{a} \dot{x}^{b} \to \gamma^{i} e\indices{^{a}_{i}} (\partial_{a} + (\omega_a)_{ij} \Sigma^{ij} + i A_).
\end{equation}
The insertion 
\begin{equation}
  \exp(-i \oint_{S^1} A_a(x (\tau)) \dv{x^{a}}{\tau} \dd[]{\tau})
\end{equation}
is the holonomy / Wilson line around $x(S^1)$, i.e.~the \emph{phase} a charged particle would acquire as it travels around a loop $x(S^1) \subset M$.

\begin{wrapfigure}{R}{0.3\columnwidth}
  \centering
  \def\svgwidth{0.2\columnwidth}
  \input{lectures/l11f1.pdf_tex}
  \caption{A charged fermion moving along trajectory $\Gamma$ in a background field $A$.}
  \label{fig:l11f1}
\end{wrapfigure}
Consider Fig.~\ref{fig:l11f1}. If $\Psi$ is parallel transported from $x$ to $y$ along $\Gamma$, meaning that $V^{a} (\partial_a + i A_a) \Psi = 0$, where $V$ is tangent to $\Gamma$, 
\begin{equation}
  \Psi(y) = \exp(-i \int_x^y A_a(x) V^{a}(x) \dd[]{\tau}) \Psi(x).
\end{equation}
Since our $\mathcal{N} = 1$ theory gives a spinor on $M$, we also get a contribution from its magnetic moment
\begin{equation}
  \frac{1}{2} F_{ab} \gamma^{a} \gamma^{b} = \frac{1}{4} F_{ab} [\gamma^{a}, \gamma^{b}] = F_{ab} \Sigma^{ab}.
\end{equation}
\begin{example}[]
  In $n = \dim (M) = 3$, then $\Sigma^{ab} = \epsilon^{abc} \sigma^{c}$ and $F_{ab} \Sigma^{ab} \mapsto \boldsymbol \sigma \cdot \vb{B}$.
\end{example}
The modified action is still invariant under the original supersymmetry transformations
\begin{equation}
  \delta x^{a} = \epsilon \psi^{a} \qquad \delta \psi^{a} = i \epsilon \dot{x}^{a},
\end{equation}
with supercharge $Q = i g_{ab} \psi^{a} \dot{x}^{b}$. This is quantised as the gauge covariant Dirac operator $\cancel{D}$.
Thus, when computing the supertrace we still get cancellation
\begin{equation}
  \text{Str}_{\mathscr{H}} (e^{-\beta \hat{H}}) = \text{ind}(\cancel{D}).
\end{equation}

What will happen on the path integral side? The new terms in $S[x, \psi]$ are independent of the target space metric $g$ so they do not change the fact that we localise on constant maps. They also do not affect the fluctuation contributions, because they do not come with any $\lambda$.
They the result is only modified by $S[x_0, \psi_0] = -\frac{1}{2} F_{ab} (x_0) \psi_0^{a} \psi_0^{b}$.
\begin{equation}
  \implies \text{ind} (\cancel{D}) = \int_M \left( \hat{A}(M) e^{-F} \right)^{(n)}.
\end{equation}
