% lecture notes by Umut Özer
% course: susy
\lhead{Lecture 4: January 28}

\section{The Duistermaat--Heckmann Theorem}%
\label{sec:the_duistermaat_heckmann_theorem}

\begin{definition}[symplectic manifold]
  A \emph{symplectic manifold} $(M, \omega)$ is a smooth manifold $M$ of dimension $\dim_{\mathbb{R}}(M) = 2n$ on which we have a $2$-form $\omega$, which is
  \begin{itemize}
    \item closed: $d\omega = 0$,
    \item non-degenerate: $\omega(X, Y) = 0$ for all vector fields $Y$ iff $X = 0$.
  \end{itemize}
\end{definition}
Equivalently, the non-degeneracy condition can be expressed as $\omega^n = \det(\omega_{ab}) dx^1 \wedge \dots \wedge dx^{2n}$ is non-vanishing, and therefore provides a (Liouville) volume form.

Let $(M, \omega)$ be a symplectic manifold.
Suppose $X$ is a vector field on $(M, \omega)$  with $\omega$  invariant along the flow generated by $X$.
 \begin{figure}[tbhp]
  \centering
  \def\svgwidth{0.3\columnwidth}
  \input{lectures/l4f1.pdf_tex}
  \caption{}
  \label{fig:l4f1}
\end{figure}
This means that the Lie derivative vanishes
\begin{equation}
  0 = \mathcal{L}_X \omega = \left( \iota_X d + d \iota_X \right) \omega = d(\iota_X \omega),
\end{equation}
where in the last equation we used that $d\omega = 0$.

\begin{definition}[Hamiltonian]
  \label{def:hamvec}
  We say $X$ is a \emph{Hamiltonian vector field} if there exists a map $h \colon M \to \mathbb{R}$ such that $\iota_X \omega = - dh$.
\end{definition}
\begin{example}[]
  Let $M = \mathbb{R}^{2n}$  and $\omega = dp_i \wedge dq^{i}$ . Take $X = \frac{\partial}{\partial q^{i}}$. Then $\iota_X \omega = - dp_i$.
  So translations are Hamiltonian with $p_{i}$ as the Hamiltonian function.
\end{example}

We will be interested in compact $(M, \omega)$ with $\partial M = \emptyset$. We will also require that $X$ generates a $U(1)$\footnote{Strictly speaking it does not need to be $U(1)$. It can be other groups as well, but this is the simplest case.} action on $M$, meaning that generic orbits of  $X$  are circles.
\begin{example}[]
  Consider $M = S^2$  with  $\omega = \sin \theta d\theta \wedge d\phi$ .
  \begin{figure}[tbhp]
    \centering
    \def\svgwidth{0.4\columnwidth}
    \input{lectures/l4f2.pdf_tex}
    \caption{Integral curves on $S^2$.}
    \label{fig:l4f2}
  \end{figure}
  Take $X = \frac{\partial }{\partial \phi}$. There are two fixed points at the north and south poles, as illustrated in \ref{fig:l4f2}.
\end{example}
Then the Duistermaat--Heckmann theorem states that for any $\alpha \in \mathbb{R}$ 
\begin{equation}
  \int_M e^{i \alpha h(x)} \frac{\omega^n}{n!}
\end{equation}
localises to fixed points $x_* \in M$, where $X(x_*) = 0$ .

\begin{example}[$(M, \omega) = (S^2, \sin \theta d\theta \wedge d\phi)$]
  Have $X = - \frac{\partial }{\partial \phi}$ and $h = \cos \theta$. The integral can simply be done without using a fancy localisation theorem. Use the substitution $z = \cos \theta$:
  \begin{equation}
    \label{eq:4-int}
    \int_{S^2} e^{i \alpha \cos \theta} \sin \theta d\theta \wedge d\phi = 2 \pi \int_{-1}^{+1} e^{i \alpha z} \dd[]{z} = \frac{2 \pi}{i \alpha} \left[ e^{i \alpha} - e^{-i \alpha} \right],
  \end{equation}
  which is simply the value of $e^{i \alpha h(x_*)}$ at the north and south poles.
\end{example}

\begin{proof}
  We can derive this using supersymmetry. Our `fields' are $(x^{a}, \psi^{b})$, where the $\psi^{a}$ transform as vectors on $M$. Thus the space of fields is $\mathcal{C} = \Pi TM$.
  A generic superfield is 
  \begin{equation}
    F(x, \psi) = f(x) + \rho_a(x) \psi^{a} + g_{ab} (x) \psi^{a} \psi^{b} + \dots + h(x) \psi^1 \psi^2 \cdots \psi^{2n}.
  \end{equation}
  \begin{leftbar}
    The $\Pi$ is just a notation that reminds us that the $\psi^{a}$ are anti-commuting.
  \end{leftbar}
  As before, we can identify the space of smooth functions $C^\infty (\Pi T M) = \Omega^* (M)$ to be the space of polyforms on the manifold.
  \begin{leftbar}
    On any coordinate patch we have a supervector space given by $\mathbb{R}^{2n \suchthat 2n}$.
    For a general curved manifold we might need to worry about what happens on the overlaps of the coordinate patches, but we are not going to.
  \end{leftbar}
  We choose our action to have 2 parts. Firstly,
  \begin{equation}
    S_0 = -i \alpha \left( h(x) + \omega_{ab}(x) \psi^{a} \psi^{b} \right).
  \end{equation}
  \begin{leftbar}
    \begin{claim}
      This is invariant under supersymmetry transformations generated by the vector field
      \begin{equation}
	\mathcal{Q} = \psi^{a} \frac{\partial }{\partial x^{a}} + X^{a}(x) \frac{\partial }{\partial \psi^{a}}.
      \end{equation}
    \end{claim}
    \begin{proof}
      \begin{equation}
	\frac{i}{\alpha} Q(S_0) = \psi^{a} \partial_{a} h + \partial_{a} \omega_{bc} \psi^{b} \psi^{b} + 2 X^{a} \omega_{ab} \psi^{b}
      \end{equation}
      Now the second part vanishes since $d\omega = 0$.
      \begin{equation}
	\dots = \psi^{a} (\partial_{a} h + 2 X^{b} \omega_{ba}) = 0
      \end{equation}
      since $\iota_X \omega = - dh$ by Def.~\ref{def:hamvec}
    \end{proof}
  \end{leftbar}
  We can write ``$\mathcal{Q} = d + \iota_X$'', giving
  \begin{equation}
    \frac{1}{2} \left\{ \mathcal{Q}, \mathcal{Q} \right\} = (d + \iota_X)^2 = d \iota_X + \iota_X d = \mathcal{L}_X.
  \end{equation}
  So $\mathcal{Q}^2 = 0$ on forms that are invariant along the flow of $X$.
  We now deform $S$ by picking a positive definite metric $g$ on $M$. Then for a constant $\lambda \in \mathbb{R}$ that measures the deformation, we have
  \begin{equation}
    S_{\lambda} = S_0 + \lambda \mathcal{Q} (g(X, \psi)).
  \end{equation}
  Provided the metric is invariant under the flow, $\mathcal{Q}(S_{\lambda}) = \lambda Q^2 (g_{ab} X^{a} \psi^{b}) = 0$. However, this corresponds to $\lambda \mathcal{L}_X (g_{ab} X^{a} dx^{b})$.

  The partition function of this is as always
  \begin{equation}
    Z = \int_{\mathrlap{\Pi TM}} e^{-S_{\lambda}} \dd[2n]{x} \dd[2n]{\psi}
  \end{equation}
  \begin{exercise}
    Check that the measure $\dd[2n]{x} \dd[2n]{\psi}$ is invariant under $\text{Diff}(M)$.
  \end{exercise}
  Again, if we differentiate this with respect to the parameter $\lambda$, we have
  \begin{equation}
    -\dv{Z}{\lambda} = \int \mathcal{Q}(g(X, \psi)) e^{-S_\lambda} \dd[2n]{x} \dd[2n]{\psi}
    = \int \mathcal{Q} (g(X, \psi) e^{-S_\lambda}) \dd[2n]{x} \dd[2n]{\psi}  =0
  \end{equation} 
  Because out whole action is supersymmetric, we were able to make $\mathcal{Q}$ act on everything in the first equality.
  Hence $Z_\lambda$ is, as before, independent of $\lambda$.

  In particular, this is useful since
  \begin{align}
    Z(\lambda &= 0) = \int_{\Pi T M} e^{-S_0} \dd[2n]{x} \dd[2n]{\psi} = (i \alpha)^n \int_M e^{i \alpha h(x)} \text{Pfaff}(\omega_{ab}) \dd[2n]{x} \\
	      &= (i\alpha)^n \int_M e^{i \alpha h(x)} \frac{\omega^n}{n!},
  \end{align}
  which is the Duistermaat--Heckmann integral.
  \begin{leftbar}
    We have recast the problem into supersymmetric language.
  \end{leftbar}
  \begin{remark}
    The deformation term has two pieces $\mathcal{Q} (g(X, \psi)) = (\partial_{c} g_{ab} X^{a}) \psi^{c} \psi^{b} + g(X, X)$. The second term is the important bit; it is purely bosonic and positive definite. Hence as we scale $\lambda$ to a very large value $\lambda \to \infty$, we only get contributions from a neighbourhood of any critical points, where the vector field has zero length and thus vanishes $X(x_*) = 0$.
  \end{remark}
  We know that $Z_\lambda$ is independent of $\lambda$, so the original integral must localise.
  We can evaluate $\lim_{\lambda \to \infty} Z_\lambda$ by using steepest descent.
  \begin{equation}
  Z_\lambda = \lim_{\lambda \to \infty} \left\{ Z_\lambda \right\} \sim \frac{(2\pi)^n}{(i \alpha)^n} \sum_{\substack{x_* \in M \\ X(x_*) = 0}} e^{i \alpha h(x_*)} \left. \frac{\epsilon^{a_1 b_1 \dots a_n b_n} (\partial_{a_1} X_{b_1}) \dots (\partial_{a_n} X_{b_n})}{\sqrt{\det \partial_{a} \partial_{b} g(X, X)}} \right\rvert_{x = x_*}
  \end{equation}
  where $X_{b} = g_{bc} X^{c}$.
  Localisation tells us that this result is exact.
\end{proof}
