% lecture notes by Umut Özer
% course: susy
\lhead{Lecture 8: February 11}

The adjoint of the supercharge is expected to strip off a form
\begin{equation}
  Q^{\dagger} = i \hat{p}_a \hat{\psi}{}^a \to \iota_{\partial / \partial x^a} \frac{\partial }{\partial x^{a}} = d^{\dagger},
\end{equation}
where the contraction $\iota_{\frac{\partial }{\partial x^{a}}} (d x^{b}) = \delta\indices{^{b}_{a}}$ .

The operator $d^{\dagger}$ is the adjoint of $d$ with respect to the inner product on $\mathscr{H}$ and can be also be written as $d^{\dagger} = (-1)^{n (p+1) + 1} \star d \star {}$ acting on $\Omega^p (\mathbb{R}^n)$, where $\star \colon \Omega^p \to \Omega^{n-p}$ is the \emph{Hodge star}.
This follows from the adjoint statement:
\begin{equation}
  \int \alpha \wedge \star d^{\dagger} \beta = (\alpha, d^{\dagger} \beta) = (d\alpha, \beta) = \int_{\mathbb{R}^n} d\alpha \wedge \star \beta,
\end{equation}
and integration by part.
Note also that $d^{\dagger} \colon \Omega^p \to \Omega^{p-1}$ as expected for a contraction.

Our action is also invariant under the (bosonic) $U(1)$ transformations that leave $x$ unchanged but rotate the phase of $\psi$ and $\overline{\psi}{}$ opposite ways
\begin{equation}
  x^{a} \to x^{a} , \qquad \psi^{a} \to e^{i \alpha} \psi^{\alpha}, \qquad \overline{\psi}{} \to e^{-i \alpha} \overline{\psi}{}.
\end{equation}
The associated Noether current is $F = \delta_{ab} \overline{\psi}{}^{a} \psi^{b}$ quantised as the Fermion number operator.
This allows us to treat the space of polyforms $\mathscr{H} \simeq \Omega^*(\mathbb{R}^n, \mathbb{C})$ as a supervector space, with the bosonic and fermionic parts of the Hilbert space being
\begin{equation}
  \mathscr{H}_{B} = \bigoplus_{p \text{ even}} \Omega^{p}(\mathbb{R}^n, \mathbb{C}) \qquad \text{and} \qquad \mathscr{H}_F = \bigoplus_{p \text{ odd}} \Omega^p (\mathbb{R}^n, \mathbb{C}).
\end{equation}
Indeed $F$  simply reads off the homogeneity of the form.
\begin{remark}
  If $\omega \in \Omega^p$ and $\rho \in \Omega^q$, then $\omega \wedge \rho = (-1)^{pq} \rho \wedge \omega$.
\end{remark}

\section{\texorpdfstring{$\mathcal{N} = 1$}{One-dimensional} SQM and Spinors }%
\label{sec:$n = 1$_one_dimensional_sqm_and_spinors_}

$\mathcal{N} = 1$ SQM is a theory of $n$ real fermions with action
\begin{equation}
  S[x, \psi] = \frac{1}{2} \int \delta_{ab} \dot{x}^{a} \dot{x}^{b} + i \delta_{ab} \psi^{a} \dot{\psi}^{b} \dd[]{\tau}.
\end{equation}
\begin{remark}
  Note that the fermionic part of the action would not make sense if $\psi^{a}$ were bosonic, since integration by part would induce a minus sign; this form of an action would therefore vanish for bosonic fields.
  For fermionic fields, the minus sign is cancelled by the change in order due to the integration by parts.
\end{remark}
This is invariant under the supersymmetry
\begin{equation}
  \label{eq:n1susy}
  \delta x^{a} = \epsilon \psi^{a} \qquad \delta\psi^{a} = i \epsilon \dot{x}^{a},
\end{equation}
which is generated by the supercharge
\begin{equation}
  Q = i \delta_{ab} \dot{x}^{a} \psi^{b}.
\end{equation}
However, not that the bosonic $U(1)$  is broken to a $\mathbb{Z}_2$  subgroup; we cannot change the phase of $\psi$ , only the sign.

Upon quantisation, we have standard commutation relations between $\hat{p}_{b}$  and $\hat{x}^{a}$  as $[\hat{p}_b, \hat{x}^{a}] = i \delta\indices{^{a}_{b}}$  as before. But since
\begin{equation}
  \pi_a = \frac{\delta L}{\delta \dot{\psi}^{a}} = \frac{i}{2} \psi^{b} d_{ab},
\end{equation} 
the fermionic field is its own conjugate momentum!
It obeys the standard anticommutation relations, which become
\begin{equation}
  \label{eq:7-cliff}
  \{\hat{\psi}^{a}, \hat{\psi}^{b}\} = 2 \delta^{ab}.
\end{equation}
To find a representation of this $\mathscr{H}$, we need to split the $\hat{\psi}^a$ 's into raising and lowering operators.
We can do this in the language of forms. However, the $\hat\psi$'s are most naturally interpreted as $\gamma$-matrices acting on the space of spinors, since they obey \eqref{eq:7-cliff}, which is exactly the Clifford algebra.

First suppose $n$  is even. Over $\mathbb{C}$ , construct $\frac{n}{2}$  raising and lowering operators as
\begin{equation}
  \gamma^{i}_{\pm} = \frac{1}{2} \left( \gamma^{2i} \pm i \gamma^{2i + 1} \right),
\end{equation}
for $i = 1, \dots, \frac{n}{2}$ .
These obey the algebra of creation and annihilation operators
\begin{equation}
  \{ \gamma^i_{pm}, \gamma^j_{\pm} \} = 0 \qquad \{\gamma^i_+, \gamma^j_-\} = \delta^{ij},
\end{equation}
just as in $\mathscr{N} = 2$ SQM, but in half the dimension.

Starting from a spinor $\chi$ that obeys $\gamma^{i}_- \chi = 0$  for all $i$, we construct a basis of the space of spinors by acting with each raising operator  $\gamma^{i}_+$  at most once (since they anti-commute, as before).
Hence we obtain a representation of the spin group $\text{Spin(n)}$ , with dimension $2^{n / 2}$; for each value of $i$, there is a choice of whether to use or not use the raising operator.
\begin{example}[]
  For $n = 4$, we have $2^{4 / 2} = 4 $ components. This is the Dirac spinor that we are familiar with from QED.
\end{example}

The generators of $\text{Spin}(n)$  act on this representation by $\Sigma^{ab} = \frac{1}{4} [\gamma^a, \gamma^b]$ .
These generators obey
\begin{equation}
  [\Sigma^{ab}, \Sigma^{cd}] = \delta^{bc} \Sigma^{ad} + \delta^{ad} \Sigma^{bc} - \delta^{ac} \Sigma^{bd} - \delta^{bd} \Sigma^{ac}.
\end{equation}
of $\text{Spin}(n) \simeq SO(n)$ .
Similarly to the familiar story in $n = 4$, since  $\Sigma^{ab}$  are quadratic in the $\gamma$ 's, states with an odd / odd number of creation operators $\psi^{i}_+$ 's acting on the vacuum $\chi$ do not mix under $SO(n)$  transformations.
Hence, the representation is \emph{reducible}; there are two invariant non-trivial subspaces.

\begin{definition}[chirality matrix]
  We define the \emph{chirality matrix} as the product of all the $\gamma$  matrices
  \begin{equation}
    \gamma = (i)^{n / 2} \gamma^1 \gamma^2 \dots \gamma^{n} = \frac{i^{n / 2}}{n!} \epsilon_{a_1 a_2 \dots a_n} \gamma^{a_1} \gamma^{a_2} \dots \gamma^{a_n}.
  \end{equation}
\end{definition}
\begin{remark}
  This is the generalisation of $\gamma^5$ in $d = 4$. We could call this $\gamma^{2n+1}$, but we will just call it $\gamma$ to save writing large indices.
\end{remark}
As in $d = 4$, we have
 \begin{equation}
  \gamma^2 = 1, \qquad \{\gamma, \gamma^{a}\} = 0, \qquad [\gamma, \Sigma^{ab}] = 0.
\end{equation}
Since $\gamma^2 = 1$, its eigenvalues are  $\pm 1$.
\begin{definition}[]
  We say a spinor is \emph{chiral} if it is in the $+1$ eigenspace of $\gamma$ and \emph{antichiral} if in the $-1$ eigenspace.
\end{definition}
Hence, the space of spinors splits into spinors of definite chirality
\begin{equation}
  S = S^+ \oplus S^-.
\end{equation}
\begin{remark}
  In $d = 4$, this is the statement that we can decompose the $4$-component Dirac spinor into two $2$-component Weyl spinors.
\end{remark}
Here, $\gamma$ plays the role of $(-1)^F$.
We do not have an operator $F$ in this case, since we do not have the $U(1)$ symmetry; however, we do have the operator $(-1)^F$.
\begin{remark}
  For $n$ odd, we can construct a $2^{\floor*{\frac{n}{2}}}$-dimensional spin representation as before.
  However, now this representation is irreducible as $\gamma^n$ appears in some of the $\Sigma^{ab}$.
\end{remark}

To summarise, we represent $\mathscr{H}$ in $\mathcal{N} = 1$ SQM as the space $\Gamma(\mathbb{R}^n, S)$  of ($L^2$-integrable) Dirac spinors on $\mathbb{R}^n$ . When acting on this space, the supercharge becomes
\begin{equation}
  Q = i \psi^a p_a \to \cancel{\partial},
\end{equation}
the Dirac operator.
When $n$  is even, we can split this 
\begin{equation}
  \cancel{\partial} = \cancel{\partial} \left( \frac{1 + \gamma}{2} \right) + \cancel{\partial} \left( \frac{1 - \gamma}{2} \right) = \cancel{\partial}_+ + \cancel{\partial}_-, 
\end{equation}
where the $\frac{1}{2}( 1 \pm \gamma)$ project onto $S^{\pm}$, the spaces of definite chirality.
Due to the anticommutation relation $\{\gamma^a, \gamma\} = 0$, the chirality operator anticommutes with $\cancel{\partial}$, so
\begin{equation}
  \cancel{\partial}_+ \colon \Gamma(\mathbb{R}^n, S^+) \to \Gamma(\mathbb{R}^n, S^-) \quad \text{and} \quad \cancel{\partial}_- \colon \Gamma(\mathbb{R}^n, S^-) \to \Gamma(\mathbb{R}^n, S^+).
\end{equation}
In particular, $\cancel{\partial}^2_+ = \cancel{\partial}_-^2 = 0$.
