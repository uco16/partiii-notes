% lecture notes by Umut Özer
% course: susy
\lhead{Lecture 12: February 25}

\chapter{Supersymmetric QFT in \texorpdfstring{$d = 2$}{Two Dimensions}}%
\label{cha:susy_qft_in_two_dimensions}

Consider $\mathbb{R}^2$ with coordinates $(t, x) = (x^0, x^1)$ and metric $\eta = \text{diag}(+, -)$.
We have worldsheet Lorentz transformations
\begin{equation}
  \begin{pmatrix}
  x^0 \\
  x^1 \\
  \end{pmatrix} \mapsto
  \begin{pmatrix}
   \cosh\gamma & \sinh \gamma \\
   \sinh \gamma & \cosh \gamma \\
  \end{pmatrix}
  \begin{pmatrix}
  x^0 \\
  x^1 \\
  \end{pmatrix}
\end{equation}
so $\text{Lorentz} \simeq SO(1, 1)$, which is Abelian.
Hence, its irreps are $1$-dimensional. In particular, Dirac spinors have $2^{d / 2} = 2$ components, but chiral spinors\footnote{We are in even dimension, so we can define a chirality matrix $\gamma$ as in \eqref{eq:chirality}.} have just one component $\psi_+$ or $\psi_-$ transforming as $\psi_{\pm} \mapsto e^{\pm \gamma / 2} \psi_{\pm}$.

\section{\texorpdfstring{$\mathcal{N} = (2, 2)$}{2, 2} Superspace and Superfields}%
\label{sec:2_2_superspace}

Let $\mathbb{R}^{2 \mid 4}$ be the superspace with 10 coordinates $(x^{\pm}, \theta^{\pm}, \overline{\theta}{}^{\pm})$, where $x^{\pm} =  x^0 \pm x^1$, $\overline{\theta}{}^{\pm} = (\theta^{\pm})^*$ and $\theta^{\pm} \mapsto e^{\pm \gamma / 2} \theta^{\pm}$ under $SO(1, 1)$ transformations.

We introduce fermionic derivatives
\begin{equation}
  \label{eq:12-Q}
  \mathcal{Q}_{\pm} = \frac{\partial }{\partial \theta^{\pm}} + i \overline{\theta}{}^{\pm} \frac{\partial }{\partial x^{\pm}}, \qquad \overline{\mathcal{Q}}{}_{\pm} = -\frac{\partial }{\partial \overline{\theta}{}^{\pm}} - \theta^{\pm} \frac{\partial }{\partial x^{\pm}},
\end{equation}
acting on $\mathbb{R}^{2 \mid 4}$.
These obey
\begin{equation}
  \label{eq:12-star}
  \{\mathcal{Q}_{\pm}, \overline{\mathcal{Q}}_{\pm}\} = -2 i \frac{\partial }{\partial x^{\pm}},
\end{equation}
with all other anticommutators of $\mathcal{Q}$'s, $\overline{\mathcal{Q}}{}$'s trivial.
The $-i \partial_{\pm}$ generate translations on $\mathbb{R}^2$ and will be realised in our theory by $H \pm P$ so we recognise this \eqref{eq:12-star} as our susy algebra.
The infinitesimal supersymmetry transformation is generated by
\begin{equation}
  \label{eq:12-delta}
  \delta = \epsilon_+ \mathcal{Q}_- + \epsilon_- \mathcal{Q}_+ - \overline{\epsilon}{}_+ \overline{\mathcal{Q}}{}_- - \overline{\epsilon}{}_- \overline{\mathcal{Q}}{}.
\end{equation}
To write a susy theory, need superfields on our worldsheet $\mathbb{R}^{2 \mid 4}$. In the basic case, these are just functions 
\begin{equation}
  F(x^{\pm}, \theta^{\pm}, \overline{\theta}{}^{\pm}) = f(x^{\pm}) + \theta^+ \psi_+ (x^{\pm}) + \theta^- \psi_- (x^{\pm}) + \dots + \theta^+ \theta^- \overline{\theta}{}^{+} \overline{\theta}{}^{-} D(x^{\pm}).
\end{equation}
\begin{leftbar}
  It is conventional to call the top-component of a real superfield $D$.
\end{leftbar}
Altogether, $F$ has $2^4 = 16$ component fields, since each of the four $\theta^{\pm}, \overline{\theta}{}^{\pm}$ can or cannot appear.
Under supersymmetry transformations, the components transform via
\begin{equation}
  F \mapsto F + \delta F,
\end{equation}
where $\delta$ is the vector field defined in \eqref{eq:12-delta}.
This is pretty messy. However, note that the highest component $D(x^{\pm})$ changes by a bosonic total derivative:
\begin{equation}
  D(x^{\pm}) \mapsto D(x^{\pm}) + \partial_{\pm} ( \dots).
\end{equation}
This is since the fermionic derivatives \eqref{eq:12-Q} have one component which strips off a fermion, moving terms in $F$ towards the left, and another component which adds a fermion, moving components along the right.
However, this second operation, which is the only one that adds to $D(x^{\pm})$, comes with a bosonic derivative.
\begin{definition}[chiral superderivatives]
  We will often be interested in smaller superfields. To do this, introduce \emph{chiral superderivatives}
  \begin{equation}
    D_{\pm} = \frac{\partial }{\partial \theta^{\pm}} - i \overline{\theta}{}^{\pm} \frac{\partial }{\partial x^{\pm}}, \qquad \overline{D}{}_{\pm} = - \frac{\partial }{\partial \overline{\theta}{}^{\pm}} + i {\theta}^{\pm} \frac{\partial }{\partial x^{\pm}}.
  \end{equation}
\end{definition}
These obey
\begin{equation}
  \{D_{\pm}, \overline{D}{}_{\pm}\} = +2i \partial_{\pm}
\end{equation}
and, importantly
\begin{equation}
  \{D_+, \mathcal{Q}_{\pm}\} = 0 = \{D_+, \overline{Q}{}_{\pm}\},
\end{equation}
and likewise for $D_-, \overline{D}{}_{\pm}$.
\begin{definition}[chiral superfield]
  A \emph{chiral superfield} is a superfield $\Phi$ that obeys $\overline{D}{}_{\pm} \Phi = 0$.
  Its complex conjugate $\overline{\Phi}{} = (\Phi)^*$ obeys $D_{\pm} \overline{\Phi}{} = 0$ and is called \emph{antichiral}.
\end{definition}
Under supersymmetry, $\Phi \mapsto \Phi + \delta \Phi$, but since all $\Delta$'s anticommute with all $\mathcal{Q}$'s,
\begin{equation}
  \overline{D}{}_{\pm} (\delta \Phi) = \delta (\overline{D}{}_{\pm} \Phi) = 0,
\end{equation}
so chiral superfields remain chiral under supersymmetry.
This is why these derivatives are useful: they are compatible with supersymmetry.
Note that if $\Phi_1$ and $\Phi_2$ are chiral, so too is $\Phi_1 \Phi_2$ and $W(\Phi_1)$ for any holomorphic function $W(z)$.

The conditions $\overline{D}{}_{\pm} \Phi = 0$ mean that $\Phi$ can depend on $\overline{\theta}{}^{\pm}$ only through $y^{\pm} = x^{\pm} - i \theta^{\pm} \overline{\theta}{}^{\pm}$, which is annihilated by $\overline{D}{}_{\pm}$.
Hence $\Phi = \Phi(y^{\pm}, \theta^{\pm})$ and the component expansion
\begin{align}
  \Phi(y^{\pm}, \theta^{\pm}) &= \phi(y^{\pm}) + \theta^+ \psi_+ (y^{\pm}) + \theta^- \psi_- (x^{\pm}) + \theta^+ \theta^- F(y^{\pm}) \\
  \begin{split}
    \label{eq:12-comp}
    &= \phi(x^{\pm}) -i \theta^+ \overline{\theta}{}^+ \partial_+ \phi(x^{\pm}) - i \theta^- \overline{\theta}{}^- \partial_- \phi (x^{\pm}) - \theta^+ \overline{\theta}{}^+ \theta^- \overline{\theta}{}^- \partial_+ \partial_- \phi(x^{\pm}) \\
    &\qquad {}+ \theta^+ \psi_+ (x^{\pm}) - i \theta^+ \theta^- \overline{\theta}{}^- \partial_- \psi_+ (x^{\pm}) - i \theta^- \theta^+ \overline{\theta}{}^- \partial_+ \psi_- (x^{\pm}) + \theta^+ \theta^- F(x^{\pm}).
  \end{split}
\end{align}
\begin{remark}
  It is conventional to call the top-component of a chiral superfield $F$.
\end{remark}
To work out the susy transformations of the component fields, change variables
\begin{equation}
  (x^{\pm}, \theta^{\pm}, \overline{\theta}{}^{\pm}) \mapsto (y^{\pm}, \theta^{\pm}).
\end{equation}
If we do that, we have
\begin{align}
  \mathcal{Q}_{\pm} &= \left.\frac{\partial }{\partial \theta^{\pm}}\right\rvert_{x^{\pm}, \overline{\theta}{}^{\pm}} + i \overline{\theta}{}^{\pm} \left.\frac{\partial }{\partial x^{\pm}} \right\rvert_{\theta, \overline{\theta}{}} \\
  &= \left. \frac{\partial }{\partial \theta^{\pm}} \right\rvert_{y, \overline{\theta}{}} + \left. \frac{\partial y^{\pm}}{\partial \theta^{\pm}} \right\rvert_{x, \overline{\theta}{}} \left. \frac{\partial }{\partial y^{\pm}} \right\rvert_{\theta, \overline{\theta}{}} + i \overline{\theta}{}^{\pm} \left. \frac{\partial }{\partial y^{\pm}} \right\rvert_{\theta, \overline{\theta}{}} \\
  &= \left. \frac{\partial }{\partial \theta^{\pm}} \right\rvert_{y, \overline{\theta}{}}.
\end{align}
Similarly, 
\begin{equation}
  \overline{\mathcal{Q}}{}_{\pm} = - \left.\frac{\partial }{\partial \overline{\theta}{}^{\pm}} \right\rvert_{y, \theta} - 2 i \theta^{\pm} \left. \frac{\partial }{\partial y^{\pm}} \right\rvert_{\theta, \overline{\theta}{}}.
\end{equation}
Using these, we find (exercise)
\begin{align}
  \delta \phi &= \epsilon_+ \psi_- - \epsilon_- \psi_+ \\
  \delta \psi_{\pm} &= \pm 2 i \overline{\epsilon}{}_{\mp} \partial_{\pm} \phi + \epsilon_{\mp} F \\
  \delta F &= -2 i \overline{\epsilon}{}_{+} \partial_- \psi_+ - 2 i \overline{\epsilon}{}_- \partial_+ \psi_-.
\end{align}
In particular, again the highest term $\theta^+ \theta^- F(x^{\pm})$ transforms only by a total derivative.

\subsection{Supersymmetric Actions in \texorpdfstring{$d = 2$}{Two Dimensions}}%
\label{sub:supersymmetric_actions}

There are two basic ways of constructing any supersymmetric action in two dimensions.

\subsubsection*{Using Real Superfields}%
%\label{sub:using_real_superfields}

We use the observation that the highest terms in real or chiral superfields transform only by total derivatives $\partial_{\pm} ( \dots )$ to construct manifestly supersymmetric actions.
Let $K(F, \Phi, \overline{\Phi}{})$ be a real function of real superfields $F^i$ and chiral superfields $\Phi^a$.
Then $K$ itself has a $\theta$-expansion, so is a real superfield.
Hence the action
\begin{equation}
  \int_{\mathbb{R}^{2 \mid 4}} K(F, \Phi, \overline{\Phi}{}) \dd[2]{x} \dd[4]{\theta} = \int_{\mathbb{R}^2} K(F, \Phi, \overline{\Phi}{}) \rvert_{\theta^2 \overline{\theta}{}^2} \dd[2]{x}.
\end{equation}
Only the top-component---the term that has all four $\theta$'s in the expansion---will survive.
Because the top-component $K(F, \Phi, \overline{\Phi}{})\rvert_{\theta^2 \overline{\theta}{}^2}$ only changes by a total bosonic derivative, the action is supersymmetric (up to boundary terms).
$K$ is called the \emph{Kähler potential}, and $\int K \dd[2]{x} \dd[4]{\theta}$ is called a \emph{$D$-term}.

\subsubsection*{Using Chiral Superfields}%
%\label{sub:using_chiral_superfields}

Alternatively, let $W(\Phi)$ be a holomorphic function, so $W(\Phi)$ is itself a chiral superfield.
The integral over the full superspace would just vanish, since it does not depend on $\overline{\theta}{}$.
However, if we integrate only over $\mathbb{R}^{2 \mid 2}_c$, then
\begin{equation}
  \int W (\Phi) \dd[2]{y^{\pm}} \dd[]{\theta^+} \dd[]{\theta^-} = \int_{\mathbb{R}^2} W(\phi)_{\theta^+ \theta^-} \dd[2]{y},
\end{equation}
which is again supersymmetric (up to boundary terms).
We call $W(\Phi)$ the \emph{superpotential} and the integral $\int W(\Phi) \dd[2 \mid 2]{x}$ is called an \emph{$F$-term}.

It will turn out that the kinetic terms will always appear from the Kähler potential, whereas the superpotential gives the interacting potential.
Our generic action will take the form
\begin{equation}
  S[K, \Phi, \overline{\Phi}{}] = \int_{\mathbb{R}^{2 \mid 4}} K (\Phi, \overline{\Phi}{}) \dd[2 \mid 4]{x} 
  + \biggl[\int_{\mathbb{R}^{2 \mid 2}_c} W \dd[2]{x} \dd[2]{\theta} + \int \overline{W}{} \dd[2]{x} \dd[2]{\overline{\theta}}\biggr].
\end{equation}

\subsection{The Wess--Zumino Model}%
\label{sub:the_wess_zumino_model}

The Wess--Zumino model is a theory of a single chiral superfield $\Phi$, where the Kähler potential is 
\begin{equation}
  K(\Phi, \overline{\Phi}{}) = \abs{\Phi}^2.
\end{equation}
From the component expansion \eqref{eq:12-comp}, we have
\begin{multline}
  \overline{\Phi}{} \Phi\rvert_{\theta^4} = - \overline{\phi}{} \partial_+ \partial_- \phi + \partial_+ \overline{\phi}{} \partial_- \phi + \partial_- \overline{\phi}{} \partial_+ \phi - (\partial_* \partial_- \overline{\phi}{}) \phi \\
  + i \overline{\psi}{}_+ \partial_- \psi_+ - i (\partial_- \overline{\psi}{}_+) \psi_+ + i \overline{\psi}{} \partial_+ \psi_- - i (\partial_+ \overline{\psi}{}_-) \psi_- + \abs{F}^2.
\end{multline}
Hence, after bosonic intergration by parts, we obtain
\begin{equation}
  \frac{1}{2} \int \overline{\Phi}{} \Phi \dd[4]{\theta} \dd[2]{x} = \int_\mathbb{R} \abs{\partial_0 \phi}^2 - \abs{\partial_1 \phi}^2 + i \overline{\psi}{}_- \partial_+ \psi_- + i \overline{\psi}{}_+ \partial_- \psi_+ + \frac{1}{2} \abs{F}^2.
\end{equation}
Thus the Kähler potential has given us the kinetic terms for our fields. To get some interactions, we should include a superpotential $W(\Phi)$.
When expanding the superpotential, we should treat our superfield $W(\Phi (y^{\pm}, \theta^{\pm}))$ as a chiral superfield,
\begin{equation}
  \Phi(y^{\pm, \theta^{\pm}}) = \phi(y^{\pm}) + \theta^+ \psi_+ (y^{\pm}) + \theta^- \psi_- (\psi^{\pm}) + \theta ^+ \theta^- F,
\end{equation}
since the superpotential action is an integral over the chiral superspace $\mathbb{R}_c^{2 \mid 2}$.
The superpotential integral in the action is going to strip off only the $W(\Phi) \rvert_{\theta^2}$ component, which is
\begin{equation}
  W(\Phi) \rvert_{\theta^2} = W'(\phi) F - W'' (\phi) \psi_+ \psi_-.
\end{equation}
Therefore, the interaction action is
\begin{align}
  S_{\text{int}} (\phi, \phi_{\pm}, F) &= \int W(\Phi) \dd[2]{\theta} \dd[2]{y} + \int \overline{W} (\overline{\Phi}{}) \dd[2]{\overline{\theta}{}} \dd[2]{\overline{y}{}} \\
				       &= \int_{\mathbb{R}^2} \left[ W'(\phi) F - W''(\phi) \psi_+ \psi_- + \overline{W}{}' (\overline{\phi}{}) \overline{F}{} - \overline{W}{}'' (\overline{\phi}{}) \overline{\psi}{}_- \overline{\psi}{}_+ \right] \dd[2]{x}.
\end{align}
In both integrals in the first line, $y, \overline{y}{}$ are dummy bosonic variables, which have been treated as $x$ in going to the second line.
The field $F$ has a purely algebraic equation of motion:
\begin{equation}
  \overline{F}{} = -2 W'(\phi), \qquad F = -2 \overline{W}{}'(\overline{\phi}{}).
\end{equation}
Consequently, the field is \emph{auxiliary} and can be eliminated even at the quantum level. Doing so leaves us with a scalar potential and Yukawa interactions:
\begin{equation}
  V(\phi) = \abs{W'(\phi)}^2, \qquad W''(\phi) \psi_+ \psi_-.
\end{equation}
\begin{remark}
  There might be mistakes with the factors of two here. However, the important thing is that $V(\phi)$ is the mod squared of something.
\end{remark}

\subsection{Symmetries of the Wess--Zumino Model}%
\label{sub:symmetries_of_the_wess_zumino_model}

\subsection*{Supersymmetries}%

Using superspace has made manifest that the Wess--Zumino model is invariant under supersymmetry transformations.
\begin{exercise}
  You can check that the corresponding Noether currents, called \emph{supercurrents}, are
  \begin{align}
    G_{\pm}^0 &= 2 \partial_{\pm} \overline{\Phi}{} \psi_{\pm} \mp i \overline{\psi}{}_{\mp} \overline{W}{}'(\overline{\phi}{})  \\
    G_{\pm}^1 &= \mp 2 \partial_{\pm} \overline{\phi}{} \psi_{\pm} - i \overline{\psi}{}_{\mp} \overline{W}{}'(\overline{\phi}{}),
  \end{align}
  and similarly for $\overline{G}{}^{0, 1}_{\pm}$.
\end{exercise}
These have vector indices on the worldsheet, since they are currents, but they also have spinor indices since they are currents for transformations on spinors.
The corresponding supercharges
\begin{equation}
  Q_{\pm} = \int_{x^0 = \text{const.}} G^0 _{\pm} \dd[]{x^1}, \qquad
  \overline{Q}{}_{\pm} = \int_{x^0 = \text{const.}} \overline{G}{}^0 _{\pm} \dd[]{x^1}.
\end{equation}
\begin{remark}
  Under Lorentz transformation with rapity $\gamma$, we have, 
  \begin{equation}
    Q_{\pm} \mapsto e^{\mp \gamma / 2} Q_{\pm}, \qquad 
    \overline{Q}{}_{\pm} \mapsto e^{\mp \gamma / 2} \overline{Q}{}_{\pm}.
  \end{equation}
  So $Q$'s and $\overline{Q}{}$'s are chiral \emph{spinors}, while the $G^{\mu}_{\pm}$ transform as vectors $\otimes$ spinors.
\end{remark}

\subsection*{Bosonic Symmetries}%

The WZ model also has bosonic symmetries. Consider following phase transformation of the $\theta$'s:
\begin{equation}
  \Phi(x^{\pm}, \theta^{\pm}, \overline{\theta}{}^{\pm}) \mapsto \Phi(x^{\pm}, e^{\mp i \alpha} \theta^{\pm}, e^{\pm i \alpha} \overline{\theta}{}^{\pm}),
\end{equation}
for $\alpha \in \mathbb{R}$.
These are called axial $U(1)$ transforms ($U(1)_A$), since they act differently on $\theta^+$ and $\theta^-$.
This transformation leaves $\theta^+ \theta^-$ and $\overline{\theta}{}^+ \overline{\theta}{}^-$ and $\theta ^+ \theta^- \overline{\theta}{}^+ \overline{\theta}{}^-$ invariant. Because of this, the Kähler term $\int K(\Phi, \overline{\Phi}{}) \dd[4]{\theta}$ and the superpotential term $\int W(\Phi) \dd[2]{\theta}$ will also be invariant.
So this is a symmetry, no matter which potentials we use.
\begin{remark}
  We have not said that the overall field $\Phi$ does not have any charge. We have only said how the $\theta$'s transform.
\end{remark}
We can take it alternatively to act on the component fields.
Instead of rotating the $\theta$'s, we can get the same transformations by acting as
\begin{equation}
  \phi (x^{\pm}) \mapsto \phi(x^{\pm}), \qquad \psi_{\pm} (x^{\pm}) \mapsto e^{\mp i \alpha} \psi_{\pm} (x^{\pm}), \qquad F(x^{\pm}) \mapsto F(x^{\pm}).
\end{equation}
Since this is a symmetry, there is an associated Noether current $J^{\mu}_A$ with time and space components
\begin{equation}
  J^0_A = \overline{\psi}{}_+ \psi_+ - \overline{\psi}{}_- \psi_-, \qquad
  J^1_A = -\overline{\psi}{}_+ \psi_+ - \overline{\psi}{}_- \psi_-,
\end{equation}
and charge
\begin{equation}
  F_A = \int_{\mathrlap{x^0 = \text{const.}}} \; J_A^0 \dd[]{x^1}.
\end{equation}
We also consider $U(1)_V$ (vector) transformations, which rotate $\theta^+$ ($\overline{\theta}{}^+$) the same way as $\theta^-$ ($\overline{\theta}{}^-$):
\begin{equation}
  \Phi(x^{\pm}, \theta^{\pm}, \overline{\theta}{}^{\pm}) \mapsto \Phi(x^{\pm}, e^{-i \beta} \theta^{\pm}, e^{i \beta} \overline{\theta}{}^{\pm}).
\end{equation}
This leaves $\theta^2 \overline{\theta}{}^2$ invariant, so is a symmetry of $\int K(\Phi, \overline{\Phi}{}) \dd[4]{\theta}$. However, the measure $\dd[2]{\theta}$ is not invariant, so to make it a symmetry of $\int W(\Phi) \dd[2]{\theta}$, we need to assign a $U(1)_V$ charge to $\Phi$ as a whole:
\begin{equation}
  \Phi(x^{\pm}, \theta^{\pm}, \overline{\theta}{}^{\pm}) \mapsto e^{i q \beta} \Phi(x^{\pm}, e^{-i \beta} \theta^{\pm}, e^{i \beta} \overline{\theta}{}^{\pm}).
\end{equation}

\begin{example}[monomial]
  If the superpotential is just a monomial $W(\Phi) = c \phi^k$, choosing $\Phi$ to have charge $q = k / 2$ will balance the transformation incurred by the $\theta^+ \theta^-$.
\end{example}

\subsection{Vacuum Moduli Spaces}%

Our scalar potential $V(\phi) = \abs{W'(\phi)}^2$ (or more generally $ V(\phi^a) = \sum_a \abs{\frac{\partial W}{\partial \phi^a}}^2$ for several superfields) is positive definite. However, we know from the supersymmetry algebra that supersymmetric states all have $\bra{\Omega} H \ket{\Omega} \geq 0$.
In particular, the ground state of this $(1 + 1)$-dimensional QFT must have non-negative energy. This in the vacuum, we require $\phi(x, t) = \phi_0$, where $\phi_0$ is a critical point the superpotential.

\begin{example}[]
  Let $W (\phi) = \frac{1}{2} m \phi^2 + \frac{1}{3} \lambda \phi^3$. We have $W'(\phi) = m \phi + \lambda \phi^2$, so $W'(\phi) = 0$ when  $\phi = 0, - m / \lambda$.
  In this case, the space of possible vacua, called the \emph{vacuum moduli space}, is 
  \begin{equation}
    \mathcal{M} = \{0, -\frac{m}{\lambda}\}.
  \end{equation}
  The scalar potential $V(\phi) = \abs{W'(\phi)}^2$ is sketched in \ref{fig:l12f1}.
  \begin{figure}[ht]
    \centering
    \inkfig[0.5]{l12f1}
    \caption{A superpotential $W$ with vacuum moduli space $\left\{0, -m / \lambda\right\}$.}
    \label{fig:l12f1}
  \end{figure}
\end{example}

\begin{example}[]
  Take the theory of 3 chiral superfields $X, Y, Z$ with $W(X, Y, Z) = XY Z$.
  Then
  \begin{equation}
    V(X, Y, Z) = 0 \implies \partial_X W = Y Z  = 0 \quad \partial_Y W = X Z = 0 \quad \partial_Z W = X Y = 0.
  \end{equation}
  Hence
  \begin{equation}
    \mathcal{M} = \{X = Y = 0\} \cup \{Y = Z = 0\} \cup \{Z = X = 0\},
  \end{equation}
  which is three copies of $\mathbb{C}$, meeting at the origin. 
  This is shown in Fig.~??
\end{example}
