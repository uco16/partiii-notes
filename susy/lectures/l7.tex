% lecture notes by Umut Özer
% course: susy
\lhead{Lecture 7: February 06}

Just as for the harmonic oscillator, we have fermionic \emph{coherent states} labelled by a fixed Grassmann parameter $\eta^a$ defined as
 \begin{equation}
  \ket{\eta} = e^{-\eta^a \hat{\overline{\psi}{}}_a} \ket{0},
\end{equation}
which are eigenstates of the lowering operators $\hat{\psi}^a \ket{\eta} = \eta^a \ket{\eta}$ .
These are analogous to position eigenstates $\hat{x}^a \ket{x'} = x'{}^a \ket{x'}$  in the bosonic system.
As such, they are useful building blocks for the path integral.
\begin{claim}
  \begin{itemize}
    \item The unit operator on the Hilbert space can be expanded as
    \begin{equation}
      1_\mathcal{H} = \int e^{- \overline{\eta}{}_a \eta^a} \ket{\eta} \bra{\overline{\eta}{}} \dd[]{\eta}
    \end{equation}
    \item If $\hat{A} \colon \mathcal{H}_B \oplus \mathcal{H}_F \to \mathcal{H}_B \oplus \mathcal{H}_F$ , then the supertrace is
      \begin{equation}
	\text{Str}_{\mathcal{H}}(\hat{A}) \coloneqq \Tr_{\mathcal{H}_B}(\hat{A}) - \Tr_{\mathcal{H}_F}(\hat{A}) = \int e^{- \overline{\eta}{}_a \eta^a} \bra{\overline{\eta}{}} \hat{A} \ket{\eta} \dd[2]{\eta},
      \end{equation}
      whereas the normal trace is
      \begin{equation}
	Tr_{\mathcal{H}_B}(\hat{A}) + \Tr_{\mathcal{H}_F}(\hat{A}) = \int e^{-\overline{\eta}{}_a \eta^a} \bra{-\overline{\eta}{}} \hat{A} \ket{\eta} \dd[2]{\eta},
      \end{equation}
      where $\bra{\overline{\eta}{}} = \bra{0} \exp(- \hat{\psi}^a \overline{\eta}{}_a)$ is conjugate to $\ket{\eta}$.
  \end{itemize}
\end{claim}
\begin{proof}
  Exercise.
\end{proof}
\begin{exercise}
  Check signs!
\end{exercise}

\section{Path Integral for Fermions}%
\label{sec:path_integral_for_fermions}

With the action $S = \int i \overline{\psi}{}_a \dot{\psi}^a - V(\overline{\psi}{}, \psi) \dd[]{\tau}$, then $H = V(\overline{\psi}{}, \psi)$ . 
Taking anticommutators if needed, we can choose to bring the $\hat{\psi}$  operators to the right of all $\hat{\overline{\psi}{}}$  operators in $\hat{H}$ .

Let $\ket{\chi}$ , $\ket{\chi'}$  be fermionic coherent states. As before, we define the heat kernel, the propagator between $\ket{\chi}$  and $\ket{\chi'}$  to be 
\begin{align}
  \bra{\overline{\chi}{}'} e^{-\beta H} \ket{\chi} &= \bra{\overline{\chi}{}'} e^{- \Delta \tau H} e^{- \Delta \tau H} \dots e^{- \Delta \tau H} \ket{\chi} \\
  &= \bra{\overline{\chi}{}'} e^{-\Delta \tau H} \ket{\eta_{N-1}} \dots \bra{\overline{\eta}{}_2} e^{-\Delta \tau H} \ket{\eta_1} \bra{\overline{\eta}{}_1} e^{- \Delta \tau H} \ket{\chi} \prod_{i = 1}^{N-1} e^{- \overline{\eta}{}_i \eta_i} \dd[2n]{\eta_i}.
\end{align}
We have for very small time intervals, $\Delta \tau$ infinitesimal, 
\begin{equation}
  \bra{\overline{\eta}{}_{i+1}} e^{- \Delta \tau H (\hat{\overline{\psi}{}}, \hat{\psi})} \ket{\eta_{i}} = e^{-\Delta \tau H(\overline{\eta}{}_{i + 1}, \eta_i)} \bra{\overline{\eta}{}_{i + i}} \ket{\eta_i} = e^{-\Delta \tau H(\overline{\eta}{}_{i+1}, \eta_i)} e^{\overline{\eta}{}_{i + 1} \eta_i}
\end{equation}
where we just took the linear term from the right hand side. The normal ordered Hamiltonian can then be replaced (to linear order) by the value on its eigenfunctions.

Thus 
\begin{align}
  \bra{\overline{\chi}{}'} e^{- \beta H} \ket{\chi} &= \lim_{N \to \infty} \int \exp(\sum_{i=1}^{N} \overline{\eta}{}_i \eta_{i-1} - \Delta \tau V(\overline{\eta}{}_i, \eta_{i-1}))
  \prod_{i = 1}^{N-1} e^{- \overline{\eta}{}_i \eta_i} \dd[2n]{\eta_i} \\
				       &= \lim_{N \to \infty}  \int \exp(-\sum_{i=1}^{N} \left[ \overline{\eta}{}_i \frac{\eta_i - \eta_{i-1}}{\Delta \tau} - V(\overline{\eta}{}_i, \eta_{i-1}) \right]\Delta \tau) e^{\overline{\eta}{}_N \eta_N} \prod_{i = 1}^{N-1} \dd[2n]{\eta_i},
\end{align}
where $\eta_0 \coloneqq \chi$ and $\eta_N \coloneqq\chi'$. The argument of $\exp(\dots)$ is a discretised version of the Euclidean action
\begin{equation}
  S[\eta, \overline{\eta}{}] = \int_0^{\beta} \left[ \overline{\eta}{} \dot{\eta} + V(\overline{\eta}{}, \eta) \right] \dd[]{\tau}.
\end{equation}

Thus, formally
\begin{equation}
  \bra{\overline{\chi}{}'} e^{-\beta H} \ket{\chi} = \int e^{-S[\overline{\psi}{}, \psi]} e^{\overline{\psi}{}(\beta) \psi(\beta)} \pdd{\psi} \pdd{\overline{\psi}{}},
\end{equation}
where $\psi(0) = \chi$  and $\psi(\beta) = \chi'$ .

To construct the partition function, we must first choose whether $\psi$  is periodic, meaning $\psi(\tau + \beta) = \psi(\tau)$ , or antiperiodic $\psi(\tau + \beta) = -\psi(\tau)$ . The antiperiodic version is allowed since each term in the action $S[\overline{\psi}{}, \psi]$ must contain an even number of fermions.

Let us look at both of these cases separately.
\begin{description}
  \item[periodic] This gives the supertrace, since
    \begin{equation}
      \text{Str}_{\mathcal{H}}(e^{-\beta H}) = \bra{\overline \chi} e^{- \beta H} \ket{\chi} e^{- \overline{\chi}{} \chi} \dd[2n]{\chi} = \int_{\mathrlap{\text{periodic}}} e^{-S[\overline{\psi}{}, \psi]} \pdd{\psi} \pdd{\overline{\psi}{}}.
    \end{equation}
  \item[antiperiodic] We now obtain the trace, since
    \begin{equation}
      \Tr(e^{-\beta H}) = \int \bra{- \overline{\chi}{}} e^{- \beta H} \ket{\chi} e^{- \overline{\chi}{} \chi} \dd[2n]{\chi} = \int_{\mathrlap{\text{antiperiodic}}} e^{-S[\overline{\psi}{}, \psi]} \pdd{\psi} \pdd{\overline{\psi}{}}.
    \end{equation}
\end{description}
\begin{remark}
  Often, the `usual' trace is called the \emph{partition function} of the theory, whereas the supertrace is (in SUSY theories) known as the \emph{Witten index}.
\end{remark}

\section{Supersymmetric Quantum Mechanics (SQM)}%
\label{sec:supersymmetric_quantum_mechanics_sqm_}

There are two different (closely related) types, depending on whether we have complex fermions ($\mathcal{N} = 2$ SQM) or real fermions ($\mathcal{N} = 1$ SQM).

For complex $\mathbb{C}$-fermions, the simplest action is
\begin{equation}
  S[x, \overline{\psi}{}, \psi] = \int \frac{1}{2} \delta_{ab} \dot{x}^a \dot{x}^b + i \delta_{ab} \overline{\psi}{}^a \dot{\psi}^b \dd[]{t},
\end{equation}
where we have Minkowski time on the worldline.
We have canonical momenta $p_a = \delta_{ab} \dot{x}^b$ and $\pi = i \overline{\psi}{}_a$.
Quantising this theory leads to canonical commutation relations for the bosons and anticommutation relations for the fermions, 
\begin{equation}
  [\hat{p}_a, \hat{x}^b] = -i \delta^b_a, \qquad \{\hat{\overline{\psi}{}}_a, \hat{\psi}^b\} = \delta_a^b,
\end{equation}
just as it did for the two theories separately.

It is natural to take the Hilbert space to be $\mathcal{H} = \Omega^*(\mathbb{R}^n, \mathbb{C})$ the space of all ($\mathbb{C}$-valued) forms on $\mathbb{R}^n$. Explicitly, if $\hat{\psi}^a \ket{0} = 0$  for all $a = 1, \dots, n$ , then
\begin{equation}
  f^{ab \dots c} (x) \hat{\overline{\psi}{}}{}^a \hat{\overline{\psi}{}}{}^b \dots \hat{\overline{\psi}{}}{}^c \ket{0} \leftrightarrow f(x) = f_{ab \dots c} (x) dx^a \wedge dx^b \wedge \dots \wedge dx^c.
\end{equation}
If $f$ and $g$ are forms of the same degree, then we have
\begin{equation}
  (f, g) = 
  \int_{\mathbb{R}^n} \overline{f}{} \wedge \star g = \int_{\mathbb{R}^n} \overline{f^{ab \dots c} (x)}{} g_{ab \dots c} (x) \dd[n]{x}.
\end{equation}
If their degree differs, then $(f, g) = 0$.
We also require the norm to be finite $ \norm{f}^2 = (f, f) < \infty$.

The $\mathcal{N} = 2$ SQM action is invariant under SUSY transformations
\begin{equation}
  \delta x^a = \epsilon \overline{\psi}{}^a - \overline{\epsilon}{} \psi^a, \qquad \delta \psi^a = i \epsilon \dot{x}^a, \qquad \delta \overline{\psi}{}^a = -i \overline{\epsilon}{} \dot{x}^a,
\end{equation}
where $\epsilon, \overline{\epsilon}{}$  are constant Grassmann parameters. Unlike in $d = 0$ , here we have Noether charges $Q$  and $\overline{Q}{}$, which generate these (by Poisson brackets).
\begin{equation}
  Q = +i \delta_{ab} \dot{x}^a \overline{\psi}{}^b, \qquad
  \overline{Q}{} = -i \delta_{ab} \dot{x}^a {\psi}{}^b,
\end{equation}
where $\delta = \overline{\epsilon}{} Q + \epsilon \overline{Q}{}$ .

As usual, we have a Hamiltonian
\begin{equation}
  H = p_a \dot{x}^a + \pi_a \dot{\psi}^a - L = \frac{1}{2} \delta^{ab} p_a p_b.
\end{equation}
\begin{claim}
  The Poisson brackets of our charges turn out to be
  \begin{equation}
    \{ Q, \overline{Q}{} \}_{\text{PB}} = -2i H,
  \end{equation}
  realising the SUSY algebra in $d = 1$.
\end{claim} 

Representing the Hilbert space in terms of polyforms $\mathscr{H} \simeq \Omega^* (\mathbb{R}^n, \mathbb{C})$, then the supercharge becomes
\begin{equation}
  \hat{Q} = i \hat{p}_a \hat{\overline{\psi}}{}^a \rightarrow dx^a \wedge \frac{\partial }{\partial x^a} = d,
\end{equation}
the exterior derivative.
The interpretation of $\hat{\overline{Q}{}}$ is slightly more subtle.
