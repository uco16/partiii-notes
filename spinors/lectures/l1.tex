% lecture notes by Umut Özer
% course: spinors
\lhead{Lecture 1: January 18}

\begin{shadequote}
  No one fully understands spinors. Their algebra is formally understood but their general significance is mysterious. In some sense they describe the square root of geometry and just as understanding the square root of $(-1)$ took centuries, the same may be true of spinors.\par\emph{Michael Atiyah}
\end{shadequote}

\begin{shadequote}
  The existence of spinor structure appears on physical grounds to be a reasonable condition to impose on ANY cosmological model in GR.\par\emph{Robert Geroch}
\end{shadequote}

\subsection*{References}%

\begin{itemize}
  \item David Tong --- QFT
  \item John Stewart --- Ad.~GR., CUP 2008
  \item Penrose \& Rindler --- Spinors \& Spacetime, Vol 1 (\& 2), CUP 1987
  \item \'Elie Cartan --- Theory of Spinors, Dover 2003
  \item Peter O'Donnell --- Introduction to $2$-spinors in GR, World Scientific 2003
\end{itemize}

\section{Spin Structures}%
\label{sec:spin_structures}

In differential geometry, we can define a \emph{spin structure} on an orientable Riemannian manifold $(M, g)$ in terms of associated spinor bundles. This will allow us to give a definition of spinors in differential geometry.

Spin structures have applications in mathematical physics---most notably QFT---and are necessarily present in any theory with fermions.

They are also of purely mathematical interest in areas such as differential geometry, algebraic topology, K theory, and spin geometry.

\subsection*{Some applications}%

\begin{enumerate}[1)]
  \item Three dimensions: non-relativistic electrons and other fermions
  \item Dirac spinors \& the Dirac equation $\rightarrow$ quantum states of relativistic electrons
  \item QFT $\rightarrow$ many-particle systems
  \item Applications to symplectic geometry, gauge theory, and complex algebraic geometry.
\end{enumerate}

\subsection*{Examples}%

These are possible topics that, depending on the interest in the class, we might cover during this course:
\begin{enumerate}[1)]
  \item Petrov classification: spacetime, $SL(2, \mathbb{C})$ spinors.\par This can generalise to higher dimensions (still an open research topic).
  \item Particle physics --- intrinsic angular momentum ``spin''.\par This again generalises to higher dimensions.
  \item Curved spacetime (CST) $\rightarrow$ null and causal structures [null congruences $\rightarrow$ classify]
  \item Zero rest mass (ZRM) free fields for any spin
  \item Can build a spacelike foliation of spacetime using $SU(2)$ spinors and then use this to quantise spin-$\frac{3}{2}$ ZRM on a curved background.
\end{enumerate}

Other examples include  supersymmetry, quantum gravity, solutions to the gravitational field equations, $SU(2)$  spinors in Witten gravitational energy, topological massive gauge fields, soldiering forms due to Ashtekar, \dots

\section{One of a number of possible definitions}%
\label{sec:one_of_a_number_of_possible_definitions}

There are many possible definitions of spinors. We can define them as elements of a representation space of the spin group (or its Lie algebra).
Spinors also arise as elements of a vector space that carries a linear representation of the Clifford algebra.

Both the spin group and its Lie algebra are embedded inside the Clifford algebra in a natural way.

Sometimes the Clifford algebra is easier to work with.

Start by choosing an orthogonal basis. Then choose ``gamma matrices'', which generate the Clifford algebra.
The $\Gamma$ 's have to satisfy anti-commutation relations for the Clifford algebra. The spinors are column vectors on which the matrices act.

\begin{example}[Euclidean 3 $d$]
  The Pauli spin matrices $\sigma^{a}$  are a set of gamma matrices. The associated spinors are $2$ -component complex column vectors.
\end{example}

The particular matrix representations of the Clifford algebra, and what we mean by `spinors' involves a choice of \emph{basis} and a choice of gamma matrices.

As a representation of the spin group the realisation of spinors (as complex column vectors) might be irreducible.
Otherwise, we will decompose it into spin-$\frac{1}{2}$  Weyl representations of the dimension is even.

\section{Clifford Algebras}%
\label{sec:clifford_algebras}

\begin{definition}[$R$-algebra]
  Let $R$ be a commutative ring. An $R$-algebra is an additive Abelian group $A$ with the structure of a ring and also $R$-module such that the ring multiplication is $\mathbb{R}$-linear:
  \begin{equation}
    r \cdot (xy) = (r \cdot x) y = x (r \cdot y).
  \end{equation}
  $A$ is called \emph{unital} if the exists a unit element $1 \cdot x = x = x \cdot 1$.
\end{definition}

\begin{definition}[Clifford Algebra]
  A Clifford algebra $Cl(V, Q)$ is a unital associative alebra, generated by a vector space $V$ that comes with a quadratic form $Q$.

  ``Generated by $V$'' means that there is a linear map $i \colon V \to Cl(V, Q)$ such that $i(v)^2 = Q(V) \cdot 1$ for all $v \in V$.

  $V$ is defined by the following universal property: For any associative algebra $A$ and any linear map $j\colon V \to A$ such that $j(v)^2 = Q(v) 1_A$, there is a unique algebra homomorphism $f\colon Cl(V, Q) \to A$ such that the following diagram commutes:
  \begin{equation}
    \begin{tikzcd}
      V \arrow{r}{i} \arrow[swap]{dr}{j = f \circ i} & Cl(V, Q) \arrow{d}{f} \\
		     & A
    \end{tikzcd}
  \end{equation}
\end{definition}
