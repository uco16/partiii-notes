% lecture notes by Umut Özer
% course: sm
\lhead{Lecture 1: January 18}

\chapter{Introduction and History}%
\label{cha:introduction_and_history}

\section*{Prerequisites}%
\label{sec:prerequisites}

It is necessary to have attended the \emph{Quantum Field Theory} and the \emph{Symmetries, Fields, and Particles} courses, or to be familiar with the material covered in them. It would be advantageous to attend the \emph{Advanced Quantum Field Theory} course during the same term as this course, or to study renormalisation and non-abelian gauge fixing.

\section{Introduction}%
\label{sec:introduction}

\begin{definition}[standard model]
  A theoretical physics construction (theory, model) that describes all known elementary particles and their interactions based on relativistic quantum field theory (QFT).
\end{definition}

The Standard Model of particle physics is the most successful application of QFT we currently have. Based on the gauge group $SU(3) \times SU(2) \times U(1)$, it accurately describes, at the time of writing, all experimental measurements involving strong, weak, and electromagnetic interactions.

\subsection*{Ingredients}%

\begin{enumerate}[(i)]
  \item spacetime: $3 + 1$ dimensional Minkowski space\par
    symmetry: Poincar\'e group
  \item particles: 
    \begin{description}
      \item[spin $s = 0$] Higgs
      \item[spin $s = 1 / 2$] three families of quarks and leptons
    \end{description}
  \item interactions:
    \begin{description}
      \item[$s = 1$] three gauge interactions
      \item[$s = 1$] gravity\footnote{as important as it is, we will not be concerned with gravity for most of this course}
    \end{description}
    Gauge (local) symmetry: $SU(3)_C \times SU(2)_L \times U(1)_Y \xrightarrow[Breaking]{Symmetry} SU(3)_C \times U(1)_{EM}$
    \begin{description}
      \item[C] color: strong
      \item[L] left: electroweak
      \item[Y] hypercharge
    \end{description}
    These are related via $Q = T_3 + Y$.\par
    %F1 spontaneous symmetry breaking, mexican hat potential
\end{enumerate}

Particle representations\footnote{numbers tell us representations under $(C, L; Y)$}:
\begin{itemize}
  \item Quarks and Leptons: $\overbrace{3}^{\mathclap{\text{families (flavour)}}}[\underbrace{(3, 2; \frac{1}{6})}_{\mathclap{Q_L}} + \underbrace{(\overline{3}, 1; -\frac{2}{3})}_{\mathclap{U_R}} + \underbrace{(\overline{3}, 1; \frac{1}{3})}_{\mathclap{d_R}} + \underbrace{(1, 2; -\frac{1}{2})}_{\mathclap{L_L}} 
    + \underbrace{(1, 1; 1)}_{\mathclap{e_R}} + \underbrace{(1, 1; 0)}_{\mathclap{\nu_R}}]$ 
  \item Higgs: $(1, 2; -\frac{1}{2})$ 
  \item Gauge: $\underbrace{(8, 1; 0)}_{\mathclap{gluons}} + \underbrace{(1, 3; 0)}_{\mathclap{W^{\pm}, \mathbb{Z}}} + \underbrace{(1, 1; 0)}_{\mathclap{\gamma}}$ 
\end{itemize}

\subsection*{Comments}%

\begin{itemize}
  \item interactions given by QFT
  \item main tool: symmetry
  \item total symmetry: spacetime $\otimes$ internal (gauge)\footnote{Theorem: cannot mix these two symmetries. Supersymmetry provides a way around this.}
  \item also accidental (global) symmetries $\sim$ baryon + lepton number
  \item plus approximate (flavour) symmetries:
  \item very rigid: $\sum Y = \sum Y^3 = 0$\footnote{gravitational anomaly}, $\#3 = \# \overline{3}$, $\# 2$ even
  \item rich structure (3 phases: Coulomb, Higgs, confining)
\end{itemize}

\subsection*{Motivation}%

Why to learn about the SM?
\begin{itemize}
  \item It is fundamental.
  \item It is based on elegant principles of symmetry.
  \item It is true!
    \begin{itemize}
      \item outstanding predictions: ($\mathbb{Z}^0, W^{\pm}$, Higgs, \dots)
      \item precision tests: \par
	anomalous magnetic dipole moment of the electron: \begin{equation}a = \frac{g - 2}{2} = (1159.65218091 \pm 0.00000026) \times 10^{-6}\end{equation} 
	fine structure constant (at $E \ll 10^3$GeV):
	\begin{equation}
	  \alpha^{-1} = \frac{\hbar c}{e^2} = 137.035999084(21)
	\end{equation}
    \end{itemize}
  \item It is the best test of QFT.
  \item It is incomplete!
\end{itemize}
