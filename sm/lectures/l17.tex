% lecture notes by Umut Özer
% course: sm
\lhead{Lecture 17: February 25}

The Abelian Higgs model describes the effective field theory of superconductivity.
In this case, $\langle \phi \rangle \sim e^- e^-$ is the Cooper pair and $A^{\mu} A_{\mu}$ gives a magnetic field, which is energetically disfavoured.  This is the Meissner effect.
The penetration depth $R = \frac{1}{m_A}$ coincides with the mass of the gauge field.
In the Yukawa case, it tells us how deeply the magnetic field enters into the superconductor.
Moreover, we have vortex lines with correlation length $\xi = \frac{1}{m_{\eta}}$ and $\xi >R$ gives type I, whereas $\xi < R$ gives type II, where you can have vortices.
Superconductivity is an example where you break to a discrete symmetry.

Since we map from a circular vacuum to three dimensional space, we have what is called \emph{cosmic strings}:
\begin{figure}[ht]
    \centering
    \inkfig[0.4]{cosmic-strings}
    \caption{Cosmic Strings}
    \label{fig:cosmic-strings}
\end{figure}

\section{Anomalies}%
\label{sec:anomalies}

\begin{leftbar}
  Non-examinable.
\end{leftbar}

We have finished the spontaneous symmetry breaking part but for completeness we should mention another mechanism of symmetry breaking.
\begin{definition}[anomalies]
  \emph{Anomalies} appear whenever you have classical symmetries, which are broken by quantisation.
\end{definition}
In quantum theory, we have much for than a Lagrangian. It might be possible that the symmetries of the Lagrangian might disappear upon quantisation.
Recall that in quantum field theory, we work with the path integral formulation of quantum mechanics
\begin{equation}
  \int \pdd{\phi} e^{i S[\phi]}.
\end{equation}
Consider a classical symmetry $\phi \to \phi'$, defined by the fact that the action transforms as $S \to S$.
But the path integral measure also needs to be invariant for the physics to stay the same! The measure transforms as
\begin{equation}
  \pdd{\phi} \to \pdd{\phi'} \mathcal{J}.
\end{equation}
If the Jacobian $\mathcal{J} \neq 0$, then the symmetry is broken.
The presence of anomalies has two different effects depending on the type of symmetry.
In a global symmetry, the symmetry is simply broken and nothing bad happens.
However, in the case of local (gauge) symmetries, anomalies are a killer: they imply that the theory is inconsistent.
Every single gauge theory that we deal with has to be free of anomalies.
It can be shown that chiral theories are automatically anomaly-free.
This subject is very broad and deep, and we will not go into too much detail here, but it is encouraged to read up more on anomalies.

\subsection{Anomalies in QED}%
\label{sub:anomalies_in_qed}

Consider for instance QED:
\begin{equation}
  \mathscr{L} = -\frac{1}{4} F^{\mu\nu} F_{\mu\nu} + \overline{\psi}{}_L (i \cancel{\partial} - e \cancel{A}) \psi_L + \overline{\psi}{}_R (i \cancel{\partial} - e \cancel{A})\psi_R - m \overline{\psi}{}_L \psi_R - m \overline{\psi}{}_R \psi_L.
\end{equation}
For the massless limit $m \to 0$, we have two symmetries:
\begin{align}
  \psi &\to e^{i \alpha} \psi, & \psi &\to e^{i \beta \gamma_5} \psi \label{eq:17-sym} \\
  \psi_L & \to e^{i (\alpha + \beta)} \psi_L, & \psi_R &\to e^{i(\alpha + \beta)} \psi_R.
\end{align}
We can then find conserved currents
\begin{equation}
  J_V^{\mu} = \overline{\psi}{}\gamma^{\mu} \psi \qquad J_A^{\mu} = \overline{\psi}{} \gamma^{\mu} \gamma^5 \psi,
\end{equation}
with $\partial^{\mu} J_{\mu} = \partial^{\mu} J_{\mu 5} = 0$.
We call $J_V^{\mu}$ the \emph{vector current} and $J_A^{\mu}$ the \emph{axial current}.
For the massive case $m \neq 0$, 
\begin{equation}
  \partial_{\mu} J_V^{\mu} = 0, \qquad \partial_{\mu} J_A^{\mu} = 2 i m \overline{\psi}{} \gamma^5 \psi,
\end{equation}
only the vector current is conserved.

We will now see that, even in the massless limit, the axial current conservation will be broken.  Consider the following \emph{triangle diagrams}
\begin{equation}
  \begin{gathered}
    \feynmandiagram[transform shape, scale=1][horizontal=a to b] {
      a -- [boson, momentum=$q$] b -- [fermion, momentum'=$l + p$] u -- [boson, momentum=$p$] ru [particle=\(\nu\)],
      b -- [fermion, rmomentum=$l - k$] d -- [boson, momentum=$k$] rd [particle=\(\lambda\)],
      u -- [fermion, momentum'=$l$] d,
      rd -- [draw=none] ru,
    };
  \end{gathered}
  \quad + \qquad
  \begin{gathered}
    \feynmandiagram[transform shape, scale=1][horizontal=a to b] {
      a -- [boson, momentum=$q$] b -- [fermion, momentum'=$l + k$] u -- [boson, momentum=$k$] ru [particle=\(\lambda\)],
      b -- [fermion, rmomentum=$l - p$] d -- [boson, momentum=$p$] rd [particle=\(\nu\)],
      u -- [fermion, momentum'=$l$] d,
      rd -- [draw=none] ru,
    };
  \end{gathered}
\end{equation}
These kinds of diagrams created a big problem in the community, since people did not know how to deal with it.
They are the source of the anomalies.
\begin{equation}
  \int \dd[4]{x} e^{i q \cdot x} \bra{p, k} J^{\mu}_A(x) \ket{0} = \bdelta^4(p + k - q) \epsilon_{\nu}(p) \epsilon_{\lambda}(k) \mathcal{M}^{\mu\nu\lambda} (p, k).
\end{equation}
Computing $\partial_{\mu} J^{\mu}_{A}$ is equivalent to $q_{\mu} \mathcal{M}^{\mu\nu\lambda}$.
We want to check whether the Ward identity is satisfied:
\begin{equation}
  q_{\mu} M^{\mu\nu\lambda} \epsilon_{\nu} \epsilon_{\lambda} \stackrel{?}{=} 0.
\end{equation}
We observe that both cannot vanish since
\begin{align}
  \bra{p, k} \partial_{\mu} J^{\mu 5} \ket{0} &= -\frac{e^2}{16 \pi^2} \epsilon^{\mu\nu\alpha\beta} (-i p_{\mu}) \epsilon_{\nu} (p) (-i k_{\alpha}) \epsilon_{\beta} (k) \\
  &= -\frac{e^2}{2\pi^2} \bra{p, k} e^{\alpha\lambda\beta\nu} F_{\alpha\beta} F_{\lambda\nu} \ket{0}. \\
  \implies \partial_{\mu} J_A^{\mu} &= -\frac{e^2}{16 \pi ^2} \epsilon^{\mu\nu\alpha\beta} F_{\mu\nu} F_{\alpha\beta}.
\end{align}
This is caled the \emph{Adler--Bell--Jackiw anomaly}.
To obtain this, there is an integral
\begin{equation}
  \int_{-\infty}^{\infty} \left( f(x + a) - f(x) \right) \dd[]{x}.
\end{equation}
Naively, these seem to cancel each other, giving zero.
However, if we expand the first term in a Taylor series, we obtain 
\begin{equation}
  \int \left( f'(x)a + \dots \right) \dd[]{x} = f(\infty)- f(-\infty),
\end{equation}
where $f(\infty), f(-\infty)$ are constants that do not cancel each other.

For the path integral,
\begin{equation}
  \int \pdd{\psi} \pdd{\overline{\psi}{}} \pdd{A} e^{i \int \dd[4]{x} \left( -\frac{1}{4} F_{\mu\nu}^2 + i \overline{\psi}{} \cancel{D} \psi \right)}.
\end{equation}
Under the transformations \eqref{eq:17-sym}, of the form
\begin{equation}
  \psi \to \Delta \psi, \qquad \overline{\psi}{} \to \Delta_c \overline{\psi}{},
\end{equation}
we have $S \to S$, but the measure transforms as
\begin{align}
  \pdd{\overline{\psi}{}} \pdd{\psi} &\to (\mathcal{J}_c \mathcal{J})^{-1} \pdd{\overline{\psi}{}} \pdd{\psi}, \\
				     &= e^{\Tr \ln \Delta} = e^{\int \dd[4]{x} \bra{x} \Tr \ln \Delta (x) \ket{x}}
\end{align}
where $\mathcal{J} = \det \Delta$, $\mathcal{J}_c = \det \Delta_c$.
For $\Delta = e ^ i \beta \gamma_5$, we have
\begin{equation}
  \mathcal{J} = e^{-i \int \dd[4]{x} \left( \beta \frac{e^{2}}{32 \pi^2} \epsilon^{\mu\nu\alpha\beta} F_{\mu\nu} F_{\alpha\beta} \right)}.
\end{equation}
The result which we obtain from this:
\begin{equation}
  \partial_{\mu} \langle J^{\mu 5} \mathcal{O}(x, \dots) \rangle = -\frac{e^2}{16 \pi^2} \langle \epsilon^{\mu\nu\alpha\beta} F_{\mu\nu} F_{\alpha\beta} \mathcal{O}(x, \dots) \rangle
\end{equation}
is in fact true to all loop orders.

\subsection{Anomalies in Yang--Mills Theory}%
\label{sub:anomalies_in_yang_mills_theory}

In the non-Abelian case, the corresponding diagrams contribute
\begin{equation}
  \Tr(T^{a} T^{b} T^{c}) + \Tr(T^{a} T^{c} T^{b})
\end{equation}
We define
\begin{equation}
  A^{abc} = 2 \Tr[T^{a}_R \{T^{b}_R, T^{c}_R\}] = A(R) d^{abc},
\end{equation}
where the subscript $R$ denotes a particular fermion representation and $d$ the fundamental.
Then the axial current is
\begin{equation}
  \partial_{\mu} (J_A^{a})^{\mu} = \left( \sum_{\text{left}} A(R_l) - \sum_{\text{right}} A(R_r) \right) \frac{g^2}{128 \pi^2} d^{abc} \epsilon^{\mu\nu\alpha\beta} F^{b}_{\mu\nu} F^{c}_{\alpha\beta}
\end{equation}
The key thing to compute is the quantity in parenthesis.
If it is equal to zero, the theory is safe. If not, the theory is dead.
This is a test of consistency for the theory.
In general, the important thing is the difference between left- and right-handed spinors.
In a chiral theory, in which left and right handed representations are the same, the anomalies will cancel automatically and the theory is consistent.
For $U(1)^3$,
\begin{equation}
  \begin{gathered}
    \feynmandiagram[transform shape, scale=1][horizontal=a to b] {
      a [particle=\(\gamma\)] -- [boson] b -- [fermion] c -- [fermion] d -- [fermion] b,
      c -- [boson] e [particle=\(\gamma\)],
      d -- [boson] f [particle=\(\gamma\)],
      e -- [draw=none] f,
    };
  \end{gathered}
  \quad \sim \quad A \propto \Tr(Q^3)
  \begin{cases}
    = 0 , & \text{consistent} \\
    \neq 0, & \text{inconsistent}  
  \end{cases}
\end{equation}

\subsection*{Gravity}%

Finally, we can consider gravity.
\begin{equation}
  \begin{gathered}
    \feynmandiagram[transform shape, scale=1][horizontal=a to b] {
      a [particle=\(\gamma\)] -- [boson] b -- [fermion] c -- [fermion] d -- [fermion] b,
      c -- [gluon] e [particle=\(h\)],
      d -- [gluon] f [particle=\(h\)],
      e -- [draw=none] f,
    };
  \end{gathered}
  \quad ~ \quad A \propto \Tr(Q)
  \begin{cases}
    = 0 , & \text{if consistent} \\
    \neq 0, & \text{if inconsistent}  
  \end{cases},
\end{equation}
where $h$ denotes a graviton.
