% lecture notes by Umut Özer
% course: sm
\lhead{Lecture 7: February 01}

\section{Unitary Representations of the Poincaré Group}%
\label{sec:unitary_representations_of_the_poincare_group}

This is due to Wigner in 1939.

\subsection{Rotation Group}%
\label{sub:rotation_group}

We already know how to find the representations of the rotation group in $3D$  $SO(3) \simeq SU(2)$ .
Its generators $J_i$  obey $[J_i, J_j] = i \epsilon_{ijk} J_k$ .
\begin{definition}[Casimir operator]
  The \emph{Casimir operator} is
  \begin{equation}
    J^2 = J_1^2 + J^2_2 + J^2_3, \qquad [J^2, J_i] = 0.
  \end{equation}
\end{definition}
The representations or multiplets of the rotation group, which are a collection of states, are labelled by the eigenvalues of the Casimir operators
\begin{equation}
  J^2 \ket{j} = j (j + 1) \ket{j}, \qquad j = 0, \frac{1}{2}, 1, \dots
\end{equation}
Within one representation, we can diagonalise one of the $J_i$  generators, for instance $J_3$ . Its eigenvalues are $j_3$. Acting on any given state, labelled by $j_3$, in the representation that is labelled by $j$, we have
 \begin{equation}
  J_3 \ket{j, j_3} = j_3 \ket{j, j_3}, \qquad j_3 = -j, -j+1, \dots, +j.
\end{equation}

\begin{remark}
  For a compact group, the number of Casimir operators is the rank of the group, but for non-compact group we have to try by hand.
\end{remark}

\subsection{Poincaré Group}%
\label{sub:poincare_group}

We follow exactly the same steps for the Poincaré group.
The generators are $P^{\mu}$ and $M^{\mu\nu}$, obeying the commutation relations \eqref{eq:pmcom} and \eqref{eq:mcom}.
\begin{definition}[Casimir operator]
  The \emph{Casimir operators} for the Poincaré group are
  \begin{equation}
    C_1 = P^{\mu} P_{\mu} \qquad C_2 = W^{\mu} W_{\mu}.
  \end{equation}
\end{definition}
\begin{remark}
  Note that we now have two Casimirs, so the representations will be indicated by two labels.
\end{remark}
\begin{claim}
  The Casimir operators commute with the Poincaré algebra's generators:
  \begin{equation}
    [C_{1, 2}, P^{\mu}] = 0 = [C_{1, 2}, M^{\mu\nu}].
  \end{equation}
\end{claim}
\begin{exercise}
  Prove this!
\end{exercise}
\begin{definition}[Pauli--Ljubanski vector]
  The \emph{Pauli--Ljubanski vector} is
  \begin{equation}
    W_M \coloneqq \frac{1}{2} \epsilon_{\mu\nu\rho\sigma} P^{\nu} M^{\rho\sigma}.
  \end{equation}
\end{definition}
\begin{claim}
  \label{claim:plv-pm}
  The Pauli--Ljubanski vector and the Poincaré generators have the following commutation relations
  \begin{equation}
    [W_{\mu}, P_{\nu}] = 0, \qquad
    [W_{\mu}, M_{\rho\sigma}] = i (\eta_{\mu\rho} W_{\sigma} - \eta_{\mu\sigma} W_{\rho}).
  \end{equation}
\end{claim}
\begin{claim}
  Using Claim~\ref{claim:plv-pm}, one can show
  \begin{equation}
    [W_{\mu}, W_{\nu}] = -i \epsilon_{\mu\nu\rho\sigma} W^{\rho} P^{\sigma},
  \end{equation}
  so the $W_{\mu}$ do not form an algebra, except if the $P^{\mu}$ are fixed!
\end{claim}

As before, we label representations of the Poincaré algebra by eigenvalues $C_1$ and $C_2$
\begin{equation}
  \ket{C_1, C_2}.
\end{equation}
Within an irreducible representation (irrep), we now need to pick a subset of generators that can be simultaneously diagonalised. Let us pick the $P^{\mu}$, with eigenvalue $p^{\mu}$, since they already commute with each other.
Since $C_1 = P^{\mu} P_{\mu}$, the eigenvalues $p^{\mu} p_{\mu}$ can be bigger than, equal to, or less than zero.
\begin{description}
  \item[$p^{\mu} p_{\mu}> 0$] In other words, $p^{\mu}$ is timelike. We can choose coordinates in which $p^{\mu} = (m, 0,0,0)$.
    It is immediately obvious that this vector is invariant under rotations $J_i$. We say that $SO(3)$ is the \emph{little group}.
    Now $C_1 = P^{\mu} P_{\mu} = m^2$ and since $W_{\mu} = (0, -m J_i)$, we have $C_2 = m^2 J^2$.
    The multiplet is then labelled by
    \begin{equation}
      \ket{c_1, c_2; p^{\mu}, j_3} = \ket{m, j; p^{\mu}, j_3}.
    \end{equation}
    These are \emph{massive one-particle states}!
    \begin{remark}
      We cannot overemphasise the importance of this result: elementary particles are irreducible representations of the Poincaré algebra.
    \end{remark}
  \item[$P^{\mu} P_{\mu} = 0$] We can find a frame in which $p^{\mu} = (E, 0, 0, E)$ and have $C_1 = 0$.
    The $W_{\mu}$ are given by
    \begin{equation}
      (W_0, W_1, W_2, W_3) = E(J_3, -J_1 + K^2, -J_2 -K_1, -J_3).
    \end{equation}
    The commutation relations are
    \begin{equation}
      [W_1, W_2] = 0, \qquad [W_3, W_1] = -i E W_2, \qquad [W_3, W_2]= i E W_1.
    \end{equation}
    The little group is the $2D$ Euclidean group $E_2$, which has infinite-dimensional representations. However, the corresponding particles have never been observed.
    \begin{remark}
      So far, we do not have a satisfactory explanation of why this particle should not be there.
      Weinberg gives a phenomenological explanation, whereas Wigner gave an explanation in terms of the heat capacity. The lecturer proposed an explanation in terms of string theory.
    \end{remark}
    Set $W_1 = W_2 = 0$. Then $W_3$ is generating $SO(2)$, rotations around $x_3$.
    We have
    \begin{equation}
      W_{\mu} = E J_3 (1, 0, 0, -1) \propto p_{\mu}.
    \end{equation}
    Finally, the second Casimir operator is $C_2 = W^{\mu} W_{\mu} = 0$.
    The irreducible representation is labelled
    \begin{equation}
      \ket{0, 0; p^{\mu}, \lambda} \coloneqq \ket{p^{\mu}, \lambda}, 
    \end{equation}
    where $\lambda = 0, \pm \frac{1}{2}, \pm 1, \dots$ is the eigenvalue of $J_3$, which is called \emph{helicity}
    \begin{equation}
      e^{2 \pi i \lambda} \ket{p^{\mu}, \lambda} = \pm \ket{p^{\mu}, \lambda}.
    \end{equation}
    \begin{remark}
      We will identify these with the Higgs ($\lambda = 0$), quarks and leptons ($\pm \frac{1}{2}$), gauge bosons ($\pm 1$) and the graviton ($\pm 2$). There are no higher spin particles!
      The $\lambda = \pm \frac{3}{2}$ particle, dubbed the gravitino, has not yet been found.
    \end{remark}
    \begin{remark}
      There is another state with $P^{\mu} P_{\mu} = 0$, which is simply $p^{\mu} = (0,0,0,0)$. This is what people call the vacuum; a state without any particles.
    \end{remark}
  \item[$P^{\mu} P_{\mu} < 0$] These are tachyonic states.
\end{description}

\begin{remark}
  Weinberg's book (Chapter 2) will go into further detail on this.
\end{remark}
