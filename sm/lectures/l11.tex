% lecture notes by Umut Özer
% course: sm
\lhead{Lecture 11: February 11}

\section{Origin of Gauge (Local) Symmetries}%
\label{sec:origin_of_gauge_local_symmetries}

Consider a massless helicity-1 field.
\begin{equation}
  A_{\mu}(x) = \sum_{\lambda=\pm 1} \int \dd[]{p} \left( \epsilon_{\mu}(p^{\mu}, \lambda) a_{p \lambda} e^{i p x} + \epsilon^*_{\mu} (p^{\mu}, \lambda) a^{\dagger}_{p \lambda} e^{-i p x} \right),
\end{equation}
where $\epsilon_{\mu}$  is the polarisation vector.
As it is written, $A_{\mu}$  has four degrees of freedom $\mu = 0,1,2,3$.
However, we know that the massless helicity-1 field only has two degrees of freedom $\lambda = \pm 1$, so we need some (Lorentz invariant) constraints. We can impose
\begin{equation}
  \label{eq:11-1}
  p^{\mu} \epsilon_{\mu} = 0,
\end{equation}
which leads us from $4$  to $3$  degrees of freedom; this would be enough for a massive vector with $j_3 = -1, 0, 1$, but it is not satisfactory for the massless particle. 
In fact, there are no other Lorentz invariant constraints. Thus we know that the $\epsilon_{\mu}$  have some extra degree of freedom. The constraint \eqref{eq:11-1} leaves open a redundancy
\begin{equation}
  \label{eq:11-2}
  \epsilon_{\mu} \equiv \epsilon_{\mu} + \alpha p_{\mu},
\end{equation}
which defines an equivalence class for the $\epsilon_{\mu}$ .
Transforming back to position space, the redundancy \eqref{eq:11-2} becomes the gauge invariance condition
\begin{equation}
  A_{\mu} \equiv A_{\mu} + \partial_{\mu} \alpha.
\end{equation}
The origin of gauge invariance lies in the Lorentz invariant description of a massless helicity-1 field.
One might say that Lorentz invariance implies gauge invariance!
\begin{remark}
  The polarisation vector $\epsilon_{\mu}$ is not a Lorentz covariant object despite carrying a vector index.
  This is because it transforms as $\epsilon_{\mu} \to \Lambda\indices{_{\mu}^{\nu}} \epsilon_{\nu} + \alpha p_{\mu}$.
\end{remark}

Similarly, for  helicity $\lambda = 2$, we work with a field $h_{\mu\nu}$ and polarisations $\epsilon_{\mu\nu}$. We end up with a similar reundancy
\begin{equation}
  \epsilon_{\mu\nu} \equiv \epsilon_{\mu\nu} + \alpha_{\mu} p_{\nu} + p_{\mu} \alpha_{\nu} \implies h_{\mu\nu} \equiv h_{\mu\nu} + \partial_{\mu} \alpha_{\nu} + \partial_{\nu} \alpha_{\mu}.
\end{equation}
Again, the diffeomorphism invariance of (linearised) general relativity is implied by the Lorentz invariance of a massless helicity-2 particle.

All the beauty of the geometry of general relativity, or the gauge symmetry, is gone. It is all really Lorentz invariance.

Any amplitude (recall: $S_{\alpha\beta} = \delta_{\beta\alpha} + (2\pi) \delta (p_\alpha - p_{\beta}) M_{\alpha\beta}$) will be of the form
\begin{equation}
  M_{\alpha\beta} (p^{\mu}_i, \lambda_i) = \epsilon^{\mu} (M_{\mu})_{\alpha\beta}.
\end{equation}
The redundancy \eqref{eq:11-2} due to Lorentz invariance implies the \emph{Ward identity}
\begin{equation}
  \boxed{p^{\mu} M_{\mu} = 0}.
\end{equation}

\subsection{Charge Conservation}%
\label{sub:charge_conservation}

\begin{figure}[tbhp]
  \centering
  \begin{minipage}[t]{0.5\columnwidth}
    \centering
    \begin{tikzpicture}
      \begin{feynman}
	\vertex (a) at (-2, 0.33);
	\vertex (b) at (-2, 1);
	\vertex (c) at (-2, -1);
	\vertex (d) at (-2, -0.33);
  
	\vertex (f) at (2, 1);
	\vertex (g) at (2, -1);
  
	\tikzfeynmanset{every vertex={large, blob}};
	\vertex (v) at (0,0);
  
	\diagram* {
      (a) -- [fermion] (v),
      (b) -- [fermion] (v) --[fermion] (f),
      (c) -- [fermion] (v) --[fermion, edge label=$p^{\mu}$] (g),
      (d) -- [fermion] (v),
	};
      \end{feynman}
    \end{tikzpicture}
    \caption{}
    \label{fig:l11d1}
  \end{minipage}%
  \begin{minipage}[t]{0.5\columnwidth}
    \centering
    \begin{tikzpicture}
      \begin{feynman}
  
	\vertex (a) at (-2, 0.33);
	\vertex (b) at (-2, 1);
	\vertex (c) at (-2, -1);
	\vertex (d) at (-2, -0.33);
  
	\vertex (e) at (2, 0) {$q^{\mu}$};
	\vertex (f) at (2, 1);
	\vertex (g) at (2, -1) {$p^{\mu}$};
  
	\tikzfeynmanset{every vertex={small, dot}};
	\vertex[label=$e$] (i) at ($(0,0)!0.5!(g)$);
  
	\tikzfeynmanset{every vertex={large, blob}};
	\vertex (v) at (0,0);
  
	\diagram* {
	  (a) -- [fermion] (v),
	  (b) -- [fermion] (v) --[fermion] (f),
	  (c) -- [fermion] (v) --[fermion] (i) -- [fermion] (g),
	  (d) -- [fermion] (v),
	  (i) -- [boson] (e),
	};
      \end{feynman}
    \end{tikzpicture}
    \caption{}
    \label{fig:l11d2}
  \end{minipage}
\end{figure}
Consider the interaction illustrated in Fig.~\ref{fig:l11d1}, which is described by some matrix element $M_0$.
Fig.~\ref{fig:l11d2} depicts the same interaction, except that we added a `soft photon' with momentum $q^{\mu} \ll p^{\mu}$ to the line with momentum $p^{\mu}$. This interaction has matrix element $\mathcal{M}$. 
The most general vertex is $\Gamma^{\mu} = F p^{\mu} + Q q^{\mu}$, where $F, Q$ are functions of $p^2, q^2$, and $p \cdot q$.
Computing $\epsilon^{\mu} \Gamma_{\mu}$ and $q^{\mu} q_{\mu} = 0$, we can forget the $Q$ term.
If $p^2 = m^2$ and $q^2 = 0$, then $F = F(\frac{p \cdot q}{m^2}) \approx F(0)$ without loss of generality.
Thus we have
\begin{equation}
  M = M_0 \times \left( \frac{\epsilon^{\mu} F p_{\mu}}{(p + q)^2 - m^2} \right) \simeq M_0 \times \left( \frac{F \epsilon \cdot p}{2 p \cdot q} \right)
\end{equation}

Adding soft photons to all external lines gives the Ward identity
\begin{equation}
  M = M_0 \left( -\sum_{\text{incoming}} \frac{p_i \cdot \epsilon}{2 p_i \cdot q} F_i(0) + \sum_{\text{outgoing}} \frac{p_i \cdot \epsilon}{2 p_i \cdot q} F_i(0) \right).
\end{equation}
This gives \emph{charge conservation}
\begin{equation}
  \sum_{\mathclap{\text{ingoing}}} F_i (0) - \sum_{\mathclap{\text{outgoing}}} F_i(0) = 0,
\end{equation}
where $F_i(0) \coloneqq Q_i$.

Let us do the same for $\lambda = 2$. This time we add a soft graviton instead of a soft photon.
After going through the same considerations, one arrives at the Ward identity
\begin{equation}
  \sum_{\text{in}} k_i p_i^{\nu} - \sum_{\text{out}} k_i p^{\nu}_i = 0.
\end{equation}
The extra factors of the $p_i^{\nu}$ complicate this.
How can you have an interaction where all the momenta are conserved and then this new combination of momenta is also conserved? This is only possible if all $k_i$ are equal!
This is the \emph{principle of equivalence}: all particles interact gravitationally with the same strength.

What happens for $\lambda = 3$?
We find that $ \sum g_i p_i^{\mu} p_j^{\nu} $ is conserved.
This implies that all $g_i = 0$.
This is an extremely important result: there are no interacting theories for massless particles with helicity higher than $\lambda = 2$.
Detailed proofs are given in \cite[Vol.~1, Chapter 13]{weinberg} and in \cite[Chapter 9]{schwartz}.
