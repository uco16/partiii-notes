% lecture notes by Umut Özer
% course: sm
\lhead{Lecture 11: February 11}

\section{Origin of Gauge (Local) Symmetries}%
\label{sec:origin_of_gauge_local_symmetries}

Consider a massless helicity-1 field.
\begin{equation}
  A_{\mu}(x) = \sum_{\lambda=\pm 1} \int \dd[]{p} \left( \epsilon_{\mu}(p^{\mu}, \lambda) a_{p \lambda} e^{i p x} + \epsilon^*_{\mu} (p^{\mu}, \lambda) a^{\dagger}_{p \lambda} e^{-i p x} \right),
\end{equation}
where $\epsilon_{\mu}$  is the polarisation vector.
As it is written, $A_{\mu}$  has four degrees of freedom $\mu = 0,1,2,3$.
However, we know that the massless helicity-1 field only has two degrees of freedom $\lambda = \pm 1$, so we need to impose  (Lorentz invariant) constraints:
Imposing
\begin{equation}
  \label{eq:11-1}
  p^{\mu} \epsilon_{\mu} = 0,
\end{equation}
leads us from $4$  to $3$  degrees of freedom; this would be enough for a massive vector with $j_3 = -1, 0, 1$, but it is not satisfactory for the massless particle. 
In fact, there are no other Lorentz invariant constraints. Thus we know that the $\epsilon_{\mu}$  have some extra degree of freedom. The constraint \eqref{eq:11-1} leaves open a redundancy
\begin{equation}
  \label{eq:11-2}
  \epsilon_{\mu} \equiv \epsilon_{\mu} + \alpha p_{\mu},
\end{equation}
which defines an equivalence class for the $\epsilon_{\mu}$ .
Transforming back to position space, the redundancy \eqref{eq:11-2} becomes the gauge invariance condition
\begin{equation}
  A_{\mu} \equiv A_{\mu} + \partial_{\mu} \alpha.
\end{equation}
The origin of gauge invariance lies in the Lorentz invariant description of a massless helicity-1 field.
One might say that Lorentz invariance implies gauge invariance!
\begin{remark}
  The polarisation vector $\epsilon_{\mu}$ is not a Lorentz covariant object despite carrying a vector index.
  This is because it transforms as $\epsilon_{\mu} \to \Lambda\indices{_{\mu}^{\nu}} \epsilon_{\nu} + \alpha p_{\mu}$.
\end{remark}

Similarly, for  helicity $\lambda = 2$, we work with a field $h_{\mu\nu}$ and polarisations $\epsilon_{\mu\nu}$. One then has to impose the redundancy 
\begin{equation}
  \epsilon_{\mu\nu} \equiv \epsilon_{\mu\nu} + \alpha_{\mu} p_{\nu} + p_{\mu} \alpha_{\nu} \implies h_{\mu\nu} \equiv h_{\mu\nu} + \partial_{\mu} \alpha_{\nu} + \partial_{\nu} \alpha_{\mu}.
\end{equation}
Again, the diffeomorphism invariance of (linearised) general relativity is implied by the Lorentz invariance of a massless helicity-2 particle.

All the beauty of the geometry of general relativity, or the gauge symmetry, is gone. It is all really Lorentz invariance.

Any amplitude (recall: $S_{\alpha\beta} = \delta_{\beta\alpha} + (2\pi) \delta (p_\alpha - p_{\beta}) M_{\alpha\beta}$) will be of the form
\begin{equation}
  M_{\alpha\beta} (p^{\mu}_i, \lambda_i) = \epsilon^{\mu} M_{\mu}.
\end{equation}
The redundancy \eqref{eq:11-2} due to Lorentz invariance implies the \emph{Ward identity}
\begin{equation}
  \boxed{p^{\mu} M_{\mu} = 0}
\end{equation}

\subsection{Charge Conservation}%
\label{sub:charge_conservation}

Consider the following interaction
\begin{equation}
  \begin{tikzpicture}
    \begin{feynman}

      \vertex (a) at (-2, 0.33);
      \vertex (b) at (-2, 1);
      \vertex (c) at (-2, -1);
      \vertex (d) at (-2, -0.33);

      \vertex (e) at (2, 0.33);
      \vertex (f) at (2, 1);
      \vertex (g) at (2, -1);
      \vertex (h) at (2, -0.33);

      \tikzfeynmanset{every vertex={blob}};
      \vertex (v) at (0,0);

      \diagram* {
	(a) -- [fermion] (v) --[fermion] (e),
	(b) -- [fermion] (v) --[fermion] (f),
	(c) -- [fermion] (v) --[fermion] (g),
	(d) -- [fermion] (v) --[fermion] (h),
      };
    \end{feynman}
  \end{tikzpicture}
\end{equation}
This is described by a matrix element $M_0$.

Now add a `soft photon'
\begin{equation}
  \begin{tikzpicture}
    \begin{feynman}

      \vertex (a) at (-2, 0.33);
      \vertex (b) at (-2, 1);
      \vertex (c) at (-2, -1);
      \vertex (d) at (-2, -0.33);

      \vertex (e) at (2, 0.33);
      \vertex (f) at (2, 1);
      \vertex (g) at (2, -1);
      \vertex (h) at (2, -0.33);

      \vertex (j) at (1.2, -1);

      \tikzfeynmanset{every vertex={blob}};
      \vertex (v) at (0,0);

      \tikzfeynmanset{every vertex={small, dot}};
      \vertex (i) at ($(0,0)!0.5!(g)$);

      \diagram* {
	(a) -- [fermion] (v) --[fermion] (e),
	(b) -- [fermion] (v) --[fermion] (f),
	(c) -- [fermion] (v) --[fermion] (i) -- [fermion] (g),
	(d) -- [fermion] (v) --[fermion] (h),
	(i) -- [boson] (j),
      };
    \end{feynman}
  \end{tikzpicture}
\end{equation}
which has matrix element $\mathcal{M}$. 

LOTS OF STUFF MISSED

Adding soft photons to all external lines gives the Ward identity
\begin{equation}
  M = M_0 \left( -\sum_{\text{incoming}} \frac{p_i \cdot \epsilon}{2 p_i \cdot q} F_i(0) + \sum_{\text{outgoing}} \frac{p_i^{\mu} \epsilon_{\mu}}{2 p_i \cdot \vb{q}} F_i(0) \right).
\end{equation}
This gives \emph{charge conservation}
\begin{equation}
  \sum_{\mathclap{\text{ingoing}}} F_i (0) - \sum_{\mathclap{\text{outgoing}}} F_i(0) = 0,
\end{equation}
where $F_i(0) \coloneqq Q_i$.

Let us do the same for $\lambda = 2$.
SOFT GRAVITON DIAGRAM
At the end, the Ward identity will tell us
\begin{equation}
  \sum_{\text{in}} k_i p_i^{\nu} - \sum_{\text{out}} k_i p^{\nu}_i = 0.
\end{equation}
The extra factors of the $p_i^{\nu}$ complicate this.
How can you have an interaction where all the momenta are conserved and then this new combination of momenta is also conserved? This is not possible except if all $k_i$ are equal!
This is the \emph{principle of equivalence}: all particles interact gravitationally with the same strength.

What happens for $\lambda = 3$?
We find that $ \sum g_i p_i^{\mu} p_j^{\nu} $ is conserved.
This implies that all $g_i = 0$.
This is an extremely important result: there are no interacting theories for massless particles with helicity higher than $\lambda = 2$.
The proof is given in Chapter 13 of Weinberg Vol.~1 and also in Schwarz' book Chapter 9.
