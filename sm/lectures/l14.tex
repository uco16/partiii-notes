% lecture notes by Umut Özer
% course: sm
\lhead{Lecture 14: February 18}

\chapter{Broken Symmetries}%
\label{cha:broken_symmetries}

\section{Motivation}%
\label{sec:motivation}

So far spin/helicity $0, \pm \frac{1}{2}$ OK

massless helicity $\pm 1 \implies$  QED, Yang--Mills

massless helicity $\pm 2 \implies$ gravity

No more interactions.

But massless Yang--Mills fields have not been observed.

What about massive $s = 1$?
A massive spin-1 field $A_{\mu}(x, t)$ is associated with a polarisation vector $\epsilon_{\mu}(p)$.
It has three polarisation states $j_z = -1, 0, 1$.
Since $A_{\mu}(x, t)$ has 4 degrees of freedom, we impose the condition $p^{\mu}\epsilon_{\mu} = 0$ to get three polarisations. There is no gauge redundancy.

Pick $p^{\mu} = (E, 0,0, p_z)$ with the condition that $E^2 - p_z^2 = m^2$.
We have two transverse polarisations $\epsilon^{\mu}_1 = (0,1,0,0)$ and $\epsilon^{\mu}_2 = (0,0,1,0)$ and one longitudinal polarisation.
With the normalisation condition $\epsilon_{\mu}^2 = -1$, there is only one possibility: $\epsilon^{\mu}_L = (\frac{p_z}{m}, 0,0, \frac{E}{m})$.

Suppose we are at high energies, where $E \gg m$. We then have $p \approx E$.
Then the longitudinal polarisation becomes $\epsilon^{\mu}_L \approx \frac{E}{m}(1,0,0,1)$.
The amplitudes will have a term $M \sim g^2 \epsilon^0_L \epsilon^z_L \sim g^2 \frac{E^2}{m^2}$. This blows up!
However, the amplitudes are probabilities, which have to be less than one; so this blowing-up breaks unitarity. The theory of massless spin-1 particles must have a cutoff; it cannot be valid until high energies since it breaks perturbative unitarity.
\begin{example}[]
  For a mass $m \simeq 100$GeV and coupling $g \sim 0.1$, the energy has to be less than $E \lesssim 1$Tev.
\end{example}
Massive spin-1 particles lead to interactions that can be valid only at small energies and have to be superseded by a consistent UV completion.

To search for a UV completion?

\section{Spontaneously Broken Discrete Symmetries}%
\label{sec:spontaneously_broken_discrete_symmetries}

Let us warm up to this concept slowly by considering discrete symmetries.
Take the theory of a real scalar field with symmetry $\phi \to -\phi$.
The most general renormalisable Lagrangian is
\begin{equation}
  \mathscr{L} = \frac{1}{2} \partial_{\mu} \phi \partial^{\mu} \phi - V_{\pm}(\phi).
\end{equation}
There are two possible scalar potentials
\begin{equation}
  V_{\pm}(\phi) = \pm \frac{1}{2} m^2 \phi^2 + \frac{\lambda}{4} \phi^4 + \kappa_{\pm}.
\end{equation}
Here, $\lambda > 0$ for stability and $\kappa_{\pm}$ is introduced to be able to tune the value of $V$ at the minimum.
\begin{remark}
  Gauge invariance is imposed only for massless particles. For massive particles, the most general theory involves a mass-term, which breaks gauge invariance. This is totally okay, since gauge invariance is only secondary, coming from our requirement of Lorentz invariance.
\end{remark}

\begin{figure}[tbhp]
  \centering
  \def\svgwidth{0.4\columnwidth}
  \input{lectures/l14f1.pdf_tex}
  \caption{}
  \label{fig:l14f1}
\end{figure}
The potential $V_+$ has a minimum at $\phi = \phi_0 = 0$ and is symmetric under $\phi \to -\phi$, as illustrated in Fig.~\ref{fig:l14f1}.
We say the ``symmetry is manifest''.

In the quantum theory, we want to investigate vacuum expectation values (VEV)
\begin{equation}
  \langle \phi \rangle \coloneqq \bra{0}\phi \ket{0} = \int \pdd{\phi} \phi e^{\frac{i}{\hbar} \int \mathscr{L} \dd[4]{x}}.
\end{equation}
Taking the classical limit $\hbar \to 0$, we recover the minimum of the potential $\langle \phi \rangle \to \phi_0$.

Perturbations around the vacuum
\begin{equation}
  \phi = \phi_0 + \sigma(x) = \sigma(x).
\end{equation}
The Lagrangian becomes
\begin{equation}
  \mathscr{L} = \frac{1}{2} \partial^{\mu} \sigma \partial_{\mu} \sigma - \frac{1}{2} m^2 \sigma^2 - \frac{\lambda}{4} \sigma^4,
\end{equation}
so  $\sigma$ is a particle of mass $m$.

We can write the quartic $V_-$ as 
\begin{equation}
  V_- = \frac{\lambda}{2} \left( \phi^2 - v^2 \right)^2, \qquad v \coloneqq \sqrt{\frac{m^2}{\lambda}}.
\end{equation}
This potential is illustrated in Fig.~\ref{fig:l14f2}.
\begin{figure}[tbhp]
  \centering
  \def\svgwidth{0.4\columnwidth}
  \input{lectures/l14f2.pdf_tex}
  \caption{If you sit in the vacua $\phi = \pm v$, you do not see the symmetry of the potential.}
  \label{fig:l14f2}
\end{figure}
We have two vacua $\langle \phi \rangle = \pm v$ degenerate.
Perturbing around the vacuum $\phi = \pm v + \sigma(x)$ gives
\begin{equation}
  \mathscr{L} = \frac{1}{2} \partial_{\mu} \sigma \partial_{\mu} \sigma - \left( \lambda v^2 \sigma^2 \pm \lambda v \sigma^3 + \frac{\lambda}{4} \sigma^4 \right).
\end{equation}
In this case $\lambda v^2 \sigma^2 = m^2 \sigma^2 + \dots$. So $\sigma$ is a particle of squared mass $2m^2 \geq 0$.
The second derivative of the potential gives the mass of the particle.
\begin{equation}
  m^2 = \left.\frac{\partial^2 V_-}{\partial \phi^2} \right\rvert_{\phi^2 = v^2} = 2 m^2.
\end{equation}
We got a cubic term in the Lagrangian potential.
The symmetry $\phi \to -\phi$ is \emph{hidden} (spontaneously broken).
However, the symmetry is of course still there since we have $\sigma \to -\sigma \mp 2v$. Some people say the symmetry is ``non-linearly realised'', when we have $\phi \to a\phi + X$, where $X$ is something else.

If we would have expanded around $\phi = 0$, we have a tachyonic mass 
\begin{equation}
  \left.\frac{\partial^2 V}{\partial \phi^2} \right\rvert_{\phi = 0} = - m^2.
\end{equation}
In principle, you can consider superpositions of the two vacua, which recovers the symmetry.
This is discussed in \cite[Vol.~2]{weinberg}; the upshot is that in large systems you still always have to choose one particular vacuum.

This system is very simple, just possessing a discrete symmetry, but is nonetheless very rich.
Consider a thermodynamic system where the coefficient of $\phi^2$ depends on the temperature.
We have $m^2 \propto (T - T_c)$.
For very high temperatures, $T \gg T_c$, we have a potential like $V_+$. Reducing the temperature to $T < T_c$, the symmetry is broken to the potential $V_-$.
This model explains systems ranging from magnetic materials to cosmological systems.
These models can have rich properties such as domain walls and other things we discussed in the \emph{Statistical Field Theory} course.
