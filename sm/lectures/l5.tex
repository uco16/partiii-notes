% lecture notes by Umut Özer
% course: sm
\lhead{Lecture 5: January 28}

\subsection*{Comment 2}%


The algebra of $SO(3, 1)$ is determined by the algebra of $SU(2) \times SU(2)$.
Define Hermitian operators $J_{i} = \frac{1}{2} \epsilon_{ijk} M_{jk}$ and $K_{i} = M_{0i}$. Their algebra arises directly from the commutators \eqref{eq:mcom} of the $M_{ij}$:
\begin{equation}
  [J_{i}, J_{j}] = i \epsilon_{ijk}, 
  \qquad [J_{i}, K_k] = i \epsilon_{ijk} K_k,
  \qquad [K_{i}, K_{j}] = -i \epsilon_{ijk} J_k.
\end{equation}
We can also define $A_i = \frac{1}{2} (J_i + i K_i)$ and $B_i = \frac{1}{2} (J_i - i K_i)$. These are not Hermitian.
However, this gives a nice separation between the algebras
\begin{equation}
  [A_i, A_j] = i \epsilon_{ijk} A_k, \qquad
  [B_i, B_j] = i \epsilon_{ijk} B_k, \qquad
  [A_i, B_j] = 0,
\end{equation}
and in each case the $A$- and $B$-subalgebra look like the algebra of the $J$'s: These are like $SU(2)$ algebras, except they are not Hermitian.


\subsection*{Representations of $SU(2) \times SU(2)$}%

For representations of $SU(2) \times SU(2)$, recall that that the $SU(2)$ states are labelled by half-integers $j = 0, \frac{1}{2}, \dots$.
Then the $A_i$ and $B_i$ algebra states are labelled by $A, B = 0, \frac{1}{2}, \dots$ respectively.
Therefore, the representations of $SO(3, 1)$ can be labelled by specifying $(A, B)$.
\begin{remark}
  Under parity
  \begin{align}
    P\colon \qquad &J_i \to J_i \\
		   & K_i \to -K_i \\
		   & A_i \leftrightarrow B_i \\
		   & (A, B) \leftrightarrow (B, A)
  \end{align}
  Therefore, we can denote either one of these, say $(A, B)$, as `left'. Then $(B, A)$ is `right' and vice-versa.
\end{remark}

\subsection*{Comment 3}%

We have the homomorphism
\begin{equation}
  SO(3, 1) \simeq SL(2, \mathbb{C}).
\end{equation}

Consider first $SO(3, 1)$.
Let $X = X_{\mu} e^{\mu} = (X_0, X_1, X_2, X_3)$  denote a $4$ -vector.
Under Lorentz transformation $X \to \Lambda X$, where $\Lambda \in SO(3, 1)$, the modulus squared $\abs{X}^2 = X^2_0 - X^2_1 - X^2_2 - X^2_3$  remains invariant.

Now consider the space of $2 \times 2$  matrices with basis given by the Pauli matrices
\begin{equation}
  \sigma^{\mu} \coloneqq 
  \left\{ 
    \begin{pmatrix}
     1 & 0 \\
     0 & 1 \\
    \end{pmatrix},
    \begin{pmatrix}
     0 & 1 \\
     1 & 0 \\
    \end{pmatrix},
    \begin{pmatrix}
     0 & -i \\
     i & 0 \\
    \end{pmatrix},
    \begin{pmatrix}
     1 & 0 \\
     0 & -1 \\
    \end{pmatrix}
  \right\}
\end{equation}
We can then write any matrix $\widetilde{X}$ as a linear combination of these
\begin{equation}
  \label{eq:5-map}
  \widetilde{X} = X_{\mu} \sigma^{\mu} = 
  \begin{pmatrix}
   X_0 + X_3 & X_1 - i X_2 \\
   X_1 + i X_2 & X_0 - X_3 \\
  \end{pmatrix}.
\end{equation}
Taking the components $(X_0, X_1, X_2, X_3)$ to be the same as the $4$-vector above, this is just another way of representing the same information.
Furthermore, the action of $SL(2, \mathbb{C})$ on $\widetilde{X}$ is
\begin{equation}
  \widetilde{X} \to N \widetilde{X} N^{\dagger}, \qquad N \in SL(2, \mathbb{C}).
\end{equation}
Since $N \in SL(2, \mathbb{C})$, we have $\det N = 1$.  The determinant has the form $\det \widetilde{X} = X^2_0 - X^2_1 - X^2_2 - X^2_3$. Exactly as in the case of $SO(3, 1)$, this quantity is kept invariant.

This is the defining feature. As such, the map from $SL(2, \mathbb{C}) \to SO(3, 1)$ defined by \eqref{eq:5-map} is homomorphic.
This map is $2$ to $1$ since it maps $N = \pm \mathbb{1} \to \Lambda = \mathbb{1}$.

\begin{claim}
  But $SL(2, \mathbb{C})$ is \emph{simply connected.}
\end{claim}
\begin{proof}
  Polar decomposition: We can write $N = e^{h} U$, where $h = h^{\dagger}$ is hermitian and $U = (U^{\dagger})^{-1}$ unitary.
  Since the eigenvalues of a hermitian matrix are positive, the trace of $h$ is positive.
  Then, using that $\det e^h = e^{\tr h}$, we find that $\det N = 1$ implies that $\tr h = 0$ and $\det U = 1$.
  \begin{equation}
    h =
    \begin{pmatrix}
     a & b+ic \\
     b-ic & -a \\
    \end{pmatrix}
    \qquad U = 
    \begin{pmatrix}
     x + i y & z + i w \\
     -z + i w & x - i y \\
    \end{pmatrix}.
  \end{equation}
  For $h$, the variables $a, b, c \in \mathbb{R}$ define the manifold $\mathbb{R}^3$.
  Similarly the components of $U$ have the condition $x^2 + y^2 + z^2 + w^2 = 1$, defining the manifold $S^3$.
  Therefore, we have that $SL(2, \mathbb{C})$ has the manifold structure $\mathbb{R}^3 \times S^3$, which is simply connected.
\end{proof}
\begin{corollary}
  Thus, the $SO(3, 1)$ manifold, which is obtained from a $2$ to $1$ map from $SL(2, \mathbb{C})$, is $\mathbb{R}^3 \times S^3 / \mathbb{Z}_2$.
\end{corollary}

\subsection*{Representations of $SL(2, \mathbb{C})$}%

\begin{definition}[fundamental]
  The \emph{fundamental representation} $\psi_{\alpha}$, $\alpha = 1, 2$ is given by
  \begin{equation}
    \psi'_{\alpha} = N\indices{_{\alpha}^{\beta}} \psi_{\beta}.
  \end{equation}
  The $\psi_{\alpha}$ transforming in this way are called \emph{left-handed Weyl spinors}.
\end{definition}
\begin{definition}[conjugate]
  The \emph{antifundamental} or \emph{conjugate representation} is given by \emph{right-handed Weyl spinors} $\overline{\chi}_{\dot{\alpha}}$, $\dot{\alpha} = 1, 2$ transforming as
  \begin{equation}
    \overline{\chi}{}'_{\dot{\alpha}} = (N^*)\indices{_{\dot{\alpha}}^{\dot{\beta}}} \overline{\chi}_{\dot{\beta}}.
  \end{equation}
\end{definition}
\begin{definition}[contravariant]
  The \emph{contravariant representation} is
  \begin{equation}
    \psi'{}^{\alpha} = \psi^{\beta} (N^{-1})\indices{_{\beta}^{\alpha}}, 
    \qquad \overline{\chi}{}'{}^{\dot{\alpha}} = \overline{\chi}{}^{\dot{\beta}} (N^*{}^{-1})\indices{_{\dot{\beta}}^{\dot{\alpha}}}.
  \end{equation}
\end{definition}

To raise and lower indices, we need \emph{invariant tensors}:
\begin{description}
  \item[SO(3, 1)] $\eta^{\mu\nu} = (\eta_{\mu\nu})^{-1}$
  \item[SL(2, $\mathbb{C}$)] $\epsilon^{\alpha\beta} = \epsilon^{\dot{\alpha} \dot{\beta}} = 
    \begin{pmatrix}
     0 & 1 \\
     -1 & 0 \\
    \end{pmatrix} = -\epsilon_{\alpha\beta} = -\epsilon_{\dot{\alpha} \dot{\beta}}$ 
\end{description}
Invariance means that $\epsilon^{\alpha\beta}$ transforms as
\begin{equation}
  \epsilon'{}^{\alpha\beta} = \epsilon^{\rho\sigma} N\indices{_{\rho}^{\alpha}} N\indices{_{\sigma}^{\beta}} = \epsilon^{\alpha\beta} \det N = \epsilon^{\alpha\beta}
\end{equation}
