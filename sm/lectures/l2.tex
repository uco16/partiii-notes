% lecture notes by Umut Özer
% course: sm
\lhead{Lecture 2: January 21}

Take another look at the quarks and leptons.
For $Q_L: (3, 2, \frac{1}{6})$ the second entry tells us that these are doublets under $SU(2)$. This means that we have $Q_L = \begin{pmatrix} v_L \\ d_L \\ \end{pmatrix} $.

\section{History}%
\label{sec:history}

Weinberg has a good paper on advice he gives to young researchers. One of the advices is to study history and the history of physics in particular. This is because it gives us some sense of how physical theories developed and also that it makes us feel part of a bigger development in the pursuit of knowledge, no matter how small our contributions may seem.

\begin{description}
  \item[t < 20\textsuperscript{th} century:] only two interactions (gravity and electromagnetism)\par discreteness of matter not established
  \item[1896] Radioactivity (Becquerel, Pierre and Marie Curie, Rutherford) $\alpha, \beta, \gamma$ rays\par
    This was a big discovery at the time; there is no inherent stability in nature! This was a manifestation that pointed to the existence of other interactions.
  \item[1897] J.~J.~Thomson: discovered the \emph{electron} ($e^-$) and measured $e / m$ a few hundred meters from where we are right now in Cambridge (close to The Eagle so you can enjoy that on your visit too).
    This was the first particle ever discovered, marking the beginning of particle physics.
  \item[1900's]
    \begin{description}
      \item[1900-1930] Quantum mechanics developed. Probably the biggest revolution ever in science.
      \item[1905] Special relativity. These two still are the two basic theories to study in nature.
	The nature of quantum mechanics also implies that light behaves as a particle, which we now know as the photon.

	In the same decade, Rutherford's group also discovered the atom.
    \end{description}
  \item[1910's] Francis Aston (1919) defined the `whole number rule', for the ratio of different atomic nuclei to the hydrogen mass. This led to the discovery of the proton.

    Cosmic rays were studied, in particular by using cloud chambers.

    Einstein's theory of general relativity.
  \item[1920's] Bose, Fermi statistics.
    \par Beginning of QFT (Jordan, Heisenberg, Dirac, \dots)
    \par Dirac equation. This predicted a positive particle, which he thought could be the proton \dots
  \item[1930's] \dots but then came to predict that this is the \emph{positron} ($e^+$), which he just squeezed into the introduction of his paper on magnetic monopoles (1931).\par
    \begin{description}
      \item[1932] Anderson\footnote{There were multiple people who arguably should get some more credit for this. Blackett discovered the positron but did not publish it fast enough. There were also a Russian scientist and a graduate student at CalTech who did similar discoveries.} discovered it.
      \item[1932] Chadwick discovered the \emph{neutron}.
      \item[1930] Pauli predicted the \emph{neutrino}: $\beta$-decay: $n \to p + e^- + \overline{\nu}$. Fermi described this in terms of a four-point field theory, illustrated in Fig.~\ref{fig:l2d1}.
      \begin{figure}[tbh]
	\centering
	\feynmandiagram[transform shape, scale=0.7][horizontal=a to b, layered layout] {
	  a [particle=\(n\)] -- b [dot] -- c [particle=\(p\)],
	  b -- d [particle=\(e^{-}\)],
	  b -- e [particle=\(\overline{\nu}\)],
	};
	\caption{Four-point description of $\beta$-decay.}
	\label{fig:l2d1}
      \end{figure}
    \item[1934] Yukawa theory of strong interactions: scalar mediators of strong interactions (Pions).
      Potential $V(r) \sim e^{-mr} / r$, where $m \sim 100$MeV is the pion mass.
      This explained the short range and kept field theory going, so people started to search for this new particle.
    \item[1936] Anderson discovered the \emph{muon} ($m \sim 100$MeV).
    \item[1932] Heisenberg, 1936 (Condon et al.) introduced isospin $n \leftrightarrow p$.
      He thought that in the same way that the electron has spin-up and spin-down, the proton and neutron have such similar properties that they also have an internal symmetry.
    \end{description}
  \item[1940's] The history of physics is different to the history you learn at school; at much happens in physics during war-time.
    \begin{description}
      \item[1947] Lamb shift, QED (Schwinger, Feynman + Tomonaga + Dyson)\par Pions $\pi$ were discovered, explaining why the naive picture of Yukawa made sense.
    \end{description}
  \item[1950's] 
    \begin{itemize}
      \item A time of great optimism. Particle accelerators and bubble chambers were built (E > MeV).
    People say that the 50's were a decade of wealth; the numbers of particles were also very rich. Dozens of new particles were discovered (kaons, hyperons, \dots), mostly strongly interacting (\emph{hadrons}), which are now classified into mesons (bosons) and baryons (fermions).
    \item Strangeness (Gell-Mann, Nishijima, Pais)
    \item Parity Violation (Lee and Yang) in 1936, Wu discovered it in 1957
    \item Discovery of (anti-)neutrino (Cowan- Reines, 1956)
    \item $V-A$ property of weak interactions (Marshak, Sudarshan, 1957)\footnote{Four experiements seemed to deny their theory. However, the theory was so strong that they were convinced that these experiments must have been wrong. All four actually turned out to be wrong.}
    \item Pontecorvo; neutrino oscillations proposed
    \item Yang--Mills 1954, non-Abelian gauge theory.
      In QED, there is a $U(1)$ gauge theory giving a massless photon.
      In Yang--Mills theory, with a greater group such as $SU(2)$, that this should give further massless / long-range particles. But these have never been seen so Pauli predicted correctly that this theory was not relevant to nature.
    \end{itemize}

\end{description}
