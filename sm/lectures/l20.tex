% lecture notes by Umut Özer
% course: sm
\lhead{Lecture 20: March 03}

\subsection{Coupling to Fermions}%
\label{sub:coupling_to_fermions}

We will simultaneously treat the leptons $L^{i}_L$ and quarks $Q^{i}_L$, which come in three families
\begin{align}
  L^{i}_L &=
  \left\{ 
    \begin{pmatrix}
    \nu_{e L} \\
    e_L \\
    \end{pmatrix}
    ,
    \begin{pmatrix}
    \nu_{\mu L} \\
    \mu_L \\
    \end{pmatrix}
    ,
    \begin{pmatrix}
    \nu_{\tau L} \\
    \tau_L \\
    \end{pmatrix}
  \right\}_{Y = - 1 / 2} \\
  Q^{i}_L &=
  \left\{ 
    \begin{pmatrix}
    u_{L} \\
    d_L \\
    \end{pmatrix}
    ,
    \begin{pmatrix}
    c_{L} \\
    s_L \\
    \end{pmatrix}
    ,
    \begin{pmatrix}
    t_{L} \\
    b_L \\
    \end{pmatrix}
  \right\}_{Y = 1 / 6}
\end{align}
\begin{align}
  u^{i}_R &= \left\{ u_R, c_R, t_R \right\}_{Y = -2 / 5} & d_R^{i} &= \left\{ d_R, s_R, b_R \right\}_{Y = -1 / 3} \\
  e^{i}_R &= \left\{ e_R, \mu_R, \tau_R \right\}_{Y = -1} & \nu_R^{i} &= \left\{ \nu_{e R}, \nu_{\mu R}, \nu_{\tau R} \right\}_{Y = 0}
\end{align}
Hypercharges specified by anomaly cancellation:
\begin{equation}
  2 \Tr[T^{a} \left\{ T^{b}, T^{c} \right\}]_R = A(R) d^{abc},
\end{equation}
where the anomaly consistency condition is $A(R) = 0$.
\begin{equation}
  \left.
  \begin{gathered}
    \feynmandiagram[transform shape, scale=1][horizontal=a to b] {
      a [particle=\(U(1)\)] -- [boson] b -- c --d -- b,
      c -- [boson] e,
      d -- [boson] f,
      e -- [draw=none] f,
    };
  \end{gathered}
  \ \right\} \
  \text{U(1), SU(2), SU(3), gravity}
\end{equation}
\begin{itemize}
  \item $U(1)^3: \sum_{\text{left}} Y^3 - \sum_{\text{right}} Y^3 = 0$
  \item $SU(2)^2 \times U(1): Y_L + 3 Y_R = 0$
  \item $SU(2)^3 \times U(1): 2 Y_Q - Y_u - Y_d = 0$
  \item gravity$^2 \times U(1): (2 Y_L - Y_e - Y_u) + e(Y_Q - Y_u - Y_d) = 0$
\end{itemize}
With $\nu_R$, also $U(1)_{B - L}$ satisfies the conditions.

\subsection{Electroweak Interactions of Fermions}%
\label{sub:electroweak_interactions_of_fermions}

We split the Lagrangian
\begin{equation}
  \mathscr{L}_{\text{fermions}} = \mathscr{L}^{\text{fermions}}_{\text{kinetic}} + \mathscr{L}_{\underbrace{\text{Higgs-fermions}}_{\mathclap{\text{Yukawa}}}}
\end{equation}
The general expression for the kinetic energy is
\begin{equation}
  \mathscr{L}_{\text{kinetic}}^{\text{fermions}} = i \overline{L}{}_i \cancel{D} L_i + i \overline{Q}{}_{L_i} \cancel{D} Q_{L_i} + i \overline{e}{}_{R_i} \cancel{D} e_{R_i} + i \overline{\nu}{}_{R_i} \cancel{ D} \nu_{R_i} + i \overline{u}{}_{R_i} \cancel{D} u_{R_i}  + i \cancel{d}_{R_i} \cancel{D} d_{R_i},
\end{equation}
and the covariant derivative is
\begin{align}
  D_{\mu} &= \partial_{\mu} - i g W_{\mu}^{a} T^{a} - i g' B_{\mu} Y \\
	  &= \partial_{\mu} - \frac{i g}{\sqrt{2}} \left( W_{\mu}^+ T^+ + W^-_{\mu} T^- \right) - \frac{i g Z_{\mu}}{\cos \theta_W} \left( T^3 - \sin^2 \theta_W Q \right)-i e A_{\mu} Q,
\end{align}
where $T^{\pm} = T^1 \pm i T^2$.

The Yukawa potential is
\begin{equation}
  \mathscr{L}_{\text{Yukawa}} = \mathscr{L}_{\text{Higgs-leptons}} + \mathscr{L}_{\text{Higgs-quarks}},
\end{equation}
with
\begin{equation}
  \mathscr{L}_{\text{Higgs-quarks}} = - y^{d}_{ij} \overline{Q}{}^{i}_{L} H d_{R}^{j} - y^{u}_{ij} \overline{Q}{}^{i}_{L} \widetilde{H} u^{j}_{R} + \text{h.o.},
\end{equation}
where $\widetilde{H} \coloneqq i \sigma_2 H^*$, and $y^{u}_{ij}, y^{d}_{ij}$ are constants, called the Yukawas.
These couplings are allowed by gauge invariance.
Since the standard model has chiral symmetry, direct mass mass terms for quarks are not allowed by gauge invariance. Quarks get mass because $\langle H \rangle \neq 0$. Yukawa couplings to the Higgs give mass to fermions.
This gives a mass term
\begin{equation}
  \mathscr{L}_{\text{mass-quarks}} = -\frac{v}{\sqrt{2}} \left[ \overline{d}{}^{i}_{L} y^{d}_{ij} d^{j}_{R} + \overline{u}{}^{i}_{L} y^{u}_{ij} u^{j}_{R} \right] + \text{h.o.}
\end{equation}

Let us diagonalise the mass matrix
\begin{equation}
  y^{d} = U_{d} M_{d} K^{\dagger}_{d}, \qquad y^{u} = U_{u} M_{u} K_{u}^{\dagger},
\end{equation}
where $K, U$ are unitary and $M_u, M_d$ are diagonal and real.
Also
\begin{equation}
  d_L \to U_d d_L, \qquad d_R \to K_d d_R \qquad u_L \to U_u u_L \qquad u_R \to K_u u_R.
\end{equation}
The mass term is
\begin{equation}
  \mathscr{L}_{\text{mass}} = -\frac{v}{\sqrt{2}} \left[ \overline{d}{}^{i}_{L} (M_d)_{ii} d^{i}_{R} + \overline{u}{}^{i}_{L} (M_u)^{ii} u^{i}_R \right] + \text{h.o.}.
\end{equation}
Thus, the masses are
\begin{equation}
  m_d = \frac{v}{\sqrt{2}} (M_d)_{ii}, \qquad m_{u_i} = \frac{v}{\sqrt{2}} (M_u)_{ii}.
\end{equation}
With these transformations, the masses are diagonal but the kinetic terms are not.
The problem arises with the covariant derivatives.
\begin{multline}
  \mathscr{L}_{\substack{\text{quarks} \\ \text{Higgs} \\ \text{gauge}}} = \mathscr{L}_{\text{kin}} + \frac{e}{\sin \theta_W} Z_m J^Z_m + e A_{\mu} J^{\mu}_{EM} - m^{j}_{d} (\overline{d}{}_L^{j} d^{j}_R + \overline{d}{}^{j}_R d^{j}_L) - m^{j}_{u} (\overline{u}{}^{j}_{L} u^{j}_{R} + \overline{u}{}^{j}_{R} u^{j}_{L}) \\
  + \frac{e}{\sqrt{2} \sin \theta_W} \bigl[ W^+_{\mu} \underbrace{\overline{u}{}_L \gamma^{\mu} (V_{\text{CKM}})^{ii} d^{j}_{L}}_{\mathclap{J^+_{\mu}}} + W^-_{\mu} \underbrace{\overline{d}{}^{i}_{L} \gamma^{\mu} (V_{\text{CKM}}^+)^{ij} u_L^{j}}_{\mathclap{J^-_{\mu}}} \bigr],
\end{multline}
where the \emph{Cabibbo--Kobayashi--Maskawa matrix} $V_{\text{CKM}}$ arises from the diagonalisation and is unitary
\begin{equation}
  V_{\text{CKM}} = U^{\dagger}_u U_d = 
  \begin{pmatrix}
   V_{ud} & V_{us} & V_{ub} \\
   V_{cd} & V_{cs} & V_{cb} \\
   V_{td} & V_{ts} & V_{tb} \\
  \end{pmatrix}.
\end{equation}
Note that this is a $3 \times 3$ unitary matrix, which generally has $9$ free parameters.
We can further reduce this number by $5$ since we have $U(1)^6$ symmetries $d^{i}_{R L} \to e^{i \alpha_i} d_{R, L}$ and $u^{i}_{R, L} \to e^{i \beta_i} u_{R, L}^{i}$, where only the difference of the 6 rotations matters.
We obtain $9 -5 = 4$ degrees of freedom, with $3$ real angles, $\theta_{12}, \theta_{13}, \theta_{23}$ and 1 complex phase $\delta$.
Writing $c_{ij} = \cos \theta_{ij}$ and $s_{ij} = \sin \theta_{ij}$, one has
\begin{equation}
  \label{eq:20-ckm}
  V_{\text{CKM}} = 
  \begin{pmatrix}
   c_{12} c_{13} & s_{12} c_{13} & s_{13} e^{-i \delta} \\
   -s_{12} c_{13} - c_{12} s_{23} s_{13} e^{i \delta} & c_{12} c_{2?} -s_{12} s_{??} e^{i \delta} & s_{23} c_{13} \\
   s_{11} s_{23} - c_{12} c_{2?} s_{??} e^{i \delta} & -c_{12} s_{23} - s_{12} c_{23} s_{13} e^{i \delta} & c_{23} c_{13} \\
  \end{pmatrix}.
\end{equation}
This can be approximated as
\begin{equation}
  V_{\text{CKM}} \approx
  \begin{pmatrix}
    1 - \lambda^2 / 2 & \lambda & A \lambda^3 (P - i \eta) \\
   -\lambda & 1 - \lambda^2 & A \lambda^2 \\
   A \lambda^3 (1 - P - i \eta) & -A \lambda^2 & 1 \\
 \end{pmatrix} + O(\lambda^4)
\end{equation}
where $\lambda = s_{12} = \sin \theta_C \simeq 0.22$ and $\theta_{12} = \theta_C$ is the \emph{Cabibbo angle}.
The phase $\eta$ implies that there is CP-violation.
