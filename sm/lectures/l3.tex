% lecture notes by Umut Özer
% course: sm
\lhead{Lecture 3: January 23}

\begin{description}
  \item[1960's] 
    \begin{description}
      \item[1961] Eightfold way (Gell-Mann, Neemann)

	Put order to zoo of discovered particles by considering representations of $SU(3)_{\text{flavour}}$.
	\begin{figure}[tbhp]
	  \centering
	  \def\svgwidth{0.4\columnwidth}
	  \input{lectures/eightfold.pdf_tex}
	  \caption{The eightfold way is the $8$-dimensional representation of $SU(3)$.}
	  \label{fig:eightfold}
	\end{figure}
	By considering the $10$-dimensional representation of $SU(3)$, depicted in Fig.~\ref{fig:l3f2} the $\Omega^-$ was predicted.
	\begin{figure}[tbhp]
	  \centering
	  \def\svgwidth{0.4\columnwidth}
	  \input{lectures/l3f2.pdf_tex}
	  \caption{The $10$-dimensional representation of $SU(3)$.}
	  \label{fig:l3f2}
	\end{figure}
      \item{1964} Gell-Mann, Zweig came up with the theory of \emph{quarks}.
	This theory was not accepted at the time since three quarks needed to be in the same state for some particles, violating Pauli's exclusion principle.

	\begin{itemize}
	  \item $3 \oplus \overline{3} \rightarrow$ Mesons $s =0$; $3 \otimes \overline{3} = 8 + 1$
	  \item $3 \oplus 3 \otimes 3 \rightarrow$ baryons $s = \frac{1}{2}$; $3 \otimes 3 \otimes 3 = 10 + 8 + 8 +1$
	\end{itemize}
      \item[1964] Greenberg, $1965$ (Nambu and Han) $\rightarrow$ colour
      \item[1967] Deep inelastic scattering. Evidence for substructure in the proton nucleus.

      \hrulefill

    \item[1961] Symmetry breaking (Nambu, Goldstone, Salam, Weinberg),
      % again symmetry breaking figure
      Goldstone bosons (massless)
    \item[1964] Higgs Mechanism (Higgs, Brout, Englert, Kibble, Guralnik, Haden)

      If the broken symmetry is local, then
      \begin{itemize}
	\item the gauge field is massive
	\item the Goldstone boson is is eaten and leaves behind a physical massive particle (Higgs)
      \end{itemize}

      The problem that Pauli pointed out to Yang and Mills is solved! Now you can have non-Abelian gauge symmetries, and broken symmetries.
    \item[1967-8] Weinberg, Salam, (Ward) tried non-Abelian gauge theory for the strong interaction, which failed. Trying it for the weak interaction gave Electroweak unification 
      \begin{equation}
	\underbrace{SU(2)}_{\mathclap{L}} \times \underbrace{U(1)}_{\mathclap{Y}} \to \underbrace{U(1)}_{\mathclap{\text{EM}}}
      \end{equation}
      (Glashow 1962 identified $SU(2) \times U(1)$)
    \item[1964] experimental discovery of CP violation (Cronin, Fitsch) $\implies$ particle $\leftrightarrow$ antiparticle
    \end{description}
  \item[1970's] Glashow--Illoporlos--Maiani (GIM) mechanism.
      Explain no FCNC $\implies$ new quark: \emph{charm} $c$.
      As such, the magic number of three, leading Gell-Mann to quarks, is not magic at all.
      The previous symmetry was only approximate, which was not noticed since $c$ is very massive.
      In hindsight it was obvious that we needed a fourth quark:
      \begin{description}
	\item[1969] Jackiw--Bell--Adler; Anomalies. Need partner of s: $ \begin{pmatrix} c \\ s \\ \end{pmatrix}_L $.
	\item[1973] 
	  \begin{itemize}
	    \item weak neutral currents discovered
	    \item Asymptotic Freedom (Gross, Wilczek, Politzer)
	      \begin{figure}[tbhp]
	        \centering
	        \def\svgwidth{0.5\columnwidth}
	        \input{lectures/l3f3.pdf_tex}
	        \caption{The running coupling gives hope for unification.}
	        \label{fig:l3f3}
	      \end{figure}
	  \end{itemize}
	\item[1974] $J / \psi$ discovered $\rightarrow$ \emph{charm}
	\item[1975-9] jets (quarks, gluons), for instance $e^+ e^- \to qq$ gives 2 jets, but $e^+ e^- \to q g q$ gives 3 jets.
	  \begin{equation}
	    R = \frac{e^+ e^- \to \text{hadrons}}{e^+ e^- \to \text{muons}} = \frac{33}{9}
	  \end{equation}
	  depends on the number of colours. This gave evidence for $3$ colours, confirming the idea of quarks.
      \end{description}
  \item[1980's]
    \begin{description}
      \item[1983] $Z^0, W^{\pm}$ discovered
    \end{description}
  \item[1990's]
    \begin{description}
      \item[1995] \emph{top quark} discovered. This was not a surprise since people already knew about the bottom quark, which needed a partner. We end up with three families
	\begin{equation}
	  \begin{pmatrix}
	  u \\
	  d \\
	  \end{pmatrix},
	  \begin{pmatrix}
	  c \\
	  s \\
	  \end{pmatrix},
	  \begin{pmatrix}
	  t \\
	  b \\
	  \end{pmatrix}
	\end{equation}
    \end{description}
  \item[2000's] \emph{Tau neutrino}
  \item[2012] Higgs!
\end{description}

We are lucky to be taking this course in a time where the standard model is essentially solved. In this course we will see that this structure is essentially forced on us. The structure is very rigid.
