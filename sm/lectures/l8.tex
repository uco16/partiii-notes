% lecture notes by Umut Özer
% course: sm
\lhead{Lecture 8: February 04}

\section{Discrete Spacetime Symmetries}%
\label{sec:discrete_spacetime_symmetries}

We know of two discrete symmetries already. We have seen parity $P$ , with $\Lambda_P = \text{diag}(+1, -1, -1, -1)$  and time reversal $T$  with $\Lambda_T = \text{diag}(-1, +1, +1, +1)$ .
Previously we ignored them because these are not continuously connected to the identity.
These transformations move us between the various disconnected pieces of the Lorentz group.

We can represent these as operators acting on a Hilbert space
\begin{equation}
  P = U(\Lambda_P, 0) \qquad T = U(\Lambda_T, 0).
\end{equation}
\begin{leftbar}
  It will turn out that $P$ is unitary while $T$ is not.
\end{leftbar}

Let us first consider how $P$ and $T$ act on operators of the Hilbert space.
For any $U(\Lambda, a)$ , these act as
\begin{equation}
  P U P^{-1} = U (\Lambda_P, \Lambda, \Lambda_P^{-1}, \Lambda_P a), \qquad
  T U T^{-1} = U (\Lambda_T, \Lambda, \Lambda_T^{-1}, \Lambda_T a).
\end{equation}
Infinitesimally, where we take the Lorentz transformations to be $\Lambda\indices{^{\mu}_{\nu}} = \delta\indices{^{\mu}_{\nu}} + \omega\indices{^{\mu}_{\nu}}$  and $\alpha^{\mu} = \epsilon^{\mu}$ , with $\omega\indices{^{\mu}_{\nu}}, \epsilon^{\mu} \ll 1$. We also write its unitary representation as $ U(\Lambda, a) = \mathbb{1} - \frac{i}{2} \omega_{\mu\nu} M^{\mu\nu} + i \epsilon_{\mu} P^{\mu}$.
Then
\begin{equation}
  P J_i P^{-1} = J_i, \qquad
  P K_i P^{-1} = -K_i, \qquad
  P P_i P^{-1} = -P_i, \qquad
  P P_0 P^{-1} = P_0.
\end{equation}

Naively, we expect $T P_0 T = -P_0$ , but this would imply negative energy.
\begin{theorem}[Wigner]
  Transformations on a Hilbert space preserving probabilities are either unitary and linear or antiunitary and antilinear.
\end{theorem}
\begin{proof}
  Weinberg Vol 1.
\end{proof}
\begin{description}
  \item[Unitary and Linear:] We have two states $\ket{\phi}, \ket{\psi}$. Then
    \begin{equation}
      \bra{U \phi} \ket{U \psi} = \bra{\phi}\ket{\psi} \qquad U (\alpha \phi + \beta \psi) = \alpha U \phi + \beta U \psi.
    \end{equation}
  \item[Antiunitary and Antilinear:] With similar states, we have instead
    \begin{equation}
      \bra{U \phi} \ket{U \psi} = \bra{\phi}\ket{\psi}^* \qquad U (\alpha \phi + \beta \psi) = \alpha^* U \phi + \beta^* U \psi.
    \end{equation}
    In particular, if $\alpha$ is imaginary, then it changes sign, so `$U$ does not commute with $i$'.
\end{description}
Then pick $T$ to be antiunitary and antilinear ($Ti = -i T$).
\begin{equation}
  T J_i T^{-1} = -J_i, \qquad
  T K_i T^{-1} = K_i, \qquad
  T P_i T^{-1} = -P_i, \qquad
  T P_0 T^{-1} = P_0.
\end{equation}

Now we want to know how these act on particle states.
\begin{claim}
  For massive particles, we have
  \begin{align}
    \ket{m, i; p^{\mu}, j_3} &\xrightarrow{P} \eta_P \ket{m, j; -p^{\mu}, j_3}, \\
			     &\xrightarrow{T} \eta_T (-1)^{j-j_3} \ket{m, j; -p^{\mu}, -j_3},
  \end{align}
  where $\eta_P$ and $\eta_T$ represent some phase.
  For massless particles, 
  \begin{align}
    \ket{p^{\mu}, \lambda} &\xrightarrow{P} \eta_P e^{\mp i \pi \lambda} \ket{-p^{\mu}, -\lambda}, \\
			     &\xrightarrow{T} \eta_T e^{\pm i \pi \lambda} \ket{-p^{\mu}, -\lambda}.
  \end{align}
\end{claim}
\begin{proof}
  See Weinberg Vol I.
\end{proof}

\subsection*{Comments}%

\begin{itemize}
  \item If $\lambda \neq 0$, then $\ket{p^{\mu}, \lambda} \to \ket{-p^{\mu}, -\lambda}$, so the states come in two polarisations $\pm\lambda$, so $\lambda = 0, \pm \frac{1}{2}, \pm 1, \pm \frac{3}{2}, \dots$
    This happens whenever parity is conserved. For most of the interactions, such as the photon or the graviton, this is true.
    However, for the weak interaction, parity is not conserved.
    We interpret the second helicity eigenvalue to belong to its antiparticle.
  \item For a massive particle of spin $j$, there are $2j+1$ polarisation states, since $j_3 = -j, \dots, +j$.
    For instance, a massive spin-1 particle has $3$ polarisation states. However, the massless object of helicity $\lambda$ has always only two states $\pm \lambda$. This difference between becomes even more pronounced for higher spins.
  \item We have not yet talked about any fields here. Special relativity and quantum mechanics is enough to imply the existence of particles.
    Fields will be introduced as a way to describe interactions.
\end{itemize}

\section{From Particles to Fields}%
\label{sec:from_particles_to_field}

We do not only want to see that single particle states exist, but it is also extremely important to be able to describe \emph{interactions} among many particles.
Such interactions can be described by a Hamiltonian
\begin{equation}
  H = H_0 + H_{\text{interaction}},
\end{equation}
where $H_0$  is the Hamiltonian of the free, non-interacting theory.

There are some conditions on the observables:
\begin{enumerate}[(i)]
  \item Lorentz and translation invariance
  \item Unitarity (guarantees that probabilities are conserved). Unitary evolution $U = e^{-i H t}$, where $H$ is Hermitian.
  \item Locality. This is where fields enter. We want to describe interactions at spacetime points, so we need some local structure there. The interactions are described as functions of $x$ and $t$ and those functions are the fields.
    The principle of \emph{Cluster decomposition} means that things that are far away do not interact.
\end{enumerate}
If all of these conditions are satisfied, then we are necessarily driven towards the introduction of fields.
