% lecture notes by Umut Özer
% course: sm
\lhead{Lecture 18: February 27}

\chapter{Electroweak Interactions}%
\label{cha:electroweak_interactions}

\section{Introduction}%
\label{sec:ew-introduction}

\subsection{QED Interaction Processes}%
\label{sub:qed_interaction_processes}


Recall in QED, we had the basic interaction vertex
\begin{equation}
  \begin{gathered}
    \feynmandiagram[transform shape, scale=1][small,horizontal=b to c] {
      a [particle=\(\psi\)] -- b [small, dot] -- [boson] c [particle=\(\gamma\)],
      d [particle=\(\psi\)] -- b,
    };
  \end{gathered}
\end{equation}
The current is $J^{\mu} = i \overline{\psi}{}\gamma^{\mu} \psi$.
We can have various physical processes: \par
\begin{minipage}{0.5\columnwidth}
  \begin{itemize}
  \item $e^- e^- \to e^- e^-$
  \item $e^+ e^- \to e^+ e^-$
  \item $e^- \gamma \to e^- \gamma$
  \end{itemize}
\end{minipage}%
\begin{minipage}{0.5\columnwidth}
  \begin{itemize}
    \item $e^- e^- \to e^- e^-$
    \item $e^+ e^- \to \mu^+ e^-$
    \item  \dots
  \end{itemize}
\end{minipage}
\begin{remark}
  In the whole standard model, the last interaction is not possible. However, in QED itself, the lepton number does not need to be conserved.
\end{remark}
Also, we have scalar QED with the diagrams
\begin{equation}
  \begin{gathered}
    \feynmandiagram[transform shape, scale=1][small, horizontal=b to c] {
      a [particle=\(\phi\)] -- [scalar] b [small, dot] -- [boson] c [particle=\(\gamma\)],
      d [particle=\(\phi\)] -- [scalar] b,
    };
  \end{gathered}
  \qquad
  \begin{gathered}
    \feynmandiagram[transform shape, scale=1][small, horizontal=a to c] {
      a [particle=\(\phi\)] -- [scalar] b [small, dot] -- [boson] c [particle=\(\gamma\)],
      d [particle=\(\phi\)] -- [scalar] b -- [boson] e [particle=\(\gamma\)],
    };
  \end{gathered}
\end{equation}

\subsection{Weak Interaction Processes}%

\begin{enumerate}[(i)]
  \item Leptonic:
    \begin{itemize}
      \item $\mu^- \to e^- \nu_{\mu} \overline{\nu}{}_e$
      \item $\nu_{\mu} e^- \to \nu_{\mu} e^-$
      \item $\nu_{\mu} e^- \to \nu_e \mu^-$
    \end{itemize}
  \item Semi-leptonic:
    \begin{itemize}
      \item $n \to p + e^- + \overline{\nu}{}_e$\par
	($d \to u + e^- + \overline{\nu}{}_e$)
    \end{itemize}
  \item Non-leptonic:
    \begin{itemize}
      \item $\Lambda^0 \to p + \pi^-$ \par
	($s \to u + d + u$)
    \end{itemize}
\end{enumerate}
People found all of these interactions and needed to come up with a description to unify these.
A nice proposal came from Fermi: he realised that all of these are four-body interactions, so we can have a \emph{Fermi coupling}
\begin{equation}
  \label{eq:fermi}
  \begin{gathered}
    \feynmandiagram[transform shape, scale=1][small, horizontal=a to b] {
      a [particle=\(\psi\)] -- v [small, dot, label=$G_F$] -- b [particle=\(\psi\)],
      c [particle=\(\psi\)] -- v -- d [particle=\(\psi\)],
    };
  \end{gathered}
  \quad 
  \mathscr{L} = G_F \psi^4
\end{equation}
Automatically, looking at the dimension, we have
\begin{equation}
  G_F \sim \frac{1}{M^2}
\end{equation}
and therefore the Lagrangian is non-renormalisable.

The natural thing to expect is that for energies smaller than a certain mass, this coupling is working, but needs to be replaced by something else at higher energies.
This is akin to how electromagnetism sees a direct interaction between electrons; the photon does not appear at low energies.
In such higher energies, we introduce the $W^{\pm}$ such that the first leptonic interaction becomes
\begin{equation}
  \begin{gathered}
    \feynmandiagram[transform shape, scale=1][horizontal=a to b, layered layout] {
      a [particle=\(\mu^-\)] -- [fermion] b -- [fermion] c [particle=\(\nu_{\mu}\)],
      b -- [boson, edge label=$W^-$] d -- [fermion] e [particle=\(e^-\)],
      d -- [fermion] f [particle=\(\overline{\nu}{}_e\)],
    };
  \end{gathered}
\end{equation}

\subsection*{Properties of Weak Interactions}%

The basic interaction vertex is
\begin{equation}
  \begin{gathered}
    \feynmandiagram[transform shape, scale=1][small, horizontal=b to c] {
      a -- b -- [boson, edge label=$W^{\pm}\, Z^0$] c,
      d -- b,
    };
  \end{gathered}
\end{equation}
These interactions are short range (long lifetimes), which in analogy to the Yukawa potential, going as $e^{-M / r} / r$, means that the mediators must be massive $M_{W^\pm}, M_{Z^0} \neq 0$.
For energies less than $M_{W^{\pm}}, M_{Z^0}$, we effectively have the Fermi interaction \eqref{eq:fermi}.
Moreover, interactions are \emph{chiral}, meaning that they break parity.

We know that the only way to describe these interactions in field theory valid at small and large energies is through a spontaneously broken gauge symmetry.

\subsection{Identifying the Gauge theory}%

Start with electrons and neutrinos.
We know that electrons $e_L, e_R$ can be left- and right-handed.
However, neutrinos $\nu_L$ only exist in their left-handed version, since we have not observed right-handed neutrinos in experiments.
We cannot put particles with differing chiralities into the same multiplet, since it would not be Lorentz invariant.
In general, we could have a doublet $
\begin{pmatrix}
\nu_l \\
e_L \\
\end{pmatrix} $ and a singlet, which just contains $e_R$.
This corresponds to the group
\begin{equation}
  \label{eq:ew-group}
  G = SU(2)_L \times U(1) \times U(1)_R
\end{equation}
with generators $T^a, Q_L, Q_R$ respectively.
We know the fundamental representation for $SU(2)$ is given by the Pauli matrices
\begin{equation}
  T^a = \frac{1}{2} \sigma^{a},
\end{equation}
acting on the doublet $
\begin{pmatrix}
\nu_L \\
e_L \\
\end{pmatrix} $. Moreover, the $Q_L$ and $Q_R$ act as
\begin{equation}
  Q_L \left\{ 
    \begin{gathered}
      \begin{pmatrix}
      \nu_L \\
      e_L \\
      \end{pmatrix}
      = \frac{1}{2}
      \begin{pmatrix}
      \nu_L \\
      e_L \\
      \end{pmatrix} \\
      e_R = 0
    \end{gathered}
  \right. ;
  \qquad
  Q_R \left\{ 
    \begin{gathered}
      \begin{pmatrix}
      \nu_L \\
      e_L \\
      \end{pmatrix}
      = 0 \\
      e_R = 1
    \end{gathered}
  \right.
\end{equation}
From this, we define the \emph{hypercharge} $Y$ and \emph{electron lepton number} $L_e$ as
\begin{align}
  Y &\coloneqq -Q_R - Q_L & \implies Y 
  \begin{pmatrix}
    \nu_L \\
    e_L \\
  \end{pmatrix}
  &= -\frac{1}{2}
  \begin{pmatrix}
  \nu_L \\
  e_L \\
  \end{pmatrix}
  , & 
  Y e_R &= -e_R, \\
  L_e &\coloneqq 2 Q_L + Q_R & \implies L_e 
  \begin{pmatrix}
    \nu_L \\
    e_L \\
  \end{pmatrix}
  &=
  \begin{pmatrix}
  \nu_L \\
  e_L \\
  \end{pmatrix}
  , & 
  L_e e_R &= e_R.
\end{align}
We then define the \emph{electric charge}
\begin{equation}
  \boxed{Q \coloneqq T^3 + Y}
\end{equation}
which acts on the mutliplets as
\begin{equation}
  Q 
  \begin{pmatrix}
  \nu_L \\
  e_L \\
  \end{pmatrix}
  = 
  \begin{pmatrix}
  0 \\
  -e_L \\
  \end{pmatrix}
  ,\qquad Q e_R = -e_R.
\end{equation}
But remember that the gauge group \eqref{eq:ew-group} contains also a $U(1)_{L_e}$ factor.
However, there is evidence that there is no gauge field corresponding to $U(1)_{L_e}$.
Therefore, we work with
\begin{equation}
  SU(2)_L \times U(1)_Y,
\end{equation}
with charge $Q = T^3 + Y$.

\section{Glashow--Weinberg--Salam Model}%
\label{sec:glashow_weinberg_salam_model}

An element of this group is $U = e^{i \alpha_a T_a} e^{i \beta Y}$.
The corresponding gauge fields are
\begin{align}
  SU(2)_L &\colon W_{\mu}^{a}, & W^{a}_{\mu\nu} &\coloneqq \partial_{\mu} W^{a}_{\nu} - \partial_{\nu} W_{\mu} - i g f^{abc} [W^{b}_{\mu}, W^{c}_{\nu}] \\
  U(1)_Y &\colon B_{\mu}, & B_{\mu\nu} &\coloneqq \partial_{\mu} B_{\nu} - \partial_{\nu} B_{\mu}.
\end{align}
These change as
\begin{align}
  \delta W^{a}_{\mu} &= \frac{1}{g} \partial_{\mu} \alpha^{a} - \epsilon^{abc} \alpha^{b} W^{c}_{\mu} \\
  \delta B_{\mu} &= \frac{1}{g'} \partial_{\mu} \beta,
\end{align}
where $g, g'$ are gauge couplings.
Once we impose renormalisability, the most general Lagrangian that gives us spontaneous symmetry breaking is
\begin{equation}
  \mathscr{L} = -\frac{1}{4} (W_{\mu\nu})^2 - \frac{1}{4} (B_{\mu\nu})^2 + D_{\mu} H D^{\mu} H^+ + m^2 H^+ H - \lambda (H^+ H)^2 + \kappa,
\end{equation}
where the SSB scalar $H$, the \emph{Higgs}, is a complex $SU(2)_L$ doublet.
It is by convention chosen to give the Higgs a hypercharge of $Y = \frac{1}{2}$:
\begin{equation}
  H = 
  \begin{pmatrix}
  H_+ \\
  H_0 \\
  \end{pmatrix}_{Y = 1 / 2}.
\end{equation}
The covariant derivative is
\begin{equation}
  D_{\mu} H = \partial_{\mu} H - i g W^{a}_{\mu} T^{a} H - \frac{i}{2} g' B_{\mu} H.
\end{equation}
For a broken symmetry, $\langle H \rangle \neq 0$, pick
\begin{equation}
  \langle H \rangle = \frac{1}{\sqrt{2}} 
  \begin{pmatrix}
  0 \\
  v \\
  \end{pmatrix}
  , \qquad v = \frac{m}{\sqrt{\lambda}},
\end{equation}
where $v$ is the vacuum expectation value.
Expand around the vacuum:
\begin{equation}
  H = \frac{1}{\sqrt{2}} e^{-i \xi^{a} ?? T_a}
  \begin{pmatrix}
  0 \\
  h(x) + v \\
  \end{pmatrix},
\end{equation}
where we identify the $\xi_{a}$ as the Goldstones, and $h(x)$ as the real Higgs boson.
Unitary:
\begin{equation}
  \partial_{\mu} \xi^{a} T^{a} - g W^{a}_{\mu} T^{a} - g' B_{\mu}  \to -g W_{\mu}^{a} T^{a} - g'B_{\mu},
\end{equation}
eating the Goldstones.
