% lecture notes by Umut Özer
% course: sm
\lhead{Lecture 19: February 29}

The Lagrangian is then a sum of quadratic and interaction terms
\begin{equation}
  \mathscr{L} = \mathscr{L}^{\text{quadratic}} + \mathscr{L}^{\text{interaction}}
\end{equation}
with
\begin{multline}
  \mathscr{L}^{\text{quadratic}} = -\frac{1}{4} (W_{\mu\nu}^{a})^2 - \frac{1}{4} (B_{\mu\nu})^2 + \frac{1}{2} \partial_{\mu} h \partial^{\mu} h - m^2_h h^2 \\
  + \frac{1}{2}
  \begin{pmatrix}
   0 & v \\
  \end{pmatrix}
  \left[ (i g W^{a}_{\mu} T^{a} + \frac{i}{2} g' B_{\mu}) (-ig [W^{a}]^{\mu} T^{a} - \frac{i}{2} g' B^{\mu}) \right] 
  \begin{pmatrix}
  0 \\
  v \\
  \end{pmatrix}.
\end{multline}
The term in square brackets is
\begin{equation}
  \frac{g^2 v^2}{8 ??}  \left[ (W^1_{\mu})^2 + (W^2_{\mu})^2 + (\frac{g'}{g} B_{\mu} - W^3_{\mu})^2 \right].
\end{equation}
Let us diagonalise this mass matrix by introducing
\begin{align}
  Z^0_{\mu}  &\coloneqq W^3_{\mu} \cos \theta_W - B_{\mu} \sin \theta_W, & \cos\theta_W &= \frac{g}{\sqrt{g^2 + g'{}^2}} \\
  A_{\mu} &\coloneqq W_{\mu}^3 \sin\theta_W + B_{\mu} \cos\theta_W, & \sin \theta_W &= \frac{g'}{\sqrt{g^2 + g'{}^2}},
\end{align}
where $\theta_W$ is called the \emph{weak mixing angle}.

Also, we define
\begin{equation}
  W^{\pm} \coloneqq \frac{1}{\sqrt{2}} \left( W_{\mu}^1 \mp i W^2_{\mu} \right).
\end{equation}
From this we can read off the mass spectrum, which is shown in Table \ref{tab:l19t1}.
\begin{table}[htpb]
  \centering
  \begin{tabular}{c c}
   mass & experimentally \\
   $m_h = \sqrt{2 \lambda} v = \sqrt{2} m$ & $m_h = 125.2$GeV \\
   $m_{W^{\pm}} = \frac{1}{2} g v$ & $m_W = 80.38$GeV \\
   $m_{Z^0} = \frac{1}{2} v \sqrt{g^2 + g'{}^2} = m_W / \cos \theta_W > m_W$ & $m_Z = 91.1876$GeV \\
   $m_A = 0$ & $m_\gamma < 10^{-18}$eV.
  \end{tabular}
  \caption{Electroweak boson masses.}
  \label{tab:l19t1}
\end{table}
We identify $h$ as the Higgs boson and $A$ as the photon, which we know from QED.

\subsection*{Why is $A_{\mu}$ massless?}%

The reason why $A$ is massless is that there is still some unbroken symmetry, meaning that $\delta \langle H \rangle = 0$.
Recall
\begin{align}
  \delta \langle H \rangle &= (i \alpha^{a} T^{a} + i \beta Y) \langle H \rangle \\
   &= \delta 
  \begin{pmatrix}
  0 \\
  v / \sqrt{2} \\
  \end{pmatrix}
  = \frac{i}{2 \sqrt{2}} 
  \begin{pmatrix}
   \alpha_3 + \beta & \alpha_1 - i \alpha_2 \\
   \alpha_1 + i \alpha_2 & -\alpha_2 + \beta \\
  \end{pmatrix}
  \begin{pmatrix}
  0 \\
  v \\
  \end{pmatrix}
  = \frac{iv}{2 \sqrt{2}}
  \begin{pmatrix}
  \alpha_1 - i \alpha_2 \\
  \beta - \alpha_3 \\
  \end{pmatrix}
\end{align}
If we want $\delta \langle H \rangle = 0$, then we need
\begin{equation}
  \alpha_1 = \alpha_2 = 0, \qquad \beta = \alpha_3.
\end{equation}
Thus
\begin{equation}
  U = e^{i \alpha_3 T_3} e^{i \beta Y} = e^{i \beta (T_3 + Y)} = e^{i \beta Q}
\end{equation}
This unbroken symmetry is generated by $Q$, which we identified by physical reasoning as the electric charge.
This justifies the identification of $A_{\mu}$ with the photon.
\begin{equation}
  SU(2)_L \times U(1)_Y \xrightarrow{\langle H \rangle} U(1)_{EM}.
\end{equation}
From this we can count the number of Goldstone modes
\begin{equation}
  \text{number}(\xi^{a}) = 3 = \dim SU(2) \times U(1) - \dim U(1) = 4 - 1  = 3.
\end{equation}

\subsection{Charges of Physical Fields}%
\label{sub:charges_of_physical_fields}

Consider a global transformation $e^{i \beta Q}$ and let us see what happens to the Higgs.
\begin{equation}
  \delta H = \left[ \frac{i \beta}{2} 
    \begin{pmatrix}
     1 & 0 \\
     0 & -1 \\
    \end{pmatrix}
    + \frac{i \beta}{2}
    \begin{pmatrix}
     1 & 0 \\
     0 & 1 \\
    \end{pmatrix} \right] H
    = i \beta 
    \begin{pmatrix}
     1 & 0 \\
     0 & 0 \\
    \end{pmatrix} = i \beta 
    \begin{pmatrix}
    H_+ \\
    0 \\
    \end{pmatrix}.
\end{equation}
Therefore, $H_+$ has electric charge $+1$, whereas $H_0$ has electric charge $0$.

Under the same transformation, the gauge fields change as
\begin{equation}
  \delta W_{\mu}^{a} = -\epsilon^{abc} \alpha^{b} W_{\mu}^{c}.
\end{equation}
Using $\alpha_1 = \alpha_2 = 0$ and $\alpha_3 = \beta$, we have
\begin{equation}
  \delta 
  \begin{pmatrix}
  W_{\mu}^1 \\
  W^2_{\mu} \\
  \end{pmatrix} = \beta
  \begin{pmatrix}
  W_{\mu}^2 \\
  -W_{\mu}^1 \\
  \end{pmatrix} \quad \implies \quad
  \delta W^{\pm}_{\mu} = \pm i \beta W^{\pm}_{\mu}.
\end{equation}
Therefore, $W^{\pm}_{\mu}$ have charges $\pm 1$.

Finally, 
\begin{equation}
  \delta Z^0_{\mu} = \delta A_{\mu} = 0.
\end{equation}
Therefore, both $Z^0_{\mu}$ and $A_{\mu}$ are uncharged.

\begin{remark}
  In the covariant derivative, 
  \begin{equation}
    D_{\mu} H = (\partial_{\mu} + i g W_{\mu}^3 T^3 + i g' B_{\mu} + \dots) H,
  \end{equation}
  where the second term contains $g \sin \theta_W A_{\mu}$, whereas the third term is $g' \cos \theta_W A_{\mu}$. We identify the electromagnetic coupling as
  \begin{equation}
    e = g \sin \theta_W = g' \cos \theta_W.
  \end{equation}
\end{remark}

In summary, the parameters of the electroweak theory are listed in Table~\ref{tab:l19t2}.
\begin{table}[htpb]
  \centering
  \begin{tabular}{c c c c}
    $m$ & $\lambda$ & $g$ & $g'$\\
    $e = 0.303$, & $\theta_W = \arcsin 0.223$ & $m_h = 125$GeV & $m_W = 80$GeV
  \end{tabular}
  \caption{Parameters}
  \label{tab:l19t2}
\end{table}
In particular, we obtain the fine-structure constant as
\begin{equation}
  \alpha = \frac{e^2}{2 \pi}.
\end{equation}

Let us now expand the original Lagrangian, expanded with the Higgs.
The quadratic part in the Lagrangian is
\begin{align}
  \mathscr{L}^{\text{quadratic}} = -\frac{1}{4} F_{\mu\nu}^2 - \frac{1}{4} Z_{\mu\nu}^2 - \frac{1}{2} (\partial_{\mu}  W_{\nu}^+ - \partial_{\nu} W_{\mu}^+) (\partial_{\mu} W^-_{\mu} - \partial_{\nu} W^-_{\mu})\\
  + \frac{1}{2} m^2_Z Z^{\mu} Z_{\mu} + m_W^2 W^+{}^{\mu} W^-_{\mu} + \frac{1}{2} \partial_{\mu} h \partial^{\mu} h - \frac{1}{2} m^2_h h^2
\end{align}
The interaction Lagrangian is quite unwieldy when fully expanded, and we split it into cubic and quartic pieces:
\begin{align}
  \mathscr{L}^{\text{interaction}} &= \mathscr{L}^{\text{cubic}} + \mathscr{L}^{\text{quartic}}.
\end{align}
??

Importantly, all of this depends only on the four parameters in Table~\ref{tab:l19t2}.

Examples of Feynman rules include
\begin{equation}
  \begin{gathered}
    \feynmandiagram[transform shape, scale=1][small, horizontal=a to b] {
      u [particle=\(W^+_{\mu}\)] -- [anti charged boson] a [small, dot] -- [boson] b [particle=\(Z\)],
      d [particle=\(W^-_{\nu}\)] -- [charged boson] a,
    };
  \end{gathered}
  \qquad
  \begin{gathered}
    \feynmandiagram[transform shape, scale=1][small, horizontal=u to b, layered layout] {
      u [particle=\(W^+_{\mu}\)] -- [anti charged boson] a [small, dot] -- [boson] b [particle=\(Z^0_{\alpha}\)],
      d [particle=\(W^-_{\nu}\)] -- [charged boson] a -- [boson] c [particle=\(Z^0_{\alpha}\)],
    };
  \end{gathered}
  \quad
  \begin{gathered}
    \feynmandiagram[transform shape, scale=1][vertical=u to v, small] {
      a [particle=\(W\)] -- [boson] v [small, dot] -- [boson] b [particle=\(W\)],
      u -- [scalar, edge label=$h$] v,
    };
  \end{gathered}
\end{equation}

\begin{example}[]
  Consider the process ??
\end{example}
