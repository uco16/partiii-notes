% lecture notes by Umut Özer
% course: sft
\lhead{Lecture 15: November 15}

\subsection{Epsilon Expansion in O(N) Models}%
\label{sub:epsilon_expansion_in_o_n_models}

As before, as we go away from $d = 4$, the Wilson-Fischer fixed point opens up.
\begin{equation}
  \mu_*^2 = -\frac{1}{2} \frac{N + 2}{N + 8} \Lambda^2 \epsilon \qquad \widetilde{g}_* \frac{2\pi^2}{N+8} \epsilon.
\end{equation}
From here we can go ahead and calculate all the critical exponents of the $O(N)$ Model.
\begin{equation}
  \alpha = \frac{4 - N}{2(N + 8)} \epsilon, \quad \beta = \frac{1}{2}- \frac{3}{2(N + 8)} \epsilon, \quad \gamma = 1 + \frac{N +2}{2(N + 8)} \epsilon, \quad \delta = 3 + \epsilon.
\end{equation}

\subsection{Goldstone Bosons in \texorpdfstring{$d=2$}{two dimensions}}%
\label{sub:goldstone_bosons_in_d_2}

We saw before that under spontaneous continuous symmetry breaking, we get Goldstone bosons.
The arguments that led to Goldstone bosons (difference of generators of symmetries) did not seem to care about the number of dimensions.
However, in $d = 2$ something very special happens.
Recall again the discrete $\mathbb{Z}_2$ symmetry. We saw that in the lower critical dimension $d = 1$ for the Ising model, the domain walls dominate the physics and destroy the ordered phase.
This is because it cost us some energy to change from the vacuum at $-m^0$ to $+m^0$ or vice-versa.
The fluctuations that take us from one vacuum to the other destroy the ordered phase.
%F1
For continuous symmetries, it costs us no energy. We will discover, with a slightly hand-waving argument again, that the domain wall fluctuations in the vacuum will destroy the ordered phase in $d = 2$. Afterwards, we will perform the full-blown RG calculation to see what is really going on.

\subsection*{XY-Model}%

%F2

Consider a system, which starts out with the expectation value of its Goldstone modes at $\langle \theta (\vb{x}) \rangle = 0$.
Now let us calculate fluctuations around the vacuum with the two-point correlation
\begin{equation}
  \langle [\theta(\vb{x}) - \theta(0)]^2 \rangle = 2 \langle \theta^2 (\vb{x}) \rangle - 2 \langle \theta(\vb{x}) \theta(0) \rangle
\end{equation}
The second term is
\begin{equation}
  \langle \theta(\vb{x}) \theta(0) \rangle = \frac{1}{\theta M_0^2} \int_0^\Lambda \bdd[d]{k} \frac{e^{-i\vb{k} \cdot \vb{x}}}{ k^2} \sim
  \begin{cases}
    \Lambda^{d-2} - r^{2-d} & d > 2 \\
    \log(\Lambda r) & d = 2 \\
    r - \Lambda^{-1} & d = 1
  \end{cases}
\end{equation}
This is a manifestation of the following theorem:
\begin{theorem}[Mermin-Wagner Theorem] \label{thm:mermin-wagner}
  A continuous symmetry cannot be spontaneously broken in $d = 2$.
\end{theorem}

Goldstone's theorem tells us that there should be massless modes when symmetries are spontaneously broken. However, when we study the behaviour of massless Goldstone modes in $d = 2$, we see that the fluctuations grow to arbitrary large scales, similar to the domain walls in the Ising model.

\subsection*{Sigma Models}%

Historically, particle theorists wanted to understand pions, sigma mesons, and other particles, in terms of the theory of Goldstone bosons.
Let $\phi(\vb{x}) = (\phi^1(\vb{x}), \dots, \phi_N(\vb{x}) )$.
Ordered phase $\langle \abs{\phi} \rangle \neq 0$ $\implies$ Vacuum Manifold $\sim S^{N-1}$.
We get ungapped Goldstone modes and gapped (massive) longitudinal modes.
%F2 Mexican hat with both modes marked: One along base and one along hat curve right troph
The mass is the second derivative of a potential.
Have $\phi\phi = M_0^2$. Rescale $\phi \to \vb{n} \cdot \vb{n} = 1$.
The potential is $V(\phi) = \lambda/2 (\phi \cdot \phi - M_0^2)$.
Free energy: 
\begin{equation}
  F{n} = \int \dd[d]{x} \frac{1}{2 e^2} (\grad \vb{n}) \cdot (\grad \vb{n}), \qquad e^2 = \frac{1}{\gamma M_0^2}.
\end{equation}

\subsection*{Path Integral}%

\begin{equation}
  Z = \int \pdd{\vb{n}} \delta(\vb{n}(\vb{x})^2 - 1) \exp(-\frac{1}{2e^2} \int \dd[d]{x} (\grad \vb{n}) \cdot (\grad \vb{n})), \qquad [e^2] = 2 - d.
\end{equation}
We can think of this as a coupling.
Rewriting $\vb{n}(\vb{x}) = (\boldsymbol\pi(\vb{x}), \sigma(\vb{x}))$, where $\pi$ is an $N-1$ dimensional vector, while $\sigma$ is just one field.
This tells us that $\sigma^2(\vb{x}) = 1 - \boldsymbol\pi(\vb{x}) \cdot \boldsymbol\pi(\vb{x})$.
The free energy then takes the following form:
\begin{align}
  F[\boldsymbol\pi(\vb{x})] &= \int \dd[d]{x} \frac{1}{2e^2} \left\{ (\grad \boldsymbol\pi)^2 + (\grad \sigma)^2 \right\} \\
			    &= \int \dd[d]{x} \frac{1}{2e^2} \left[ (\grad \boldsymbol\pi)^2 + \frac{(\boldsymbol\pi \cdot \grad \boldsymbol\pi)^2}{1 - \boldsymbol\pi \cdot \boldsymbol\pi} \right].
\end{align}
We see that the Goldstone bosons actually have interactions.

\subsection*{Background Field Method}%

We want to see what happens to the Goldstone bosons as we go to the IR with the renormalisation group.
Polyakov developed the background field method, which uses the geometry.
We first split the field into IR and UV modes as usual. The long wavelengths also satisfy the constraint $\widetilde{n} \cdot \widetilde{n} = 1$.
To introduce the short wavelength modes, we introduce \emph{frame fields}.
\begin{definition}[]
  The \emph{frame fields} are basis of $N-1$ unit vectors $e^{a}_{\alpha} (\vb{x})$, with $a = 1, \dots, N$ and $\alpha = 1, \dots, N-1$ that are orthogonal to $\widetilde{n} ^{a}$, meaning that $e^{a}_{\alpha} e^{a}_{\beta} = \delta_{\alpha\beta}$ and $\widetilde{n} ^{a} e^{a}_{\alpha} = 0$ for all $\alpha$.
\end{definition}
\begin{leftbar}
  \begin{remark}
    There is a rotational ambiguity in the definition of this basis. We just pick any one for this calculation.
  \end{remark}
\end{leftbar}
Introduce short wavelength modes $\chi_{\alpha}(x)$ as
\begin{equation}
  n^{a}(\vb{x}) = \widetilde{n} ^{a}(\vb{x}) (1 - \chi^2(x))^{1/2}
  + \sum_{\alpha = 1}^{N -1} \chi_{a}(x) e^{a}_{\alpha}(x) \implies n^2 = 1.
\end{equation}
The interaction term of the action is
\begin{equation}
  F_I[\widetilde{n}^{a}, \chi_{\alpha}] = \frac{1}{2e^2} \int \dd[d]{x} \left[ -\chi^2 (\grad \widetilde{n})^2 + \chi_{\alpha} \chi_{\beta} \grad e^{a}_{\alpha} \grad e^{a}_{\beta} + \cancel{2\grad \widetilde{n}^{a} \grad(\chi_{\alpha} a^{a}_{\alpha})} \right]
\end{equation}
\begin{equation}
  \langle F_I \rangle = \frac{1}{2e^2} \int \dd[d]{x} \left( -\delta_{\alpha\beta} (\grad \widetilde{n}^{a})^2 + \grad e^{a}_{\alpha} \grad e^{a}_{\beta} \right) \langle \chi_{\alpha} \chi_{\beta} \rangle
\end{equation}
\begin{equation}
  \langle \chi_{\alpha}(\vb{x}) \chi_{\beta}(\vb{x}) \rangle = e^2 \delta_{\alpha\beta} I_{d}, \qquad
  I_{d} = \frac{\Omega_{d-1}}{(2\pi)^{d}} \Lambda^{d-2} \times
  \begin{cases}
    \zeta-1 & d=1 \\
    \log(\zeta) & d = 2 \\
    1 - \zeta^{2-d} & d \geq 3
  \end{cases}
\end{equation}
Using $\widetilde{n}^{a} \widetilde{n}^{b} + e^{a}_{\alpha} e^{b}_{\alpha} = \delta^{ab}$,
\begin{equation}
  \Rightarrow \grad e^{a}_{\alpha} \grad e^{a}_{\alpha} = \grad \widetilde{n}^{a} \grad \widetilde{n}^{a} + \dots
\end{equation}
Finally, the correction to the free energy is $\langle F_I \rangle = (2 -N) I_d \int \dd[d]{x} \frac{1}{2} (\grad \widetilde{n})^2$. This gives a correction to the coupling
\begin{equation}
  \frac{1}{(e')^2} = \frac{1}{e_0^2} + (2-N) I_d
\end{equation}
After the whole RG calculation, we find the renormalised coupling
\begin{equation}
  \implies \frac{1}{e^2(\zeta)} = \zeta^{d-2} \left[ \frac{1}{e_0^2} + (2-N) I_d \right].
\end{equation}
