% lecture notes by Umut Özer
% course: sft
\lhead{Lecture 8: October 30}
\chapter{The Renormalisation Group}%
\label{cha:the_renormalisation_group}

The conceptional tools that we learn in the context of RG could be some of most important theoretical ideas of the last century. There are aspects of RG that we knew before, but the framework, developed by Kadanoff, Fischer, Wilson, and others, gave a unifying picture of these ideas.
We will mostly be working in momentum space; these guys mostly worked in real space, using the language of operator product expansions.

\begin{leftbar}
  \begin{remark}
    Wilson was a genius, but famously did not publish very often; he had published something like two papers when Pauli suggested for him to be tenured at Cornell. Nowadays, another Wilson would not exist; he likely would not even get a postdoc\dots
  \end{remark}
\end{leftbar}

\section{Basic Introduction}%
\label{sec:basic_introduction}

Consider a very general free energy
\begin{equation}
  \label{eq:general-action}
  F[\phi] = \int \dd[d]{x} \left[ \frac{1}{2} (\grad \phi)^2 +  \frac{1}{2}\mu^2 \phi^2 + g\phi^4 + \dots + g_{126} \phi^{126} + \dots + \tilde g_{2} (\laplacian \phi)^2 + \dots \right],
\end{equation}
where $\mu^2 \sim T - T_c$. The form of $F$ is only restricted by symmetry and analyticity.
We refer to the $g_i$ as \emph{Wilson coefficients}.
And we will be consistent in referring to the free energy by the more general term \emph{action}.
The partition function is then
\begin{equation}
  Z = \int \pdd{\phi} e^{-F[\phi]}.
\end{equation}
However, to fully describe the validity of the theory, we also need to specify a cutoff $\Lambda \sim \frac{\pi}{a}$, such that the high momentum fields vanish $\phi_{\vb{k}} = 0$ if $\abs{\vb{k}} > \Lambda$. 
The renormalisation group (RG) answers the question of what happens to the coupling constants when this cutoff is changed.
We usually refer to change of the Wilson coefficients as RG \emph{flow}.

\section{Flowing to the IR}%
\label{sec:flowing_to_the_ir}

The IR (infrared) is a historical term that refers to long distances.
Similarly, going to the UV (ultraviolet) refers to smaller distances---often used in synonym for high energy theories that we do not yet know about.

Imagine we only care about predictions on distance scales $L \gg a$, where $a$ is the original coarse-graining scale.
We can then define a theory with a new, slightly lower cutoff $\Lambda' = \Lambda/ \varepsilon$, where $\varepsilon > 1$. This means that all Fourier modes with $1 / L < \abs{\vb{k}} < \Lambda$ have to be accounted for.
With this new cutoff, something must have happened to the Fourier modes which were between $\Lambda'$ and $\Lambda$ that are now gone. We say that these modes have been \emph{integrated out}. As we will see, this will be a well-defined concept.

We can expand the Fourier modes as
\begin{equation}
  \phi_{\vb{k}} = \phi^-_{\vb{k}} + \phi^+_{\vb{k}},
\end{equation}
where
\begin{equation}
  \phi_{\vb{k}}^- =
  \begin{cases}
    \phi_{\vb{k}} & \abs{\vb{k}} < \Lambda' \\
    0 & \abs{\vb{k}} > \Lambda'
  \end{cases} ,
  \qquad
  \phi_{\vb{k}}^+ =
  \begin{cases}
    \phi_{\vb{k}} & \Lambda' < \abs{\vb{k}} < \Lambda \\
    0 & \abs{\vb{k}} > \Lambda
  \end{cases} .
\end{equation}
We refer to $\phi^-$ as the IR modes and $\phi^+$ as the UV modes.
The action then splits into the form
\begin{equation}
  F[\phi] = F_0 [\phi^-] + F_0 [\phi^+] + F_I [\phi^-, \phi^+],
\end{equation}
where $F_0$ is the Gaussian part and $F_I$ is the non-Gaussian part, which is often called the \emph{interaction part}, of the action.

\section{Partition Function}%
\label{sec:partition_function}

Let us see what happens to the partition function $Z$ on such a split of Fourier modes:
\begin{align}
  Z &= \int \left[\prod_{\abs{\vb{k}}< \Lambda} \dd[d]{\phi_{\vb{k}}}\right] e^{-F[\phi_{\vb{k}}]} \\
    &= \int \left[\prod_{\abs{\vb{k}}< \Lambda'} \dd[d]{\phi^-_{\vb{k}}}\right] e^{-F_0[\phi^-_{\vb{k}}]}
    \int \left[ \prod_{\Lambda' < \abs{\vb{k}} < \Lambda} \dd[d]{\phi^+_{\vb{k}}} \right] e^{-F_0[\phi^+_{\vb{k}}} e^{-F_I [\phi^+_{\vb{k}}, \phi^+_{\vb{k}}]}\\
    &= \int \left[ \prod_{\abs{\vb{k}} < \Lambda'} \dd[d]{\phi^-_{\vb{k}}} \right] e^{-F'[\phi^-_{\vb{k}}]}.
\end{align}
This defines the \emph{Wilsonian effective action} $F'$. In general, this will be of the same form as our original action in Equation \eqref{eq:general-action}, except that the coupling constants are shifted
\begin{equation}
  F'[\phi] = \int_{}^{} \dd[d]{x} \left[ \frac{1}{2} \gamma'(\grad \phi)^2 + \frac{1}{2} \mu'^2 \phi^2 + g' \phi^4 + \dots \right].
\end{equation}
However, we cannot directly compare $F'$ and $F$ to see how the theory is changed. This is because of the different cutoffs in the Fourier integrals.

\section{Momentum Rescaling}%
\label{sec:momentum_rescaling}

To compare the theories, we need to restore the cutoff from $\Lambda'$ to $\Lambda$ of the original theory.
We therefore rescale $\vb{k} \to \vb{k}' = \zeta \vb{k}$, which corresponds to $\vb{x} \to \vb{x}' = \vb{x} / \zeta$, where $\zeta = \Lambda / \Lambda'$.
Finally, we can also also do a field rescaling $\phi \to \phi'$ to absorb $\gamma'$ into the definition of the field: $\gamma' (\grad \phi)^2 \to (\grad \phi')^2$.
Note that these rescalings do not change any physical predictions; they only let us compare the two theories with each other.
The new action, with the rescaled modes and fields, is
\begin{equation}
  F_{\zeta} [\phi'] = \int_{}^{} \dd[d]{x} \left[ \frac{1}{2} (\grad \phi')^2 + \frac{1}{2} \mu^2(\zeta) \phi'^2 + g(\zeta) \phi'^4 + \dots \right].
\end{equation}
If $\varepsilon \approx 1$---if we only integrate out an infinitesimal region in Fourier space close to the cutoff---then the coupling constants change continuously. We can apply the RG transformations repeatedly to obtain an RG flow in the space of values of coupling constants.

\begin{wrapfigure}{R}{0.35\columnwidth}
  \centering
  \def\svgwidth{0.3\columnwidth}
  \input{lectures/l8f1.pdf_tex}
  \caption{RG flow in parameter space}
  \label{fig:l8f1}
\end{wrapfigure}
This space of coupling constant configurations might have \emph{basins of attractions}---regions from which you always flow to a certain fixed points, no matter which configuration of $\{g_i \}$ you started with.
At these fixed points, the theory does not change anymore under a rescaling. This scale invariance is mathematically expressed in \emph{conformal symmetry}. These conformal field theories are extremely interesting to study!
The flow of the coupling constants with scale is described by the \emph{beta functions} $\dv*{g}{\zeta} \propto \sum \beta_i.$
Note that the renormalisation group is not really a group, since there does not exist a unique inverse to each transformation---it is a semi-group.
\begin{example}[]
  Beta decay was historically understood (before we discovered the $W$ and $Z$ bosons) as a four field interaction as depicted in Figure \ref{fig:beta1}.
  \begin{figure}[bt]
    \begin{subfigure}[t]{0.5\textwidth}
      \centering
      \feynmandiagram [layered layout, horizontal=mu to a] {
	mu [particle=\(\mu\)] -- [fermion] a,
	a -- [fermion] numu [particle=\(\nu_\mu\)],
	a -- [fermion] e [particle=$e$],
	a -- [fermion] nue [particle=\(\nu_e\)]
      };
      \caption{IR theory: the four fields interact at a point.}
      \label{fig:beta1}
    \end{subfigure}
    \begin{subfigure}[t]{0.5\textwidth}
      \centering
      \feynmandiagram [layered layout, horizontal=a to b] {
	a [particle=\(\mu\)] -- [fermion] b -- [fermion] c [particle=\(\nu^\mu\)],
        b -- [charged boson, edge label=\(W^-\)] d,
	d -- [fermion] e [particle=\(e\)],
	d -- [fermion] f [particle=\(\nu_e\)],
      };
      \caption{UV theory: the interaction is mediated by a $W$-boson.}
      \label{fig:beta2}
    \end{subfigure}
    \caption{Meson decay}
  \end{figure}
  However this neglected the UV interactions. Nowadays, we understand this interaction to be mediated by a $W$-boson as depicted in Figure \ref{fig:beta2}.
\end{example}
