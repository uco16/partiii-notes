% lecture notes by Umut Özer
% course: sft
\lhead{Lecture 10: November 04}

\begin{table}[htpb]
  \centering
  \begin{tabular}{|c |c |c|}
    \hline
    Heat Capacity & $c = \pdv[2]{f}{t} \sim t^{-\alpha}$ & $\alpha = 2 - d \nu$ \\
    Magnetisation & $\phi \sim t^{-\beta}$ & $\beta = \frac{d - 2 + \mu}{2}$ \\
    Susceptibility & $\chi \sim t^{-\gamma}$ & $\gamma = \nu(2 - \eta)$ \\
    External Field & $\phi\sim B^{1 / \delta}$ & $\delta = \frac{d + 2 - \eta}{d - 2 + \eta}$ \\
    \hline
  \end{tabular}
  \caption{Critical Exponents}
  \label{tab:critexp}
\end{table}

From \ref{tab:critexp}, we can see that we have found relations between exponents that writes the four parameters in terms of only $\nu$ and $\eta$.

\begin{table}[htpb]
  \centering
  \begin{tabular}{c | c c c c c c}
     & $\alpha$ & $\beta$ & $\gamma$ & $\delta$ & $\eta$ & $\nu$ \\
     \hline
    MF & $(4 - d) / 2$ & $1/2$ & $1$ & $3$ & $0$ & $1/2$ \\
    $d = 2$ & $0$ & $1 / 8$ & $7/4$ & $15$ & $1/4$ & $1$ \\
    $d = 3$ & $0.11$ & $0.33$ & $1.24$ & $4.79$ & $0.04$ & $0.63$ \\
  \end{tabular}
  \caption{}
  \label{tab:critexpvals}
\end{table}

This shows that the scaling arguments that we employed are extremely powerful.
We were able to recover physical relationships between things like magnetisation and heat capacity by simply considering the scaling of certain quantities.

\subsection{Relevant, Irrelevant, Marginal?}%
\label{sub:relevant_irrelevant_marginal_again}

We saw that the free energy is an integral over many terms. Amongst these, we have terms of the following form
\begin{equation}
  F[\phi] \sim \int \dd[d]{x} \left\{ \dots + g_O O(x) + \dots \right\}
\end{equation}
Suppose that $O(\vb{x}) \to O'(\vb{x}') = \xi^{\Delta O} O(\vb{x})$. If we have an operator that scales like this, we know by dimensional analysis that the scaling dimension of the associated coupling constant is $\Delta g_O = d - \Delta O$.
Now we have various possible cases. We say that the operator is \emph{relevant} if $\Delta O < d$, \emph{irrelevant} if $\Delta 0 > d$, and \emph{marginal} if $\Delta O = d$.

\subsection{The Gaussian Fixed Point}%
\label{sub:the_gaussian_fixed_point}

Consider a free action
\begin{equation}
  F_0 [\phi] = \int_{\mathbb{R}^d} \dd[d]{x} \left[ \frac{1}{2} (\grad \phi)^2 + \frac{1}{2} \mu_0^2 \phi^2 \right] = \int_0^\Lambda \bdd[d]{k} \frac{1}{2}( \abs{\vb{k}}^2 + \mu_0^2) \phi_{\vb{k}} \phi_{-\vb{k}}.
\end{equation}

Following the RG steps, we split the field into UV and IR fields, giving
\begin{equation}
  \label{eq:l10gauss}
  F_0[\phi] = F_0[\phi^-] + F_0[\phi^+].
\end{equation}
The Wilsonian effective action is then obtained by considering
\begin{equation}
  e^{-F'[\phi']} = \int \pdd{\phi^+} e^{-F_0[\phi^+]} e^{-F_0[\phi^-]} = N e^{-F_0[\phi^-]}.
\end{equation}
We then need to rescale $\vb{k} \to \vb{k}' = \zeta \vb{k}$ and $\phi_{\vb{k}} \to \phi_{\vb{k}}' = \zeta^{-\omega} \phi^-_{\vb{k}}$.
Then, the effective action is
\begin{equation}
  F_0[\phi'] = \int_0^{\Lambda} \bdd[d]{k} \frac{1}{2 \zeta^d} \qty(\frac{k^2}{\zeta^2} + \mu_0^2) \zeta^{2\omega} \phi'_{\vb{k}} \phi'_{-\vb{k}}.
\end{equation}
To make this agree with the previous action, we rescale $\mu^2 (\zeta) = \zeta^2 \mu_0^2$.
Therefore, the correlation length $\epsilon \sim 1/\mu^2$ also gets rescaled to $\epsilon \to \epsilon/\zeta$.
We have fixed points whenever $\dv[]{\mu^2}{\zeta} = 0$. There are only two possibilities for this. Either, $\mu^2 = \infty$. Since $\mu^2 \sim (T - T_c)$, this corresponds to a state of infinite temperature.
The other possibility is $\mu^2 = 0$. This is the interesting one because the correlation length becomes infinite. We call this the \emph{Gaussian fixed point}.
Even within this very simple calculation, the RG analysis allows us to tell that the interesting physics lies at the point where the correlation length becomes infinite.

\subsection*{Around the Fixed Point}%

Let us now look at some of the other couplings.
We can consider a general action with higher order interactions
\begin{equation}
  F[\phi] = \int \dd[d]{x} \left[ \frac{1}{2} (\grad \phi)^2 + \frac{1}{2} \mu_0^2 \phi^2 + \sum_{n=4}^{\infty} g_{0, n} \phi^n \right]
\end{equation}
After splitting the fields, we then get an extra interaction term $F_0[\phi^-, \phi^+]$, which mixes the low and high wavenumbers, as compared to \eqref{eq:l10gauss} when splitting the action
\begin{equation}
  \label{eq:broken-action}
  F_0[\phi] = F_0[\phi^-] + F_0[\phi^+] + F_0[\phi^-, \phi^+].
\end{equation}
However, let us first consider a simple spatial rescaling.
When rescaling $\vb{x} = \vb{x} / \zeta$, and $\phi'(\vb{x}) \to \zeta^{\Delta \phi} \phi(\vb{x})$, we get
\begin{equation}
  F'[\phi'] = \int \dd[d]{x} \zeta^d \left[ \frac{1}{2} \zeta^{-2 -2\Delta \phi} (\grad' \phi')^2 + \frac{1}{2} \mu_0^2 \zeta^{-2 \Delta \phi} \phi'^2 + \sum g_{0, n} \zeta^{-n \Delta \phi} \phi'^n \right].
\end{equation}
We see that $g_n (\zeta) = \zeta^{d - n \Delta \phi} g_{0, n}$. 
In particular, considering the $\phi^4$ coupling is irrelevant for $d > 4$, marginal for $d = 4$, and relevant for $d < 4$.
%F1
This model shows us why $d = 4$ is so special.
As we are going toward the IR, this tells us that there is an enormous difference between $d < 4$ and $d > 4$.
We can tune our coupling constants in such a way that we are exactly at the Gaussian fixed point.
In general, we will have to tune each relevant parameter in the theory to hit this fixed point.
This can be seen in figure %F2
, explaining the difference of the Ising model and liquid-gas phase transitions.

\subsection{Symmetry Breaking}%
\label{sub:symmetry_breaking}

We have already seen that turning on a magnetic field $B$ in the Ising model corresponds to a $\mathbb{Z}_2$-breaking.
There is a very important thing to know: Typically, the RG evolution will not break symmetries.
We started off with some action which respected some symmetry. This symmetry is independent of momentum. This means that when breaking the fields into UV and IR parts, the new action \eqref{eq:broken-action} will also respect the symmetry.
The statement of symmetry is an RG independent statement.
However, we can also have \emph{spontaneous symmetry breaking}, where the symmetry is still present, but invisible in the IR.
If we turn on a magnetic field, we have $F \sim \alpha \phi$.
%F2
