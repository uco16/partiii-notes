% lecture notes by Umut Özer
% course: sft
\lhead{Lecture 16: November 18}

\section{Asymptotic Freedom and \texorpdfstring{$d=2+\epsilon$}{dimension two plus epsilon}}%
\label{sec:asymptotic_freedom_and_dimension_two_plus_epsilon}

Calculating the beta function as before,
\begin{equation}
  \beta(e) = \dv[]{e}{s} = (N-2) \frac{e^3}{4 \pi}
\end{equation}
for $N \geq 3$, we get a positive beta function $\beta(e) > 0$.
As we go the IR, the theory is getting more strongly coupled. Conversely, the theory is weakly coupled in the far UV.
This property is referred to as \emph{asymptotic freedom}.

In $d  =2$ dimensions, Theorem \ref{thm:mermin-wagner} of Mermin and Wagner tells us that we do not expect to have an ordered phase. This is exactly what happens here because the Goldstone bosons are so strongly interacting; this is similar to how it does not make sense to talk about quarks on large distance scales, since they are confined within a proton.
Because of this, the weakly coupled phase is unstable.

For $d = 2 + \epsilon$, 
\begin{equation}
  \dv[]{e}{s} = - \frac{\epsilon}{2} e + (N-2) \frac{e^3}{4 \pi} \Lambda_{\epsilon}.
\end{equation}
A Wilson-Fischer fixed point exists for 
\begin{equation}
  e^2_* = \frac{2\pi \epsilon}{N - 2} \Lambda^{-\epsilon}.
\end{equation}
For $\epsilon \ll 1$, this occurs at small coupling, hence exists in the UV.

\subsection{Critical Exponents}%
\label{sub:critical_exponents}

It is tempting to identify the coupling $1 / e^2 \sim 1 / T$, based on its function as a prefactor to the free energy.
This would mean we identify $e^2_*$ with the critical temperature $T_*$.
Linearising around the fixed point:
\begin{equation}
  \dv[]{(\delta e^2)}{s} = \epsilon \delta e^2.
\end{equation}
This gives us $\Delta_t$.

The scaling of the correlation length with temperature is $\xi \sim t^{-\nu}$. This tells us that $\nu = 1 / \epsilon$ (from our previous arguments of page \pageref{64}).

Plugging in $\epsilon = 1$, $N = 3$ we obtain table \ref{tab:16-exp}.
\begin{table}[tbhp]
  \centering
  \begin{tabular}{c | c c}
    & $\eta$ & $\nu$ \\
    \hline
    MF & 0 & $1/2$ \\
    $d = 4 - \epsilon$ & 0 & $0.61$ \\
    $d = 2 + \epsilon$ & $1$ & 1 \\
    actual & $0.0386$ & $0.702$ \\
  \end{tabular}
  \caption{}
  \label{tab:16-exp}
\end{table}

\section{Kosterlitz-Thouless Transition (\texorpdfstring{$N=2$}{N = 2})}%
\label{sec:kosterlitz_thouless_transition_n_2}

We saw, for $O(N \geq 3)$ that the Goldstone modes are interacting, and gapped in $d = 2$.
But the Mermin-Wagner theorem does not care about $N$; in $d = 2$, we are not supposed to have an ordered phase regardless of the value of $N$.
However, for the $X$--$Y$-model ($N = 2$), things are more interesting.
Consider the two-point function: At high $T$, $\langle \psi^{\dagger}(\vb{x}) \psi(0) \rangle = \flatfrac{e^{- r / \xi}}{\sqrt{r}}$, with $\quad \xi \sim \frac{1}{\mu^2}$.
At low $T$, write $\psi \sim M e^{i\theta}$, long distance physics dominated by $\theta$.
To leading order, we plug this into the free energy to have $F[\theta] = \frac{1}{2e^2} \int \dd[2]{x} (\grad \theta)^2$. At low $T$, $e^2 \ll 1$.
The two-point function is
\begin{equation}
  \langle \theta(x) \theta(0) \rangle = - \frac{e^2}{2\pi} \log (\Lambda r) \qquad \text{recall} \quad \sim \int \bdd[d]{k} \frac{e^{-i k \cdot x}}{k^2}.
\end{equation}
But $\langle \psi_{\dagger}(\vb{x}) \psi(0) \rangle \sim \langle e^{-i \theta(x)} e^{i \theta(0)} \rangle = \langle e^{-i (\theta(x) - \theta(0))} \rangle$. We can think of solving this as an application of Wick's theorem, since it is a Gaussian distribution, in which case
\begin{equation}
  \langle \psi_{\dagger}(\vb{x}) \psi(0) \rangle \sim \exp(- \left\langle \frac{1}{2} (\theta(x) - \theta(0))^2 \right\rangle) \sim \frac{1}{r_{\eta}} \quad \eta = \frac{e^2}{2 \pi}.
\end{equation}
In summary, at high temperature, we have an exponential behaviour, while at low temperature, we have a power law behaviour. This power law behaviour is over a range of temperatures $e^2 \sim T$, rather than one specific temperature.
This makes it seem like a phase transition is occurring!

\subsection{Vortices}%
\label{sub:vortices}

Our field $\theta(x)$ is periodic, unlike in the Ising model. We can have field configurations that wind.
This means that if we integrate over some closed contour $C$ in $\mathbb{R}^2$, we have
\begin{equation}
  \oint _C (\grad \theta) \cdot \dd[]{\vb{x}} = 2 \pi n.
\end{equation}
For $n = 1$, this is called a \emph{vortex}, whereas $n = -1$ is called an \emph{anti-vertex}.
These are examples of \emph{topological defects}; these are global properties, not local ones.
%F1 Iron filings
The free energy scales like $\sim n^2$, since we have $\theta^2$ in the free energy. This means that $n = 1$ vortices are preferred.
\begin{equation}
  F_{\text{vortex}} = \frac{1}{2e^2} \int \dd[2]{x} (\grad \theta \cdot \grad \theta) = \frac{\pi n^2}{e^2} \log (\frac{L}{a}) + F_{\text{core}},
\end{equation}
where $L$ is the IR cutoff and $a$ the UV cutoff.
Inserting this into the path integral, we get that
\begin{equation}
  p(\text{vortex}) = \left( \frac{L}{a} \right)^2 \frac{e^{-F_{\text{vortex}}}}{Z} = \frac{e^{-F_{\text{core}}}}{Z} \left( \frac{L}{a} \right)^{2 - \pi/e^2}.
\end{equation}
For $e^2 > e^2_{\text{KT}} = \frac{\pi}{2}$, the vortices are unsuppressed (KT for Kosterlitz-Thouless). And since $e^2 \sim T$, there is a phase transition at $T_{\text{KT}}$, and the power law behaviour is destroyed.
The `Kosterlitz-Thouless transition' is a \emph{topological phase transition}---driven by topological defects---which arises due to global properties.
It is responsible for superfluid films and the melting of $2D$ lattices.
