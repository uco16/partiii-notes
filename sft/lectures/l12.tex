% lecture notes by Umut Özer
% course: sft
\lhead{Lecture 12: November 08}
Equipped with Wick's theorem, we are now able to deal with the term $\langle \phi^+_{\vb{q}_1} \phi^+_{\vb{q}_2}\phi^+_{\vb{q}_3}\phi^+_{\vb{q}_4} \rangle$ in \eqref{eq:11-order-2}.
Let us insert the result \eqref{eq:11-wick}. It turns out that the term $\langle \phi^+_{\vb{q}_1}\phi^+_{\vb{q}_2} \rangle \langle \phi^+_{\vb{q}_3}\phi^+_{\vb{q}_4} \rangle$ is cancelled by the corresponding term in $\langle F_I \rangle^2$. This corresponds to the fact that we only need to calculate connected cumulants in the logarithmic expansion $\log \langle e^{-F_I} \rangle$.
The next term we consider is $\langle \phi^+_{\vb{q}_1}\phi^+_{\vb{q}_3} \rangle \langle \phi^+_{\vb{q}_2}\phi^+_{\vb{q}_4} \rangle$. The second order contribution to the effective action associated with this term is
\begin{align}
  &\int \left[ \prod_{j=1}^4 \dd[d]{q_i} \right] \langle \phi^+_{\vb{q}_1}\phi^+_{\vb{q}_3} \rangle \langle \phi^+_{\vb{q}_2}\phi^+_{\vb{q}_4} \rangle_+ \delta^d (\vb{k}_1 + \vb{k}_2 + \vb{q}_1 + \vb{q}_2) \delta^d (\vb{k}_3 + \vb{k}_4 + \vb{q}_3 + \vb{q}_4) \\
		      &= \int \dd[d]{q_1} \dd[d]{q_2} G_0 (q_1) G_0(q_2) \delta^d (\vb{k}_1 + \vb{k}_2 + \vb{q}_1 + \vb{q}_2) \delta^d(\vb{k}_3 + \vb{k}_4 - \vb{q}_1 - \vb{q}_2) \\
		      &= \int \dd[d]{q} G_0(q) G_0(\abs{\vb{k}_q + \vb{k}_2 + \vb{q}}) \delta^d(\vb{k}_1 + \vb{k}_2 + \vb{k}_3 + \vb{k}_4)
\end{align}
Where we used out previous result for the propagator:
\begin{equation}
  \langle \phi^+_{\vb{q}_1} \phi^+_{\vb{q}_3} \rangle_+ = \bdelta^d(\vb{k}_1 + \vb{k}_3) G_0(\vb{k}_1), \qquad G_0(\vb{k}_1) = \frac{1}{\vb{k}_1^2 + \mu_0^2}.
\end{equation}
As a result, we find that
\begin{equation}
  \frac{1}{2} \langle F_I^2 \rangle_+ \sim \left( \frac{4!}{2!} \right)^2 g_0^2 \int_{0}^{\Lambda/\zeta} \left[ \prod_{i=1}^4 \bdd[d]{k_i} \phi_{\vb{k}_i}^- \right] f(\abs{\vb{k}_1 + \vb{k}_2}) \bdelta^d(\sum_i \vb{k}_i),
\end{equation}
where the function $f$ is
\begin{equation}
  f(\vb{k}) = \int_{\Lambda/\zeta}^{\Lambda}  \bdd[d]{q} \left[ \frac{1}{q^2 + \mu_0^2} \times \frac{1}{(\vb{k} + \vb{q})^2 + \mu_0^2} \right].
\end{equation}
We can Taylor expand this 
\begin{equation}
  f(\vb{k}) \approx \int_{\Lambda/\zeta}^{\Lambda} \frac{\bdd[d]{q}}{(q^2 + \mu_0^2)^2} \left[ 1 + O(\abs{\vb{k}}^2) \right].
\end{equation}
Finally, we rescale 
\begin{equation}
  g_0 \to g_0' = g_0 - 36 g_0^2 \int_{\Lambda/\zeta}^\Lambda \frac{\dd[d]{q}}{(q^2 + \mu_0^2)^2}.
\end{equation}
\begin{leftbar}
  \begin{remark}
    `Do not dismiss this calculation, as it underpins your very own existence.'
  \end{remark}
\end{leftbar}

\subsection{Feynman Diagrams}%
\label{sub:feynman_diagrams}

In Feynman diagram language, things like the cancellation of disconnected cumulants are translated very naturally in the fact that only connected diagrams are allowed.
When we expand $\log \langle e^{-F_I[\phi_{\vb{k}}^-, \phi^+_{\vb{k}}]} \rangle$, we get terms of the form $g_0^p (\phi^-)^n (\phi^+)^l$. Each of these terms gives an integral. We can keep track of the various different terms arising from this expansion by using Feynman diagrams.
Let us recap the Rules (similar to \emph{Quantum Field Theory}):
\begin{itemize}
  \item Each $\phi^-_{\vb{k}}$ is an external solid line.
  \item Each $\phi^+_{\vb{k}}$ is an internal dotted line.
  \item Dotted lines are always connected at both ends.
  \item Each vertex joins $4$ lines with $g_0$.
  \item Each line has momentum $\vb{k}$, conserved at vertices.
  \item Symmetry factors
\end{itemize}
\begin{leftbar}
  \begin{remark}
    A common misconception is that Feynman diagrams are just a useful visualisation for the physics of scattering of particles.
    However, Feynman diagrams in this context are much more than that; we should think of each diagram as an equation; computing terms in the perturbative expansion of the effective action with the above rules.
  \end{remark}
\end{leftbar}
At lowest order, we have the following diagrams:
\begin{align}
  \text{`Tree'} \qquad
  \feynmandiagram[inline=(v.base), horizontal=a to b] {
    a -- [fermion,  edge label=\(\vb{k}_1\)] v [dot] -- [anti fermion,  edge label=\(\vb{k}_4\)] b,
    c -- [fermion,  edge label'=\(\vb{k}_2\)] v -- [anti fermion, edge label'=\(\vb{k}_3\)] d,
  };
  &\implies g_0 \int \dd[4]{x} (\phi^-)^4 \\
  \text{`Loop'}
  \feynmandiagram[inline=(a.base)] {
    a -- [scalar, loop, min distance=2cm, in=-135, out=-45] a,
    a -- [scalar, loop, min distance=2cm] a,
  };
  &\sim g_0(\phi^+)^4 \\
  \text{Interesting}
  \feynmandiagram[inline=(v.base), horizontal=a to b] {
    a -- [draw=none] b,
    a -- [fermion,  edge label=\(\vb{k}_1\)] v [particle=\(g_0\)] -- [scalar, loop, min distance=2cm] v,
    v -- [anti fermion, edge label=\(\vb{k}_2\)] b,
  };
  &= 6g_0 \int_{\Lambda/\zeta}^\Lambda \frac{\bdd[d]{q}}{q^2 + m_0^2} \int \dd[d]{x} (\phi^-)^2
\end{align}
\begin{leftbar}
  \begin{remark}
    Symmetry factor: how many ways are there to connect two lines to four dots?
  \end{remark}
\end{leftbar}
At second order, we have the following diagrams
\begin{equation}
  \feynmandiagram[inline=(a.base), horizontal=a to b] {
    i1 -- [fermion, edge label'=\(\vb{k}_1\)] a -- [anti fermion, edge label=\(\vb{k}_2\)] i2,
    a -- [charged scalar, half left, edge label=$k_1 + k_2 + q$] b -- [charged scalar, half left, edge label=$q$] a,
    f1 -- [fermion, edge label=\(\vb{k}_3\)] b -- [anti fermion, edge label=\(\vb{k}_4\)] f2,
  };
  \implies 36 g_0^2 \int_0^{\Lambda/\zeta} \left[ \prod_{i=1}^4 \bdd[d]{k_i} \phi_{\vb{k}_i}^- \right] f(\vb{k}_1 + \vb{k}_2) \bdelta^d(\sum_i \vb{k}_i)
\end{equation}
Also we have disconnected diagrams, which are cancelled by $\langle F_I \rangle^2$.
We also have some extra diagrams at second order, which correspond to diagrams that we have not considered before
\begin{equation}
  \feynmandiagram[inline=(a.base), horizontal=a to b, layered layout] {
    a -- [fermion, edge label=\(\vb{k}\)] b -- [scalar] c -- [anti fermion, edge label=\(\vb{k}\)] d,
    b -- [scalar, half left] c -- [scalar, half left] b,
  };
  \implies g_0^2 \int \bdd[d]{k} \frac{1}{2} A(k, \Lambda) \phi^-_{\vb{k}} \phi^-_{-\vb{k}},
\end{equation}
where $A(k, \Lambda)$ can be expanded as
\begin{equation}
  A(k, \Lambda) \approx A(0) + \frac{1}{2} k^2 A'' (0) + \dots
\end{equation}
The first term $A(0)$ is a new correction to $\mu^2$ while $A''(0)$ corrects $(\grad \phi)^2$.
The fact that this type of correction only pops up at two loops is very special to $\phi^4$ theory.

\subsection{\texorpdfstring{$\beta$}{Beta}-Functions}%
\label{sub:beta_functions}

Let us write the new cutoff as $\Lambda' = \Lambda/\zeta = \Lambda e^{-s}$.
We then define the beta functions $\beta_n(g_n)$ as 
\begin{equation}
  \beta_n(g_n) \coloneqq \dv[]{g_n}{s} 
  \begin{cases}
    > 0 & \text{$g_n$ in the IR}\\
    < 0 & \text{otherwise}
  \end{cases}
\end{equation}
\begin{equation}
  \dv[]{g_n}{s} = (d - \frac{1}{2} n d + n ) g_n, \qquad g_n(s) = \exp{\frac{d - nd/2 + n}{s}} g_{0, n}
\end{equation}
