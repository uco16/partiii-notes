% lecture notes by Umut Özer
% course: sft
\lhead{Lecture 7: October 28}
\section{Green's Functions}%
\label{sec:green_s_functions}

Consider a multi-dimensional Gaussian. This is not a path integral, but rather a large but finite number of integrals.
We can then diagonalise this, separating this integral into $n$ Gaussian integrals. This will evaluate to the square root of the product of the eigenvalues, which gives the determinant
\begin{equation}
  \int_{-\infty}^{\infty} \dd[n]{y} e^{-\frac{1}{2} \vb{y}^T \cdot G^{-1} \cdot \vb{y}} = \sqrt{\det(2\pi G)}.
\end{equation}
Similarly, we can deal with an extra linear term $\vb{B}^T \cdot \vb{y}$ by changing variables and completing the square. This gives
\begin{equation}
  \int_{-\infty}^{\infty} \dd[n]{y} e^{-\frac{1}{2} \vb{y}^T \cdot G^{-1} \cdot \vb{y} + \vb{B}^T \cdot \vb{y}} = \sqrt{\det(2\pi G)} e^{\frac{1}{2} \vb{B}^T \cdot G \cdot \vb{B}}.
\end{equation}

Let us apply this to path integrals now. Consider again the action
\begin{align}
  F[\phi(x)] &= \int_{}^{} \dd[d]{x} \left[ \frac{1}{2} \phi(\grad\phi)^2 + \frac{1}{2} \mu^2\phi^2 + B \phi\right] \\
	     &= \iint \dd[d]{x} \dd[d]{y} \frac{1}{2} \phi(x) G^{-1} (x, y) \phi(y) + \int_{}^{} \dd[d]{x} B(x) G(x)
\end{align}
Then we have the inverse $G^{-1}(x, y) = \delta^d(x - y) (-\gamma\laplacian_{y} + \mu^2)$ in the sense that it acts on $G(x, y)$ to give, in analogy to matrix multiplication,
\begin{equation}
  \int_{}^{} \dd[4]{z} G^{-1}(x, z) G(z, y) = \delta^d(x, y).
\end{equation}
The correlation function is then
\begin{equation}
  (-\phi \laplacian_x + \mu^2) \langle \phi(x) \phi(0) \rangle = \frac{1}{\beta}\delta^d(x).
\end{equation}
This is telling us that the fluctuations solve the Euler-Lagrange equations.
\begin{leftbar}
  \begin{remark}
    Green's functions allow us to solve differential equations:
    Consider a differential operator $O(x)$ acting on a function $F(x)$ to give
    \begin{equation}
      \label{eq:l_7_operator}
      O(x) F(x) = H(x).
    \end{equation}
    If we can solve the equation
    \begin{equation}
      O(x) G(x, y) = \delta^d(x - y),
    \end{equation}
    then we can solve equation \eqref{eq:l_7_operator} by integrating
    \begin{equation}
      F(x) = \int_{}^{} \dd[d]{y} H(y) G(x - y)
    \end{equation}
    so that as expected we have
    \begin{equation}
      O(x) F(x) = \int_{}^{} \dd[d]{y} H(y) \delta^d(x - y) = H(x).
    \end{equation}
  \end{remark}
\end{leftbar}

\subsection{Connection to Susceptibility}%
\label{sub:connection_to_susceptibility}

In the mean field theory, we defined the magnetic susceptibility to be $\chi = \pdv*{m}{B}$. In the continuum limit $m \to \phi(x)$, where we go from mean-field theory to a theory with local fluctuations, we can define a local version of the magnetic susceptibility as
\begin{equation}
  \chi(x, y) \coloneqq \fdv{\phi(x)}{B(y)} = \beta \langle \phi(x) \phi(y) \rangle \sim \int \pdd{\phi} \fdv{\phi}{B} e^{\beta F[\phi]}
\end{equation}
We can also define a global susceptibility by integrating over all space
\begin{equation}
  \chi = \int_{}^{} \dd[d]{x} \chi(x, 0) = \beta \int_{}^{} \dd[d]{x} \langle \phi(x) \phi(0) \rangle.
\end{equation}

We can now start to ask a few more physical questions.
Recall that we found the functional form of the correlator to have a particular scaling behaviour:
\begin{equation}
  \langle \phi(x) \phi(y) \rangle \sim 
  \begin{cases}
    \frac{1}{r^{d-2}} & r \ll \varepsilon \\
    \frac{e^{-r/2}}{r^{(d-1)/2}} & r \gg \varepsilon.
  \end{cases}
  \implies 
  \mu^2 \sim \abs{T - T_c}, \text{ and } \xi \sim \frac{1}{\abs{T - T_c}^{1/2}}.
\end{equation}
This means that correlations are related across the entire system, across all distance scales, around the critical temperature $T_c$.
Physically, we intuitively understand that at high temperatures $T \gg T_c$, there is so much energy in the system, so many fluctuations, that different parts of the system are practically uncorrelated.
As we lower the temperature, the fluctuations decrease in magnitude and the size of patches with aligned spins grows, meaning that the correlation length starts to grow.
As $T \to T_c$, we find patches of all shapes and sizes at all scales, as the correlation length goes to infinity.

\section{Critical Exponents (Again\ldots)}%
\label{sec:critical_exponents_againldots_}

The correlation length goes like $\xi \sim \abs{T - T_c}^{-\nu}$, where $\nu = \frac{1}{2}$ and $\langle \phi(x) \phi(y) \rangle \sim r^{-(d-2 + \eta)}$, where $\eta = 0$.


\begin{table}[htpb]
  \centering
  \begin{tabular}{|c|c|c|c|}
    \hline
    & MFT & $d = 2$ & $d = 3$ \\
    \hline
    $\eta$ & 0 & ${1}/{4}$ & $0.0363\dots$ \\
    $\nu$ & $\frac{1}{2}$ & $1$ & $0.6300$ \\
    \hline
  \end{tabular}
  \caption{caption}
  \label{tab:label}
\end{table}


\subsection{Upper Critical Dimension}%
\label{sub:upper_critical_dimension}

Recall that from mean-field theory, the one point correlation is $\langle \phi(x) \rangle = \pm m_0$. The path integral told us that there are also fluctuations. 
A sensible question to ask is: How big are the fluctuations relative to $m_0$?
This tells us whether the mean-field theory is good or bad.
\begin{equation}
  R = \frac{\int_{0}^{\xi} \dd[d]{x} \langle \phi(x)\phi(0) \rangle}{\int_{0}^{\xi} \dd[d]{x} m_0^2} \sim \frac{1}{m_0^2 \xi^d} \int_{0}^{\xi} \dd[]{r} \frac{r^{d-1}}{r^{d-2}} \sim \frac{\xi^{2-d}}{m_0^2}
\end{equation}
where we used that in polar coordinates, $\dd[d]{x} = r^{d-1} \dd[]{r} \dd[]{\Omega_d}$.
Therefore $m_0 \sim \abs{T - T_c}^{1/2}$ and $\xi = \abs{T - T_c}^{-1/2}$ implies that $R = \abs{T-T_c}^{(d-4)/2}$.

\section{Translating to QFT}%
\label{sec:translating_to_qft}

There is a very strong and non-accidental connection between the tools we are using in this course and the tools used in quantum field theory.
In statistical field theory, we have a path integral $Z = \int \pdd{\phi} e^{-\beta \int_{}^{} \dd[d]{x} F[\phi]}$. Here, $d$ is entirely spatial. There is no imaginary term.

In Quantum field theory, we have
\begin{equation}
  Z = \int \pdd{\phi} e^{\frac{i}{\hbar} \int \dd[d]{x} \mathcal{L}[\phi]}.
\end{equation}
Here $d$ is spacetime and we have a factor of $\frac{i}{\hbar}$.

To translate from one to the other, we can use a \emph{Wick rotation}, which is a change of coordinates such that $\tau = it$. This means that our operators become
\begin{equation}
  \partial_t^2 - \laplacian \to -(\partial_\tau^2 + \laplacian).
\end{equation}
The quantum field theory path integral then becomes a Euclidean path integral such as the one in statistical field theory:
\begin{equation}
  Z = \int \pdd{\phi} e^{\frac{i}{\hbar} S[\phi]} \to \int \pdd{\phi}e^{-\frac{1}{\hbar} S_E[\phi]}.
\end{equation}

\begin{table}[htpb]
  \centering
  \begin{tabular}{|c|c|}
    \hline
    High energy physics & Statistical field theory \\
    \hline
    Quantum fluctuations & Thermal fluctuations \\
    QFT in $d-1$ dimensions & SFT in $d$ dimensions\\
    $\hbar$ & $\beta$ \\
    \hline
  \end{tabular}
  \caption{Analogy between QFT and SFT.}
  \label{tab:analogy}
\end{table}
There are places where this analogy breaks down. For example, there are situations where there are important topological properties of QFT. In this case, the analogy becomes limited; it is difficult to phrase topological questions in the language of statistical mechanics.
