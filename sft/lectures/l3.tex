% lecture notes by Umut Özer
% course: sft
\lhead{Lecture 3: October 18}
\chapter{Landau-Ginzburg Free Energy}%
\label{cha:landau_ginzburg_free_energy}

\section{Model Building}%
\label{sec:model_building}

This will be the first attempt of writing down a \emph{theory} by postulating a free energy. In the case of particle physics, this corresponds to the action.
There are certain guiding principles that help construct such a theory. Such `common sense' principles include
\begin{description}
  \item[Locality] All interactions are \emph{local}. Only nearest neighbours matter.
    This means the free energy functional is just one integral over one spatial component
    \begin{equation}
      F[m(x)] = \int_{}^{} \dd[d]{x} f[m(x)].
    \end{equation}
    In general, one could imagine non-local theories where we have terms of the form \[\int_{}^{} \dd[d]{x} \dd[d]{y} f[m(x), m(y)]. \]
    In the continuum Ising model, we have some almost non-local terms. These are the gradient terms
    \begin{equation}
      \pdv{m(\vb{x})}{\vb{x}} \simeq \lim_{a \to 0} \frac{m(\vb{x} + a\hat{\vb{x}}) - m(\vb{x})}{a} . 
    \end{equation}
    There are certain pitfalls, such as causality violation, which arise when we can include non-local terms. For more information, see Weinberg Volume 1.
  \item[Translational and Rotational Invariance] The symmetry of the continuum theory should reflect the symmetries of the lattice. The Ising model has a large-scale rotational symmetry, which will be recovered when going to the continuum limit.
    Similarly, there is a discrete translational symmetry which will become a continuum symmetry.
    This means, we have terms like $(\grad m)^2$, but not $\cancel{\vb{n} \cdot \grad m}$.

    Moreover, the Ising model with $B = 0$ has a $\mathbb{Z}_2$ symmetry: $S_i \to -S_i$. Therefore, the continuum limit must have a symmetry under $m(\vb{x}) \to -m(\vb{x})$.
  \item[Analyticity] We require the free energy $F$ to have a well-behaved Taylor expansion in the magnetisation field $m(\vb{x})$.
    This allows terms like $m^2(\vb{x})$, $(\grad m(\vb{x}))^4$, \ldots, $m^8(\vb{x})$. However, we are not allowed to have non-analytic terms like $\sqrt{m(\vb{x})}$, $\log(m(\vb{x}))$, or $\frac{1}{m(\vb{x})}$.
    \begin{leftbar}
      \begin{remark}
        Non-analyticities like these usually signal that we overlooked some important ingredient in the theory.
      \end{remark}
    \end{leftbar}
  \item[Derivative Expansion] If the variation of the gradient is slow over the length scale $a$ of the lattice spacing, $(a \grad) \grad m(\vb{x}) \ll \grad m(\vb{x})$, then higher derivatives can be ignored.
    As we go to higher number of derivatives, we go to more separated neighbour interactions.
\end{description}

\begin{leftbar}
  \begin{remark}
    The Higgs interaction, among many others, starts to generate logarithmic terms when we coarse-grain high-energy processes.
  \end{remark}
\end{leftbar}

\section{Zero external field}%
\label{sec:zero_external_field}

When $B = 0$, the free energy is
\begin{equation}
  F[m(\vb{x})] = \int_{}^{} \dd[d]{x} \left[ \frac{1}{2} \alpha_2(T) m^2 + \frac{1}{4} \alpha_4(T)m^4 + \frac{1}{2} \gamma(T) (\grad m)^2 + \ldots \right],
\end{equation}
where $m = m(\vb{x})$, $\alpha_2(T) \simeq T - T_c$, and $\alpha_4(T) \sim \frac{1}{3}T$.

\subsection{The Saddle Point Approximation}%
\label{sub:the_saddle_point_approximation}

We want to find minimise the free energy functional $F[m(\vb{x})]$.
Consider a field configuration $m(\vb{x})$ and perturb it by $\delta m(\vb{x})$.
The free energy then changes by
\begin{align}
  \delta F &= \int_{}^{} \dd[d]{x} \left[ \alpha_2 m \delta m + \alpha_4 m^3 \delta m + \gamma \grad m \cdot \grad \delta m \right] \\
	   &= \int_{}^{} \dd[d]{x} \left[ \alpha_2 m + \alpha_4 m^3 - \gamma \laplacian m \right] \delta m
\end{align}
The implicit assumption here is that the variation vanishes at the boundaries of the integral.
We can take this as a definition of the functional derivative
\begin{equation}
  \fdv{f}{m} = \alpha_2 m + \alpha_4 m^3 - \gamma \laplacian m.
\end{equation}
For an Extremum, $\left.\fdv{F}{m}\right\rvert_m = 0$ $\implies \gamma \laplacian m = \alpha_2 m + \alpha_4 m^3$. This is sometimes called the \emph{Euler-Lagrange equation} or the \emph{equation of motion}.

\subsubsection{Solutions}%
\label{subsub:solutions}

The simplest solution is $m=\text{const}$. There are two cases:
\begin{equation}
  m = 
  \begin{cases}
    0 & \alpha_2 > 0 \\
    \pm m_0 = \pm \sqrt{\frac{-\alpha_2}{\alpha_4}} & \alpha_2 < 0
  \end{cases}
\end{equation}
This recovers the mean-field approximation, confirming that our original mean-field Ansatz was not so bad after all.
Moreover, analysing the other solutions allows us to understand the mean-field solution better.
The other solutions are called \emph{domain walls}.

\subsubsection{Domain Walls}%
\label{subsub:domain_walls}

We can also imagine that different parts of the lattice take on different solutions, so that both $\pm m_0$ coexist. In that case, there will be an asymptotic transition function between the domain walls at spatial infinity. 
\begin{figure}[tbhp]
  \centering
  \def\svgwidth{0.5\columnwidth}
  \input{lectures/domainwall.pdf_tex}
  \caption{The transition function asymptotes to the domain walls as $x = \abs{\vb{x}} \to \infty$}
  \label{fig:domainwall}
\end{figure}
This transition function cannot just be a step function, since the derivative would be discontinuous, giving infinite energy.
Solving $\gamma \dv[2]{m}{x} = \alpha_2 m + \alpha_4^2$ gives
\begin{equation}
  m = m_0 \tanh(\frac{x - x_0}{w}), \qquad w = \sqrt{\frac{-2\gamma}{\alpha}}.
\end{equation}
The energy associated with this solution can be calculated to give $F_{\text{DW}} - F_{\text{GS}} \sim L^{d-1} \sqrt{\frac{-\gamma \alpha_2^3}{\alpha_4}}$.
Here we see something interesting happening; in one dimension, the free energy is independent of the size of the system, whereas in higher dimensions it does.
This exponent $d-1$ will tell us about why the mean-field theory fails in one-dimension: the domain-walls completely take over and dominate over the mean-field theory contributions.
