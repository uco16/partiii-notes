% lecture notes by Umut Özer
% course: sft
\lhead{Lecture 2: October 14}

\chapter{Landau Approach to Phase Transitions}%
\label{cha:landau_approach_to_phase_transitions}

\begin{description}
  \item[Landau Theory] Free energy + Symmetry
  \item[Effective Free Energy] $F(m) =N f(m)$, where $m$ is the \emph{order parameter}
    \begin{align}
      f(m) &= -Bm - Jqm^2 - T[\log(2) - \frac{1}{2} (1 + m) \log(1 + m) - \frac{1}{2}(1- m) \log(1-m)]\\
      &\simeq - T \log(2) - Bm + \frac{1}{2}(T - Jq) m^2 + \frac{1}{12} Tm^4 + \cdots
    \end{align}
  \item[Saddle Point] Equilibrium at minimum of $f(m)$
\end{description}

\section{Phase transition in the Ising Model}%
\label{sec:phase_transition_in_the_ising_model}

Without external magentic field $B = 0$, we have
\begin{equation}
  f(m) =\frac{1}{2}(T - Jq)m^2 + \frac{1}{12}Tm^5 + \cdots
\end{equation}
 
%F1

As we lower the temperature the potential becomes flat. Below the critical temperature $T_C$, we obtain two minima at $\pm m_0$, where
\begin{equation}
  m_0 = \sqrt{\frac{3(T_c - T)}{T}}
\end{equation}

We call a phase transition an \emph{$n$-th order phase transition} if it has a discontinuity in the $n$-th derivative.
% F2
In this case, we have a second order phase transition. 

\begin{leftbar}
  \begin{remark}
    The reason for this discontinuity is the large number of degrees of freedom. Since $N \rightarrow \infty$, discontinuities show up, but on the level of individual spins, all processes are continuous.
  \end{remark}
\end{leftbar}

\begin{definition}[Heat Capacity]
  The heat capacity of the system is
\begin{equation}
  C = \pdv{\langle E \rangle}{T} = \beta^2 \pdv[2]{\log(Z)}{\beta}, \qquad E = -\pdv{\log(Z)}{\beta}
\end{equation}
  where $Z = e^{-\beta N f(m_{\text{min}})}$.
\end{definition}
In the Landau Approach:
\begin{align}
  T &> T_C \qquad & f(m_{\text{min}}) &= 0 \\
  T &< T_c & f(m_{\text{min}}) &= -\frac{3}{4} \frac{(T_c - T)^2}{T}
\end{align}

% F3

\section{Spontaneous Symmetry Breaking}%
\label{sec:spontaneous_symmetry_breaking}

At high temperatures, we have
\begin{equation}
  f(m)= \frac{1}{2}(T - T_c)m^2 + \frac{1}{12} T m^4.
\end{equation}
This has a $\mathbb{Z}_2$ symmetry under interchange $m \to -m$.

\begin{definition}[Spontaneous Symmetry Breaking]
The ground state does not respect a symmetry of the action / free energy.
\end{definition}

\section{Non-Zero Magnetic Field}%
\label{sec:non_zero_magnetic_field}

If we have $B \neq 0$, then we have a term that is linear in $m$:
\begin{equation}
  f(m) = -Bm + \frac{1}{2}(T-T_c) m^2 + \frac{1}{12} T m^4 + \cdots
\end{equation}

% F4

In the absence of the magnetic field, the two tracks split. However, in this case, the tracks do not meet and there is no discontinuity and no phase transition either.
However, we can manufacture a phase transition in the following way.

\subsection{Manufacturing a Phase Transition}%
\label{sub:manufacturing_a_phase_transition}

% F5
We increase the magnetic field in such a way as to increase the energy of the state in which the system found itself via spontaneous symmetry breaking.
The system will remain in that state, even until we increase the magnetic field so much that it is only a local, not the global minimum.
After some time spent in this meta-stable state, the magnetisation jumps to the lower energy state.
This is a first order phase transition.
% F6

At the critical point $T = T_c$, we can solve for the magnetisation as a function of the magnetic field. We find that 
\begin{equation}
  m \sim 
  \begin{cases}
    B^{1/3} & B < 0 \\
    -\abs{B}^{1/3} & B > 0
  \end{cases}
\end{equation}

\begin{definition}[Magnetic Susceptibility]
  \begin{equation}
    \chi = \pdv{m}{B}\rvert_T
  \end{equation}
\end{definition}

Since we have $m \simeq \frac{B}{T - T_c}$, the susceptibility is $\chi \sim \abs{T - T_c}^{-1}$.
These are predictions which can be tested in experiments.

\section{Validity of Mean Field Theory}%
\label{sec:validity_of_mean_field_theory}

The mean field theory (MFT) gave an excellent qualitative picture of the phase transitions. But how well does it actually work?
There were a number of assumptions: We neglected any spatial variations by assuming that any spin takes the same value as the average. We also neglected any interactions, such as corrections from the quartic term in $f(m)$.

It turns out that whether or not MFT works depends on the number of dimensions:
\begin{description}
  \item[$d = 1$:] Total Failure. In one dimension there is never any phase transition. We call this the \emph{lower critical dimension} $d_l$.
  \item[$d = 2,3$:] Works-\emph{ish}. The qualitative picture is right, but the details are not correct.
  \item[$d \geqslant~4$:] Works remarkably well. In $d \geq 4$, it gives exactly the right answer and matches experiment.
    We call this the \emph{upper critical dimension} $d_c$.
\end{description}

In general theories, not restricted to the Ising model, we call the dimension in which MFT fails completely the \emph{lower critical dimension} $d_l$ and the theory for which it always works the \emph{upper critical dimension} $d_c$.

\begin{leftbar}
  \begin{remark}
    This can be understood in terms of the number of nearest-neighbours. The fluctuations grow in a different way depending on the dimension of the system.
  \end{remark}
\end{leftbar}

\section{Critical Exponents}%
\label{sec:critical_exponents}

We found that in the absence of an external magnetic field, the magnetisation scaled as $m \sim \abs{T_c - T}^{\beta}$, where $\beta = \frac{1}{2}$.
Moreover, the heat capacity scaled as $C \sim C_{\pm} \abs{T- T_c}^{-\alpha}$, where $\alpha = 0$, the susceptibility scaled as $\chi \sim \abs{T - T_c}^{-\gamma}$ where $\gamma = 1$. Finally, the magnetisation also scaled as $m \sim B^{1/\delta}$ where $\delta = 3$.

The actual values are tabulated in \ref{tab:criticalexp}.

\begin{table}[htpb]
  \centering
  \begin{tabular}{|c|c|c|c|c|}
    \hline
     & MF & d=2 & d=3 & d=4 \\
     \hline
    $\alpha$ & $0$ & $O(\log)$ & $0.1101\ldots$ & \\
    $\beta$ & $\frac{1}{2}$ & $0.3264\ldots$ & & \\
    $\gamma$ & 1 & $\frac{7}{4}$ & &  \\
    $\delta$ & 3 & 15 & $4.7898\ldots$ & \\
    \hline
  \end{tabular}
  \caption{Critical Exponents in various dimensions.}
  \label{tab:criticalexp}
\end{table}

This leads us to the concept of \emph{universality} where two systems that have different underlying physical processes actually have the same critical exponents.
