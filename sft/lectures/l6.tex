% lecture notes by Umut Özer
% course: sft
\lhead{Lecture 6: October 25}
We can take this further. There are more interesting quantities than the average value $\langle \phi(x) \rangle$; we can obtain the \emph{connected two-point function} by taking two functional derivatives
\begin{equation}
  \frac{1}{\beta^2} \frac{\delta^2 \log Z}{\delta B(x) \delta B(y)} = \langle \phi(x)\phi(y) \rangle_B - \langle \phi(x) \rangle \langle \phi(y) \rangle_B.
\end{equation}
This automatically calculates the fluctuations about the average, which is quite neat.
Setting $B = 0$ for $T > T_c$, we get
\begin{equation}
  \frac{1}{\beta^2} \left.\frac{\delta^2 \log(Z)}{\delta B(x) \delta B(y)} \right\rvert_{B=0} = \langle \phi(x) \phi(y) \rangle_{B = 0}.
\end{equation}
This two-point function is a very important quantity that pops up in various places all around theoretical physics.
\section{The Gaussian Path Integral}%
\label{sec:the_gaussian_path_integral}

Moving to Fourier space, we have the action
\begin{equation}
  F[\phi_{\vb{k}}] = \int_{}^{} \bdd[d]{k} \left[ \frac{1}{2} (\gamma k^2 + \mu^2) \phi_{\vb{k}} \phi_{-\vb{k}} + B_{-\vb{k}} \phi_{\vb{k}} \right].
\end{equation}
We can define a shifted momentum mode
\begin{equation}
  \hat \phi_{\vb{k}} = \phi_{\vb{k}} + \frac{B_{\vb{k}}}{\gamma k^2 + \mu^2}.
\end{equation}
When we do this, the integral over momentum space simplifies greatly:
\begin{equation}
  F[\hat \phi_{\vb{k}}] = \int_{}^{} \bdd[d]{k} \left[ \frac{1}{2} (\gamma k^2 + \mu^2) \abs{\hat \phi_{\vb{k}}}^2 + \frac{1}{2} \frac{\abs{B_{\vb{k}}}^2}{(\gamma k^2 + \mu^2)} \right].
\end{equation}
The path integral measure, written out as a product of integrals, is
\begin{equation}
  Z = \left[\prod_{\vb{k}} \int \dd{\phi_{\vb{k}}} \dd{\phi_{-\vb{k}}}\right] e^{-\beta F[\hat \phi_{\vb{k}}]}.
\end{equation}
And therefore, we have the full partition function with external magnetic field as
\begin{equation}
  Z[B_{\vb{k}}] = e^{-\beta F_{\text{thermo}}} \exp(\frac{\beta}{2} \int_{}^{} \bdd[d]{k} \frac{\abs{B_{\vb{k}}}^2}{(\gamma k^2 + \mu^2)}).
\end{equation}
We have separated out the path integral over $\phi$ into the front.
One might think that this drops all the information about the fields $\phi$. However, this information is still in the right side of the integral, since the form of the factor $(\gamma k^2 + \mu^2)$ was determined by $\phi$.
Taking the inverse Fourier transform, we have
\begin{equation}
  Z[B(\vb{x})] = e^{-\beta F_{\text{thermo}}} e^{\frac{\beta}{2} \int \dd[d]{x} \dd[d]{y} B(\vb{x}) G(\vb{x} - \vb{y}) B(\vb{x})},
\end{equation}
where we have
\begin{equation}
  G(\vb{x} - \vb{y}) = \int_{}^{} \bdd[d]{k} \frac{e^{-i \vb{k} \cdot (\vb{x} - \vb{y})}}{(\gamma k^2 + \mu^2)}.
\end{equation}
Therefore, we have
\begin{equation}
  \left.\langle \phi(\vb{x}) \phi(\vb{y}) \rangle \right\rvert_{B = 0} = \frac{1}{\beta} G(\vb{x} - \vb{y})
\end{equation}
This is our version of the Feynman propagator.
There is no special direction picked out, so $G(\vb{x} - \vb{y}) = G(r)$, where $r^2 = \sum_i x_i^2$.
\begin{equation}
  G(r) = \frac{1}{\gamma} \int_{}^{} \bdd[d]{k} \frac{e^{-i \vb{k} \cdot \vb{x}}}{k^2 + \frac{1}{\varepsilon^2}}, \qquad \varepsilon^2 = \frac{\gamma}{\mu^2}.
\end{equation}
The only length scale in the theory is the \emph{correlation length} $\varepsilon$.
We can perform this integral explicitly. 
However, we can also write
\begin{equation}
  \frac{1}{k^2 + \frac{1}{\varepsilon^2}} = \int_{0}^{\infty} \dd[]{t} e^{-t (k^2 + \frac{1}{\varepsilon^2})}
\end{equation}
\begin{leftbar}
  \begin{remark}
    c.f. Laplace transform and Schwinger parameter.
  \end{remark}
\end{leftbar}
Then
\begin{align}
  G(r) &= \frac{1}{\gamma} \int_{}^{} \bdd[d]{k} \int_{0}^{\infty} \dd[]{t} e^{-t ( k^2 + \varepsilon^{-2}) - i \vb{k} \cdot \vb{x}} \\
   &= \frac{1}{\gamma} \int_{}^{} \bdd[d]{k} \int_{0}^{\infty} \dd[]{t} e^{-t ( k + \frac{i x}{2 t})^2 - \frac{r^2}{4 t} - \frac{t}{\varepsilon^2}} \\
   &= \frac{1}{(4 \pi)^{d/2} \gamma} \int_{0}^{\infty} \dd{t} t^{-d/2} e^{- \frac{r^2}{4 t} - \frac{t}{\varepsilon^2}} \\
   &\sim \int_{0}^{\infty} \dd{t} e^{-S(t)},
\end{align}
where we can use the saddle point approximation
\begin{align}
  S(t) &= \frac{r^2}{4t} + \frac{t}{\varepsilon^2} + \frac{d}{2} \log(t) \\
       &\approx S(t_*) + \frac{S''(t_*) t^2}{2} + \dots, \qquad t_* = \frac{\varepsilon^2}{2} \left( -\frac{d}{2} + \sqrt{\frac{d^2}{4} + \frac{r^2}{\varepsilon}} \right).
\end{align}
There are two interesting limits.
\begin{enumerate}
  \item $r \gg \varepsilon$: We find that $t_*$ scales as $t_* \sim \frac{1}{2} r \varepsilon$. This means that the Green's function scales as
    \begin{equation}
      G(r) \sim \varepsilon^{-(d-3)/2} \frac{e^{-r/\varepsilon}}{r^{(d-1)/2}}.
    \end{equation}
    The correlations are dying off exponentially fast; on large distance scales, it looks like a completely uncorrelated theory.
    There are no modes propagating easily on large distances.
  \item $r \ll \varepsilon$: We have $t_* \sim \frac{r^2}{4 d}$, and therefore $G(r) \sim r^{2-d}$.
    On small distances, we have the same polynomial scaling as one would expect from dimensional analysis.
    There correlations and modes that propagate freely.
\end{enumerate}
In particle physics, this $\varepsilon$ is related to the mass of the particle. Heavy particles give the Yukawa potential and they do not propagate. Light or massless particles, like with gravity or electromagnetism, have polynomial behaviour as in the second type.
For higher order interactions, like $\phi^4$, we would not be able to solve this explicitly.
