% lecture notes by Umut Özer
% course: sft
\lhead{Lecture 14: November 13}

\section{Continuous Symmetries}%
\label{sec:continuous_symmetries}

This will be our first step beyond the Ising model. 
Like with music and musical notes and chords, there is an infinite collection of theories---determined by their free energy---we can study. However, just like in music, only a small subset of these are worth considering.

Phases of matter are characterised by symmetries.
There are external (spacetime) and internal (field, like U(1)) symmetries.
The symmetry of the ground state (H) may be different from the symmetry of the free energy (G).
For example, in the Ising model, the ground state at high temperature $T > T_c$ respects the $\mathbb{Z}_2$ symmetry of the theory. However, once we go to low temperatures $T < T_c$, the spins pick out a direction and the ground state has no symmetry. We say the symmetry of the theory has been \emph{spontaneously broken}.
Of course, if we had a magnetic field, then the theory had no $\mathbb{Z}_2$ symmetry to begin with.

In general, we build our theories by making a hypothesis of G, and seeing whether their predictions match up with nature.
Many systems are characterised by their symmetrised. One example is given by christals. However, some of the most interesting symmetries we come across are internal symmetries.
\begin{leftbar}
  \begin{remark}
    Sometimes, these things mix up. Think of lowering the temperature in a crystal. We spontaneously break spacetime symmetries at around the same time as we break internal symmetries. 
  \end{remark}
\end{leftbar}

\section{O(N) Models}%
\label{sec:o_n_models}

We have $N$ scalar fields $\phi(\vb{a}) = (\phi_1(\vb{x}), \dots, \phi_{N}(\vb{x}))$. Of the theory respects $O(n)$ symmetry, the theory will be invariant under 
\begin{equation}
  \phi_a (\vb{x}) \to R \indices{_a^b} \phi_b(\vb{x}) \qquad R ^T R = \mathbb{1}.
\end{equation}

We may now write down the most general action
\begin{equation}
  F[\phi(\vb{x})] = \int \dd[d]{x} \left[ \frac{\gamma}{2} \sum_i (\grad \phi_i) \cdot (\grad \phi_i) + \frac{\mu^2}{2} \phi \cdot \phi + g (\phi \cdot \phi)^2 + \dots \right],
\end{equation}

\begin{example}[O(2): XY-Model]
  A particularly interesting model of this is the \emph{XY-Model}, which is simply $O(2)$.
  We package our fields $\phi = (\phi_1, \phi_2)$ into a complex field $\psi(\vb{x}) = \phi_1 (\vb{x}) + i \phi_2 (\vb{x})$, using the isomorphism between $U(1) \sim SO(2)$.
  The free energy is then
  \begin{equation}
    F[\psi(x)] = \int \dd[d]{x} \left[ \grad \psi^* \cdot \grad \psi + \frac{\mu^2}{2} \abs{\psi}^2 + g \abs{ \psi}^4 + \dots \right]
  \end{equation}

  As we will see in the next sections, this model describes Bose-Einstein condensates and superfluids.
\end{example}

\begin{example}[O(3): Heisenberg Model]
\end{example}

These models are kind of `Ising-Plus'. However, these symmetries also give us some more interesting phenomena.


\subsection{Goldstone Bosons}%
\label{sub:goldstone_bosons}

\begin{theorem}[]
  For every spontaneously broken continuous symmetry, you get an exactly massless scalar.
\end{theorem}
These (Nambu-)Goldstone bosons show up in all areas of physics.
Why does this happen? 
For discrete symmetries we have a finite number of vacua.
\begin{example}[$\mathbb{Z}_2$]
  In the Ising model, the spins can either all point up or all down, $\langle \phi \rangle = \pm m_0$.
  Therefore, there are two different vacua in the low temperature theory.
\end{example}
However, for continuous symmetries, the magnitude of the direction is fixed, whereas the direction of the field is not fixed.
$\langle \abs{\phi} \rangle = M_0  = \sqrt{-\mu^2 / (4g)}$.
This means that we have a sphere of ground states.

%F1 Mexican hat 

The ground state is at the bottom of the Mexican hat potential depicted in F1. 
If we generalise this to the $O(n)$ model, the vacua are the $(n-1)$-sphere $S^{n-1}$.
This means it costs us no energy to move around the bottom of the potential.
There is no Boltzmann suppression in producing these; they are massless.
Other names for these Goldstone bosons are \emph{spin waves}, \emph{soft modes}, or \emph{gapless modes}.
At low temperatures, you are bound to produce these modes since they cost no energy; they dominate the low energy behaviour.

We can count the number of Goldstone bosons we expect to have, by considering the number of generators.
The number of Goldstone bosons is the number of broken symmetries $\dim(G) - \dim(H)$.
In each case we get a vacuum manifold, which can be written as the quotient group
\begin{equation}
  S^{N-1} = \frac{O(N)}{O(N-1)}.
\end{equation}
The vacuum expectation value of $\phi$ in $O(N)$ is $\langle \phi \rangle = (M_0, 0, \dots, 0)$, where the zero vector remaining is the $O(N-1)$ symmetry.
This means that the number of Goldstone bosons for $O(N)$ is $N-1$.

\begin{example}[XY-Model]
  Parametrise $\psi(\vb{x}) = (M_0 + \widetilde{M}(\vb{x})) e^{i \theta(\vb{x})}$.
  The free energy takes the form
  \begin{equation}
    F[M, \theta] = \int \dd[d]{x} \left\{ \frac{\gamma}{2} (\grad \widetilde{M})^2 + \abs{\mu^2}\widetilde{M}^2 + g \widetilde{M}^4 + \dots + \frac{\gamma}{2} M_0^2 (\grad \theta)^2 + \gamma M_0 \widetilde{M} (\grad \theta)^2+ \dots \right\}
  \end{equation}
  The symmetry is being remembered in the vacuum, meaning that it costs no energy to rotate; the theory has a \emph{shift symmetry} since $\grad \theta \to \grad(\theta + a) = \grad \theta$.
\end{example}

We saw that approaching the critical point, the correlation length went to infinity, effectively giving us a massless mode, which has long-range correlations.
Here, we do not have to tune anything; below the critical temperature, we get Goldstone bosons with infinite correlation length without having to tune the temperature, the magnetic field etc, to get to the critical point.
Indeed, they show some very non-trivial behaviour.

\subsection{Critical Exponents}%
\label{sub:critical_exponent}

If we sit at the critical temperature, we expect to have a theory with critical exponents as we had before.
Although the $O(n)$ models look similar to the Ising model, they are actually in a different universality class.

In $d = 3$, the exponents are particularly interesting, as tabulated in \ref{tab:14-1}.
\begin{table}[htpb]
  \centering
  \begin{tabular}{c | c c}
     & $\eta$ & $\nu$ \\
     \hline
     MF & $0$ & $1/2$ \\
     Ising & $0.0363$ & $0.6300$ \\
     $N = 2$ & $0.0385$ & $0.6718$ \\
     $N = 3$ & $0.0386$ & $0.702$ \\
  \end{tabular}
  \caption{Critical exponents}
  \label{tab:14-1}
\end{table}
This tiny difference in critical exponent sets apart superfluids from magnets; this is known as the \emph{lambda transition}. In particular, the heat capacity $C \sim \abs{T - T_c}^{-\alpha}$, $\alpha = 2 - 3 \nu$ is given by $\alpha_{O(2)} = -0.16$, which gives a divergence at $T = T_c$ for superfluids.
%F2

\subsection{RG in the O(N) Model}%
\label{sub:rg_in_the_o_n_model}

Naive dimensional analysis gives us
\begin{equation}
  [\phi] = \frac{d-2}{2}, \qquad [\mu_0^2] = 2, \qquad [g_0] = 4 - d.
\end{equation}
However, we now have $F_I[\phi] \sim \int \dd[d]{x} g \phi^4$, where $\phi^4 \sim \phi \cdot \phi \phi \cdot \phi$, which we realise in Feynman diagrams by splitting the vertex
\begin{equation}
  \feynmandiagram[inline=(v.base), horizontal=a to b] {
    a -- v -- b,
    c -- v -- d
  };
  \to
  \feynmandiagram[inline=(v.base), horizontal=v to r] {
    a -- v -- r -- b,
    c -- v,
    r -- d,
  };
\end{equation}
%Fix: elongated vertex

To order $O(g_0)$, we can calculate the contributions from the following Feynman diagrams
%Diagrams F3
%F4
Performing the epsilon expansion, we get
\begin{equation}
  \dv[]{\mu^2}{s} = 2\mu^2 + \frac{N + 2}{2\pi^2} \frac{\Lambda^4}{(\Lambda^2 + \mu^2)} \widetilde{g}^2 \qquad \dv[]{\widetilde{g}}{s} = \epsilon \widetilde{g} - \frac{N + 8}{2\pi^2} \frac{\Lambda^4}{\Lambda^2 + \mu^2} \widetilde{g}^2,
\end{equation}
where $\widetilde{g} = \Lambda^{-\epsilon} g$.
Flowing towards the Wilson-Fischer fixed point, we have
\begin{equation}
  \mu_*^2 = -\frac{1}{2} \frac{N + 2}{N + 8} \Lambda^2 \epsilon, \qquad \widetilde{g}_* = \frac{2\pi^2}{N + 8} \epsilon
\end{equation}
