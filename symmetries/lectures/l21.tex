% lecture notes by Umut Özer
% course: symmetries
\lhead{Lecture 21: November 28}

\begin{definition}[]
  The \emph{level} of a root is the weight in its root string expansion.
\end{definition}

\section{Reconstructing \texorpdfstring{$\mathfrak{g}$}{the Lie Algebra} from \texorpdfstring{$A^{ij}$}{the Cartan Matrix}}%
\label{sec:reconstructing_the_lie_algebra_from_the_cartan_matrix}

The Cartan matrix determines\footnote{up to choice of one vector (such as $\alpha_{(1)}$) in $\mathbb{R}^r$} the simple roots $\alpha_{(i)}$,  $i = 1, \dots, r = \text{rank}(\mathfrak{g})$ via
\begin{equation}
  A^{ij} = \frac{2(\alpha_{(i)}, \alpha_{(j)})}{(\alpha_{(j)}, \alpha_{(j)})} = \frac{2 \abs{\alpha_{(i)}}}{\abs{\alpha_{(j)}}} \cos(\varphi_{ij})
\end{equation}
We have the same picture as in Fig.~\ref{fig:l18f1}.
The remaining roots are found by constructing root strings. As we know from Claim \ref{cl:19-ii}, the length of the $\alpha_{(i)}$-string through $\alpha_{(j)}$ is $ l_{ij} = 1 - A_{ji} \in \mathbb{N}^+$.

\begin{example}[$\mathfrak{g} = A_2$]
  \dynkin{A}{2} \quad $\mathfrak{g} \simeq \mathfrak{su}_{\mathbb{C}}(3)$ \par
  The Cartan matrix is $A = 
  \begin{pmatrix}
   2 & -1 \\
   -1 & 2 \\
  \end{pmatrix} $, from which we can read off that $\mathfrak{g}$  has two simple roots $\alpha, \beta$ with
   \begin{equation}
     \frac{2(\alpha, \beta)}{(\alpha, \alpha)} = \frac{2(\beta, \alpha)}{(\beta, \beta)} = -1
  \end{equation}
  Therefore, $\abs{\alpha} = \abs{\beta}$ and $\cos(\theta_{\alpha, \beta}) = -\frac{1}{2} \implies \theta = \frac{2\pi}{3}$. This geometry is illustrated in Fig.~\ref{fig:l21f1}.

  \begin{figure}[tbhp]
    \centering
    \def\svgwidth{0.3\columnwidth}
    \input{lectures/l21f1.pdf_tex}
    \caption{The simple roots of $A_2$ obey $\abs{\alpha} = \abs{\beta}$.}
    \label{fig:l21f1}
  \end{figure}

  To find the remaining roots we use that
  \begin{itemize}
    \item $\alpha, \beta \in \Phi_S \implies \pm (\alpha - \beta) \not \in \Phi$ 
    \item the length of the $\alpha$-string through $\beta$  is \begin{equation}
	l_{\alpha, \beta} = 1 - 2 \frac{(\alpha, \beta)}{(\alpha, \alpha)} = 2
    \end{equation}  
    and the length of the $\beta$-string through $\alpha$ is 
      \begin{equation}
	l_{\beta, \alpha} = 1 - 2 \frac{(\beta, \alpha)}{(\beta, \beta)} = 2
    \end{equation}  
    So $\beta + n \alpha, \alpha + n' \beta \in \Phi$ only for $n, n' \in \left\{ 0, 1 \right\}$
    \item $\alpha, \beta, \alpha + \beta \in \Phi$ and $-\alpha, -\beta, -(\alpha + \beta) \in \Phi$
    \item the $\alpha$- and $\beta$-strings through $\alpha + \beta$ (and vice-versa) yield no additional roots
  \end{itemize}
  The new root has squared length
  \begin{align}
    (\alpha + \beta, \alpha + \beta) &= (\alpha, \alpha) + (\beta, \beta) + 2 (\alpha, \beta) \\
				     &= (\alpha, \alpha) [1 + 1 - 1] = (\alpha, \alpha)
  \end{align}

  \subsection*{Root system for $A_2$}%
  The root set is
  \begin{equation}
    \Phi = \left\{ \pm \alpha, \pm \beta, \pm (\alpha + \beta) \right\},
  \end{equation}
  where all the roots have the same length
  \begin{equation}
    \abs{\alpha} = \abs{\beta} = \abs{\alpha + \beta}.
  \end{equation}
  This is illustrated in \ref{fig:l21f2}.
  \begin{figure}[tbhp]
    \centering
    \def\svgwidth{0.4\columnwidth}
    \input{lectures/l21f2.pdf_tex}
    \caption{The root system of $A_2 \simeq \mathfrak{su}_{\mathbb{C}}(3)$.}
    \label{fig:l21f2}
  \end{figure}

  The CW basis for $A_2$ is:
  \begin{equation}
    \left\{ H^1, H^2, E^{\alpha}, E^{-\alpha}, E^{\beta}, E^{-\beta}, E^{\alpha + \beta}, E^{-(\alpha + \beta)} \right\}
  \end{equation}
  Since $\mathfrak{g} = A_2 \simeq \mathfrak{su}_{\mathbb{C}}(3)$, we have $\dim \mathfrak{g} = 3^2 - 1 = 8$, as expected.
\end{example}  

\begin{exercise}
  Find the root system of $B_2$.
\end{exercise}

\chapter{Representation Theory II}%
\label{cha:representation_theory_ii}

To find the finite-dimensional rep of $\mathfrak{su}_{\mathbb{C}}(2)$, the idea was to work in the CW basis and work with step operators.
Because we have now generalised the CW basis to general simple Lie algebras, we can use the same idea.

\section{Weights}%
\label{sec:weights}

Let $R$ be a representation of $\mathfrak{g}$ with dimension $N$.
For example, 
\begin{equation}
  \begin{gathered}
    H^i \\
    E^{\alpha}
  \end{gathered}
  \quad \mapsto \quad
  \begin{gathered}
    R(H^i) \\
    R(E^\alpha)
  \end{gathered}
  \quad \in \text{Mat}_N(\mathbb{C})
  \qquad i = 1, \dots, r \quad \alpha \in \Phi
\end{equation}

The $H^i$ are  \emph{diagonalisable} in the sense that $[H^i, H^{j}] = 0$ . The representation matrices also commute since $ [R(H^{i}), R(H^{j})] = R([H^{i}, H^{j}]) = 0$. So if one is diagonalisable, then all of them are.
We will assume that $R(H^i)$  are diagonalisable for $i = 1, \dots, r$.
The fact that they commute with each other means that $\left\{ R(H^{i}) \right\}$ are \emph{simultaneously diagonalisable}.

This means that the vector space $V \simeq \mathbb{C}^N$ on which they act is spanned by the simultaneous eigenvectors of $\left\{ R(H^i) \right\}$.
Just as for the  $\mathfrak{su}_{\mathbb{C}}(2)$ case, we can write our representation space $V$ as
 \begin{equation}
  V = \bigoplus_{\lambda \in S_R} V_{\lambda}
\end{equation}
a decomposition into a direct sum of linear subspaces associated with the eigenvalues $\lambda$, where $\forall v \in V_\lambda$, we have
\begin{equation}
  R(H^{i}) v = \lambda^{i} v \qquad 
  \begin{gathered}
    \lambda^{i} \in \mathbb{C} \\
    i = 1, \dots, r.
  \end{gathered}
\end{equation}

For $\mathfrak{su}_{\mathbb{C}}(2)$, we called the eigenvalue $\lambda$ the  \emph{weights}. In this case, they are vectors rather than just numbers.

\begin{definition}[]
  The eigenvalue $\lambda \in \mathfrak{h}^*$ is a \emph{weight} of the representation $R$. We denote by $S_R$ the associated weight set.
\end{definition}

\begin{remark}
  Weights $\lambda \in S_R \subset \mathfrak{h}^*$ have multiplicities $m_{\lambda} = \dim V_{\lambda} \geq 1$.
\end{remark}

This motivates the following rephrasing of Def.~\ref{def:roots}:
\begin{definition}[]
  Roots $\alpha \in \Phi$ are the weights of the adjoint representation $R(X) = ad_X$.
\end{definition}

The key point for $\mathfrak{su}_{\mathbb{C}}(2)$ was that we generated the other weight vectors from the highest weight by applying the step operators.
We likewise consider the action of step operators $R(E^{\alpha})$ , with root $\lambda \in \Phi$, on vectors  $v$ belonging to a particular subspace  $V_\lambda$ with weight  $\lambda$.
 \begin{align}
  R(H^{i}) R(E^\alpha) v &=  R(E^\alpha) R(H^{i}) v + [R(H^{i}), R(E^{\alpha})] v \\
			 &= (\lambda^i+ \alpha^i) R(E^\alpha) v
\end{align}
\begin{leftbar}
  \begin{note}
    This is similar to how we solve the QHO with step operators in quantum mechanics, showing that $a \ket{n}$ is an eigenvalue of $H$ with energy $n - 1$.
  \end{note}
\end{leftbar}
Thus, for all vectors $v \in V_\lambda$, we find that
\begin{equation}
  R(E^\alpha) v 
  \begin{cases}
    \in V_{\lambda + \alpha}, & \text{if } \lambda + \alpha \in S_R \\
    = 0, & \text{otherwise}
  \end{cases}
\end{equation}

Recall that for $\mathfrak{su}_{\mathbb{C}}(2)$, the weights were all integers.
Here, we can use again the fact that the Lie algebra contains a lot of $\mathfrak{su}_{\mathbb{C}}(2)$ subalgebras.
This allows us to say something precise about the weights.

Consider the action of $\mathfrak{sl}(2)_\alpha$ of a particular $\mathfrak{su}_{\mathbb{C}}(2)$  subalgebra
with generators
\begin{equation}
  \{R(h^\alpha), \quad R(e^{\alpha}), \quad R(e^{-\alpha}) \} 
\end{equation}
on $V$.
Representation matrices obey the  $\mathfrak{su}_{\mathbb{C}}(2)$  commutation relations
\begin{align}
  [R(h^\alpha), R(e^{\pm \alpha})] &= \pm 2 R(e^{\pm \alpha}) \\
  [R(e^{+\alpha}), R(e^{- \alpha})] &= R(h^\alpha) \\
\end{align}

Each generator defines a linear map $V \to V$.
This means that $V$ is a representation space for some representation of $\mathfrak{sl}(2)_\alpha$ ($R_\alpha$ is finite-dimensional but not necessarily  irreducible).
Recall the definition 
\begin{equation}
  h^\alpha = \frac{2}{(\alpha, \alpha)} H^\alpha \qquad H^\alpha = (\kappa^{-1})_{ij} \alpha^{i} H^{j}
\end{equation}
Using linearity, we have that $\forall v \in V_{\lambda}$,
\begin{align}
  R(h^{\alpha}) v &= \frac{2}{(\alpha, \alpha)} (\kappa^{-1})_{ij} \alpha^{j} R(H^{j}) v \\
						   &= \left( \frac{2}{(\alpha, \alpha)} (\kappa^{-1})_{ij} \alpha^{i} \lambda^{j} \right) v \\
						   &= \frac{2 (\alpha, \lambda)}{(\alpha, \alpha)} \cdot v
\end{align}
We learned something new; the weights $\lambda$ have to obey the following quantisation condition:
 \begin{equation}
  \label{eq:23}
   \boxed{\frac{2(\alpha, \lambda)}{(\alpha, \alpha)} \in \mathbb{Z} \qquad
    \begin{gathered}
      \forall \lambda \in S_R \\
      \alpha \in \Phi
    \end{gathered}}
\end{equation}

