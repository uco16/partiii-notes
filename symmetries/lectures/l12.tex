% lecture notes by Umut Özer
% course: symmetries
\lhead{Lecture 12: November 07}
Hence, we have unique candidates for the irreps of dimensions in Table \ref{tab:12-reps}.
\begin{table}[htpb]
  \centering
  \begin{tabular}{|c|c|c|}
    \hline
     & & Dimension \\
     \hline
    $R_0 = d_0$ & trivial representation & $1$  \\
    $R_1 = d_f$ & fundamental representation & $2$ \\
    $R_2 = d_{adj}$ & adjoint representation & $3$ \\
    \hline
  \end{tabular}
  \caption{}
  \label{tab:12-reps}
\end{table}

\subsection{Angular Momentum in Quantum Mechanics}%
\label{sub:angular_momentum_in_quantum_mechanics}

What we have here described is analogous to the theory of angular momentum in quantum mechanics.
Recall that the total angular momentum consists of orbital and spin angular momenta.
We have a Hermitian operator $\vb{J} = (J_1, J_2, J_3)$ of total angular momentum.
Its corresponding eigenstates are labelled by $j \in \mathbb{Z}/2$, $j \geq 0$, and $m \in \left\{ -j, -j + 1, \dots, j-1, j \right\}$:
\begin{align}
  J^2 \ket{j, m} &= \hbar j (j+1) \ket{j, m}  \\
  J_3 \ket{j, m} &= \hbar m \ket{j, m},
\end{align}
where $J^2 = J_1^2 + J_2^2 + J_3^2$.
In the Cartan representation, we have the correspondance
\begin{align}
  J_3 &= \frac{1}{2} R(H) \\
  J_\pm = J_1 \pm i J_2 &= R(E_\pm)
\end{align}
The highest weight is $\Lambda = 2 j \in \mathbb{Z}^+$ and the other weights are $\lambda = 2m \in \mathbb{Z}$. The associated states are
\begin{equation}
  v_\Lambda \sim \ket{j, j} \qquad v_\lambda \sim \ket{j, m}.
\end{equation}

\subsection{\texorpdfstring{$SU(2)$}{SU(2)} reps from \texorpdfstring{$\mathfrak{su}(2)$}{its Lie algebra's} reps}%
\label{sub:su2-reps-from-lsu2-reps}

Locally, we can parametrise group elements $A \in SU(2)$ as $A = \text{Exp}(X)$, where $X$ is an element of the corresponding Lie algebra $\mathfrak{su}(2)$.
Starting from an irrep $R_\Lambda$ of $\mathfrak{su}(2)$, we have a representation 
\begin{equation}
  D_\Lambda(A) = \text{Exp} \bigl(R_\Lambda(X)\bigr).
\end{equation}
As before, $\Lambda \in \mathbb{N}_0$.
We will now use the relationship between $SU(2)$ and $SO(3)$ that we established. In particular, you can think of $SO(3)$ as the quotient
\begin{equation}
  SO(3) \simeq \frac{SU(3)}{\mathbb{Z}_2},
\end{equation}
corresponding to the identification of antipodal points $A \sim -A$.
This condition implies that the representation matrices of $A$ and $-A$ must be the same.
In other words, for a representation of $SO(3)$, we require that $\forall A \in SU(3)$
\begin{equation}
  D_\Lambda(-A) = D_\Lambda(A).
\end{equation}
Using the representation property, we can see that this is actually equivalent to the simpler condition
\begin{equation}
  D_\Lambda (-\mathbb{1}_2) = D_{\Lambda}(\mathbb{1}_2).
\end{equation}
Now if we represent $-\mathbb{1}_2 = \text{Exp}(i\pi H)$ with 
\begin{equation}
  H = 
  \begin{pmatrix}
   1 & 0 \\
   0 & -1 \\
  \end{pmatrix}, 
  \qquad \text{Exp}(i \pi H) = 
  \begin{pmatrix}
   e^{i\pi} &  \\
    & e^{-i\pi} \\
  \end{pmatrix} = -\mathbb{1}.
\end{equation}
Therefore, the representation is
\begin{align}
  D_\Lambda(-\mathbb{1}_2) &= \text{Exp}(i\pi R(\Lambda)(H)), \qquad \text{with } R_\Lambda(H) = 
  \begin{pmatrix}
   \Lambda &  &  &  &  \\
    & \Lambda-2 &  &  &  \\
    &  & \ddots &  &  \\
    &  &  & -\Lambda + 2 &  \\
    &  &  &  & -\Lambda \\
  \end{pmatrix} \\
			   &= \text{Exp} \left( 
           \begin{pmatrix}
            e^{i\pi \Lambda} &  &  &  \\
			     & e^{i \pi (\Lambda - 2)} &  &  \\
             &  & \ddots &  \\
             &  &  & e^{-i \pi \Lambda} \\
           \end{pmatrix}
			   \right).
\end{align}
We find that $D_\Lambda(-\mathbb{1}_2)$ has eigenvalue $\text{exp}(i\pi\lambda) = (-1)^{\lambda} = (-1)^\Lambda$. Hence, 
\begin{equation}
  D_\Lambda (-\mathbb{1}_2) = D_\Lambda (+\mathbb{1}_2) = \mathbb{1}_{\Lambda + 1} \qquad \text{iff} \quad \Lambda \in 2 \mathbb{Z}.
\end{equation}
Now we have two cases:
\begin{itemize}
  \item $\Lambda \in 2\mathbb{Z} \implies$ $D_\Lambda$ is a representation of $SU(2)$ and $SO(3)$.
  \item $\Lambda \in 2\mathbb{Z} + 1 \implies$ $D_\Lambda$ is a representation of $SU(2)$, but \emph{not} of $SO(3)$.
\end{itemize}

\section{New representations from old}%
\label{sec:new_representations_from_old}

\begin{definition}[conjugate rep]
  If $R$ is a representation of a \emph{real} Lie algebra $\mathfrak{g}$, we define a \emph{conjugate representation} by $\bar R(X) = R(X)^* \quad \forall X \in g$.
\end{definition}
Sometimes, we find that $\bar R \simeq R$.
\begin{leftbar}
  \begin{remark}
    We will find that the if a particle transforms under a representation $R$, its anti-particles will transform under $\bar R$.
  \end{remark}
\end{leftbar}

\begin{definition}[direct sum]
  Suppose we are given representations $R_1$ and $R_2$ of any $\mathfrak{g}$ (not necessarily real) with representation spaces $V_1$ and $V_2$ of dimensions $d_1$ and $d_2$. We then define a \emph{direct sum} $R_1 \oplus R_2$ as a new representation that acts on the representation space
  \begin{equation}
    V_1 \oplus V_2 = \left\{ v_1 \oplus v_2 \mid v_1 \in V_1, v_2 \in V_2 \right\}
  \end{equation}
  as 
  \begin{equation}
    (R_1 \oplus R_2)(X) (v_1 \oplus v_2) = (R_1(X) v_1) \oplus (R_2(X)v_2) \in V_1 \oplus V_2
  \end{equation}
  for all $X \in \mathfrak{g}$.
\end{definition}

The matrix corresponding to the linear map $(R_1 \oplus R_2)(X)$ is in block matrix notation
\begin{equation}
  (R_1 \oplus R_2)(X)= 
  \begin{pmatrix}
    R_1(X) & 0 \\
    0 & R_2(X) \\
  \end{pmatrix}
\end{equation}
