% lecture notes by Umut Özer
% course: symmetries
\lhead{Lecture 18: November 21}

In the Cartan-Weyl basis,
\begin{equation}
  [H^{i}, E^{\delta}] = \delta^{i} E^{\delta}, \qquad \forall \delta \in \Phi, \forall i = 1, \dots, r,
\end{equation}
the Killing form components are
\begin{align}
  \kappa^{ij} &= \kappa(H^{i}, H^{j}) = \frac{1}{\mathcal{N}} \Tr[ad_{H^{i}} \circ ad_{H^{j}}] \\
   &= \frac{1}{\mathcal{N}} \sum_{\delta \in \Phi} \delta^{i} \delta^{j},
\end{align}
where $\mathcal{N} \in \mathbb{R}$ is a normalisation constant.
Hence for $\alpha, \beta \in \Phi$, where we raise and lower indices with the Killing form, 
\begin{align}
  (\alpha, \beta) &= \alpha^{i} \beta^{j}(\kappa^{-1})_{ij} = \alpha_{i} \beta^{i} = \alpha_{i} \beta_{j} \kappa^{ij} \\
		  &= \frac{1}{\mathcal{N}} \sum_{\delta \in \Phi} \alpha_{i} \delta^{i} \delta^{j} \beta_{j} = \frac{1}{\mathcal{N}} \sum_{\delta \in \Phi} (\alpha, \delta) (\beta,\delta) \label{eq:18-17}
\end{align}
From \eqref{eq:17-15}, we have
\begin{equation}
  R_{\alpha, \beta} = \frac{2 (\alpha, \beta)}{(\alpha, \alpha)} \in \mathbb{Z} \subset \mathbb{R}.
\end{equation}
Moreover, from \eqref{eq:18-17}, this is
\begin{equation}
  \frac{2}{(\beta, \beta)} = \frac{1}{\mathcal{N}} \sum_{\delta \in \Phi} R_{\alpha, \delta} R_{\beta, \delta} \in \mathbb{R}
\end{equation}
This gives a series of implications: For all $\alpha, \beta \in \Phi$ we have
\begin{equation}
  N \in \mathbb{R} \implies \text{LHS} \in \mathbb{R} \implies \text{RHS} \in \mathbb{R}
  \implies (\beta, \beta) \in \mathbb{R} \implies \boxed{(\alpha, \beta) \in \mathbb{R}}
\end{equation}

\section{Real geometry of roots}%
\label{sec:real_geometry_of_roots}

Recall that roots $\alpha \in \Phi$ are elements of $\mathfrak{h}^*$, the dual of the CSA $\mathfrak{h} \in \mathfrak{g}$.

\begin{claim}
  In fact, the roots span $\mathfrak{h}^*$,
  \begin{equation}
    \mathfrak{h}^* = \text{Span}_{\mathbb{C}}\left\{\alpha \in \Phi\right\}.
  \end{equation}
\end{claim}
\begin{proof}
  Assume for contradiction that the roots $\alpha \in \Phi$ do not span $\mathfrak{h}^*$. This means $\exists \lambda \in \mathfrak{h}^*$ such that $\forall \alpha \in \Phi$, 
  \begin{equation}
    (\lambda, \alpha) = (\kappa^{-1})_{ij} \lambda^{i} \alpha^{j} = \kappa^{ij} \lambda_{i} \alpha_{j} = 0.
  \end{equation}
  Then we can build $H_{\lambda} = \lambda_{i} H^{i} \in \mathfrak{h}$ such that $ [H_{\lambda}, H] = 0$ and $[H_{\lambda}, E^{\alpha}] = (\lambda, \alpha) E^{\alpha} = 0$ for all $H \in \mathfrak{h}$ and $\alpha \in \Phi$.
  Since these form a basis of $\mathfrak{g}$, we find that $[H_{\lambda}, X] = 0$ for all $X \in \mathfrak{g}$.
  In other words, $\mathfrak{g}$ has a non-trivial ideal $\mathfrak{i} = \text{Span}_{\mathbb{C}}\left\{H_{\lambda}\right\}$, which contradicts our assumption that $\mathfrak{g}$ is simple.
  Thus, $\mathfrak{h}^* = \text{Span}_{\mathbb{C}}\left\{\alpha \in \Phi\right\}$ .
\end{proof}

\begin{corollary}
  Since $\dim(\mathfrak{h}^*) = \dim(\mathfrak{h}) = r = \text{rank}(\mathfrak{g})$, we can therefore find $r$ roots, $\left\{ \alpha_{(i)} \in \Phi \suchthat i = 1, \dots, r \right\}$, which provide a basis for the complex vector space $\mathfrak{h}^*$.
\end{corollary}
As such, we can write any root $\beta \in \Phi$ as $\beta = \sum_{i=1}^{r} \beta^{i} \alpha_{(i)}$, where $\beta^{i} \in \mathbb{C}$ can in general be complex.
\begin{definition}[]
  We define a real subspace $\mathfrak{h}^*_{\mathbb{R}} \subset \mathfrak{h}^*$ as
  \begin{equation}
    \mathfrak{h}^*_{\mathbb{R}} \coloneqq \text{Span}_{\mathbb{R}}\left\{ \alpha_{(i)}, i = 1, \dots, r\right\}.
  \end{equation}
\end{definition}
The coefficients $\beta^i $ solve
\begin{equation}
  (\beta, \alpha_{(j)}) = \sum_{i=1}^{r} \beta^i (\alpha_{(i)}, \alpha_{(j)}).
\end{equation}
From the fact that $(\alpha, \beta) \in \mathbb{R}$, we can immediately infer that $\beta^i \in \mathbb{R}$.
This is the same as saying that the root $\beta$ actually lives in the real subspace $\mathfrak{h}^*_{\mathbb{R}}$:
\begin{equation}
  \boxed{\beta \in \mathfrak{h}^*_{\mathbb{R}} \qquad \forall\beta \in \Phi}
\end{equation}

Consider now the inner product of two arbitrary elements $\lambda, \mu \in \mathfrak{h}^*_{\mathbb{R}}$.
By linearity, this is related to the inner product of the basis,
\begin{equation}
  (\lambda ,\mu) = \sum_{i, j =1}^{r} \lambda^{i} \mu^{j} (\alpha_{(i)}, \alpha_{(j)}) \in \mathbb{R}.
\end{equation}
Hence, the inner product of any two roots is real.
From \eqref{eq:18-17}, we have the length of a vector as
\begin{equation}
  (\lambda, \lambda) = \frac{1}{\mathcal{N}} \sum_{\delta \in \Phi} \lambda_{i} \delta^{i} \delta^{j} \lambda_{j} = \frac{1}{\mathcal{N}} \sum_{\delta \in \Phi} (\lambda, \delta)^2 \geq 0.
\end{equation}
As a result of the non-degeneracy of the inner product $\kappa$, we have thus
\begin{equation}
  \abs{\lambda}^2 = 0 \iff (\lambda, \delta) = 0 \quad \forall \delta \in \Phi \iff \lambda = 0.
\end{equation}

\subsection*{Summary}%

The roots $\alpha \in \Phi$ live in the \emph{real vector space} $\mathfrak{h}^*_{\mathbb{R}} \simeq \mathbb{R}^r$, where $r = \text{Rank}(\mathfrak{g})$, equipped with a \emph{Euclidean inner product}:
For all vectors $\lambda, \mu \in \mathfrak{h}^*_{\mathbb{R}}$, 
\begin{enumerate}
  \item $(\lambda, \mu) \in \mathbb{R}$ 
  \item $(\lambda, \lambda) \geq 0$ 
  \item $(\lambda,  \lambda) = 0 \iff \lambda = 0$
\end{enumerate}

%F1
\begin{wrapfigure}{R}{0.3\columnwidth}
  \centering
  \def\svgwidth{0.25\columnwidth}
  \input{lectures/l18f1.pdf_tex}
  \caption{}
  \label{fig:l18f1}
\end{wrapfigure}

\begin{definition}[]
  As $(\alpha, \alpha) > 0 \quad \forall \alpha \in \Phi$, we can define a \emph{length} $\abs{\alpha} \coloneqq + (\alpha, \alpha)^{1 / 2} > 0$.
\end{definition}
The inner product of two vectors $\alpha, \beta$ takes the standard form
\begin{equation}
  (\alpha, \beta) = \abs{\alpha} \abs{\beta} \cos\theta_{\alpha, \beta}
\end{equation}

The \emph{angle} $\theta_{\alpha, \beta} \in [0, \pi]$ is constrained by quantisation \eqref{eq:17-15}:
\begin{subequations}
  \begin{align}
    \frac{2 (\alpha, \beta)}{(\alpha, \alpha)} &= \frac{2 \abs{\beta}}{\abs{ \alpha}} \cos \theta_{\alpha, \beta} \in \mathbb{Z} \label{eq:19a} \\
    \frac{2 (\beta , \alpha)}{(\beta, \beta)} &= \frac{2 \abs{\alpha}}{\abs{\beta}} \cos \theta_{\alpha, \beta} \in \mathbb{Z} \label{eq:19b}
  \end{align}
\end{subequations}
Multiplying \eqref{eq:19a} $\times$ \eqref{eq:19b}, we find that $4 \cos^2 \theta_{\alpha, \beta} \in \mathbb{Z}$, which implies that
\begin{equation}
  \cos\theta_{\alpha, \beta} = \pm \frac{1}{2} \sqrt{n}.
\end{equation}
Solutions to this can only be found for $n \in \left\{ 0, 1, 2, 3, 4 \right\}$ and are enumerated in Table \ref{tab:18-1}.
\begin{table}[btp]
  \centering
  \begin{tabular}{|c|c|c|c|}
    \hline
    $n$ & sign & $\theta$ &  \\
    \hline
    $0$ & $\pm$ & $\theta = \frac{\pi}{2}$ & $(\alpha, \beta) = 0$ \\
    $4$ & $+$ & $0$ & $\alpha = \beta$ \\
	& $-$ & $\pi$ & $\alpha = -\beta$ \\
    1, 2, 3 & $+$ & $\pi / 6, \pi/4, \pi/3$ & $(\alpha, \beta) > 0$ \\
	    & $-$ & $2\pi/3, 3 \pi/4, 5 \pi / 6$ & $(\alpha, \beta) <0$ \\
    \hline
  \end{tabular}
  \caption{}
  \label{tab:18-1}
\end{table}

\begin{exercise}
  Show that if $\alpha \in \Phi$, then  $k \alpha \in \Phi$ only for  $k = \pm 1$.
\end{exercise}

\begin{figure}[tbph]
  \centering
  \def\svgwidth{0.5\columnwidth}
  \input{lectures/l18f2.pdf_tex}
  \caption{In $\mathbb{R}^2$ we can find an $\mathbb{R}^{2-1}$-dimensional line, which separates the positive from the negative roots.}
  \label{fig:l18f2}
\end{figure}

\section{Simple Roots}%
\label{sec:simple_roots}


The root system is some finite set of root vectors living in $\mathbb{R}^r$. If $\beta \in \Phi$, then so is  $-\beta \in \Phi$.
This motivates the following definition. 
We choose some plane in $\mathfrak{h}^*_{\mathbb{R}}$ of dimension  $\mathbb{R}^{r-1}$ in which none of the root vectors lie, such as illustrated in %F2
for the case of $\mathbb{R}^2$.

We can always find such a plane since there are finitely many roots, but a continuous infinity of choices of such a plane.
This divides the roots into positive and negative roots
\begin{equation}
  \Phi = \Phi_+ \cup \Phi_-.
\end{equation}
This satisfies
\begin{enumerate}
  \item $\alpha \in \Phi_+ \iff -\alpha \in \Phi_-$
  \item if $\alpha + \beta \in \Phi$ and $\alpha, \beta \in \Phi_+$, then $\alpha + \beta \in \Phi_+$
  \item if $\alpha, \beta \in \Phi_-$ and $\alpha + \beta \in \Phi$, then $\alpha + \beta \in \Phi_-$
\end{enumerate}
\begin{definition}[]
  A \emph{simple root} is a positive roots that cannot be written as a sum of two positive roots.
  \begin{equation}
    \delta \in \Phi_S \quad \iff \quad
    \begin{gathered}
      \delta \in \Phi_+ \\
      \delta \neq \alpha + \beta \quad \forall \alpha, \beta \in \Phi_+
    \end{gathered}
  \end{equation}
\end{definition}
We will see that these simple roots form a particularly simple basis of the Cartan-Weyl basis, which we will use to completely determine the Lie algebra.
