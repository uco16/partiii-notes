\lhead{Lecture 3: October 15}

\subsection{Embedded Submanifolds}%
\label{sub:embedded submanifolds}

We can define properties of manifold in an intrinsic way. However, for some of these properties it is significantly easier to define manifolds as subspaces embedded in real space.
\begin{theorem}[Embedding Theorem]
  Consider the subspace of $\mathbb{R}^m$ defined as
  \begin{equation}
    \mathcal{M} = \left\{ \vb{x} \in \mathbb{R}^m \mid \mathcal{F}_\alpha(\vb{x}) = 0 \right\}
  \end{equation}
  where the constraint $\mathcal{F}_\alpha \colon \mathbb{R}^m \to \mathbb{R}$, with $\alpha = 1, \cdots, l$ is a smooth map.  Then $\mathcal{M}$ is a manifold of dimension $D = m-l$ if and only if the Jacobian matrix
  \begin{equation}
    (\mathcal{J})_{\alpha, i} = \pdv{\mathcal{F}_\alpha}{x_i}
  \end{equation}
  has maximal rank $l$ everywhere on $\mathcal{M}$.
\end{theorem}

\begin{example}[$S^2$]
We can realise the two-sphere $S^2$ as a manifold embedded in three dimensional Euclidean space
\begin{equation}
  \mathcal{M} = \left\{\vb{x} \in \mathbb{R}^3 \mid \abs{\vb{x}} = R \right\}
\end{equation}
The solution space of $x^2 + y^2 + z^2 -R^2 = F$ defines a manifold.
In this case, the Jacobian 
\begin{equation}
  \mathcal{J} = \left( \pdv{\mathcal{F}}{x}, \pdv{\mathcal{F}}{y}, \pdv{\mathcal{F}}{z}\right) = 2(x, y, z)
\end{equation}
has rank 1 except at $x = y = z = 0$, but that point is not contained in $\mathcal{M}$, so that is allowed by the theorem.
\end{example}

\begin{definition}[connected]
  A manifold is said to be \emph{connected} if there is a smooth path between any two points on the manifold.
\end{definition} 

%F1

\begin{definition}[simply connected]
  A manifold is said to be \emph{simply connected} if all loops are `trivial', in the sense that they can be continuously contracted to a point.
\end{definition}

%F2

\begin{example}[]
The spherical surface $S^2$ is simply connected, while the torus $T^2$ is not.
\end{example}

\begin{definition}[compact]
  An embedded manifold is \emph{compact} if it is closed and bounded.\footnote{This is the Heine--Borel theorem. On a general manifold we can define compactness from the underlying topology.}
\end{definition}

%F3

\begin{example}[]
The sphere is a compact space, while the hyperboloid is not.
\end{example}

\begin{definition}[submanifold]
  A \emph{submanifold} is a subspace of a manifold that is also a manifold.
\end{definition}

\begin{definition}[Lie subgroup]
  A \emph{Lie subgroup} is a subset of a Lie group which is also a Lie group.
\end{definition}


\section{Matrix Lie Groups}%
\label{sec:matrix_lie_groups}

Matrix multiplication is closed, associative, and there exists a unit element $ e = 1_n \in \text{Mat}_n(F)$.  Here, $F$ can be any field, but we will mostly be working with $F = \mathbb{R}$ or $\mathbb{C}$.
However, the set of matrices $\text{Mat}_n(F)$ is not a group under matrix multiplication since not all matrices are \emph{invertible}.

\begin{definition}[general linear group]
The general linear group of dimension $n$ is the set of $n \times n$ matrices with non-vanishing determinant
\begin{equation}
  GL(n, F) = \left\{ M \in \text{Mat}_n(F) \mid \det M \neq 0 \right\}
\end{equation}
guaranteeing invertibility.
\end{definition}

\begin{definition}[special linear group]
The special linear group has unit determinant
\begin{equation}
  SL(n, F) = \left\{ M \in GL(n, F) \mid \det M = 1 \right\}
\end{equation}
\end{definition}

These conditions are enough to guarantee that these are groups. In particular, the closure property follows from
\begin{equation}
  \det(M_1 M_2) = \det(M_1) \det(M_2) \qquad \forall M_1, M_2 \in \text{Mat}_n(F).
\end{equation}

This is connected to the embedding theorem in the following way.
Taking $SL(n, \mathbb{R})$ we apply the embedding theorem with $m = n^2$. The number of constraints is $l = 1$ due to the determinant being constraint to unity:
\begin{equation}
  \label{eq:3-constraint}
  F_1(M) = \det M - 1
\end{equation}

To apply the embedding theorem we have to calculate the Jacobian. 
It is useful to recall the definition of a minor:
\begin{definition}[minor]
  Let $M \in \text{Mat}_n(\mathbb{R})$ be an $n \times n$ matrix with real entries. We define the \emph{minor} $\hat M^{(ij)}$ of each element of the matrix as the $(n-1) \times (n-1)$ matrix with $i$\textsuperscript{th} row and $j$\textsuperscript{th} column deleted.
\end{definition}
Using this definition, we can differentiate \eqref{eq:3-constraint} as follows
\begin{equation}
  \pdv{F_1}{M_{ij}} = \pm \det(\hat M^{(ij)}).
\end{equation}
Therefore, the Jacobian $\pdv{F_1}{M_{ij}}$ has rank 1, unless all the determinants of the minors vanish.
This is equivalent to the determinant of the matrix vanishing
\begin{equation}
  \det(\hat M^{(ij)}) = 0 \iff \det(M) = 0 \neq 1
\end{equation}

This shows that $SL(n, \mathbb{R})$ is a smooth manifold of dimension $n^2 - 1$.
We could do this with $SL(n, \mathbb{C})$ by splitting the coordinates and then the detminant condition in its real and imaginary parts. For the general linear groups, we define them by removing a condition.

\begin{exercise}
  Complete the proof that $SL(n, \mathbb{R})$ is a Lie group. For this, you have to convince yourself that matrix multiplication, considered element by element, provides a smooth map.
\end{exercise}

This gives us four families of matrix Lie groups.
We have established that
\begin{equation}
  \text{dim}(SL(n, \mathbb{R})) = n^2 - 1.
\end{equation}
A similar argument allows us to find that
\begin{equation}
  \text{dim}(SL(n, \mathbb{C})) = 2n^2 - 2.
\end{equation}
Moreover, for the general linear group we can find that
\begin{equation}
  \text{dim}(GL(n, \mathbb{R})) = n^2, \qquad \text{dim}(GL(2, \mathbb{C})) = 2n^2.
\end{equation}
\begin{leftbar}
  \begin{remark}
    We talk here about the \emph{real dimension}, i.e.~the dimension of the Lie group as embedded into the real manifold $\mathbb{R}^n$.
  \end{remark}
\end{leftbar}

\subsection{Compact Subgroups of \texorpdfstring{$GL(n, \mathbb{R})$}{GL(n, R)}}%
\label{sub:subgroup_of_gl_n_r}

\subsection*{The Orthogonal Groups}%

\begin{definition}[orthogonal groups]
  The \emph{orthogonal groups} are defined to be those elements of the general linear groups which satisfy the orthogonality condition
  \begin{equation}
    O(n) = \left\{ M \in GL(n, \mathbb{R}) \mid M M^T = 1_n \right\}.
  \end{equation}
\end{definition}

\begin{definition}[orthogonal transformations]
Orthogonal transformations are of the form
\begin{equation}
  \vb{v} \in \mathbb{R}^n \to \vb{v}' = M \cdot \vb{v} \in \mathbb{R}^n
\end{equation}
  where $M \in O(n)$ is an orthogonal matrix.
\end{definition}

Orthogonal transformations can be thought of as linear transformations which preserve the length of vectors since
\begin{equation}
  \abs{\vb{b}'} = \vb{v}'^T \cdot \vb{v}' = \vb{v} \cdot M^T M \cdot \vb{v} = \vb{v}^T \cdot \vb{v} = \abs{\vb{v}}^2.
\end{equation}

If we have an orthogonal real matrix, we have
\begin{equation}
  \det(M M^T) = \det(M)^2 = 1,
\end{equation}
so its determinant is $\det(M) = \pm 1$.

From this, by continuity, we can tell that $O(n)$ is not a connected manifold. Indeed, $O(n)$ must have two \emph{connected components}. Moreover, only one of these can contain the identity element.
As such, we can consider the space which only includes the connected components of the identity:
\begin{definition}[special orthogonal group]
  The \emph{special orthogonal groups} $SO(n)$ are defined to be the subsets of $O(n)$ with positive unit determinant:
\begin{equation}
  SO(n) \coloneqq \left\{ M \in O(n) \mid \det(M) = 1 \right\}.
\end{equation}
\end{definition}

How do we distinguish between matrices which have $\det(M) = \pm 1$?
Given a frame $\left\{ \vb{v}_1, \cdots, \vb{v}_n \right\} \in \mathbb{R}^n$, an orthogonal transformation
\begin{equation}
  \vb{v}_a \in \mathbb{R}^n \to \vb{v}_a' = M \cdot \vb{v}_a \in \mathbb{R}^n, \qquad M \in O(n)
\end{equation}
preserves the \emph{orientation} of a frame, i.e.~the sign of the volume element
\begin{equation}
  \Omega = \varepsilon^{i_1, \ldots, i_n} v_1^{i_1}v_2^{i_2} \cdots v_n^{i_n},
\end{equation}
where $\varepsilon$ is the $n$-dimensional alternating tensor, only if $M \in SO(n)$.
Elements of $SO(n)$ correspond to \emph{rotations}, whereas elements of $O(n)$ with $\det(M) = -1$ correspond to some mixture of \emph{rotation} and \emph{reflection}.

\begin{exercise}
  Use the embedding theorem to check that $O(n)$ is a manifold and that its dimension is
  \begin{equation}
    \dim[O(n)] = \dim[SO(n)] = \frac{1}{2}n(n-1).
  \end{equation}
  Remember to show that the Jacobian matrix has maximal rank.
\end{exercise}
