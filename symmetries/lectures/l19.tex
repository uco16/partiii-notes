% lecture notes by Umut Özer
% course: symmetries
\lhead{Lecture 19: November 23}

\begin{definition}[]
  The \emph{height} of a root is the perpendicular distance from the hyperplane, see Fig.~\ref{fig:l18f2}.
\end{definition}

\subsection{Simple Roots}%
\label{sub:simple_roots}

\begin{equation}
  \delta \in \Phi_S \quad \iff \quad \delta \in \Phi_+ \quad \delta \neq \alpha + \beta \quad \forall \alpha, \beta \in \Phi_+
\end{equation}
\begin{claim}
  \label{claim:19-1}
  If $\alpha, \beta \in \Phi_S$, then $\alpha - \beta$ is not a root.
\end{claim}
\begin{proof}
  Suppose $\alpha - \beta \in \Phi$. Either
  \begin{description}
    \item[$\alpha - \beta \in \Phi_+$] $\alpha = \alpha - \beta + \beta \implies \alpha$ not simple \smashtimes
    \item[$\beta - \alpha \in \Phi_+$] $\beta = \beta - \alpha + \alpha \implies \beta$ not simple \smashtimes
  \end{description}
\end{proof}
\begin{claim}
  If $\alpha, \beta \in \Phi_S$, then ` $\alpha$-string through $b$ ' has length
  \begin{equation}
    l_{\alpha, \beta} = 1 - \frac{2(\alpha, \beta)}{(\alpha, \alpha)} \in \mathbb{N}.
  \end{equation}
\end{claim}
\begin{proof}
  String has roots
  \begin{equation}
    S_{\alpha, \beta} = \left\{ \beta + n \alpha \in \Phi, \quad n \in \mathbb{Z}, \quad n_- \leq n \leq n_+ \right\}
  \end{equation}
  with $n_+ \geq 0$ and $n_- \leq 0$.
  Now Eq.~\eqref{eq:17-15} implies that $(n_+ + n_-) = -\frac{2 (\alpha, \beta)}{(\alpha, \alpha)} \in \mathbb{Z}$. From Claim~\ref{claim:19-1}, we have
  \begin{align}
    \alpha, \beta \in \Phi_S &\implies \beta - \alpha \not \in \Phi  \\
			     &\implies n_- = 0
  \end{align}
  Again from \eqref{eq:17-15}, we have
  \begin{equation}
    \label{eq:19-20}
    n_+ = \frac{-2 (\alpha, \beta)}{(\alpha, \alpha)} \in \mathbb{Z}_{\geq 0}
  \end{equation}
  Length of string is then
  \begin{equation}
    l_{\alpha, \beta} = n_+ + 1 = 1 - \frac{2 (\alpha, \beta)}{(\alpha, \alpha)} \in \mathbb{N}
  \end{equation}
\end{proof}
\begin{claim}
  Let $\alpha, \beta \in \Phi_S$ be distinct simple roots, meaning that $\alpha \neq \beta$. Then the inner product is negative
  \begin{equation}
    (\alpha, \beta) \leq 0.
  \end{equation}
\end{claim}
\begin{claim}
  \label{cl:19-4}
  Any positive root $\beta \in \Phi_+$ can be written as
  \begin{equation}
    \beta = \sum_i c_i \alpha_{(i)},
  \end{equation}
  an integer combination of simple roots, where $\alpha_{(i)} \in \Phi_S$ and $c_{i} \in \mathbb{Z}_{\geq 0}$
\end{claim}
\begin{proof}
  The statement is true if $\beta \in \Phi_S$ is a simple root.
  If $\beta \not \in \Phi_S$, then we must have that  $\beta = \beta_1 + \beta_2$ for $\beta_1, \beta_2 \in \Phi_+$.
  Now if $\beta_1, \beta_2 \in \Phi_S$, then we are done. If not, then we subdivide again. We repeat this; this process can only terminate on a linear combination of simple roots as in the claim. Since the space is finite dimensional, and there is a highest root, then this process must indeed terminate.
\end{proof}

\begin{remark}
  Similarly, we can decompose negative roots in terms of negative simple roots.
\end{remark}
This is the nice property of the simple roots! Any other roots can be written as a linear combination of them.
To show that they are the basis, we still need to show the following:
\begin{claim}
  \label{cl:19-5}
  Simple roots are linearly independent.
\end{claim}
\begin{proof}
  Let $\lambda = \sum_{i \in I} c_{i} \alpha_{(i)}$, with $c_{i} \in \mathbb{R}\setminus{\{0\}}$, $\alpha_{(i)} \in \Phi_S$ and the index set is $I \subset \Phi_S$.
  Then simple roots being linearly independent means that $\lambda \neq 0$ unless $c_{i} = 0$ for all $i$.
  Define $\lambda_+ = \sum_{i\in I_+} c_{i} \alpha_{i}$, and $\lambda_- = -\sum_{i \in I_+} c_{i} \alpha_{(i)}$, with $I = I_+ \cup I_-$ and $I_+ = \left\{ i \in I \suchthat c_{i} > 0 \right\}$, $I_- = \left\{ i \in I \suchthat c_i < 0 \right\}$.
  Now $\lambda = \lambda_+ - \lambda_-$ where $\lambda_+, \lambda_-$ not both zero
  \begin{align}
    (\lambda, \lambda) &= (\lambda_+, \lambda_+) + (\lambda_-, \lambda_-) - 2(\lambda_+, \lambda_-) \\
		       &> -2 (\lambda_+, \lambda_-) \\
		       &= 2 \sum_{i\in I_+} \sum_{j \in I_-} c_{i} c_{j} (\alpha_{(i)}, \alpha_{(j)})
  \end{align}
  Now $(\alpha_{(i)}, \alpha_{(j)}) \leq 0$. Moreover, $c_{i} > 0$, $c_{j} > 0$. Therefore, we have that $(\lambda, \lambda) > 0$. Since we have a non-degenerate inner product, we must have $\lambda \neq 0$.
\end{proof}

Finally, to form a basis, we need to make sure that there are enough of them:
\begin{claim}
  There are exactly $r = \text{Rank}[\mathfrak{g}]$ simple roots.
\end{claim}
\begin{proof}
  By Claim \ref{cl:19-5}, simple roots are linearly independent and therefore $\abs{\Phi_S} \leq r$.
  Suppose for contradiction that $\abs{\Phi_S} < r$. Then there would exist a vector $\lambda \in \mathfrak{h}^*_\mathbb{R}$, which is orthogonal to all simple roots:
  \begin{equation}
    (\lambda, \alpha) = 0 \qquad \forall \alpha \in \Phi_S.
  \end{equation}
  But all roots are either positive or negative. And by Claim \ref{cl:19-4}, we can write all roots as linear combinations of simple roots. Therefore, $\lambda$ is orthogonal to all roots
  \begin{equation}
    (\lambda, \alpha) = 0 \qquad \forall \alpha \in \Phi.
  \end{equation}
  Then $H_{\lambda} = \lambda_{i} H^{i} \in \mathfrak{h}$
  \begin{align}
    [H_{\lambda}, H] &= 0 \qquad \forall H \in \mathfrak{h} \\
    [H_{\lambda} E^{\alpha}] &= (\lambda, \alpha) E^{\alpha}  = 0 \qquad \forall \alpha \in \Phi
  \end{align}
  This would imply that $\mathfrak{g}$ has a non-trivial ideal $I = \text{Span}_{\mathbb{C}}\left\{H_{\lambda}\right\}$. This contradicts the assumption that $\mathfrak{g}$ is simple.
\end{proof}

\begin{corollary}
  We can now choose the simple roots $\alpha_{(i)}$, $i = 1, \dots, r$, as a basis for $\mathfrak{h}^*_{\mathbb{R}}$.
  \begin{equation}
    B = \left\{ \alpha \in \Phi_S \right\} = \left\{ \alpha_{(i)} \right\}.
  \end{equation}
\end{corollary}

\begin{definition}[]
  The inner products of simple roots form the \emph{Cartan matrix}
  \begin{equation}
    A_{ij} = \frac{2(\alpha_{(i)}, \alpha_{(j)})}{(\alpha_{(j)}, \alpha_{(j)})} \qquad i,j = 1, \dots, r
  \end{equation}
\end{definition}

\begin{equation}
  \left\{ h^{i} = h^{\alpha_{(i)}}, e^{i}_{\pm} e^{\pm \alpha_{(i)}} \right\}, \qquad i = 1, \dots, r.
\end{equation}
Then we have the brackets
\begin{align}
  [h^{i}, h.j] &= 0 \\
  [h^{i}, e^{j}_\pm] &= \pm A^{ji} e^{j}_\pm \qquad \text{(no summation)} \\
  [e_+^{i}, e^{j}_-] &= \delta^{ij} h^{i}
\end{align}
This is the \emph{Chevali (?) basis}.
We will see that we can restrain the Cartan matrix very heavily, and that any simple Lie algebra is determined by its Cartan matrix.
