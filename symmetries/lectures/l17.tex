% lecture notes by Umut Özer
% course: symmetries
\lhead{Lecture 17: November 19}
Therefore, if $\alpha + \beta \in \Phi$, $\alpha + \beta \neq 0$,
 \begin{equation}
   [E^{\alpha}, E^{\beta}] =
   \begin{cases}
     N_{\alpha, \beta} E^{\alpha + \beta} & \text{if } \alpha + \beta \in \Phi \\
     0 & \text{otherwise} \\
   \end{cases}
\end{equation}

If $\alpha + \beta = 0$, then
 \begin{equation}
   [H^{i}, [E^{\alpha} , E^{-a}]]  =0 \quad \forall i = 1, \dots, r \quad\implies\quad [E^{\alpha} , E^{-\alpha}] \in \mathfrak{h}.
\end{equation}
We will define an element
\begin{equation}
  H^{\alpha} \coloneqq \frac{[E^{\alpha} , E^{-\alpha}]}{\kappa ( E^{\alpha} , E^{-\alpha})}.
\end{equation}
Consider the Killing form $\kappa(H^{\alpha} , H)$ for all $H \in \mathfrak{h}$. This is
\begin{align}
  \kappa(H^{\alpha} , H) &= \frac{1}{\kappa(E^{\alpha} , E^{-\alpha})} \kappa([E^{\alpha} , E^{-\alpha} ], H) \\
			 &\stackrel{\eqref{eq:killing-invariance}}{=} \frac{1}{\kappa(E^{\alpha}, E^{-\alpha})} \kappa (E^{\alpha}, [E^{-\alpha}, H]) \\
			 &\stackrel{\eqref{eq:16-3}}{=} \alpha(H) \cancel{\frac{\kappa(E^{\alpha}, E^{-\alpha})}{\kappa(E^{\alpha}, E^{-\alpha})}}.
\end{align}
Thus, this element solves
\begin{equation}
  \label{eq:17-star}
  \kappa(H^{\alpha}, H) = \alpha(H) \quad \forall H \in \mathfrak{h}.
\end{equation}

In components, $H^{\alpha} = c\indices{_{i}^{\alpha}} H^{i}$, where $H = c_{i} H^{i} \in \mathfrak{h}$.
Using \eqref{eq:17-star}, we have that $\kappa^{ij} c\indices{_{i}^{\alpha}} c\indices{_{j}} = \alpha^{j} c_{j}$ $\implies \kappa^{ij} c\indices{_{i}^{\alpha}}  = \alpha^{j}$ $\implies c\indices{_{i}^{\alpha}} = (\kappa^{-1})_{ij} \alpha^{j}$
\begin{equation}
  \label{eq:17-7}
  \implies \boxed{H^{\alpha} = c\indices{_{i}^{\alpha}} H^{i} = (\kappa^{-1})_{i} \alpha^{j} H^{i}}.
\end{equation}

\subsection*{Cartan-Weyl basis}%

\begin{equation}
  \left\{ H^{i}, E^{\alpha} \mid i = i, \dots, r \quad \alpha \in \Phi \subset \mathfrak{h}^* \right\}
\end{equation}
Brackets:
\begin{align}
  [H^{i}, H^{j}] &= 0 \quad \forall i, {j} = 1, \dots, r \\
  [H^{i}, E^{\alpha}] &= \alpha^{i} E^{\alpha} \quad \forall \in \Phi \\
  [E^{\alpha}, E^{\beta}] &= 
  \begin{cases}
    N_{\alpha, \beta} E^{\alpha + \beta} & \alpha + \beta \in \Phi \\
    \kappa(E^{\alpha}, E^{-\alpha}) H^{\alpha} & \alpha + \beta = 0 \\
    0 & \text{otherwise}.
  \end{cases}
\end{align}

To simplify this further, consider the brackets of $H^{\alpha} \in \mathfrak{h}$ defined above. For all $\alpha, \beta \in \Phi$,
 \begin{align}
   [H^{\alpha}, E^{\beta}] &\stackrel{\eqref{eq:17-7}}{=} (\kappa^{-1})_{ij} \alpha^{j} [H^{j}, E^{\beta}] \\
			   &\stackrel{\eqref{eq:16-2}}{=} (\kappa^{-1})_{ij} \alpha^{i} \beta^{j} E^{\beta} \\
			   &\stackrel{\eqref{eq:16-6}}= (\alpha, \beta) E^{\beta}
\end{align}
Now $\forall \alpha \in \Phi$, we can define
\begin{equation}
  e^{\alpha} = \sqrt{\frac{2}{(\alpha, \alpha)\kappa(E^{\alpha}, E^{-\alpha})}} E^{\alpha}, \qquad 
  h^{\alpha} = \frac{2}{(\alpha, \alpha)} H^{\alpha}.
\end{equation}
We have not proven that $(\alpha, \alpha) \neq 0$, but this is true (see for example Fuchs \& Schweigert pp.87).
Now we have the algebra in its final form: $\forall \alpha, \beta \in \Phi$, 
\begin{subequations} \label{eq:17-10}
  \begin{gatherbox}
    [h^{\alpha}, h^{\beta}] = 0, \qquad
    [h^{\alpha}, e^{\beta}] = \frac{2 (\alpha, \beta) e^{\beta}}{(\alpha, \alpha)}, \\
    [e^{\alpha}, e^{\beta}] = 
    \begin{cases}
      n_{\alpha, \beta} e^{\alpha + \beta} & \alpha + \beta \in \Phi \\
      h^{\alpha} & \alpha + \beta = 0 \\
      0 & \text{otherwise},
    \end{cases}
  \end{gatherbox}
\end{subequations}
where $(\alpha, \beta) = (\kappa^{-1})_{ij} \alpha^{i} \beta^{j}$ and $n_{\alpha\beta}$ are undetermined constants.

\section*{Appendix II}%

\begin{definition}[]
  A real Lie algebra $\mathfrak{g}_{\mathbb{R}}$ is of \emph{compact type} if there is a basis for which the Killing form is negative definite
  \begin{equation}
    \kappa^{ab} = -\kappa \delta^{ab}, \qquad \kappa \in \mathbb{R}^+
  \end{equation}
\end{definition}

\begin{theorem}[]
  Every complex semi-simple Lie algebra $\mathfrak{g}$ of finite dimension has a real form $\mathfrak{g}_{\mathbb{R}}$ of compact type.
\end{theorem}

\begin{exercise}
  Show that $\mathfrak{g} = \mathfrak{g}_1 \oplus \mathfrak{g}_2 \oplus \dots \oplus \mathfrak{g}_n$, where $\mathfrak{g}_i$ for $i = 1, \dots, n$ are simple.
\end{exercise}


\section{\texorpdfstring{$\mathfrak{su}(2)$}{Lie algebra of SU(2)} subalgebra}%
\label{sec:lie_algebra_of_su_2_subalgebra}

We know from Claim \ref{claim:16-inverse-root-is-root} that $\alpha \in \Phi \iff -\alpha \in \Phi$. Hence for each pair $\pm \alpha \in \Phi$, we have a subalgebra of $\mathfrak{g}$ with basis $\left\{ h^{\alpha}, e^{\alpha}, e^{-\alpha} \right\}$. From \eqref{eq:17-10}, we find the brackets
\begin{equation}
  \label{eq:17-11}
  [h^{\alpha}, e^{\pm \alpha}] = \pm 2 e^{\pm \alpha} \qquad [e^{\alpha}, e^{-\alpha}] = h\alpha,
\end{equation}
which are the same relations as \eqref{eq:L(su2)} for $\mathfrak{su}_{\mathbb{C}}(2)$.
This subalgebra is called $\mathfrak{sl}(2)_{\alpha}$. We have one such $\mathfrak{su}_{\mathbb{C}}(2)$ subalgebra for each root.

\subsection{Consequences: `Root Strings'}%

\begin{definition}[]
  For $\alpha, \beta \in \Phi$, where $\alpha \neq \beta$, the \emph{$\alpha$-string passing through $\beta$} is the set of all roots of the form $\beta + l {\alpha}$ for some $l \in \mathbb{Z}$:
  \begin{equation}
    S_{\alpha, \beta} = \left\{ \beta + l \alpha \in \Phi \mid l \in \mathbb{Z} \right\}
  \end{equation}
\end{definition}
The corresponding vector subspace of $\mathfrak{g}$ is $ V_{\alpha, \beta} = \text{Span}_{\mathbb{C}}\left\{ e^{\beta + l \alpha} \mid \beta + l \alpha \in S_{\alpha, \beta}\right\}.$
Now consider the action of $\mathfrak{sl}(2)_{\alpha}$ on $V_{\alpha, \beta}$.
\begin{equation}
  [h^{\alpha}, e^{\beta + l \alpha}] \stackrel{\eqref{eq:17-10}}{=} \frac{2 (\alpha, \beta + l \alpha)}{(\alpha, \alpha)} \; e^{\beta + l \alpha} 
  = \qty( \frac{2 (\alpha, \beta)}{(\alpha, \alpha)} + 2 l) \; e^{\beta + l \alpha} \label{eq:17-12}
\end{equation}
\begin{equation}
  \text{and} \qquad [e^{\pm \alpha}, e^{\beta + l \alpha}] 
  \begin{cases}
    \propto e^{\beta + (l  \pm 1)\alpha} & \text{if } \beta + (l \pm 1) \alpha \in \Phi \\
    = 0 &  \text{otherwise}
  \end{cases}
\end{equation}
The step operators carry us up or down the root string, but we always stay inside $V_{\alpha, \beta}$.
This means that the vector space $V_{\alpha, \beta}$ is \emph{closed} or \emph{invariant} under the ajoint action of the subalgebra $\mathfrak{sl}(2)_{\alpha}$.
Therefore, $V_{\alpha, \beta}$ is the representation space for some representation $R$ of $\mathfrak{sl}(2)_{\alpha}$. 
Moreover, \eqref{eq:17-12} tells us that the weight set of $R$ is
\begin{equation}
  \label{eq:17-13}
  S_R = \left\{ \frac{2 (\alpha, \beta)}{(\alpha ,\alpha)} + 2 l \suchthat \beta + l \alpha \in \Phi \right\}
\end{equation}
As roots of $\mathfrak{g}$ are finite in number and non-degenerate, $R$ is finite-dimensional and irreducible. 
Therefore, we must have $R \simeq R_{\Lambda}$ for some $\Lambda \in \mathbb{Z}_{\geq 0}$.
The associated weight set is 
\begin{equation}
  \label{eq:17-14}
  S_R = \left\{ -\Lambda, -\Lambda + 2, \dots, +\Lambda \right\}.
\end{equation}
Comparing \eqref{eq:17-13} and \eqref{eq:17-14}, we find two things. Firstly, $l = n \in \mathbb{Z}$ takes on all the values $n_{-} \leq n \leq n_{+}$ for some $n_{\pm} \in \mathbb{Z}$. In other words, the set of allowed roots forms an unbroken `string' $S_{\alpha, \beta} = \left\{ \beta + n \alpha \suchthat n_- \leq n \in \mathbb{Z} \leq n_+ \right\}$ of length $n_+ - n_- + 1$.
Secondly, comparing the upper and lower boundaries gives $(-\Lambda) = \frac{2 (\alpha, \beta)}{(\alpha, \alpha)} + 2 n_{-}$ and $+\Lambda = \frac{2 (\alpha, \beta)}{(\alpha, \alpha)} + 2 n_{+}$. Adding these, we find that the weight vectors or roots have angles such that their inner products are quantised in integer units:
\begin{equation}
  \label{eq:17-15}
  \boxed{\frac{2 (\alpha, \beta)}{(\alpha, \alpha)} = 2 \frac{\abs{\beta}}{\abs{\alpha}} \cos(\theta_{\alpha,\beta}) = -(n_{+}+ n_{-}) \in \mathbb{Z}}
\end{equation}

