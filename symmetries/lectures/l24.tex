% lecture notes by Umut Özer
% course: symmetries
\lhead{Lecture 24: December 05}

Now the symmetry condition \eqref{eq:symmetry} on the Lagrangian implies that we have a conserved Noether charge.
To couple scalar field to electromagnetism, we gauge the $U(1)$.
Previously, $g$ was simply an element of $U(1)$. Now, we promote $g$ to a field that depends on spacetime
\begin{equation}
  g\colon \mathbb{R}^{3, 1} \to {U}(1) \qquad
  \begin{gathered}
    \phi \to g(x) \phi \\
    \phi^* \to g^{-1}(x) \phi^*
  \end{gathered}
\end{equation}

Consider again the infinitesimal transformation $g = \exp(\epsilon X)$
\begin{equation}
  \delta_X \phi = \epsilon X(x) \phi.
\end{equation}

Once we gauge the symmetry, $\mathcal{L}_\phi$ is no longer invariant. To restore gauge invariance, we introduce a \emph{covariant derivative} $D_{\mu} = \partial_{\mu} + A_{\mu}$, where we introduced the gauge field $A_{\mu}\colon \mathbb{R}^{3, 1} \to \mathfrak{u}(1) \simeq i \mathbb{R}$, which lives in the Lie algebra of the gauge group.
The gauge field transforms as
\begin{equation}
  \delta_X A_{\mu} = - \epsilon \partial_{\mu} X.
\end{equation}

We find a consistent coupling by taking the invariant Lagrangian
\begin{equation}
  \mathcal{L} = \frac{1}{4 g^2} F_{\mu\nu} F^{\mu\nu} + (D_{\mu} \phi)^* (D^{\mu} \phi) - W(\phi^* \phi),
\end{equation}
where we replaced the ordinary derivatives by covariant ones.

In fact, we know that this is the correct Lagrangian since the variation with respect to the gauge fields gives the correct classical equations of motion.

As far as we know, this is the unique way of getting renormalisable spin-$1$ interacting theories for $U(1)$ .
We want to generalise this successful Lagrangian to other symmetry groups.

\section{Non-Abelian Gauge Theory}%
\label{sec:non_abelian_gauge_theory}

We have a gauge symmetry $\gamma$ based on a Lie group  $G$.
As in electromagnetism, start with a global symmetry:
We obtain a global symmetry by choosing a representation  $D$  of the Lie group $G$  of dimension $N$.
In general, we will think of a complex representation; in this case we can identify the representation space with  $V \simeq \mathbb{C}^N$ .
We will define the standard Hermitian inner product $(u, v) \coloneqq \vb{u}^{\dagger} \cdot \vb{v}$ $\forall u, v \in V$.
If we then have a field  $\phi \colon \mathbb{R}^{3, 1} \to V$ , then the Lagrangian $\mathcal{L}_\phi = (\partial_{\mu}, \partial^{\mu} \phi) - W [(\phi, \phi)]$  generalises the analysis of the $U(1)$  case.
The transformation of the field 
\begin{equation}
  \label{eq:24-non-abel-sym-trans}
  \phi \to D(g) \phi \qquad \forall g \in G
\end{equation}
is defined to be a  \emph{global non-Abelian symmetry transformation} if it preserves $\mathcal{L}_\phi$.
This places a restriction on the $D$ : We require that $D$  is unitary, i.e.~$D(g)^{\dagger} D(g) = \mathbb{1}_N$ .

It is quite important that $G$  is a compact group; only compact Lie groups have unitary representations of finite dimension.

As in the Abelian case, we want to focus on the infinitesimal transformations. In a Lie group, we do this by expanding near the identity:
\begin{equation}
  g = \text{Exp}(\epsilon X) \qquad
  \begin{gathered}
    \epsilon \ll 1 \\
    X \in \mathfrak{g}
  \end{gathered}
\end{equation}
From our earlier analysis, we know that the representation $D$ of the Lie group induces a representation $R \colon \mathfrak{g} \to \text{Mat}_N(\mathbb{C})$ of the Lie algebra by
\begin{equation}
  D(g) = \text{Exp}(\epsilon R(X)).
\end{equation}

\begin{definition}[]
  A `\emph{unitary}' representation $R(X)$ of the Lie algebra $\mathfrak{g}$ is anti-hermitian:
  \begin{equation}
    R(X) = - R(X)^{\dagger} \qquad \forall X \in \mathfrak{g}.
  \end{equation}
\end{definition} 

Near the identity ($\epsilon \ll 1$), we have
\begin{equation}
  D(G) \approx \mathbb{1}_N + \epsilon R(X) + O(\epsilon^2)
\end{equation}
This means we can write down an infinitesimal version of the symmetry transformation \eqref{eq:24-non-abel-sym-trans} as $\phi \to \phi + \delta_X \phi$ with  $\delta_X \phi = \epsilon R(X) \phi \in V$ .

We want to imitate the procedure of the Abelian case in electromagnetism to gauge the symmetry. We upgrade $X$  to be spacetime dependent:
\begin{equation}
  X \colon \mathbb{R}^{3, 1} \to \mathfrak{g}
\end{equation}
In particular, the symmetry transformation becomes
\begin{equation}
  \delta_X\phi = \epsilon R(X(x)) \phi \in V.
\end{equation}
Under this transformation $\mathcal{L}_\phi$  is no longer invariant. As before we introduce a gauge field $A_{\mu} \colon \mathbb{R}^{3, 1} \to \mathfrak{g}$, living in the Lie algebra of the gauge group.
The gauge transformation is
\begin{equation}
  \label{eq:24-gtrans}
  \delta_X A_{\mu} = - \epsilon \partial_{\mu} X + \epsilon [X, A_{\mu}]
\end{equation}
The first term is the same as the Abelian gauge transformation.
Indeed it is quite important that we recover Abelian gauge theory in the limit of our couplings going to zero.
However, the symmetries also allow us to a dd the unique additional term (up to coefficients) on the right.
It is an element of the Lie algebra, has the right dimension, Lorentz indices, etc \dots

We need to add this term because we want to define a covariant derivative.
The natural candidate is
\begin{equation}
  \boxed{D_{\mu} \phi = \partial_{\mu} \phi + R(A_{\mu}) \phi}.
\end{equation}
\begin{claim}
  With these definitions, we indeed have a covariant derivative. In other words, for all spacetime dependent $X \in \mathfrak{g}$, we have
  \begin{equation}
    \delta_X(D_{\mu} \phi) = \epsilon R(X) D_{\mu} \phi.
  \end{equation}
\end{claim}
\begin{proof}
  We need to compute the variation in the covariant derivative. Variation is a linear operator satisfying the Leibniz rule:
  \begin{align}
    \delta_X (D_{\mu} \phi) &= \delta_X (\partial_{\mu} \phi + R(A_{\mu}) \phi) \\
			    &= \partial_{\mu}(\delta_X \phi) + R(A_{\mu}) \phi_X \phi + R(\delta_X A_{\mu}) \phi.
  \end{align}
  We have used the fact that derivatives and representations are linear operations, meaning that we can take the variation inside them.
  \begin{exercise}
    Verify this property!
  \end{exercise}
  Applying \eqref{eq:24-gtrans} to the latter terms, we have
  \begin{align}
    \dots &= \partial_{\mu} ( \epsilon R(X) \phi) + \epsilon R(A_{\mu})R(X) \phi - \epsilon R(\partial_{\mu} X) \phi + R([X, A_{\mu}])\phi. \\
	  &= \cancel{\epsilon R(\partial_{\mu} X) \phi} + \epsilon R(X) \partial_{\mu} \phi + \epsilon R(X) R(A_{\mu}) \phi + \cancel{\epsilon [R(A_{\mu}, R(X))] \phi} - \cancel{\epsilon R(\partial_{\mu} X) \phi} + \cancel{\epsilon [R(X), R(A_{\mu})]\phi} \\
	  &= \epsilon R(X) \partial_{\mu} \phi + \epsilon R(X) R(A_{\mu}) \phi \\
	  &= \epsilon R(X) D_{\mu} \phi.
  \end{align}
\end{proof}
This means that we have a good analogue of the covariant derivative of the Abelian case.
Now the variation of the inner product is
\begin{equation}
  \delta_X (D^{\mu} \phi, D_{\mu} \phi)= \epsilon (R(X) D_{\mu} \phi, D^{\mu} \phi) + \epsilon (D_{\mu} \phi, R(X) D^{\mu} \phi) = 0,
\end{equation}
using the fact that for a unitary representation we have $R(X)^{\dagger} = - R(X)$ .

\subsection{Action for the Gauge Field}%
\label{sub:action_for_the_gauge_field}

The last thing we have to do is to find the appropriate generalisation of the Maxwell action for the non-Abelian gauge field $A_{\mu} \colon \mathbb{R}^{3, 1} \to \mathfrak{g}$; again, gauge invariance is going to do most of the work for us.
We define the field strength tensor to be
\begin{equation}
  F_{\mu\nu} = \partial_{\mu} A_{\nu} - \partial_{\mu} A_{\nu} + [A_{\mu}, A_{\nu}].
\end{equation}
\begin{claim}
  \label{cl:24-1}
  Only with these definitions does the field strength tensor transform with the adjoint representation, meaning that
  \begin{equation}
    \delta_X(F_{\mu\nu}) = \epsilon [X, F_{\mu\nu}] \in \mathfrak{g}.
  \end{equation}
\end{claim}
\begin{remark}
  In an Abelian Lie algebra, all the brackets will vanish and $F_{\mu\nu}$ is gauge invariant.
\end{remark}
\begin{proof}
  The variation obeys the Leibniz rule and can be commuted with derivatives
  \begin{equation}
    \delta_X(F_{\mu\nu}) = \partial_{\mu} (\delta_X A_{\nu}) - \partial_{\nu} ( \delta_X A_{\mu}) + [\delta_X A_{\mu}, A_{\nu}] + [A_{\mu}, \delta_X A_{\nu}].
  \end{equation}
  As we have four terms and each term gets a gauge variation with two terms, we will end up with an expression of eight terms; however, these will reduce as we will see now.
  \begin{align}
    \dots &= \cancel{-\epsilon \partial_{\mu} \partial_{\nu} X} + \epsilon \partial_{\mu} ([X, A_{\nu}]) + \cancel{\epsilon \partial_{\nu} \partial_{\mu} X} - \cancel{\epsilon \partial_{\nu} ([X, A_{\mu}])} \nonumber \\ 
	  & \qquad - \cancel{\epsilon [\partial_{\mu} X, A_{\nu}]} - \cancel{\epsilon [A_{\mu}, \partial_{\nu}X]} + \epsilon [[X, A_{\mu}] A_{\nu}] + \epsilon[A_{\mu}, [X, A_{\nu}]] \\
	  &= \epsilon [X, \partial_{\mu} A_{\nu}] - \epsilon[X, \partial_{\nu} A_{\mu}] - \epsilon [A_{\nu}, [X, A_{\mu}]] + \epsilon [A_{\mu}, [A_{\nu}, X]] \\
	  &= \epsilon [X, \partial_{\mu} A_{\nu} - \partial_{\nu} A_{\mu}] + \epsilon [X, [A_{\mu}, A_{\nu}]] \\
	  &= \epsilon [X, F_{\mu\nu}],
  \end{align}
  where we have made use of the Jacobi identity.
\end{proof}

\begin{definition}[]
  The \emph{Yang-Mills Lagrangian} $\mathcal{L}_{YM}$  is defined by using the Killing form (which is the unique invariant inner product in the Lie algebra)
  \begin{equation}
    \mathcal{L}_{\text{YM}} = \kappa[F_{\mu\nu}, F^{\mu\nu}] \times \frac{1}{g^2}.
  \end{equation}
\end{definition}
\begin{claim}
  The covariance property of Claim \ref{cl:24-1} is enough to show that this is invariant.
\end{claim}
\begin{proof}
  We have $\forall X \in \mathfrak{g}$
  \begin{align}
    \delta_X \mathcal{L}_A &= \frac{1}{g^2} \kappa (\delta_X F_{\mu\nu}, F^{\mu\nu}) + \frac{1}{g^2} \kappa(F^{\mu\nu}, \delta_X F^{\mu\nu}) \\
			   &= \frac{1}{g^2} [\kappa([X, F_{\mu\nu}], F^{\mu\nu}) + \kappa(F_{\mu\nu}, [X, F^{\mu\nu}])] = 0.
  \end{align}
\end{proof}

\begin{remark}
  You can show that this is the only Lagrangian that is quadratic in the field strength that you can write down.
\end{remark}

For $\mathfrak{g} = \mathscr{L}_{\mathbb{C}}[G]$ with representation $R_{\Lambda}$, have
\begin{equation}
  \mathcal{L} = \frac{1}{g^2} \kappa (F_{\mu\nu}, F^{\mu\nu}) + \sum_{\Lambda \in S} (D_{\mu} \phi^{\Lambda}, D^{\mu} \phi_{\Lambda}) - W[(\phi_{\Lambda}, \phi_{\Lambda})]
\end{equation}
