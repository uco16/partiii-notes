% lecture notes by Umut Özer
% course: symmetries
\lhead{Lecture 9: October 31}

\section{Baker-Campell-Hausdorff Identity}%
\label{sec:baker_campell_hausdorff_identity}

Setting $t = 1$, we have a map
\begin{equation}
  \text{Exp}\colon \mathfrak{g} \to G.
\end{equation}
\begin{claim}
  The exponential map is one-to-one in some neighbourhood of the identity $e\in G$.
\end{claim}
Given elements of the Lie algebra $X, Y \in \mathfrak{g}$, we can construct Lie group elements $g_X = \text{Exp}(X) \in G$ and $g_Y = \text{Exp}(Y) \in G$.
Group multiplication can then be recovered by requiring
\begin{equation}
  g_X g_Y = g_Z = \text{Exp}(Z) \in G.
\end{equation}
By comparing the left and hight hand side, we get the \emph{Baker-Campbell-Hausdorff} (BCH) formula
\begin{equation}
  \label{eq:BCH-formula}
  Z = X + Y + \frac{1}{2}[X, Y] + \frac{1}{12}([X, [X, Y]] - [Y, [X, Y]) + \dots,
\end{equation}
where $(\dots)$ reflects the terms that are quartic and higher in the matrices.
This series allows us to reconstruct from the bracket of the Lie algebra close to the identity the multiplication law of the group.
\begin{leftbar}
  \begin{remark}
    The formula is given purely in terms of sums of nested commutators of matrices. This reflects the fact that this is a formula in the Lie algebra. In fact, the same formula holds and makes sense even if we do not deal with matrix Lie groups.
    Moreover, this series is not unique since we can always shift around the higher order terms with the Jacobi identity.
  \end{remark}
\end{leftbar}
Therefore, $\mathfrak{g}$ determines $G$ in some neighbourhood of the identity (where the BCH series converges). More generally, $\text{Exp}$ is not globally one-to-one. In particular,
\begin{claim}
  The map $\text{Exp}$ is not surjective when $G$ is not connected.
\end{claim}
\begin{example}[$G= O(n)$]
  The Lie algebra of $O(n)$ is the set of real anti-symmetric $n \times n$ matrices:
  \begin{equation}
    \mathscr{L}(O(n)) = \left\{ x \in \text{Mat}_n(\mathbb{R}) \mid X + X^T = 0 \right\}
  \end{equation}
  Recall that the matrices $X \in \mathscr{L}(O(n))$ are traceless $\text{Tr} X = 0$.
  To find out which connected component $\text{Exp}(X)$ is in, we use the identity
  \begin{equation}
    \det(\text{Exp}(X)) = \exp(\text{Tr}X) = +1.
  \end{equation}
  Therefore, for any element $X$ of the Lie algebra, $\text{Exp}(X) \in SO(n)$. Since the image of the exponential map is not the whole of $O(n)$, but only $SO(n) \subset O(n)$, this is not surjective.
\end{example}
\begin{claim}
  In general, for compact Lie groups $G$, the image of $\mathfrak{g}$ under $\text{Exp}$ is the connected component of the identity.
\end{claim}
\begin{claim}
  The exponential map is not injective when $G$ has a $U(1)$ subgroup.
\end{claim}
\begin{example}[$G = U(1)$]
  The Lie algebra of $U(1)$ is $\mathscr{L}(U(1)) = \left\{ X = ix \in \mathbb{C} \mid x \in \mathbb{R} \right\}$. Then $g = \text{Exp}(X) = \exp(ix) \in U(1)$. Lie algebra elements $ix$ and $ix + 2 \pi i$ yield the same group element.
  The inverse of $\text{Exp}$ is multi-valued so it is not injective.
\end{example}

\section{SU(2) vs SO(3)}%
\label{sec:su_2_vs_so_3_}

We know that the Lie algebras have the same structure constants, meaning that they are isomorphic:
\begin{equation}
  \mathscr{L}(SU(2)) \simeq \mathscr{L}(SO(3)).
\end{equation}
Although we cannot construct an isomorphism $SU(2) \not\simeq SO(3)$, we can construct a \emph{double-covering}, that is a globally 2-to-one map
\begin{equation}
  \begin{split}
    d \colon SU(2) \ &\to\  SO(3) \\
    A \ &\mapsto\  d(A),
  \end{split}
\end{equation}
where $d(A)_{ij} = \frac{1}{2} \tr_2 ( \sigma_i A \sigma_j A^{\dagger})$. Note that $d(A) = d(-A)$, $\forall A \in SU(2)$.
This map provides an isomorphism between groups
\begin{equation}
  SO(3) \simeq SU(2) / \mathbb{Z}_2, \qquad \mathbb{Z}_2 = \left\{ \mathbb{1}_2, -\mathbb{1}_2 \right\} \text{ centre of SU(2)}.
\end{equation}
However, in this case this is more than merely an isomorphism between groups: Lie groups are also manifolds. 
The manifold of $SU(2)$ is $\mathcal{M}(SU(2)) \simeq S^3$.
\begin{equation}
  S^3 \simeq \left\{ \vb{x} \in \mathbb{R}^4 \mid \abs{\vb{x}}^2 = 1 \right\}.
\end{equation}
Moreover, the manifold of $SO(3)$ is $S^3$ with antipodal points identified.
Quotienting out $\mathbb{Z}_2$ is the same as taking $S^3$ and identifying antipodal points $\vb{x} \sim -\vb{x}$.
\begin{figure}[tbhp]
  \centering
  \def\svgwidth{0.3\columnwidth}
  \input{lectures/l9f1.pdf_tex}
  \caption{$S^3$ with antipodal points identified.}
  \label{fig:l9f1}
\end{figure}
This is written as
\begin{equation}
  \mathcal{M}(SO(3)) \simeq S^3_+ \cup \left\{ \text{equator with antipodal points identified} \right\},
\end{equation}
where $S^3_+$ is the upper hemisphere $x_3 \geq 0$. Note that $S^+_3 \simeq B_3$.
%Fig with upper hemisphere S3+ simeq B3 circle with two opposite points marked

\chapter{Representation Theory}%
\label{cha:representation_theory}

\begin{definition}
  A \emph{representation} of a group $G$ is a map $D \colon G \to \text{Mat}_n(F)$, $n \in \mathbb{N}$, $F = \mathbb{R} \text{ or } \mathbb{C}$, that preserves the structure  of the group. In particular, for all $g_1, g_2 \in G$ we have
  \begin{equation}
    D(g_1) D(g_2) = D(g_1 g_2).
  \end{equation}
\end{definition}
\begin{definition}[]
  A representation $D$ is \emph{faithful} if it is injective; in other words, distinct group elements have distinct representations.
\end{definition}
\begin{definition}
  For a \emph{Lie group} $G$, a representation $D$ is a group representation in the sense above, but the map $D$ must be smooth.
\end{definition}
\begin{definition}[]
  For a \emph{Lie algebra} $\mathfrak{g}$, a \emph{representation} $d$ is a map $d: \mathfrak{g} \to \text{Mat}_n(F)$ that preserves the structure of the Lie algebra. In particular, for all $X_1, X_2 \in \mathfrak{g}$ and $\alpha, \beta \in F$ we have
  \begin{enumerate}
    \item $[d(X_1), d(X_2)] = d([X_1, X_2])$
    \item $d(\alpha X_1 + \beta X_2) = \alpha d(X_1) + \beta d(X_2)$
  \end{enumerate}
\end{definition}
\begin{leftbar}
  \begin{remark}
    Again, representations of Lie groups need not be faithful.
  \end{remark}
\end{leftbar}
\begin{definition}[]
  The \emph{dimension} of a representation is the dimension $n$ of the matrices.
\end{definition}
\begin{definition}[]
  Matrices act on a vector space $V \simeq F^n$ known as \emph{representation space}.
\end{definition}
\begin{leftbar}
  \begin{remark}
    In a physical context, this will be the Hilbert space of states on which the matrix operator act.
  \end{remark}
\end{leftbar}

\begin{claim}
  There is a direct relation between representations of a Lie group $G$ and of elements in the Lie algebra $\mathfrak{g}$.
  Take a representation $D$ of a matrix Lie group $G$. Note that in general, $n = \dim D \neq \dim G = m$. We will construct a corresponding representation of the Lie algebra.
  To do this, we use again the correspondence between tangent vectors and curves.
  For $X \in \mathfrak{g}$, define a curve $C_X \colon I \in \mathbb{R} \to G$ that maps $t \mapsto g_X(t)$. We then expand $g_X(t) \simeq \mathbb{1}_n + t X + \dots$ and define the representation $d$ of the Lie algebra as
  \begin{equation}
    d(X) = \dv{}{t} \left. \left( D(g(t)) \right) \right\rvert_{t=0} \in \text{Mat}_n(F).
  \end{equation}

  %2 Nov l10
  Map the curve onto $D(g(t)) \in GL(n, F) \subset \text{Mat}_n(F)$ in the space of matrices.
  We then Taylor expand this around the origin to give
  \begin{equation}
    D(g(t)) = D(\mathbb{1}_m) + \left.\dv{D(g(t))}{t}\right\rvert_{t = 0} t + O(t^2).
  \end{equation}
  We have of course $D(\mathbb{1}_m) = \mathbb{1}_n$. Moreover, we define 
  \begin{equation}
    d(X) \coloneqq \left.\dv{D(g(t))}{t}\right\rvert_{t=0} \qquad \forall X \in \mathfrak{g}.
  \end{equation}
  This provides a representation of $\mathfrak{g}$.
\end{claim}


