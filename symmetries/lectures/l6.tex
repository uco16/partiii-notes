% lecture notes by Umut Özer
% course: symmetries
\lhead{Lecture 6: October 22}

\chapter{Lie Algebras from Lie Groups}%
\label{cha:lie_algebras_from_lie_groups}

So far, we have introduced the concepts of Lie group and Lie algebras.
These appear to be very different objects. We will now relate these two concepts by showing that every Lie group gives rise to a Lie algebra, by considering the tangent space near its identity element.

\section{Preliminaries}%
\label{sec:preliminaries}

Let $M$ be a smooth manifold of dimension $D$. Pick a point $p \in M$.
Because of the manifold structure, we can introduce a coordinate chart $\left\{ x^i \in \mathbb{R} \right\}$, with $i = i, \ldots, D$, in some region $\mathcal{P} \subset M$.
Without loss of generality, we can choose the coordinates such that the point $p$ corresponds to the origin $x^i = 0$, $\forall i$.

\begin{definition}
  The \emph{tangent space} $T_p M$ to $M$ at $p$ is a $D$-dimensional vector space spanned by differential operators $\left\{ \pdv*{}{x^j} \right\}$ acting on functions $f \colon M \to \mathbb{R}$.
  An element of the tangent space is called a \emph{tangent vector}. 
  We can expand every tangent vector $V$ as a linear combination of the differential operators
  \begin{equation}
    V = v^i \pdv{}{x^i} \in T_p  M,
  \end{equation}
  where the real coefficients $v^i \in \mathbb{R}$ are called the \emph{components} of $V$.
\end{definition}
This decomposition tells us that tangent vectors act on a function $f = f(x)$ as
\begin{equation}
  V \cdot f = v^i \left.\pdv{f(x)}{x^i} \right\rvert_{x = 0}
\end{equation}
where the evaluation at $x = 0$ signifies that we are dealing with a tangent space at point $p$, which is at the origin of our coordinates.

\begin{definition}[smooth curve]
  A \emph{smooth curve} $C$ on a manifold $M$ is a smooth map from an interval $I \in \mathbb{R}$ to $M$.
  In coordinates, we can use a parameter $t \in I$ to parametrise the curve as
  \begin{equation}
    C\colon t \mapsto x^i(t).
  \end{equation}
  We will work with curves in which the coordinates $\{ x^i (t) \}$ are continuous and differentiable at least once.
  Moreover, we choose coordinates in which $x^i(0) = 0$, $\forall i = 1, \ldots, D$.
\end{definition}
\begin{figure}[htpb]
  \centering
  \def\svgwidth{0.3\columnwidth}
  \input{lectures/tangentcurve.pdf_tex}
  \caption{Tangent vector $V_c$ to curve $C\colon I \to M$ at point $p \in M$.}
  \label{fig:tangentcurve}
\end{figure}
Now consider a curve $C\colon I \to M$ passing through the point $p$ at $x^i(0) = 0$. The tangent vector to $C$ at $p$ is
\begin{equation}
  V_c = \dot x^i(0) \pdv{}{x^i} \in T_p M,
\end{equation}
where the derivatives with respect to the parameter $t$ are denoted $\dot x^i(t) = \dv*{x^i(t)}{t}$.
Intuitively, in a physical picture where $C$ is the trajectory of the particle, parametrised by a time coordinate $t$, the tangent vector $V_c \in T_p M$ corresponds to the velocity vector at $t = 0$.
Every smooth curve has a tangent vector at every point it passes through.
This construction will also work for the end points of the parameter interval.
% F2 path with heavy dots at the end, $I = [0, L]$ written next to it

\section{The Lie Algebra \texorpdfstring{$\mathscr{L}(G) = \mathfrak{g}$}{of G}}%
\label{sec:the_lie_algebra_of_G}

Let $G$ be a Lie group of dimension $D$. Recall that Lie groups are also manfolds; we will apply the definitions above to examine certain tangent spaces on the manifold structure of $G$. Let $\left\{ \theta^i \right\}$, with $i = 1, \ldots, D$, be a set of coordinates in some coordinate patch $\mathcal{P}$ containing the identity $e \in \mathcal{P} \subset G$.
These coordinates allow us to parametrise any element $g \in \mathcal{P}$ as $g = g(\theta) \in G$. The statement that the coordinates are taken to be centered at the identity $e$ translates to $g(0) = e$.

% F3 figure just as in notes
\begin{figure}[htpb]
  \centering
  \def\svgwidth{0.55\columnwidth}
  \input{lectures/tangentspace.pdf_tex}
  \caption{Tangent space $T_p G$ at point $p$ in the manifold $G$.}
  \label{fig:tangentspace}
\end{figure}
\begin{claim}
  Let $G$ be a Lie group and consider the tangent space $T_e  G$ at the identity element $e \in G$.
  We can define a Lie bracket operation
  \begin{equation}
    [\cdot, \cdot]: T_e G \times T_e G \to T_e G
  \end{equation}
  so that the tangent space becomes a Lie algebra $\mathscr{L}(G) = \{T_e G, [\cdot , \cdot]\}$ when we equip it with this bracket structure.
\end{claim}
\begin{proof}
  It is evident that the tangent space $T_e G$ at the identity is a $D$-dimensional vector space. 
  We have to find a suitable bracket that satisfies Definition \ref{def:lie_algebra} and makes this tangent space into a Lie algebra.
  Showing this is easiest for matrix Lie groups.
  Let $G \subset \text{Mat}_n (F)$ be a subspace of $n \in \mathbb{N}$ dimensional matrices over the field $F = \mathbb{R}$ or $\mathbb{C}$.
  We can map tangent vectors to matrices by constructing the following map:
  \begin{equation}
    \begin{split}
      f \colon T_e G &\to \text{Mat}_n(F) \\
      v^i \pdv{}{\theta^i} &\mapsto v^i \left.\pdv{g(\theta)}{\theta^i} \right\lvert_{\theta = 0}.
    \end{split}
  \end{equation}
  The map $f$ is injective and linear.
  This allows us to identify the tangent space $T_e G$ with the subset of $\text{Mat}_n(F)$ spanned by $ \left\{ \left. \pdv*{g(\theta)}{\theta^i}\right\rvert_{\theta = 0} \suchthat i = 1, \ldots, D \right\}$.
  Due to this identification with matrices, the \emph{matrix commutator}, defined for any two $X, Y \in T_e G$ as
  \begin{equation}
    [X, Y] \coloneqq XY - YX,
  \end{equation}
  provides an obvious candidate for the bracket.
  It is easy to check that the defining properties of Definition \ref{def:lie_algebra} of the Lie algebra hold.
  Note in particular that the Jacobi identity follows from the associativity of matrix multiplication.
  
  It remains to show \emph{closure}: For any two tangent vectors $X, Y$, we need to show that their commutator $[X, Y]$ is itself also a tangent vector.
  To achieve this, we use the correspondence between tangent vectors and \emph{curves} on a manifold.
  Let $C: I \to G$ be a smooth curve passing through the identity $e$.
  \begin{equation}
    \begin{split}
      C \colon &I \subset \mathbb{R} \to G \\
       &t \mapsto g(t)
    \end{split}
  \end{equation}
  The parameter is chosen in such a way that $t=0$ parametrises the identity matrix: $g(0) = I_N$.
  By the chain rule, we can differentiate a group element $g$ as
  \begin{equation}
    \dv{g(t)}{t} = \dv{\theta^i (t)}{t} \pdv{g(\theta)}{\theta^i}.
  \end{equation}
  Consider the derivative at the origin:
  \begin{equation}
    \dot g(0) = \left. \dv{g(t)}{t}\right\rvert_{\mathrlap{t=0}} = \dot \theta^i(0) \left.\pdv{g(\theta)}{\theta^i}\right\rvert_{\mathrlap{\theta = 0}} \in T_e  G.
  \end{equation}
  This is the tangent vector to the curve $C$ at $e$.
  We also have an explicit representation $\dot g(\theta) \in \text{Mat}_n (F)$ for the matrix Lie group. However, they are not in general elements of the Lie group $G$.

  Near $t = 0$, we have a Taylor expansion
  \begin{equation}
    g(t) = I_n + X t + O(t^2)
  \end{equation}
  where the term $X$ appearing in the first order expansion is a tangent vector $X = \dot g(0) \in \mathscr{L}(G)$.

  Given two elements $X_1, X_2 \in \mathscr{L}(G)$, we can find smooth curves $C_1: t \mapsto g_1(t) \in G$ and $C_2: t \mapsto g_2(t) \in G$, passing through the origin $g_1(0) = g_2(0) = I_n$, such that $\dot g_1(0) = X_1$ and $\dot g_2(0) = X_2$.

  Near $t = 0$, another way of saying what we just said is that
  \begin{equation}
    g_1(t) = I_n + X_1 t + W_1 t^2 + O(t^3), \qquad
    g_2(t) = I_n + X_2 t + W_2 t^2 + O(t^3)
  \end{equation}
  for some $W_1, W_2 \in \text{Mat}_n (F)$.

  Define a new curve
  \begin{equation}
    h(t) = g_1^{-1}(t) g_2^{-1}(t) g_1(t) g_2(t) \in G.
  \end{equation}
  Since this is a composition of smooth maps, $h$ is itself smooth.
  Equivalently, we have $\forall t \in I$
  \begin{equation}
    \label{eq:h}
    g_1(t)g_2(t) = g_2(t) g_1(t) h(t).
  \end{equation}
  Expanding this near $t=0$ in steps,
  \begin{align}
    g_1(t) g_2(t) &= I_n + (X_1 + X_2)t + (X_1 X_2 + W_1 + W_2) t^2 + O(t^3) \\
    g_2(t) g_1(t) &= I_n + (X_1 + X_2)t + (X_2 X_1 + W_1 + W_2) t^2 + O(t^3).
  \end{align}
  Here we can already see the terms $X_1 X_2$ and $X_2 X_1$ that we will want to isolate for our Lie bracket.

  Now $h(t)$ has a Taylor series of the form
  \begin{equation}
    h(t) = I_n + h_1 t + h_2 t^2 + O(t^3).
  \end{equation}
  Plugging this into \eqref{eq:h} and using the above expansions for $g_1 g_2$ and $g_2 g_1$, we find that the coefficients in the Taylor series expansion are given by
  \begin{equation}
    h_1 = 0 \qquad h_2 = (X_1 X_2 - X_2 X_1) = [X_1, X_2].
  \end{equation}
  Thus, the curve $h_3$ is written in its Taylor series expansion as
  \begin{equation}
    h(t) = I_n + t^2 [X_1, X_2] + O(t^3).
  \end{equation}
  However, a curve should have its tangent vector in the linear term of its parameter.
  To have a sneeky way around this issue, we redefine our parameter $s = t^2 \geq 0$, where $e$ lies at the end point.
  This defines a new curve $C_3: s \mapsto g_3(s) = h(\sqrt{s})$, which near $s = 0$ has the form
  \begin{equation}
    g_3(s) = I_n + s[X_1, X_2] + O(s^{3/2}).
  \end{equation}
  This is a good curve in $C^1$, provided that $s \geq 0$. 
  Finally, we can isolate the commutator by taking the derivative and evaluating it at $s = 0$:
  \begin{equation}
    [X_1, X_2] = \left.\dv{g_3(s)}{s}\right\rvert_{\mathrlap{s = 0}} = \dot g_3(0) \in \mathscr{L}(G)
  \end{equation}
  This demonstrates the closure property of the Lie algebra, since the bracket of two tangent vectors gives another.
\end{proof}
Note that the second derivative of $g_3$ will give us a negative power, and as $s \to 0$, the second derivative would blow up. So $\ddot g_3(0)$ does not exist. As slightly longer proof can construct good $C^n$ curves for any $n$, but $n = 1$ is sufficient for this proof.
