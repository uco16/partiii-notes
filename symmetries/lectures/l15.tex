% lecture notes by Umut Özer
% course: symmetries
\lhead{Lecture 15: November 15}

In fact, for simple Lie algebras, the Killing form $\kappa$ is the unique invariant inner product, up to an overall scalar multiple.

\begin{definition}[]
  A Lie algebra is \emph{semi-simple} if it has no Abelian ideals.
\end{definition}

\begin{claim}
  A finite-dimensional semi-simple Lie algebra can be written as the direct sum of a finite number of simple Lie algebras.
\end{claim}
\begin{proof}
  Sheet 2 Qu 9b.
\end{proof}
\begin{leftbar}
  \begin{note}
    The underlying set for the vector space direct sum $\mathfrak{g}_1 \oplus \mathfrak{g}_2$ is 
    \begin{equation}
      \mathfrak{g} \times \mathfrak{g}' = \left\{ (x, x') \suchthat x \in \mathfrak{g},\; x' \in \mathfrak{g}' \right\}.
    \end{equation}
    where scalar multiplication and addition is defined as
    \begin{gather}
      c(x, x') \coloneqq (cx, cx') \\
      (x, x') + (y, y') \coloneqq (x + y, x' + y').
    \end{gather}
    In order to make this a Lie algebra, we define the commutator to be
    \begin{equation}
      [(x, x'), (y, y')] \coloneqq([x, y], [x', y']).
    \end{equation}
    Note that the subsets $\mathfrak{g}, \mathfrak{g}'$ commute with each other: $[(x, 0), (0, x')] = 0$.
  \end{note} 
\end{leftbar}

We will now try to find out under what conditions the Killing form is non-degenerate.
\begin{theorem}[Cartan]
  \label{thm:cartan}
  The Killing form $\kappa$ is non-degenerate if and only if the associated Lie algebra $\mathfrak{g}$ is semi-simple.
\end{theorem}
\begin{proof}[Proof of forward direction only]
  Assume that $\kappa$ is non-degenerate.
  Suppose for contradiction that $\mathfrak{g}$ is not semi-simple.
  This means that $\mathfrak{g}$ has an Abelian ideal $\mathfrak{j}$.
  Denote $\dim(\mathfrak{g}) = D$ and $\dim(\mathfrak{j}) = d$.
  Choose a basis
  \begin{equation}
    B = \left\{ T^{a} \right\} = \left\{ T^{i} \mid i = 1, \dots, d \right\} \cup \left\{ T^{\alpha} \mid \alpha =1, \dots, D-d \right\},
  \end{equation}
  where $\left\{ T^{i} \right\}$ span $\mathfrak{j}$. As $\mathfrak{j}$ is Abelian, we must have $[T^{i}, T^{j}] = 0$, $\forall i, j$.
  Moreover, as $\mathfrak{j}$ is an ideal, $[T^{\alpha}, T^{j}] = f \indices{^{\alpha j}_{\kappa}} T^{k} \in \mathfrak{j}$ and therefore $f^{ij}_{a} = 0$ and $f^{\alpha j}_{\beta} =0$.
  For $X = X_{a} T^{a} \in \mathfrak{g}$ and $Y = Y_{j} T^{j} \in \mathfrak{j}$, we have $\kappa[X, Y] = \kappa^{ai} X_{a} Y_{i}$ with
  \begin{equation}
    \kappa^{ai} = f^{ad}_{c} f^{ic}_{d} = f^{aj}_{\alpha} f^{i\alpha}_{j} = 0.
  \end{equation}
  Therefore, we have $K[X, Y] = 0$ for all $X \in \mathfrak{j}$ and all $X \in \mathfrak{g}$. In other words, $\kappa$ is degenerate, which contradicts the assumption.
  Hence, $\mathfrak{g}$ is semi-simple.
\end{proof}

\begin{definition}[compact type]
  A real Lie algebra $\mathfrak{g}_{\mathbb{R}}$ is of \emph{compact type} if there is a basis for which the Killing form is negative definite
  \begin{equation}
    \kappa^{ab} = -\kappa \delta^{ab}, \qquad \kappa \in \mathbb{R}^+
  \end{equation}
\end{definition}

\begin{leftbar}
  \begin{claim}
    If $\mathscr{L}(G)$ is of compact type, then $G$ is compact (as a topological manifold).
  \end{claim}
\end{leftbar}


\begin{theorem}[]
  Every complex semi-simple Lie algebra $\mathfrak{g}$ of finite dimension has a real form $\mathfrak{g}_{\mathbb{R}}$ of compact type.
\end{theorem}

\begin{exercise}
  Show that $\mathfrak{g} = \mathfrak{g}_1 \oplus \mathfrak{g}_2 \oplus \dots \oplus \mathfrak{g}_n$, where $\mathfrak{g}_i$ for $i = 1, \dots, n$ are simple.
\end{exercise}



\section{Complexification}%
\label{sec:complexification}

Given a \emph{real} Lie algebra $\mathfrak{g}$, we can find a basis $\left\{ T^{a} \right\}$, $a = 1, \dots, \dim \mathfrak{g}$, with real structure constants
\begin{equation}
  \label{eq:15-star}
  [T^{a}, T^{b}] = f^{ab}_{c} T^{c}, \qquad f^{ab}_{c} \in \mathbb{R}.
\end{equation}
\begin{definition}[]
  Given a real Lie algebra $\mathfrak{g} = \text{Span}_{\mathbb{R}} \left\{ T^{a} \right\}$, its \emph{complexification} is $\mathfrak{g}_{\mathbb{C}} = \text{Span}_{\mathbb{C}} \left\{ T^{a} \right\}$.
  We say that $\mathfrak{g}$ is a \emph{real form} of $\mathfrak{g}_\mathbb{C}$.
\end{definition}
Together with the bracket \eqref{eq:15-star}, $\mathfrak{g}_{\mathbb{C}}$ is a \emph{complex Lie algebra.}
A complex Lie algebra can have multiple real forms.
\begin{example}[]
  Consider the Lie algebra of $SU(2)$:
  \begin{equation}
    \mathscr{L}(SU(2)) = \mathfrak{su}(2) = \text{span}_{\mathbb{R}} \left\{ T^{a} = -\frac{i\sigma_{a}}{2}  \suchthat a = 1,2,3 \right\}.
  \end{equation}
  This is the set of $2 \times 2$ traceless anti-Hermitian matrices.
  Its complexification is
  \begin{equation}
    \mathscr{L}_{\mathbb{C}}(SU(2)) = \mathfrak{su}_{\mathbb{C}}(2) = \text{Span}_{\mathbb{C}}\left\{T^{a} = -\frac{i\sigma_{a}}{2} \suchthat a = 1,2,3\right\},
  \end{equation}
  which is the set of $2 \times 2$ traceless complex matrices.
\end{example}

\begin{definition}
  The \emph{Cartan-Weyl basis} for $\mathfrak{su}_{\mathbb{C}}(2)$ is given by the \emph{Cartan element} $H = 2 i T^3$ and the two elements $E_\pm = i T^1 \pm T^2$ with brackets
  \begin{equation}
    [H, E_\pm] = \pm 2 E_{\pm}, \qquad [E_+, E_-] = H.
  \end{equation}
\end{definition}

\subsection{Recovering a Real Representation}%

Let us star from a representation $R$ of $\mathfrak{su}_{\mathbb{C}}(2)$ with
\begin{subequations}
  \begin{align}
    \label{eq:15-star2}
    [R(H), R(E_{\pm})] &= \pm 2 R(E_\pm) \\
    [R(E_+), R(E_-)] &= R(H).
  \end{align}
\end{subequations}
We can pass back to original basis via
\begin{subequations}
  \begin{align}
    R(T^1) &= \frac{1}{2i} \bigl(R(E_+) + R(E_-)\bigr), \\
    R(T^2) &= \frac{1}{2i} \bigl(R(E_+) - R(E_-)\bigr), \\
    R(T^3) &= \frac{1}{2i} R(H).
  \end{align}
\end{subequations}
For all $X \in \mathfrak{su}(2)$, we can expand $X = X_{a} T^{a}$ for some ${X_{a} \in \mathbb{R}}$. Set $R(X) = X_{a} R(T^{a})$ to get a representation of $\mathfrak{su}(2)$.


\section{Cartan Subalgebra}%
\label{sec:cartan_subalgebra}

\begin{definition}[ad-diagonalisable]
  We say that an element $X \in \mathfrak{g}$ is \emph{ad-diagonalisable} (or \emph{semisimple}), if the map $\text{ad}_X$ is diagonalisable, meaning that there exists a basis $B = \left\{T^a\right\}$ of $\mathfrak{g}$ such that $[X, T^a]$ is proportional to $T^a$ for any element of $B$.
\end{definition}
\begin{definition}[Cartan subalgebra]
  A \emph{Cartan subalgebra} $\mathfrak{h}$ of $\mathfrak{g}$ is a maximal Abelian subalgebra consisting entirely of ad-diagonalisable elements:
  \begin{enumerate}
    \item ad-diagonalisable: $H \in \mathfrak{h} \implies H$ is ad-diagonalisable,
    \item Abelian: $H, H' \in \mathfrak{h} \implies [H, H'] = 0$,
    \item maximal: $X \in \mathfrak{g}$ with $[X, H] = 0$ for all $H \in \mathfrak{h}$ $\implies$ $X \in \mathfrak{h}$.
  \end{enumerate}
\end{definition}
\begin{claim}
  All possible Cartan subalgebras of $\mathfrak{g}$ are isomorphic and have the same dimension. 
\end{claim}
This motivates the following definition:
\begin{definition}
  The \emph{rank} of a Lie algebra $\mathfrak{g}$ is the dimension of a Cartan subalgebra $\mathfrak{h}$ of $\mathfrak{g}$.
  \begin{equation}
    \operatorname{rank} \mathfrak{g} \coloneqq \dim \mathfrak{h}.
  \end{equation}
\end{definition}
\begin{leftbar}
  In physics, $\operatorname{rank} \mathfrak{g}$ provides the maximal number of quantum numbers which can be used to label (at least partially) the states of a physical system that has $\mathfrak{g}$ as its symmetry algebra. \cite{fuchs}
\end{leftbar}


\begin{example} 
  Let $\mathfrak{g} = \mathfrak{su}_{\mathbb{C}}(2) = \text{span}_{\mathbb{C}} \left\{ H, E_{\pm} \right\}$.  The Cartan element $H$ is ad-diagonalisable, but $E_{\pm}$ are not, so we can choose $\mathfrak{h} = \text{Span}_{\mathbb{C}}\left\{H\right\}$ to be a Cartan subalgebra. Hence $\text{rank}(\mathfrak{g}) = 1$. 
\end{example}
Given a Cartan subalgebra $\mathfrak{h}$, we can choose a basis $\left\{ H^{i}\right\}$, $ i = 1, \dots, r$, where $[H^{i}, H^{j}] = 0$.
\begin{example}[]
  Consider the set of traceless complex $n \times n$-matrices, denoted $\mathfrak{sl}(n, \mathbb{C})$.
  Choose $(H^{i})_{\alpha\beta} = \delta_{\alpha i} \delta_{\beta i} - \delta_{\alpha i + 1} \delta_{\beta i + 1}$.
  Then 
  \begin{equation}
    \operatorname{rank} [\mathfrak{sl}(n, \mathbb{C})] = n-1.
  \end{equation}
\end{example}

\section{Roots}%
\label{sec:roots}

The fact that the Cartan subalgebra is Abelian implies
\begin{description}
  \item[$\implies$] by the defining property of the adjoint rep.: $ [\text{ad}_{H^{i}}, \text{ad}_{H^{j}}] = \text{ad}_{[H^{i}, H^{j}]} = 0 $,
  \item[$\implies$] the $r$ linear maps $\text{ad}_{H^{i}}\colon \mathfrak{g} \to \mathfrak{g}$, with $i = d, \dots, r$ are \emph{simultaneously diagonalisable},
  \item[$\implies$] the Lie algebra $\mathfrak{g}$ is spanned by simultaneous eigenvectors of $\text{ad}_{H^{i}}$,
    %\begin{equation}
    %  \boxed{ad_{H^{i}} (E^{\alpha}) = [H^{i}, E^{\alpha}] = \alpha^{i} E^{\alpha}, \quad i = d, \dots, r}
    %\end{equation}
\end{description}
