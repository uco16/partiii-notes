% lecture notes by Umut Özer
% course: symmetries
\lhead{Lecture 15: November 15}

In fact, for simple Lie algebras, the Killing form $\kappa$ is the unique invariant inner product, up to an overall scalar multiple.

\begin{definition}[]
  A Lie algebra is \emph{semi-simple} if it has no Abelian ideals.
\end{definition}

\begin{exercise}[Sheet 2 Qu 9b]
  Show that a finite-dimensional semi-simple Lie algebra can be written as the direct sum of a finite number of simple Lie algebras.
\end{exercise}

We will now try to find out under what conditions the Killing form is non-degenerate.
\begin{theorem}[by Cartan]
  The Killing form $\kappa$ is non-degenerate if and only if the associated Lie algebra $\mathfrak{g}$ is semi-simple.
\end{theorem}
\begin{proof}[Proof of forward direction only]
  Assume that $\kappa$ is non-degenerate.
  Suppose for contradiction that $\mathfrak{g}$ is not semi-simple.
  This means that $\mathfrak{g}$ has an Abelian ideal $\mathfrak{j}$.
  Denote $\dim(\mathfrak{g}) = D$ and $\dim(\mathfrak{j}) = d$.
  Choose a basis
  \begin{equation}
    B = \left\{ T^{a} \right\} = \left\{ T^{i} \mid i = 1, \dots, d \right\} \cup \left\{ T^{\alpha} \mid \alpha =1, \dots, D-d \right\},
  \end{equation}
  where $\left\{ T^{i} \right\}$ span $\mathfrak{j}$. As $\mathfrak{j}$ is Abelian, we must have $[T^{i}, T^{j}] = 0$, $\forall i, j$.
  Moreover, as $\mathfrak{j}$ is an ideal, $[T^{\alpha}, T^{j}] = f \indices{^{\alpha j}_{\kappa}} T^{k} \in \mathfrak{j}$ and therefore $f^{ij}_{a} = 0$ and $f^{\alpha j}_{\beta} =0$.
  For $X = X_{a} T^{a} \in \mathfrak{g}$ and $Y = Y_{j} T^{j} \in \mathfrak{j}$, we have $\kappa[X, Y] = \kappa^{ai} X_{a} Y_{i}$ with
  \begin{equation}
    \kappa^{ai} = f^{ad}_{c} f^{ic}_{d} = f^{aj}_{\alpha} f^{i\alpha}_{j} = 0.
  \end{equation}
  Therefore, we have $K[X, Y] = 0$ for all $X \in \mathfrak{j}$ and all $X \in \mathfrak{g}$. In other words, $\kappa$ is degenerate, which contradicts the assumption.
  Hence, $\mathfrak{g}$ is semi-simple.
\end{proof}

\section{Complexification}%
\label{sec:complexification}

Given a \emph{real} Lie algebra $\mathfrak{g}$, we can find a basis $\left\{ T^{a} \right\}$, $a = 1, \dots, \dim \mathfrak{g}$, with real structure constants
\begin{equation}
  \label{eq:15-star}
  [T^{a}, T^{b}] = f^{ab}_{c} T^{c}, \qquad f^{ab}_{c} \in \mathbb{R}.
\end{equation}
\begin{definition}[]
  Given a real Lie algebra $\mathfrak{g} = \text{Span}_{\mathbb{R}} \left\{ T^{a} \right\}$, we define the \emph{complexification} $g_{\mathbb{C}} = \text{Span}_{\mathbb{C}} \left\{ T^{a} \right\}$.
\end{definition}
Together with the bracket \eqref{eq:15-star}, $g_{\mathbb{C}}$ is a \emph{complex Lie algebra.}
\begin{example}[]
  Consider the Lie algebra of $SU(2)$:
  \begin{align}
    \mathscr{L}(SU(2)) = \mathfrak{su}(2) &= \text{span}_{\mathbb{R}} \left\{ T^{a} = -\frac{i\sigma_{a}}{2} \mid a = 1,2,3 \right\} \\
		     &= \left\{ 2 \times 2 \text{ traceless anti-Hermitian matrices} \right\}.
  \end{align}
  Its complexification is
  \begin{align}
    \mathscr{L}_{\mathbb{C}}(SU(2)) = \mathfrak{su}_{\mathbb{C}}(2) &= \text{Span}_{\mathbb{C}}\left\{T^{a} = -\frac{i\sigma_{a}}{2} \mid a = 1,2,3\right\} \\
    &= \left\{ 2 \times 2 \text{ traceless complex matrices} \right\}.
  \end{align}
\end{example}

The Cartan-Weyl basis for $\mathfrak{su}_{\mathbb{C}}(2)$ is given by $H = 2 i T^3$ and $E_\pm = i T^1 \pm T^2$ with brackets
\begin{equation}
  [H, E_\pm] = \pm 2 E_{\pm} \qquad [E_+, E_-] = H.
\end{equation}

\subsection*{Appendix}%

Starting from a representation $R$ of $\mathfrak{su}_{\mathbb{C}}(2)$ with
\begin{subequations}
  \begin{align}
    \label{eq:15-star2}
    [R(H), R(E_{\pm})] &= \pm 2 R(E_\pm) \\
    [R(E_+), R(E_-)] &= R(H).
  \end{align}
\end{subequations}
Pass back to original basis via
\begin{subequations}
  \begin{align}
    R(T^1) &= \frac{1}{2i} \bigl(R(E_+) + R(E_-)\bigr) \\
    R(T^2) &= \frac{1}{2} \bigl(R(E_+)) - R(E_-)\bigr) \\
    R(T^3) &= \frac{1}{2i} R(H).
  \end{align}
\end{subequations}
For all $X \in \mathfrak{su}(2)$, we can expand $X = X_{a} T^{a}$ for some ${X_{a} \in \mathbb{R}}$. Set $R(X) = X_{a} R(T^{a})$ to get a representation of $\mathfrak{su}(2)$.


\section{Cartan Classification}%
\label{sec:cartan_classification}

Can classify \emph{all} finite-dimensional simple complex $\mathfrak{g}$ (Cartan 1894).
\begin{definition}[]
  We say that $X \in \mathfrak{g}$ is \emph{ad-diagonalisable} (AD), if $\text{ad}_X \colon \mathfrak{g} \to \mathfrak{g}$ is diagonalisable.
\end{definition}
\begin{definition}[]
  A \emph{Cartan subalgebra} (CSA) $\mathfrak{h}$ of $\mathfrak{g}$ is a maximal Abelian subalgebra,
  \begin{enumerate}
    \item $H \in \mathfrak{h} \implies H$ is (AD)
    \item $H, H' \in \mathfrak{h} \to  [H, H'] = 0$
    \item if $X \in \mathfrak{g}$ and $[X, H] = 0$ for all $H \in \mathfrak{h}$, then $X \in \mathfrak{h}$
  \end{enumerate}
\end{definition}
In fact, all possible CSAs of $\mathfrak{g}$ are isomorphic and have the same dimension, $r = \dim \mathfrak{h} \in \mathbb{N}$, which is the \emph{rank} of $\mathfrak{g}$.
\begin{example}[]
  $\mathfrak{g} = \mathfrak{su}_{\mathbb{C}}(2) = \text{span}_{\mathbb{C}} \left\{ H, E_{\pm} \right\} \implies \text{rank}(\mathfrak{g}) = 1$, $H$ is (AD): $[H, E_{\pm}] = \pm 2 E_{\pm}$ and $[H, H] = 0$, but $E_\pm$ are \emph{not}.
\end{example}
$\mathfrak{h} = \text{span}_{\mathbb{C}} \left\{ H \right\}$ is choice of CSA. Choose basis $\left\{ H^{i} \mid i = 1, \dots, r \right\}$, $[H^{i}, H^{j}] = 0$.
\begin{example}[]
  $\mathfrak{su}_{\mathbb{C}}(N) = \left\{ \text{traceless complex $n \times n$-matrices} \right\}$ (why not anti-hermitian?).
  Choose $(H^{i})_{\alpha\beta} = \delta_{\alpha i} \delta_{\beta i} - \delta_{\alpha i + 1} \delta_{\beta i + 1}$.
  Then $\text{rank} [\mathfrak{su}_{\mathbb{C}}(N)] = N-1$, $[H^{i}, H^{j}] = 0$ for all $i,j = 1, \dots, r$.
  \begin{description}
    \item[$\implies$] $(ad_{H^{i}} \circ ad_{H^{j}} - ad_{H^{j}} \circ ad_{H^{i}}) = 0$ (def.~property of adjoint rep)
    \item[$\implies$] $r$ linear maps $ad_{H^{i}}\colon \mathfrak{g} \to \mathfrak{g}$, with $i = d, \dots, r$ are \emph{simultaneously diagonalisable}
    \item[$\implies$] $\mathfrak{g}$ is spanned by simultaneous eigenvectors $ad_{H^{i}}$
      \begin{equation}
	\boxed{ad_{H^{i}} (E^{\alpha}) = [H^{i}, E^{\alpha}] = \alpha^{i} E^{\alpha}, \quad i = d, \dots, r}
      \end{equation}
  \end{description}
\end{example}
