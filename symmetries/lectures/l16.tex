% lecture notes by Umut Özer
% course: symmetries
\lhead{Lecture 16: November 16}
\begin{itemize}
  \item Jacobi
    \begin{equation}
      \label{eq:16-alpha}
      [X, [Y, Z]] + [Y, [Z, X]] + [Z, [X, Y]] = 0
    \end{equation}
  \item Adjoint repn
    \begin{equation}
      \label{eq:16-beta}
      ad_{[X, Y]} = ad_X \circ ad_Y - ad_Y \circ ad_X
    \end{equation}
  \item Killing form invariance
    \begin{equation}
      \label{eq:16-delta}
      \kappa([Z, X], Y) + \kappa(X, [Z, Y]) =0
    \end{equation}
\end{itemize}

Cartan-Weyl basis
\begin{equation}
  \mathfrak{g} = \mathfrak{su}_{\mathbb{C}}(2) \qquad \left\{ H, E_+, E_- \right\}
\end{equation}
\begin{equation}
  [H, E_{\pm}] = \pm 2E_{\pm} \qquad [E_+, E_-] = H
\end{equation}

Cartan subalgebra $\mathfrak{h} \subset \mathfrak{g}$ of dimension $\rank(\mathfrak{g})$.
Basis for $\mathfrak{h}$ is $\left\{ H^{i} \right\}$, $i = 1, \dots, r$ with
\begin{equation}
  \label{eq:16-1}
  [H^{i}, H^{j}] = 0 \quad \forall i, j
\end{equation}
$\mathfrak{g}$ is spanned by simultaneous eigenvectors of $ad_{H^{i}} \colon \mathfrak{g} \to \mathfrak{g}$.
\begin{description}
  \item[Zero eigenvalues] correspond to $\left\{ H^{i} \right\}$ $j = 1, \dots, r$
    \begin{equation}
      ad_{H^{i}}(H^{j}) = [H^{i}, H^{j}] = 0
    \end{equation}
  \item[Non-zero eigenvalues] $\left\{ E^{\alpha} \right\}$, $\alpha \in \Phi$
    \begin{equation}
      \label{eq:16-2}
      ad_{H^{i}}(E^{\alpha}) = [H^{i}, E^{\alpha}] = \alpha^{i} E^{\alpha}
    \end{equation}
    By construction, $\alpha^{i} \in \mathbb{C}$ are not all zero. The components $\alpha$ define a \emph{root} of the Lie algebra.
\end{description}
For a general element $H \in \mathfrak{h}$, we can write $H = e_{i} H^{i}$ with $e_{i} \in \mathbb{C}$.
By \eqref{eq:16-2}, we have
\begin{equation}
  \label{eq:16-3}
  [H, E^{\alpha}] = \alpha(H) E^{\alpha}
\end{equation}
with $\alpha(H) = e_{i} \alpha^{i} \in \mathbb{C}$.
Each root determines a \emph{linear map} $\alpha \colon \mathfrak{h} \to \mathbb{C}$. We should think of $\alpha$ as an element of $\mathfrak{h}^*$, which is the dual space to $\mathfrak{h}$.
In general, we have a problem of degenerate eigenvalues.
In a full proof of the Cartan construction, one can prove that the roots are non-degenerate if $\mathfrak{g}$ is simple; each root corresponds to precisely one of the step generators $E^{\alpha}$.
A more complete story is given by Fuchs and Schweigert.
We will assume this from now on.
This means that the set $\Phi$ of roots of the Lie algebra must consist of $d-r$ distinct elements of $\mathfrak{h}^*$. Here $d = \dim(\mathfrak{g})$ and $r = \dim(\mathfrak{h})$.
\begin{definition}[]
  The \emph{Cartan-Weyl basis} is $B = \left\{ H^{i} \mid i=1, \dots, r \right\} \cup \left\{ E^{\alpha} \mid \alpha \in \Phi \right\}$.
\end{definition}
By Cartan's theorem, if $\mathfrak{g}$ is simple, then the Killing form with normalisation $N$,
\begin{equation}
  \label{eq:16-4}
  \kappa(X, Y) = \frac{1}{N} \Tr[ad_X \circ ad_Y],
\end{equation}
is non-degenerate.
We will always assume $\mathfrak{g}$ is simple from now on.
Let us evaluate $\kappa$ in the Cartan-Weyl basis.
\begin{claim}
  \begin{enumerate}
    \item $\forall H \in \mathfrak{h}$, $\forall \alpha \in \Phi$, we have $\kappa(H, E^{\alpha}) = 0$
    \item $\forall \alpha, \beta \in \Phi$, such that $\alpha + \beta \neq 0$, we have $\kappa(E^{\alpha}, E^{\beta}) =0$
    \item non-degeneracy of $\kappa$ then implies that $\forall H \in \mathfrak{h}$, $\exists H' \in \mathfrak{h}$ such that $\kappa(H, H') \neq 0$.
  \end{enumerate}
\end{claim}
\begin{proof}[Proof of 1]
  \item For any $H' \in \mathfrak{h}$, we use the relation \eqref{eq:16-3} to write
    \begin{equation}
      \alpha(H') \kappa(H, E^{\alpha}) = \kappa(H, [H', E^{\alpha]})
    \end{equation}
    We then use the invariance \eqref{eq:16-delta} of the Killing form, we get
    \begin{equation}
      \dots = -\kappa ([H', H], E^{\alpha})
    \end{equation}
    Finally, using \eqref{eq:16-1}, we have
    \begin{equation}
      \dots = -\kappa (0, E^{\alpha}) = 0
    \end{equation}
    \begin{equation}
      \alpha(H') \neq 0 \quad \implies \quad \kappa(H, E^{\alpha}) = 0.
    \end{equation}
\end{proof}
\begin{proof}[Proof of 2]
  For all $H' \in \mathfrak{h}$, we have
  \begin{align}
    (\alpha(H') + \beta(H')) \kappa(E^{\alpha}, E^{\beta}) &\stackrel{\eqref{eq:16-3}}{=} \kappa([H', E^{\alpha]}, E^{\beta}) + \kappa(E^{\alpha}, [H', E^{\beta}]) \\
							   &\stackrel{\eqref{eq:16-delta}}{=} 0
  \end{align}
  Nor for all $\alpha,\beta \in \Phi$ such that $\alpha + \beta \neq 0$, we have 
  \begin{equation}
    \alpha(H') + \beta(H') \neq 0 \text{ for some } H' \quad \implies \quad \kappa(E^{\alpha}, E^{\beta}) = 0.
  \end{equation}
\end{proof}
\begin{proof}[Proof of 3]
  For some $H \in \mathfrak{h}$, assume for contradiction that $\kappa(H, H') = 0$, $\forall H' \in \mathfrak{h}$. From 1, $\kappa(H, E^{\alpha}) = 0$ for all $\alpha \in \Phi$. This means that
  \begin{equation}
    \kappa(H, X) = 0 \quad \forall X \in \mathfrak{g} \quad \implies \quad \kappa \text{ degenerate}.
  \end{equation}
  This is a contradiction.
\end{proof}
The third statement implies that $\kappa$ is a non-degenerate inner product on $\mathfrak{h}$
\begin{equation}
  \kappa(H, H') = \kappa^{ij} l_{i} l'_{j},
\end{equation}
where $H = l_{i} H^{i}$ and $H' = l_{i}' H^{i}$.
This means that $\kappa^{ij} = \kappa(H^{i}, H^{j})$ is an invertible $r \times r$ matrix. Explicitly, there exists an inverse matrix $\kappa^{-1}$ such that $(\kappa^{-1})_{ij} \kappa^{jk} = \delta^{k}_{i}$.
$\kappa^{-1}$ defines a non-degenerate inner product on the dual space $\mathfrak{h}^*$.
This is where the roots live! Suppose we have roots $\alpha, \beta \in \Phi$ such that
\begin{equation}
  [H^{i}, E^{\alpha}] = \alpha^{i} E^{\alpha}, \qquad [H^{i}, E^{\beta}] = \beta^{i} E^{\beta}.
\end{equation}
We then define an inner product
\begin{equation}
  \label{eq:16-6}
  \boxed{(\alpha, \beta) \coloneqq (\kappa^{-1})_{ij} \alpha^{i} \beta^{j}}
\end{equation}
\begin{claim} \label{claim:16-inverse-root-is-root}
  If $\alpha \in \Phi$ is a root, then $-\alpha \in \Phi$ is too and $\kappa(E^{\alpha}, E^{-\alpha}) \neq 0$.
\end{claim}
\begin{proof}
  \begin{description}
    \item[From 1:] $\kappa(E^{\alpha}, H) = 0 \quad \forall H \in \mathfrak{h}$
    \item[From 2:] $\kappa(E^{\alpha}, E^{\beta}) = 0 \quad \forall \beta \in \Phi$ with $\alpha \neq -\beta$
  \end{description}
  Hence, unless $-\alpha \in \Phi$ and $\kappa(E^{\alpha}, E^{\alpha}) \neq 0$, we have $\kappa(E^{\alpha}, X) = 0$ for all elements $X \in \mathfrak{g}$, which would mean that $\kappa$ is degenerate, being a contradiction.
\end{proof}

\section{Algebra in Cartan-Weyl basis}%
\label{sec:algebra_in_cartan_weyl_basis}

So far, we have the following brackets:
\begin{align}
  [H^{i}, H^{j}] &= 0 \quad \forall i , j = 1, \dots, r \\
  [H^{i}, E^{\alpha}] &= \alpha^{i} E^{\alpha} \quad \alpha \in \Phi.
\end{align}
It remains to evaluate $[E^{\alpha}, E^{\beta}] \quad \forall \alpha, \beta \in \Phi$.
We can use the Jacobi identity \eqref{eq:16-alpha}, 
\begin{align}
  [H^{i}, [E^{\alpha}, E^{\beta}]] &\stackrel{\eqref{eq:16-alpha}}{=} [E^{\alpha}, [E^{\beta}, H^{i}]] - [E^{\beta}, [H^{i}, E^{\alpha}]] \\
				   &\stackrel{\eqref{eq:16-3}}{=} (\alpha^{i} + \beta^{i}) [E^{\alpha}, E^{\beta}]
\end{align}
