% lecture notes by Umut Özer
% course: symmetries
\lhead{Lecture 16: November 16}

The eigenvectors of $ad_{H^{i}}$ within the Cartan subalgebra can be distinguished by their eigenvectors:
\begin{description}
  \item[Zero eigenvalues] correspond to the \emph{Cartan elements} $H^{i}$:
    \begin{equation}
      \text{ad}_{H^{i}}(H^{j}) = [H^{i}, H^{j}] = 0
    \end{equation}
  \item[Non-zero eigenvalues] correspond to \emph{step-operators} $E^{\alpha}$:
    \begin{equation}
      \label{eq:16-2}
      \operatorname{ad}_{H^{i}}(E^{\alpha}) = [H^{i}, E^{\alpha}] = \alpha^{i} E^{\alpha}
    \end{equation}
    By construction, $\alpha^{i} \in \mathbb{C}$ are not all zero.
\end{description}

For a general element $H \in \mathfrak{h}$, we can write $H = c_{i} H^{i}$ with $c_{i} \in \mathbb{C}$.
By \eqref{eq:16-2}, we have
\begin{equation}
  \label{eq:16-3}
  [H, E^{\alpha}] = \alpha(H) E^{\alpha},
\end{equation}
with $\alpha(H) = c_{i} \alpha^{i} \in \mathbb{C}$.

\begin{leftbar}
  When comparing the eigenvalue equation \eqref{eq:16-3} to the general expansion $[H, T^{a}] = \sum_b \xi(H)\indices{^{a}_{b}} T^{b}$, it follows that the eigenvalues of $\text{ad}_H$ are the roots of the characteristic equation $\det(\xi(H) - \alpha(H) \mathbb{1}) = 0$.
\end{leftbar}
\begin{definition}[roots]
  \label{def:roots}
  The non-zero eigenvalues $\alpha(H)$ of $\text{ad}_H$ are called the \emph{roots} of the Lie algebra. 
\end{definition}
\begin{leftbar}
  It is a matter of convention that only the non-zero eigenvalues are called roots.
\end{leftbar}

For any fixed element $H \in \mathfrak{h}$, the eigenvalue $\alpha(H)$ of $E^{\alpha}$ is some complex number which depends linearly on $H$.  Each root therefore gives a linear function $\alpha\colon \mathfrak{h} \to \mathbb{C}$, which means that $\alpha$ is an element of the dual space $\mathfrak{h}^*$.  

\begin{leftbar}
  As $\mathfrak{g}$ is spanned by the eigenvectors of $\text{ad}_H$, it can be written as a direct sum of vector spaces $\gamma_{\alpha}$ according to
  \begin{equation}
    \mathfrak{g} = \bigoplus_{\alpha} \mathfrak{g}_{\alpha}, \qquad \mathfrak{g}_{\alpha} = \left\{x \in \mathfrak{g} \suchthat [H, x] = \alpha(H) \cdot x \text{ for all } H \in \mathfrak{h} \right\}.
  \end{equation}
  Separating out the Cartan subalgebra $\mathfrak{h} = \mathfrak{g}_0$, we obtain the \emph{root space decomposition} of $\mathfrak{g}$ relative to $\mathfrak{h}$:
  \begin{equation}
    \mathfrak{g} = \mathfrak{h} \oplus \bigoplus_{\alpha \neq 0} \mathfrak{g}_{\alpha}.
  \end{equation}
  \begin{remark}
    In Yang--Mills theory, gauge bosons carry the adjoint representation of some simple Lie algebra. Their quantum numbers are zero if they correspond to generators of the Cartan subalgebra and are otherwise given by the non-zero roots, in which case the bosons are charged and self-interact.
    In electro-weak theory, the $W^{\pm}$ correspond to the roots of an $\mathfrak{sl}(2)$ algebra and hence are charged, whereas a linear combination of the photon and $Z$ boson, which are uncharged, correspond to the Cartan subalgebra. In Abelian gauge theories like electrodynamics there are no roots and gauge bosons do not self-interact.
  \end{remark}
\end{leftbar}

In general, we may have a problem of degenerate eigenvalues.  In a full proof of the Cartan construction, one can prove the following\footnote{A more complete story is given by Fuchs and Schweigert.}:
\begin{claim}
  \label{cl:non-deg}
  The roots are non-degenerate if $\mathfrak{g}$ is simple; each root $\alpha$ corresponds to the one-dimensional subspace spanned by the step-generator $E^{\alpha}$.
\end{claim}
\begin{corollary}
  This means that the set $\Phi$ of roots of the Lie algebra $\mathfrak{g}$ must consist of $(\dim \mathfrak{g} - \operatorname{rank} \mathfrak{g})$ distinct elements of $\mathfrak{h}^*$.
\end{corollary}
\begin{definition}[]
  The \emph{Cartan-Weyl basis} is $B = \left\{ H^{i} \mid i=1, \dots, r \right\} \cup \left\{ E^{\alpha} \mid \alpha \in \Phi \right\}$.
\end{definition}

\subsection{Inner Product on the Root Space}%
\label{sub:inner_product_on_the_root_space}

To analyse the root system, we will define an inner product on the space $\mathfrak{h}^*$ of roots via the Killing form on $\mathfrak{g}$.

By Cartan's theorem, Theorem \ref{thm:cartan}, the Killing form with normalisation $N$,
\begin{equation}
  \label{eq:16-4}
  \kappa(X, Y) = \frac{1}{N} \Tr[ad_X \circ ad_Y],
\end{equation}
is non-degenerate if $\mathfrak{g}$ is simple. We will always assume $\mathfrak{g}$ is simple from now on.
Let us evaluate $\kappa$ in the Cartan-Weyl basis.
\begin{claim}
  \label{cl:killingform}
  \begin{enumerate}
    \item $\forall H \in \mathfrak{h}$, $\forall \alpha \in \Phi$, we have $\kappa(H, E^{\alpha}) = 0$
    \item $\forall \alpha, \beta \in \Phi$, such that $\alpha + \beta \neq 0$, we have $\kappa(E^{\alpha}, E^{\beta}) =0$
    \item non-degeneracy of $\kappa$ then implies that $\forall H \in \mathfrak{h}$, $\exists H' \in \mathfrak{h}$ such that $\kappa(H, H') \neq 0$.
  \end{enumerate}
\end{claim}
\begin{proof}[Proof of 1]
  For any $H' \in \mathfrak{h}$, we use the relation \eqref{eq:16-3} to write
  \begin{equation}
    \alpha(H') \kappa(H, E^{\alpha}) = \kappa(H, [H', E^{\alpha} ])
  \end{equation}
  Using the invariance \eqref{eq:killing-invariance} of the Killing form, this becomes
  \begin{align}
    \alpha(H') \kappa(H, E^{\alpha}) &= \kappa ([H, H'], E^{\alpha})\\
     &= \kappa (0, E^{\alpha}) = 0
  \end{align}
  \begin{equation}
    \alpha(H') \neq 0 \quad \implies \quad \kappa(H, E^{\alpha}) = 0.
  \end{equation}
\end{proof}
\begin{proof}[Proof of 2]
  For all $H' \in \mathfrak{h}$, we have
  \begin{align}
    [\alpha(H') + \beta(H')] \kappa(E^{\alpha}, E^{\beta}) &\stackrel{\eqref{eq:16-3}}{=} \kappa([H', E^{\alpha}], E^{\beta}) + \kappa(E^{\alpha}, [H', E^{\beta}]) \\
							   &\stackrel{\eqref{eq:killing-invariance}}{=} 0
  \end{align}
  Nor for all $\alpha,\beta \in \Phi$ such that $\alpha + \beta \neq 0$, we have 
  \begin{equation}
    \alpha(H') + \beta(H') \neq 0 \text{ for some } H' \quad \implies \quad \kappa(E^{\alpha}, E^{\beta}) = 0.
  \end{equation}
\end{proof}
\begin{proof}[Proof of 3]
  Take any $H \in \mathfrak{h}$ and assume for contradiction that $\kappa(H, H') = 0$, $\forall H' \in \mathfrak{h}$. From 1, $\kappa(H, E^{\alpha}) = 0$ for all $\alpha \in \Phi$. This means that $\forall X \in \mathfrak{g}$, we have $\kappa(H, X) = 0$. In other words, $\kappa$ is degenerate, which is a contradiction.
\end{proof}
The third statement implies that $\kappa$ is a non-degenerate inner product on $\mathfrak{h}$
\begin{equation}
  \kappa(H, H') = \kappa^{ij} c_{i} c'_{j},
\end{equation}
where $H = c_{i} H^{i}$ and $H' = c_{i}' H^{i}$.
This means that $\kappa^{ij} = \kappa(H^{i}, H^{j})$ is an invertible $r \times r$ matrix. Explicitly, there exists an inverse matrix $\kappa^{-1}$ such that $(\kappa^{-1})_{ij} \kappa^{jk} = \delta^{k}_{i}$.
This inverse matrix $\kappa^{-1}$ defines a non-degenerate inner product on the dual space $\mathfrak{h}^*$.
This is where the roots live! Suppose we have roots $\alpha, \beta \in \Phi$ such that
\begin{equation}
  [H^{i}, E^{\alpha}] = \alpha^{i} E^{\alpha}, \qquad [H^{i}, E^{\beta}] = \beta^{i} E^{\beta}.
\end{equation}
We then define an inner product
\begin{equation}
  \label{eq:16-6}
  \boxed{(\alpha, \beta) \coloneqq (\kappa^{-1})_{ij} \alpha^{i} \beta^{j}}
\end{equation}
\begin{claim} \label{claim:16-inverse-root-is-root}
  If $\alpha \in \Phi$ is a root, then $-\alpha \in \Phi$ is too and $\kappa(E^{\alpha}, E^{-\alpha}) \neq 0$.
\end{claim}
\begin{proof}
  From Claim \ref{cl:killingform}.1, $\kappa(E^{\alpha}, H) = 0$ holds for all $H \in \mathfrak{h}$.
  From Claim \ref{cl:killingform}.2, $\kappa(E^{\alpha}, E^{\beta}) = 0$ holds for all roots $\beta \in \Phi$ with $\alpha \neq -\beta$.
  Hence, unless $-\alpha \in \Phi$ and $\kappa(E^{\alpha}, E^{-\alpha}) \neq 0$, we have $\kappa(E^{\alpha}, X) = 0$ for all elements $X \in \mathfrak{g}$, which would mean that $\kappa$ is degenerate, being a contradiction.
\end{proof}

\section{Algebra of the Cartan-Weyl Basis}%
\label{sec:algebra_in_cartan_weyl_basis}

So far, we have the following brackets:
\begin{subequations}
  \begin{align}
    [H^{i}, H^{j}] &= 0, \qquad \forall i , j = 1, \dots, r, \\
    [H^{i}, E^{\alpha}] &= \alpha^{i} E^{\alpha}, \qquad \alpha \in \Phi.
  \end{align}
\end{subequations}
It remains to evaluate $[E^{\alpha}, E^{\beta}]$ for the roots $\alpha, \beta \in \Phi$.
We can use the Jacobi identity
\begin{align}
[H^{i}, [E^{\alpha}, E^{\beta}]] &= [E^{\alpha}, [E^{\beta}, H^{i}]] - [E^{\beta}, [H^{i}, E^{\alpha}]] \\
				   &\stackrel{\eqref{eq:16-3}}{=} (\alpha^{i} + \beta^{i}) [E^{\alpha}, E^{\beta}]
\end{align}
