% lecture notes by Umut Özer
% course: symmetries
\lhead{Lecture 7: October 24}
\begin{example}[G = SO(n)]
  Let $g(t) = R(t) \in SO(n)$ for all $t \in \mathcal{I} \subset \mathbb{R}$, with $R(0) = I_n$ be a curve on the group manifold $G = SO(n)$.
  This implies that
  \begin{equation}
    R^T(t) R(t) = I_n \quad \forall t \in \mathcal{I}.
  \end{equation}
  Differentiating with respect to $t$, we have
  \begin{equation}
    \dot R^T (t) R(t) + R^T(t) \dot R(t) = 0, \qquad \forall t \in \mathcal{I}.
  \end{equation}
  If $x_i = \dot R(0)$, then at $t =0$ we get $x^T + x = 0$.
  Since continuity automatically implies $\det R(t) = +1$, there is no further constraint from imposing this.
  Therefore,
  \begin{equation}
    \mathscr{L}(O(n)) \simeq \mathscr{L}(SO(n)) = \{ X \in \text{Mat}_n(\mathbb{R}) \mid X^T = -X \}.
  \end{equation}
  Counting the parameters in these matrices, we then have 
  \begin{equation}
    \dim(\mathscr{L}(SO(n))) = \frac{1}{2} n (n - 1) = \dim(SO(n)).
  \end{equation}
\end{example}

\begin{example}[G = SU(n)]
  Let $g(t) = U(t) \in SU(n)$ be a curve passing through the origin, with $U(0) = I_n$.
  As in the previous example, we differentiate the unitarity property $U^{\dagger}(t) U(t) = I_n$ with respect to $t$ and evaluate at $t = 0$ to give
  \begin{equation}
    Z^{\dagger} + Z = 0
  \end{equation}
  for the derivatives $Z = \dot U(0) = \mathscr{L}(SU(n))$.
  Here, in contrast to the previous example, we \emph{will} get an additional constraint from $\det U(t) = 1$ $\forall t \in \mathbb{R}$.

  \begin{exercise}
    Show that $\det U(t) = 1 + t \Tr Z + O(t^2)$
  \end{exercise}
  Therefore, the constraint on the determinant imposes that the traces $\Tr Z$ vanish.
  The higher order terms in the Taylor series will not impose any additional constraints.
  Hence, we can use the same Lie algebra as $SO(n)$, except that we have to constrain the matrices be traceless:
  \begin{equation}
    \mathscr{L}(SU(n)) = \{Z \in \text{Mat}_n(\mathbb{C}) \mid Z^{\dagger} = -Z, \; \Tr Z = 0\}
  \end{equation}
  From this, we find that $\dim(\mathscr{L} (SU(n))) = 2n^2 - n^2 - 1 = n^2 - 1$; the dimension of the Lie algebra is the same as the dimension of the Lie group $\dim(SU(n))$.
\end{example}

\begin{example}[]
  \begin{equation}
    \dim SU(2) = \dim SO(3) = 3
  \end{equation}
  Let $G = SU(2)$, then from our considerations above we know that the Lie algebra $\mathscr{L}(SU(2))$ is the set of $2\times 2$ traceless anti-hermitian matrices.
  The basis for the Lie algebra of $SU(2)$ can be built from the hermitian Pauli matrices: for $a = 1, 2, 3$, they satisfy
  \begin{equation}
    \sigma_a = \alpha_a^{\dagger} \qquad \text{and} \qquad \Tr \sigma_a = 0.
  \end{equation}
  A basis element in the Lie algebra of $SU(2)$ can then be written
  \begin{equation}
    T^a = -\frac{1}{2} i \alpha_a,
  \end{equation}
  where the $i$ guarantees that $T^a$ is anti-hermitian.
  At the moment, the upper or lower placement of indices is purely notational at the moment, but will gain significance later.
  In order to compute the structure constants, we recall that the Pauli matrices obey the following identity:
  \begin{equation}
    \sigma_a \sigma_b = \delta_{ab} I_2 + i \varepsilon_{abc} \sigma_c.
  \end{equation}
  Using this, the structure constants $f \indices{^{ab}_c}$ are, according to their definition, obtained from the commutator
  \begin{equation}
    [T^a, T^b] = -\frac{1}{4} [\sigma_a, \sigma_b] = -\frac{1}{2} i \varepsilon_{abc} \sigma_c = f \indices{^{ab}_c} T^c .
  \end{equation}
  Therefore, we simply read off
  \begin{equation}
    \boxed{f \indices{^{ab}_c} = \varepsilon_{abc}} \qquad a,b,c = 1,2,3.
  \end{equation}
\end{example}

\begin{example}[G = SO(3)]
  From our previous analysis, we know that the Lie algebra of $SO(3)$ consists of
  \begin{equation}
    \mathscr{L}(SO(3)) = \{ 3 \times 3 \text{ real anti-symmetric matrices} \}.
  \end{equation}
  We can write down a convenient basis as
  \begin{equation}
    \widetilde T^1 = 
    \begin{pmatrix}
     0 & 0 & 0 \\
     0 & 0 & -1 \\
     0 & 1 & 0 \\
    \end{pmatrix},
    \qquad
    \widetilde T^2 = 
    \begin{pmatrix}
     0 & 0 & 1 \\
     0 & 0 & 0 \\
     -1 & 0 & 0 \\
    \end{pmatrix},
    \qquad
    \widetilde T^3 = 
    \begin{pmatrix}
     0 & -1 & 0 \\
     1 & 0 & 0 \\
     0 & 0 & 0 \\
    \end{pmatrix}
  \end{equation}
  This can be used to find the structure constants as
  \begin{equation}
    [\widetilde T^a, \widetilde T^b] = f \indices{^{ab}_c} \widetilde T^c
  \end{equation}
  with $f \indices{^{ab}_c} = \varepsilon_{abc}$ with $a,b,c = 1,2,3$.
  We find that the Lie algebras are the same
  \begin{equation}
    \mathscr{L}(SO(3)) \cong \mathscr{L}(SU(2)).
  \end{equation}
  However, the Lie groups themselves are not isomorphic: 
  \begin{equation}
    SO(3) \not\cong SU(2).
  \end{equation}
  We learn that different Lie groups can lead to the same Lie algebra. However, we will see later that this degeneracy will be easy to deal with.
  In fact, there is a relation between these two Lie groups:
  \begin{equation}
    SO(3) \simeq \frac{SU(2)}{\mathbb{Z}_2}.
  \end{equation}
  \begin{leftbar}
    \begin{remark}
      This is because there is a \emph{double cover}, or two-to-one, surjective homomorphism
      \begin{equation}
	\phi\colon \ SU(2) \to SO(3), \qquad \text{ker}(\phi) = \mathbb{Z}_2 = \{\pm 1\}.
      \end{equation}
    \end{remark}
  \end{leftbar}
\end{example}

\section{Diffeomorphisms}%
\label{sec:diffeomorphisms}

A Lie group is a very special type of manifold. In particular, for each element $h$ in the Lie group $G$, we have two smooth maps from the group to itself that come from left and right group multiplication,
\begin{align}
  L_h &\colon G \to G \qquad g \in G \mapsto hg \in G, \\
  R_h &\colon G \to G \qquad g \in G \mapsto gh \in G,
\end{align}
known as \emph{left-} and \emph{right-translations} respectively.

\begin{claim}
  These maps are \emph{surjective}, which means that for every $g' \in G$, there is an element $g \in G$, such that $L_h(g) = g'$. 
\end{claim}
\begin{proof}
  To see this, set $g = h^{-1} g'$.
\end{proof}

\begin{claim}
  These maps are also \emph{injective}, which means that for all $g, g' \in G$, we have that if $L_h(g) = L_h(g)$, then we must have $g = g'$.
\end{claim}
\begin{proof}
  \(
    L_h(g) = L_h(g') \implies gh = gh' \implies g = g'.
    \)
\end{proof}
\begin{definition}[]
  Since $L_h$ and $R_h$ are both surjective and injective---they are said to be \emph{one-to-one}, or \emph{diffeomorphisms} of $\mathcal{M}(G)$---the \emph{inverse map}$(L_h)^{-1} = L_{h^{-1}}$ exists and is smooth.
\end{definition}

\begin{wrapfigure}{L}{0.37\columnwidth}
  \centering
  \def\svgwidth{0.3\columnwidth}
  \input{lectures/l8f1.pdf_tex}
  \label{fig:l8f1}
\end{wrapfigure}

Now introduce coordinates $\{ \theta^i \}$, $i = 1, \dots, D$ in some region containing the identity element. We can then parametrise every $g = g(\theta) \in G$, and we can choose the parametrisation such that $g(0) = e$.
Assuming that we map to an element $g'$ that is in the same coordinate patch as $g$, let $g' = g(\theta') = L_h(g) = h g(\theta)$.
In coordinates, $L_h$ is specified by $D$ real functions $\theta'^i = \theta'^i(\theta)$, $i = 1, \dots, D$.
Since $L_h$ is a diffeomorphism, the Jacobian matrix
\begin{equation}
  J \indices{^i_j}(\theta) = \pdv{\theta'^i}{\theta^j}
\end{equation}
exists and is invertible, which is equivalent to saying that $\det J \neq 0$.
