\lhead{Lecture 1: October 10}
\chapter{Introduction}%
\label{cha:introduction}

In this course, we will cover Lie groups $G$, Lie algebras $\mathscr{L}(G)$, and their representations.

\section*{Resources}%
\label{sec:resources}

\begin{itemize}
  \item Notes (online): Manton, Osborn, Gutowski (Cartan classification)
  \item Book: Fuchs \& Schweigert (Ch 1-7) ``Symmetries, Lie Algebras and Representations''
\end{itemize}

\section{Symmetry and Groups}%
\label{sec:groups_and_lie_groups}

\begin{definition}
A \emph{symmetry} is a transformation of dynamical variables that leaves the form of physical laws invariant.
\end{definition}

\begin{example}[Rotation]
  Consider a vector $\vb{x} \in \mathbb{R}^3$. A rotation of this vector is a transformation
  \begin{equation}
    \vb{x} \to \vb x' = M \vdot \vb x,
  \end{equation}
  where $M$ is a $3\times 3$ matrix, which is \emph{orthogonal} ($M M^T = 1_{3}$) and \emph{special} ($\det M = 1$).  Applying a rotation to Newton's laws gives:
  \begin{align}
    \vb F &= m \dv[2]{\vb x}{t} \\
    \rightarrow \vb F' &= M \vdot \vb F' = m \dv[2]{\vb x'}{t},
  \end{align}
  so rotations of the coordinates are symmetries of Newtonian physics.
\end{example}

If two transformations individually leave a set of physical laws invariant, their combination also will. This property of symmetry transformations is captured in saying that they have a \emph{group} structure:

\begin{definition}[Group]
 A group is a (finite or infinite) set $G$ with the following properties:
\begin{enumerate}
  \item Closure: $g_1, g_2 \in G \implies g_1 g_2 \in G$
  \item Unit: $\exists e \in G: eg = ge = g \forall g \in G$
  \item Inverse: $\forall g \in G$, $\exists g^{-1} \in G : g^{-1}g = g^{-1} = e$
  \item Associativity: $\forall g_1, g_2, g_3 \in G: (g_1 g_2) g_3 = g_1 (g_2 g_3)$
\end{enumerate}
\end{definition} 

In physics, we will in general deal with unconstrained, infinite groups. In general, say for rotations in three dimensions, $G = SO(3)$, the order in which we apply the symmetry transformations makes a difference, but there are some groups for which it does not.

\begin{definition}[Abelian]
A commutative group $G$, where $g_1 g_2 = g_2 g_1$, is called \emph{Abelian}.
\end{definition}

\begin{exercise}
  Show that the set $SO(3)$ of all $3\times 3$ special orthogonal matrices, representing rotations, form a group under matrix multiplication.
\end{exercise}

\begin{remark}
  A rotation in $\mathbb{R}^3$ depends continuously on 3 parameters $\hat n \in S^2$ and $\theta \in [0, \pi]$.
\end{remark}

\section{Lie Groups and Lie Algebras}%
\label{sec:lie_groups_and_lie_algebras}

% L1 F1, F2
\begin{figure}[htpb]
  \centering
  \def\svgwidth{0.6\columnwidth}
  \input{lectures/l1f1.pdf_tex}
  \caption{Locally, the torus $\mathbb{T}^2$ looks like $\mathbb{R}^{2}$}
  \label{fig:l1f1}
\end{figure}

A \emph{Lie group} $G$ is a group which is also a \emph{smooth manifold}.  Due to the algebraic properties of the group each $g \in G$ defines a map from the group to itself
\begin{equation}
  \begin{split}
    L_g \colon G &\to G \\
    h &\mapsto gh
  \end{split}
\end{equation}
Roughly speaking, this definition will require that group and manifold structures must be compatible (e.g.~$L_{g}$ must be smooth).  As we will see in Chapter~\ref{cha:lie_algebras_from_lie_groups}, this map will allow us to move around the manifold. As a result, $G$ is almost completely determined by its behaviour ``near'' the identity $e$.
In other words, the Lie group is almost completely defined by infinitesimal symmetry transformations.

Rather than the whole of the manifold, we will only think about the tangent space $T_e (G)$ of the identity.

Tangent vectors at $e$, $\vb v_1, \vb v_2 \in T_e (G)$, equipped with a bracket $[, ]: T_e(G) \times T_e(G) \rightarrow T_e(G)$, define a \emph{Lie algebra} $\mathscr{L}(G)$.
Lie groups are (almost) determined by their Lie algebra.

\section{Key Result: Cartan Classification (1895)}%
\label{sec:key_result_cartan_classification}

The Cartan Classification arises from the question of whether we can have a finite dimensional Lie algebra. The problem reduces to analysing \emph{simple} Lie algebras, which all Lie algebras can be built from.

\begin{theorem}[Cartan Classification]
  All finite-dimensional semi-simple Lie algebras (over $\mathbb{C}$) belong to one of
  \begin{itemize}
    \item four infinite families $A_n, B_n, C_{n}, D_n$, where $n \in \mathbb{N}$,
    \item five exceptional cases $E_6, E_7, E_8, G_2, F_4$.
  \end{itemize}
\end{theorem}

This basically lists the allowed gauge theories, since the gauge groups must correspond to the Lie algebras in this list. All physical groups come from the low-$n$ values of the infinite families. In more modern theoretical physics, exceptional Lie groups also show up. For example, in String theory certain anomalies only cancel with combinations like $E_8 \oplus E_8$ or $\mathfrak{so}(32) = D_8$.

\section{Symmetries in Quantum Physics}%
\label{sec:quantum_phyiscs}

By Noether's theorem, symmetries imply the existence of conserved quantities.

\begin{example}
  The invariance rotational symmetry in $\mathbb{R}^3$ implies the conservation of angular momentum $\vb L = (L_1, L_2, L_3)$.
\end{example}

Instead of the classical phase space, quantum mechanical
\begin{itemize}
  \item states are vectors $\ket{\psi}$ in a (potentially infinite dimensional) Hilbert space $\mathcal{H}$,
  \item observables are linear operators $\hat O \colon \mathcal{H} \rightarrow \mathcal{H}$. These do not commute $\hat O_1 \hat O_2 \neq \hat O_2 \hat O_1$.
\end{itemize}

\subsection{Angular Momentum}%
\label{sub:angular_momentum}

\begin{wrapfigure}{R}{0.31\columnwidth}
  \centering
  \def\svgwidth{0.25\columnwidth}
  \input{lectures/so3.pdf_tex}
  \caption{The angular momentum operators form a basis of the tangent space of $SO(3)$.}
  \label{fig:so3}
\end{wrapfigure}

The angular momentum operators $\hat L_1, \hat L_2 \hat L_3$ have the structure of the $\mathscr{L}(SO(3))$ lie algebra, where the commutator plays the role of the lie bracket:
\begin{equation}
  \label{eq:ang-mom-com}
  [\hat L_i, \hat L_j] = i \hbar \varepsilon_{ijk} \hat L_k.
\end{equation}
These operators form a basis of the tangent space $T_e (SO(3))$.

