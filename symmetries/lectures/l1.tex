\lhead{Lecture 1: October 10}
\chapter{Introduction}%
\label{cha:introduction}

In this course, we will cover Lie groups $G$, Lie algebras $\mathscr{L}(G)$, and their Representations.

\section*{Resources}%
\label{sec:resources}

\begin{itemize}
  \item Notes (online): Martan, Osbord, Gutowski (Cartan classification)
  \item Book: Fuchs \& Schwegert (Ch 1-7) ``Symmetries \dots''
\end{itemize}

\section{Introduction}%
\label{sec:introduction}

\begin{definition}
A \emph{symmetry} is a transformation of dynamical variables leaves the form of physical laws invariant.
\end{definition}

\begin{example}[Rotation]
  \begin{equation}
    \vb x \in \mathbb{R}^{3} \longrightarrow \vb x' = M \vdot \vb x \in \mathbb{R}^{3},
  \end{equation}
  where $M$ is a $3\times 3$ matrix, which is orthogonal ($M M^T = 1_{3}$) and ``special'' ($\det M = 1$).

  We can do this with Newton's laws:
  \begin{equation}
    \vb F = m \dv[2]{\vb x}{t} \rightarrow \vb F' = M \vdot \vb F' = m \dv[2]{\vb x'}{t}.
  \end{equation}
\end{example}

\section{Groups and Lie Groups}%
\label{sec:groups_and_lie_groups}

\begin{definition}[Group]
 A group is a (finite or infinite) set $G$ with the following properties:
\begin{enumerate}
  \item Closure: $g_1, g_2 \in G \implies g_1 g_2 \in G$
  \item Unit: $\exists e \in G: eg = ge = g \forall g \in G$
  \item Inverse: $\forall g \in G$, $\exists g^{-1} \in G : g^{-1}g = g^{-1} = e$
  \item Associativity: $\forall g_1, g_2, g_3 \in G: (g_1 g_2) g_3 = g_1 (g_2 g_3)$
\end{enumerate}
\end{definition} 

In physics, we will in general deal with unconstrained, infinite groups.

\begin{definition}[Abelian Groups]
A commutative group $G$, where $g_1 g_2 = g_2 g_1$, is called \emph{Abelian}.
\end{definition}

The set of all $3\times 3$ special orthogonal matrices, representing rotations, form a group under matrix multiplication.

\begin{exercise}
Check this!
\end{exercise}

A rotation in $\mathbb{R}^3$ depends continuously on 3 parameters $\hat n \in S^2, \theta \in [0, \pi]$.

\begin{definition}[Lie Group]
A Lie group $G$ is a group which is also a \emph{smooth manifold}.
\end{definition}

% L1 F1, F2
\begin{figure}[htpb]
  \centering
  \def\svgwidth{0.6\columnwidth}
  \input{lectures/l1f1.pdf_tex}
  \caption{Locally, the torus $\mathbb{T}^2$ looks like $\mathbb{R}^{2}$}
  \label{fig:l1f1}
\end{figure}

Due to the algebraic properties of the group each $g \in G$ defines a map from the group to itself
\begin{equation}
  \begin{split}
    L_g \colon G &\to G \\
    h &\mapsto gh
  \end{split}
\end{equation}
Roughly speaking, this definition will require that group and manifold structures must be compatible (e.g.~$L_{g}$ must be smooth).
As we will see in Chapter~\ref{cha:lie_algebras_from_lie_groups}, this map will allow us to move around the manifold. As a result, $G$ is almost completely determined by its behaviour ``near'' the identity $e$.
In other words, the Lie group is almost completely defined by infinitesimal symmetry transformations.

Rather than the whole of the manifold, we will only think about the tangent space $T_e (G)$ of the identity.

Tangent vectors at $e$, $\vb v_1, \vb v_2 \in T_e (G)$, equipped with a bracket $[, ]: T_e(G) \times T_e(G) \rightarrow T_e(G)$, define a \emph{Lie algebra} $\mathscr{L}(G)$.
Lie groups are (almost) determined by their Lie algebra.

\section{Key Result: Cartan Classification (1895)}%
\label{sec:key_result_cartan_classification}

This arises from the question of whether we can have a finite dimensional Lie algebra. The problem reduces to analysing \emph{simple} Lie algebras, which all Lie algebras can be built from.

\begin{theorem}[Cartan Classification]
  All finite-dimensional semi-simple Lie algebras (over $\mathbb{C}$) belong to one of
  \begin{enumerate}
    \item four infinite families $A_n, B_n, C_{n}, D_n$, where $n \in \mathbb{N}$
    \item five exceptional cases $E_6, E_7, E_8, G_2, F_4$
  \end{enumerate}
\end{theorem}

This basically lists the allowed gauge theories, since the gauge groups must correspond to the Lie algebras in this list. All physical groups come from the low-$n$ values of the infinite families. In more modern theoretical physics, exceptional Lie groups also show up. For example, in String theory, certain anomalies only cancelled with combinations like $E_8 \oplus E_8$ or $\mathfrak{so}(32) = D_8$.

\section{Classical Physics}%
\label{sec:classical_physics}

By Noether's theorem, symmetries $\implies$ conserved quantities.

\begin{example}
  invariance under rotations in $\mathbb{R}^3 \implies$ conserved angular momentum $\vb L = (L_1, L_2, L_3)$
\end{example}

\section{Quantum Physics}%
\label{sec:quantum_phyiscs}

Instead of phase space, 
\begin{itemize}
  \item states are now vectors $\ket{\psi}$ in a (potentially infinite dimensional) Hilbert space $\mathcal{H}$
  \item observables are linear operators $\hat O: \mathcal{H} \rightarrow \mathcal{H}$. These do not commute $\hat O_1 \hat O_2 \neq \hat O_2 \hat O_1$
\end{itemize}

The angular momentum operators $\hat L_1, \hat L_2 \hat L_3$ obey the $\mathscr{L}(SO(3))$ lie algebra, lie bracket commutator:
\begin{equation}
  [\hat L_i, \hat L_j] = i \hbar \varepsilon_{ijk} \hat L_k.
\end{equation}

% F3
\begin{figure}[htpb]
  \centering
  \def\svgwidth{0.3\columnwidth}
  \input{lectures/so3.pdf_tex}
  \caption{The angular momentum operators form a basis of the tangent space of $SO(3)$.}
  \label{fig:so3}
\end{figure}

These operators form a basis of the tangent space $T_e (SO(3))$ where the commutator is the lie bracket.

