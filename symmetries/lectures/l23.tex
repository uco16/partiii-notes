% lecture notes by Umut Özer
% course: symmetries
\lhead{Lecture 23: December 03}

\section{Irreducible Representations of \texorpdfstring{$A_2 \simeq \mathfrak{su}_{\mathbb{C}}(3)$}{the Complexified Lie Algebra of SU(3)}}%
\label{sec:irreducible representations_of_lie_algebra_of_su_3_}

Recall that the highest weight of an irreducible $\mathfrak{sl}(2)$ representation determines the dimension of that representation. The same is true for representations of $A_2$.
\begin{claim}
  The representations $R_{(\Lambda_1, \Lambda_2)}$ of $A_2$ are labelled by the highest weight $\Lambda = \Lambda_1 \omega_{(1)} + \Lambda_2 \omega_{(2)}$, where $(\Lambda_1, \Lambda_2) \in \mathbb{N}_0^2$ are the Dynkin labels of the representation. Their dimensions are
  \begin{equation}
    \dim(R_{(\Lambda_1, \Lambda_2)}) = \frac{1}{2} (\Lambda_1 + 1) (\Lambda_2 + 1) (\Lambda_1 + \Lambda_2 + 2).
  \end{equation}
\end{claim}
\begin{corollary}
  This formula is symmetric under interchange of $\Lambda_1 \leftrightarrow \Lambda_2$.  This means that whenever $\Lambda_1 \neq \Lambda_2$, then we get pairs of representations
  \begin{equation}
    R_{(\Lambda_2, \Lambda_1)} = \overline{R}_{(\Lambda_1, \Lambda_2)}.
  \end{equation}
\end{corollary}
Using the formula to compute the $A_2$ irreducible representations of lowest dimension, we obtain Table~\ref{tab:23-1}. Commonly, the irreducible representations are referred to by their dimension, written in a bold font. For instance, we denote the fundamental and anti-fundamental representations of $A_2$ as $R_{(1, 0)} = \boldsymbol 3$ and $R_{(0,1)} = \overline{\boldsymbol 3}{}$ respectively.

\begin{table}[tbhp]
  \centering
  \begin{tabular}{ c | c  c }
    $A_2$ irrep & dim & name \\
    \hline
    $R_{(0,0)}$ & $\boldsymbol1$ & trivial \\
    $R_{(1, 0)}$ & $\boldsymbol 3$ & fundamental \\
    $R_{(0, 1)} = \overline{R}_{(1, 0)}$ & $\overline{\boldsymbol 3}{}$ & anti-fundamental \\
    $R_{(2, 0)}$ & $\boldsymbol 6$ &  \\
    $R_{(0, 2)} = \overline{R}_{(2, 0)}$ & $\overline{\boldsymbol 6}{}$ &  \\
    $R_{(1, 1)}$ & $\boldsymbol 8$ & adjoint \\
    $R_{(3, 0)}$ & $\boldsymbol{10}$ & \\
    $R_{(0, 3)} = \overline{R}_{(3, 0)}$ & $\overline{\boldsymbol{10}}{}$ &  \\
  \end{tabular}
  \caption{Irreducible representations (irreps) of $A_2$ with lowest dimensions.}
  \label{tab:23-1}
\end{table} 

Since $\dim(A_2) = \dim(\mathfrak{su}_{\mathbb{C}}(3)) = 8$, the eight dimensional representation $R_{(1,1)} = \boldsymbol 8$ is the natural candidate to be the adjoint representation.
\begin{figure}[tbhp]
  \centering
  \def\svgwidth{0.4\columnwidth}
  \input{lectures/l23f1.pdf_tex}
  \caption{The weight lattice for the $R_{(1, 1)}$ representation of $A_2$. Applying the step-down operators to the highest weight $\omega_1 + \omega_2 = \alpha + \beta$ gives the $\abs{S_{(1, 1)}} = 8$ weights.}
  \label{fig:l23f1}
\end{figure}
Starting from the highest weight vector $\omega_1 + \omega_2$ and successively applying the step down operator algorithm, we recover the other roots.  This is shown in Figure \ref{fig:l23f1}. The weights of the adjoint representation are the roots. From the diagram, we see that the weight set, expressed in terms of simple roots $\alpha$ and $\beta$, is 
\begin{equation}
  \label{eq:23-weights}
  S_{(1, 1)} = \left\{\alpha + \beta, \alpha, \beta, 0, -\alpha, -\beta, -(\alpha + \beta) \right\}.
\end{equation}
The representation space is just the Lie algebra itself, generated by the step operators and two Cartan elements
\begin{equation}
  A_2 = \operatorname{span}\left\{ E^{\alpha}, E^{-\alpha}, E^{\beta}, E^{\beta}, E^{\alpha+ \beta}, E^{-(\alpha + \beta)}, H^1, H^2 \right\}.
\end{equation}
Since $\dim R_{(1, 1)} = 8$, and the 6 roots are non-degenerate by Claim \ref{cl:non-deg}, the weight $(0, 0)$ must have multiplicity two, corresponding to the two Cartan elements $H^1$ and $H^2$. The weights \eqref{eq:23-weights} correspond to the generators of the Lie algebra, so $R_{(1, 1)}$ is indeed the adjoint representation. 

\section{Tensor Products Revisited}%
\label{sec:tensor_products_revisited}

The representation of a composite system, say of two particles, is given by the tensor product of the individual particle representations.

Let $R_{\Lambda}$ and $R_{\Lambda'}$  be irreducible representations of the simple complex Lie algebra $\mathfrak{g}$. The corresponding representation spaces are $V_\Lambda$ and  $V_{\Lambda'}$, where $\Lambda$ and $\Lambda'$ denote the highest weight of the weight sets $S_{\Lambda}, S_{\Lambda'} \subset \mathfrak{h}_{\mathbb{R}}^*$.
The representation space is a direct sum of subspaces corresponding to each weight
 \begin{equation}
  V_{\Lambda} = \bigoplus_{\lambda \in S_\Lambda} V_{\lambda}.
\end{equation}
In Sec.~\ref{sub:tensor_product} we defined the tensor product of two arbitrary representations.

\begin{claim}
  If $\lambda \in S_\Lambda$ and $\lambda' \in S_{\Lambda'}$, then $\lambda + \lambda' \in \mathcal{L}_\omega [\mathfrak{g}]$ is a weight of the tensor product representation $R_{\Lambda} \oplus R_{\Lambda'}$.
\end{claim}
\begin{proof}
  Let $v_{\lambda} \in V_{\lambda}$ be a weight vector. Recall that this satisfies the eigenvalue equation
  \begin{equation}
    R_{\Lambda}(H^{i}) v_{\lambda} = \lambda^{i} v_{\lambda}.
  \end{equation}
  And similarly for $v_{\lambda'} \in V_{\lambda'}$.
  Basis vectors for $V_{R_\Lambda \oplus R_{\Lambda'}} = V_{\Lambda} \oplus V_{\Lambda'}$ are tensor products $v_{\lambda} \oplus v_{\lambda'}$ of pairs of weight vectors from the original representation.
  Then by Definition \ref{def:tensor-product-rep} of the tensor product representation, we have
  \begin{align}
    (R_{\Lambda} \otimes R_{\Lambda'})(H^{i}) (v_{\lambda} \otimes v_{\lambda'}) &= R_{\Lambda}(H^{i}) v_{\lambda} \otimes v_{\lambda'} + v_\lambda \otimes R_{\Lambda'} \otimes R_{\Lambda'} (H^{i}) v_{\lambda}' \\
										 &= (\lambda + \lambda') (v_{\lambda} \otimes v_{\lambda'})
  \end{align}
\end{proof}

For finite, simple, complex $\mathfrak{g}$, the tensor product representation is reducible:
\begin{equation}
  R_{\Lambda} \otimes R_{\Lambda'} = \bigoplus_{\Lambda'' \in \mathcal{L}_\omega} \mathcal{N}_{\Lambda, \Lambda'}^{\Lambda''} R_{\Lambda''}.
\end{equation}

\begin{example}[$\mathfrak{g} = A_2$]
  Have $R_{(1, 0)} \otimes R_{(1, 0)}$. The weight set is $S_{(1, 0)} = \{\omega_1, \omega_2 - \omega_1, -\omega_2\}$. Keeping track of the degeneracy of weights, here in particular $\omega_2$ and $-\omega_1$, we first find the sum of the weight sets and then decompose it:
  \begin{align}
    S_{(1, 0) \otimes (1, 0)} &= \left\{ 2 \omega_1, \omega_2, \omega_2, \omega_1 - \omega_2, \omega_1 - \omega_2, -\omega_1, -\omega_1, -2\omega_1 - 2\omega_2, -2\omega_2 \right\} \\
    &= \left\{ 2\omega_1, \omega_2, \omega_1 - \omega_2, -\omega_1, -2\omega_1 -2\omega_2, -2\omega_2 \right\} \cup \left\{ \omega_2, \omega_1 - \omega_2, -\omega_1 \right\} \\
    &= S_{(2, 0)} \cup S_{(0, 1)}
  \end{align}
  This decomposition of the weight set implies that $R_{(1, 0)} \otimes R_{(1, 0)} = R_{(2, 0)} \oplus R_{(0, 1)}$.  In the dimensional notation of Table \ref{tab:23-1}, this translates into $\boldsymbol 3 \otimes \boldsymbol 3 = \boldsymbol 6 \oplus \overline{\boldsymbol 3}{}$, which makes it obvious that the dimensions work out.
\end{example}

\subsection{Strong Interactions, Quarks and the Eightfold Way}%
\label{sub:strong_interactions_and_the_eightfold_way}

\begin{leftbar}
  Not lectured. Taken from Prof.~Manton's notes.
\end{leftbar}
This is where our discussion comes back to the observation of the \emph{eightfold way} from Sec.~\ref{sub:degeneracy_of_the_spectrum}.
Each of the weights corresponds to a particle, which transform according to the $SU(3)_{\text{flavour}}$ approximate symmetry. Since the rank of $A_2$ is $2$, these particles are simultaneous eigenstates of two quantum numbers, which we choose to be \emph{isospin} $I = H^1 / 2$ and \emph{hypercharge} $Y = (H^1 + 2 H^2) / 3$. Electric charge is related to these by $Q = I + Y / 2$. Particles in irreducible representations have approximately the same mass, and related strong interactions.
The particles transforming under the fundamental representation $\boldsymbol 3$ of flavour $SU(3)$ are the quarks $q$, while antiquarks $\overline{q}{}$ are in $\overline{\boldsymbol 3}{}$. This is shown in Fig.~\ref{fig:quarks}.

\begin{figure}[btph]
  \centering
  \def\svgwidth{0.9\columnwidth}
  \input{lectures/quarks.pdf_tex}
  \caption{The quarks $q$ transform under $\boldsymbol 3$ and anti-quarks $\overline{q}{}$ under $\overline{\boldsymbol 3}{}$ of flavour $SU(3)$.}
  \label{fig:quarks}
\end{figure}

All other hadrons, such as the mesons $q \overline{q}{}$ and the baryons $qqq$, are multi-quark states and can be obtained by tensor products.  In particular, the particles in the \emph{eightfold way} of Fig.~\ref{fig:eightfold} are the baryon octet $\boldsymbol 8$, which arises in the decomposition of $qqq \simeq \boldsymbol 3 \otimes \boldsymbol 3 \otimes \boldsymbol 3 = \boldsymbol{10} \oplus \boldsymbol 8 \oplus \boldsymbol 8 \oplus \boldsymbol 1$, alongside the baryon decuplet $\boldsymbol{10}$.
The $SO(3)$ gauge invariance of physical states in QCD explains why only colour singlets $\boldsymbol 3 \otimes \overline{\boldsymbol 3}{}$ ($q \overline{q}{}$), $\boldsymbol 3 \otimes \boldsymbol 3 \otimes \boldsymbol 3$ ($q q q$), and $\overline{\boldsymbol 3}{} \otimes \overline{\boldsymbol 3}{} \otimes \overline{\boldsymbol 3}{}$ ($\overline{q}{} \overline{q}{} \overline{q}{}$) are observed, but does not tell us why colour singlets of $qq \overline{q}{} \overline{q}{}$ or $qq qq qq$ states are not observed.

\chapter{Gauge Theory}%
\label{cha:gauge_theory}

We will assume that most of the basics of Abelian gauge theory have been covered in other courses. We will develop these quickly and then move on to non-Abelian gauge theories.

Gauge theories have other kinds of symmetries than those that we have been dealing with; they are redundancies in the way we describe our system.

\section{Electromagnetism}%
\label{sec:electromagnetism}

The classic example is the gauge invariance of electromagnetism. Consider a field
\begin{equation}
  \mathcal{A}_{\mu} = 
  \begin{cases}
    \Phi, & \mu = 0 \\
    A_{i}, & \mu = i = 1, 2, 3
  \end{cases}
\end{equation}
Gauge transformations are $\mathcal{A}_{\mu} \to \mathcal{A}_{\mu} + \partial_{\mu} \alpha$, where $\alpha = \alpha(\vb{x}, t)$ and leave the fields invariant.
In particular, they leave the field strength tensor $\mathcal{F}_{\mu\nu} = \partial_{\mu} \mathcal{A}_{\nu} - \partial_{\nu} \mathcal{A}_{\mu}$ invariant.

The Lagrangian is
\begin{equation}
  \mathcal{L}_{\text{EM}} = -\frac{1}{4 g^2} \mathcal{F}^{\mu\nu} \mathcal{F}_{\mu\nu}
\end{equation}

We will work with a redefined gauge potential, which is imaginary, and field strength tensor
\begin{equation}
  A_{\mu} = -i \mathcal{A}_{\mu} \qquad F_{\mu\nu} = -i \mathcal{F}_{\mu\nu}.
\end{equation}
We will use this to reconcile it with the fact that the gauge field naturally lives in the Lie algebra of the symmetry group, which we represented by complex numbers $e^{i \theta} \in \mathbb{C}$.

\subsection{Coupling to Matter}%
\label{sub:coupling_to_matter}

Define a complex scalar field $\phi \colon \mathbb{R}^3 \to \mathbb{C}$ with the standard Lagrangian
\begin{equation}
  \mathcal{L}_\phi = \partial_{\mu} \phi^* \partial^{\mu} \phi - W(\phi \phi^*).
\end{equation}
Choosing the potential $W$ to only depend on the modulus makes sure that this Lagrangian is invariant under a global $U(1)$ symmetry:
\begin{equation}
  \begin{gathered}
    \phi \to g\phi \\
    \phi^* \to g^{-1} \phi^*
  \end{gathered}
  \qquad g = \exp(i \delta) \in U(1), \quad \delta \in [0, 2\pi].
\end{equation}
Consider now the infinitesimal symmetry generator
\begin{equation}
  g = \exp(\epsilon X) \simeq 1 + \epsilon X + O(\epsilon^2).
\end{equation}
Working close to the identity element means taking $\epsilon \ll 1$. Moreover, $X \in \mathfrak{u}(1) \simeq i \mathbb{R}$.
We write the infinitesimal symmetry transformation as 
\begin{equation}
  \begin{gathered}
    \phi \to \phi + \delta_X \phi \\
    \phi^* \to \phi^* + \delta_X \phi^*
  \end{gathered}
  \qquad
  \begin{gathered}
    \delta_X \phi = \epsilon X \phi \\
    \delta_X \phi^* = -\epsilon X \phi^*.
  \end{gathered}
\end{equation}
We say that this transformation is a \emph{symmetry} if the variation $\delta_X \mathcal{L}$ of the Lagrangian vanishes. 
