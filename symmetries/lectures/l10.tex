% lecture notes by Umut Özer
% course: symmetries
\lhead{Lecture 10: November 02}
\begin{proof}
  For any $X_1, X_2 \in \mathscr{L}(G)$, we construct two curves $C_i \colon t \mapsto g_i(t) \in G$, $i = 1, 2$, with starting point $g_1(0) = g_2(0) = \mathbb{1}_m$ and $\dot g_1(0) = X_1$, $\dot g_2(0) = X_2$.
  As on page 41, define
  \begin{align}
    h(t) &= g_1^{-1}(t)g_2^{-1}(t)g_1(t)g_2(t) \in G \\
	 &= \mathbb{1}_m + t^2 [X_1, X_2] + O(t^3).
  \end{align}
  As $D$ is a representation of $G$, we have for all $t$ that
  \begin{align}
    D(h) &= D(g_1^{-1} g_2^{-1} g_1 g_2) = D(g_1)^{-1} D(g_2)^{-1} D(g_1) D(g_2) \\
    D(g_1(t)) &= D(\mathbb{1}_m + t X_1 + \dots) \\
	      &= D(\mathbb{1}_m) + td(X_1) + O(t^2) \\
    D(g_2(t)) &= D(\mathbb{1}_n) + td(X_2) + O(t^2)
  \end{align}
  Therefore, we have
  \begin{align}
    D(h(t)) &= D(\mathbb{1}_m + t^2 [X_1, X_2] + O(t^3)) \\
	    &= D(\mathbb{1}_m) + t^2 \left.\dv[2]{D(h(t))}{t}\right\rvert_{t = 0} + \dots \\
	    &= \mathbb{1}_n + d([X_1, X_2]) t^2 + \dots \\
    D(h) &= D(g_1)^{-1} D(g_2)^{-1} D(g_1) D(g_2)
  \end{align}
  We then plug in Taylor lens (?) and copare terms at $O(t^2)$ to give
  \begin{equation}
    d([X_1, X_2]) = [d(X_1), d(X_2)]
  \end{equation}
  The second property of Lie brackets follows automatically from the definition.
\end{proof}
\begin{exercise}
  Let $d$ be a representation of $\mathscr{L}(G)$. For all elements of $G$ of the form $g = \text{Exp}(X)$, with $X \in \mathscr{L}(G)$. Then define
  \begin{equation}
    D(g) = D(\text{Exp} X) \coloneqq \text{Exp} (d(X)).
  \end{equation}
  Under what circumstances does this give a good representation of $G$?
\end{exercise}

\chapter{Representations of Lie Algebras}%
\label{cha:representations_of_lie_algebras}

Let $\mathfrak{g}$ be a Lie algebra of dimension $D$ and let $X \in \mathfrak{g}$.
For any matrix Lie algebra $\mathfrak{g} = \mathscr{L}(G)$ of some matrix Lie group $G \subset \text{Mat}_n(F)$, we can always write down three canonical representations: For all $X \in \mathfrak{g}$, we define
\begin{description}
  \item[the trivial representation $d_0$] with dimension $\dim(d_0) = 1$ by $d_0(X) = 0$.
  \item[the fundamental representation $d_f$] with dimension $\dim(d_f) = n$ by $d_f(X) = X$.
  \item[the adjoint representation $d_{\text{Adj}}$] with dimension $\dim(d_{\text{Adj}}) = \dim( \mathfrak{g}) = D$. To define the adjoint representation $d_{\text{Adj}}(X)$, we first consider the linear map $ad_X$ defined by
    \begin{equation}
      \begin{split}
        ad_X \colon \mathfrak{g} \ &\to\  \mathfrak{g} \\
	Y \ &\mapsto\  [X,Y]
      \end{split}
    \end{equation}
    $ad_X$ is equivalent to a $D \times D$ matrix. For a choice of basis $B = \left\{ T^a \right\}$, for $a = 1, \dots, D$, we can expand two elements $X, Y \in \mathfrak{g}$ as
    \begin{equation}
      X = X_a T^a \qquad Y = Y_a T^a.
    \end{equation}
    In particular, as we have seen before, the bracket of $X, Y$ is determined by the bracket of the generators $T^a$
    \begin{equation}
      [X, Y] = X_a Y_b [T^a, T^b] = X_a Y_b f \indices{^a^b_c}T^c.
    \end{equation}
    Similarly, the $c$\textsuperscript{th} component of $ad_X(Y)$ is
    \begin{equation}
      [ad_X(Y)]_c = (R_X)^b_c Y_b \quad \implies \quad \boxed{(R_X) \indices{^b_c} = X_a f \indices{^a^b_c}}.
    \end{equation}
    The adjoint representation is then defined for all $X \in \mathfrak{g}$ by
    \begin{equation}
      d_{\text{Adj}}(X) = ad_X.
    \end{equation}
    More concretely, we can define it in a particular basis as $[d_{\text{Adj}}(X)] \indices{^b_c} = (R_X) \indices{^b_c}.$ 
\end{description}
\begin{leftbar}
  \begin{remark}
    The trivial and adjoint representations exist for all Lie algebras.
  \end{remark}
\end{leftbar}
\begin{claim}
  We want to show that the adjoint representation is indeed a representation.
  Let us check the defining properties of a representation.
  For all $X, Y \in \mathfrak{g}$, we must have
  \begin{equation}
    \label{eq:l10star}
    [d_{\text{Adj}}(X), d_{\text{Adj}}(Y)] = d_{\text{Adj}}([X, Y])
  \end{equation}
\end{claim}
\begin{proof}
  We have $d_{\text{Adj}}(X) = ad_X$ and $d_{\text{Adj}}(Y) = ad_Y$.
  Hence, for all elements $Z \in \mathfrak{g}$ of the Lie algebra, we have
  \begin{align}
    \bigl(d_{\text{Adj}}(X) \circ d_{\text{Adj}}(Y) \bigr)(Z) &= [X, [Y, Z]] \\
    \bigl(d_{\text{Adj}}(Z) \circ d_{\text{Adj}}(Y) \bigr)(X) &= [Z, [Y, X]] \\
  \end{align}
  Subtracting these two, we get
  \begin{equation}
    [d_{\text{Adj}}(X), d_{\text{Adj}}(Y)](Z) = [X, [Y, Z]] - [Y, [X, Z]]
  \end{equation}
  This is the left hand side of \eqref{eq:l10star}. Subtracting the right hand side we get
  \begin{align}
    (\text{LHS}- \text{RHS}) (Z) &= [X, [Y, Z]] - [Y, [X, Z] ] - [[X, Y], Z] \\
				 &= [X, [Y, Z] ] + [Z, [X, Y] ] + [Y, [Z, X] ]= 0
  \end{align}
  where we used the Jacobi identity.
  We see that Jacobi identity of the Lie algebra therefore ensures that the Lie algebra has an adjoint representation on itself.
  Property two of the Lie algebra is satisfied since $ad_X$ is linear in $X$.
\end{proof}
