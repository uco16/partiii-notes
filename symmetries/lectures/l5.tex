% lecture notes by Umut Özer
% course: symmetries
\lhead{Lecture 5: October 19}
In this course, we are interested in classifying Lie groups and Lie algebras. We have isomorphisms and homeomorphisms, maps which preserve group and manifold structure respectively.

\begin{definition}[isomorphism]
  Two Lie groups $G$ and $G'$ are \emph{isomorphic} $(G \simeq G')$ if there exists a one-to-one smooth map $J : G \to G'$ such that for all $g_1, g_2 \in G$, we have $J(g_1 g_2) = J(g_1) J(g_2)$.
\end{definition}

Let us look at some low-dimensional examples of unitary groups:
\begin{example}[$G = U(1)$]
  Let $U(1) = \left\{ z \in \mathbb{C} \mid \abs{z} = 1 \right\}$. A general element $z = e^{i \theta}$ of $G = U(1)$, parametrised by $\theta \in \mathbb{R}$ with identification $\theta \sim \theta + 2\pi$, corresponds to a unique element
  \begin{equation}
    M(\theta) =
    \begin{pmatrix}
     \cos\theta & -\sin\theta \\
     \sin\theta & \cos\theta \\
    \end{pmatrix}
  \end{equation}
  of $G' = SO(2)$ via the map
  \begin{equation}
    \begin{split}
      J \colon U(1) &\to SO(2) \\
      z(\theta) = e^{i\theta} &\mapsto M(\theta).
    \end{split}
  \end{equation}
  The map $J$ is one-to-one and
  \begin{equation}
    J(z(\theta_1) z(\theta_2)) = M(\theta_1 + \theta_2) = M(\theta_1) M(\theta_2) = J(z(\theta_1)) J(z(\theta_2)).
  \end{equation}
  This implies that $U(1) \simeq SO(2)$.
\end{example}

\begin{example}[$G = SU(2)$]
  This is a three-dimensional group. The matrix parametrised as $U = a_0 I_2 + i \vb{a} \cdot \boldsymbol\sigma$, where $\boldsymbol\sigma = (\sigma_1, \sigma_2, \sigma_3)$ and $a_0 \in \mathbb{R}$, $\vb{a} = (a_1, a_2, a_3) \in \mathbb{R}^3$, is an element of $SU(2)$ provided that 
  \begin{equation}
    a_0^2 + a_1^2 + a_2^2 + a_3^2 = 1.
  \end{equation}
  This implies that $M(SU(2)) \simeq S^3 \subset \mathbb{R}^4$.
  Since 
  \begin{equation}
    \pi_1(SU(2)) \simeq \left\{ 1 \right\}, \qquad \pi_1(SO(3)) = \mathbb{R}_2,
  \end{equation}
  this means that $SU(2) \not\simeq SO(3)$.
\end{example}

\chapter{Lie Algebras}%
\label{cha:lie_algebras}

\begin{definition}[Lie algebra]
  A \emph{Lie algebra} $\mathfrak{g}$ is a vector space over a field $F = \mathbb{R}, \mathbb{C}$ with a bracket, 
  \begin{equation}
    [, ]: \mathfrak{g} \times \mathfrak{g} \to \mathfrak{g}
  \end{equation}
  with the following properties for all $X, Y, Z \in \mathfrak{g}$:
  \begin{enumerate}
    \item Anti-symmetry: $[X, Y] = -[Y, X]$.
    \item (Bi-)linearity: $[\alpha X + \beta Y, Z] = \alpha[X, Z] + \beta [Y, Z]$, for all coefficients $\alpha, \beta \in F$.
    \item Jacobi identity: $[X, [Y, Z] ] + [Y, [Z, X] ] + [Z, [X, Y] ] = 0$.
  \end{enumerate}
\end{definition}

Let $V$ be a vector space that has a product $*: V \times V \to V$, which is associative, meaning that for all $X, Y, Z \in V$,
\begin{equation}
  (X * Y) * Z = X*(Y*Z).
\end{equation}
Moreover, the product is distributive over the field
\begin{equation}
  Z * (\alpha X + \beta Y) = (\alpha Z*X + \beta Z*Y).
\end{equation}
Then, we obtain a Lie algebra from the following definition of a Lie bracket
\begin{equation}
  [X, Y] = X*Y - Y*X.
\end{equation}
One example we have in mind is the case where $V$ is the vector space of matrices and $*$ is matrix multiplication. 
\begin{leftbar}
  \begin{remark}
    Compare this to the Lie algebra of differential operators/vectors in differential geometry.
  \end{remark}
\end{leftbar}

\begin{definition}[dimension]
  The dimension of a Lie algebra $\mathfrak{g}$ is the dimension of its underlying vector space.
\end{definition}

Choose a basis $B$ for $\mathfrak{g}$
\begin{equation}
  B = \left\{ T^a, a=1, \ldots, n = \dim(\mathfrak{g}) \right\}.
\end{equation}
Then any $X \in \mathfrak{g}$ can be written as
\begin{equation}
  X = X_a T^a \coloneq \sum_{a=1}^n X_a T^a
\end{equation}
where $X_a \in F$. Bracket of elements $X, Y \in \mathfrak{g}$ can then be written as
\begin{equation}
  [X, Y] = X_a Y_b [T^a, T^b].
\end{equation}
Therefore, knowing the brackets of basis elements allows us to construct the full Lie algebra:
\begin{equation}
  [T^a, T^b] = f \indices{^a^b_c} T^c .
\end{equation}
The \emph{structure constants} $f \indices{^a^b_c}$ therefore define the Lie algebra, and two Lie algebras are isomorphic if they have the same structure constants.
Note however, that structure constants are basis dependent. We will want to find a way to classify Lie algebras that is independent of our choice of basis.

\section{Structure Constants}%
\label{sec:structure_constants}

Let $f \indices{^a^b_c} \in F$, $a, b, c = 1, \ldots, \dim(\mathfrak{g})$ be structure constants of a Lie algebra $\mathfrak{g}$.
The axioms of Lie algebras then imply
\begin{enumerate}
  \item $\implies f \indices{^a^b_c} = -f \indices{^b^a_c}$
  \item $\implies f \indices{^a^b_c} f \indices{^c^d_e} + f \indices{^d^a_c} f \indices{^c^b_e} + f \indices{^b^d_c} f \indices{^c^a_e} = 0$
\end{enumerate}

\begin{definition}[isomorphism]
  Two Lie algebras $\mathfrak{g}$ and $\mathfrak{g}'$ are said to be \emph{isomorphic}, $\mathfrak{g} \simeq \mathfrak{g}'$ if there exists a linear, one-to-one map $f: \mathfrak{g} \to \mathfrak{g}'$ such that
  \begin{equation}
    [f(X), f(Y)] = f([X, Y]), \qquad \forall X, Y \in \mathfrak{g}.
  \end{equation}
\end{definition}

\begin{definition}[subalgebra]
  A \emph{subalgebra} $\mathfrak{h} \subset \mathfrak{g}$ is a vector subspace of $\mathfrak{g}$ which is also a Lie algebra.
\end{definition}

\begin{definition}[ideal]
  An \emph{ideal} of $\mathfrak{g}$ is a subalgebra $\mathfrak{h}$ of $\mathfrak{g}$ with
  \begin{equation}
    [X, Y] \in \mathfrak{h}, \qquad \forall X \in \mathfrak{g}, Y \in \mathfrak{h}
  \end{equation}
  The notion of ideal roughly corresponds to the concept of a normal subgroup.
\end{definition}

\begin{example}[trivia algebras]
  Every Lie algebra $\mathfrak{g}$ has two `trivial' ideals:
  \begin{equation}
    \mathfrak{h} = \left\{ 0 \right\} \qquad \text{and} \qquad \mathfrak{h} = \mathfrak{g}.
  \end{equation}
\end{example}
\begin{example}[derived algebra]
  The \emph{derived algebra}
  \begin{equation}
    i = [\mathfrak{g}, \mathfrak{g}] \coloneq = \text{Span}_F \left\{ [X, Y] \mid X, Y \in \mathfrak{g} \right\}
  \end{equation}
  is an ideal of $\mathfrak{g}$.
\end{example}
\begin{example}[centre]
  The \emph{centre} of $\mathfrak{g}$
  \begin{equation}
    \xi(\mathfrak{g}) = \left\{ X \in \mathfrak{g} \mid [X, Y] = 0 \quad \forall Y \in \mathfrak{g} \right\}.
  \end{equation}
\end{example}

\begin{definition}[abelian]
  An Abelian Lie algebra is such that all brackets vanish:
  \begin{equation}
    [X, Y] = 0 \qquad \forall X, Y \in \mathfrak{g}
  \end{equation}
\end{definition}

For an Abelian Lie algebra, the Lie algebra is equal to its own centre $\mathfrak{g} = \xi(\mathfrak{g})$ and the ideal is trivial $i(g) = \left\{ 0 \right\}$.

\begin{definition}[simple]
  A Lie algebra $\mathfrak{g}$ is said to be \emph{simple} if it is non-Abelian and it has no non-trivial ideals.
\end{definition}
This implies that for simple Lie algebras, $\xi(\mathfrak{g}) = \left\{ 0 \right\}$ and $i(\mathfrak{g}) = \mathfrak{g}$.

The main theorem that we will work up to is the \emph{Cartan classification}, which will allow us to classify all finite-dimensional, simple, complex Lie algebras $\mathfrak{g}$.
