% lecture notes by Umut Özer
% course: symmetries
\lhead{Lecture 20: November 26}

There are no arbitrary structure constants in this set of brackets.
In fact, this is not yet a basis (despite us naming it that); we only have $2 \times \text{rank}(\mathfrak{g})$ step generators associated with the simple roots. In general, the number of step generators is $\dim(\mathfrak{g}) - \text{rank}(\mathfrak{g})$, which is typically greater than $2r$.

The full algebra is generated by repeated brackets subject to the \emph{Chevally-Serre relations}:
\begin{equation}
  \boxed{(\text{ad}_{e^i_\pm})^{1 - A^{ji}} e_\pm^{j} = 0} \qquad \forall i, j = 1, \dots, r.
\end{equation}

This encodes the fact that the `$\alpha_{(i)}$ string through $\alpha_{(j)}$' has length
\begin{equation}
  l_{ij} = 1 - \frac{2(\alpha_{(i)}, \alpha_{(j)})}{(\alpha_{(i)}, \alpha_{(i)})} = 1 - A^{ji},
\end{equation}
which follows from Claim \ref{cl:19-ii}.

Let us see how the full algebra is generated. Consider first
\begin{equation}
  [e_+^{i}, e_+^{j}] = [e^{+\alpha_{(i)}}, e^{+\alpha_{(j)}}] \propto
  \begin{cases}
    e^{\alpha_{(j)} + \alpha_{(i)}} & \alpha_{(i)} + \alpha_{(j)} \in \Phi \\
    0 & \text{else}
  \end{cases}
\end{equation}
Applying $\text{ad}_{e^i_+}$ again, we get
\begin{equation}
  [e_+^{i}, [e_+^{i}, e_+^{j}]] \propto e^{\alpha_{(j)} + 2 \alpha_{(i)}} \qquad \text{if } \alpha_{(j)} + 2 \alpha_{(i)} \in \Phi
\end{equation}
Repeatedly applying the bracket, we find 
\begin{equation}
  (\text{ad}_{e_+^{i}})^n e^{j}_{+} = \underbrace{[e_+^{i}, \dots [e_+^{i}, [e_+^{i},}_{\mathclap{n \text{ times}}} e_+^{j}]]] \propto e^{\alpha_{(j)} + n \alpha_{(i)}} \qquad \text{if } \alpha_{(j)} + n \alpha_{(i)} \in \Phi
\end{equation}
Hence, by the Chevally-Serre relations, $\alpha_{(j)} + n \alpha_{(i)} \in \Phi$ if and only if $n < 1 - A^{ji} = l_{ij}$.


\section{Reconstructing \texorpdfstring{$\mathfrak{g}$}{the Lie Algebra} from \texorpdfstring{$A^{ij}$}{the Cartan Matrix}}%
\label{sec:reconstructing_g_from_a_ij}

We will now proceed with the Cartan classification in multiple parts.
Starting with generators corresponding with the simple roots, we generate other roots with the root strings. Their brackets are given by iterating the Jacobi identities and the Serre-Chevally relations.

\begin{claim}
  A finite-dimensional simple complex Lie algebra is uniquely determined by its Cartan matrix.
\end{claim}
If this is true, then to classify Lie algebras, we want to find out about these Cartan matrices.

\subsection{Constraints on the Cartan Matrix}%
\label{sub:constraints_on_cartan_matrix}

The Cartan matrix $ A^{ij} = 2\frac{(\alpha_{(i)}, \alpha_{(j)})}{(\alpha_{(j)}, \alpha_{(j)})} \in \mathbb{Z} $ has the following properties:
\begin{enumerate}[a)]
  \item $A^{ii} = 2$ for $i = 1, \dots, r$
  \item $A^{ij} = 0 \iff A^{ji} = 0$; zeros on the off-diagonal come in pairs
  \item Claim \ref{cl:19-iii} tells us that $A^{ij} \in \mathbb{Z}_{\leq 0}$ for $i \neq j$
  \item $\det A > 0$
  \item $\mathfrak{g}$ is simple $\implies$ A is irreducible, meaning it is not block diagonal:
    \begin{equation}
      A \neq 
      \begin{pmatrix}
      A^{(1)} & 0 \\
      0 & A^{(2)}
      \end{pmatrix}
    \end{equation}
\end{enumerate}

\begin{proof}[Proof of d)]
  Let $\lambda, \mu \in \mathfrak{h}^*_{\mathbb{R}}$. We employ Einstein summation convention $\lambda = \sum_{i=1}^{r} \lambda^{i} \alpha_{(i)} = \lambda^{i} a_{(i)}$. Then $(\lambda, \mu) = (\kappa^{-1})_{ij} \kappa^{i} \mu^{j}$ with $(\kappa^{-1})^{ij} = (\alpha_{(i)}, \alpha_{(j)})$, with $i, j = 1, \dots, r$.
  The inverse Killing form $\kappa^{-1}$ is a real symmetric matrix. This means we can diagonalise it with some orthogonal matrix $O$
  \begin{equation}
    O (\kappa^{-1}) O^T = \text{diag}(l_1, \dots, l_r), \qquad l_i \in \mathbb{R}.
  \end{equation} 
  For each eigenvalue $l$, there is an eigenvector $v_{l} = v_{l}^{i} \alpha_{(i)}$ which satisfies
  \begin{equation}
    (\kappa^{-1})_{ij} v_{l}^{j} = l \delta_{ij} v_{l}^{j}
  \end{equation}
  Its length squared is therefore
  \begin{equation}
    (v_{l}, v_{l}) = (\kappa^{-1})_{ij} v^{i}_{l} v^{j}_{l} = l \delta_{ij} v_l^{i} v_l^{j} = \abs{v_{l}}^2 > 0
  \end{equation}
  Hence, every eigenvalue $l > 0$ is positive, and $\text{det}(\kappa^{-1}) > 0$.
  Now the Cartan matrix is $A^{ij} =  S^{ik} D\indices{_{k}^{j}}$, where $S^{ik} = (\alpha_{(i)}, \alpha_{(j)}) = (\kappa^{-1})_{ik}$ and $D\indices{_k^j} = \frac{2}{(\alpha_{(j)}, \alpha_{(j)})} \delta^{j}_{k}$ . Hence $\det A = \det (D) \det (S) > 0$.
\end{proof}
\begin{proof}[Proof of e)]
  Suppose that $A = \begin{pmatrix}
      A^{(1)} & 0 \\
      0 & A^{(2)}
      \end{pmatrix} $ is reducible.
  The simple root set then decomposes into $ \Phi_S = \Phi^{(1)} \cup \Phi^{(2)}$ with $(\alpha, \beta) = 0$ whenever $\alpha \in \Phi^{(1)}$ and $\beta \in \Phi^{(2)}$.
  As the simple roots span the root space, we have $ \mathfrak{h}^*_{\mathbb{R}} = R_1 \oplus R_2, $ where $R_i = \text{Span}_{\mathbb{R}}\left\{\alpha \in \Phi^{(i)}\right\}$.
  As the inner product is \emph{Euclidean}, these are orthogonal $R_2 = (R_1)_{\perp}$ and their intersection $R_1 \cap R_2 = \emptyset$ is empty.
  Now consider 
  \begin{equation} 
    \mathfrak{g}_1 = \text{Span}_{\mathbb{C}}\left\{h^{\alpha}, e^{\alpha} \suchthat \alpha \in \Phi^{(1)}\right\}.
  \end{equation}
  \begin{claim}
    $\mathfrak{g}_1$ is an ideal of $\mathfrak{g}$.
  \end{claim}
  \begin{leftbar}
    \begin{proof}[Proof]
      Check $[X, Y] \in \mathfrak{g}_1$ for all $X \in \mathfrak{g}$ and $Y \in \mathfrak{g}_1$.
      Any non-trivial case $X = e^{\alpha'}$ , where $\alpha' \in R_2 = (R_1)_{\perp}$ gives
       \begin{equation}
	[e^{\alpha'}, h^{\alpha} ] = -2 \frac{(\alpha,\alpha')}{(\alpha, \alpha)} e^{\alpha'} = 0 \qquad \forall \alpha \in R_1
      \end{equation}
       \begin{equation}
	[e^{\alpha'}, e^{\alpha} ]
	\begin{cases}
	  \propto e^{\alpha' + \alpha} & \alpha + \alpha' \in \Phi \\
	  = 0 & \text{otherwise}.
	\end{cases}
      \end{equation}
    \end{proof}
  \end{leftbar}
  Reconstructing the root system with root strings we find for $\alpha \in \Phi^{(1)}$ and $\beta \in \Phi^{(2)}$
  \begin{equation}
    l_{\alpha, \beta} = 1 - 2 \frac{(\alpha, \beta)}{(\alpha, \alpha)} = 1,
  \end{equation}
  which means that all roots $\alpha \in \Phi$ belong to either $\alpha \in R_1$ or $\alpha \in R_2$.
  Then consider $\mathfrak{g}_1 = \text{Span}_{\mathbb{C}}\left\{h^{\alpha}, e^{\alpha} \suchthat \alpha \in R_1\right\}$.
  \begin{claim}
    $\mathfrak{g}_1$ is a non-trivial ideal of $\mathfrak{g}$\footnote{See Gutowski Proposition 6.}.
  \end{claim}
  \begin{leftbar}
    \begin{proof}
      Check $[X, Y] \in \mathfrak{g}_1$ for all $X \in \mathfrak{g}$ and $Y \in \mathfrak{g}_1$.
      The only non-trivial bracket is $X = e^{\beta}$ for $\beta \in R_2$ and $Y = e^{\alpha}$ for $\alpha \in R_1$
      \begin{equation}
        [e^{\beta} , e^{\alpha}] 
	\begin{cases}
	  \propto e^{\beta + \alpha} & \beta + \alpha \in \Phi \\
	  = 0 & \text{otherwise}
	\end{cases}
      \end{equation}
      Now if $\alpha \in R_1$ and $\beta \in R_2$ then $\alpha + \beta \neq R_1 \cup R_2$ or $R_1 \cap R_2$ is non-empty. Hence $[e^{\beta} , e^{\alpha}] = 0$.
    \end{proof}
  \end{leftbar}
  Therefore, $A$ being reducible implies that $\mathfrak{g}$ is not simple.
  Hence, $\mathfrak{g}$ being simple implies that $A$ is irreducible.
\end{proof}

For $r = \text{rank}(\mathfrak{g}) = 1$ we simply have $A = 2$. However, for $r = 2$ we have a Cartan matrix of the form
\begin{equation}
  \begin{pmatrix}
    2 & -m \\
    -n & 2
  \end{pmatrix}
  \qquad m, n \in \mathbb{N}_0
\end{equation}
From property d), we get that $4 - mn > 0$ and up to exchange of $m$ and $n$ we therefore have only the pairs $(1, 1), (1, 2), (1, 3)$.
We want to represent the data of the Cartan matrix in a way that is invariant under this relabelling.

\begin{exercise}
  Show that $A^{ij} A^{ji} = \left\{ 0,1,2,3 \right\}$ for $i \neq j$ and enumerate the possible solutions for $A^{ij}$.
\end{exercise}

\begin{exercise}
  Show that a simple Lie algebra $\mathfrak{g}$ has simple roots of at most two different lengths.
\end{exercise}

\section{Dynkin Diagrams}%
\label{sec:dynkin_diagrams}

Dynkin diagrams the data contained in the Cartan matrix in diagrammatic form.
\begin{enumerate}[]
  \item Draw a node for each $\alpha_{(i)} \in \Phi_S$, $i = 1, \dots, r$.
\item Join nodes $\alpha_{(i)}$ and $\alpha_{(j)}$ with $\text{max}(\abs{A}^{ij}, \abs{A}^{ji}) \in \{0, 1, 2, 3\}$ lines.
  \item If roots have different lengths, then draw an array from long to short\footnote{This distinction makes sense since there can only be two different simple root lengths.}.
\end{enumerate}
If we relabel the points in the diagram, we get the same diagram. The process of enumerating the roots of the Lie algebra comes down to enumerating Dynkin diagrams.
The classification of Cartan matrices / Dynkin diagrams satisfying constraints a) - e) is a purely combinatoric problem.

\newcommand*{\Dynk}[2]{#1_{#2} & \dynkin{#1}{#2}}

Let us translate what we learned at the level of the Cartan matrix for ranks $r = 1,2$ into diagrams:
\[ \begin{array}{ll}
  r = 1 & \dynkin{A}{1} \\
  r = 2 & \dynkin{A}{2}
	 \quad \dynkin{B}{2}
	 \quad \dynkin{G}{2}
\end{array} \]

In 1894, Cartan showed that the diagrams are classified into infinite families labelled by $\text{rank}(\mathfrak{g}) = n \in \mathbb{N}^+$:
\[ \begin{array}{lll}
    A_n & \dynkin{A}{} & \mathfrak{su}_{\mathbb{C}}(n+1) \\
    B_n & \dynkin{B}{} & \mathfrak{so}_{\mathbb{C}}(2n+1) \\
    C_n & \dynkin{C}{} & \mathfrak{sp}_{\mathbb{C}}(2n) \\
    D_n & \dynkin{D}{} & \mathfrak{so}_{\mathbb{C}}(2n) \\
  \end{array} \qquad
  \begin{array}{ll}
    \Dynk{E}{6} \\
    \Dynk{E}{7} \\
    \Dynk{E}{8} \\
    \Dynk{F}{4} \\
    \Dynk{G}{2} \\
  \end{array} \]
\begin{leftbar}
  \begin{note}
    We have not talked about the symplectic group $Sp(2N)$ in this course.
  \end{note}
\end{leftbar}
For $n = 1$, we have $A_1 \simeq B_1 \simeq C_1$, corresponding to the fact that $ \mathfrak{su}_{\mathbb{C}}(2) \simeq \mathfrak{so}_{\mathbb{C}}(3) \simeq \mathfrak{sp}_{\mathbb{C}}(2)$.
For $n = 2$, we have $B_2 \simeq C_2$  and thus $ \mathfrak{so}_{\mathbb{C}}(5) \simeq \mathfrak{sp}_{\mathbb{C}}(4)$.
For $n = 3$, $D_3 \simeq A_3$, meaning that $ \mathfrak{so}_{\mathbb{C}}(6) \simeq \mathfrak{su}_{\mathbb{C}}(4)$.
