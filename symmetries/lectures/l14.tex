% lecture notes by Umut Özer
% course: symmetries
\lhead{Lecture 14: November 12}
For the Lie algebra of $SU(2)$, we have already describe the full set of irreducible representations $R_\Lambda$ with $\dim(R) = \Lambda + 1$, where $\Lambda \in \mathbb{N}_0$.
Let us consider the tensor product representations $R_{\Lambda} \otimes R_{\Lambda'}$. Because $\mathfrak{su}(2)$ is simple, it is fully reducible. In other words, we can decompose the product representation into a sum over the irreps $R_{\Lambda} \otimes R_{\Lambda'} \simeq \bigoplus_{\Lambda'' \in \mathbb{N}_0} \mathcal{L}_{\Lambda, \Lambda'}^{\Lambda''} R_{\Lambda''}$. Our task is now to find the multiplicities $\mathscr{L}$, which describe how often each irrep enters the decomposition.
Let us use the Cartan basis. The vector space $V_\Lambda$ has basis $\left\{ v_{\lambda} \right\}$, $\lambda \in S_\Lambda = \left\{ -\Lambda, -\Lambda +2, \dots, + \Lambda \right\}$.
The eigenvectors $R_\Lambda(H) v_{\lambda} = \lambda v_{\lambda}$. Similarly, we have a second vector space $V_{\lambda'}$ with basis $\left\{ v'_{\lambda'} \right\}$.
Construct a basis for $V_\Lambda \otimes V_{\Lambda'}$ as $B = \left\{ v_{\lambda} \otimes v'_{\lambda} \mid \lambda \in S_\Lambda, \lambda' \in S_{\Lambda'} \right\}$.
We now want to understand the weights of the new representation. To work out this weight, we act on a general basis vector with the representation of the Cartan element $H$. By definition, the tensor product representation acts as
\begin{align}
  (R_\Lambda \otimes R_{\Lambda'})(H) (v_{\lambda} \otimes v_{\lambda'}) &= (R_\Lambda (H) v_{\lambda}) \otimes v'_{\lambda'} + v_{\lambda} \otimes (R_{\Lambda'} (H) v'_{\lambda'}) \\
									 &= (\lambda + \lambda') (v_{\lambda} \otimes v_{\lambda'}).
\end{align}
The eigenvalues are just the sums of the eigenvalues of the basis elements. Therefore, we can deduce that $R_\Lambda \otimes R_{\Lambda'}$ has weight set $S_{\Lambda, \Lambda'} = \left\{ \lambda + \lambda' \mid \lambda \in S_{\Lambda}, \lambda' \in S_{\Lambda'} \right\}$. Sometimes, this sum adds up the same weight in $n$ different cases. We then have to keep track of these $n$ multiplicities.
We can then construct the representation, which is the sum of irreducibles, explicitly.
The irreps are uniquely determined by identification of their weight set.
We must take this weight set and decompose it into unions of the weight sets associated with the irreducibles; we will see that there is a unique way to do this.
Consider first the highest weight. This will only come when $\lambda = \Lambda$ and $\lambda' = \Lambda'$. Since there is only one way to obtain the sum $\Lambda + \Lambda'$, the irrep $R_{\Lambda + \Lambda'}$ must appear in the decomposition with multiplicity one!
In other words, $\mathcal{L}^{\Lambda + \Lambda'}_{\Lambda, \Lambda'} = 1$.
We can decompose the tensor product as
\begin{equation}
  R_{\Lambda} \otimes R_{\Lambda'} = R_{\Lambda + \Lambda'} \oplus \widetilde{R}_{\Lambda, \Lambda'}.
\end{equation}
The problem is reduced to finding the remainder $\widetilde{R}_{\Lambda, \Lambda'}$, which will have weight set $\widetilde{S}_{\Lambda, \Lambda'}$ where $S_{\Lambda, \Lambda'} = S_{\Lambda + \Lambda'} \cup \widetilde{S}_{\Lambda, \Lambda'}$. 
We remove the weight set
\begin{equation}
  S_{\Lambda + \Lambda'} = \left\{ -\Lambda - \Lambda', \dots, + \Lambda + \Lambda' \right\}, 
\end{equation}
and find the highest weight of the remainder $\widetilde{R}_{\Lambda, \Lambda'}$ and keep repeating until the remainder is empty.

\begin{example}[$\Lambda = \Lambda' = 1$]
  In this case, the weight set of both representations consists of two elements
  \begin{equation}
    S_1 = \left\{ -1 , +1 \right\}.
  \end{equation}
  This is the representation a spin-$\frac{1}{2}$ particle carries.
  Then the weight set of the tensor product is 
  \begin{align}
    S_{1, 1} &= \left\{ -1, +1 \right\} + \left\{ -1, +1 \right\} = \left\{ -2, 0, 0, +2 \right\}. \\
	     &= \left\{ -2, 0, +2 \right\} \cup \left\{ 0 \right\},
  \end{align}
  where $\left\{ -2, 0, +2 \right\}$ is the weight set of the highest weight representation $R_2$. We find that
  \begin{equation}
    R_1 \otimes R_1 = R_2 \oplus R_0.
  \end{equation}
  In quantum mechanics, this means that two spin-$\frac{1}{2}$ particles form states of spin-$1$ (triplet) and spin-$0$ (singlet).
  Note that the dimensions match as well.
\end{example}
\begin{exercise}[Sheet 2, Qu 8]
  Consider two irreps of $SU(2)$: $\Lambda' = M$ and $\Lambda = N$. Then show, by applying the above algorithm, that $R_N \otimes R_M = R_{\abs{M-N}} \oplus R_{\abs{M-N} + 2} \oplus \dots \oplus R_{N + M}$.
  For $SU(2)$ this is essentially the whole story.
\end{exercise}

\chapter{Cartan Classification}%
\label{cha:cartan_classification}

Our ultimate aim of this chapter will be to classify all finite-dimensional simple complex Lie algebras $\mathfrak{g}$ (Cartan 1894). However, before we can do this we will have to build up some machinery.

\section{The Killing Form}%
\label{sec:the_killing_form}

We can specify an inner product on $\mathbb{R}^3$, by mapping $\vb{u}, \vb{v} \to \vb{u} \cdot \vb{v}$. 
Similarly, we might specify this scalar product by giving a metric such as $u_i, v_j \to \delta^{ij} u_i v_j$.

\begin{definition}[Inner Product]
  Given a vector space $V$ over $F = \mathbb{R}$ or $\mathbb{C}$, an \emph{inner product} is a bilinear, symmetric map $i: V \times V \to F$.
\end{definition}

\begin{definition}[]
  We say that $i$ is \emph{non-degenerate} if $\forall v \in V, v \neq 0$ there is a $w \in V$ such that the inner product $i(v, w) \neq 0$.
\end{definition}
\begin{leftbar}
  \begin{remark}
    This amounts to saying that there is no vector which is orthogonal to itself and all other vectors in the vector space.
  \end{remark}
\end{leftbar}
Is there a `natural' inner product that we can write down on a Lie algebra $\mathfrak{g}$?
Yes! This is the \emph{Killing form}.
\begin{definition}[]
  The \emph{Killing form} $\kappa$ of a Lie algebra $\mathfrak{g}$ over a field $F$ is the map
  \begin{equation}
    \begin{gathered}
      \kappa \colon \\
      \qquad
    \end{gathered}
    \begin{gathered}
      \mathfrak{g} \times \mathfrak{g} \\
      (X, Y)
    \end{gathered}
    \quad
    \begin{gathered}
      \to \\
      \mapsto
    \end{gathered}
    \quad
    \begin{gathered}
      F \\
      \tr(\text{ad}_{X} \circ \text{ad} _{Y})
    \end{gathered}
  \end{equation}
\end{definition}
Let us find out what this means explicitly in components.
The action of the inner $ad$-map composition is
\begin{equation}
  \begin{split}
    (ad_X \circ ad_Y) \colon \mathfrak{g} \ &\to\  \mathfrak{g} \\
    Z \ &\mapsto\  [X,[Y, Z]].
  \end{split}
\end{equation}
Choose a basis $\left\{ T^a \right\}$, where $a = 1, \dots, D=\dim(\mathfrak{g})$, for the Lie algebra $\mathfrak{g}$. 
We then know that by definition of the structure constants $f \indices{^{ab}_c}$, we have $[T^a, T^b] = f \indices{^{ab}_c} T^c$.
Expanding the elements of the Lie algebra in terms of this basis, we see that
\begin{align}
  [X, [Y, Z]] &= X_a Y_b Z_c [T^a, [T^b, T^c]] \\
	      &= X_a Y_b Z_c f \indices{^{ad}_e} f \indices{^{bc}_d} T^e \\
	      &= M(X, Y) \indices{^a_e} Z_c T^e
\end{align}
with $M(X, Y) \indices{^c_e} = X_a Y_b f \indices{^{ad}_e} f^{bc_d}$. This is the explicit form of the inner $ad$-map composition.
To find the explicit map defined by the Killing form, we take the trace of this:
\begin{equation}
  \kappa(X, Y) = \Tr_D [M(X, Y)] = \kappa^{ab} X_a Y_b,
\end{equation}
where, concretely, the Killing form is given by
\begin{equation}
  \boxed{\kappa^{ab} = f^{ad}_c f^{bc}_d}
\end{equation}
This is manifestly symmetric.

Now what does `natural' mean? 
\begin{claim}
  The Killing form $\kappa$ remains invariant under the adjoint action of the Lie algebra $\mathfrak{g}$, meaning that $\forall X, Y, Z \in \mathfrak{g}$
  \begin{equation}
    \label{eq:14-1}
    \boxed{\kappa([Z, X], Y) + \kappa(X, [Z, Y]) = 0}.
  \end{equation}
\end{claim}
\begin{proof}
  By the defining property of the adjoint representation, we have
  \begin{equation}
    ad_{[Z, X]} = ad_Z \circ ad_X - ad_X \circ ad_Z.
  \end{equation}
  Therefore, the first term of \eqref{eq:14-1} is
  \begin{align}
    \kappa([Z, X], Y) &= \Tr[ad_{[Z, X]} \circ ad_Y] \\
		      &= \Tr[ad_Z \circ ad_X \circ ad_Y] - \Tr[ad_X \circ ad_Z \circ ad_Y]
  \end{align}
  Moreover, the second term of \eqref{eq:14-1} is
  \begin{equation}
    \kappa (X, [Z, Y]) = \Tr[ad_X \circ ad_Z \circ ad_Y] - \Tr[ad_X \circ ad_Y \circ ad_Z].
  \end{equation}
  By cyclic invariance of the trace $\Tr(ABC) = \Tr(CAB) = \Tr(BCA)$, we find that these two terms exactly cancel.
\end{proof}
