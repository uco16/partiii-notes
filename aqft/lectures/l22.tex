% lecture notes by Umut Özer
% course: aqft
\lhead{Lecture 22: March 07}

\subsection{Faddeev--Popov Ghost Fields}%
\label{sub:fadeev_popov_ghost_fields}

We write the Faddeev--Popov determinant as a functional integral over a new set of \emph{spinless} Grassmann fields $c, \overline{c}{}$
\begin{equation}
  \label{eq:22-1}
  \det(\frac{\delta G^{a}(x)}{\delta \alpha^{b}(y)}) = \det(\frac{1}{g} \partial^{\mu} D_{\mu}) = \int \pdd{c} \pdd{\overline{c}{}} \exp[-\int \dd[4]{x} \overline{c}{}^{a}(-\partial^{\mu} D_{\mu}^{ab}) c^{b}],
\end{equation}
which adds to the Lagrangian the term
\begin{equation}
  \mathscr{L}_{\text{gh}} = -\overline{c}{}^{a} \partial^{\mu} D_{\mu} ^{ab} c^{b}.
\end{equation}
Since this Lagrangian is the adjoint derivative, we see that $c, \overline{c}{}$ have to transform in the adjoint representation.
To give the correct identity, $c$ and $\overline{c}{}$ must be anticommuting fields that are scalars under Lorentz transformation, meaning that the quantum excitations of these fields have the wrong relation between spin and statistics to be physical particles.
Nevertheless, we can treat these excitations as additional particles, called \emph{Faddeev--Popov ghosts}, in the computation of Feynman diagrams.
We will see that this is a completely benign trick since the unphysical $c$-fields, which violate the spin-statistics theorem, will end up not contributing to any scattering amplitudes.

We can integrate by parts to have an equivalent Lagrangian
\begin{align}
  \mathscr{L}_{\text{gh}} &= \partial^{\mu} \overline{c}{}^{a} D^{ab}_{\mu} c^{b} \\
			  &= \partial^{\mu} \overline{c}{}^{a} \partial_{\mu} c^{a} - i g (\partial^{\mu} \overline{c}{}^{a}) A^{c}_{\mu} (T^{c} _{\text{adj}})^{ab} c^{b} \\
			  &= \partial^{\mu} \overline{c}{}^{a} \partial_{\mu} c^{a} - g f^{abc} A_{\mu} ^{c} (\partial^{\mu} \overline{c}{}^{a}) c^{b}
\end{align}
which gives the same action $S_{\text{gh}}$.
The first term gives the ghost propagator while the second term gives the interaction vertex between the $c$ ghosts and the gauge field $A$.
For $U(1)$, the structure constants $f^{abc} = 0$ vanish. So the ghosts decouple and are not needed.

\subsection{Gauge Fixing}%
\label{sub:gauge_fixing}

We want to rewrite the $\delta$-function as a \emph{gauge fixing} term in the action,
\begin{equation}
  \label{eq:22-2}
  \delta[G^{a}(x)] = \delta[\partial^{\mu} A_{\mu}^{a}(x) - w^{a}(x)] \to e^{-S_{\text{gf}}}.
\end{equation}
In order to do this, we want to multiply $Z$ by a Gaussian in $\omega$ with width $\xi$,
\begin{equation}
  Z \to \exp(-\int \dd[4]{x} \frac{\omega^2}{2 \xi}) Z
\end{equation}
and integrate over $\omega$. As $\xi \to 0$, this becomes a $\delta$-function.

The $\omega$-integral is then easy since the $\delta$-functional \eqref{eq:22-2} simply replaces $\omega^{a}(x)$ with $\partial^{\mu} A_{\mu}^{a}(x)$.
We then have the action
\begin{equation}
  Z \propto \int \pdd{A} \pdd{[c, \overline{c}{}]} \exp(-S_{YM} - S_{\text{gh}} - S_{\text{gf}}),
\end{equation}
with
\begin{equation}
  \label{eq:22-gf}
  S_{\text{gf}} = \int \dd[4]{x} \frac{1}{2 \xi} (\partial^{\mu} A_{\mu}^{a})^2.
\end{equation}

\section{Feynman Rules for Fermions, Gauge Bosons, and Ghosts}%
\label{sec:feynman_rules_for_fermions_and_gauge_bosons_and_ghosts}

Our Lagrangian has three different kinds of fields: gauge, fermion, and ghost
\begin{equation}
  \label{eq:l-nonab}
  \mathscr{L} = \frac{1}{4} (F^{a}_{\mu\nu})^2 + \overline{\psi}{} (\cancel{D} + m) \psi + \frac{1}{2 \xi} (\partial^{\mu} A_{\mu}^{a}) ^2 + \partial_{\mu} c^{a} \partial^{\mu} \overline{c}{}^{a} - g f^{abc} A_{\mu}^{c} \partial^{\mu} \overline{c}{}^{a} c^{b}.
\end{equation}
%where the index $a$ is summed over the generators of the gauge group $G$, and the fermion multiplet $\psi$ belongs to an irreducible representation $r$ of $G$.

The propagators are:
\begin{align}
  &\text{Fermions} & 
  \begin{gathered}
    \feynmandiagram[transform shape, scale=1][horizontal=a to b] {
      a [particle=\(i\)] -- [fermion, momentum=$p$] b [particle=\(j\)],
    };
  \end{gathered}
  &= \left( \frac{1}{i \cancel{p} + m} \right)^{ij} = \left( \frac{-i \cancel{p} + m}{p^2 + m^2} \right)^{ij} \\
  &\text{Gauge bosons} & 
  \begin{gathered}
    \feynmandiagram[transform shape, scale=1][horizontal=a to b] {
      a [particle=\(\nu \text{, } b\)] -- [gluon, momentum=$k$] b [particle=\(\mu \text{, } a\)],
    };
  \end{gathered}
  &= \frac{1}{k^2} \left( \delta^{\mu\nu} - (1 - \xi) \frac{k^{\mu} k^{\nu}}{k^2} \right)\delta^{ab} \\
  &\text{Ghosts} & \label{eq:gh-prop}
  \begin{gathered}
    \feynmandiagram[transform shape, scale=1][horizontal=a to b] {
      a [particle=\(a\)] -- [charged ghost, momentum=$q$] b [particle=\(b\)],
    };
  \end{gathered}
  &= \frac{\delta^{ab}}{q^2} \quad \text{(massless)}
\end{align}

The interactions come from writing out $D_{\mu}$ and $F_{\mu\nu}$:
\begin{equation}
  \label{eq:3-boson-vertex}
  \begin{gathered}
    \feynmandiagram[transform shape, scale=1][horizontal=a to b, small] {
      a [particle=\(\nu \text{, } b\)] -- [gluon, momentum=$p$] b,
      u [particle=\(\mu \text{, } a\)] -- [gluon, momentum=$k$] b,
      d [particle=\(\rho \text{, } c\)] -- [gluon, momentum=$q$] b,
    };
  \end{gathered}
  = -g f^{abc} \left[ \delta^{\mu\nu} (k - p)^{\rho} + \delta^{\nu\rho} (p - q)^{\mu} + \delta^{\rho\mu} (q - k)^{\nu} \right],
\end{equation}
from term $g f^{abc} (\partial_{\mu} A^{a}_{\nu}) A^{\mu, b} A^{\nu, c}$.
\begin{align}
  \begin{gathered}
    \feynmandiagram[transform shape, scale=1][horizontal=a to u, small] {
      a [particle=\(\nu \text{, } b\)] -- [gluon] b,
      u [particle=\(\mu \text{, } a\)] -- [gluon] b,
      d [particle=\(\rho \text{, } c\)] -- [gluon] b,
      e [particle=\(\sigma \text{, } d\)] -- [gluon] b,
    };
  \end{gathered}
  = 
  \begin{aligned}
      -g^2 \bigl[ &f^{eab} f^{ecd} (\delta^{\mu\rho} \delta^{\nu\sigma} - \delta^{\mu\sigma} \delta^{\nu\rho}) \\
    		&f^{ace} f^{bde} (\delta^{\mu\nu} \delta^{\rho\sigma} - \delta^{\mu\sigma} \delta^{\nu\rho}) \\
    		& f^{ade} f^{bce} (\delta^{\mu\nu} \delta^{\rho\sigma} - \delta^{\mu\rho} \delta^{\nu\sigma})
      \bigr],
  \end{aligned}
\end{align}
from the $\frac{1}{4} g^2 (f^{abc} A_{\mu}^{b} A_{\nu}^{c})(f^{ade} A^{\mu, d} A^{\nu, e})$ term.
Moreover, taking $T_{ij}$ to be in the fundamental representation, we have
\begin{equation}
  \begin{gathered}
    \feynmandiagram[transform shape, scale=1][horizontal=i to j, small] {
      i [particle=\(i\)] -- [fermion] v -- [fermion] j [particle=\(j\)],
      u [particle=\(\mu \text{, }a\)] -- [gluon] v,
    };
  \end{gathered}
  = ig \gamma^{\mu} T^{a}_{ij},
  \qquad
  \begin{gathered}
    \feynmandiagram[transform shape, scale=1][horizontal=i to j, small] {
      i [particle=\(b\)] -- [charged ghost] v -- [charged ghost, momentum=$p$] j [particle=\(a\)],
      u [particle=$\mu \text{, } c$] -- [gluon] v,
    };
  \end{gathered}
  = -g f^{abc} p^{\mu},
\end{equation}
where the momentum $p^{\mu}$ is the one associated with the outgoing ghost $\overline{c}{}$.
\begin{remark}
  All of these vertices involve the same coupling constant $g$.
  In fact, the coupling constants of all three nonlinear terms in the Yang--Mills Lagrangian must be equal in order to preserve the Ward identity and avoid the production of bosons with unphysical polarisation states.
  Conversely, the non-Abelian gauge symmetry guarantees that these couplings are equal, giving a consistent theory of physical vector particle interactions \cite[pp.~508]{peskin}.
\end{remark}


\section{One-Loop Renormalisation}%
\label{sec:one_loop_renormalisation}

\begin{leftbar}
  The Lagrangian \eqref{eq:l-nonab} of non-Abelian gauge theory contains no interactions of dimension higher than $4$ and is therefore renormalisable, meaning that the divergences can be removed by a finite number of counterterms.

  However, as in QED, the gauge symmetry implies restrictions on the divergences.
  In QED, the Ward identity implies the relation
  \begin{equation}
    \label{eq:16.57}
    q^{\mu} \left( 
      \begin{gathered}
        \feynmandiagram[transform shape, scale=1][small, horizontal=a to b, layered layout] {
          a -- [boson] b [blob] -- [boson] c,
        };
      \end{gathered}
    \right) = 0,
  \end{equation}
  which in turn implies that the photon self-energy diagrams have the structure
  \begin{equation}
    \label{eq:16.58}
    \begin{gathered}
      \feynmandiagram[transform shape, scale=1][horizontal=a to b, small, layered layout] {
        a --[boson] b [blob] -- [boson] c,
      };
    \end{gathered}
    = (q^2 \delta^{\mu\nu} - q^{\mu} q^{\nu}) \Pi(q^2).
  \end{equation}
  The only divergence possible is a logarithmically divergent contribution to $\Pi(q^2)$.
  In non-Abelian gauge theories, \eqref{eq:16.57} still holds and therefore the self-energy again has the Lorentz structure \eqref{eq:16.58}, although the contributions and cancellations ins this vacuum polarisation diagram are much more complex.
\end{leftbar}

\subsection{Vacuum Polarisation}%
\label{sub:vacuum_polarisation}

Our aim will be to show \eqref{eq:16.58} and calculate the gauge boson self-energy $\Pi(q^2)$. At one-loop order, we have $5$ diagrams contributing, all of which are truncated:
\begin{align}
  &\text{Fermion loop:} & \mathfrak{M}_F^{ab, \mu \nu} &= 
  \begin{gathered}
    \feynmandiagram[transform shape, scale=1][horizontal=a to b, layered layout] {
      a [particle=\(\mu \text{, } a\)] -- [gluon, insertion={[size=5pt]0.5}] b -- [half left, looseness=1, momentum=$p$, fermion] c -- [half left, looseness=1, momentum=$p - q$, fermion] b,
      c -- [gluon, insertion={[size=5pt]0.5}] d [particle=\(\nu \text{, } b\)],
    };
  \end{gathered} \label{eq:22-fermionloop} \\
  &\text{Gauge loops:} & \mathfrak{M}_3^{ab, \mu \nu} &= 
  \begin{gathered}
    \feynmandiagram[transform shape, scale=1][horizontal=a to b, layered layout] {
      a [particle=\(\mu \text{, } a\)] -- [gluon, insertion={[size=5pt]0.5}] b -- [half left, looseness=1, momentum=$p$, gluon] c -- [half left, looseness=1, momentum=$p - q$, gluon] b,
      c -- [gluon, insertion={[size=5pt]0.5}] d [particle=\(\nu \text{, } b\)],
    };
  \end{gathered} \label{eq:22-gaugeloop} \\
  & & \mathfrak{M}_4^{ab, \mu \nu} &= 
    \feynmandiagram[transform shape, scale=1][horizontal=a to b, layered layout] {
      a [particle=\(\mu \text{, } a\)] -- [gluon, insertion={[size=5pt]0.5}] b -- [loop, min distance=2cm, in=135, out=45, gluon] b,
      b -- [gluon, insertion={[size=5pt]0.5}] d [particle=\(\nu \text{, } b\)],
    }; \label{eq:22-gaugelooptwo}\\
  &\text{Ghost loop:} & \mathfrak{M}_{\text{gh}}^{ab, \mu \nu} &= 
  \begin{gathered}
    \feynmandiagram[transform shape, scale=1][horizontal=a to b, layered layout] {
      a [particle=\(\mu \text{, } a\)] -- [gluon, insertion={[size=5pt]0.5}] b -- [half left, looseness=1, momentum=$p$, ghost] c -- [half left, looseness=1, momentum=$p - q$, ghost] b,
      c -- [gluon, insertion={[size=5pt]0.5}] d [particle=\(\nu \text{, } b\)],
    };
  \end{gathered} \label{eq:22-ghostloop} \\
  &\text{Counterterm:} & \mathfrak{M}_{ct}^{ab, \mu \nu} &=
  \begin{gathered}
    \feynmandiagram[transform shape, scale=1][horizontal=a to b, layered layout] {
      a [particle=\(\mu \text{, } a\)] -- [gluon, insertion={[size=5pt]0.5}] b [square dot] -- [gluon, insertion={[size=5pt]0.5}] c [particle=\(\nu \text{, } b\)],
    };
  \end{gathered}
\end{align}

The fermion loop \eqref{eq:22-fermionloop} is as in QED: Using dimensional regularisation, we have
\begin{equation}
  \mathfrak{M}_F^{ab \mu \nu} = - \Tr(T^{a} T^{b}) (ig)^2 \mu^{\epsilon} \int \bdd[d]{p} \frac{\Tr[(-i \cancel{p} + m) \gamma^{\mu} (-i (\cancel{p} - \cancel{q}) \gamma^\nu)]}{[(p - q)^2 + m^2][p^2 + m^2]}.
\end{equation}
Now the generators of the fundamental representation of $SU(N)$ obey the trace relation \eqref{eq:t-f}
\begin{equation}
  \Tr(T^{a} T^{b}) = T_F \delta^{ab} = \frac{1}{2} \delta^{ab}.
\end{equation}
The rest is as in QED.
\begin{equation}
  \mathfrak{M}_{F}^{ab \mu \nu} = -\frac{1}{2} \delta^{ab} \left( q^2 \delta^{\mu\nu} - q^{\mu} q^{\nu} \right) \frac{g^2}{2 \pi^2} \int_0^1 \dd[]{x} x(1 - x) \left[ \frac{2}{\epsilon} - \gamma + \ln( \frac{4 \pi \mu^2}{\Delta}) \right]
\end{equation}
where the momentum term in parentheses is expected for massless gauge boson and
\begin{equation}
  \Delta = m^2 + q^2 x(1 - x).
\end{equation}
