% lecture notes by Umut Özer
% course: aqft
\lhead{Lecture 19: February 29}

This derivation can be extended for $n$-point functions. For example, 
\begin{equation}
  \partial^2_z \langle \phi(z) \phi(x) \phi(y) \rangle = - \delta^{(4)} (z - x) \langle \phi(y) \rangle - \delta^{(4)} (z - y) \langle \phi(x) \rangle.
\end{equation}

\subsection{Adding Interactions}%
\label{sub:adding_interactions}

Let us now add an interaction Lagrangian $\mathscr{L}_{\text{int}}[\phi]$ to the action:
\begin{equation}
  S[\phi] = \int \dd[4]{y} \left( -\frac{1}{2} \phi \partial^2 \phi + \mathscr{L}_{\text{int}}[\phi] \right).
\end{equation}
The classical equations of motion become $\partial^2 \phi = \mathscr{L}'_{\text{int}}[\phi]$.
We Taylor-expand by employing functional derivation
\begin{equation}
  \mathscr{L}_{\text{int}} [\phi + \epsilon] \approx \mathscr{L}_{\text{int}} [\phi] + \epsilon(x) \mathscr{L}'_{\text{int}}[\phi].
\end{equation}
Therefore, the addition of the potential $\mathscr{L}_{\text{int}}[\phi]$ contributes and additional term
\begin{equation}
  - \int \dd[4]{z} \epsilon(z) \mathscr{L}'_{\text{int}}[\phi(z)]
\end{equation}
to the bracket in the path integral \eqref{eq:18-2}.
Thus, \eqref{eq:18-3} is modified to 
\begin{equation}
  \int \dd[4]{z} \epsilon(z) \int \pdd{\phi} e^{-S} \left[ \phi(x) \partial^2_z \phi(z) + \delta^{(4)}(z - x) - \phi(x) \mathscr{L}'_{\text{int}}[\phi(z)] \right] = 0.
\end{equation}
Hence, the Schwinger--Dyson equation for the $2$-point function for the interacting theory is
\begin{equation}
  \label{eq:scalar-sd}
  \partial^2_z \langle \phi(z) \phi(x) \rangle = \langle \mathscr{L}_{\text{int}}'[\phi(z)] \, \phi(x) \rangle - \delta^{(4)} (z - x).
\end{equation}

If we have more field insertions, the Schwinger--Dyson equations add contact interactions, contracting the field on which the operator acts with all the other fields in the correlator. For example, with three fields:
\begin{equation}
  \partial^2_x \langle \phi(x) \phi(y) \phi(z) \rangle = \langle \mathscr{L}'_{\text{int}}[\phi(x)] \; \phi(y) \phi(z) \rangle - \delta^4 (x - z) \langle \phi(y) \rangle - \delta^4 (x - y) \phi(z).
\end{equation}
In this way, the complete set of Schwinger--Dyson equations can be derived. In general, for a massive scalar field \cite[Sec.~14.7]{schwartz}
\begin{multline}
  (\partial^2_x + m^2) \left\langle \phi(x) \phi(x_1) \dots \phi(x_n) \right\rangle = \left\langle \mathscr{L}'_{\text{int}}[\phi(x)] \phi(x_1) \dots \phi(x_n) \right\rangle \\
  - \sum_i \delta^4(x - x_i) \left\langle \phi(x_1) \dots \phi(x_{i-1}) \phi(x_{i+1}) \dots \phi(x_n) \right\rangle,
\end{multline}
where $\mathscr{L}'_{\text{int}}[\phi] = \frac{\delta }{\delta \phi} \mathscr{L}_{\text{int}}[\phi]$ is the variational derivative of the interaction Lagrangian, and we are using $\langle \dots \rangle$ as an abbreviation for $\bra{\Omega} T \left\{ \dots \right\} \ket{\Omega}$ for time-ordered matrix elements in the interacting vacuum.
\begin{leftbar}
  \begin{remark}
    These relations can be derived from the assumptions that the interacting quantum fields satisfy
    \begin{equation}
      (\partial^2 + m^2) \phi = \mathscr{L}'_{\text{int}}[\phi], \qquad [\phi(x), \partial_0 \phi(y)] = \delta^3(x - y).
    \end{equation}
  \end{remark}
\end{leftbar}
The Schwinger--Dyson equations should be interpreted as saying that the difference between the quantum and classical equations of motion is given by contact terms, which specify the quantum theory.

\subsection{Symmetries}%
\label{sub:symmetries}

As shown in Appendix \ref{apx:variational_calculus}, the variational derivatives of the action $S$ and the Lagrangian $\mathscr{L}$ are related via
\begin{equation}
  \frac{\delta S}{\delta \phi} = \frac{\delta \mathscr{L}}{\delta \phi} - \partial_{\mu} \left( \frac{\partial \mathscr{L}}{\partial (\partial_{\mu} \phi)} \right).
\end{equation}

For the action for a scalar Klein--Gordon field, this is
\begin{equation}
  \frac{\delta S}{\delta \phi} =  \mathscr{L}'_{\text{int}}[\phi] - \partial^2 \phi.
\end{equation}
Therefore, the Schwinger--Dyson equation \eqref{eq:scalar-sd} becomes
\begin{equation}
  \label{eq:sd-s}
  \left\langle \frac{\delta S}{\delta \phi(z)} \phi(x) \right\rangle = \delta^{(4)} (z - x).
\end{equation}

Suppose we have a symmetry, meaning that $\delta \mathscr{L} = 0$ under $\phi \to \phi + \epsilon$. We often have $\epsilon(x) = \alpha(x) \phi(x)$. Then, as shown in \eqref{eq:apx-cur} of Appendix \ref{apx:variational_calculus}, we have
\begin{equation}
  \partial_{\mu} j^{\mu} = - \frac{\delta S}{\delta \phi} \epsilon.
\end{equation}
This means that we have $\partial_{\mu} j^{\mu} = 0$ when the field equations are satisfied, which we expect from a Noether current. Together with \eqref{eq:sd-s}, this yields the \emph{Ward--Takahashi identity}
\begin{equation}
  \frac{\partial }{\partial z^{\mu}} \langle j^{\mu}(z) \phi(x) \rangle = - \delta^{(4)} (z - x) \langle \epsilon(x) \rangle.
\end{equation}

\section{Ward--Takahashi Identity}%
\label{sec:ward_takahashi_identity}

In the derivation of Noether's theorem, we perform a variation of the field that is also a global symmetry of the Lagrangian, which leads to the existence of a classically conserved current.
Performing a similar variation on the path integral and following the steps that led to the Schwinger--Dyson equations will produce a general and powerful relation among correlation functions known as the Ward--Takahashi identity. This not only implies the usual Ward identity and gauge invariance, but since it is non-perturbative it will also play an important role in the renormalisation of QED \cite[Sec.~14.8]{schwartz}.

\subsection{Schwinger--Dyson for Fermions}%
\label{sub:schwinger_dyson_for_fermions}

Consider $\mathscr{L} = \overline{\psi}{} \cancel{\partial} \psi + (\text{non-derivative terms})$.
Under a transformation
\begin{equation}
  \psi(x) \to e^{i \alpha(x)} \psi(x), \qquad \overline{\psi}{}(x) \to \overline{\psi}{} e^{-i \alpha(x)},
\end{equation}
we pick up an extra piece
\begin{equation}
  \overline{\psi}{} \cancel{\partial} \psi \to \overline{\psi}{}\cancel{\partial} \psi + i \overline{\psi}{} \gamma^{\mu} \psi \partial_{\mu} \alpha.
\end{equation}
We want to be able to compute the expectation value of our fields, so let us investigate the propagator $\langle \psi(x_1) \overline{\psi}{}(x_2) \rangle$.
We expand about small $\alpha(x)$ and follow the same steps as in Sec.~\ref{sec:schwinger_dyson_equations_and_scalars} to find that terms of order $O(\alpha)$ must vanish
\begin{multline}
  0 = \int \pdd{\psi} \pdd{\overline{\psi}{}} e^{-S} \left[ i \int \dd[4]{x} \overline{\psi}{}(x) \gamma^{\mu} \psi(x) \partial_{\mu} \alpha(x) \right] \psi(x_1) \overline{\psi}{}(x_2) \\
  + \int \pdd{\psi} \pdd{\overline{\psi}{}} e^{-S} \left[ i \alpha(x_1) \psi(x_1) \overline{\psi}{}(x_2) - i \alpha(x_2) \psi(x_1) \overline{\psi}{}(x_2) \right].
\end{multline}
Once again we integrate by parts and factor out the $\alpha(x)$. Requiring this to vanish for any $\alpha(x)$ gives the fermion Schwinger--Dyson equation associated with charge conservation
\begin{equation}
  \label{eq:sd-qed}
  \partial_{\mu} \langle j^{\mu}(x) \psi(x_1) \overline{\psi}{}(x_2) \rangle = \left[ - \delta^{(4)}(x - x_1) + \delta^{(4)}(x - x_2) \right] \langle \psi(x_1) \overline{\psi}{}(x_2) \rangle,
\end{equation}
where $j^{\mu}(x) = i \overline{\psi}{}(x) \gamma^{\mu} \psi(x)$ is the Noether current for QED. This is a non-perturbative relation between the correlation functions.
If has the same qualitative content as the other Schwinger--Dyson equations: the classical equations of motion, in this case $\partial_{\mu} j^{\mu} = 0$, hold within time-ordered correlation functions up to contact interactions.

\subsection{Ward--Takahashi Identity in QED}%
\label{sub:ward_takahashi_identity_in_qed}

The Schwinger--Dyson equation is going to be more useful for us in momentum space.
Let us Fourier transform to obtain ``off-shell'' amplitudes, where we are not imposing momentum conservation.
The Fourier transform of the matrix element of the current with fields is
\begin{equation}
  M^{\mu}(p, q_1, q_2) \coloneqq \int \dd[4]{x} \dd[4]{x_1} \dd[4]{x_2} e^{i p \cdot x} e^{i q_1 \cdot x_1} e^{-i q_2 \cdot x_2} \langle j^{\mu}(x) \psi(x_1) \overline{\psi}{}(x_2) \rangle = 
  \begin{gathered}
    \feynmandiagram[transform shape, scale=0.6][horizontal=a to b] {
      a -- [fermion, momentum=$q_1$] c [small, dot] -- [fermion, momentum'=$p_2$] b,
      u [particle=\(j^{\mu}\)] -- [boson, momentum=$p$, insertion={[size=5pt]0.5}] c,
    };
  \end{gathered},
\end{equation}
where we have chosen signs so that the momenta represent $j(p) + e^-(q_1) \to e^-(q_2)$.
Similarly, we define
\begin{equation}
  M(q_1, q_2) \coloneqq \int \dd[4]{x_1} \dd[4]{x_2} e^{i q_1 \cdot x_1} e^{-i q_2 \cdot x_2} \langle \psi(x_1) \overline{\psi}{}(x_2) \rangle =
  \begin{gathered}
    \feynmandiagram[transform shape, scale=1][horizontal=a to b, layered layout, small] {
      a -- [fermion, momentum=$q_1$] v[small, dot] -- [fermion, momentum=$q_2$] b,
    };
  \end{gathered},
\end{equation}
with signs to represent $e^-(q_1) \to e^-(q_2)$, so that
\begin{equation}
  M(q_1 + p, q_2) \coloneqq \int \dd[4]{x} \dd[4]{x_1} \dd[4]{x_2} e^{i q_1 \cdot x_1} e^{-i q_2 \cdot x_2} \delta^4(x - x_1) \langle \psi(x_1) \overline{\psi}{}(x_2) \rangle
\end{equation}
is the Fourier transform of the first term on the right of \eqref{eq:sd-qed}. The second term is similar and so the Fourier transform of the QED Schwinger--Dyson equation \eqref{eq:sd-qed} is
\begin{equation}
  i p_{\mu} M^{\mu}(p, q_1, q_2) = M(q_1 + p, q_2) - M(q_1, q_2 - p).
\end{equation}
This is known as a \emph{Ward--Takahashi} identity, and it can be represented diagrammatically as
\begin{equation}
  i p_{\mu} \biggl(
  \begin{gathered}
    \feynmandiagram[transform shape, scale=1][small,vertical=c to u] {
      a -- [fermion, edge label=$q_1$] c [small, dot] -- [fermion, edge label=$p_2$] b,
      u [particle=\(j^{\mu}\)] -- [boson, insertion={[size=5pt]0.5}, momentum=$p$] c,
    };
\end{gathered} \biggr) = 
  \begin{gathered}
    \feynmandiagram[transform shape, scale=1][horizontal=a to b, layered layout, small] {
      a -- [fermion, edge label=$q_1 + p$] v[small, dot] -- [fermion, edge label=$q_2$] b,
    };
  \end{gathered} \quad - \quad
  \begin{gathered}
    \feynmandiagram[transform shape, scale=1][horizontal=a to b, layered layout, small] {
      a -- [fermion, edge label=$q_1$] v[small, dot] -- [fermion, edge label=$q_2 - p$] b,
    };
  \end{gathered}.
\end{equation}
These are not Feynman diagrams for $S$-matrix elements, since the momenta are not on-shell.
Instead, they are Feynman diagrams for correlation functions, also sometime called \emph{off-shell $S$-matrix} elements.  The Feynman rules are the usual momentum space Feynman rules with the addition of propagators for external lines and without the external polarisation (i.e.~without removing the stuff that the LSZ formula removes).
Momentum is not necessarily conserved, which is why we can have $q_1 + p$ coming in and $q_2$ going out.
Moreover, we have at no point used any perturbation theory, so these hold non-perturbatively.

\subsection{Relation to Renormalisation}%
\label{sub:renormalisation}

In QED, we have
\begin{equation}
  \mathscr{L} = \frac{1}{4} Z_3 F^2 + Z_2 \overline{\psi}{} \cancel{\partial} \psi + Z_2 Z_m m \overline{\psi}{} \psi + Z_1 e \overline{\psi}{} \cancel{A} \psi.
\end{equation}
Our aim will be to prove that $Z_1 = Z_2$, as we assumed in the previous chapter.
Let 
\begin{equation}
  \mathcal{M}(q_1, q_2) = \bdelta^4(q_1 - q_2) G(\cancel{q_1}).
\end{equation}
Also, consider for now a massless electron $m = 0$, since we are not interested in the renormalisation of the mass term for now.
The renormalised propagator is then
\begin{equation}
  G(\cancel{p}) = \frac{1}{Z_2} \frac{1}{i \cancel{q}}.
\end{equation}

After one-loop renormalisation, we got the vertex function $ \Gamma^{(2)}{}^{\mu} \sim e Z_1 \gamma^{\mu}.$
We want to extend this vertex function to off-shell.
This is done by taking the Green's function and amputating (see notes)
\begin{equation}
  \Gamma^{\mu}(p, q_1, q_2) \bdelta^4(p + q_1 - q_2) = \int \dd[4]{x} \dd[4]{x_1} \dd[4]{x_2} e^{i p \cdot x} e^{i q_1 \cdot x_1} e^{-i q_2 \cdot x_2} G^{-1} (q_1) \langle j^{\mu}(x) \psi(x_1) \overline{\psi}{}(x_2) \rangle G^{-1} (p + q_1).
\end{equation}
Relating the inverse Green's function to what we had before in momentum space, we have
\begin{equation}
  G^{-1}(q_1) \mathcal{M}^{\mu}(p, q_1, q_2) ??
\end{equation}
