% lecture notes by Umut Özer
% course: aqft
\lhead{Lecture 10: February 08}

Let us decorate the original theory by adding $0$ subscripts
\begin{equation}
  \mathscr{L}_0 = \frac{1}{2} (\partial \phi_0)^2 + \frac{1}{2} m_0^2 \phi_0^2 + \frac{\lambda_0}{4!}\phi_0^4.
\end{equation}
Generically, contribution from loops spoil this and we have to rescale 
\begin{equation}
  \phi_0 = Z_{\phi}^{\frac{1}{2}} \phi.
\end{equation}
The rescaling $Z_\phi$  is determined by proper normalisation for LSZ, namely that 
\begin{equation}
  \bra{\Omega} \widetilde{\phi}(0) \ket{1} = 1.
\end{equation}
The Lagrangian becomes
\begin{equation}
  \mathscr{L}_0 = \frac{Z_\phi}{2} (\partial \phi)^2 + \frac{Z_\phi}{2}m_0^2 \phi^2 + \frac{Z_\phi^2\lambda}{4!} \phi^4.
\end{equation}

What we want to do is separate out two sets of terms.
We want to write the original Lagrangian in terms of the renormalised Lagrangian and any counter terms.
In particular, the renormalised Lagrangian should be of the same form as the original action, giving
\begin{align}
  \mathscr{L}_0 &= \mathscr{L}_{\text{ren}} + \mathscr{L}_{\text{ct}}, \label{eq:renct} \\
		&= \left[ \frac{1}{2} (\partial \phi)^2 + \frac{1}{2} m^2 \phi^2 + \frac{\lambda}{4!} \phi^4 \right] + 
		\left[ \frac{\delta Z_\phi}{2} (\partial \phi)^2 + \frac{\delta m^2}{2} \phi^2 + \frac{\delta \lambda}{4!} \phi^4 \right] \label{eq:10-star}.
\end{align}
Equating coefficients, we have
\begin{equation}
  \delta Z_\phi = Z_\phi - 1, \qquad \delta m^2 = Z_\phi m_0^2 - m^2, \qquad \delta \lambda = Z_\phi^2 \lambda_0 - \lambda .
\end{equation}
Of course the Feynman rules for the renormalised Lagrangian $\mathscr{L}_{\text{ren}}$ are the same as for the original Lagrangian $\mathscr{L}_0$ but with the coefficients, $m^2 \& \lambda$, interpreted as the renormalised ones.
We need to find the Feynman rules for the counter terms $\mathscr{L}_{\text{ct}}$:
\begin{equation}
  \begin{gathered}
    \feynmandiagram[transform shape, scale=1][horizontal=a to b, layered layout] {
      a -- b[square dot] -- c,
    }; \\
    -p^2 \delta Z_\phi
  \end{gathered}, \qquad
  \begin{gathered}
    \feynmandiagram[transform shape, scale=1][horizontal=a to b, layered layout] {
      a -- [insertion=1] b -- c,
    }; \\
    -\delta m^2
  \end{gathered}, \qquad
  \begin{gathered}
    \feynmandiagram[transform shape, scale=0.5][horizontal=a to b] {
      a -- v [dot, red] -- b,
      c -- v -- d,
    };
  \end{gathered}
  - \delta \lambda.
\end{equation}
Generally, $\delta Z_\phi, \delta m^2, \delta \lambda$ are $O(\hbar)$ at most.
Therefore, tree diagrams containing $\mathscr{L}_{\text{ct}}$ vertices are the same order as 1-loop diagrams from $\mathscr{L}_{\text{ren}}$.

The 2-point vertex is
\begin{equation}
  \widetilde{\Gamma}^{(2)}(p) = [\widetilde{G}^{(2)}(p)]^{-1} = p^2 + m^2 - \Pi(p^2).
\end{equation}
From $\mathscr{L}_{\text{ren}}$, we get $\Pi_1(p^2)$ at one loop just as in \eqref{eq:9-pi}, but with $m^2, \lambda$ interpreted as renormalised quantities of \eqref{eq:10-star}.
From $\mathscr{L}_{\text{ct}}$, 
\begin{equation}
  \Pi_{1, \text{ct}} = AMPUTATED DIAGRAMS = - \delta m^2 - p^2 \delta Z_\phi.
\end{equation}
Finite result for $\Pi_{i \text{ren}} = \Pi_1(p^2) + \Pi_{1, \text{ct}}$  is obtained by choosing
 \begin{equation}
  \delta Z_\phi = 0 \qquad \text{and} \qquad \delta m^2 = -\frac{\lambda}{32 \pi^2} \left[ \Lambda^2 - m^2 \ln(1 + \frac{\Lambda^2}{m^2}) \right].
\end{equation}
With this choice $\Pi_{1, \text{ren}} = 0$ .
The freedom to choose where to put finite points is called the \emph{renormalisation scheme}.
This arbitrary choice obviously looses some predictability, but we do it to make progress with the divergences.
The above scheme is called the \emph{on-shell} scheme, based on the requirement that
\begin{equation}
  \Pi_{\text{ren}}(-m^2_{\text{phys}}) \stackrel{!}{=} m^2 - m^2_{\text{phys}} \qquad \text{and} \qquad
  \left.\frac{\partial \Pi_{\text{ren}}}{\partial p^2}\right\rvert_{p^2 = -m^2_{\text{phys}}} = 0.
\end{equation}
The first term usually cancels out $m^2 - m^2_{\text{phys}} = 0$.

With the on-shell scheme, 
\begin{equation}
  \widetilde{G}^{(2)}(p) = \frac{1}{p^2+ m^2 - \Pi_{\text{ren}}(p^2)} = \frac{1}{p^2 + m^2_{\text{phys}}}.
\end{equation}
This has a pole at $p^2 = -m^2_{\text{phys}}$, which is the reason for the name \emph{on-shell}.
We are giving up on predicting the mass of the particle; we dial it in by hand after obtaining it from some experimental measurement.
The residue from the LSZ is $1$.

Next, choose $\delta \lambda$ to cancel the divergences in $\Gamma_1^{(4)}(0,0,0,0)$.
We have $\widetilde\Gamma_{1, \text{ct}}^{(4)} = -\delta \lambda$. Choosing
\begin{equation}
  \delta \lambda = \frac{3 \lambda^2}{32 \pi^2} \ln(\frac{\Lambda^2}{m^2} - 1).
\end{equation}
After a bit of algebra, this gives
\begin{align}
  \lambda_{\text{eff}} \coloneqq \widetilde{\Gamma}_{\text{ren}}^{(4)}(0,0,0,0) &= \lambda + \widetilde{\Gamma}_1^{(4)}(0,0,0,0) + \widetilde{\Gamma}_{1, \text{ct}}^{(4)} \\
										&= \lambda - \frac{3\lambda^2}{32 \pi^2} \left[ \ln(1 + \frac{m^2}{\Lambda^2}) + \frac{m^2}{m^2 + \Lambda^2} \right].
\end{align}
This is finite as $\Lambda \to \infty$ . (In fact, the infinite piece is chosen such that $\lambda_{\text{eff}} \to \lambda$.)

This term really acts like an effective coupling, which---at least to one-loop order---incorporates the quantum corrections.
If we did this calculation to all orders in a well defined theory, it would be solved.

\section{Dimensional Regularisation}%
\label{sec:dimensional_regularisation}

Since physical predictions come from tree diagrams with effective couplings like $\lambda_{\text{eff}}$, physical quantities should be independent of the cutoff $\Lambda$ in the end.
We will do this not with a hard momentum cutoff, but with a method that works more generally for gauge theories.
Hard momentum cutoff are not compatible with gauge invariance, so we want to come up with a different regularisation method.

A more mathematically elegant way to do this is given by \emph{dimensional regularisation}.
It is a trick we do order-by-order in perturbation theory.

In the context of perturbation theory, divergences can be regulated by working in $d = 4 - \epsilon$ dimensions.
Usually, we think about  $0 < \epsilon \ll 1$, but we have already seen that taking $\epsilon \to 1$ gave us useful results in  \emph{Statistical Field Theory}.

Let us start from the original Lagrangian (and drop the zero subscripts)
\begin{equation}
  S = \int \dd[d]{x} \left[ \frac{1}{2} (\partial \phi)^2 + \frac{1}{2} (m \phi)^2 + \frac{\lambda}{4!} \phi^4 \right].
\end{equation}

\subsection{Dimensional Analysis}%
\label{sub:dimensional_analysis}

Denote by square brackets $[\bullet]$ the mass dimension and let $\hbar = c = 1$.
Given that $[S] = 0$ and $[\partial] = [m] = -[x] = 1$, we find that the field has mass dimension
\begin{equation}
  [m^2 \phi^2] = 2[m] + 2[\phi] = d \implies [\phi] = \frac{d}{2} -1.
\end{equation}
From this we find that the coupling has dimension
\begin{equation}
  [\lambda \phi^4] = d \implies [\lambda] = 4 - d = \epsilon.
\end{equation}
Working with a dimensionful coupling is annoying, so we introduce an arbitrary renormalisation scale $\mu$ with mass dimension $[\mu] = 1$, which is not to be taken to $\infty$ like the cutoff was. 
Then we write
\begin{equation}
  \lambda = \mu^\epsilon g(\mu)
\end{equation}
such that $g$ is dimensionless. Of course, $g$ will depend on whichever choice we end up making for the renormalisation scale $\mu$.
