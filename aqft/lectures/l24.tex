% lecture notes by Umut Özer
% course: aqft
\lhead{Lecture 24: March 12}

\section{BRST Symmetry [Becchi--Rouet--Stora--Tyutin]}%
\label{sec:brst_symmetry_becchi_rouet_stora_tyutin}

The idea is that we have got the gauge-fixing term $\frac{1}{2\xi} (\partial_{\mu} A^{\mu})^2$. Since it fixes the gauge, it breaks gauge invariance.
But there is still a remnant global symmetry.

\subsection{Abelian Case}%
\label{sub:abelian_case}

It is easiest if we strip away the non-Abelian details for a moment and look at QED ($U(1)$ gauge theory).
The gauge-fixed Lagrangian is
\begin{equation}
  \mathscr{L} = \frac{1}{4} (F^{a})^2 + \overline{\psi}{}(\cancel{D} + m) \psi + \frac{1}{2 \xi} (\partial_{\mu} A^{\mu})^2 - \overline{c}{} \partial^2 c.
\end{equation}
Under the infinitesimal gauge transformation
\begin{subequations}
  \label{eq:24-1}
  \begin{align}
    \psi &\to \psi + i \alpha \psi \quad \text{or} \quad \delta \psi = i \alpha \psi \\
    A_{\mu} &\to A_{\mu} + \frac{1}{e} \partial_{\mu} \alpha \quad \text{or} \quad \delta A_{\mu} = \frac{1}{e} \partial_{\mu} \alpha,
  \end{align}
\end{subequations}
where $\alpha = \alpha(x)$. The term
\begin{equation}
  (\partial_{\mu} A^{\mu})^2 \to (\partial^{\mu} A_{\mu} + \frac{1}{e} \partial^2 \alpha)^2
\end{equation}
is only invariant if $\partial^2 \alpha(x) = 0$.
\begin{remark}
  The $\overline{c}{} \partial^2 c$ term in $\mathscr{L}$ leads to equations of motion $\partial^2 c = 0 = \partial^2 \overline{c}{}$.
\end{remark}
Consider restricting our gauge transformations to $\alpha(x) = \theta c(x)$, where $\theta$ is some $x$-independent Grassmann number.  So if we then also transform
\begin{equation}
  \overline{c}{}(x) \to \overline{c}{}(x) - \frac{\theta}{e \xi} \partial^{\mu} A_{\mu}(x), \quad \text{or} \quad \delta \overline{c}{} = -\frac{\theta}{e \xi} \partial^{\mu} A_{\mu}(x)
\end{equation}
along with the previous ones \eqref{eq:24-1}, then $\mathscr{L}$ is invariant.
These are the \emph{BRST transformations of QED}.

\subsection{Non-Abelian Case}%
\label{sub:non_abelian_case}

\begin{equation}
  \mathscr{L} = \frac{1}{4} (F^{a})^2 + \overline{\psi}{}(\cancel{D} + m) \psi + \frac{1}{2 \xi} (\partial^{\mu} A_{\mu}^{a})^2 - \overline{c}{} \partial^{\mu} D_{\mu} c.
\end{equation}
We now have to account for the fact that the derivative $D_{\mu} c^{a} = \partial_{\mu} c^{a} + g f^{abc} A_{\mu}^{b} c^{c}$.
Let $\alpha^{a}(x) = \theta c^{a}(x)$. The BRST transformations are
\begin{align}
  \delta \psi_i &= i \theta c^a T^{a}_{ij} \psi_{j} \\
  \delta A_{\mu}^{a} &= \frac{\theta}{g} D_{\mu}^{ab} c^{b} \\
  \delta \overline{c}{} &= -\frac{\theta}{g \xi} \partial^{\mu} A^{a}_{\mu}.
\end{align}
The $A_{\mu}$ in the covariant derivative contributes a term
\begin{equation}
  D_{\mu} c^{a} \to D_{\mu} c^{a} - \theta f^{abc} (D_{\mu} c^{b}) c^{c}.
\end{equation}
To cancel this, we need a transformation
\begin{equation}
  \delta c^{c} = -\frac{\theta}{2} f^{abc} c^{b} c^{c}
\end{equation}
to have $\mathscr{L}$ invariant.
Thus we have a remnant global symmetry (BRST).

Sometimes it is convenient to introduce an auxiliary (non-dynamical) scalar field $B^{a}(x)$ (Nakanishi--Lautrup).
\begin{equation}
  \mathscr{L} = \frac{1}{4} (F^{a})^2 + \overline{\psi}{} (\cancel{D} + m) \psi - \frac{\xi}{2} (B^{a})^2 + B^{a} \partial^{\mu} A^{a}_{\mu} - \overline{c}{} \partial^{\mu} D_{\mu} c.
\end{equation}
We recover the original $\mathscr{L}$ by completing the square and integrating over $\pdd{B}$.
Now
\begin{align}
  \delta \overline{c}{}^{a} &= \theta B^{a}. \\
  \delta B^{a} &= 0.
\end{align}

\subsection{BRST Cohomology}%
\label{sub:brst_cohomology}

You can show that two successive transformations leave the fields invariant. That is, let $Q = Q_{\text{BRST}}$ be an operator which gives the transformed fields:
\begin{align}
  \psi &\to \psi + \theta Q \psi \\
  A_{\mu} &\to A_{\mu} + \theta Q A,
\end{align}
with $Q^2  =0$ for all fields.
In other words, the BRST operator is \emph{nilpotent}.

\begin{leftbar}
  The following is non-examinable, proofs are out of scope for this course.
\end{leftbar}

A nilpotent operator $Q$ divides the Hilbert space $\mathcal{H}$ into subspaces:
\begin{description}
  \item[closed states:] those states annihilated by $Q$
    \begin{equation}
      Q \ket{\Psi} = 0 \implies \ket{\Psi} \in \mathcal{H}_{\text{closed}}
    \end{equation}
  \item[exact states:] states in the image of $Q$
    \begin{equation}
      \exists \ket{\Phi} \in \mathcal{H} \text{ s.t. } \ket{\Psi} = Q \ket{\Phi} \implies \ket{\Psi} \in \mathcal{H}_{\text{exact}}
    \end{equation}
\end{description}
Note that $\mathcal{H}_{\text{exact}} \subset \mathcal{H}_{\text{closed}}$: If $\ket{\Psi} \in \mathcal{H}_{\text{exact}}$, then $\exists \ket{\Phi} \in \mathcal{H}$ such that
\begin{equation}
  Q \ket{\Psi} = Q^2 \ket{\Phi} \stackrel{Q^2 = 0}{=} 0 \implies \ket{\Psi} \in \mathcal{H}_{\text{closed}}.
\end{equation}
Physical states are those in the quotient (or moduli) space
\begin{equation}
  \mathcal{H}_{\text{phys}} \coloneqq \frac{\mathcal{H}_{\text{closed}}}{\mathcal{H}_{\text{exact}}}.
\end{equation}
This is called \emph{BRST cohomology}.

It turns out that only the 2 transverse polarisations belong to $\mathcal{H}_{\text{phys}}$.

