% lecture notes by Umut Özer
% course: aqft
\lhead{Lecture 2: January 21}

We may analytically continue this to imaginary time.
Let $\tau = i t$. In terms of this imaginary time, we have
 \begin{equation}
  \bra{x} e^{- \hat{H} \tau / \hbar} \ket{x_0} = \int \pdd{x} e^{-S / \hbar}.
\end{equation}
Mathematically, it makes these integrals much more well-defined, clearly convergent.
Here the least-action principle is really evident from  the $\hbar = 0$ argument.
We also see the connection to statistical physics, interpreting  $e^{-S / \hbar}$  as the Boltzmann factor $e^{-\beta H}$ .

Quantum mechanics is quantum field theory in $0 + 1$  dimensions.
We are treating space differently from time: $\hat{x}(t)$  is a field, whereas $t$  is a variable.
However, Lorentz invariance forces us to put $x$  and $t$  on the same footing.
In QFT, we solve this problem by demoting $x$ from a field to another label. We then talk about fields $\phi(x, t)$ and want to know about the behaviour of these in all of spacetime.
\begin{leftbar}
  String theory gives another ansatz to this problem by promoting again, instead of demoting operators to labels.
\end{leftbar}

\section{Integrals and their Diagrammatic Expansions}%
\label{sec:integrals_and_their_diagrammatic_expansions}

In QFT, we are interested in correlation functions.
The following discussion will be very similar to what we have seen in the \emph{Statistical Field Theory} course in Michaelmas term, also no knowledge from that course will be assumed here.

For simplicity, consider a $0$ -dimensional field $\varphi \in \mathbb{R}$. As if we are in imaginary time, let
\begin{equation}
  Z = \int_{\mathbb{R}} \dd[]{\varphi} e^{- \frac{S(\varphi)}{\hbar}}.
\end{equation} 
Assume that the action $S(\varphi)$  is an even polynomial and  $S(\varphi) \to \infty$  as $\varphi \to \pm \infty$ .

We will be interested in expectation values
\begin{equation}
  \expectationvalue{f} = \frac{1}{Z} \int \dd[]{\varphi} f(\varphi) e^{-S / \hbar}.
\end{equation}
Again, assume $f$ does not grow too fast as $\varphi \to \pm \infty$. Usually, $f$ is polynomial in $\varphi$.

\subsection{Free Theory}%
\label{sub:free_theory}

Say we have $N$ scalar fields (in $0+1$ dimensions we should really just say `variables') $\varphi_{a}$ with $a = 1, \dots, N$, with action
\begin{equation}
  S_0 (\varphi) = \frac{1}{2} M_{ab} \varphi_{a} \varphi_{b} = \frac{1}{2} \varphi^T M \varphi,
\end{equation}
where $M$ is an $N \times N$ symmetric, positive definite ($\det M > 0$) matrix.

We can diagonalise this. There exists some orthogonal $P$ such that $M = P \Lambda P^T$, where $\Lambda$ is diagonal.
Let $\chi = P^T \varphi$. Then the free partition function is
\begin{align}
  Z_0 &= \int \dd[N]{\varphi} \exp(-\frac{1}{2 \hbar} \varphi^T M \varphi) \\
      &= \int \dd[N]{\chi} \exp(- \frac{1}{2 \hbar} \chi^T \Lambda \chi) \\
      &= \prod_{c=1}^N \int \dd[]{\chi_c} e^{-\frac{\lambda_c}{2 \hbar} \chi^2} = \sqrt{\frac{(2 \pi \hbar)^N}{\det M}}
\end{align}

We want to get from the partition function to correlation functions.
We can do this by introducing an $N$-component vector of external sources $J$ to the action
\begin{equation}
  S_0 (\varphi) \to S_0 + J^T \varphi.
\end{equation}
The partition function is then
\begin{equation}
  Z_0(J) = \int \dd[N]{\varphi} \exp(-\frac{1}{2\hbar} \varphi^T M \varphi - \frac{1}{\hbar}) J^T \varphi.
\end{equation}
Similar to solving an ordinary Gaussian integral, we complete the square by writing $\widetilde{\varphi} = \varphi + M^{-1} J$. One can then solve this integral to be
\begin{equation}
  \label{eq:generating-function}
  Z_0(J) = Z_0(0) \exp(\frac{1}{2\hbar} J^T M^{-1} J).
\end{equation}
This is called the \emph{generating function}\footnote{When we go to higher dimensions, where $J = J(x)$ this will be a generating functional $Z[J(x)]$.}.
Correlation functions are obtained from differentiating with respect to the auxiliary sources $J$ and evaluating the whole expression at $J = 0$:
\begin{align}
  \expectationvalue{\varphi_a \varphi_b} &= \frac{1}{Z_0(0)} \left.\int \dd[N]{\varphi} \varphi_{a} \varphi_{b} \exp(-\frac{1}{2\hbar} \varphi^T M \varphi - \frac{1}{\hbar} J^T \varphi)\right\rvert_{J = 0}. \\
					 &= \frac{1}{Z_0(0)} \left. \int \dd[N]{\varphi}  \left( -\hbar \frac{\partial }{\partial J_a} \right) \left(-\hbar \frac{\partial }{\partial J_b}\right) \exp( \dots )  \right\rvert_{J = 0} \\
					 &= \frac{1}{Z_0(0)} \left( -\hbar \frac{\partial }{\partial J_a} \right) \left( -\hbar \frac{\partial }{\partial J_b} \right) Z_0(J) \rvert_{J=0} \\
					 &= \hbar \, (M^{-1})_{ab} \coloneqq
					 \feynmandiagram[inline=(a.base), horizontal=a to b] {
					   a [particle=\(a\)] -- b [particle=\(b\)],
					 };
\end{align}
Connecting this to the \emph{Quantum Field Theory} course, we identify this as the \emph{free propagator}.

More generally, let $l(\varphi)$ be a \emph{linear} combination of $\varphi_a$. 
 \begin{equation}
  l(\varphi) = \sum_{a = 1}^N l_a \varphi_a, \qquad l_a \in \mathbb{R}.
\end{equation}
Then the steps above are equivalent to swapping $l(\varphi)$  for $l(- \hbar \frac{\partial}{\partial J}) = -\hbar \sum_a l_a \frac{\partial }{\partial J_a}$ .

The correlation function can again be evaluated explicitly by the introduction of an auxiliary current $J$:
\begin{align}
  \expectationvalue{l^{(1)}(\varphi) \cdots l^{(p)} (\varphi)} &= \frac{1}{Z_0(0)} \left.\int \dd[N]{\varphi} \prod_{i = 1}^p l^{(i)}(\varphi) e^{-\frac{1}{2\hbar} \varphi^T M \varphi - \frac{1}{\hbar} J^T \varphi} \right\rvert_{J = 0}. \\
							       &= (-\hbar)^p \prod_{i = 1}^p l^{(i)}(\frac{\partial }{\partial J}) \left. e^{\frac{1}{2\hbar} J^T M^{-1} J} \right\rvert_{J=0}
\end{align}

In other words, if $p$  is odd, the integrand is odd in some $\varphi_a$  and the integral over $\varphi_a \in (- \infty, + \infty )$   vanishes.
For $p = 2 k$, the terms which are non-zero as $J \to 0$ have the following form.
We need half the derivatives to bring down components of  $m^{-1} J$  and half to remove the $J$ -dependence from those terms that earlier derivatives brought down.
As such, we get exactly $k$  factors of $M^{-1}$ .
This is Wick's theorem.

\begin{example}[4-point function]
  Consider the $4$-point correlation function. One can check that the above result gives
  \begin{align}
    \expectationvalue{\varphi_a \varphi_b \varphi_c \varphi_d} &= \hbar^2 \left[ (M^{-1})_{ab} (M^{-1})_{cd}
    + (M^{-1})_{ac} (M^{-1})_{bd}
  + (M^{-1})_{ad} (M^{-1})_{bc} \right] \\
  &= 
  \begin{gathered}
    \begin{tikzpicture}
      \begin{feynman}[transform shape, scale=0.35]
	\diagram[horizontal=a to b] {
	  a [particle=\(a\)] 
	  -- [draw=none] b [particle=\(b\)]
	  -- [draw=none] d [particle=\(d\)]
	  -- [draw=none] c [particle=\(c\)]
	  -- [draw=none] a,
	};
	\diagram* {
	  (a) -- (c),
	  (b) -- (d),
	};
      \end{feynman}
    \end{tikzpicture}
  \end{gathered}
  +
  \begin{gathered}
    \begin{tikzpicture}
      \begin{feynman}[transform shape, scale=0.35]
	\diagram[horizontal=a to b] {
	  a [particle=\(a\)] 
	  -- [draw=none] b [particle=\(b\)]
	  -- [draw=none] d [particle=\(d\)]
	  -- [draw=none] c [particle=\(c\)]
	  -- [draw=none] a,
	};
	\diagram* {
	  (a) -- (b),
	  (c) -- (d),
	};
      \end{feynman}
    \end{tikzpicture}
  \end{gathered}
  +
  \begin{gathered}
    \begin{tikzpicture}
      \begin{feynman}[transform shape, scale=0.35]
	\diagram[horizontal=a to b] {
	  a [particle=\(a\)] 
	  -- [draw=none] b [particle=\(b\)]
	  -- [draw=none] c [particle=\(d\)]
	  -- [draw=none] d [particle=\(c\)]
	  -- [draw=none] a,
	};
	\diagram* {
	  (a) -- (c),
	  (b) -- (d),
	};
      \end{feynman}
    \end{tikzpicture}
  \end{gathered}
  \end{align}
  We end up with three terms, one for each way of grouping the 4 fields into pairs.
\end{example}

In general, for $\expectationvalue{\varphi_{1} \dots \varphi_{2k}}$, the number of terms is the number of distinct ways of pairing the $2k$ fields. This is $(2k + 1)!! = (2k)! / (2^k k!)$; the number of permutations of $2k$ fields is $(2k)!$, but we have to divide this by the $2^k$ permutations within the pairs and the $k!$ ways of rearranging the pairs.
\begin{remark}
  For complex fields, $M$ is Hermitian but not symmetric anymore. In that case, the order of indices of $M^{-1}$ is important. We keep track of this by drawing the propagator with a directed line
  \begin{equation}
    \langle \phi_a \phi_b^* \rangle = \hbar\, (M^{-1})_{ab} \coloneqq
    \feynmandiagram[inline=(a.base), horizontal=a to b] {
      a [particle=\(a\)] -- [fermion] b [particle=\(b\)],
    };
  \end{equation}
\end{remark}
