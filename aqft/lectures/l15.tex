% lecture notes by Umut Özer
% course: aqft
\lhead{Lecture 15: February 20}

\section{Continuum Limit and Renormalisability}%
\label{sec:continuum_limit_and_renormalisability}

So far we have integrated out the high-momentum modes from $\Lambda_0$ down to $\Lambda$ and obtained a family of effective actions
\begin{equation}
  e^{-S_\Lambda^{\text{eff}}[\phi^-]} = \int_\Lambda^{\Lambda_0} \pdd{\phi^+} e^{-S_{\Lambda_0} [\phi^- + \phi^+]}.
\end{equation}
Requiring that we get the same physics out of this gave us the $\beta$-functions.
\begin{equation}
  \beta_i (\{g_j\}) = (d_i - d)g_i + \Lambda \left( \dv{\Lambda} g_i \right) (\{g_j\}).
\end{equation}

We want to take the \emph{continuum limit}\footnote{This terminology is a bit more natural in the context of a lattice theory in condensed matter systems, where $\Lambda^{-1}$ is the lattice spacing.} $\Lambda_0 \to \infty$.
Renormalisability is the sensitivity to initial couplings $g_{0 i}$.

\begin{figure}[tbhp]
  \centering
  \def\svgwidth{0.4\columnwidth}
  \input{lectures/l15f1.pdf_tex}
  \caption{}
  \label{fig:l15f1}
\end{figure}

Cases:
\begin{enumerate}[(i)]
  \item The theory has only irrelevant couplings.
    Then $g_{0 i}$ must lie in $C$.
    The $g_i(\Lambda)$ flow to the fixed point $g_i^*$ irrespective of initial conditions $g_{0 i}$.
    Therefore, the limit of $S_\Lambda^{\text{eff}}$ as $\Lambda_0 \to \infty$ exists and gives a scale-invariant theory.
    We have $g_i(\Lambda) = g^*_i$.
    No renormalisation conditions are needed.
    An example of this is given by $\mathcal{N} = 4$ Super-Yang--Mills (SYM) in $d = 4$ dimensions.
  \item There is at least $1$ relevant (or marginal) direction.
    If $\Lambda_0 \to \infty$ for fixed $g_{0i}$, the flow leaves the vicinity of the fixed point (physics no longer universal).
    Therefore, we are sensitive to initial coupling (corresponds to relevant (or marginal) operators).
    Solution: Tune initial couplings to be closer to $C$ as $\Lambda_0 \to \infty$.
    Then the flow is slower and $\Lambda_0 / \Lambda$ can be made larger.
    Renormalisation conditions are needed to fix $g_{i0}$ for relevant directions (also marginal).
    Having a finite number of conditions gives a \emph{renormalisable} theory.
    Examples of this include Yang--Mills, QCD, \dots, which are asymptotically free ($\beta(g) < 0$).
  \item Irrelevant operators need to be tuned for correct physics.
    An example of this is Fermi theory with an interaction term $(\overline{\psi}{}_i \Gamma \psi_j) (\overline{\psi}{}_k \Gamma \psi_l)$ and four-point coupling $G_F \sim g^2 / M_W^2$.
    
    Then we cannot allow couplings to flow into the fixed point. The $\Lambda_0 \to \infty$ limit cannot be taken.
    In principle, an infinite number of initial conditions are needed.
    We call this a \emph{nonrenormalisable} theory. However, this is not entirely useless, since we still have some universality. In practice, there is usually some principle by which we can order the importance of irrelevant operators.
\end{enumerate}

As we will see, $\phi^4$ and QED are \emph{perturbatively renormalisable}.
Beyond perturbation theory, one has to do more work to show that these couplings are marginally irrelevant.
The couplings of $\phi^4$ and, as we will see, QED are marginally irrelevant. We still have a finite number of renormalisation conditions.
In dimensional regularisation, we have $g(\mu) \propto [\ln \frac{\Lambda_{\phi^4}}{\mu}]^{-1}$, which is called a \emph{Landau pole} at high scales $\mu \simeq \Lambda_{\phi^4}$.
Logarithmic running is so slow that in practice we never need to worry---we do not get close to $\Lambda_{\phi^4}$.

\chapter{Quantum Electrodynamics}%
\label{cha:quantum_electrodynamics}

We will work in Euclidean spacetime with action
\begin{equation}
  S[\psi, \overline{\psi}{}, A] = \int \dd[4]{x} \left[ \frac{1}{4} F_{\mu\nu} F^{\mu\nu} + \overline{\psi}{}(\cancel{D} + m) \psi \right],
\end{equation}
where the covariant derivative\footnote{Many textbooks use the convention $\cancel{D} = \gamma^{\mu} ( \partial_{\mu} -i e A_{\mu})$.} $\cancel{D} = \gamma^{\mu} (\partial_{\mu} + i e A_{\mu})$ and $\psi, \overline{\psi}{}$ are $4$-component Grassmann variables. Moreover, $F_{\mu\nu} = \partial_{\mu} A_{\nu} - \partial_{\nu} A_{\mu}$ and the partition function is
\begin{equation}
  \mathcal{Z} = \int \pdd{\psi} \pdd{\overline{\psi}{}} \pdd{A} e^{-S[\psi, \overline{\psi}{}, A]}.
\end{equation}
In Euclidean spacetime, $\{\gamma_{\mu}, \gamma_{\nu}\} = 2 \delta_{\mu\nu}$ and $\gamma_5 = \gamma_1 \gamma_2 \gamma_3 \gamma_4$. We use the convention $\gamma_{\mu}^{\dagger} = \gamma_{\mu}$.
\begin{example}[]
  One may use the representation
  \begin{equation}
    \gamma_j = 
    \begin{pmatrix}
     0 & -i \sigma_j \\
     i \sigma_j & 0 \\
    \end{pmatrix}, \qquad 
    \gamma_4  =
    \begin{pmatrix}
     1 & 0 \\
     0 & -1 \\
    \end{pmatrix} \quad \text{or} \quad
    \begin{pmatrix}
     0 & 1 \\
     1 & 0 \\
    \end{pmatrix}.
  \end{equation}
\end{example}

\section{QED Feynman Rules}%
\label{sec:qed_feynman_rules}

\subsection*{Electron Propagator}%

Let us first find the Feynman propagator for the fermionic electrons $\psi$. We go to momentum space with Fourier convention
\begin{equation}
  \psi(x) = \int \bdd[4]{p} e^{i p \cdot x} \psi(p),
\end{equation}
where, instead of using another symbol, we distinguish the field $\psi(x)$ from its modes $\psi(p)$ by keeping the argument explicit.
The free part of the fermion action is
\begin{equation}
  \label{eq:15-f}
  S_f[\psi, \overline{\psi}{}] = \int \bdd[4]{p} \overline{\psi} (-p) (i \cancel{p} + m) \psi(p).
\end{equation}
Introducing fermionic charges $\eta, \overline{\eta}{}$, the (momentum space) generating functional is
\begin{align}
  \mathcal{Z} [\eta, \overline{\eta}{}] &= \int \pdd{\psi} \pdd{\overline{\psi}{}} \exp{ - \int \bdd[4]{p} \left[ \overline{\psi}{}(i \cancel{p} + m) \psi - \overline{\eta}{} \psi + \overline{\psi}{} \eta \right] } \\
					&= \mathcal{Z}[0, 0] e^{-\int \overline{\eta}{}(i \cancel{p} + m)^{-1} \eta}.
\end{align}
The free propagator is therefore 
\begin{equation}
  G_F(p) = \left.\frac{\delta^2 \mathcal{Z}}{\delta \overline{\eta}{} \delta \eta}\right\rvert_{\eta, \overline{\eta}{} = 0} = \frac{1}{i \cancel{p} + m}.
\end{equation}
We see that we could have just read off the propagator to be the inverse of the operator acting on the quadratic part of the Fourier-transformed action \eqref{eq:15-f}.

\subsection*{Photon Propagator}%

In the same way, the electromagnetic photon propagator is obtained from the Fourier-integral representation of the associated action
\begin{align}
  S_{EM} [A] &= \frac{1}{4} \int \dd[4]{x} F_{\mu\nu} F^{\mu\nu} \\
	     &= -\frac{1}{2} \int \dd[4]{x} A_{\mu}(x) [\delta^{\mu\nu} \partial^2 - \partial^{\mu} \partial^{\nu}] A_{\nu}(x) \\
	  &= \frac{1}{2} \int \bdd[4]{k} A_{\mu}(-k) [k^2 \delta^{\mu\nu} - k^{\mu} k^{\nu}] A_{\nu}(k),
	  \label{eq:15-ft}
\end{align}
where we integrated by parts in the second line after using $F_{\mu\nu} F^{\mu\nu} = 2 [\partial_{\mu} A_{\nu} \partial^{\mu} A^{\nu} - \partial_{\mu} A_{\nu} \partial^{\nu} A^{\mu}]$.
In going to the third sign, we pick up another minus sign due to the two factors of $i$ brought down by the derivatives.
From this we can read off the EM propagator
\begin{equation}
  \label{eq:15-prop}
  D^{\mu\nu}(k) = \frac{1}{k^2} (\delta^{\mu\nu} - \frac{k^{\mu} k^{\nu}}{k^2}) = 
  \begin{gathered}
    \feynmandiagram[transform shape, scale=1][horizontal=a to b] {
      a [particle=\(\mu\)] -- [boson, momentum=$k$] b [particle=\(\nu\)],
    };
  \end{gathered}.
\end{equation}
\begin{leftbar}
  \begin{remark}
    It turns out that there is a subtlety here that was brushed under the rug in lectures:
    As explained above, we should really look at the \emph{inverse} of the operator $k^2 \delta^{\mu\nu} - k^{\mu} k^{\nu}$ acting on the quadratic part of the Fourier-transformed action \eqref{eq:15-ft}.
    This means we want $D^{\mu\nu}(k)$ such that
    \begin{equation}
      \label{eq:15-dinv}
      (k^2 \delta_{\mu\nu} - k_{\mu} k_{\nu}) D^{\nu\rho}(k) = \delta\indices{_{\mu}^{\rho}}.
    \end{equation}
    However, this equation has no solution since the $4 \times 4$ matrix $(k^2 \delta_{\mu\nu} - k_{\mu} k_{\nu})$ is singular \cite[Sec.~9.4]{peskin}.
    Equivalently, the expression \eqref{eq:15-ft} vanishes whenever $A_{\mu}(k) = k_{\mu} \alpha(k)$ for any scalar function $\alpha(k)$. For all these functions, the integrand of $\int \pdd{A} e^{-S[A]}$ is $1$ and therefore the path integral is divergent (there is no Gaussian damping).
    In fact, this difficulty arises due to gauge invariance: the functional integral is divergent because we are redundantly integrating over a continuous infinity of physically equivalent field configurations, with the troublesome modes being those for which $A_{\mu}(x) = \frac{1}{e} \partial_{\mu} \alpha(x)$, i.e.~those that are gauge-equivalent to $A_{\mu}(x) = 0$.
    This redundancy can be eliminated by \emph{Faddeev--Popov gauge fixing}. To not be repetitive, this is explained for the more general non-Abelian case in Sec.~\ref{sec:fadeev_popov_gauge_fixing}. The upshot of this gauge-fixing procedure will be that we obtain a sensible photon propagator by including an extra term $-(\partial^{\mu} A_{\mu})^2 / 2 \xi$ \eqref{eq:22-gf} to the Lagrangian, so that \eqref{eq:15-dinv} becomes
    \begin{equation}
      [k^2 \delta_{\mu\nu} - (1-\frac{1}{\xi} k_{\mu} k_{\nu})] D^{\nu\rho}(k) = \delta\indices{_{\mu}^{\rho}},
    \end{equation}
    which is indeed invertible and has the solution
    \begin{equation}
      D^{\mu\nu}(k) = \frac{1}{k^2} \left( \delta^{\mu\nu} - (1-\xi) \frac{k^{\mu} k^{\nu}}{k^2} \right).
    \end{equation}
    When making computations, we may choose a specific value for $\xi$. In \eqref{eq:15-prop}, we used $\xi = 0$, which is called \emph{Landau gauge}. The \emph{Feynman gauge} $\xi = 1$ is also commonly used, and sometimes the \emph{Yennie gauge} $\xi = 3$ makes an appearance.
  \end{remark}
\end{leftbar}

\subsection*{QED Vertex}%

The vertex between a fermion-antifermion pair $\overline{\psi}{}, \psi$ and a photon $A_{\mu}$ is
\begin{equation}
  \begin{gathered}
    \feynmandiagram[transform shape, scale=1][small, horizontal=a to b, small] {
      a -- [fermion] v [small, dot] -- [fermion] b,
      c [particle=\(\mu\)] -- [boson] v, 
    };
  \end{gathered}
  = -i e \gamma^{\mu}
\end{equation}
which corresponds to the interaction Lagrangian $\mathscr{L} = \overline{\psi}{} (ie \gamma^{\mu} A_{\mu}) \psi$ term, which comes from expanding out the covariant derivative term $\overline{\psi}{}\cancel{D} \psi$.
