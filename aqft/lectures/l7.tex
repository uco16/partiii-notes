% lecture notes by Umut Özer
% course: aqft
\lhead{Lecture 7: February 01}

\chapter{LSZ Reduction Formula}%
\label{cha:lsz_reduction_formula}

We will now start to move beyond $d = 0$ and transition to using path integrals.
This transitory topic will also allow us to have a glimpse at renormalisation.
We will have in mind a weakly interacting theory, so that we can tell the story the way that Sredniki does. However, the results that we will show can also be rigorously proved non-perturbatively, which usually happens in the final chapters of quantum field theory textbooks.

We will connect scattering amplitudes to correlation functions.
\section{\texorpdfstring{$2 \to 2$ }{}Scattering}%
\label{sec:$2 to 2$ _scattering}

Consider a free scalar field in $3 + 1$ dimensions, built out of plane waves
\begin{equation}
  \phi(x) = \int \frac{\bdd[3]{k}}{2E_{\vb{k}}} \left[ a(\vb{k}) e^{-i k \cdot x} + a^{\dagger} (\vb{k}) e^{i k \cdot x} \right],
\end{equation}
where we are using the mostly minus metric in Minkowski space
\begin{equation}
  k \cdot x = E t - \vb{k} \cdot \vb{x},
\end{equation}
and the relativistic normalisation for $a (\vb{k})$ , changing the factor from $\sqrt{2 E} \to 2 E$  as compared to the conventions from our \emph{Quantum Field Theory} lectures.
In a free theory, where we do not need to do any perturbative expansion, we set $\hbar = c = 1$.
Look at $\int \dd[3]{x} e^{i k \cdot x} \phi(x)$  and $\int \dd[3]{x} e^{i k \cdot x} \partial_0 \phi(x)$. These have
\begin{align}
  a(\vb{k}) &= \int \dd[3]{x} e^{i k \cdot x} \left[ i \partial_0 \phi(x) + E \phi(x) \right] \\
  a^{\dagger}(\vb{k}) &= \int \dd[3]{x} e^{-i k \cdot x} \left[ -i \partial_0 \phi(x) + E \phi(x) \right].
\end{align}
Let the initial state for the free theory be a one-particle state
\begin{equation}
  \ket{\vb{k}} = a^{\dagger}(\vb{k}) \ket{\Omega},
\end{equation}
where $\ket{\Omega}$ satisfies $a(\vb{k}) \ket{\Omega} = 0$ for all $\vb{k}$ and is normalised to $\bra{\omega}\ket{\Omega} = 1$.
The one-particle states are normalised to
\begin{equation}
  \bra{\vb{k}} \ket{\vb{k}'} = (2 E) \bdelta^3(\vb{k} - \vb{k}'),
\end{equation}
where $E = \sqrt{\vb{k}^2 + m^2}$ .
We can take superpositions of these to introduce two Gaussian wavepackets
\begin{equation}
  a_n^{\dagger} \coloneqq \int \dd[3]{k} f_n (\vb{k}) a^{\dagger}(\vb{k}),
  \qquad f_n(\vb{k}) \propto \exp[- \frac{\abs{\vb{k} - \vb{k}_n}^2}{4 \sigma^2}], \qquad n = 1, 2.
\end{equation}

Let us now evolve the Gaussians into the distant past and future, where the overlap in coordinate space is negligible.
We will also assume that this works when including interactions. There is a complication since $a^{\dagger}(\vb{k})$ becomes time dependent, wherefore $a_1^{\dagger}(t)$ and $a_2^{\dagger}(t)$ depend on time.
Assume that as $t \to \pm \infty$, the interacting $a_1^{\dagger}$ and $a_2^{\dagger}$ coincide with their free theory expressions.
The initial and final states are
\begin{align}
  \ket{i} &= \lim_{t \to -\infty} a_1^{\dagger}(t) a_2^{\dagger}(t) \ket{\Omega}  \\
  \ket{f} &=\lim_{t \to +\infty}  a_{1'}^{\dagger}(t) a_{2'}^{\dagger}(t) \ket{\Omega}.
\end{align}
These are also normalised $\bra{i}\ket{i} = 1 = \bra{f}\ket{f}$ and have $\vb{k}_1 \neq \vb{k}_2$ and $\vb{k}_1' \neq \vb{k}_2'$ respectively.

We want to find the \emph{scattering amplitude} $\bra{f}\ket{i}$.
\begin{remark}
  We have for example
  \begin{align}
    a_1^{\dagger}(\infty) - a_1^{\dagger}(-\infty) &= \int_{-\infty}^{\infty}\dd[]{t} \partial_0 a_1^{\dagger} (t) \\
					     &= \int \dd[3]{k} f_1(\vb{k}) \int \dd[4]{x} \partial_0 \left[ e^{-i k \cdot x} (-i \partial_0 \phi + E \phi) \right] \\
					     &= -i \int \dd[3]{k} f_1(\vb{k}) \int \dd[3]{x} e^{-i k \cdot x} (\partial_0^2 + E^2) \phi \\
					     &= -i \int \dd[3]{k} f_1(\vb{k}) \int \dd[3]{x} e^{-i k \cdot x} (\partial_0^2 - \stackrel{\leftarrow}{\nabla}{}^2 + m^2) \phi \\
					     &= -i \int \dd[3]{k} f_1(\vb{k}) \int \dd[3]{x} e^{-i k \cdot x} (\partial^2 + m^2) \phi,
  \end{align}
  where in the last line we integrated by parts twice.
  In the free theory, the Klein--Gordon equation therefore implies that $a_1^{\dagger}(\infty) - a_1^{\dagger}(-\infty) = 0$.
  Using a similar calculation for $a^{\dagger}_j$ and then $a_j$, we obtain
  \begin{subequations}
    \label{eq:7-creations}
     \begin{align}
       a_j^{\dagger}(-\infty) &= a_j^{\dagger}(\infty) + i \int \dd[3]{k} f_j(\vb{k}) \int \dd[4]{x} e^{-i k \cdot x} (\partial^2 + m^2) \phi \\
       a_j(\infty) &= a_j(-\infty) + i \int \dd[3]{k} f_j(\vb{k}) \int \dd[4]{x} e^{i k \cdot x} (\partial^2 + m^2) \phi.
    \end{align}
  \end{subequations}
\end{remark}
Using the time-ordering operator $\mathcal{T}$, we have
\begin{equation}
  \bra{f}\ket{i} = \bra{\Omega} \mathcal{T} a_{1'}(\infty) a_{2'}(\infty)a_{1}^{\dagger}(-\infty) a_2^{\dagger}(-\infty) \ket{\Omega}.
\end{equation}
We can use the integral expressions \eqref{eq:7-creations}, to commute these annihilation and creation operators, giving the rather unwieldy equation
\begin{multline}
  \label{eq:lsz}
  \bra{f}\ket{i} = i^4 \int \dd[4]{x_1} \dd[4]{x_2} \dd[4]{x_1'}\dd[4]{x_2'} e^{-i k_1 \cdot x_1} e^{-i k_2 \cdot x_2} e^{i k_1' \cdot x_1'} e^{i k_2' \cdot x_2'} \\
   \times (\partial_1^2 + m^2) (\partial^2_2 + m^2) (\partial_{1'}^2 + m^2) (\partial_{2'}^2 + m^2) \\
   \times \bra{\Omega} \mathcal{T} \phi(x_1) \phi(x_2) \phi(x_1') \phi(x_2') \ket{\Omega},
\end{multline}
where we have taken $\sigma \to 0$ such that $f(\vb{k}_j) \to \delta^3 (\vb{k} - \vb{k}_j)$.
This is the \emph{LSZ reduction} formula.

This result can be proven without recourse to the free theory as we have done. However, the machinery needed to build this up is more involved.
This general derivation requires only weaker assumptions
\begin{enumerate}[1)]
  \item We need a unique ground state $\ket{\Omega}$ and the first excited state has to be a single particle, which means that we need a single simple pole, rather than a continuum of particles for example.
  \item We want the field $\phi \ket{\Omega}$ to be that single particle state, meaning that $ \bra{\Omega} \phi \ket{\Omega} = 0$.
    Usually, this is not a problem; if we have $\bra{\Omega} \phi \ket{\Omega} = v \neq 0$, say when we have some spontaneous symmetry breaking, then we let $\widetilde{\phi} = \phi - v$ to obtain $\bra{\Omega} \widetilde{\phi} \ket{ \Omega} = 0$.
  \item We want $\phi$ normalised such that $\bra{k} \phi(x) \ket{\Omega} = e^{i k \cdot x}$ as in the free case.
    Usually, interactions spoil this and require us to rescale $\phi \to \mathcal{Z}_\phi^{1 / 2} \phi$.

    We see the need to `renormalise', e.g.
    \begin{equation}
      \mathscr{L} = \frac{1}{2} \partial_{\mu} \phi \partial^{\mu} \phi - \frac{1}{2} m^2 \phi^2 - \frac{\lambda}{4!} \phi^4 \to 
      \mathscr{L} = \frac{\mathcal{Z}_\phi}{2} \partial_{\mu} \phi \partial^{\mu} \phi - \frac{Z_m}{2} m^2 \phi^2 - \frac{\lambda}{4!} Z_\lambda \phi^4.
    \end{equation}
\end{enumerate}
