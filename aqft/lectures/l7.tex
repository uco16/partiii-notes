% lecture notes by Umut Özer
% course: aqft
\lhead{Lecture 7: February 01}

\chapter{LSZ Reduction Formula}%
\label{cha:lsz_reduction_formula}

We will now start to move beyond $d = 0$ and transition to using path integrals.
We will connect scattering amplitudes to correlation functions, specifically vacuum expectation values.
This transitory topic will also allow us to have a glimpse at renormalisation.
We will have in mind a weakly interacting theory, so that we can use the free theory as a good aproximation. Afterwards, we comment on how the interactions lead to deviations from the free theory.
In doing so, we tell the story the way that Srednicki \cite{srednicki} does. However, the results that we will show can also be rigorously proved non-perturbatively, which usually happens in the final chapters of quantum field theory textbooks \cite{peskin, weinberg}.

\section{\texorpdfstring{$2 \to 2$ }{}Scattering}%
\label{sec:$2 to 2$ _scattering}

Consider a free scalar field in $3 + 1$ dimensions, built out of plane waves
\begin{equation}
  \phi(x) = \int \frac{\bdd[3]{k}}{2E_{\vb{k}}} \left[ a(\vb{k}) e^{-i k \cdot x} + a^{\dagger} (\vb{k}) e^{i k \cdot x} \right],
\end{equation}
where $E_{\vb{k}}^2 = \abs{\vb{k}}^2 + m^2$ and we are using the mostly-minus metric in Minkowski space so $ k \cdot x = E t - \vb{k} \cdot \vb{x}$.
The creation and annihilation operators are relativistically normalised, changing the factor from $\sqrt{2 E} \to 2 E$  as compared to the conventions from our \emph{Quantum Field Theory} lectures.
In a free theory, where we do not need to do any perturbative expansion, we set $\hbar = c = 1$.
Let us find expressions for $a(\vb{k}) = a_{\vb{k}}$ and $a^{\dagger}_{\vb{k}}$ by looking a the following integrals:
\begin{align}
  \int \dd[3]{x} e^{i k \cdot x} \phi(x) &= \frac{1}{2E} a_{\vb{k}} + \frac{1}{2E} e^{2 i E t} a^{\dagger}_{-\vb{k}} \\
  \int \dd[3]{x} e^{i k \cdot x} \partial_0 \phi(x) &= -\frac{i}{2} a_{\vb{k}} + \frac{i}{2} e^{2 i E t} a^{\dagger}_{-\vb{k}}.
\end{align}
We can solve these equations for $a_{\vb{k}}$ and $a^{\dagger}_{\vb{k}}$ to get
\begin{align}
  a(\vb{k}) &= \int \dd[3]{x} e^{i k \cdot x} \left[ i \partial_0 \phi(x) + E \phi(x) \right] \\
  a^{\dagger}(\vb{k}) &= \int \dd[3]{x} e^{-i k \cdot x} \left[ -i \partial_0 \phi(x) + E \phi(x) \right]. \label{eq:7-a-dag}
\end{align}
Let the initial state for the free theory be a one-particle state
\begin{equation}
  \ket{\vb{k}} = a^{\dagger}(\vb{k}) \ket{\Omega},
\end{equation}
where the vacuum $\ket{\Omega}$ satisfies $a(\vb{k}) \ket{\Omega} = 0$ for all $\vb{k}$ and is normalised to $\bra{\Omega}\ket{\Omega} = 1$.
The one-particle states are normalised to $ \bra{\vb{k}} \ket{\vb{k}'} = (2 E) \bdelta^3(\vb{k} - \vb{k}')$.
We can take superpositions of these to introduce two Gaussian wavepackets with mean momentum $\vb{k}_1 \neq \vb{k}_2$ and width $\sigma$:
\begin{equation}
  \label{eq:7-free-gauss}
  a_n^{\dagger} \coloneqq \int \dd[3]{k} f_n (\vb{k}) a^{\dagger}(\vb{k}),
  \qquad f_n(\vb{k}) \propto \exp[- \frac{\abs{\vb{k} - \vb{k}_n}^2}{4 \sigma^2}], \qquad n = 1, 2.
\end{equation}

Let us now evolve the Gaussians into the distant past and future, $t \to \pm \infty$, where the overlap in coordinate space is negligible.
We will also assume that this works when including interactions. Interacting theories introduce the complication that $a^{\dagger}$ becomes time dependent, wherefore $a_1^{\dagger}$ and $a_2^{\dagger}$ depend on time.
However, we will assume that the interacting $a_1^{\dagger}(t)$ and $a_2^{\dagger}(t)$ coincide with their free theory expressions \eqref{eq:7-free-gauss} in the limit of $t \to \pm \infty$.

We want to consider the case where the initial and final states are two-particle states
\begin{equation}
  \ket{i} = \lim_{t \to -\infty} a_1^{\dagger}(t) a_2^{\dagger}(t) \ket{\Omega}, \qquad
  \ket{f} =\lim_{t \to +\infty}  a_{1'}^{\dagger}(t) a_{2'}^{\dagger}(t) \ket{\Omega}.
\end{equation}
These are also normalised so that $\bra{i}\ket{i} = 1 = \bra{f}\ket{f}$ and we have $\vb{k}_1 \neq \vb{k}_2$ and $\vb{k}_1' \neq \vb{k}_2'$ respectively.
Our goal is to find the \emph{scattering amplitude} $\bra{f}\ket{i}$.

Let us find an expression that allows us to relate creation (annihilation) operators from the distant past (future) to the distant future (past).
We employ the expressions \eqref{eq:7-free-gauss} and \eqref{eq:7-a-dag} and the fundamental theorem of calculus to rewrite the difference of $a_1^{\dagger}$ at $t = \pm \infty$ as
\begin{align}
  a_1^{\dagger}(\infty) - a_1^{\dagger}(-\infty) &= \int_{-\infty}^{\infty}\dd[]{t} \partial_0 a_1^{\dagger} (t) \\
					   &= \int \dd[3]{k} f_1(\vb{k}) \int \dd[4]{x} \partial_0 \left[ e^{-i k \cdot x} (-i \partial_0 \phi + E \phi) \right] \\
					   &= -i \int \dd[3]{k} f_1(\vb{k}) \int \dd[3]{x} e^{-i k \cdot x} (\partial_0^2 + E^2) \phi \\
					   &= -i \int \dd[3]{k} f_1(\vb{k}) \int \dd[3]{x} e^{-i k \cdot x} (\partial_0^2 - \stackrel{\leftarrow}{\nabla}{}^2 + m^2) \phi \\
					   &= -i \int \dd[3]{k} f_1(\vb{k}) \int \dd[3]{x} e^{-i k \cdot x} (\partial^2 + m^2) \phi,
\end{align}
where in the fourth line the dispersion relation $E^2 = \abs{\vb{k}}^2 + m^2$ is used and $\abs{\vb{k}}^2$ turns into the laplacian $\nabla^2$ acting leftward onto the exponential. In the last line we integrated by parts twice with $f_1(\vb{k})$ ensuring that surface terms vanish.
In the free theory, the Klein--Gordon equation therefore implies that $a_1^{\dagger}(\infty) - a_1^{\dagger}(-\infty) = 0$.
After a similar calculation for $a^{\dagger}_2$ and $a_{1', 2'}$, we obtain
\begin{subequations}
  \label{eq:7-creations}
   \begin{align}
     a_j^{\dagger}(-\infty) &= a_j^{\dagger}(\infty) + i \int \dd[3]{k} f_j(\vb{k}) \int \dd[4]{x} e^{-i k \cdot x} (\partial^2 + m^2) \phi \\
     a_j'(\infty) &= a_j'(-\infty) + i \int \dd[3]{k} f_j(\vb{k}) \int \dd[4]{x} e^{i k \cdot x} (\partial^2 + m^2) \phi.
  \end{align}
\end{subequations}

The $2 \to 2$ scattering amplitude is
\begin{equation}
  \label{eq:7-amp}
  \bra{f}\ket{i} = \bra{\Omega} \mathcal{T} a_{1'}(\infty) a_{2'}(\infty)a_{1}^{\dagger}(-\infty) a_2^{\dagger}(-\infty) \ket{\Omega},
\end{equation}
where we were able to trivially insert the time-ordering operator $\mathcal{T}$, since the operators were already time-ordered.

We then use the integral expressions \eqref{eq:7-creations} to replace the creation (annihilation) operators of \eqref{eq:7-amp} with their counterparts at the distant future (past).
The time-ordering operator then moves the resulting $a_i^{\dagger}(\infty)$ to the left and the $a_{j'}(-\infty)$ to the right, annihilating the vacuum. The only non-zero term is the one with the products of the integrals, yielding the unwieldy formula
\begin{simplebox}
  \begin{multline}
    \label{eq:lsz}
    \bra{f}\ket{i} = i^4 \int \dd[4]{x_1} \dd[4]{x_2} \dd[4]{x_1'}\dd[4]{x_2'} e^{-i k_1 \cdot x_1} e^{-i k_2 \cdot x_2} e^{i k_1' \cdot x_1'} e^{i k_2' \cdot x_2'} \\
     \times (\partial_1^2 + m^2) (\partial^2_2 + m^2) (\partial_{1'}^2 + m^2) (\partial_{2'}^2 + m^2) \\
     \times \bra{\Omega} \mathcal{T} \phi(x_1) \phi(x_2) \phi(x_1') \phi(x_2') \ket{\Omega},
  \end{multline}
\end{simplebox}
where we have taken the narrow-width limit $\sigma \to 0$ of the Gaussian wave packets, turning $f(\vb{k}_j) \to \delta^3 (\vb{k} - \vb{k}_j)$.
This is a specific case of the \emph{LSZ reduction} formula.
In its more general form, it states that the information of $n \to n'$ scattering amplitudes can be recovered from the correlation functions $\bra{\Omega} \mathcal{T} \phi(x_1) \dots \phi(x_n) \phi(x'_1) \dots \phi(x'_{n'}) \ket{\Omega}$.

This result can be proven without recourse to the free theory as we have done. However, the machinery needed to build this up is more involved.
Our derivation assumed that the interactions do not alter the creation operators \eqref{eq:7-free-gauss} at $t \to \pm \infty$.
In fact, the general derivation requires only weaker assumptions:
\begin{enumerate}[1)]
  \item We need a unique ground state $\ket{\Omega}$ and the first excited state has to be a single particle, which means that we need a single simple pole, rather than a continuum of particles for example.
  \item We want the field $\phi \ket{\Omega}$ to be that single particle state, meaning that $ \bra{\Omega} \phi \ket{\Omega} = 0$.
    Usually, this is not a problem; if we have $\bra{\Omega} \phi \ket{\Omega} = v \neq 0$, say when we have some spontaneous symmetry breaking, then we let $\widetilde{\phi} = \phi - v$ to obtain $\bra{\Omega} \widetilde{\phi} \ket{ \Omega} = 0$.
  \item We want $\phi$ normalised such that $\bra{k} \phi(x) \ket{\Omega} = e^{i k \cdot x}$ as in the free case.
  Usually, interactions spoil this and require us to rescale the field $\phi \to \mathcal{Z}_\phi^{1 / 2} \phi$.
\end{enumerate}

This hints at the fact that we need to `renormalise' the field and couplings when includng interactions. For example, in $\phi^4$ theory we need to introduce renormalisation factors $Z_\phi, Z_m$, and $Z_\lambda$:
\begin{equation}
  \mathscr{L} = \frac{1}{2} \partial_{\mu} \phi \partial^{\mu} \phi - \frac{1}{2} m^2 \phi^2 - \frac{\lambda}{4!} \phi^4 \to 
  \frac{\mathcal{Z}_\phi}{2} \partial_{\mu} \phi \partial^{\mu} \phi - \frac{Z_m}{2} m^2 \phi^2 - \frac{\lambda}{4!} Z_\lambda \phi^4.
\end{equation}
We will meet this Lagrangian again in Sec.~\ref{sec:renormalisation}.
