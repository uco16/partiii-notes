% lecture notes by Umut Özer
% course: aqft
\lhead{Lecture 4: January 25}

\subsection{Diagrams with External Sources}%
\label{sub:diagrams_with_external_sources}

As we have previously mentioned, diagrams with $E = n$ external sources need to be considered for the $n$-point correlation functions.
Let us focus on those diagrams that have $E = 2$ external currents.
\begin{equation}
  Z(J) \supset [ \quad
  \begin{gathered}
    \feynmandiagram[transform shape, scale=0.4][vertical=a to b] {
      a [large, dot] -- b [large, dot],
    };
  \end{gathered}
  \ + \
  \begin{gathered}
    \begin{tikzpicture}
      \begin{feynman}
	\tikzfeynmanset{every vertex={large, dot}};
	\vertex (a);
	\tikzfeynmanset{every vertex={small, dot}};
	\vertex[below=0.5cm of a] (b);
	\tikzfeynmanset{every vertex={large, dot}};
	\vertex[below=0.5cm of b] (d);

	\draw (b) arc [start angle=180, end angle=-180, radius=0.2cm];
	\diagram* {
	  (a) -- (d),
	};
      \end{feynman}
    \end{tikzpicture}
  \end{gathered}
  \ + \ 
  \begin{gathered}
    \feynmandiagram[transform shape, scale=0.4][vertical=a to b] {
      a [large, dot] -- b [large, dot],
    };
  \end{gathered}
  \quad \
  \mathclap{\begin{gathered}
     \feynmandiagram[transform shape, scale=0.5][small, vertical=a to b] {
       a [small, dot] -- [loop, min distance=2cm, in=-135, out=-45] a -- [loop, min distance=2cm, in=+135, out=+45] a,
     };
   \end{gathered}}
  \quad + \
  \begin{gathered}
    \begin{tikzpicture}
      \begin{feynman}
	\tikzfeynmanset{every vertex={large, dot}};
	\vertex (a);
	\tikzfeynmanset{every vertex={small, dot}};
	\vertex[below=0.5cm of a] (b);
	\vertex[below=0.5cm of b] (c);
	\tikzfeynmanset{every vertex={large, dot}};
	\vertex[below=0.5cm of c] (d);

	\draw (b) arc [start angle=180, end angle=-180, radius=0.2cm];
	\draw (c) arc [start angle=180, end angle=-180, radius=0.2cm];
	\diagram* {
	  (a) -- (d),
	};
      \end{feynman}
    \end{tikzpicture}
  \end{gathered}
  +
  \begin{gathered}
    \begin{tikzpicture}
      \begin{feynman}
	\tikzfeynmanset{every vertex={large, dot}};
	\vertex (a);
	\tikzfeynmanset{every vertex={small, dot}};
	\vertex[below=0.5cm of a] (b);
	\vertex[right=0.4cm of b] (c);
	\tikzfeynmanset{every vertex={large, dot}};
	\vertex[below=0.5cm of b] (d);

	\draw (b) arc [start angle=180, end angle=-180, radius=0.2cm];
	\draw (c) arc [start angle=180, end angle=-180, radius=0.2cm];
	\diagram* {
	  (a) -- (d),
	};
      \end{feynman}
    \end{tikzpicture}
  \end{gathered}
  + \
  \begin{gathered}
    \feynmandiagram[transform shape, scale=0.4][vertical=a to b] {
      a [large, dot] -- b [large, dot],
    };
  \end{gathered}
  \
  \begin{gathered}
    \feynmandiagram[transform shape, scale=0.8][small, horizontal=a to b, layered layout] {
      a [dot] -- [half left] b [dot] -- [half left] a,
      a -- [half left, looseness=0.5] b -- [half left, looseness=0.5] a,
    };
  \end{gathered}
  \ + \ \dots ]
\end{equation}

Factor out vacuum bubble diagrams
\begin{equation}
  [ \dots ] = 
  \underbrace{ [ \
    \begin{gathered}
      \feynmandiagram[transform shape, scale=0.4][vertical=a to b] {
        a [large, dot] -- b [large, dot],
      };
    \end{gathered}
    \ + \
    \begin{gathered}
      \begin{tikzpicture}
        \begin{feynman}
          \tikzfeynmanset{every vertex={large, dot}};
          \vertex (a);
          \tikzfeynmanset{every vertex={small, dot}};
          \vertex[below=0.5cm of a] (b);
          \tikzfeynmanset{every vertex={large, dot}};
          \vertex[below=0.5cm of b] (d);
          \draw (b) arc [start angle=180, end angle=-180, radius=0.2cm];
          \diagram* {
            (a) -- (d),
          };
        \end{feynman}
      \end{tikzpicture}
    \end{gathered}
    \ + \dots ]}_{\mathclap{\text{no vacuum bubbles}}}
  \cdot \underbrace{[1 + \quad
   \mathclap{\begin{gathered}
     \feynmandiagram[transform shape, scale=0.5][vertical=a to b] {
       a [small, dot] -- [loop, min distance=2cm, in=-135, out=-45] a -- [loop, min distance=2cm, in=+135, out=+45] a,
     };
   \end{gathered}}
   \quad + \
    \begin{gathered}
      \feynmandiagram[transform shape, scale=0.4][horizontal=a to b] {
	a [small, dot] -- [half left] b [small, dot] -- [half left] a,
	a -- [half left, looseness=0.5] b -- [half left, looseness=0.5] a,
      };
    \end{gathered}
    \  + \dots
  ]}_{\mathclap{Z(0)}}
\end{equation}

The $n$-point expectation values are then given by
\begin{align}
  \expectationvalue{\phi^2} &= \frac{(-\hbar)^2}{Z(0)} \left.\frac{\partial^2 Z(J) }{\partial J^2} \right\rvert_{J = 0} \\
  &=
  \underbrace{ [ \
    \begin{gathered}
      \feynmandiagram[transform shape, scale=0.4][vertical=a to b] {
        a [large, dot] -- b [large, dot],
      };
    \end{gathered}
    \ + \
    \begin{gathered}
      \begin{tikzpicture}
        \begin{feynman}
          \tikzfeynmanset{every vertex={large, dot}};
          \vertex (a);
          \tikzfeynmanset{every vertex={small, dot}};
          \vertex[below=0.5cm of a] (b);
          \tikzfeynmanset{every vertex={large, dot}};
          \vertex[below=0.5cm of b] (d);
          \draw (b) arc [start angle=180, end angle=-180, radius=0.2cm];
          \diagram* {
            (a) -- (d),
          };
        \end{feynman}
      \end{tikzpicture}
    \end{gathered}
    \ + \
    \begin{gathered}
      \begin{tikzpicture}
	\begin{feynman}
	  \tikzfeynmanset{every vertex={large, dot}};
	  \vertex (a);
	  \tikzfeynmanset{every vertex={small, dot}};
	  \vertex[below=0.5cm of a] (b);
	  \vertex[below=0.5cm of b] (c);
	  \tikzfeynmanset{every vertex={large, dot}};
	  \vertex[below=0.5cm of c] (d);
	  \draw (b) arc [start angle=180, end angle=-180, radius=0.2cm];
	  \draw (c) arc [start angle=180, end angle=-180, radius=0.2cm];
	  \diagram* {
	    (a) -- (d),
	  };
	\end{feynman}
      \end{tikzpicture}
    \end{gathered}
    \ + \
    \begin{gathered}
      \begin{tikzpicture}
	\begin{feynman}
	  \tikzfeynmanset{every vertex={large, dot}};
	  \vertex (a);
	  \tikzfeynmanset{every vertex={small, dot}};
	  \vertex[below=0.5cm of a] (b);
	  \vertex[right=0.4cm of b] (c);
	  \tikzfeynmanset{every vertex={large, dot}};
	  \vertex[below=0.5cm of b] (d);
	  \draw (b) arc [start angle=180, end angle=-180, radius=0.2cm];
	  \draw (c) arc [start angle=180, end angle=-180, radius=0.2cm];
	  \diagram* {
	    (a) -- (d),
	  };
	\end{feynman}
      \end{tikzpicture}
    \end{gathered}
    \ + \dots ]}_{\mathclap{\text{no vacuum bubbles}}}
\end{align}

\subsection*{Symmetry factors}%

From $Z(J)$ \eqref{eq:3-star}, the $E = 2$, $V = 0$ ($P = 1$) term is
\begin{equation}
  \frac{1}{2\hbar} \frac{J^2}{m^2} \qquad
  \begin{gathered}
    \feynmandiagram[transform shape, scale=0.5][small, horizontal=a to b, layered layout] {
      a [large, dot] -- b [large, dot],
    };
  \end{gathered}
\end{equation}
We have $F = 2$ and $A = 1$, so $\frac{A}{F} = \frac{1}{2} = \frac{1}{S}$.
The expectation value is $\langle \phi^2 \rangle = \frac{\hbar}{m^2} = 
\begin{gathered}
  \feynmandiagram[transform shape, scale=0.5][small, horizontal=a to b] {
    a [large, dot] -- b [large, dot],
  };
\end{gathered} $ as expected!

$\langle \phi^{2n} \rangle$ proceeds similarly, but note that there \emph{are} disconnected diagrams
\begin{equation}
  \langle \phi^4 \rangle = 
  \begin{gathered}
    \feynmandiagram[transform shape, scale=0.5][small, vertical=a to b, layered layout] {
      a [large, dot] -- b [large, dot],
      c [large, dot] -- d [large, dot],
    };
  \end{gathered}
  + 
  \begin{gathered}
    \feynmandiagram[transform shape, scale=0.5][small, horizontal=a to b] {
      a [large, dot] -- v [small, dot] -- b [large, dot],
      c [large, dot] -- v -- d [large, dot],
    };
  \end{gathered}
  + 
  COPY FROM NOTES
\end{equation}

\section{Effective Actions}%
\label{sec:effective_actions}

\begin{definition}[Wilson effective action]
  We define $W(\phi)$ such that $ Z(J) = e^{-W (J) / \hbar}$.
\end{definition}

\begin{claim}
  $W(0)$ is the sum of all connected vacuum diagrams and $W(J)$ is the sum of all connected diagrams.
\end{claim}
\begin{proof}
  We denote the set of connected diagrams as $\{C_I\}$, which are taken to contain their respective symmetry factors.
  Any diagram $D$ is a product of connected diagrams:
  \begin{equation}
    D = \frac{1}{S_D} \prod_I (C_I)^{n_I},
  \end{equation}
  where $n_I$ is the number of times $C_I$ appears in $D$, and  $S_D$  is the symmetry factor associated with rearranging diagrams $C_I$'s, given by
  \begin{equation}
    S_D = \prod_{I} n_I!.
  \end{equation}

  \begin{example}[]
    If the diagram is
    \begin{equation}
      \begin{gathered}
	\feynmandiagram[transform shape, scale=0.5][small, vertical=a to b] {
	  a -- [loop, min distance=2cm, in=-135, out=-45] a -- [loop, min distance=2cm, in=+135, out=+45] a,
	};
      \end{gathered}
      \begin{gathered}
	\feynmandiagram[transform shape, scale=0.5][small, vertical=a to b] {
	  a -- [loop, min distance=2cm, in=-135, out=-45] a -- [loop, min distance=2cm, in=+135, out=+45] a,
	};
      \end{gathered}
      \begin{gathered}
	\feynmandiagram[transform shape, scale=0.5][small, vertical=a to b, layered layout] {
	  a -- [loop, min distance=2cm, in=135, out=45] a -- [half left] b -- [loop, min distance=2cm, in=-135, out=-45] b,
	  b -- [half left] a,
	};
      \end{gathered}
      \begin{gathered}
	\feynmandiagram[transform shape, scale=0.5][small, vertical=a to b] {
	  a -- [loop, min distance=2cm, in=-135, out=-45] a -- [loop, min distance=2cm, in=+135, out=+45] a,
	};
      \end{gathered}
      = 
      \underbrace{      \begin{gathered}
		  	\feynmandiagram[transform shape, scale=0.5][small, vertical=a to b] {
		  	  a -- [loop, min distance=2cm, in=-135, out=-45] a -- [loop, min distance=2cm, in=+135, out=+45] a,
		  	};
		        \end{gathered}
		        \begin{gathered}
		  	\feynmandiagram[transform shape, scale=0.5][small, vertical=a to b] {
		  	  a -- [loop, min distance=2cm, in=-135, out=-45] a -- [loop, min distance=2cm, in=+135, out=+45] a,
		  	};
		        \end{gathered}
		        \begin{gathered}
		  	\feynmandiagram[transform shape, scale=0.5][small, vertical=a to b] {
		  	  a -- [loop, min distance=2cm, in=-135, out=-45] a -- [loop, min distance=2cm, in=+135, out=+45] a,
		  	};
		    \end{gathered}}_{(C_1)^3}
		    \underbrace{      \begin{gathered}
					\feynmandiagram[transform shape, scale=0.5][small, vertical=a to b, layered layout] {
					  a -- [loop, min distance=2cm, in=135, out=45] a -- [half left] b -- [loop, min distance=2cm, in=-135, out=-45] b,
					  b -- [half left] a,
					};
				      \end{gathered}}_{\mathclap{c_2}}
    \end{equation}
    The disconnected parts commute. We have $n_1 = 3$ and $n_2 = 1$. The associated factor is $S_D = 3!  \cdot 1! = 6$.
  \end{example}
  Let $\{n_I\}$ be the set of integers specifying $D$, then
  \begin{align}
    \frac{Z}{Z_D} = \sum_{\{n_I\}} D
		  &= \sum_{\{n_I\}} \prod_I \frac{1}{n_I!} (c_I)^{n_I} \\
		  &= \prod_{I} \sum_{n_I} \frac{1}{n_I!} (c_I)^{n_I} \\
		  &= \exp(\sum_I c_I) \\
		  &= \exp( \text{sum of unique connected diagrams} ) \\
		  &\coloneqq e^{-(W - W_0) / \hbar},
  \end{align}
  where $W = W_0 - \hbar \sum_{I} c_I$.
\end{proof}

\begin{claim}
  $W(J)$ is the generating function for \emph{connected correlation functions}.
\end{claim}
\begin{example}
  Taking logarithms, we have
  \begin{equation}
    -\frac{1}{\hbar} W(J) = \ln (Z(J)).
  \end{equation}
  Differentiating with respect to $J$  twice and evaluating at $J = 0$ gives
   \begin{align}
     -\frac{1}{\hbar} \left.\frac{\partial^2}{\partial J^2} W \right\rvert_{J = 0} &= \frac{1}{Z(0)} \left.\frac{\partial^2 Z}{\partial J^2}\right\rvert_{J = 0} - \frac{1}{(Z(0))^2} \left.\left( \frac{\partial Z}{\partial J} \right)^2 \right\rvert_{J =0} \\
     &= -\frac{1}{\hbar^2} \left[ \langle \phi^2 \rangle - \cancel{\langle \phi \rangle}^2 \right] = \frac{1}{\hbar^2} \langle \phi^2 \rangle_{\text{connected}}.
  \end{align}
  \begin{leftbar}
    There are also theories where $\langle \phi \rangle \neq 0$.
  \end{leftbar}
  Less trivially, 
  \begin{align}
    -\frac{1}{\hbar} \left. \frac{\partial^4 W}{\partial J^4} \right\rvert_{J = 0} &= \frac{1}{Z(0)} \left. \frac{\partial^4 Z}{\partial J^4}\right\rvert_{J = 0} - \left.\left( \frac{1}{Z(0)} \frac{\partial^2 Z}{\partial J^2} \right)^2 \right\rvert_{J = 0} \\
    \langle \phi^4 \rangle_{\text{connected}} &= \langle \phi^4 \rangle - \langle \phi^2 \rangle^2
  \end{align}
\end{example}

\subsection*{Interactions}%

Consider an action for two distinguishable fields
\begin{equation}
  S(\phi, \chi) = \frac{m^2}{2} \phi^2 + \frac{M^2}{2} \chi^2 + \frac{\lambda}{4} \phi^2 \chi^2.
\end{equation}
There is no factorial behind the factor $4$.
The Feynman rules are
\begin{equation}
  \begin{gathered}
    \feynmandiagram[transform shape, scale=1][horizontal=a to b] {
      a -- [solid, edge label=$\phi$] b,
    };
  \end{gathered}
  \qquad
  \begin{gathered}
    \feynmandiagram[transform shape, scale=1][horizontal=a to b] {
      a -- [dotted, edge label=$\chi$] b,
    };
  \end{gathered}
  \qquad
  \begin{gathered}
    \feynmandiagram[transform shape, scale=1][horizontal=a to b] {
      a -- c -- [dotted] b,
      d -- [dotted] c -- f,
    };
  \end{gathered}
\end{equation}
The connected vacuum diagrams are
\begin{align}
  -\frac{W}{\hbar} &= [
  \begin{gathered}
    \feynmandiagram[transform shape, scale=0.5][horizontal=a to b] {
      a -- [loop, dotted, min distance=2cm, in=-135, out=-45] a,
      a -- [loop, min distance=2cm, in=135, out=45] a,
    };
  \end{gathered}
  + 
  \begin{gathered}
    \feynmandiagram[transform shape, scale=0.5][vertical=a to b] {
      a -- [loop, min distance=2cm, in=-135, out=-45] a -- [half right, dotted] b -- [loop, min distance=2cm, in=135, out=45] b -- [half right, dotted] a,
    };
  \end{gathered}
  + DIAGRAMS
  ] \\
  &= -\frac{\hbar\lambda}{4 m^2 M^2} + \frac{\hbar^2 \lambda^2}{m^4 M^4} [\frac{1}{16} + \frac{1}{16} + \frac{1}{8}].
\end{align}

Also from Feynman diagrams, 
\begin{align}
  \langle \phi^2 \rangle &= DIAGRAMS \\
  &= \frac{\hbar}{m^2} - \frac{\hbar^2 \lambda}{2 m^4 M^2} + \frac{\hbar^3 \lambda^2}{m^6 M^4} [\frac{1}{4} + \frac{1}{2} + \frac{1}{4}] + \dots
\end{align}
Say we do not care about the field $\chi$, maybe $M \gg m$, the high energy field is never produced in our experiments.
Then we want to `integrate out' $\chi$ by defining $W(\phi)$ as
\begin{equation}
  e^{- W(\phi) / \hbar} = \int \dd[]{\chi} e^{- S(\phi, \chi) / \hbar}.
\end{equation}
In other words, we treat $\phi^2 \chi^2$ as a source term, $J = \phi^2$ in earlier notation.
