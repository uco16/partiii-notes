% lecture notes by Umut Özer
% course: aqft
\lhead{Lecture 21: March 05}

\subsection{Non-Abelian Case}%
\label{sub:non_abelian}

Both the Wilson line and the Wilson loop can be generalised to the non-Abelian case.
Take $V(x) \in G$ to be the fundamental representation of the gauge group $G$, such as $SU(N)$, acting on the $n$-plet of fields $\psi$ as
\begin{equation}
  \psi(x) \to V(x) \psi(x).
\end{equation}
We want to construct the Wilson line that transforms in such a way to cancel out differences in transformations at different points
\begin{equation}
  \label{eq:20-nonabelian}
  W(y, x) \to V(y) W(y, x) V^{\dagger}(x)
\end{equation}
with normalisation $W(x, x) = \mathbb{1}$, the identity of $G$.

Consider an infinitesimal Wilson line (small $a$)
\begin{equation}
  W(x + an, x) = 1 + i g a n^{\mu} A_{\mu}^{a} T^{a}.
\end{equation}

Using the same methods as before, we find that the associated covariant derivative is
\begin{equation}
  D_{\mu} = \partial_{\mu} - i g \mathcal{A}_{\mu},
\end{equation}
where we use the shorthand $\mathcal{A}_{\mu} \coloneqq A_{\mu}^{a} T^{a}$ and $T^{a}$ are the generators of the Lie algebra $\mathfrak{g}$.
To find the form of the gauge transformation $\mathcal{A}_{\mu} \to \mathcal{A}'_{\mu}$, we require as in the Abelian case that the covariant derivative transforms in the same way as the field $\psi$:
\begin{align}
  D_{\mu}\psi \to D'_{\mu} \psi' &= V D_{\mu} \psi \\
  (\partial_{\mu} - i g \mathcal{A}'_{\mu}) V\psi &= V(\partial_{\mu} - i g \mathcal{A}_{\mu} \psi).
\end{align}
Solving for $\mathcal{A}'_{\mu}$ gives
\begin{equation}
  \mathcal{A}'_{\mu} = V A_{\mu} V^{-1} - \frac{i}{g} (\partial_{\mu} V) V^{-1} = V \left(\frac{i}{g} \stackrel{\leftarrow}{D}_{\mu}\right) V^{-1}.
\end{equation}
Under an infinitesimal transformation
\begin{equation}
  V(x)= \mathbb{1} + i \alpha^{a}(x) T^{a} + O(\alpha^2),
\end{equation}
the fields $\psi$ and $A^{a}_{\mu}$ transform as
\begin{align}
  \psi(x) &\to (\mathbb{1} + i \alpha^{a}(x) T^{a})\psi \\
  A^{a}_{\mu}(x) &\to A^a_{\mu} (x) + \frac{1}{g} \partial_{\mu} \alpha^{a}(x) + f^{abc} A_{\mu}^{b} (x) \alpha^{c}(x) + \frac{1}{g} \partial_{\mu} \alpha^{a}(x) \\
	   &= A^{a}(x) +\frac{1}{g} [\partial_{\mu} \delta^{ac} - i g A^{b}_\mu(x) (\underbrace{-i f^{bac}}_{(T^{b}_{\text{adj}})^{ac}})] \alpha^{c}(x) \\
	   &\coloneqq A^{a}_{\mu} (x) + \frac{1}{g} D_{\mu}^{ac} \alpha^{c}(x),
\end{align}
where we defined the gauge covariant derivative in the adjoint representation as 
\begin{equation}
  D_{\mu}^{ab} = \left[ \delta^{ab} \partial_{\mu} - i g A_{\mu}^{c} (T^{c} _{\text{adj}})^{ab} \right].
\end{equation}
This justifies our previous statement that the gauge fields transform in the adjoint representation.

\subsection{Non-Abelian Wilson Loops}%
\label{sub:non_abelian_wilson_lines}

\begin{leftbar}
  Not lectured. Taken from \cite[Chapter 15.3]{peskin}.
\end{leftbar}
We want to find the \emph{finite} Wilson line that generalises \eqref{eq:20-line} to the non-Abelian case.
Since the matrices $M$ do not in general commute at different points, and their exponentials combine non-trivially (according to the Baker--Campbell--Hausdorff formula), it would not be correct to simply replace $A_{\mu} \to A_{\mu}^{a} T^{a}$ in the exponent of the Abelian Wilson line \eqref{eq:20-line}.
Instead, we must order the matrices in a particular way.

Let $t$ be a parameter of the path $C$, running from $0$ at $x = y$ to $1$ at $x = z$.
Then we define the Wilson line as the power-series expansion of the exponential, with the matrices in each term ordered so that the higher values of $s$ stand to the left. This prescription is called \emph{path-ordering} and is denoted by the symbol $P$.
The non-Abelian Wilson line is
\begin{equation}
  W_C(z, y) = P \left\{ \exp[i g \int_0^1 \dd[]{t} \dv{x^{\mu}}{t} A^{a}_{\mu}(x(t)) T^{a}] \right\}.
\end{equation}
The expression for the non-Abelian Wilson line is analogous to the time-ordered exponential in Dyson's formula for the interaction-picture propagator, which we have met in the \emph{Quantum Field Theory} lectures last term.

Now what is the natural generalisation of the Abelian Wilson loop \eqref{eq:20-loop}?
Since the matrices do not commute, the Wilson line $W_C(y, y)$ associated with a closed path returning to $y$ is not gauge invariant, but transform as
\begin{equation}
  \label{eq:21-wnonabtrans}
  W_C(y, y) \to V(y) W_C(y, y) V^{-1}(y).
\end{equation}
We therefore define the Wilson loop for a non-Abelian gauge theory as the \emph{trace} of the non-Abelian Wilson line around a closed path:
\begin{equation}
  \tr W_C(y, y) = \tr P \left\{ \exp[i g \int_0^1 \dd[]{t} \dv{x^{\mu}}{t} A^{a}_{\mu}(x(t)) T^{a}] \right\}.
\end{equation}

\begin{example}[infinitesimal plaquette (non-Abelian)]
  Let us again consider the infinitesimal square shown in Fig.~\ref{fig:plaquette}.
  In this case, it turns out that one finds
  \begin{equation}
    P_{12}(x) = W_C(x, x) = 1 + i g a^2 F^{a}_{12} T^{a} + O(a^3),
  \end{equation}
  where $F^{a}_{\mu\nu}$ is given by the full non-Abelian expression \eqref{eq:nonabf}.
  Expanding the transformation law \eqref{eq:21-wnonabtrans} to order $a^2$, one recovers the transformation law \eqref{eq:nonabftrans} for $F^{a}_{\mu\nu}$.
  
  For $G =SU(2)$, where $T^{a} = \sigma^{i}$, we can evaluate the Wilson loop $\tr W_C(x, x)$ more explicitly, finding
  \begin{equation}
    \tr W_C(x, x) = 2 - \frac{1}{4} g^2 a^2 (F^{a}_{12})^2 + O(a^5).
  \end{equation}
  The gauge invariance of $(F^{a}_{\mu\nu})^2$ can therefore be derived from this geometrical argument, just as in the Abelian case.
\end{example}

\subsection{Non-Abelian Gauge Invariant Lagrangians}%
\label{sub:gauge_invariant_lagrangians}

We define the field strength tensor through the commutator of the covariant derivatives
\begin{equation}
  [D_{\mu}, D_{\nu}] = -i g F_{\mu\nu}^{a} T^{a}.
\end{equation}
Then the field strength tensor $F_{\mu\nu}= F_{\mu\nu}^{a} T^{a}$ has components
\begin{equation}
  \label{eq:nonabf}
  F_{\mu\nu}^{a} = \partial_{\mu} A_{\nu}^{a} - \partial_{\nu} A_{\mu}^{a} + g f^{abc} A^{b}_{\mu} A_{\nu}^{c}.
\end{equation}
Under an infinitesimal transformation,
\begin{equation}
  \label{eq:nonabftrans}
  F_{\mu\nu}^{a} \to F_{\mu\nu}^{a} - f^{abc} \alpha^{b} F^{c}_{\mu\nu},
\end{equation}
so $F_{\mu\nu}^{a}$ is not gauge invariant! However, 
\begin{equation}
  F_{\mu\nu}^{a} (F^{a})^{\mu\nu} = (F^{a})^2
\end{equation}
is gauge-invariant.
We have a Lagrangian
\begin{align}
  \mathscr{L} &= \underbrace{\frac{1}{4} (F^{a})^2}_{\mathclap{}} + \underbrace{\overline{\psi}{}_{i} \left( \cancel{\partial} \delta_{ij} - i g \cancel{A}^a T^{a}_{ij} + m \delta_{ij} \right) \psi_{j}}_{\mathclap{}} \\
	      &= \mathscr{L}_{YM} + \overline{\psi}{} (\cancel{D} + m) \psi,
\end{align}
where the Yang--Mills Lagrangian is often called the \emph{pure gauge}.
Also, we can introduce another gauge invariant term
\begin{equation}
  \mathscr{L}_{\theta} = \theta \epsilon^{\mu\nu\rho\sigma} F^{a}_{\mu\nu} F^{a}_{\rho\sigma} = 2 \theta \partial_{\mu} (\epsilon^{\mu\nu\rho\sigma} A^{a}_{\nu} F_{\rho\sigma}^{a}).
\end{equation}
As this is a total derivative, it is a boundary term which does not contribute to any order in perturbation theory. However, it can contribute a nonperturbative topological term, which we miss out on in the perturbation expansion. It violates CP and therefore also T. Experimentally the strong interaction in QCD seems to conserve CP, constraining the $\theta$ to be vanishingly small. The \emph{Strong CP Problem} is the task of finding a reason for why $\theta$ is so small.

\section{Fadeev--Popov Gauge Fixing}%
\label{sec:fadeev_popov_gauge_fixing}

Let us first think about an analogy. Consider an integral over two real variables
\begin{equation}
  Z \propto \int \dd[]{x} \dd[]{y} e^{-S(x)},
\end{equation}
where the action $S(x)$ only depends on one of the variables.
The integral over $y \in (-\infty, \infty)$ is divergent, but it is also redundant to any physics.
In QED, i.e.~in $U(1)$ gauge theory, we just set $y = 0$
\begin{equation}
  Z = \int \dd[]{x} e^{- S(x)}
\end{equation}
Life is more complicated when the gauge fields interact with themselves and the previous method will not work in non-Abelian gauge theory.
Instead, we may introduce a $\delta$-function to fix $y$ not to zero, but to an arbitrary function of $x$
\begin{equation}
  \label{eq:21-1}
  Z = \int \dd[]{x} \dd[]{y} \delta(y - f(x)) e^{-S(x)}.
\end{equation}
Maybe we do not have $y$ explicitly. All we need is that $y = f(x)$ is the unique solution to some equation $G(x, y) = 0$, which we call the \emph{gauge-fixing condition}.
To rewrite \eqref{eq:21-1} in a more general way, we use the composition rule for $\delta$-functions
\begin{equation}
  \delta\bigl(G(x, y)\bigr) = \abs{\frac{\partial G}{\partial y}}^{-1} \delta(y - f(x)),
\end{equation}
where we use the assumption that $y - f(x)$ is the unique solution to $G(x, y) = 0$.
Then, assuming further that $\frac{\partial G}{\partial y} > 0$, 
\begin{equation}
  Z = \int \dd[]{x} \dd[]{y} \frac{\partial G}{\partial y} \delta\bigl(G(x, y)\bigr) e^{-S(x)}.
\end{equation}
These assumptions are fine in perturbation theory, but we have \emph{Gribov copies} non-perturbatively.

This generalises to $n$ variables as
\begin{equation}
  Z = \int \dd[n]{x} \dd[n]{y} \det(\frac{\partial G}{\partial y}) \left[ \prod_{i = 1}^n \delta(G_i) \right] e^{-S(x)}.
\end{equation}
In gauge theory, these are really path integrals. The above form means that we can somehow separate the path integral $\pdd{A}$ into the physical fields $x$ and the gauge-equivalent degrees of freedom $y$.

\subsection{Gauge Theory and Fadeev--Popov Determinant}%
\label{sub:gauge_theory_and_fp_determinant}

Let us now apply this to gauge field theory. The gauge-fixing condition, generalising $\partial_{\mu} A^{\mu} = 0$ from QED, is
\begin{equation}
  G^{a}(x) = \partial^{\mu} A^{a}_{\mu}(x) - w^{a}(x)
\end{equation}
where now the $x$ represent spacetime labels and $w^{a}(x)$ are field-independent functions of $x$.
Under gauge transformation,
\begin{equation}
  G^{a}(x) \to G^{a}(x) + \frac{1}{g} \partial^{\mu} D^{ab}_{\mu} \alpha^{b}(x).
\end{equation}
We can extract the differential operator with a functional derivative
\begin{equation}
  \frac{\delta G^{a}(x)}{\delta \alpha^{b}(y)} = \frac{1}{g} \delta^{(4)}(x - y) (\partial^{\mu} D_{\mu}^{ab}).
\end{equation}
This gives us the Fadeev--Popov determinant
\begin{equation}
  \det( \frac{\delta G^{a}(x)}{\delta \alpha^{b}(y)} ),
\end{equation}
which are the components going into the path integral for the generating functional.
The way to make sense of this formal object, a determinant over differential operators, is to rewrite it as a path integrals over fermionic ghost fields.
