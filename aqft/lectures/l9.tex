% lecture notes by Umut Özer
% course: aqft
\lhead{Lecture 9: February 06}

\section{Vertex Functions}%
\label{sec:vertex_functions}

Recall that $W[J]$ and $\Gamma[\Phi]$ are the sum of connected and 1PI diagrams respectively.
Generalising the definition \eqref{eq:qee} from $d =0$, the quantum effective action is the Legendre transform
\begin{equation}
  \Gamma[\Phi] = W[J] - \int \dd[4]{x} J(x) \Phi(x).
\end{equation}
Then we have the familiar relations, which are now functional derivatives
\begin{equation}
  \frac{\delta W[J]}{\delta J(x)} = \Phi(x),\qquad \frac{\delta \Gamma[\Phi]}{\delta \Phi(x)} = - J(x).
\end{equation}
The connected $n$-point functions are
\begin{equation}
  G^{(n)} (x_1, \dots, x_n) = (-1)^{n+1} \prod_{i = 1}^n \frac{\delta }{\delta J(x_{i})} W[J] = \langle \phi(x_1) \dots \phi(x_n) \rangle^{\text{connected}}.
\end{equation}
Analogously, we define the $n$ -point vertex functions
\begin{equation}
  \Gamma^{(n)} (x_1, \dots, x_n) = (-1)^{n} \prod_{i = 1}^n \frac{\delta }{\delta \Phi(x_i)} \Gamma[\Phi].
\end{equation}

\begin{example}[$n = 2$]
  The $2$ -point functions are
  \begin{align}
    G^{(2)}(x, y) &= - \frac{\delta^2 W}{\delta J(x) \delta J(y)} = -\frac{\delta \Phi(y)}{\delta J(x)}, \\
    \Gamma^{(2)}(x, y) &= + \frac{\delta^2 \Gamma}{\delta \Phi(x) \delta \Phi(y)} = - \frac{\delta J(x)}{\delta \phi(y)}.
  \end{align}
  These are inverses of each other
  \begin{equation}
    \int \dd[4]{z} G^{(2)} (x, z) \Gamma^{(2)} (z, y) = \delta^{(4)} (x - y).
  \end{equation}
\end{example}
\begin{example}[Sheet I Question 4]
  We can calculate
  \begin{equation}
    \label{eq:9-star}
    G^{(3)}(x_1, x_2, x_3) = \int \dd[4]{z_1} \dd[4]{z_2} \dd[4]{z_3} \cdot G^{(n)} (x_1, z_1) G^{(n)} (x_2, z_2) G^{(n)} (x_3, z_3) \underbrace{\left( - \frac{\delta^3 \Gamma}{\delta \Phi(z_1) \delta \Phi(z_2) \delta\Phi(z_3)} \right)}_{\mathclap{\coloneqq \Gamma^{(3)}(z_1, z_2, z_3)}}
  \end{equation}
  Diagrammatically, this can be represented
  \begin{equation}
    \begin{gathered}
      \feynmandiagram[transform shape, scale=1][horizontal=a to b] {
        a [particle=\(x_1\)] -- v [blob] -- b [particle=\(x_3\)],
	c [particle=\(x_2\)] -- v,
      };
    \end{gathered}
    = 
    \begin{gathered}
      \feynmandiagram[transform shape, scale=1][horizontal=a to b] {
	a [particle=\(x_1\)] -- d [small, blob] -- v [blob, label={right:$\Gamma^{(3)}$}] -- e [small, blob, label={right:$G^{(2)}$}] -- b [particle=\(x_3\)],
	c [particle=\(x_2\)] -- f [small, blob] -- v,
      };
    \end{gathered},
  \end{equation}
  where the central blob on the right hand side corresponds to the 1PI diagrams $\Gamma^{(3)}$ and the small blobs are the propagators $G^{(2)}$.
  The vertex functions $\Gamma^{(3)}$ come from \emph{amputated} $n$-point functions.
  Equation \eqref{eq:9-star} can be inverted
  \begin{equation}
    \Gamma^{(3)}(y_1, y_2, y_3) = \int \dd[4]{x_1} \dd[4]{x_2} \dd[4]{x_3} \Gamma^{(2)}(x_1, y_1) \Gamma^{(2)}(x_2, y_2) \Gamma^{(2)}(x_3, y_3) G^{(3)}(x_1, x_2, x_3).
  \end{equation}
  It is useful to compare this to the LSZ reduction formula.
  We can generalise this to $n$ and in momentum space. This is done in Ryder Section 7.3.
\end{example}

\section{Renormalisation}%
\label{sec:renormalisation}

Consider our favourite $\phi^4$ scalar field theory in Euclidean spacetime
\begin{equation}
  S[\phi] = \frac{1}{2} \partial_{\mu} \phi \partial^{\mu} \phi + \frac{1}{2} m^2 \phi^2 + \frac{\lambda}{4!} \phi^4.
\end{equation}
The full propagator
\begin{equation}
  \widetilde{G}^{(2)}(p) = \int \dd[4]{x} e^{-i p \cdot x}\langle \phi(x) \phi(0) \rangle^{\text{conn}}
\end{equation}
can be written as a geometric series to make the connection between connected and 1PI diagrams.
Diagrammatically, 
\begin{align}
  \widetilde{G}^{(n)}(p) &=
  \begin{gathered}
    \feynmandiagram[transform shape, scale=1][horizontal=a to b] {
      a -- b,
    };
  \end{gathered}
  + 
  \begin{gathered}
    \begin{tikzpicture}
      \begin{feynman}
	%\tikzfeynmanset{every vertex={large, dot}};
	\vertex[left=1cm of a] (d);
	\vertex[blob] (a) {1PI};
	\vertex[right=1cm of a] (b);
	%\draw (b) arc [start angle=180, end angle=-180, radius=0.2cm];
	\diagram* {
      (d) -- (a) -- (b),
	};
      \end{feynman}
    \end{tikzpicture}
  \end{gathered}
  + 
  \begin{gathered}
    \begin{tikzpicture}
      \begin{feynman}
	%\tikzfeynmanset{every vertex={large, dot}};
	\vertex[left=1cm of a] (d);
	\vertex[blob] (a) {1PI};
	\vertex[blob, right=1cm of a] (e) {1PI};
	\vertex[right=1cm of e] (f);
	%\draw (b) arc [start angle=180, end angle=-180, radius=0.2cm];
	\diagram* {
	  (d) -- (a) -- (e) -- (f),
	};
      \end{feynman}
    \end{tikzpicture}
  \end{gathered}
  + \dots
  \\
  &= \frac{1}{p^2 + m^2} + \frac{1}{p^2 + m^2} \Pi(p^2) \frac{1}{p^2 + m^2} + \dots \\
  &= \frac{1}{p^2+ m^2 - \Pi(p^2)},
\end{align}
where $\Pi(p^2) = \widetilde{\Gamma}^{(2)}(p)$.
Perturbatively
\begin{equation}
  \Pi(p^2) = DIAGRAMS
\end{equation}
The first term in particular is
\begin{equation}
  DIAGRAM = -\frac{\lambda}{2} \int \bdd[4]{k} \frac{1}{k^2+ m^2}.
\end{equation}
Now in spherical polar coordinates $\dd[4]{k} = k^{3} \dd[]{k} \dd[]{\Omega}$  and in general $\dd[d]{k} = S_d \abs{k}^{d-1} \dd[]{\abs{k}}$  with $S_d = (2 \pi)^{d / 2} / \Gamma(d / 2)$ is the surface area of the unit $d$-sphere, where $\Gamma$ is the Euler-Gamma function.
We see that this integral diverges.
To tackle it, we introduce a cutoff  $\Lambda$ and change variables to $x = k^2 / m^2$ so that
 \begin{align}
   -\frac{\lambda}{2} \int^{\Lambda} \frac{\bdd[4]{k}}{k^2 +m^2} &= -\frac{\lambda S_4}{4 (2 \pi)^4} \int^{\Lambda^2 / m^2} \frac{x \dd[]{x}}{1 + x}, \\
   \Pi_i(p^2) &= -\frac{\lambda}{32 \pi^2} \left[ \Lambda^2 - m^2 \ln(1 + \frac{\Lambda^2}{m^2}) \right]. \label{eq:9-pi}
\end{align}
As expected, this is divergent as $\Lambda \to \infty$. We call this a \emph{UV divergence}.

The $4$-point function at $1$-loop is given by the following sum of diagrams
\begin{align}
  DIAGRAMS &= \frac{\lambda^2}{2} \int^\Lambda \bdd[4]{k} \frac{1}{k^2+ m^2} \sum_{p \in \{p_1 + p_2, p_1 + p_3, p_1 + p_4\}} \frac{1}{(p + k)^2 + m^2}. \\
	   &= \widetilde{\Gamma}^{(4)}(p_1, p_2, p_3, p_4)
\end{align}
As $k \to \infty$, we have $\dd[]{k} / k^4$ in the integral and we expect a logarithmic $\ln(\Lambda / m)$ divergence.
Since this integral is independent of the $p$'s, it is convenient to evaluate the integral with external momenta set to zero.
\begin{align}
  \widetilde{\Gamma}^{(4)}(0,0,0,0) &= \frac{3 \lambda^2}{2} \int^\Lambda \bdd[4]{k} \frac{1}{(k^2 + m^2)^2} = \frac{3 \lambda^2}{16 \pi^2} \int^\Lambda \frac{k^3 \dd[]{k}}{(k^2+ m^2)^2} \\
				    &= \frac{3 \lambda^2}{32 \pi^2} \left[ \ln(1 + \frac{\Lambda^2}{m^2}) - \frac{\Lambda^2}{\Lambda^2 + m^2} \right]
\end{align}
These diagrams must be dealt with.

On general grounds, we expect the full propagator to be of the form
\begin{equation}
  \widetilde{G}^{(3)} = \sum_{n} \frac{\abs{\bra{\Omega} \phi(0) \ket{n}}^2}{p^2 + m^2}
\end{equation}
where the sum (or the integral) over the states $n$ comes about from inserting a complete set of states.
The assumption is that the first excited state is a single particle $n = 1$, which dominates the expression:
 \begin{equation}
   \widetilde{G}^{(3)} = \frac{\abs{\bra{\Omega} \phi(0) \ket{1}}^2}{p^2 + m_1^2} + \dots,
\end{equation}
where the additional terms are finite when $p^2 = -m_1^2$.
We expect $m_1^2$ to be a physical mass with $\bra{\Omega}\phi(0)\ket{1} = 1$.
Loop diagrams spoil this; we have to renormalise the theory to control the divergences.
