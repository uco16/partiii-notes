% lecture notes by Umut Özer
% course: aqft
\lhead{Lecture 23: March 10}

Let us now consider the pure gauge sector. We first look at the gauge loop \eqref{eq:22-gaugeloop}, which is built using the three-gauge-boson vertex \ref{eq:3-boson-vertex}. Working in Feynman gauge $\xi = 1$, this diagram evaluates to
\begin{align}
  \mathfrak{M}_3^{ab \mu \nu} &=
  \begin{gathered}
    \feynmandiagram[transform shape, scale=1][horizontal=a to b, layered layout] {
      a [particle=\(a\text{, } \mu\)] -- [gluon, momentum=$q$, insertion={[size=5pt]0.5}] b -- [half left, looseness=1, momentum=$p$, gluon, edge label'=$c \text{, } \rho$] c -- [edge label'=$d \text{, }\sigma$, half left, looseness=1, gluon, momentum=$p - q$] b,
      c -- [gluon, momentum=$q$, insertion={[size=5pt]0.5}] d [particle=\(b \text{, } \nu\)],
    };
  \end{gathered} \\
  &= \frac{g^2 \mu^{\epsilon}}{2} \int \frac{\dd[d]{p}}{(2 \pi)^d} \frac{1}{p^2 (p - q)^2} f^{adc} f^{cdb} N^{\mu\nu},
\end{align}
where we divided by the symmetry factor $2$, and defined
\begin{multline}
  N^{\mu\nu} = [\delta^{\mu\sigma} (2q + p)^{\rho} + \delta^{\sigma\rho} (2p-q)^{\mu} + \delta^{\rho\mu} ( - p - q)^{\sigma}] \\
  \times [\delta_{\rho\sigma} (2p - q)^{\nu} + \delta\indices{_{\sigma}^{\nu}} (2q - p)_{\rho} + \delta\indices{^{\nu}_{\rho}} (-q-p)_{\sigma} ]
\end{multline}
First we want to isolate the loop momentum $p^2$ in the denominator using Feynman's trick
\begin{equation}
  \frac{1}{p^2 (p - q)^2} \propto \int_0^1 \dd[]{x} \frac{1}{(l^2 - \Delta)^2},
\end{equation}
where $l = p + qx$ and $\Delta = -x (1 - x)q^2$.
We then turn the crank on the numerator.
\begin{remark}
  A calculation like this would be too long to ask in an exam.
\end{remark}
We can also evaluate $f^{acd} f^{bcd} = C_A \delta^{ab}$, where $C_A = C_2(G)$ is the \emph{quadratic Casimir operator} of the adjoint representation. The end result is of the form
\begin{equation}
  \label{eq:23-1}
  \mathfrak{M}_3^{ab \mu \nu} = \frac{g^2 C_A}{(4 \pi)^{d / 2}} \delta^{ab} \int_0^1 \frac{1}{\Delta^{2 - d / 2}} \left\{ \left[ \Gamma(1 - \frac{d}{2}) f(x) + \Gamma(2 - \frac{d}{2}) g(x) \right] \delta^{\mu\nu} q^2 - \Gamma(2 - \frac{d}{2}) h(x) q^{\mu} q^{\nu} \right\}.
\end{equation}
It is not obvious since the $x$-dependence is in the factor, but it turns out that we cannot factor out $(q^2 \delta^{\mu\nu} - q^{\mu} q^{\nu})$, which is what we would like to see for a massless gauge boson.
This is inconsistent with the Ward identity $q_{\mu} \mathfrak{M}_3^{ab \mu \nu} = 0$.
It looks like we have developed a massive mode for the boson. However, we know of course that this is not the case. 

It turns out that adding the four-gauge-vertex diagram
\begin{equation}
  \mathfrak{M}_4^{ab, \mu \nu} = 
  \begin{gathered}
    \feynmandiagram[transform shape, scale=1][horizontal=a to b, layered layout] {
      a [particle=\(\mu \text{, } a\)] -- [momentum'=$q$, gluon, insertion={[size=5pt]0.5}] b -- [edge label=$\rho \text{, } c$, momentum'=$p$, gluon, loop, min distance=2cm, in=135, out=45] b -- [gluon, insertion={[size=5pt]0.5}, momentum'=$q$] c [particle=\(\nu \text{, } b\)],
    };
  \end{gathered} = -g^2 C_A \delta^{ab} \delta^{\mu\nu} (d - 1) \int \frac{\bdd[d]{p}}{p^2}
\end{equation}
also does not help with the desired transverse Lorentz structure. In dimensional regularisation, the integral over $p$ yields zero as $d \to 4$.

The problem is fixed by realising that we simply have not yet added the ghost loop amplitude \eqref{eq:22-ghostloop}:
\begin{align}
  \mathfrak{M}_{\text{gh}}^{ab \mu \nu} &=
  \begin{gathered}
    \feynmandiagram[transform shape, scale=1][horizontal=a to b, layered layout] {
      a [particle=\(a \text{, } \mu\)] -- [gluon, momentum=$q$, insertion={[size=5pt]0.5}] b -- [edge label'=$c$, ghost, half left, looseness=1, momentum=$p$] c -- [edge label'=$d$, ghost, momentum=$p - q$, half left, looseness=1]b,
      c -- [gluon, momentum=$q$, insertion={[size=5pt]0.5}] d [particle=\(b \text{, } \nu\)],
    };
  \end{gathered} \\
  &= - \int \bdd[d]{p} (-g f^{cad} p^{\mu}) \frac{1}{p^2} (-g f^{dbc} (p-q)^{\nu}) \frac{1}{(p-q)^2} \\
  &=  g^2 C_A \delta^{ab} \int \bdd[d]{p} \frac{p^{\mu} (p - q)^{\nu}}{p^2 (p - q)^2}.
\end{align}
There is no symmetry factor in this case, but just as for fermions we obtain a minus sign for the Grassmann loop.
Again we use Feynman's trick on the denominator
\begin{equation}
  \int \frac{\bdd[d]{p}}{p^2 (p - q)^2} = \int_0^1 \dd[]{x} \frac{\bdd[d]{l}}{[l^2 + \Delta]^2},
\end{equation}
with $l^{\mu} = (p - x q)^{\mu}$ and $\Delta = q^2 x (1 - x)$.
Then we substitute $p^{\mu} = (l + x q)^{\mu}$ in the numerator, discarding any terms linear in $l^{\mu}$ and using $l^{\mu} l^{\nu} = \delta^{\mu\nu} l^2 / d$, giving
\begin{equation}
  p^{\mu} (p - q)^{\nu} = \delta^{\mu\nu} l^2 / d - x(1 - x) q^{\mu} q^{\nu} + \text{odd terms}.
\end{equation}
Finally, using the integral identities \eqref{eq:loop-identities}, we find
\begin{equation}
  \mathfrak{M}_{\text{gh}}^{ab \mu \nu} = \frac{g^2 C_a \delta^{ab}}{(4 \pi)^{d / 2}} \int_0^1 \frac{ \dd[]{x}}{\Delta^{2 - d / 2}} \left[ \Gamma(1 - \frac{d}{2}) \frac{1}{2} x(1 - x) \delta^{\mu\nu} q^2 - \Gamma(2 - \frac{d}{2}) x (1 - x) q^{\mu} q^{\nu} \right].
\end{equation}
Now we can combine this with the result of \eqref{eq:23-1}. After some tricks, we have the sum
\begin{equation}
  \label{eq:23-2}
  \mathfrak{M}_3^{ab \mu \nu} + \mathfrak{M}_{\text{gh}}^{ab \mu \nu} = \frac{g^2 C_a \delta^{ab}}{16 \pi^2} (\delta^{\mu\nu} q^2 - q^{\mu} q^{\nu}) \left[ \frac{5}{3} \left( \frac{2}{\epsilon}  + (\ln \frac{\mu^2}{\Delta}) \text{-term} + \text{(finite)} \right) \right].
\end{equation}
\begin{remark}
  This long calculation is not examinable.
\end{remark}
Also, rescaling
\begin{equation}
  \mathfrak{M}_F^{ab\mu\nu} = \frac{g^2 T_F \delta^{ab}}{16 \pi^2} (\delta^{\mu\nu} q^2 - q^{\mu} q^{\nu}) \left[ - \frac{4}{3} \frac{2}{\epsilon} + \text{log} + \text{finite} \right]
\end{equation}

We need a counterterm to renormalise the divergence \eqref{eq:23-2}
\begin{equation}
  \begin{gathered}
    \feynmandiagram[transform shape, scale=1][horizontal=a to b, layered layout, small] {
      a -- [gluon, insertion={[size=5pt]0.5}] b [square dot] -- [gluon, insertion={[size=5pt]0.5}] c,
    };
  \end{gathered}
  = \delta_3 = \frac{g^2}{16 \pi^2} \frac{2}{\epsilon} \left[ \frac{5}{3} C_A - \frac{4}{3} T_F n_f \right].
\end{equation}
For every different type of fermion in the theory we get essentially the same diagram, which is why we just added the multiplicative factor $n_f$, which is the number of fermions (flavour).

\subsection*{Renormalised Lagrangian}%

If we unpack the whole Lagrangian into individual terms, all of which in principle can be renormalised separately, we obtain
\begin{multline}
  \mathscr{L} = \frac{1}{4} Z_3 (\partial_{\mu} A_{\nu}^{a} - \partial_{\nu} A_{\mu}^{a})^2 + \frac{1}{2 \xi} (\partial^{\mu} A_{\mu}^{a})^2
  + Z_{3g} g f^{abc} (\partial_{\mu} A_{\nu}^{a}) A_{\mu}^{b} A_{\nu}^{c} + \frac{1}{4} Z_{4g} g^2 f^{abe} f^{cde} A_{\mu}^{a} A_{\nu}^{b} A^{\mu, c} A^{\nu, d} \\
  + Z_{2'} \overline{c}{} \partial^2 c - Z_{1'} g f^{abc} (\partial_{\mu} \overline{c}{}^{a}) A_{\mu}^{b} c^{c} \\
  +Z_2 \overline{\psi}{} \cancel{\partial} \psi + Z_{m} m \overline{\psi}{} \psi + Z_1 g \overline{\psi}{} \cancel{A}^{a} T^{a} \psi.
\end{multline}
The change to the first term is what we have been calculating in this section. The second term is not renormalised.  The first line is the pure-gauge part of the action, whereas the second line is the ghost and the third the fermion part.

\begin{claim}
  We can define some sort of effective coupling $q_{\text{eff}}^2$ that is equal to the combination
  \begin{equation}
    q_{\text{eff}}^2 = \frac{Z_1^2}{Z_2^2 Z_3} g^2 \mu^\epsilon = \frac{Z_{1'}^2}{Z_{2'}^2 Z_3} g^2  \mu^\epsilon = \frac{Z_{3g}^2}{Z_3^3} g^2 \mu^\epsilon = \frac{Z_{4g}}{Z_3^2} g^2 \mu^\epsilon.
  \end{equation}
\end{claim}
In QED, we used gauge invariance to write down the Ward--Takahashi identity. Here we do not have gauge invariance, but we have another kind of symmetry, called \emph{BRST symmetry}, which will help us.

\subsection{\texorpdfstring{$\beta$}{Beta}-Functions and Asymptotic Freedom}%

We obtain the $\beta$-functions from requiring 
\begin{equation}
  \mu \dv{\mu} g_{\text{eff}} = 0.
\end{equation}
We have $Z_3 = (1 + \delta_3)$.
Need $\delta_2$ and $\delta_1$ from the diagrams
\begin{equation}
  \begin{gathered}
    \feynmandiagram[transform shape, scale=1][horizontal=a to b, small, layered layout] {
      a -- b -- c -- d,
      b -- [half left, looseness=1.5, gluon] c,
    }; \\
    \delta_2
  \end{gathered}
  \qquad \& \quad
  \underbrace{\begin{gathered}
	       \feynmandiagram[transform shape, scale=0.8][horizontal=a to b, small] {
	         u -- [gluon] v -- a -- [gluon] b -- v,
	         c -- a,
	         b -- d,
	       };
	      \end{gathered}
	      +
	      \begin{gathered}
	       \feynmandiagram[transform shape, scale=0.8][horizontal=a to b, small] {
	         u -- [gluon] v -- [gluon] a -- b -- [gluon] v,
	         c -- a,
	         b -- d,
	       }; \\
	      \end{gathered}}_{\delta_1}
\end{equation}
To write
\begin{align}
  \beta(g) &= -\frac{\epsilon}{2} g - g \mu \dv{\mu} \left( \frac{Z_1}{Z_2 Z_3^{1 / 2}} \right) \\
	   &= g \left[ -\frac{\epsilon}{2} - \mu \dv{\mu} \left( \delta_1 - \delta_2 - \frac{1}{2} \delta_3 \right) \right].
\end{align}
where we use L.O. $\beta$-function to write
\begin{equation}
  \mu \dv{\mu} g = -\frac{\epsilon}{2} d.
\end{equation}
To get
\begin{equation}
  \beta(g) = -\frac{\epsilon}{2} -\frac{g^3}{16 \pi^2} \left( \frac{11}{3} c_A - \frac{4}{3} n_f T_F \right).
\end{equation}

For $SU(3)$ (and fundamental fermions)
\begin{equation}
  C_A = N = 3, \qquad T_F = \frac{1}{2}, \qquad d = 4.
\end{equation}
\begin{equation}
  \beta(g) = -\frac{g^3}{16 \pi^2} \left( 11 - \frac{2}{3} n_f \right) \coloneqq -\frac{g^3}{16 \pi^2} \beta_0.
\end{equation}
Now $g$ is marginally relevant (for $n_f < 16$). So $g \to \infty$ in the IR.
Conversely, $g \to 0$ at high energies. This phenomenon is called \emph{asymptotic freedom}.
Let $\Lambda_{QCD}$ be the \emph{low energy} scale where $g \to \infty$ (at this order)
\begin{equation}
  g^2(\mu) = \frac{1}{\beta_0 \ln \frac{\mu^2}{\Lambda_{QCD}^2}}.
\end{equation}
This asymptotic freedom is why we use non-Abelian gauge theories to describe the strong interactions.
