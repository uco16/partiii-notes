% lecture notes by Umut Özer
% course: aqft
\lhead{Lecture 20: March 03}

\chapter{Nonabelian Gauge Theory}%
\label{cha:nonabelian_gauge_theory}

\section{Lie Groups: Facts and Conventions}%
\label{sec:lie_groups_facts_and_conventions}

Let $G$ be a continuous, connected group.
For $U \in G$, we can write $U = \exp(i \theta^{a} T^{a}) \mathbb{1}$.
The Hermitian generators are $T^{a}$ and $\theta^{a}$ are numbers parametrising $U = U(\theta)$.
The index $a$ is running over the generators.
The generators $T^{a}$ form a Lie algebra under the Lie bracket
\begin{equation}
  [T^{a}, T^{b}] = i f^{abc} T^{c},
\end{equation}
with structure constants $f^{abc}$. We can and will choose a basis where $f^{abc}$ is antisymmetric.
We will think of matrix representations in which the Lie bracket is just the matrix commutator.
This obeys a Jacobi identity
\begin{equation}
  [A, [B, C]] + [B, [C, A]] + [C, [A, B]] =0.
\end{equation}
Which implies that the structure constants obey
\begin{equation}
  f^{abd} f^{dce} + f^{bcd} f^{dae} + f^{cad} f^{dbe} = 0.
\end{equation}
There is a normalisation for the structure constants, which we choose to be
\begin{equation}
  f^{acd} f^{bcd} = N \delta^{ab}.
\end{equation}

We classify Lie groups into unitary, orthogonal, symplectic, and the exceptional groups.
In these lectures, we focus on unitary Lie groups.
If $U \in U(N)$, then $U^{\dagger} U = \mathbb{1}$ and $U \in SU(N)$ satisfies $\det U = 1$.
$SU(N)$ has $N^2 - 1$ generators, so $\dim SU(N) = N^2 - 1$.

\subsection{Representations}%
\label{sub:representations}

\begin{description}
  \item[Fundamental:] Smallest, non-trivial representation of the Lie algebra.
    For $\mathfrak{su}(N)$, these are $N \times N$ traceless, Hermitian operators.
    For infinitesimal $SU(N)$ transformations, $\phi$ transforms in the fundamental representation as
    \begin{equation}
      \phi_i = \phi_i + i \alpha^{a} (T^{a} _{\text{fund}})_{ij} \phi_{j},
    \end{equation}
    where $1 \gg \alpha^{a} \in \mathbb{R}$. The indices $a = 1, \dots, N$ label the generators and $i, j = 1, \dots, N$ are particular to the representation we are in.
    The $T^{a} _{\text{fund}}$ are Hermitian.
    We will often drop the subscript on the fundamental representation: $T^{a} = T_{\text{fund}}^{a}$.
  \item[Antifundamental:] The generators are complex conjugate of the generators of the fundamental representation \begin{equation}(T^{a} _{\text{afund}}) = -(T_{\text{fund}}^{a})^*.\end{equation}
    Infinitesimal transformations in the antifundamental representation are
    \begin{equation}
      \phi_i^* \to \phi_i^* i + \alpha^{a} (T_{\text{afund}}^{a})_{ij} \phi^*_j = \phi_{i}^* - i \alpha^{a} \phi^*_j (T_{\text{fund}}^{a})_{ji}.
    \end{equation}
  \item[Adjoint:] Acts on vector space spanned by generator matrices:
    \begin{equation}
      (T_{\text{adj}}^{a})_{ij} = -i f^{aij}, \qquad a,i, j = 1, \dots, N^2 - 1.
    \end{equation}
    Gauge fields transform in the adjoint representation.
\end{description}

\subsection{Classifying Representations}%
\label{sub:classifying_representations}

\begin{definition}[index]
  The \emph{index} $T(R)$ of a representation $R$ is defined by an inner product
  \begin{equation}
    \Tr(T^{a}_R T^{b}_R) = T(R) \delta^{ab},
  \end{equation}
  where the trace is over the representation indices (rather than the generator indices).
\end{definition}
\begin{description}
  \item[fundamental:] 
  \begin{equation}
    T^{a}_{ij} T^{b}_{ji} = \frac{1}{2} \delta^{ab} \implies T(\text{fund}) = T_F = \frac{1}{2}.
  \end{equation}
\item[adjoint:] 
  \begin{equation}
    f^{acd} f^{bcd} = N \delta^{ab} \implies T(\text{adj}) = T_A = N.
  \end{equation}
\end{description}

\begin{definition}[quadratic Casimir]
  The \emph{quadratic Casimir} $C_2(R)$ is defined by
  \begin{equation}
    T^{a}_R T^{a}_R = C_2(R) \mathbb{1}.
  \end{equation}
\end{definition}
Comparing with the definition of the index, setting $a = b$ and summing gives
\begin{equation}
  T(R) \dim G = C_2 (R) \dim R.
\end{equation}
Therefore, we find that the Casimirs for the fundamental and adjoint representations for $SU(N)$ are
\begin{align}
  C_2 (\text{fund}) &= C_F = \frac{N^2 - 1}{2N}, \\
  C_2 (\text{adj}) &= C_A = N.
\end{align}

\section{Gauge Invariance and Wilson Lines}%
\label{sec:gauge_invariance_and_wilson_lines}

One approach would be to take QED and decorate the Lagrangian with indices that tell us under which representation these transform.
We will get there. However, we will start from a slightly more geometric approach, involving Wilson lines.

\subsection{Abelian Wilson Lines}%
\label{sub:abelian_wilson_lines}

Let us revisit QED, with fermions transforming under a local $U(1)$ transformation
\begin{equation}
  \label{eq:20-trans}
  \psi(x) \mapsto e^{i \alpha(x)} \psi(y), \qquad \overline{\psi}{}(x) \mapsto \overline{\psi}{} (x) e^{-i \alpha(x)}.
\end{equation}
Then $\overline{\psi}{} \gamma \psi$ is not invariant.
Consider the derivative in the direction of a unit vector $n^{\mu}$
\begin{equation}
  n^{\mu} \partial_{\mu} \psi = \lim_{a \to 0} \frac{1}{a} \left[ \psi(x + an) - \psi(x)\right].
\end{equation}
Then fields at different points $x$ and $x + an$ transform in a slightly different way
\begin{equation}
  \psi(x + an) - \psi(x) \mapsto e^{i \alpha(x + an)} \psi(x + an) - e^{i \alpha(x)} \psi(x).
\end{equation}
A gauge covariant derivative fixes this.
We want $D_{\mu} \psi$ to transform in the same ways as $\psi$:
\begin{equation}
  D_{\mu}\psi(x) \mapsto e^{i \alpha(x)}D_{\mu} \psi(x).
\end{equation}
We define a \emph{Wilson line} (a parallel transporter) $W(y, x)$ such that its gauge transformation is
\begin{equation}
  \label{eq:20-wilsonline}
  W(y, x) \mapsto e^{i \alpha(y)} W(y, x) e^{-i \alpha(x)}.
\end{equation}
With the convention $W(x, x) = 1$, the Wilson line becomes a phase
\begin{equation}
  W(y, x) = e^{i \phi(y, x)}, \qquad \phi \in \mathbb{R}.
\end{equation}
We also assume the convention that $W(x, y) = [W(y, x)]^*$.
We define $D_{\mu}$ such that
\begin{equation}
  \label{eq:20-1}
  n^{\mu} D_{\mu} \psi = \lim_{a \to 0} \frac{1}{a} \left[ \psi(x + an) - W(x + an, x) \psi(x) \right] .
\end{equation}
Then $\overline{\psi}{} \cancel{D} \psi$ is gauge invariant.

For infinitesimal $a$, define $A_{\mu}$ in the middle of the Wilson line via
\begin{equation}
  \label{eq:20-infwilson}
  W(x + an, x) = \exp[i e a n^{\mu} A_{\mu} (x + \frac{1}{2} an)].
\end{equation}
Taking the limit in \eqref{eq:20-1}, we find the \emph{gauge covariant derivative}
\begin{equation}
  D_{\mu} \psi(x) = [\partial_{\mu} - i e A_{\mu}(x)] \psi(x)
\end{equation}
which by construction transforms in the same way as the field
\begin{equation}
  D_{\mu} \psi \mapsto e^{i \alpha(x)} D_{\mu} \psi.
\end{equation}
From this we can find the gauge transformation of $A_{\mu}$ as
\begin{equation}
  \label{eq:20-gaugetransform}
  A_{\mu}(x) \mapsto A_{\mu}(x) + \frac{1}{e} \partial_{\mu} \alpha(x).
\end{equation}
We have seen that if a field has the local transformation law \eqref{eq:20-trans}, then its covariant derivative has the same transformation law. Thus the second covariant derivative of $\psi$, as well as its covariant derivative also transform according to \eqref{eq:20-trans}:
\begin{align}
  D_{\nu} (D_{\mu} \psi) &\mapsto e^{i \alpha(x)} D_{\nu}(D_{\mu} \psi) \\
  [D_{\mu}, D_{\nu}] \psi &\mapsto e^{i \alpha(x)} [D_{\mu}, D_{\nu}] \psi. \label{eq:20-com}
\end{align}
In particular, the commutator is not itself a derivative at all:
\begin{align}
  [D_{\mu}, D_{\nu}] \psi &= [\partial_{\mu}, \partial_{\nu}] \psi - i e ([\partial_{\mu}, A_{\nu}] - [\partial_{\nu}, A_{\mu}]) \psi - e^2 [A_{\mu}, A_{\nu}] \psi \\
			  &= -ie (\partial_{\mu} A_{\nu} - \partial_{\nu} A_{\mu}) \psi.
\end{align}
Since the factor $\psi$ on the right-hand side accounts for the entire phase $e^{i \alpha(x)}$ in the transformation law \eqref{eq:20-com}, the field strength $F_{\mu\nu}$ defined as the curvature of the covariant derivative
\begin{equation}
  [D_{\mu}, D_{\nu}] = -i e (\partial_{\mu} A_{\nu} - \partial_{\nu} A_{\mu}) = i e F_{\mu\nu}
\end{equation}
must be gauge invariant in the case of Abelian gauge theory.

\subsection{Abelian Wilson Loops}%
\label{sub:abelian_wilson_loops}

Wilson lines need not be infinitesimal and the field $A_{\mu}(x)$ not continuous. Let us reverse the logic of the previous section and start from a connection $A_{\mu}$, assumed to have the transformation law \eqref{eq:20-gaugetransform}. One can then show that the expression
\begin{equation}
  \label{eq:20-line}
  W_C(z, y) = \exp[i e \int_C \dd[]{x^{\mu}} A_{\mu}(x)]
\end{equation}
transforms according to the defining relation \eqref{eq:20-wilsonline} if the integral is taken along a path $C$ that runs from $y$ to $z$. Then $W_C(z, y)$ is called the \emph{Wilson line}.
A \emph{Wilson loop} is a closed Wilson line $W_C(z, z)$, which starts and ends at the same point $y = z$. The Wilson loop $C$ encloses a surface $\Sigma$ as shown in \ref{fig:l20f3}. By Stokes' theorem, we can rewrite the Wilson loop as
\begin{equation}
  \label{eq:20-loop}
  W_C(z, z) = \exp[i e \oint_{C} A_{\mu} \dd[]{x^{\mu}}] = \exp[\frac{i e}{2} \int_\Sigma F_{\mu\nu} \dd[]{\sigma^{\mu\nu}}],
\end{equation}
where $\dd[]{\sigma^{\mu\nu}}$ is an area element on the surface $\Sigma$.  We see that the field strength $F_{\mu\nu}$ appears from gauge invariant \emph{Wilson loops}.
Conversely, since (almost) all gauge-invariant functions of $A_{\mu}$ can be built up from $F_{\mu\nu}$, this suggests that the Wilson loop $W_C(y, y)$ is the most general gauge invariant.

\begin{figure}[tbhp]
  \centering
  \begin{minipage}[t]{0.4\columnwidth}
    \centering
    \inkfig[0.5]{plaquette}
    \caption{Plaquette}
    \label{fig:plaquette}
  \end{minipage}%
  \begin{minipage}[t]{0.4\columnwidth}
    \centering
    \inkfig[0.4]{l20f3}
    \caption{Wilson loop $C$}
    \label{fig:l20f3}
  \end{minipage}
\end{figure}
\begin{example}[infinitesimal plaquette]
  Take the Wilson loop to be a square, called a \emph{plaquette}, as shown in Fig.~\ref{fig:plaquette}.
  Taking the side length $a$ to be infinitesimal, we can use the infinitesimal expression \eqref{eq:20-infwilson} for the Wilson line to expand the infinitesimal loop
  \begin{align}
    P_{12} (x) &= W(y_1, y_4) W(y_4, y_3) W(y_3, y_2) W(y_2, y_1) \\
    &\approx 1 - i e a^2 F_{12}(x) + O(a^3).
  \end{align}
  Since the Wilson loop is gauge invariant, this is another way to view the gauge invariance of $F_{\mu\nu}$ geometrically.
\end{example}
