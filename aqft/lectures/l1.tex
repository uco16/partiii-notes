% lecture notes by Umut Özer
% course: aqft
\lhead{Lecture 1: January 18}

\chapter{Path Integrals in QM}%
\label{cha:path_integrals_in_qm}

\begin{leftbar}
  The lecturer's notes can be found online at \url{www.damtp.cam.ac.uk/user/wingate/AQFT}.
\end{leftbar}

\begin{description}
  \item[Goal:] Schr\"odinger's equation $\to$ path integral
\end{description}

Consider a Hamiltonian in one dimension $\hat{H} = H(\hat{x}, \hat{p})$, where position and momentum operators satisfy the common commutation relations $[\hat{x}, \hat{p}] = i \hbar$ .
Assume the it takes the form
\begin{equation}
  \hat{H} = \frac{\hat{p}^2}{2m} + V(\hat{x}).
\end{equation}
Schr\"odinger's equation then says that the time evolution of a state $\ket{\psi(t)}$  is given by
\begin{equation}
  i\hbar \frac{\partial }{\partial t} \ket{\psi(t)} = \hat{H} \ket{\psi(t)}.
\end{equation}
This has a formal solution, giving us the time-evolution operator
\begin{equation}
  \ket{\psi(t)} = e^{-i \hat{H} t / \hbar} \ket{\psi(0)}.
\end{equation}

In the Schr\"odinger picture, the states are evolving in time whereas operators and their eigenstates are constant in time.
\begin{definition}[wavefunction]
  $\Psi(x, t) \coloneqq \bra{x} \ket{\psi(t)}$
\end{definition}
The Schr\"odinger equation then becomes
\begin{equation}
  \bra{x}\hat{H} \ket{\psi(t)} = \left( -\frac{\hbar^2}{2m} \frac{\partial^2 }{\partial x^2} + V(x) \right) \Psi(x, t).
\end{equation}
We will turn this differential equation into an integral equation, where we will sum over particle paths---a path integral.
We can introduce an integral by inserting a complete set of states $1 = \int dx_0 \ket{x_0}\bra{x_0}$.
\begin{align}
  \Psi(x, t) &= \bra{x}e^{-i \hat{H} t / \hbar} \ket{\psi(0)} \\
	     &= \int_{-\infty}^{\infty} \dd[]{x_0} \bra{x} e^{-i \hat{H} t / \hbar} \ket{x_0} \bra{x_0} \ket{\psi(0)} \\
	     &\coloneqq \int_{-\infty}^{\infty}\dd[]{x_0} \underbrace{K(x, x_0; t)}_{\mathclap{\text{`kernel'}}} \Psi(x_0, 0)
\end{align}
We repeat this insertion for $n$ intermediate times and positions.
\begin{notation}[]
  Let $0 \coloneqq t_0 < t_1 < \dots < t_n < t_{n+1} \coloneqq T$
\end{notation}
And we also want to factor the exponential into $n$  terms:
\begin{equation}
  e^{i \hat{H} T / \hbar} = e^{-\frac{i}{\hbar} \hat{H} (t_{n+1} - t_n)} \cdots e^{-\frac{i}{\hbar} \hat{H} (t_{1} - t_0)}.
\end{equation}
Then
\begin{equation}
  K(x, x_0; T) = \int_{-\infty}^{\infty} \left[ \prod_{r =1}^n \dd[]{x_r} \bra{x_{r+1}} e^{-\frac{i}{\hbar} \hat{H} (t_{r+1}-t_r)} \ket{x_r} \right] \bra{x_1}e^{-\frac{i}{\hbar} \hat{H}(t_1 - t_0)}\ket{x_0}
\end{equation}
Integrals are over all possible position eigenstates at times $t_r, r = 1, \dots, n$.
%F1

\subsection*{Free Theory}%

Consider the ``free'' theory, with $V(\hat{x}) = 0$ .
We will now play a similar but different trick to what we did before. Let us insert a complete set of momentum eigenstates $1 = \int_{-\infty}^{\infty}\bdd[]{p} \ket{p}\bra{p}$. We also note that these momentum eigenstates are plane waves $\bra{x}\ket{p} = \frac{1}{\sqrt{2 \pi \hbar}} e^{i p x / \hbar}$ .

\begin{definition}[barred differential]
  We define the normalised differential in Fourier space to be
  \begin{equation}
    \bdd{p} \coloneqq \frac{\dd[]{p}}{2 \pi \hbar}.
  \end{equation}
  In higher dimensional QFT, this generalises to
  \begin{equation}
    \bdd[n]{p} \coloneqq \frac{\dd[n]{p}}{(2 \pi \hbar)^n}
  \end{equation}
\end{definition}

The corresponding kernel is
\begin{align}
  K_0(x, x'; t) &= \bra{x}\exp(\frac{i}{\hbar} \frac{\hat{p}^2}{2m} t) \ket{x'}. \\
		&= \int_{-\infty}^{\infty}\bdd[]{p} e^{-i p^2 t / 2m \hbar} e^{i p (x - x') / \hbar} \\
		&= \sqrt{\frac{m}{2 \pi i \hbar t}} e^{\frac{im (x - x')^2}{2 \hbar t}}. \label{eq:1-kernel}
\end{align}
\begin{remark}
  \begin{equation}
    \lim_{t \to 0} \left\{ K_0(x, x'; t) \right\} = \delta(x - x').
  \end{equation}
\end{remark}
As expected from $\bra{x} \ket{x'} = \delta(x - x')$ .

From the Baker-Campbell-Hausdorff formula, we know that
\begin{align}
  e^{\epsilon \hat{A}} e^{\epsilon \hat{B}} &= \exp(\epsilon \hat{A} + \epsilon \hat{B} + \frac{\epsilon^2}{2} [\hat{A}, \hat{B}] + \dots) \neq e^{\epsilon(\hat{A} + \hat{B})} \\
  \text{for small $\epsilon$:} \qquad e^{\epsilon(\hat{A} + \hat{B})} &= e^{\epsilon \hat{A}} e^{\epsilon \hat{B}} (1 + O(\epsilon^2))
\end{align}
Letting $\epsilon = 1 / n$ and raising the above to the  $n$\textsuperscript{th} power\footnote{This step is sometimes called Suzuki--Trotter decomposition.} gives
\begin{equation}
  e^{\hat{A} + \hat{B}} = \lim_{n \to \infty} \left\{ e^{\hat{A} / n} e^{\hat{B} / n} \right\}^n.
\end{equation}
We will use this to separate kinetic and potential terms.

Take $t_{r+1} -t_r = \delta t$  with $\delta t \ll T$  and $n$ large such that $n \delta t = T$.
 \begin{equation}
   e^{-\frac{i}{\hbar} \hat{H} \delta t} = \exp(-\frac{i \hat{p}^2 \delta t}{2 m \hbar}) \exp(-\frac{i V(\hat{x}) \delta t}{\hbar}) [1 + O(\delta t^2)]
\end{equation}
Using the result \eqref{eq:1-kernel}, 
\begin{align}
  \bra{x_{r +1}} \exp(- \frac{i \hat{H}}{\hbar} \delta t)\ket{x_r} &= e^{-i V(x_r) \delta t / \hbar} K_0 (x_{r+1}, x_r ; \delta t) \\
  &= \sqrt{\frac{m}{2 \pi i \hbar \delta t}} \exp[\frac{i m}{2 \hbar} \left( \frac{x_{r+1} - x_r}{\delta t} \right)^2 \delta t - \frac{i}{\hbar} V(x_r) \delta t]
\end{align}
With $T = n \delta t$
 \begin{equation}
   K(x, x_0; T) = \int \left[ \prod_{r = 1}^n \dd[]{x_r} \right] \left( \frac{m}{2 \pi i \hbar \delta t} \right)^{\frac{n + 1}{2}} \exp{i \sum_{r = 0}^n \left[ \frac{m}{2 \hbar}\left( \frac{x_{r + 1} - x_r}{\delta t} \right)^2 - \frac{1}{\hbar} V(x_r) \right] \delta t}
\end{equation}
In the limit $n \to \infty$ , $\delta t \to 0$ with  $n \delta t = T$  fixed, the exponent becomes
\begin{equation}
  \frac{1}{\hbar}\int_0^T \dd[]{t} \left[ \frac{1}{2} m \dot{x}^2 - V(x) \right] = \int_0^T \dd{t} L(x, \dot{x}),
\end{equation}
where $L$  is the classical Lagrangian, the Legendre transformation of the classical Hamiltonian.
The classical action is $S = \int \dd[]{t} L(x, \dot{x})$.

The main result therefore is that the path integral for the kernel is
\begin{equation}
  \boxed{K(x, x_0; t) \coloneqq \bra{x}e^{-i \hat{H} t / \hbar} \ket{x_0} = \int \pdd{x} e^{\frac{i}{\hbar} S}}
\end{equation}

\begin{definition}[functional integral]
  \begin{equation}
    \pdd{x} = \lim_{\substack{\delta t \to 0 \\ n \delta T \text{ fixed}}} \left\{ (\sqrt{...}) \prod_{r = 1}^n (\sqrt{...} d x_r) \right\}
  \end{equation}
\end{definition}
We do not need to care about normalization factors.

\begin{remark}
  In the limit $\hbar \to 0$, the interference of amplitudes is dominated by the ones close to the extremal path where $\delta S$. This leads to Hamilton's principle of least action.
\end{remark}
