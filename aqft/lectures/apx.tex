% lecture notes by Umut Özer
% course: aqft
\lhead{Appendix}

\begin{appendices}
  
\chapter{Variational Calculus}%
\label{apx:variational_calculus}

\section{Variations of the Lagrangian}%
\label{sec:variations_of_the_lagrangian}

Let us make a bit more precise what we mean when we talk about the variations of the functional $\mathscr{L}$ with respect to the function $\phi$.  Say we have a small variation $\phi \to \phi + \delta \phi$.  Then, for a Lagrangian $\mathscr{L}(x) = \mathscr{L}[\phi(x), \partial \phi(x)]$, we have the variation
\begin{equation}
  \delta \mathscr{L}(x) = \left.\frac{\partial \mathscr{L}}{\partial \phi}\right\rvert_x \delta \phi(x) + \left. \frac{\partial \mathscr{L}}{\partial (\partial_{\mu} \phi)} \right\rvert_{x} \partial_{\mu} (\delta \phi(x)).
  \label{eq:19-star}
\end{equation}
From this we write
\begin{equation}
  \frac{\delta \mathscr{L}(x)}{\delta \phi(y)} = \left. \frac{\delta \mathscr{L}}{\delta \phi} \right\rvert_{x} \delta^{(4)} (y - x) + \left. \frac{\delta \mathscr{L}}{\delta (\partial _{\mu} \phi)} \right\rvert_{x} \partial_{\mu} \delta^{(4)} (y - x).
\end{equation}

\section{Variations of the Action}%
\label{sec:variations_of_the_action}

The action is given by the integral $S[\phi] = \int \dd[4]{x} \mathscr{L}[\phi(x), \partial_{\mu} \phi(x)]$. We say that $S[\phi]$ is a functional, since it depends only on the function $\phi(x)$, not on the position $x$ itself.  The functional derivative of $S$ with respect to $\phi(y)$ is defined by the associated variation \eqref{eq:19-star} of the function $\mathscr{L}$ under the integral:
\begin{align}
  \frac{\delta S [\phi]}{\delta \phi(y)} &= \int \dd[4]{x} \frac{\delta \mathscr{L}(x)}{\delta \phi(y)} \\
				      &= \int \dd[4]{x} \left[ \left. \frac{\partial \mathscr{L}}{\partial \phi} \right\rvert_{x} \delta^{(4)} (x - y) 
				      + \left. \frac{\partial \mathscr{L}}{\partial (\partial_{\mu} \phi)} \right\rvert_{x}  \partial_{\mu} \delta^{(4)}(x - y) \right] \\
				      &= \left. \frac{\partial \mathscr{L}}{\delta \phi} \right\rvert_{y} - \left. \partial_{\mu} \left( \frac{\partial \mathscr{L}}{\partial (\partial_{\mu} \phi)} \right) \right\rvert_{y}, \label{eq:19-svar}
\end{align}
where we integrated by parts in the second line to deal with the derivative of the delta function.
\begin{remark}
  The functional derivative is again a function of the coordinate $y$.
\end{remark}
This is familiar from classical field theory, where the equations of motion are such as to minimise the action $S$. This requires the left-hand side to vanish, giving the Euler--Lagrange equations. In quantum field theory the action is not exactly minimised, and the Schwinger--Dyson equations allow us to quantify this.

\section{Relation to Schwinger--Dyson}%
\label{sec:relation_to_schwinger_dyson}

Let us now insert \eqref{eq:19-svar} back into \eqref{eq:19-star}.  We obtain
\begin{align}
  \delta \mathscr{L} &= \left[ \partial_{\mu} \left( \frac{\partial \mathscr{L}}{\partial (\partial_{\mu} \phi)} \right) + \frac{\delta S}{\delta \phi} \right] \delta\phi 
  + \frac{\partial \mathscr{L}}{\partial (\partial_{\mu} \phi)} \partial_{\mu} (\delta\phi) \\
			 &= \partial_{\mu} \left( \frac{\partial \mathscr{L}}{\partial (\partial_{\mu} \phi)} \delta\phi \right) + \frac{\delta S}{\delta \phi} \delta\phi \label{eq:19-pent} \\
			 &= \partial_{\mu} j^{\mu} + \frac{\delta S}{\delta \phi} \delta \phi, 
			 \label{eq:19-cur}
\end{align}
where we identified the Noether current
\begin{equation}
  j^{\mu} = \frac{\partial \mathscr{L}}{\partial (\partial_{\mu} \phi)} \delta \phi.
\end{equation}
We can then rearrange \eqref{eq:19-pent} to obtain the functional derivative of $S$ in terms of the functional derivative of $\mathscr{L}$:
\begin{equation}
  \frac{\delta S}{\delta \phi} = \frac{\delta \mathscr{L}}{\delta \phi} - \partial_{\mu} \left( \frac{\partial \mathscr{L}}{\partial (\partial_{\mu} \phi)} \right).
\end{equation}
Moreover, from \eqref{eq:19-cur} we can see that if $\delta \mathscr{L} = 0$ under a transformation $\phi \to \phi + \delta \phi$, then
\begin{equation}
  \label{eq:apx-cur}
  \frac{\delta S}{\delta \phi} \delta \phi = -\partial_{\mu} j^{\mu}.
\end{equation}

\chapter{Loop Integral Identities}%
\label{apx:loop_integral_identities}

To evaluate loop integrals, we often deal with exactly the same tricks. We list them here for convenience. We have added useful identities from \cite[Apx.~B]{schwartz} to the ones given in class.

\section{Euler's Gamma and Beta Functions}%
\label{sec:gamma_function}

\begin{description}
  \item[Euler-$\Gamma$ function:]
    Use, for $\alpha > 0$, the analytic continuation of the factorial
    \begin{equation}
      \Gamma(\alpha) = \int_0^\infty \dd[]{x} x^{\alpha -1} e^{-x}, \qquad 
      \alpha \Gamma(\alpha) = \Gamma(\alpha + 1), \qquad 
      \Gamma(1) = 1, \qquad 
      \Gamma\left(\frac{1}{2}\right) = \sqrt{\pi}.
    \end{equation}
    This has the series expansion
    \begin{equation}
      \ln \Gamma(\alpha + 1) = -\gamma \alpha - \sum_{k=2}^{\infty} (-1)^k \frac{1}{k} \zeta(k),
    \end{equation}
    where $\gamma = \gamma_E \approx 0.577216$  is the Euler--Mascheroni constant and $\zeta(k) = \sum_{n=1}^{\infty} \frac{1}{n^k}$ is the Riemann $\zeta$-function.
    Usually we exponentiate this
    \begin{equation}
      \Gamma(\epsilon) = \frac{1}{\epsilon} - \gamma + O(\epsilon).
    \end{equation}

  \item[Euler-beta function:]
    \begin{equation}
      B(s, t) = \int_{0}^{1}\dd[]{x} u^{s-1} (1-u)^{t-1} = \frac{\Gamma(s) \Gamma(t)}{\Gamma(s + t)}.
    \end{equation}

  \item[Surface area $S_d$ of a unit $d$-sphere:]
    For integer dimension $d \in \mathbb{N}$, we can do $d$ Gaussian integrals and convert to polar coordinates
    \begin{equation}
      (\sqrt{\pi})^d = \int_{\mathbb{R}^d} \prod_{i=1}^d \dd[]{x_i} e^{-x_i^2} = S_d \int_0^{\infty} \dd[]{r} r^{d-1} e^{-r^2} = \frac{1}{2} S_d \, \Gamma(\frac{d}{2}).
    \end{equation}
    For non-integer $d \in \mathbb{C}$, we define $S_d$ via analytic continuation as
    \begin{equation}
      \label{eq:apx-sd}
      S_d = \frac{2 \pi^{d / 2}}{\Gamma(\flatfrac{d}{2})}.
    \end{equation}
\end{description}
\begin{remark}
  You will not be asked to prove these or even have these at hand in the exam.
\end{remark}


\section{Feynman Parameters}%
\label{sec:feynman_parameters}

\begin{equation}
  \frac{1}{A B} = \int_0^1 \frac{\dd[]{x}}{[A + (B - A) x]^2} = \int_0^1 \dd[]{x} \dd[]{y} \frac{\delta(x + y - 1)}{[x A + y B]^2}
\end{equation}
We often use these to complete the square in the denominator. For example
\begin{align}
  \int \frac{\bdd[4]{k}}{k^2 (k - p)^2} &= \int \bdd[4]{k} \int_0^1 \frac{\dd[]{x}}{[k^2 + k [(k - p)^2 - k^2]]^2} \\
					&= \int_0^1 \dd[]{x} \frac{\bdd[4]{k}}{[(k - xp)^2 - \Delta]^2},
\end{align}
where $\Delta = - p^2 x (1 - x)$. We can then shift $k \to k + xp$, leaving an integral that only depends on $k^2$.

\section{Scalar Integrals}%
\label{sec:scalar_integrals}

Once we have manipulated the loop integrals so that they are only functions of the magnitude of the momentum, we use
\begin{equation}
  \int \dd[d]{k} = S_d \int k^{d - 1} \dd[]{k},
\end{equation}
where $S_d$ is given in \eqref{eq:apx-sd}.
The resulting integrals over the (as usual Euclidean) momenta are evaluated as
\begin{equation}
  \int \dd[]{k} \frac{k^a}{(k^2 + \Delta)^b} = \Delta^{\frac{a + 1}{2} - b} \frac{\Gamma \left( \frac{a + 1}{2} \right) \Gamma \left( b - \frac{a + 1}{2} \right)}{2 \Gamma(b)}.
\end{equation}

\end{appendices}
