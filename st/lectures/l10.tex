% lecture notes by Umut Özer
% course: st
\lhead{Lecture 10: February 10}

We shall refer to the (Grassmann) fields $b_{ab}$ and $c^a$ as Fadeev--Popov  \emph{ghosts}.

\section{Calculating Observables}%
\label{sec:calculating_observables}

Our final form for $Z[0]$  for Riemann surfaces of genus $g$ is
\begin{equation}
  Z_g[0] = \frac{1}{\abs{\text{CKG}}} \int_{\mathrlap{M_g}} \dd[3]{t} \int \pdd{X} \pdd{b} \pdd{c} e^{i S[X, b, c]} \prod_{I} (b \mid \mu_I) \prod_{i, a} c^a(\hat \sigma_i).
\end{equation}
In general, we want to sum over all Riemann surfaces
\begin{equation}
  Z[0] = \sum_{g=0}^{\infty} e^{\lambda \chi} Z_g[0],
\end{equation}
where $\lambda$ is a constant and $\chi = lg - l$ is the Euler characteristic of the worldsheet.
It will turn out that the constant weighting $\lambda$  will be determined by the theory rather than being a free parameter.

We can compute correlation functions of observables by including them in our path integral
\begin{equation}
  \langle \phi_1 \dots \phi_n \rangle = \mathcal{N} \int \pdd{\phi} \phi_1 \dots \phi_n e^{i S[\phi]}
\end{equation}
What might our observables look like?
They will certainly need to be invariant under diffeomorphisms and Weyl transformations.
We can build diffeomorphism-invariant observables by taking an operator $\mathcal{O}(\sigma, \tau)$  on a genus $g$  worldsheet $\Sigma_g$  and integrating it over $\Sigma_g$:
 \begin{equation}
   \mathcal{O} = \int_{\Sigma_g} \dd[2]{\sigma} \mathcal{O}(\sigma, \tau).
\end{equation}
We will need to choose the $\mathcal{O}(\sigma, \tau)$  carefully in order for this to be Weyl invariant as well. We will talk about this when we come to CFTs.

So a correlation function of such observables would be:
\begin{equation}
  \langle \mathcal{O}_1 \dots \mathcal{O}_n\rangle = \sum_{g=0}^{\infty} \frac{e^{\lambda \chi}}{\abs{\text{CKG}}} \int \prod_{i=0}^{n} \dd[2]{\sigma_i} \int_{\mathrlap{M_g}} \dd[3]{t} \int \pdd{ X} \pdd{b} \pdd{c} e^{i S[X, b , c]} \prod_I (b \suchthat \mu_I) \prod_{i, a} c^a(\hat{\sigma}_i) \mathcal{O}_1(\sigma_1) \dots \mathcal{O}_n(\sigma_n),
\end{equation}
using the shorthand $\sigma_i = (\sigma_i, \tau_i)$.

\begin{remark}
  We have the instruction to divide by the conformal Killing group, and to integrate over the moduli space of Riemann surfaces $M_g$ as well as all locations $\sigma_i$. 
  We know we can fix the CKG on a sphere by keeping three locations fixed, and one on a torus.
  Instead of integrating over all points of the Riemann surface, we can thus write the instruction of dividing by the CKG as restricting what we integrate over:
  \begin{equation}
    \frac{1}{\text{CKG}} \int_{\mathrlap{M_g}} \dd[3]{t} \int \prod_{i=1}^n \dd[2]{\sigma_i} = \int_{\mathrlap{M_g}} \dd[3]{t} \int_{i=1}^{n-\kappa} \dd[2]{\sigma_i} = \int_{M_{g, n}}.
  \end{equation}
\end{remark}

So when computing the correlation functions, we end up integrating only over $n - \kappa$ locations
\begin{equation}
  \boxed{\langle \mathcal{O}_1 \dots \mathcal{O}_n \rangle = \sum_{g=0}^{\infty} e^{\lambda \chi} \int_{M_{g, n}} \int \pdd{X} \pdd{b} \pdd{c} e^{i S[X, b, c]} \int_I (b \suchthat \mu_I) \prod_{i, a} c^a (\hat{\sigma}_i) \mathcal{O}_1 \dots \mathcal{O}_n}
\end{equation}
This might seem like the final result depends on some locations, but as we will see the ghosts will take care of this.

\section{Conformal Field Theory}%
\label{sec:conformal_field_theory}

This is a wonderful topic. There could be a whole Part III course in CFT alone. It is important in String Theory but also in Statistical and Condensed Matter Physics.
Moreover, when trying to really understand QFT with mathematical rigour, your best bet is to start with low-dimensional CFT.

\subsection{Conformal Invariance in General Dimension}%
\label{sub:conformal_invariance_in_general_dimension}

Let us consider transformations that preserve angles. In particular, transformations $x^{\mu} \to x'{}^{\mu}(x)$  such that the metric on our space transforms as 
\begin{equation}
  \Lambda(x) \, \eta_{\mu\nu} = \eta_{\rho\sigma} \frac{\partial x'{}^{\rho}}{\partial x^{\mu}} \frac{\partial x'{}^{\sigma}}{\partial x^{\nu}},
\end{equation}
preserving the metric up to a local scale $\Lambda(x)$ .

Infinitesimally, we write a transformation as $x^{\mu} \to x^{\mu}(x) = x^{\mu} + \epsilon v^{\mu}(x) + \dots$ , where $\epsilon \ll 1$.
The first order in $\epsilon$, the metric transforms as
 \begin{equation}
  \eta_{\mu\nu} \to \eta_{\mu\nu} + \epsilon(\partial_{\mu} v_{\nu} + \partial_{\nu} v_{\mu}) = \Lambda(x) \, \eta_{\mu\nu}.
\end{equation}
Let $\Lambda(x) = e^{\epsilon \omega(x)} \approx 1 + \epsilon \omega(x) + \dots$ .
Thus, vector fields $v^{\mu}$ must satisfy
\begin{equation}
  \partial_{\mu} v_{\nu} + \partial_{\nu} v_{\mu} = \omega(x)\, \eta_{\mu\nu}.
\end{equation}
Taking the trace of both sides,
\begin{equation}
  \omega(x) = \frac{2}{d} \partial_{\mu} v^{\mu},
\end{equation}
where $d$ is the dimension few the spacetime.
Therefore, vector fields that generate such conformal transformations satisfy
\begin{equation}
  \label{eq:10-star}
  \partial_{\mu} v_{\nu} + \partial_{\nu} v_{\mu} = \frac{2}{d} \partial_{\lambda} v^{\lambda} \eta_{\mu\nu}.
\end{equation}

\subsection{In Two Dimensions}%
\label{sub:in_two_dimensions}

Something special happens in $d = 2$ dimensions.
Let us take the coordinates in two dimensions to be $\sigma$ and $\tau$. We take $\eta$ to be the Euclidean metric, $\text{diag}(1, 1)$.
With this metric, Eq.~\eqref{eq:10-star} boils down to two independent equations
\begin{equation}
  \boxed{\frac{\partial v_\tau}{\partial \tau} = \frac{\partial v_{\sigma}}{\partial \sigma}, \qquad \frac{\partial v_{\sigma}}{\partial \tau} = - \frac{\partial v_{\tau}}{\partial \sigma}}
\end{equation}
where $v = 
\begin{pmatrix}
v_{\sigma} \\
v_{\tau} \\
\end{pmatrix}$.
These are the Cauchy--Riemann equations!
\begin{remark}
  We can now bring to bear the full power of complex analysis to understand quantum gravity.
\end{remark}
If we introduce complex coordinates
\begin{equation}
  w = \tau + i \sigma, \qquad \overline{w}{} = \tau -i \sigma, \qquad (\omega \in \mathbb{C}),
\end{equation}
then the condition for the vector $v$ to generate a conformal transformation is that $v$ is holomorphic, meaning that $\overline{\partial}{}v = 0$.

If a theory is conformally invariant, then there is a conserved charge associated with every holomorphic $v$; there is an infinite number of conserved charges!
\begin{remark}
  Given our current understanding, this is special in $d = 2$.
  This allows us to do quantum field theory with an easy that we just do not see in higher dimensions.
\end{remark}

A particularly useful set of coordinates for our worldsheet is the exponential of $w$:
\begin{equation}
  \label{eq:z}
  z = e^{\tau + i \sigma}, \qquad \overline{z}{} = e^{\tau - i \sigma}.
\end{equation}
Under this map, the cylindrical worldsheet is mapped to the complex plane (if we choose a Euclidean metric, which we will do from now on).

\begin{figure}[tbhp]
  \centering
  \def\svgwidth{0.4\columnwidth}
  \input{lectures/l10f1.pdf_tex}
  \caption{}
  \label{fig:l10f1}
\end{figure}

\begin{figure}[tbhp]
  \centering
  \def\svgwidth{0.4\columnwidth}
  \input{lectures/l10f2.pdf_tex}
  \caption{}
  \label{fig:l10f2}
\end{figure}

Cylinder becomes radial evolution on the complex plane.
