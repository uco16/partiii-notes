% lecture notes by Umut Özer
% course: st
\lhead{Lecture 14: February 19}

Similarly, swapping $z \leftrightarrow w$ and $\mu \leftrightarrow \nu$ in \eqref{eq:13-j},
\begin{equation}
  j^{\nu}(w) j^{\mu}(z) = \normalorder{j^{\mu}(z) j^{\nu}(w)} + \frac{\eta^{\mu\nu}}{(z - w)^2},  \qquad \abs{z} < \abs{w}.
\end{equation}

Putting these together, 
\begin{equation}
  \mathcal{R} (j^{\mu}(z) j^{\nu}(w)) = \normalorder{j^{\mu}(z) j^{\nu}(w)} + \frac{\eta^{\mu\nu}}{(z-w)^2}.
\end{equation}
We often write
\begin{equation}
  \wick{\c j^{\mu}(z) \c j^{\nu}(w)} = \frac{\eta^{\mu\nu}}{(z - w)^2}
\end{equation}
and refer to this as a \emph{contraction} between $j^{\mu}(z)$ and $j^{\nu}(w)$.

For example, if we write $\partial X ^{\mu}(z) = -i \sqrt{\frac{\alpha'}{2}} j^{\mu}(z)$, we have that
\begin{equation}
  \wick{\c {\partial_z X^{\mu}(z)} \c {\partial_w X^{\nu}(w)}} = -\frac{\lambda'}{2} \frac{\eta^{\mu\nu}}{(z-w)^2}.
\end{equation}
Integrating
\begin{equation}
  \boxed{\wick{\c X^{\mu}(z) \c X^{\nu}(w)} = -\frac{\alpha'}{2} \eta^{\mu\nu} \ln (z-w)}
\end{equation}
and there is of course an analogous statement for the antiholomorphic sector. 
Here we are splitting $X^{\mu}(z, \overline{z}{}) = X^{\mu}(z) + \overline{X}{}^{\mu}(\overline{z}{})$.
\begin{remark}
  $\partial \overline{\partial}{} X^{\mu}(z, \overline{z}{}) = 0$
\end{remark}

The generalisation of these statements are given by \emph{Wick's theorem}:
\begin{align}
  \mathcal{R} l(\phi_1(z_1) \dots \phi_n (z_n)) &= \normalorder{\phi_1(z_1) \dots \phi_n(z_n)} \\
						&+ \sum_{(i, j)} \normalorder{\phi_1 (z_1) \dots \wick{\c \phi_i(z_i) \dots \c \phi_j (z_j)} \dots \phi_n(z_n)} \\
						&+ \sum_{(i, j)(k, l)} \normalorder{\phi_1 (z_1) \dots \wick{\c \phi_i(z_i) \dots \c \phi_j (z_j)} \dots \wick{\c \phi_k(z_k) \dots \c \phi_l (z_l)} \dots \phi_n(z_n)}  \\
						&+ \dots.
\end{align}

\section{Operator Product Expansions}%
\label{sec:operator_product_expansions}

Given a set of local operators $\mathcal{O}_i(w)$, the operator product expansion (OPE) characterises the behaviour of the theory at short distances.  In particular, we are interested in how the operators behave when we bring them close together:
\begin{equation}
  \lim_{w \to z} \mathcal{O}_i(w) \mathcal{O}_j(z).
\end{equation}
The operator product expansion is a description of this composite operator in terms of the operators in the theory. In other words, we can think of it as
\begin{equation}
  \lim_{z \to w} \mathcal{O}_i(w) \mathcal{O}_j(z) = \sum_{k} f\indices{_{ij}^{k}} (z-w) \mathcal{O}_k(z).
\end{equation}
We want to extract the singular terms of this limit.

For example, we could look at the following radial ordering operator:
\begin{equation}
  \mathcal{R} \left( \partial X^{\mu}(z) \partial X^{\nu} (w) \right) = \normalorder{\partial X^{\mu}(z) \partial X^{\nu}(w)} - \frac{\lambda'}{2} \frac{\eta^{\mu\nu}}{(z - w)^2}.
\end{equation}
As $z \to w$, the first term is regular, while the second term is singular (and therefore contains the bits we are interested in!)
We may write this as
\begin{equation}
  \label{eq:14-parxope}
  \mathcal{R} \left( \partial X^{\mu}(z) \partial X^{\nu} (w) \right)  = -\frac{\alpha'}{2} \frac{\eta^{\mu\nu}}{(z - w)^2} + \dots,
\end{equation}
where the dots denote the regular terms.
More often than not, we will assume that radial ordering is implicitly applied, unless otherwise stated.

We could compute the correlation function (where radial ordering is understood)
\begin{align}
  \langle X^{\mu}(z) X^{\nu}(w) \rangle &= \langle \normalorder{X^{\mu}(z) X^{\nu}(w)} \rangle - \frac{\alpha'}{2} \eta^{\mu\nu} \ln(z - w), \\
					&= -\frac{\alpha'}{2} \eta^{\mu\nu} \ln(z - w),
\end{align}
where we used that, by design, the vacuum expectation value of any normal ordered quantity will vanish.
This is the Green's function for $\frac{1}{2\pi \alpha'} \partial \overline{\partial}{}$.

Some useful operator product expansions are
\begin{equation}
  X^{\mu}(z) X^{\nu}(w) = -\frac{\alpha'}{2} \eta^{\mu\nu} \ln (z-w), \qquad
  \overline X{}^{\mu}(\overline{z}{}) \overline{X}{}^{\nu}(\overline{w}{}) = -\frac{\alpha'}{2} \eta^{\mu\nu} \ln (\overline{z}{}-\overline{w}{}).
\end{equation}
The operator product expansion of $X^{\mu}(z) \overline{X}{}^{\nu}(\overline{w}{})$ is regular.

We can use the knowledge of the $X^{\mu}$ operator product expansion to define composite operators such as the stress tensor, which, classically, is $ T(z) = -\frac{1}{\alpha'} \partial X^{\mu}(z) \partial X_{\mu}(z) $.
Using the operator product expansion \eqref{eq:14-parxope} we define the composite operator as
\begin{equation}
  T(z) = -\frac{1}{\alpha'} \lim_{m \to z} \left( \partial X^{\mu}(w) \partial X_{\mu}(z) + \frac{\alpha'}{2} \frac{D}{(z - w)^2} \right),
\end{equation}
where the dimension $D = \eta_{\mu\nu} \eta^{\mu\nu}$.


\section{\texorpdfstring{$T(z) X^{\mu}(w)$}{Stress tensor position} Operator Product Expansion and Conformal Transformations}%
\label{sec:t_x_ope_cft}

Consider $ T(z) X^{\mu}(w) = -\frac{1}{\alpha'} \normalorder{\partial X^{\nu} \partial X_{\nu} (z)} X^{\mu}(w)$.
We use Wick's theorem and the operator product expansion
\begin{equation}
  \label{eq:14-pxxope}
  \partial X^{\mu}(z) X^{\nu}(w) = -\frac{\alpha'}{2} \frac{\eta^{\mu\nu}}{z - w} + \dots
\end{equation}
to get
\begin{align}
  T(z) X^{\mu}(w) &= -\frac{2}{\alpha'} \wick{\normalorder{ \partial X^{\nu} \partial \c X_{\nu}(z)} \c X^{\mu}(w)} \\
		  &= -\frac{2}{\alpha'} \partial X_{\nu}(z) \left( -\frac{\alpha'}{2} \frac{\eta^{\mu\nu}}{z - w}  \right) + \dots \\
		  &= \frac{\partial X^{\mu}(z)}{z - w} + \dots
\end{align}
We could expand $\partial X^{\mu}(z)$ about $z = w$ as $\partial X^{\mu}(z) = \partial X^{\mu}(w) + (z-w) \partial^2 X^{\mu}(w) + \dots$.
Then we have
\begin{equation}
  T(z) X^{\mu}(w) = \frac{\partial X^{\mu}(w)}{z - w} + \dots.
\end{equation}
We can compute the conformal transformation of $X^{\mu}(w)$:
\begin{equation}
  \delta_v X^{\mu}(w) = \oint_{z = w} \frac{ \dd[]{z}}{2 \pi i} v(z) T(z) X^{\mu}(w) = \oint_{z = w} \frac{\dd[]{z}}{2 \pi i} v(z) \frac{\partial X^{\mu}(w)}{z - w} = v(w) \partial X^{\mu}(w).
\end{equation}
More generally, we expect a weight $(h, 0)$ primary field $\phi(z)$ to transform as
\begin{equation}
  \delta_v \phi(z) = v(z) \partial \phi(z) + h \partial v (z) \phi(z),
\end{equation}
at which point it is very tempting to call $X^{\mu}(z)$ a chiral primary of weight $h = 0$. However, we will see later that it is more subtle than that; there is something deeply non-local about the $X$'s.
Here we used the residue theorem:
\begin{equation}
  \frac{1}{(n - 1)!} \partial_z^{n-1} f(z) = \oint_{w = z} \frac{\dd[]{w}}{2 \pi i} \frac{f(w)}{(w - z)^n}.
\end{equation}

\begin{claim}
  This requirement fixes the $T(w) \phi(z)$ operator product expansion to have the form
  \begin{equation}
    \label{eq:14-ope}
    \boxed{T(w) \phi(z) = \frac{h}{(z - w)^2} \phi(z) + \frac{1}{z - w} \partial \phi(z) + \dots}
  \end{equation}
\end{claim}
\begin{proof}
  Exercise.
\end{proof}

From now on, we shall take this OPE \eqref{eq:14-ope} as the definition for a field $\phi(z)$ to be a chiral primary of weight $(h, 0)$.

We will be interested in OPEs for composite operators like $ T(z) e^{i k \cdot X(w))} $ or what you might consider a consistency conditions $T(z) T(w) = ?$.
Given $T$ lies at the heart of what it means for a theory to be conformally invariant, we want it to transform in a certain way. This will have interesting repercussions for the spacetime in our theory.
