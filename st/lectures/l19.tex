% lecture notes by Umut Özer
% course: st
\lhead{Lecture 19: March 02}

\subsection*{Meaning of $\mathcal{Q}_B^2 = 0$}%

Classically, the charge $\mathcal{Q}_B$ should commute with the Hamiltonian
\begin{equation}
  [\mathcal{Q}_B, H] = 0.
\end{equation}
The choice of gauge is encoded in the gauge-fixing fermion $\Psi$ of \eqref{eq:gf-ferm}, which we write
\begin{equation}
  \Psi = \int_{\Sigma} \dd[2]{\sigma} F_{\text{gf}},
\end{equation}
where $F_{\text{gf}} = \sqrt{-h} b^{ab} (h_{ab} - \hat{h}_{ab})$.
A change of gauge gives a change in the Hamiltonian
\begin{equation}
  \delta H = \{\mathcal{Q}_B, \delta F_{\text{gf}}\}.
\end{equation}
We expect this new Hamiltonian to preserve $\mathcal{Q}_B$, which requires
\begin{align}
  0 &= [\mathcal{Q}_B, \delta H] \\
    &= [\mathcal{Q}_B, \{\mathcal{Q}_B, \delta F_{\text{gf}}\}] \\
    &= - [\delta F_{\text{gf}}, \{\mathcal{Q}_B, \mathcal{Q}_B\}] - [\mathcal{Q}_B, \overbrace{\{\mathcal{Q}_B, \delta F_{\text{gf}}\}}^{\mathclap{\delta H}}] \\
    &= - [\delta F_{\text{gf}}, \{\mathcal{Q}_B, \mathcal{Q}_B\}],
\end{align}
where we used the Jacobi identity.
For this to be true for any $\delta F_{\text{gf}}$, we require that 
\begin{equation}
  \{\mathcal{Q}_B, \mathcal{Q}_B\} = 0 \quad \implies \quad \boxed{\mathcal{Q}_B^2 = 0}.
\end{equation}
Therefore, $\mathcal{Q}_B^2 = 0$ is resulting from saying that $\mathcal{Q}_B$ generates a symmetry.

\subsection*{States $\ket{\phi} \not\in \text{im}(\mathcal{Q}_B)$}%

Suppose we had a state $\ket{\chi} = \mathcal{Q}_B \ket{\Lambda}$ for some $\ket{\Lambda}$.
Clearly, since $\mathcal{Q}_B^2 = 0$, $\ket{\chi}$ is closed $\mathcal{Q}_B \ket{\chi} = 0$.
But also, if $\ket{\phi} \in \text{ker}(\mathcal{Q}_B)$, then
\begin{equation}
  \bra{\phi} \ket{\chi} = \bra{\phi} \mathcal{Q}_B \ket{\Lambda} = 0.
\end{equation}
Also
\begin{equation}
  \bra{\chi} \ket{\chi} = \bra{\Lambda} \mathcal{Q}_B^2 \ket{\Lambda} = 0.
\end{equation}
The state $\ket{\chi}$ is orthogonal to all physical states (and itself).
These are the same properties as for spurious states, which we met in Sec.~\ref{sec:spurious_states_and_gauge_invariance}.
Such states will decouple from any process that involves physical states (or even itself).
Hence, physical states are in $\text{ker}(\mathcal{Q}_B)$ but not in $\text{im}(\mathcal{Q}_B)$. 
In other words, all physical states may be defined up to the addition of an $\mathcal{Q}_B$-exact term
\begin{equation}
  \ket{\phi} \sim \ket{\phi} + \mathcal{Q}_B \ket{\Lambda}.
\end{equation}
The physical states live in the cohomology of $\mathcal{Q}_B$.
