% lecture notes by Umut Özer
% course: st
\lhead{Lecture 8: February 05}

Consider now the final piece $\delta_t h_{ab}$ of the transformation \eqref{eq:7-mt} of the metric.
We can write these as
\begin{equation}
  t^I \mu_{abI}, \qquad \mu_{abI} = \frac{\partial h_{ab}}{\partial m^I},
\end{equation}
where $m^I$ are coordinates of the moduli space of Riemann surfaces $M_g$ and $t^I \approx \delta m^I$ are tangent vectors to $M_g$.

\subsection{Moduli Space of (Unpunctured) Riemann Surfaces}%
\label{sub:moduli_space_of_unpunctured_riemann_surfaces}

The moduli space $M_g$ of a genus $g$ Riemann surface $\Sigma_g$ is schematically
\begin{equation}
  M_g = \frac{\{\text{metrics}\}}{\{\text{Diff} \times \text{Weyl}\}}.
\end{equation}
The (real) dimension of $M_g$ is as follows
\begin{equation}
  \dim(M_g) \coloneqq s =
  \begin{cases}
    0, & \text{if } g = 0 \\
    2, & \text{if } g = 1 \\
    6g-6, & \text{if } g \geq 2.
  \end{cases}
\end{equation}
\begin{example}[$g = 0$]
  All metrics on $g = 0$  surfaces\footnote{Everything we are saying pertains to two dimensional manifolds.} may be brought to the form $e^{2 \omega} 
  \begin{pmatrix}
   1 & 0 \\
   0 & 1 \\
  \end{pmatrix} $ by diffeomorphisms.
  In other words, they are trivial up to Weyl transformations.
\end{example}
\begin{example}[$g = 1$ (torus)]
  We can construct a torus by taking the complex plane $\mathbb{C}$  and imposing identifications such as
  \begin{equation}
    z \sim z + \lambda_1 m + \lambda_2 n,
  \end{equation}
  where $\lambda_1, \lambda_2$  are lattice vectors and $m, n \in \mathbb{Z}$ .
  One can show that the ratio
  \begin{equation}
    \tau = \frac{\lambda_1}{\lambda_2} \in \mathbb{C}
  \end{equation}
  is invariant under  $\text{Diff} \times \text{Weyl}$ .
  \begin{remark}
    This $\tau$ is not worldsheet time! It is called the \emph{complex structure}.
  \end{remark}
  In other words, using diffeomorphisms and Weyl transformations, we can always bring the metric to the form 
  \begin{equation}
    ds_2 = \abs{dz + \tau d \overline{z}{}}^2.
  \end{equation}
  There are two degrees of freedom, which are encoded in this one complex degree of freedom.
  Without loss of generality, we can choose $\Im \tau > 0$ . We could write the identification $z \sim z + n_a \lambda^a$ , where $n_a = (m, n)$  and $\lambda^a = 
  \begin{pmatrix}
  \lambda_1 \\
  l_2 \\
  \end{pmatrix} $ .
  If we act with an $SL(2)$  transformation
  \begin{align}
    n_a &\to U\indices{_{a}^{b}} n_{b} \\
    \lambda^{a} &\to (U^{-1})\indices{^{a}_{b}} \lambda^{b}
  \end{align}
  this statement is preserved.
  Moreover, if $U\in SL(2; \mathbb{Z})$ , which has integer entries, then $(n, m)$  remain integer.
  So the moduli space $M_1$ at $g=1$  can be identified with the upper half plane modulo $SL(2; \mathbb{Z})$.
  \begin{figure}[tbhp]
    \centering
    \def\svgwidth{0.5\columnwidth}
    \input{lectures/l8f1.pdf_tex}
    \caption{}
    \label{fig:l8f1}
  \end{figure}
\end{example}
\begin{example}[$g \geq 2$]
  It gets hard but there are analogues of the $SL(2; \mathbb{Z})$ (modular group) in all cases.
\end{example}

\subsection{Conformal Killing Vectors}%
\label{sub:conformal_killing_vectors}

\begin{definition}[Conformal Killing group]
  There is an overlap $\text{Diff} \cap \text{Weyl}$ called the \emph{conformal Killing group} (CKG).
\end{definition}

Is it possible to undo the effect of a diffeomorphism by a Weyl transformation?
In other words, we want to find a pair ($v^a, \omega$) such that
\begin{equation}
  \delta h_{ab} = \nabla_a v_b - \nabla_b v_a + 2 \omega h_{ab} = 0
\end{equation}
We take the trace to find that if
\begin{equation}
  \omega = -\frac{1}{2} \nabla_a v^a,
\end{equation}
then the diffeomorphism and Weyl transformation cancel out.
We defined $Pv$ in Eq.~\eqref{eq:Pv}.
\begin{definition}[Conformal Killing vectors]
  Vectors that satisfy $(Pv)_{ab} = 0$ are what we shall call \emph{conformal Killing vectors} (CKVs).
\end{definition}
The CKG is generated by the CKVs.
For a given genus, the (real) dimension of the conformal Killing group is
\begin{equation}
  \dim_{\mathbb{R}}(\text{CKG}) \coloneqq \kappa =
  \begin{cases}
    6, & \text{if } g = 0 \\
    2, & \text{if } g = 1 \\ 
    0, & \text{if } g \geq 2 \\ 
  \end{cases}
\end{equation}
In particular, notice that these are finite-dimensional.
We can figure out that for $g = 0$, the CKG is $SL(2, \mathbb{C})$  and on the torus $g = 1$  it is $U(1) \times U(1)$ . Moreover, for $g \geq 2$ , the CKG is empty; there is no overlap for higher $g$'s.

\begin{example}[$g = 2$]
  The conformal Killing group is $SL(2; \mathbb{C})$\footnote{Strictly speaking this should be the projective group $PSL(2; \mathbb{C})$. However, this distinction will not matter much for our purposes.}. This is called the \emph{Möbius group}. If $z \in \mathbb{C}$, then this transformation acts as
  \begin{equation}
    z \to \frac{az + b}{cz + d}, \qquad ad - bc = 1, \qquad a,b, c, d \in \mathbb{Z}.
  \end{equation}
  We can specify a particular element of the conformal Killing group at genus $g = 0$ by describing how three distinct points transform under the given map.
\end{example}

\subsection{Moduli Space of Punctured Riemann Surfaces \texorpdfstring{$M_{g, n}$}{}}%
\label{sub:moduli_space_of_punctured_riemann_surfaces}

Suppose we have a Riemann surface of genus $g$ with $n$ punctures (or marked points). What is its moduli space?
Naively, we might expect
\begin{equation}
  M_{g, n} = M_g \otimes \Sigma^n.
\end{equation}
But we also have to worry about the CKG.
If we have a genus $g =0$ surface  with $n$ punctures, we can use the conformal Killing group to fix three of the locations of the punctures (i.e.~`three degrees of freedom'), leaving only $n-3$ free punctures.
Similarly, we can fix the CKG on the torus ($g = 1$) by fixing the location of one puncture. 
\begin{leftbar}
  We will see very explicitly how this works when we look at the path integral.
\end{leftbar}
So we should really take the moduli space (as relating to our path integral) to be
\begin{equation}
  M_{g, n} = M_g \otimes \Sigma^{n- \frac{\kappa}{2}},
\end{equation}
since each puncture has two bits ($\sigma, \tau$) of information.
Therefore, the dimension of the moduli space is
\begin{equation}
  \boxed{\dim(M_{g, n}) = 6 g - 6 + 2n}
\end{equation}

