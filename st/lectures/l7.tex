% lecture notes by Umut Özer
% course: st
\lhead{Lecture 7: February 03}

They are orthogonal to every physical state, so in a correlation function they do not affect anything, even though they are physical.

Consider $\lambda_{\mu} = \widetilde{\lambda}_{\mu}$ and let us add $\ket{\chi}$ to the graviton state $\ket{h} = \epsilon_{\mu\nu} \alpha^{\mu}_{-1} \overline{\alpha}{}^{\nu}_{-1}$, so
\begin{equation}
  \ket{h'} = \ket{h} + \ket{\chi}.
\end{equation}
Since $\ket{\chi}$  decouples entirely, its inner product with all physical states vanishing, this should give identical physics.
Then
\begin{align}
  \ket{h'} &= (\epsilon_{\mu\nu} + \lambda_{\mu} k_{\nu} + \lambda_{\nu} k_{\mu}) \alpha^{\mu}_{-1} \overline{\alpha}{}^{\nu}_{-1} \ket{k} \\
  &= \epsilon'_{\mu\nu} \alpha^{\mu}_{-1}\overline{\alpha}{}^{\nu}_{-1} \ket{k}.
\end{align}
All this does it that it changes the graviton to another graviton with a different polarisation. We recognise the transformation
\begin{equation}
  \epsilon_{\mu\nu} \to \epsilon'_{\mu\nu} = \epsilon_{\mu\nu} + \lambda_{\mu} k_{\nu} + \lambda_{\nu} k_{\mu}
\end{equation}
as a (linearised) diffeomorphism, the symmetry of general relativity.
These states are interpreted as being gauge exact when we look at BRST quantisation later.
This is a hint suggesting that $a = 1$  might be a sensible physical choice.

\chapter{Path Integral Quantisation}%
\label{cha:path_integral_quantisation}

This will be our favourite way to quantise the string, since the path integral formalism gives an easy way to computing correlation functions in terms of integrals over spaces of fields with given boundary conditions.

\begin{example}[$d = 1$]
  In non-relativistic quantum mechanics, we may be interested in calculating the transition amplitude for a particle to go from $x_i (t_i)$  to $x_f(t_f)$ . The corresponding amplitude is given by a weighted sum of all possible ways of going from the start point to the end point
  \begin{equation}
    \langle x_i, t_i : x_f, t_f \rangle \propto \int_i^f \pdd{X} e^{i S[X] / \hbar}.
  \end{equation}
  For most of the calculations we want to do we can treat this just as a generating function for correlation functions and we do not need to worry too much about the path integral measure $\pdd{X}$.
  \begin{figure}[tbhp]
    \centering
    \def\svgwidth{0.4\columnwidth}
    \input{lectures/l7f1.pdf_tex}
    \caption{Path Integral}
    \label{fig:l7f1}
  \end{figure}
\end{example}

In string theory we take a very similar starting point.
We take a weighted sum of all worldsheets with initial and final conditions specified. This is illustrated as shown in Fig. %F2

We want to make sense of ($\hbar = 1$)
\begin{equation}
  \bra{\Psi_i} \ket{\Psi_f} = \frac{1}{\abs{\text{Diff} \times \text{Weyl}}} \int \pdd{X} \pdd{h} e^{i S[X, h]},
\end{equation}
where $S[X, h]$ is the Polyakov action \eqref{eq:polyakov} in flat spacetime.
We want to separate the integral over $h_{ab}$ into those $h_{ab}$ related to each other by the action of diffeomorphisms and Weyl transformations, $\text{Diff} \times \text{Weyl}$, and those that are `physically distinct'.
The sort of thing we want to do is to split the measure $\pdd{h}$ into 
\begin{equation}
  \pdd{h} = \pdd{\nu} \times \pdd{h_{\text{phys}}} \times \; \mathcal{J},
\end{equation}
where $\pdd{\nu}$ is the volume element of the infinite dimensional group $\text{Diff} \times \text{Weyl}$, and $\mathcal{J}$ is some Jacobian coming from the separation, which we can see as a change of variable.
Recall the general transformation of $h_{ab}$:
\begin{equation}
  \label{eq:7-mt}
  \delta h_{ab} = \delta_v h_{ab} + \delta_w h_{ab} + \delta_t h_{ab},
\end{equation}
where 
\begin{itemize}
  \item $ \delta_v h_{ab} = \nabla_a v_b + \nabla_b v_, $ are diffeomorphisms (with parameter $v$), which take the form of a Lie derivative,
  \item $\delta_w h_{ab} = 2 \omega h_{ab}$ are Weyl rescalings of $h_{ab}$,
  \item $\delta_t h_{ab}$ are physically distinct changes in $h_{ab}$.
\end{itemize}
%F3
\begin{figure}[tbhp]
  \centering
  \def\svgwidth{0.4\columnwidth}
  \input{lectures/l7f3.pdf_tex}
  \caption{The `space of $h_{ab}$'. Although we have not proved it, the gauge slice is chosen to cut through all orbits of $\text{Diff} \times \text{Weyl}$ exactly once.}
  \label{fig:l7f3}
\end{figure}

We notice that $\delta_n h_{ab}$ has non-zero trace. It is useful to absorb this trace part into $\delta_w h_{ab}$.
We introduce a traceless version of $\delta_v h_{ab}$:
\begin{equation}
  \label{eq:Pv}
  \boxed{(\mathcal{P} v)_{ab} = \nabla_a v_b + \nabla_b v_a - h_{ab} (\nabla_c v^c)}.
\end{equation}
And the Weyl transformation becomes 
\begin{equation}
  \delta_{\overline{\omega}{}} = 2 \overline{\omega}{} h_{ab} \qquad \overline{\omega}{} = \omega + \frac{1}{2} \nabla \cdot v.
\end{equation}
\begin{definition}[]
  We call the space of physically distinct metrics $h_{ab}$ the \emph{moduli space}.
\end{definition}
And $\delta_t h_{ab}$ is a variation of the metric in the moduli space.

\begin{leftbar}
  So far, we are only talking about diffeomorphisms connected to the identitiv since we are working infinitesimally. We can extend this to `large diffeomorphisms'.
\end{leftbar}
We will work on worldsheets in which (at least infinitesimally) all metrics can be related to each other by diffeomorphisms and Weyl transformations.

\section{A Crash Course on Riemann Surfaces}%
\label{sec:a_crash_course_on_riemann_surfaces}

Potentially, we will work on all sorts of worldsheets with all sorts of topologies.
\begin{definition}[Riemann surface]
  A \emph{Riemann surface} is a 2 (real) dimensional Riemannian manifold, where we consider metrics related by a Weyl transformation $ h_{ab} \to e^{2 \omega} h_{ab} $ to be equivalent.
\end{definition}
Schematically, we might say that the space of Riemann surfaces is the space of Riemannian manifolds modulo Weyl transformations.

\subsection{Worldsheet Genus and Punctures}%
\label{sub:worldsheet_genus_and_punctures}

These things will distinguish different Riemann surfaces.
\begin{definition}[genus]
  For $\Sigma$  without boundary (closed strings)\footnote{For a cylinder, we might need to worry about the infinite past and future? We will discuss this later, but it turns out that it is not quite like a boundary. We will discuss this in terms of punctures.} the topology of $\Sigma$  is encapsulated in the Euler characteristic
  \begin{equation}
    \chi = \frac{1}{4\pi} \int_\Sigma \dd[2]{\sigma} \sqrt{-h} R(h) = 2 - 2g,
  \end{equation}
  where $g = 0, 1, \dots$ is the \emph{genus} of $\Sigma$.
\end{definition}
% F4 sphere, donut, two holes, \dots with g =0 g=1, g=2 below it
Considering Fig.%F4
, this starts to look a bit like a loop expansion, where $g$  will be related to the loop order in perturbation theory.
\begin{definition}[punctures]
  \emph{Punctures} on $\Sigma$ are marked points on $\Sigma$, i.e.~points at a particular position.
\end{definition}
%F5 donut with three crosses on it
\begin{example}[]
  The Riemann surface illustrated in Fig %F5
  has $g = 1$ and $3$ marked points.
\end{example}
