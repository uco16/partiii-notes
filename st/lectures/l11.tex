% lecture notes by Umut Özer
% course: st
\lhead{Lecture 11: February 12}

\subsection{The Witt Algebra}%
\label{sub:the_witt_algebra}

We have seen that conformal transformations are generated by (anti-)holomorphic vectors $v(z)$ and $\overline{v}{}(z)$, i.e.
\begin{equation}
  z \to z + v(z) + \dots \qquad \overline{z}{} \to \overline{z}{} + \overline{v}{}(\overline{z}{}) + \dots
\end{equation}

We can write $v(z)$ as a power series expansion
\begin{equation}
  v(z) = \sum_{n} v_n z^{n+1}.
\end{equation}
We might allow for the possibility to have some isolated poles in this sum, but we will not worry too much about that for now.
For now the exponent $z^{n+1}$ is just a convention.
Infinitesimally, we have the transformation
\begin{equation}
  z \to z + \sum_{n} v_n z^{n+1},
\end{equation}
which we can think of as being generated by the operators
\begin{equation}
  l_n = -z^{n+1} \frac{\partial }{\partial z} \qquad \text{i.e.} \quad z \to z - \sum_{n} v_n l_n z.
\end{equation}
The operators $l_n$  satisfy the Witt algebra
\begin{equation}
  \boxed{[l_n, l_m] = (n-m)l_{n+m}}
\end{equation}
The first time we saw this, the $l_n$  were the Fourier modes of the stress tensor. Here, another way to look at these is as the generators of conformal transformations.
This hints at a connection between the stress tensor and conformal transformations.

\subsection{Conformal Fields}%
\label{sub:conformal_fields}

Some definitions:
\begin{definition}[chiral field]
  A \emph{chiral field} is a field $\phi(z)$ that depends on $z$ alone.
\end{definition}
\begin{definition}[conformal dimension]
  The \emph{conformal dimension} is a number $\Delta = h + \overline{h}{}$ that tells us how a field transforms under scaling.

  Suppose we have a field $\phi(z, \overline{z}{})$ and rescale $(z, \overline{z}{}) \to (\lambda z, \overline{\lambda}{} \overline{z}{}) = (z', \overline{z}{}')$,
  where $\lambda$ and $\overline{\lambda}{}$ need not necessarily be complex conjugates of each other.
  Then
  \begin{equation}
    \phi(z, \overline{z}{}) \to \phi'(z', \overline{z}{}') = \lambda^{h} \overline{\lambda}{}^{\overline{h}{}} \phi(\lambda z, \overline{\lambda}{} \overline{z}{}).
  \end{equation}
\end{definition}
\begin{remark}
  A chiral field has $\overline{h}{} = 0$.
\end{remark}

\begin{definition}[primary field]
  Under the transformation $z \to f(z)$ , a \emph{primary field} transforms as
  \begin{equation}
    \phi(z, \overline{z}{}) \to \left( \frac{\partial f}{\partial z} \right)^h \left( \frac{\partial \overline{f}{}}{\partial \overline{z}{}} \right)^{\overline{h}{}} \phi \bigl(f(z), \overline{f}{}(\overline{z}{})\bigr),
  \end{equation}
  where $f$ and $\overline{f}{}$ need not be complex conjugates of each other.
  What we are really saying is that primary fields are tensors under conformal transformations (not just under diffeomorphisms).
\end{definition}

Similarly to how we focused on just the left-moving fields before, we will look only at chiral primary fields. (The correspondence between left (right) moving modes and (anti) holomorphic fields is given by the definition of our coordinates \eqref{eq:z}.)

For an infinitesimal transformation $f(z) = z + v(z) + \dots$
\begin{gather}
  \left( \frac{\partial f}{\partial z} \right)^h \approx (1 + \partial v)^h \approx 1 + h \partial v \\
  \phi(f(z)) = \phi(z + v + \dots) = \phi(z) + v \partial \phi(z) + \dots
\end{gather}
and similarly for $\overline{z}{} \to \overline{z}{} + \overline{v}{}(\overline{z}{}) + \dots$ .
We find
\begin{equation}
  \boxed{ \delta_{v, \overline{v}{}} \phi(z, \overline{z}{}) = \left( h \partial v + \overline{h \partial v}{} + v \partial + \overline{v \partial}{} \right) \phi(z, \overline{z}{}) }
\end{equation}

\subsection{Conformal Transformations and the Stress Tensor}%
\label{sub:conformal_transformations_and_the_stress_tensor}

We start with out favourite Polyakov action in flat space,
\begin{equation}
  S[X] = -\frac{1}{4 \pi \alpha'} \int_\Sigma \dd[2]{\sigma} \partial_{a} X^{\mu} \partial^{a} X^{\nu} \eta_{\mu\nu}.
\end{equation}
Under a conformal transformation, the embedding field transforms as
\begin{equation}
  \delta_v X^{\mu} = v^{a} \partial_{a} X^{\mu}.
\end{equation}
We would like to find the Noether current associated with this.
Under such a transformation, the action changes by
\begin{equation}
  \delta_v S[X] = \frac{1}{2\pi} \int_\Sigma \dd[2]{\sigma} (\partial^{a} v^{b}) T_{ab}.
\end{equation}
We can integrate by parts to obtain
\begin{equation}
  \delta_v S[X] = -\frac{1}{2\pi} \int_\Sigma \dd[2]{\sigma} v^{a} (\partial^{b} T_{ab}) = 0.
\end{equation}
For this to be a symmetry under all $v^{a}$, there has to be a conserved current
\begin{equation}
  \label{eq:11-cons}
  \boxed{\partial^{a} T_{ab} = 0 = \partial^{b} T_{ab}}
\end{equation}
We could define conserved charges $Q_{\pm}$, which in lightcone coordinates ($\sigma^{\pm} = \tau \pm \sigma$) look like
\begin{equation}
  Q_{\pm} = \frac{1}{2\pi} \oint \dd[]{\sigma} v^{\pm}(\sigma) T_{\pm \pm} (\sigma),
\end{equation}
where the normalisation is a convention.
Infinitesimal conformal transformations are given by the Poisson bracket of of some field with $Q_{\pm}$ .
In particular, 
\begin{equation}
  \label{eq:11-pb}
  \delta_{v} X^{\mu} = \{ Q_{+} + Q_-, X^{\mu} \}_{\text{PB}},
\end{equation}
where $\{\ , \ \}_{\text{PB}}$  is the classical Poisson bracket.

This classical statement gives us some hint that the stress tensor, thoguht of as a constraint, generates these conformal transformations and explains why the Witt algebra is appearing in these two apparently different circumstances.
Put another way, the constraints coming from the vanishing of the stress tensor put physical conditions on the spacetime. We would like to recast this intuition into understanding how conformal symmetry gives rise to various properties of spacetime physics.
We will soon look at a quantum version of \eqref{eq:11-pb}.

\section{Complex Coordinates}%
\label{sec:complex_coordinates}

In this section, we use $z = e^{\tau + i \sigma}$, $\overline{z}{} = e^{\tau -i \sigma}$ and take $h_{ab}$ to be Euclidean on $\Sigma$.
In these coordinates, the stress tensor has only two non-trivial coordinates
\begin{equation}
  T \coloneqq T_{zz} =-\frac{1}{\alpha'} \partial X^{\mu} \partial X^{\nu} \eta_{\mu\nu}, \qquad 
  \overline{T}{} \coloneqq T_{\overline{zz}} =-\frac{1}{\alpha'} \overline{\partial}{} X^{\mu} \overline{\partial}{} X^{\nu} \eta_{\mu\nu}.
\end{equation}
The off-diagonal entries $T_{z\overline{z}{}} = 0$ vanish as they are the trace of the stress tensor.
Moreover, 
\begin{equation}
  \dd[]{t} \dd[]{\sigma} = -\frac{\dd[]{z} \dd[]{\overline{z}{}}}{2i \abs{z}^2}.
\end{equation}
The path integral becomes
\begin{equation}
  \int \pdd{X} e^{i S[X]} \to \int \pdd{X} e^{-S[X]},
\end{equation}
where the action is
\begin{equation}
  S[X] = -\frac{1}{2\pi \alpha'} \int_\Sigma \dd[2]{z} \partial X^{\mu} \overline{\partial}{} X^{\nu} \eta_{\mu\nu}.
\end{equation}

The equation of motion for the $X^{\mu}$  is now $\partial \overline{\partial}{} X^{\mu} = 0$. This means that we have the split
\begin{equation}
  X^{\mu}(z, \overline{z}{}) = X^{\mu}_L (z) + X^{\mu}_R(\overline{z}{}).
\end{equation}
Finally, our conservation equation \eqref{eq:11-cons} becomes
\begin{equation}
  \overline{\partial}{}T(z) = 0 \qquad \partial \overline{T}{}(\overline{z}{}) = 0.
\end{equation}

\subsection{Ward Identities and Conformal Transformations}%
\label{sub:ward_identities_and_conformal_transformations}

Given an infinitesimal conformal transformation, how do we compute it in terms of the operators in our theory.
Ultimately, the fields that we are interested in are the embedding fields $X^{\mu}$, but also the ghosts $b$ and $c$.
We will describe a general primary field as $\phi(z, \overline{z}{})$. Later we shall choose $\phi$ to be either $X^{\mu}$ or one of the $b, c$ ghost fields, which came from the quantum field theoretic description of the Fadeev--Popov determinant.

Let us consider the change in a correlation function $\langle \phi_1 \dots \phi_n \rangle = \int \pdd{\phi} e^{-S[\phi]} \phi_1 \dots \phi_n$ resulting from an infinitesimal transformation $\phi \to \phi' = \phi + \delta \phi$.
\begin{equation}
  \langle \phi'(z_1) \dots \phi'(z_n) \rangle = \int \pdd{\phi'} e^{-S[\phi']} \phi'(z_1) \dots \phi'(z_n),
\end{equation}
where we write $\phi'(z_i)$ as a shorthand for $\phi'(z_i, \overline{z}{}_i)$.
We shall assume that the measure $\pdd{\phi'} = \pdd{\phi}$ does not change (meaning that there are no anomalies in the theory).
However, in general the action might change. To leading order, we have
\begin{align}
    \langle \phi'(z_1) \dots\phi'(z_n) \rangle &= \int \pdd{\phi} e^{-S[\phi] - \delta S[\phi]} (\phi(z_1) + \delta \phi(z_1)) \dots (\phi(z_n) + \delta \phi(z_n)) \\
					       &= \langle \phi(z_1) \dots \phi(z_n) \rangle - \int \pdd{\phi} e^{-S[\phi]} \delta S[\phi] \phi(z_1) \dots \phi(z_n) \\
					       & \qquad + \int \pdd{\phi} e^{-S[\phi]} \sum_{k=1}^{n} \phi(z_1) \dots \phi(z_{k-1}) \delta \phi(z_k) \phi_{z_{k+1}} \dots \phi(z_n).
\end{align}
For our theory to be invariant under the change $\phi \to \phi'$ we require  $\langle \phi'1 \dots \phi'_n \rangle = \langle \phi_1 \dots \phi_n \rangle$ . Thus
\begin{equation}
  \langle \delta S[\phi]\, \phi(z_1) \dots \phi(z_n) \rangle = \sum_{k=1}^{n} \langle \phi(z_1) \dots \delta \phi(z_k) \dots \phi(z_n) \rangle.
\end{equation}
