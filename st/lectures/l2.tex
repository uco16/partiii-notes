% lecture notes by Umut Özer
% course: st
\lhead{Lecture 2: January 22}

\chapter{The Classical Particle and String}%
\label{cha:the_classical_particle_and_string}

In this section we will start systematically constructing a theory of quantum strings.
As a precursor to this discussion, we will need to talk a bit about non-relativistic quantum mechanics.

In non-relativistic QM, we treat time ($t$) as a parameter and position ($\hat{x}$) as an operator.
Obviously, this kind of distinction should not survive in a relativistic theory. This means there are some choices to be made. 

\begin{description}
  \item[Second quantisation:] Both $x^{i}$ and $t$ are considered parameters. They are not the fundamental objects we quantise. Instead, we quantise quantum fields $\phi(\vb{x}, t)$, which are the basic objects of our theory.
    We require that the fields transform in appropriate ways under Lorentz transformations.
    This is historically overwhelmingly the most useful way to do it, seeing its success in quantum field theory and the standard model.
  \item[First quantisation:] Here we make the other choice: we elevate $t$ to being an operator.
    This is the natural framework for describing the relativistic embedding of worldlines (-sheets, -volumes) into spacetime.
    Here, $X^{\mu} = (x^{i}, t)$ is an operator, the fundamental object we quantise, and there will be some other natural parameter entering the theory.
\end{description}

There is such a thing as string field theory, which employs the viewpoint of second quantisation. However, apart from a few exceptions, there is not much that you do not also obtain from first quantisation string theory.
However, first quantisation string theory has made significant advances on problems that have not yet be solved with second quantisation.

\subsection{Worldlines and Particles}%
\label{sub:worldlines_and_particles}

We consider the embedding of a worldline $\mathcal{L}$  into spacetime $M$.
The basic field is the embedding  $X^{\mu} \colon \mathcal{L} \to M$. An action that describes this embedding might be
\begin{equation}
  \label{eq:2-a}
  S[X] = -m \int_{x_2}^{x_1} \dd[]{s} = -m \int_{\tau_1}^{\tau_2} \dd[]{\tau} \sqrt{-\eta^{\mu\nu} \dot{X}^{\mu} \dot{X}^{\nu}},
\end{equation}
where $\tau$ (a parameter) is the proper time and $X^{\mu}(\tau_2) = x_2^{\mu}$, $X^{\mu}(\tau_1) = x_1^{\mu}$ are end points of the world line.
\begin{figure}[tbhp]
  \centering
  \def\svgwidth{0.4\columnwidth}
  \input{lectures/l2f1.pdf_tex}
  \caption{}
  \label{fig:l2f1}
\end{figure}

The momentum conjugate to $X^{\mu}(\tau)$  is $P_{\mu}(\tau) = - m \frac{\dot{X_{\mu}}}{\sqrt{- \dot{X}^2}}$. This satisfies $\dot{X}^2 = \eta_{\mu\nu} \dot{X}^{\mu} \dot{X}^{\nu}$. 
This satisfies $P^2 + m^2 = 0$ identically, an \emph{on-shell condition}.

\subsection*{Symmetries}%

\begin{description}
  \item[Rigid symmetry:] $X^{\mu}(\tau) \to \Lambda\indices{^{\mu}_{\nu}} X^{\nu}(\tau) - a^{\mu}$, where $\Lambda\indices{^{\mu}_{\nu}}$ is a Lorentz transformation and $a^{\mu}$ is a (constant) displacement.
  \item[Reparameterisation invariance:] $\tau \to \tau + \xi(\tau)$. The embedding $X^{\mu}$ changes as 
    \begin{equation}
      X^{\mu}(\tau) \to X^{\mu}(\tau + \xi) = X^{\mu}(\tau) + \xi \dot{X}^{\mu}(\tau) + \dots.
    \end{equation}
    To first order, $\delta X^{\mu}(\tau) = \xi \dot{X}^{\mu} (\tau)$.
\end{description}

There is a rewriting of this action \eqref{eq:2-a} that makes life a bit easier.
We do this by introducing a new auxiliary field $e(\tau)$, which is a one-form, on the worldline $\mathcal{L}$.
\begin{equation}
  \label{eq:2-b}
  S[X, e] = \frac{1}{2} \int_{\mathcal{L}} \dd[]{\tau} \left( e^{-1} \eta_{\mu\nu} \dot{X}^{\mu} \dot{X}^{\nu} - em^2 \right).
\end{equation}
\begin{leftbar}
  If you like, you can think of $e$ as something like a one-dimensional metric, which sets a scale for distances on the line.
\end{leftbar}
\begin{leftbar}
  We will be able to do something analogous for strings, which will be very useful!
\end{leftbar}

The equations of motion for $X^{\mu}(\tau)$ and $e(\tau)$ are 
\begin{equation}
  \dv{\tau} (e^{-1} \dot{X}^{\mu}) = 0.
\end{equation}
However, the equation of motion for $e$  will not depend on $\dot{e}$, but be purely algebraic:
\begin{equation}
  \label{eq:2-1}
  \dot{X}_2 + e^2 m^2 = 0.
\end{equation}
As such, $e$ can be thought of as a Lagrange multiplier, enforcing a constraint.
What is the constraint that it enforces?

The momentum conjugate to $X^{\mu}$  is 
\begin{equation}
  \label{eq:2-2}
  P_{\mu} = e^{-1} \dot{X}_{\mu}.
\end{equation}
Combining \eqref{eq:2-1} and \eqref{eq:2-2}, we can eliminate $e$ to find precisely the on-shell condition $P^2 + m^2 = 0$.

\begin{leftbar}
  The auxiliary field $e$ basically enforces energy-momentum conservation on the worldline.
\end{leftbar}

\begin{exercise}
  We can write $e^{-1} = m / \abs{\dot{X}}$. Plug this into the action \eqref{eq:2-b} to find that $S[X, e]$, subject to the equation of motion for $e(\tau)$ gives precisely the action \eqref{eq:2-a}.
\end{exercise}
There are two reasons why we should consider this new action instead. Firstly, in the $m \to 0$ limit, it will be easy to show that it describes null worldlines. Secondly, since there is no square root in this theory, it will be easier to quantise.
This is at the cost of having to introduce a new non-dynamical field $e$.

The action $S[X, e]$ has the following symmetries:
\begin{itemize}
  \item Poincar\'e invariance (where $e$ is invariant)
  \item Reparameterisation invariance: 
    \begin{equation}
      \delta X^{\mu} = \xi \dot{X}^{\mu} \qquad \delta e = \dv{\tau} (\xi e)
    \end{equation} 
    provided these variations vanish on the endpoints.
    \begin{leftbar}
      Note that this transformation law under diffeomorphisms shows that $e$ needs to transform like a one-form.
    \end{leftbar}
\end{itemize}

\begin{leftbar}
  Locally, we can think of $e$ as a pure gauge.
\end{leftbar}

\begin{remark}
  We could generalise $\eta_{\mu\nu} \to g_{\mu\nu}\big(X(\tau)\big)$. 
  The model becomes highly non-linear.
  We will look at this later in the context of strings.
\end{remark}

\section{Classical Strings: The Nambu--Goto Action}%
\label{sec:classical_strings}

Our starting point will be the \emph{Nambu-Goto Action}.
We use units where $\hbar = c = 1$ throughout.
\begin{figure}[tbhp]
  \centering
  \def\svgwidth{0.4\columnwidth}
  \input{lectures/l2f2.pdf_tex}
  \caption{}
  \label{fig:l2f2}
\end{figure}
The fundamental degree of freedom is
\begin{equation}
  X \colon \sigma \to M,
\end{equation}
where $M$ is called the \emph{target space}.

\begin{definition}[Nambu--Goto Action]
  The Nambu--Goto action is
  \begin{equation}
    S[X] = -\frac{1}{2\pi \alpha'} \int_{\Sigma} \dd[]{\tau} \dd[]{\sigma} \sqrt{-\det(\eta_{\mu\nu} \partial_{a} X^{\mu} \partial_{b} X^{\nu})},
  \end{equation}
  where $\alpha'$ is a constant with dimensions of (spacetime) area and $\sigma^{a} = (\tau, \sigma)$  and $\partial_{a} = \frac{\partial}{\partial \sigma^{a}}$ .
\end{definition}

\begin{remark}
  One often speaks of the \emph{string length} $l_2 = 2 \pi \sqrt{\alpha'}$.
\end{remark}

\begin{definition}[string tension]
  We introduce the \emph{string tension} $T = \frac{1}{2 \pi \alpha'}$.
\end{definition}

\begin{leftbar}
  This turns out to be a good starting point; quantising this puts us on the right track. However, it is horrendously difficult to actually perform this quantisation. Instead, we will consider another action.
\end{leftbar}

\begin{definition}[Polyakov Action]
  The Polyakov action is
  \begin{equation}
    S[X, h] = -\frac{1}{4 \pi \alpha'} \int_{\Sigma} \dd[2]{\sigma} \sqrt{-h} h^{ab} \eta_{\mu\nu} \partial_{a} X^{\mu} \partial_{b} X^{\nu}.
  \end{equation}
\end{definition}

The new feature as compared to the Nambu--Goto action is the new field $h_{ab}$, which is a metric on $\Sigma$.
It is non-dynamical; there are no Einstein-Hilbert terms and similarly to $e$ previously it can be thought of as enforcing a constraint.
We will find that $h_{ab}$ is extremely important.
\begin{leftbar}
  This form is quite suggestive. All this is, when quantised, is a two-dimensional massless Klein--Gordon field in three dimensions. It is difficult to find an easier theory than this.
\end{leftbar}
