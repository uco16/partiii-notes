% lecture notes by Umut Özer
% course: st
\lhead{Lecture 5: January 29}

\section{A First Look at the Quantum Theory}%
\label{sec:a_first_look_at_the_quantum_theory}

If we choose the metric to be Minkowski up to Weyl transformation, we get a two-dimensional Klein--Gordon equation of motion.
However, there is also the constraint $T_{ab} = 0$, which distinguishes the two-dimensional massless Klein--Gordon field from bosonic string theory.
There are two approaches we can take.
Either we impose the constraint first and then quantise, or we do it the other way around.
\emph{Lightcone quantisation} goes the first route, which has historically been very successful.
However, in this course, we will not do this!

Instead, we will quantise the unconstrained theory and then impose the constraint $T_{ab} = 0$ as a physical condition  on the Hilbert space of states.

\subsection{Canonical Quantisation}%
\label{sub:canonical_quantisation}

We quantise by replacing our Poisson brackets with commutators:
\begin{equation}
  \Bigl\{ \quad , \quad \Bigr\}_{\text{PB}} \to -i [\quad , \quad ].
\end{equation}
Giving equal time commutators:
\begin{equation}
  [X^{\mu}(\sigma), X^{\nu}(\sigma')] = 0, \qquad [P_{\mu}(\sigma), P_{\nu}(\sigma')] = 0, \qquad [P_{\mu}(\sigma), X^{\nu}(\sigma')] = -i \delta\indices{_{\mu}^{\nu}} \delta(\sigma - \sigma').
\end{equation}
Using the mode expansions for $X^{\mu}$  and $P_{\mu}$ , e.g.
\begin{equation}
  X^{\mu}(\sigma, \tau) = x^{\mu} + p^{\mu} \alpha' \tau + i \sqrt{\frac{\alpha'}{2}} \sum_{n \neq 0} \frac{1}{n}\left( \alpha_n^{\mu} e^{-in (\tau - \sigma)} + \overline{\alpha}_n{}^{\mu} e^{-in (\tau + \sigma)} \right).
\end{equation}
The commutation relations for the modes are
\begin{equation}
  [\alpha^{\mu}_m, \alpha^{\nu}_n] = m \delta_{m + n, 0} \eta^{\mu\nu}, \qquad
  [\alpha^{\mu}_m, \overline{\alpha}^{\nu}_n] = 0, \qquad
  [\overline{\alpha}^{\mu}_m, \overline{\alpha}^{\nu}_n] = m \delta_{m + n, 0} \eta^{\mu\nu}.
\end{equation}
For each direction $\mu$, these are an infinite number of ladder operator relations ($(\alpha_n^{\mu})^{\dagger} = \alpha^{\mu}_{-n}$).

\begin{definition}[vacuum]
  We introduce a \emph{vacuum state} $\ket{0}$ on the worldsheet $\Sigma$ such that
  \begin{equation}
    \alpha_n^{\mu} \ket{0} = 0 \qquad n \geq 0.
  \end{equation}
\end{definition}

We recall the Fourier modes of $T_{ab}$  are $L_n$  and $\overline{L}_n$, where
\begin{equation}
  \label{eq:5-1}
  L_m = \frac{1}{2} \Sigma_n \alpha_{m - n} \cdot \alpha_n.
\end{equation}
These are called the \emph{Virasoro operators}.
\begin{leftbar}
  The contraction is given by the spacetime Minkowski metric.
  Later we will briefly talk about how things change with a curved metric.
\end{leftbar}
This expression Eq.~\eqref{eq:5-1} is ambiguous for $m = 0$ because $\alpha_n$ and $\alpha_{-n}$ do not commute. Going from the classical to the quantum theory, operator ordering starts to matter. 
We shall take
\begin{equation}
  \label{eq:5-2}
  L_0 = \frac{1}{2} \alpha_0^2 + \sum_{n > 0} \alpha_{-n} \cdot \alpha_n.
\end{equation}
\begin{leftbar}
  The ambiguity is still there as we will see later. We just shifted the ambiguity into the states.
\end{leftbar}

Instead of thinking about $x^{\mu}$, $p_{\mu}$ and $T_{ab}$, it will be more useful to think about the mode operators $\alpha^{\mu}_m, \overline{\alpha}^{\mu}_m$ and $L_m$ and $\overline{L}_m$.

\subsection{Physical State Conditions}%
\label{sub:physical_state_conditions}

We shall be imposing the conditions $T_{ab} = 0$  on the Hilbert space of the theory.

\begin{notation}[]
  Let us define $N = \sum_{n > 0} \alpha_{-n} \cdot \alpha_n$ and $\overline{N} = \sum_{n > 0} \overline{\alpha}_{-n} \cdot \overline{\alpha}_n$.
\end{notation}
\begin{leftbar}
  We think of the $\alpha_n$ with positive (negative) $n$ to be annihilation (creation) operators.
  In that case $N$ is like a number operator.
  However, they are not quite like the ladder operators for the quantum harmonic operator, and $N$ is not quite a number operator; it is more like a weighted number operator.
\end{leftbar}

Then $L_0 = \frac{\alpha'}{4} p^2 + N$, and $\overline{L}_0 = \frac{\alpha'}{4} p^2 + \overline{N}$, where $\alpha_0^{\mu} = \sqrt{\frac{\alpha'}{2}} p^{\mu}$.

It would be too strong a condition for the operator to vanish $T_{ab} = 0$  identically as an operator expression; there would not be much of a theory left if we did that.
\begin{definition}[physical states]
  Instead, we require that $T_{ab} \ket{\phi} = 0$ for $\ket{\phi}$ to be a \emph{physical state}.
\end{definition}
This means that $L_n \ket{\phi} = 0$ and  $\bra{\phi} L_{-n} = 0$ for $n > 0$ .

We shall also require $L_0 \ket{\phi} = a \ket{\phi}$ and $\overline{L}_0 \ket{\phi} = a \ket{\phi}$, where the constant $a$  reflects the ambiguity in defining the ordering of operators constituting $L_0$ , Eq.~\eqref{eq:5-2}, in the quantum theory.

For now we shall take $a = 1$.
 \begin{leftbar}
  We will find that we can interpret the canonical quantised states in a natural way when $a = 1$.
\end{leftbar}

\begin{definition}[]
  Let us define $L^{\pm} = L_0 \pm \overline{L}_0$.
\end{definition}
Then our conditions for a state $\ket{\phi}$ to be physical translate to
\begin{equationbox}
  \label{eq:5-3}
  (L^+_0 - 2) \ket{\phi} = 0, \qquad
  L^-_0 \ket{\phi} = 0, \qquad
  L_n \ket{\phi} = 0 = \overline{L}_n \ket{\phi} \quad \text{for } n > 0.
\end{equationbox}

\section{The Spectrum}%
\label{sec:the_spectrum}

\subsection{The Tachyon}%
\label{sub:the_tachyon}

\begin{definition}[]
  We can construct a (spacetime) momentum eigenstate as
  \begin{equation}
    \ket{k} = e^{i k \cdot x} \ket{0}
  \end{equation}
\end{definition}
In terms of a position basis in the target space, the momentum operator is $p_{\mu} = -i \frac{\partial }{\partial x^{\mu}}$, so $p_{\mu} \ket{k} = k_{\mu} \ket{k}$.

From the point of view of a massless Klein--Gordon theory, we would just stop here. However, we have got to check the physicality conditions, which we imposed. 
Straightforwardly, we get from \eqref{eq:5-3} that
\begin{equation}
  L_n \ket{k} = 0 = \overline{L}_n \ket{k}.
\end{equation}
For the other two, we need to work a bit harder. We can write $L_0^- = N - \overline{N}$. This is sometimes called \emph{level matching}.
\begin{leftbar}
  This is the only constraint coupling the left- and right-moving sectors together.
\end{leftbar}
Here, $N = \overline{N} = 0$.
Finally, we only need to check that $(L_0^+ - 2) \ket{k} \stackrel{!}{=} 0$ .
\begin{equation}
  (L_0^+ - 2) \ket{k} = \left( \frac{\alpha'}{2} p^2 + N + \overline{N} - 2 \right) \ket{k} = \left( \frac{\alpha'}{2} k^2 - 2 \right) \ket{k} = 0.
\end{equation}
This gives us a condition on $k^2$ : $k^2 - \frac{4}{\alpha'} = 0$. 
If we compare this with the energy-momentum condition $k^2 + M^2 = 0$, we see that the spacetime interpretation of this is that the state has negative mass squared:
\begin{equation}
  \boxed{M^2 = -\frac{4}{\alpha'}}
\end{equation}
The state $\ket{k}$ has a spacetime interpretation as a tachyon.
