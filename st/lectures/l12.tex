% lecture notes by Umut Özer
% course: st
\lhead{Lecture 12: February 14}

Let us now apply this to the bosonic string theory.
Take $S$ to be the Polyakov action (with gauge fixed worldsheet metric) and $\delta \phi= v^{a} \partial_{a} \phi$ .
\begin{exercise}
  Under such a transformation the action varies as
  \begin{equation}
    \delta S[\phi] = \frac{1}{2\pi} \int_{\Sigma} \dd[2]{\sigma} (\partial_a v_b) T^{ab},
  \end{equation}
  where $T^{ab}$ is the stress tensor.
\end{exercise}
We have
\begin{equation}
  \frac{1}{2\pi} \int_{\Sigma} \dd[2]{\sigma} (\partial^{a} v^{b}) \langle T_{ab} \phi_1 \dots \phi_n \rangle = \sum_{k=1}^{n} \langle \phi_1 \dots \delta \phi_k \dots \phi_n \rangle.
\end{equation}

Our worldsheet looks locally like a cylinder with coordinates $(\sigma, \tau)$.
\begin{figure}[tbhp]
  \centering
  \def\svgwidth{0.8\columnwidth}
  \input{lectures/l12f1.pdf_tex}
  \caption{}
  \label{fig:l12f1}
\end{figure}
We map each of these cylinders to the complex plane using $z = e^{\tau +i \sigma}$  and glue the regions together using (holomorphic) transition functions. This is illustrated in Fig.~\ref{fig:l12f1}.

If we take the boundaries $C_i$ to points  (the local $\tau$ coordinate goes to $-\infty$), we can associate the state on the boundary $C_i$ to a local operator  $\phi(z_i, \overline{z}{}_i)$ .

We choose our parameter $v^{a}$  to be:
\begin{itemize}
  \item zero on all $C_i$, except at $C_{\omega}$, which is the boundary associate to the operator at $z_i = \omega$, i.e.~$\delta_v \phi_i = 0$ except for $\delta_v \phi(\omega, \overline{\omega}{})$,
  \item of the form $v^{a} = (v(z), \overline{v}{}(\overline{z}{}))$,
  \item arbitrary in the interior of $\Sigma$.
\end{itemize}
For this choice of $v^{a}$, we have
\begin{equation}
  \label{eq:12-1}
  \frac{1}{2\pi} \int_{\Sigma} \dd[2]{\sigma} (\partial_{a} v_{b}) T^{ab} = \langle \phi_1 \dots \delta \phi(\omega, \overline{\omega}{}) \dots \phi_n \rangle.
\end{equation}
To put it another way, we look at a transformation where only one field changes for simplicity.
We can rewrite the left-hand side of \eqref{eq:12-1} as
\begin{equation}
  \frac{1}{2\pi} \int_{\Sigma} \dd[2]{\sigma} \partial^{a} (v^{b} \langle T_{ab}(\sigma) \phi_1 \dots \phi_n \rangle) -\frac{1}{2\pi} \int_{\Sigma} \dd[2]{\sigma} v^{b} \partial^{a} \langle T_{ab}(\sigma) \phi_1 \dots \phi_n \rangle.
\end{equation}
The first term is a boundary term ($\partial \Sigma = \bigcup_i C_i$).
The only boundary contribution comes from $C_{\omega}$, where $v^{a}$ has holomorphic ($v(z)$) and antiholomorphic ($\overline{v}{}(\overline{z}{})$) components and is
\begin{equation}
  \frac{1}{2\pi i} \oint_{C_\omega} \dd[]{z} v(z) \langle T(z) \phi_1 \dots \phi_n \rangle - \frac{1}{2\pi i} \oint_{C_\omega} \dd[]{\overline{z}{}} \langle \overline{T}{}(\overline{z}{})) \phi_1 \dots \phi_n \rangle,
\end{equation}
where $T_{zz}(z) = T(z)$  and $T_{\overline{zz}{}} (\overline{z}{}) = \overline{T}{}(\overline{z}{})$.
Thus
\begin{multline}
  \langle \phi_1 \dots \delta_v \phi(\omega, \overline{\omega}{}) \dots \phi_n \rangle = - \frac{1}{2\pi} \int_{\Sigma} \dd[2]{\sigma} v^{b}  \partial^{a} \langle T_{ab} \phi_1 \dots \phi_n \rangle  \\
  + \frac{1}{2\pi i} \oint_{C_{\omega}} \dd[]{z} v(z) \langle T(z) \phi_1 \dots \phi_n \rangle
  + \frac{1}{2\pi i} \oint \dd[]{\overline{z}{}} \langle \overline{T}{}(\overline{z}{}) \phi_1 \dots \phi_n \rangle
\end{multline}
The $v$'s appear in very different ways here.
In the term on the left hand side and the other two terms on the right, $v$ appears as a puncture.
However, the first term on the right is different.
But $v^{a}$ is arbitrary in the interior of $\Sigma$, so we must have
\begin{equation}
  \partial^{a} \langle T_{ab} (\sigma) \phi_1 \dots \phi_n \rangle = 0.
\end{equation}
This is the analogue of the classical (Noether) statement $\partial^{a} T_{ab} = 0$.

All fields $\phi_i \neq \phi(\omega, \overline{\omega}{})$ did not contribute to the calculation and just came along for the ride.
We conclude that we have the general operator statement
\begin{equation}
  \boxed{\delta_v \phi(\omega, \overline{\omega}{}) = \oint_{C_{\omega}} \frac{\dd[]{z}}{2\pi i } v(z) T(z) \phi(\omega, \overline{\omega}{}) - \oint_{C_{\omega}} \frac{\dd[]{\overline{z}{}}}{2\pi i } \overline{v}{}(\overline{z}{}) T(\overline{z}{}) \phi(\omega, \overline{\omega}{})}
\end{equation}
This is understood to hold as an equality once inserted into a correlation function.

We can start to see why it was so important to go to complex coordinates.
By the residue theorem, the contributions on the right-hand side come from points where the functions on the right have poles.
The singularity structure of operators as we bring them close together contains information about the physics.
It will turn out that the local structure of the theory is contained within these singularities; not all singularities are bad!


\section{Radial Ordering}%
\label{sec:radial_ordering}

Our task now is to understand better the singularity structure of these operators.
Let us go back to the cylindrical worldsheet of Fig.~\ref{fig:l10f1}.
Mapping to the complex plane with $z = e^{\tau + i \sigma}$ in Fig.~\ref{fig:l10f2}, we see that the notion of time ordering $\tau_1 > \tau_2$  translates into radial ordering $\abs{z_1} > \abs{z_2}$ .
\begin{definition}[radial ordering]
  We introduce the notion of \emph{radial ordering}:
  \begin{equation}
    \boxed{\mathcal{R}(A(z) B(w)) \coloneqq
    \begin{cases}
      A(z) B(w), & \text{if } \abs{z} > \abs{w} \\
      B(w) A(z), & \text{if } \abs{z} < \abs{w}
    \end{cases}}
  \end{equation}
\end{definition}
We will eventually work our way towards Wick's theorem for radial ordered products.
