% lecture notes by Umut Özer
% course: st
\lhead{Lecture 21: March 06}

\section{Massless Modes}%
\label{sec:massless_modes}

Consider a correlation function
\begin{equation}
  \langle \phi_1 \dots \phi_n \rangle_\eta = \int \pdd{X} e^{-S[X]} \phi_1 \dots \phi_n
\end{equation}
in flat spacetime ($g_{\mu\nu} = \eta_{\mu\nu}$).
How does this change if we let
\begin{equation}
  g_{\mu\nu} = \eta_{\mu\nu} \mapsto \eta_{\mu\nu} + \epsilon_{\mu\nu} e^{i k \cdot X}.
\end{equation}
The Polyakov action changes by 
\begin{equation}
  S[X] \mapsto S[X] - \frac{1}{4 \pi \alpha'} \int_{\sigma} \dd[2]{\sigma} \sqrt{-h} h^{ab} \epsilon_{\mu\nu} e^{i k \cdot X} \partial_{a} X^{\mu} \partial_{b} X^{\nu}.
\end{equation}
Assuming the change in $g_{\mu\nu}$ is small, then to leading order, the change in the correlation function is the insertion of the operator
\begin{equation}
  \int_{\Sigma}	\dd[2]{\sigma} \sqrt{-h}h^{ab} \epsilon_{\mu\nu} \partial_{a} X^{\mu} \partial_{b} X^{\nu} e^{i k \cdot X}.
\end{equation}
This leads us to the graviton vertex operator:
\begin{equation}
  V_g = g_c \int_{\Sigma} \dd[2]{z} \epsilon_{\mu\nu} \partial X^{\mu} \overline{\partial}{} X^{\nu} e^{i k \cdot X(z, \overline{z}{})}, \qquad \epsilon_{\mu\nu} = \epsilon_{\nu\mu}.
\end{equation}
We require all of our operators in the BRST cohomology. We found that a condition for this vertex operator to be BRST invariant if the integrand is of conformal weight $(1,1)$.
\begin{exercise}
  Show that this happens when $k^2 = 0$, $\epsilon_{\mu\nu} k^{\nu} = 0$, and $\epsilon_{\mu\nu} k^{\mu} = 0$.
\end{exercise}

When first looking at the quantum string, we found exactly these conditions as physical state conditions for the graviton. Here they fall out naturally of the BRST considerations. 

Once we have a $(1, 1)$ primary field, we can construct two operators.
In addition to $V_g$ also have the unintegrated vertex operator 
\begin{equation}
  U_g = g_c c(z) \overline{c}{}(z) \epsilon_{\mu\nu} \partial X^{\mu} \overline{\partial}{} X^{\nu} e^{i k \cdot X(z, \overline{z}{})},
\end{equation}
which is invariant under BRST.

The state/operator correspondence gives
\begin{equation}
  \lim_{z \to 0} U_{g} = g_c c_1 \overline{c}{}_1 \epsilon_{\mu\nu} \alpha^{\mu}_{-1} \overline{\alpha}{}^{\nu}_{-1} \ket{k}.
\end{equation}
Ignoring the ghost bits, this mode term is precisely what we interpreted as the graviton state.

Also 
\begin{description}
  \item[$B$-field:] ($b_{\mu\nu} = - b_{\nu\mu}$)
    \begin{align}
      U_B &= g_c b_{\mu\nu} c(z) \overline{c}{}(z) \partial X^{\mu} \overline{\partial}{} X^{\nu} e^{i k \cdot X} \\
      V_B &= g_c \int_{\Sigma} \dd[2]{z} b_{\mu\nu} \partial X^{\mu} \overline{\partial}{} X^{\nu} e^{i k \cdot X}.
    \end{align}
    The similarity between the $B$-field and graviton is not accidental, but we will not talk much about the underlying similarities in this course.
  \item[Dilaton:]
    \begin{align}
      U_\phi &= g_c c(z) \overline{c}{}(z) \phi \partial X^{\mu} \overline{\partial}{} X^{\nu} e^{i k \cdot X} \\
      V_\phi &= g_c \int_{\Sigma} \dd[2]{z} \phi \partial X^{\mu} \overline{\partial}{} X^{\nu} e^{i k \cdot X}.
    \end{align}
\end{description}

\subsection*{Massive States}%

Because the exponential $e^{i k \cdot X(z, \overline{z}{})}$ is not a conformal scalar, we can construct infinitely many other vertex operators.
The massive states will be subject to renormalisation, which would take us beyond the scope of the course.
Secondly, the unbroken gauge symmetry is found in the massless sector, which is really the interesting part of string theory.
Finally, we do not really have much time left in this course, and need to be a bit selective in the choice of topics we are discussing.

\section{S-Matrix}%
\label{sec:s_matrix}

We have weight $(h, \overline{h}{}) = (1, 1)$ operators $\phi_i = \phi(z_i, \overline{z}{}_i)$ and we compute the $n$-point amplitude
\begin{equation}
  A_{n} = \sum_{g = 0}^{\infty} g_c^{2 g - 2 + n} \int_{\mathrlap{M_g}} \dd[s]{t} \int \prod_{i = \kappa + 1}^n \dd[2]{z_i} \int \pdd{X} \pdd{[b, \overline{b}{}]} \pdd{[c, \overline{c}{}]} e^{-S[X, b, c, \overline{b}{}, \overline{c}{}]} \prod_{I = 1}^s (b, \mu_I) \prod_{j=1}^\kappa c(z_j) \overline{c}{}(\overline{z}{}_j) \phi_1 \dots \phi_n
\end{equation}
where $\kappa = \dim(\text{CKG})$, $s = \dim(M_g)$.

Let us introduce some notation, which will make the BRST invariants (almost) obvious.
\begin{equation}
  V_i = g_c \int \dd[2]{z_i} \phi_i, \qquad U_i = g_c c(z_i) \overline{c}{}(\overline{z}{}_i) \phi_i.
\end{equation}
Since the $\phi_i$ are weight $(1, 1)$ fields, we know that these are BRST invariant.

This gives us something, which looks a little bit more palatable.
\begin{equation}
  A_n = \sum_{g=0}^{\infty} g_c^{2g-2} \int_{\mathrlap{M_g}} \dd[s]{t} \langle \prod_{I=1}^s (b, \mu_I) \prod_{i=1}^{\kappa} U_i \prod_{j=\kappa + 1}^{n} V_i \rangle_{X, b, c}.
\end{equation}

\subsection*{Tree-level}%


We will focus on tree-level amplitudes, where $g = 0$:
In this case, $M_g$ is a point; we can ignore the integral over the moduli space and $s = 0$.
The CKG is $SL(2; \mathbb{C})$, so the complex dimension $\kappa = 3$; we fix three of our punctures.
Then,
\begin{align}
  A_n^{\text{tree}} &= g_c^{-2} \langle \prod_{i=1}^3 U_i \prod_{j=4}^n V_j \rangle_{X, b, c}. \\
		    &= g_c^{n-2} \int \prod_{i=4}^n \dd[2]{z_i} \langle \prod_{j=1}^{3} c(z_j) \overline{c}{}(\overline{z}{}_j) \rangle_{b, c} \langle \phi_1 \dots \phi_n \rangle_{X}
\end{align}
To construct the scattering amplitudes for $n > 3$, we need both kinds of vertex operator, despite ultimately giving the same physical interpretation in target space, as we have seen for the graviton.

\subsection{The Ghost Bit}%
\label{sub:the_ghost_bit}

The invariance under the CKG should guarantee that the expression does not depend on the location of the $n -3$ punctures, although this is not at all obvious.

Consider
\begin{equation}
  \langle c(z_1) c(z_2) c(z_3) \rangle =  \bra{0} c(z_1) c(z_2) c(z_3) \ket{0}
\end{equation}
We can use our knowledge of the expansion of the $c$ ghosts, which have conformal weight $-1$:
\begin{equation}
  c(z) = \sum_{n} c_n z^{-n + 1}, \qquad c_n \ket{0} = 0 \quad n > 1,
\end{equation}
we find
\begin{equation}
  \langle c(z_1) c(z_2) c(z_3) \rangle = K (z_1 - z_2) (z_2 - z_3) (z_3 - z_1),
\end{equation}
where $K = \bra{0} c_{-1} c_0 c_1 \ket{0} = 1$ is our choice of normalisation.

We get a similar answer for the $\overline{c}{}$ ghosts.
Therefore, the ghost bits gives
\begin{equation}
  \boxed{ \langle \prod_{j=1}^{3} c(z_j) \overline{c}{}(\overline{z}{}_j) \rangle = \abs{z_1 - z_2}^2 \abs{z_2 - z_3}^2 \abs{z_3 - z_1}^2 }
\end{equation}
This is a universal, appearing in all tree-level scattering amplitudes.
So 
\begin{equation}
  A_n^{\text{tree}} = g_c^{n-2} \abs{z_1 - z_2}^2 \abs{z_2 - z_3}^2 \abs{z_3 - z_1}^2 \int \prod_{j=4}^{n} \dd[2]{z_j} \langle \prod_{i=1}^{n} \phi(z_i, \overline{z}{}_i) \rangle_X.
\end{equation}

\section{Calculations Using the Path Integral}%
\label{sec:calculating_using_the_path_integral}

We want to compute the $n$-point function
\begin{equation}
  \label{eq:n-pt}
  \langle \phi_1 \dots \phi_n \rangle_X = \int \pdd{X} e^{-S_P[X]} \phi_1 \dots \phi_n,
\end{equation}
where $S_P[X] = -\frac{1}{2\pi \alpha'} \int_{\Sigma} \dd[2]{z} \partial X^{\mu} \overline{\partial}{} X^{\nu} \eta_{\mu\nu}$.
To do this, we introduce a source term 
\begin{equation}
  S_J[X] = \int_{\Sigma} \dd[2]{z} X^{\mu} (z, \overline{z}{}) J_{\mu} (z, \overline{z}{})
\end{equation}
into the path integral. Consider the (normalised) generating functional
\begin{equation}
  Z [J] = Z[0]^{-1} \int \pdd{X} e^{-S_P[X] - S_J[J, X]}.
\end{equation}
